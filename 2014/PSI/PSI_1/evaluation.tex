\section{Evaluation}
\label{sec:Evaluation}

We implemented our algorithm of abstract translation in a tool built on top of 
FsYacc\footnote{http://fsharppowerpack.codeplex.com/wikipage?title=FsYacc}. We completely 
reused LALR generator, but implemented custom interpreter with stack copying ability. 

Our tool was evaluated on a migration of a real-world project from MS-SQL Server 2005 to Oracle 11gR2. 
The original system contained 850 stored procedures and more than 3000 dynamic queries. 
The total size of the system was 2,7 million lines of code. More than half of all queries were 
complex; the number of query-generating operators varied from 7 to 212. The average number of 
query-generating operators was 40. We used PC workstation with Intel Core i7 2.6 GHz and 16 GB of RAM.

The results of comparison of two abstract translation implementations are presented in the Table~\ref{results}.

The first implementation was directly based on abstract parsing algorithm. That version was not adapted 
to process complex queries and turned system into active swapping. The analysis did not finish
in acceptable time. Timeout (64 seconds) was added to limit one query processing time. Experiments 
showed that increasing timeout did not increase the number of processed queries. The number of queries,
whose analysis was terminated by a timeout is shown in the table under the category 
"Dynamic SQL-queries with exponential growth of parsing forest".

The second implementation utilized state merging. State merging reduced the number of queries with 
exponential growth of parsing forest from 253 to 42, i.e. approximately in six times.

In the table below we present statistics for dynamic SQL query processing by two algorithms: original algorithm with timeout and algorithm with states merging.

Partially processed queries are those with non-empty parsing forest but with parsing or lexing errors. 
This category is the most difficult to deal with because error may be a false positive. Such situation may 
occur if query which triggers error can not actually be generated at run time.

\begin{table}
\begin{center}

\caption{Comparison of the original algorithm with timeout and the algorithm with state merging}
\begin{tabular}[c c c]{| p{5cm} | p{1.1cm} | p{1.1cm} |}
\hline
Category description & Original algorithm with timeout & The algorithm with state merging
\\
\hline
The total number of dynamic SQL queries & 3122 & 3122
\\
\hline
The number of successfully processed dynamic SQL queries & 2181 & 2253
\\
\hline
 & &
\\
\hline
\bfseries{The number of partially processed dynamic SQL queries} & 408 & 522
\\
\hline

 %\ \ \ \ 
 Lexer errors & 283 & 289
\\
\hline

 %\ \ \ \ 
 Parser errors & 354 & 468
\\
\hline
 & &
\\
\hline

\bfseries{The number of not processed dynamic SQL queries} & 533 & 347
\\
\hline
 %\ \ \ \
  Lexer errors & 140 & 134
\\
\hline

 %\ \ \ \
 Parser errors & 280 & 305
\\
\hline

 %\ \ \ \ 
 Dynamic SQL queries with exponential growth of parsing forest. & 253 & 42

\\
\hline
 & &
\\
\hline


Percentage of successfully processed dynamic SQL queries & 69.86\% & 72.17\%
\\
\hline

Percentage of partially processed dynamic SQL queries & 13.07\% & 16.72\%
\\
\hline

Percentage of dynamic SQL queries with non-empty forest & 82.93\% & 88.89\%
\\
\hline
 
\end{tabular}
\label{results}
\end{center}

\end{table}

