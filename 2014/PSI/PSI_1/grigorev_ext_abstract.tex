% This is LLNCS.DEM the demonstration file of
% the LaTeX macro package from Springer-Verlag
% for Lecture Notes in Computer Science,
% version 2.4 for LaTeX2e as of 16. April 2010
%
\documentclass{llncs}
%
\usepackage{makeidx}  % allows for indexgeneration
\usepackage[T2A]{fontenc} 
\usepackage[utf8]{inputenc}
\usepackage[english]{babel}
\usepackage{graphicx}
\usepackage{indentfirst}
\usepackage{hyperref}

%
\begin{document}
%
%\frontmatter          % for the preliminaries
%
%\pagestyle{headings}  % switches on printing of running heads
%\addtocmark{Hamiltonian Mechanics} % additional mark in the TOC
%
%
\tableofcontents
%
\mainmatter              % start of the contributions
%
\title{From Abstract Parsing to Abstract Translation}
%
%\titlerunning{Hamiltonian Mechanics}  % abbreviated title (for running head)
%                                     also used for the TOC unless
%                                     \toctitle is used
%
\author{Semyon Grigorev\inst{1} 
\and Iakov Kirilenko\inst{2}
}
%
%\authorrunning{Ivar Ekeland et al.} % abbreviated author list (for running head)
%
%%%% list of authors for the TOC (use if author list has to be modified)
\tocauthor{Semyon Grigorev
, Iakov Kirilenko
}
%

\institute{St. Petersburg State University \\
198504, Universitetsky prospekt 28, Peterhof, St. Petersburg, Russia. \\
\email{rsdpisuy@gmail.com}
\and
St. Petersburg State University \\
198504, Universitetsky prospekt 28, Peterhof, St. Petersburg, Russia. \\
\email{jake@math.spbu.ru}
}

\maketitle              % typeset the title of the contribution

\begin{abstract}

String-embedded languages translation is one of the problems which can be faced during database applications and information systems migration. There are two approaches to solve this problem: static translation or translation in run time. Our research deals with the first one since there is no solution for embedded languages static translation problem.

Doh, Kim, and Schmid's algorithm can be used for parsing of dynamically constructed expressions. In this article we propose its adaptation intended for string-embedded languages translation. We describe modifications of the basic algorithm which allow to process semantics of embedded languages. An algorithm of parsing forest minimization is also presented. We discuss some issues of real-world dynamic SQL queries processing and provide possible solutions. Finally, we present results of implemented algorithm application in real-world software reengineering project aimed to migrate information system from MS-SQL server to Oracle server.

\keywords{Embedded languages, two-staged languages, string-embedded languages, abstract parsing,  abstract translation, LALR, dynamic SQL}
\end{abstract}

\section{Introduction}

Most programming languages allow to use string-embedded languages: dynamically build expressions in other languages as string values and evaluate them with intended processor (database, web browser, or interpreter). In the current work we use dynamic SQL as an example of string-embedded languages. The mechanism of dynamic SQL queries contraction and execution was specified in ISO SQL standard in 1992 ~\cite{ISO} and most DBMS support this. 

Dynamically constructed expressions is a string expressions in any programming language and for compiler they are just string constants. If we want to change execution method or runtime of this expression (e.g. change target database from MySQL to Oracle) then we possibly should transform it. For classical case we have a translator from one language to another, and the process called "translation". Now we want to describe similar process for string-embedded languages and we call it "abstract translation".

Dynamic queries translation is necessary for information systems reengineering and database applications migration from one database management system to another~\cite{NetDbTransform}. Translation of one dialect of SQL to another is often required during this process. If source code contains dynamic queries then we should guarantee that all string values which are used for dynamic queries building remain correct after translation. By correctness of resulting values we mean that queries formed from them are correct in the context of target DBMS.
 
Various objects modifications may be performed during information system reengineering for example renaming or dead code elimination . In this case we should guarantee not only syntax correctness of translation results but also semantic correctness in the context of performed environment modifications. 

For practical usage it is very important that information systems can contain a big number of dynamic queries and each one of them can contains hundreds of branches~\cite{TiunovaUIInt}. The necessity of analyzing all possible values for each query can cause  performance problems and exponential growth of required resources.

There are no tools for string-embedded languages translation. The existing production tools for database applications development, such as PL-SQL Developer ~\cite{PLSQL}, SwisSQL ~\cite{SwissSQL} or SQL Ways ~\cite{SQLWays}, allow to translate stored SQL code but do not support dynamic SQL translation. There is a set of tools for string-embedded languages processing, such as PHP String Analyzer ~\cite{PHPSA} or Alvor, but these tools also can not be used for dynamically constructed expression translation.

One of the possible approaches of dynamic expressions translation is a translation in run time with implementation of custom function for statements execution. It means that the final version of statement constructed during program execution will be executed with custom function which translate statement from old language to current before execution. This approach was successfully applied for information system migration~\cite{OpenSystemsDBMS}.

The second possible approach is static translation. It means that it can be done statically as part of information system migration process by the similar way as classical translation. So,  we should generalize translation techniques for application to string embedded languages.  As a result of abstract translation we should calculate new values for all string variables to ensure   correctness of dynamically constructed expressions in the target system. As we have already mentioned, the problem of static string-embedded languages translation is not solved yet.

Contribution of this article is an adaptation and application of the abstract parsing algorithm ~\cite{AbstrParsing} to solve the problem of string-embedded languages translation. We describe necessary modifications of basic abstract parsing algorithm to make it applicable to string-embedded languages translation in the context of real-world information systems migration. 

The paper is organized as follows: The next Section introduces and describe abstract parsing. Section~\ref{sec:AbstractTranslation} introduces abstract translation and describes basic ideas of string-embedded languages translation algorithm. Section~\ref{sec:Optimizations} describes optimizations of abstract translation algorithm discussed in the previous Section. Section~\ref{sec:Evaluation} presents results of proposed algorithm application to real-world migration of the information system which contains more than 2 million lines of code and more than 3000 dynamic SQL queries from MS-SQL Server to Oracle and Section~\ref{sec:Conclusion} discuss some problems of abstract translation and possible ways to solve them.



%
% ---- Bibliography ----
%
\begin{thebibliography}{}
  
\bibitem{RelaxedLALR}
Ефимов А.А., Кириленко Я.А. Построение ослабленного LALR-транслятора на основе анализа грамматики на избыточность // Системное программирование. Т. 4, вып. 1, 2009. С. 79–103.  

\bibitem{mart}
Мартыненко Б.К. Языки и трансляции. Издательство Санкт-Петербургского университета, 2008. 257 с. 

\bibitem{TiunovaUIInt}
Мосиенко М.А., Тиунова А.Е. Интеграция программной логики с пользовательским интерфейсом при реинжиниринге приложений // Системное программирование. Т. 1, вып. 1, 2004. С. 199–224.

\bibitem{NetDbTransform}
Трошин С.Л. Преобразование сетевых баз данных в реляционные: задачи и подходы // Системное программирование. Т. 1, вып. 1. 2004. С. 282–310.

\bibitem{ALVOR1}
Annamaa A., Breslav A., Kabanov J. e.a. An interactive tool for analyzing embedded SQL queries. Programming Languages and Systems. LNCS, vol. 6461. Springer: Berlin; Heidelberg. 2010. P. 131–138.

\bibitem{ALVOR2}
Annamaa A., Breslav A., Vene V. Using abstract lexical analysis and parsing to detect errors in string-embedded DSL statements // Proceedings of the 22nd Nordic Workshop on Programming Theory. Marina Walden and Luigia Petre, editors. 2010. P. 20-22.

\bibitem{StringExpr}
\emph{Aske Simon Christensen, Mller A., Michael I. Schwartzbach.} Precise analysis of string expressions // Proc. 10th International Static Analysis Symposium (SAS), Vol. 2694 of LNCS. Springer-Verlag: Berlin; Heidelberg, June, 2003. P. 1–18.

\bibitem{Grune}
Grune D., Ceriel J. H. Jacobs. Parsing techniques: a practical guide. Ellis Horwood, Upper Saddle River, NJ, USA, 1990. P. 322.

\bibitem{SAofStrVal}
Costantini G., Ferrara P., Cortesi F. Static analysis of string values // Proceedings of the 13th international conference on Formal methods and software engineering, ICFEM’11. Springer-Verlag: Berlin; Heidelberg, 2011. P. 505–521.

\bibitem{ISO}
ISO. ISO/IEC 9075:1992: Title: Information technology — Database languages — SQL. 1992. P. 668.

\bibitem{JSA}
Java String Analyzer. URL: \href{http://www.brics.dk/JSA/}{http://www.brics.dk/JSA/}

\bibitem{AbstrParsing}
Kyung-Goo Doh, Hyunha Kim, David A. Schmidt. Abstract parsing: Static analysis of dynamically generated string output using LR-parsing technology // Proceedings of the 16th International Symposium on Static Analysis, SAS’09. Springer-Verlag: Berlin; Heidelberg, 2009. P. 256–272.

\bibitem{PHPSA}
PHP String Analyzer. URL: \href{http://www.score.is.tsukuba.ac.jp/~minamide/phpsa/}{http://www.score.is.tsukuba.ac.jp/$\sim$minamide/phpsa/}

\bibitem{PLSQL}
PL/SQL Developer. URL: \href{http://www.allroundautomations.com/plsqldev.html}{http://www.allroundautomations.com/plsqldev.html}

\bibitem{OpenSystemsDBMS}
Shapot M., Popov E. Database reengineeing // Open Systems.DBMS. Number 4. 2004.    

\bibitem{SQLWays}
SQL Ways. URL: \href{http://www.ispirer.com/products}{http://www.ispirer.com/products}

\bibitem{SwissSQL}
SwissSQL. URL: \href{http://www.swissql.com/}{http://www.swissql.com/}

\bibitem{SAForInject}
Xiang Fu, Xin Lu, Peltsverger B. e.a. A static analysis framework for detecting SQL injection vulnerabilities // Proceedings of the 31st Annual International Computer Software and Applications Conference. Vol. 01, COMPSAC’07, Washington, DC, USA, IEEE Computer Society, 2007. P. 87–96.

\end{thebibliography}%\clearpage
%\addtocmark[2]{Author Index} % additional numbered TOC entry
%\renewcommand{\indexname}{Author Index}
%\printindex
%\clearpage
%\addtocmark[2]{Subject Index} % additional numbered TOC entry
%\markboth{Subject Index}{Subject Index}
%\renewcommand{\indexname}{Subject Index}
%\input{subjidx.ind}
\end{document}
