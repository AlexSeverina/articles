%%% Проверка используемого TeX-движка %%%
\usepackage{ifxetex}

%%% Поля и разметка страницы %%%
\usepackage{lscape}                                % Для включения альбомных страниц
\usepackage{geometry}                              % Для последующего задания полей
\usepackage{float}

%%% Математические пакеты %%%
\usepackage{amsthm,amsfonts,amsmath,amssymb,amscd} % Математические дополнения от AMS

%%% Кодировки и шрифты %%%
\ifxetex
  \usepackage{polyglossia}                         % Поддержка многоязычности
  \usepackage{fontspec}                            % TrueType-шрифты
\else
  \usepackage{cmap}                                % Улучшенный поиск русских слов в полученном pdf-файле
  \usepackage[T2A]{fontenc}                        % Поддержка русских букв
  \usepackage[utf8]{inputenc}                      % Кодировка utf8
  \usepackage[english, russian]{babel}             % Языки: русский, английский
  \usepackage{pscyr}                               % Красивые русские шрифты
\fi

%%% Оформление абзацев %%%
\usepackage{indentfirst}                           % Красная строка

%%% Цвета %%%
\usepackage[usenames]{color}
\usepackage{color}
\usepackage{colortbl}

%%% Таблицы %%%
\usepackage{longtable}                             % Длинные таблицы
\usepackage{multirow,makecell,array}               % Улучшенное форматирование таблиц

%%% Общее форматирование
\usepackage[singlelinecheck=off,center]{caption}   % Многострочные подписи
\usepackage{soul}                                  % Поддержка переносоустойчивых подчёркиваний и зачёркиваний
\usepackage{icomma}                                % Запятая в десятичных дробях

%%% Библиография %%%
\usepackage{cite}

%%% Гиперссылки %%%
\usepackage[plainpages=false,pdfpagelabels=false]{hyperref}

%%% Изображения %%%
\usepackage{graphicx}                              % Подключаем пакет работы с графикой
\usepackage{epstopdf}

%%% Опционально %%%
% Следующий пакет может быть полезен, если надо ужать текст, чтобы сам текст не править, но чтобы места он занимал поменьше
%\usepackage{savetrees}

% Этот пакет может быть полезен для печати текста брошюрой
%\usepackage[print]{booklet}