\documentclass{beamer}
\usepackage{beamerthemesplit}
\usepackage{wrapfig}
\usetheme{SPbGU}
\usepackage{pdfpages}
\usepackage{amsmath}
\usepackage{mathtools}
\usepackage{cmap} 
\usepackage[T2A]{fontenc} 
\usepackage[utf8]{inputenc}
\usepackage[english,russian]{babel}
\usepackage{indentfirst}
\usepackage{amsmath}
\usepackage{tikz}
\usepackage{multirow}
\usepackage[noend]{algpseudocode}
\usepackage{algorithm}
\usepackage{algorithmicx}
\usepackage{ stmaryrd }
\usepackage{qtree}
\usetikzlibrary{shapes,arrows}
\usepackage{fancyvrb}
\usepackage{minted}

\beamertemplatenavigationsymbolsempty

\newcommand{\derive}[0]{\xRightarrow[]{*}}
\newcommand{\derives}[0]{\xRightarrow[]{}}
\newcommand{\derivek}[1]{\xRightarrow[]{#1}}
\newcommand{\deriveg}[1]{\xRightarrow[#1]{*}}
\newcommand{\derivegone}[1]{\xRightarrow[#1]{}}

\title[]{Теория автоматов и формальных языков}
\subtitle[]{Контекстно-свободные языки: нисходящий анализ}
\institute[]{
Санкт-Петербургский государственный электротехнический университет <<ЛЭТИ>>\\
}

\author[]{Екатерина Вербицкая}

\date{25 октября 2016г.}

\definecolor{orange}{RGB}{179,36,31}

\begin{document}
{
  \begin{frame}
    \titlepage
  \end{frame}
}


\begin{frame}[fragile]
  \transwipe[direction=90]
  \frametitle{В предыдущей серии}
  \begin{itemize}
    \item Восходящий и нисходящий синтаксический анализ
    \item LL-грамматики
  \end{itemize}
\end{frame}

\begin{frame}[fragile]
  \transwipe[direction=90]
  \frametitle{Нисходящий синтаксический анализ}
  \begin{itemize}
    \item Top-down parsing
    \item Начинаем разбирать со стартового нетерминала, применяем правила грамматики, пока не получим строку
    \begin{itemize}
      \item С откатом ([full] backtracking)
      \item Без отката (without backtracking)
    \end{itemize}
  \end{itemize}
\end{frame}

\begin{frame}[fragile]
  \transwipe[direction=90]
  \frametitle{Нисходящий синтаксический анализ с откатом}
  \begin{itemize}
    \item Метод грубой силы, bruteforce
    \item Перебираем все возможные варианты разбора, если что-то пошло не так --- возвращаемся к началу и пробуем снова
  \end{itemize}
\end{frame}  

\begin{frame}[fragile]
  \transwipe[direction=90]
  \frametitle{Bruteforce parsing: пример}

  $$
  \begin{array}{crcccl}
  &S& \rightarrow & a A d & | & a B \\
  &A& \rightarrow & b     & | & c \\
  &B& \rightarrow & c c d & | & d d c
  \end{array}
  $$
  
  $\omega = a d d c$ \pause 
  
  $S \pause \derives a A d \pause \derives a b d$ \pause --- не подходит, откатываемся \pause 
  
  $S \derives a A d \pause \derives a c d$ \pause --- не подходит, откатываемся \pause 
  
  $S \pause \derives a B \pause \derives a c c d$ \pause --- не подходит, откатываемся \pause 
  
  $S \derives a B \pause \derives a d d c$ \pause --- ура!
  
  ~\\~
  
  Проблема: ну очень уж долго работает: экспоненциальное время!
\end{frame}

\begin{frame}[fragile]
  \transwipe[direction=90]
  \frametitle{Нисходящий синтаксический анализ без отката}
  \begin{itemize}
    \item Рекурсивный спуск (recursive descent parsing)
    \begin{itemize}
      \item Для каждого нетерминала написана функция
      \item Функции для нетерминалов рекурсивно вызывают друг друга
    \end{itemize}
  \end{itemize}
\end{frame}

\begin{frame}[fragile]
  \transwipe[direction=90]
  \frametitle{Рекурсивный спуск: пример}
  Код
\end{frame}

\begin{frame}[fragile]
  \transwipe[direction=90]
  \frametitle{Нисходящий синтаксический анализ без отката: LL(k)}
  \begin{itemize}
    \item Идея: откат запрещен, но разрешен предпросмотр
    \item По (нескольким) следующим терминалам принять решение о том, какую продукцию использовать
    \item Как и предыдущие 2 подхода не может обрабатывать леворекурсивные правила грамматики
    \item Достаточно хорош для используемых на практике языков
  \end{itemize}
\end{frame}

\begin{frame}[fragile]
  \transwipe[direction=90]
  \frametitle{Леворекурсивные правила грамматики}
  \begin{itemize}
    \item Явная (непосредственная) левая рекурсия
    \begin{itemize}
      \item $A \rightarrow A \beta$
    \end{itemize}
    \item Неявная левая рекурсия
    \begin{itemize}
      \item $A \rightarrow \alpha A \beta, \, \alpha \derive \varepsilon$
    \end{itemize}
    \item Взаимная рекурсия
    \begin{itemize}
      \item $A \rightarrow \alpha B \beta, \, B \rightarrow \gamma A \delta, \, \alpha \derive \varepsilon, \gamma \derive \varepsilon$
    \end{itemize}
  \end{itemize}
\end{frame}

\begin{frame}[fragile]
  \transwipe[direction=90]
  \frametitle{Избавление от левой рекурсии}
  \begin{itemize}
   \item $A \rightarrow A \alpha \, | \, \beta \Leftrightarrow A \rightarrow \beta A', \, A' \rightarrow \varepsilon \, | \, \alpha A'$
  \end{itemize} \pause
  \begin{itemize}
    \item $E \rightarrow E + T \, | \, T \Leftrightarrow E \rightarrow T E', \, E' \rightarrow \varepsilon \, | \, + T E'$
  \end{itemize} \pause

\begin{tabular}{p{5.5cm} p{6cm}}
  
\Tree [.E [.E [.E T ] + T ] + T ]  
& 
\Tree [.E T [.E' + T [.E' + T [.E' $\varepsilon$ ] ] ] ] 

\end{tabular}
\end{frame}

\begin{frame}[fragile]
  \transwipe[direction=90]
  \frametitle{Избавление от левой рекурсии: более общий случай}
  \begin{itemize}
   \item $A \rightarrow A \alpha_1 \, | \, A \alpha_2 \, | \, \dots \, | \, A \alpha_n \, | \, \beta_1 \, | \, \beta_2 \, | \, \dots \, | \, \beta_k$
  \end{itemize}
  \begin{itemize}
   \item $A \rightarrow \beta_1 A' \, | \, \beta_2 A' \, | \, \dots \, | \, \beta_k A'$
   \item $A' \rightarrow \varepsilon \, | \, \alpha_1 A' \, | \, \alpha_2 A' \, | \, \dots \, | \, \alpha_n A' $
  \end{itemize}
\end{frame}

\begin{frame}[fragile]
  \transwipe[direction=90]
  \frametitle{Избавление от взаимной левой рекурсии}
  \begin{itemize}
   \item Избавляемся от $\varepsilon$-продукций
   \item Упорядочиваем правила по индексу нетерминала
   \item Добиваемся того, чтобы не было правил вида $A_i \rightarrow A_j \alpha, j \leq i$
   \begin{itemize}
     \item Перебираем все $A_i$
     \item Перебираем все $A_j, 1 \leq j < i$
     \item Для каждого правила $p: \, A_i \rightarrow A_j \gamma$
     \begin{itemize}
       \item Удалить правило $p$
       \item Для каждого правила $A_j \rightarrow \delta_1 \,|\, \dots \, | \, \delta_k$ Добавить правила $A_i \rightarrow \delta_l$
     \end{itemize}
     \item Устранить непосредственную левую рекурсию для $A_i$
   \end{itemize}
  \end{itemize}
\end{frame}

 
\begin{frame}[fragile]
  \transwipe[direction=90]
  \frametitle{Левая факторизация грамматики}
  \begin{itemize}
   \item Выделяем наибольший общий префикс продукций $A \rightarrow \alpha \beta \, | \, \alpha \gamma \Rightarrow A\rightarrow \alpha A', \, A' \rightarrow \beta \, | \, \gamma$
  \end{itemize}
\end{frame}

\begin{frame}[fragile]
  \transwipe[direction=90]
  \frametitle{Пример}
  $$
  \begin{array}{crcl}
  &S& \rightarrow & a S S b S \\
  & &           | & a S a S b \\
  & &           | & a b b \\
  & &           | & b 
  \end{array}
  $$
 \pause
  $$
  \begin{array}{crcl}
  &S & \rightarrow & a S' \\
  &  &           | & b \\
  
  &S'& \rightarrow & S S b S \\
  &  &           | & S a S b \\
  &  &           | & b b 
  \end{array}
  $$ 
  \pause
  $$
  \begin{array}{crcl}
  &S  & \rightarrow & a S' \, | \, b \\
  
  &S' & \rightarrow & S S'' \, | \, b b \\
  
  &S''& \rightarrow & S b S \, | \, a S b 
  \end{array}
  $$
\end{frame}

\begin{frame}[fragile]
  \transwipe[direction=90]
  \frametitle{Множество FIRST}
  \begin{itemize}
   \item Множество символов, которые могут появиться первыми во время вывода из данной сентенциальной формы
   \item $FIRST(a \alpha) = \{ a \}, \, a \in V_T, \alpha \in (V_T \cup V_N)^*$
   \item $FIRST(\varepsilon) = \{ \varepsilon \}$
   \item $FIRST(\alpha \beta) = FIRST(\alpha) \cup (FIRST(\beta), if \varepsilon \in FIRST(\alpha))$
   \item $FIRST(S) = FIRST(\alpha) \cup FIRST(\beta), S \rightarrow \alpha \, | \, \beta $
  \end{itemize}
\end{frame}

\begin{frame}[fragile]
  \transwipe[direction=90]
  \frametitle{Множество FIRST: пример}
  $$
  \begin{array}{crcl}
  &S  & \rightarrow & a S' \\
  
  &S' & \rightarrow & A b B S' \, | \, \varepsilon \\
  
  &A  & \rightarrow & a A' \, | \, \varepsilon \\
  &A' & \rightarrow & b \, | \, a \\
  &B  & \rightarrow & c \, | \, \varepsilon  
  \end{array}
  $$ \pause
  
  \begin{itemize}
    \item $FIRST(S) = \{ a \}$ \pause
    \item $FIRST(A) = \{ a, \varepsilon \}$ \pause
    \item $FIRST(A') = \{ a, b \}$ \pause
    \item $FIRST(B) = \{ c, \varepsilon \}$ \pause
    \item $FIRST(S') = \{ a, b, \varepsilon \}$ 
  \end{itemize}
\end{frame}

\begin{frame}[fragile]
  \transwipe[direction=90]
  \frametitle{Множество FOLLOW}
  \begin{itemize}
   \item Множество символов, которые могут появиться в некотором выводе сразу после данной сентенциальной формы
   \item Положим $FOLLOW(X) = \varnothing $
   \item Если $X$ --- стартовый нетерминал, $FOLLOW(X) = FOLLOW(X) \cup \{ \$ \}$ --- символ конца строки
   \item Для всех правил вида $A \rightarrow \alpha X \beta$, $FOLLOW(X) = FOLLOW(X) \cup (FIRST(\beta) \setminus \{ \varepsilon\})$
   \item Для всех правил вида $A \rightarrow \alpha X$ и $A \rightarrow \alpha X \beta$, где $\varepsilon \in FIRST(\beta)$, $FOLLOW(X) = FOLLOW(X) \cup FOLLOW(A)$
   \item Повторять последние 2 пункта, пока можно что-то добавлять
  \end{itemize}
\end{frame}

\begin{frame}[fragile]
  \transwipe[direction=90]
  \frametitle{Множество FOLLOW: пример}
  $$
  \begin{array}{crcl}
  &S  & \rightarrow & a S' \\
  
  &S' & \rightarrow & A b B S' \, | \, \varepsilon \\
  
  &A  & \rightarrow & a A' \, | \, \varepsilon \\
  &A' & \rightarrow & b \, | \, a \\
  &B  & \rightarrow & c \, | \, \varepsilon  
  \end{array}
  $$ \pause
  
  \begin{itemize}
    \item $FOLLOW(S) = \{ \$ \}$ \pause
    \item $FOLLOW(S') = \{ \$ \} \, (S \rightarrow a S')$  \pause
    \item $FOLLOW(A) = \{ b \} \, (S' \rightarrow A b B S')$ \pause
    \item $FOLLOW(A') = \{ b \} \, (A \rightarrow a A')$ \pause
    \item $FOLLOW(B) = \{ a, b, \$ \} \, (S' \rightarrow A b B S', \varepsilon \in FIRST(S'))$ 
  \end{itemize}
\end{frame}

\begin{frame}[fragile]
  \transwipe[direction=90]
  \frametitle{LL(1)-анализ}
  \begin{itemize}
   \item Нисходящий синтаксический анализ с предпросмотром одного символа
   \item Читает вход слева направо (L: left-to-right), строит левый вывод в грамматике (L: leftmost)
   \item Состоит из:
   \begin{itemize}
     \item Входного буфера (откуда читается входная строка)
     \item Стека (для промежуточных данных)
     \item Таблицы анализатора (управляет процессом разбора)
   \end{itemize}
   \item Работает за $O(n)$, где $n$ --- длина входной строки
  \end{itemize}
\end{frame}

\begin{frame}[fragile]
  \transwipe[direction=90]
  \frametitle{Таблица LL(1)-анализатора}
  $$
  S \rightarrow ( S ) \, | \, \varepsilon
  $$  
  
  Размещаем продукции в таблице (по горизонтали --- нетерминалы; по вертикали --- терминалы + $\$ $)
  \begin{itemize}
    \item Продукции вида $A \rightarrow \alpha$ --- в ячейки $(A, a)$, где $a \in FIRST(A)$
    \item Продукции вида $A \rightarrow \varepsilon$ --- в ячейки $(A, a)$, где $a \in FOLLOW(A)$
  \end{itemize}   ~\\~
  
\begin{center}
\begin{tabular}{ l || c | c || c | c | r }
  N & FIRST & FOLLOW & ( & ) & $\$ $ \\ \hline  
  $S$ & \pause $\{ (, \varepsilon \}$ & $\{ ), \$ \}$ & \pause $S \rightarrow (S)$ & \pause $S \rightarrow \varepsilon$ & $S \rightarrow \varepsilon$ 
\end{tabular} 
\end{center} 
\end{frame}  

\begin{frame}[fragile]
  \transwipe[direction=90]
  \frametitle{Синтаксический анализ (доска)}
  $$
  S \rightarrow ( S ) \, | \, \varepsilon
  $$    

\begin{center}
\begin{tabular}{ l || c | c || c | c | r }
  N & FIRST & FOLLOW & ( & ) & $\$ $ \\ \hline  
  $S$ & $\{ (, \varepsilon \}$ & $\{ ), \$ \}$ & $S \rightarrow (S)$ & $S \rightarrow \varepsilon$ & $S \rightarrow \varepsilon$ 
\end{tabular}  
\end{center}

$\omega = (()) \$ $

Стек: $\$, S, ), S, (, ), S, ($ 
  

\end{frame}  

\begin{frame}[fragile]
  \transwipe[direction=90]
  \frametitle{Когда LL-анализ не возможен}
  \begin{itemize}
    \item Леворекурсивные правила
    \item Когда при построении таблицы в одну ячейку нужно записать больше одной записи
    \begin{itemize}
      \item FIRST-FIRST конфликт
      \begin{itemize}
        \item $A \rightarrow \alpha \, | \, \beta, FIRST(\alpha) \cap FIRST(\beta) \neq \varnothing $ 
        \item $E \rightarrow T + E \, | \, T * E$
      \end{itemize}
      \item FIRST-FOLLOW конфликт
      \begin{itemize}
        \item $FIRST(A) \cap FOLLOW(A) \neq \varnothing$
        \item $S \rightarrow A a b, A \rightarrow a \, | \, \varepsilon$
      \end{itemize}
    \end{itemize} 
    \item Как с этим бороться? 
    \begin{itemize}
      \item Избавиться от левой рекурсии
      \item Избавиться от недетерминизма
      \item Факторизовать грамматику
      \item Использовать аннотации (если есть)
      \item Переписать грамматику
      \item Использовать более одного символа предпросмотра
    \end{itemize}
  \end{itemize}
\end{frame} 

\end{document}
