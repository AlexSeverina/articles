\documentclass{beamer}
\usepackage{beamerthemesplit}
\usepackage{wrapfig}
\usetheme{SPbGU}
\usepackage{pdfpages}
\usepackage{amsmath}
\usepackage{mathtools}
\usepackage{cmap} 
\usepackage[T2A]{fontenc} 
\usepackage[utf8]{inputenc}
\usepackage[english,russian]{babel}
\usepackage{indentfirst}
\usepackage{amsmath}
\usepackage{tikz}
\usepackage{multirow}
\usepackage[noend]{algpseudocode}
\usepackage{algorithm}
\usepackage{algorithmicx}
\usetikzlibrary{shapes,arrows}
\usepackage{fancyvrb}
\newtheorem{rutheorem}{Теорема}
\newtheorem{ruproof}{Доказательство}
\newtheorem{rudefinition}{Определение}
\newtheorem{rulemma}{Лемма}
\beamertemplatenavigationsymbolsempty

\title[]{Теория автоматов и формальных языков}
\subtitle[]{Регулярные языки}
\institute[]{
Санкт-Петербургский государственный электротехнический университет <<ЛЭТИ>>\\
}

\author[]{Екатерина Вербицкая}

\date{ сентября 2016г.}

\definecolor{orange}{RGB}{179,36,31}

\begin{document}
{
  \begin{frame}
    \titlepage
  \end{frame}
}

\begin{frame}[fragile]
  \transwipe[direction=90]
  \frametitle{Регулярная грамматика}
  \textbf{Праволинейная грамматика} --- грамматика, все правила которой имеют следующий вид:
  \begin{itemize}
    \item $A \rightarrow \omega B$ или $A \rightarrow \omega$, где $A, B \in V_N, \omega \in V^*$
  \end{itemize}


  \textbf{Леволинейная грамматика} --- грамматика, все правила которой имеют следующий вид:
  \begin{itemize}
    \item $A \rightarrow B \omega$ или $A \rightarrow \omega$, где $A, B \in V_N, \omega \in V^*$
  \end{itemize}

\pause 

  \begin{rutheorem}[]
    Пусть L --- формальный язык. 

    $\exists G_r$ --- праволинейная грамматика, т.ч. $L = L(G_r) \Leftrightarrow \exists G_l$ --- леволинейная грамматика, т.ч. $L = L(G_l) $
  \end{rutheorem}
\pause
  \textbf{Регулярная грамматика} --- праволинейная или леволинейная грамматика
\end{frame}

\begin{frame}[fragile]
  \transwipe[direction=90]
  \frametitle{Регулярный язык}
  \textbf{Регулярный язык} --- язык, для которого существует регулярная грамматика, порождающая его
  \begin{itemize}
    \item 
  \end{itemize}

\end{frame}


\begin{frame}[fragile]
  \transwipe[direction=90]
  \frametitle{}
  \begin{itemize}
    \item 
  \end{itemize}
\end{frame}

\begin{frame}[fragile]
  \transwipe[direction=90]
  \frametitle{}
  \begin{itemize}
    \item 
  \end{itemize}
\end{frame}

\begin{frame}[fragile]
  \transwipe[direction=90]
  \frametitle{}
  \begin{itemize}
    \item 
  \end{itemize}
\end{frame}

\begin{frame}[fragile]
  \transwipe[direction=90]
  \frametitle{}
  \begin{itemize}
    \item 
  \end{itemize}
\end{frame}

\begin{frame}[fragile]
  \transwipe[direction=90]
  \frametitle{}
  \begin{itemize}
    \item 
  \end{itemize}
\end{frame}

\begin{frame}[fragile]
  \transwipe[direction=90]
  \frametitle{}
  \begin{itemize}
    \item 
  \end{itemize}
\end{frame}

\begin{frame}[fragile]
  \transwipe[direction=90]
  \frametitle{}
  \begin{itemize}
    \item 
  \end{itemize}
\end{frame}

\begin{frame}[fragile]
  \transwipe[direction=90]
  \frametitle{}
  \begin{itemize}
    \item 
  \end{itemize}
\end{frame}

\begin{frame}[fragile]
  \transwipe[direction=90]
  \frametitle{}
  \begin{itemize}
    \item 
  \end{itemize}
\end{frame}

\begin{frame}[fragile]
  \transwipe[direction=90]
  \frametitle{}
  \begin{itemize}
    \item 
  \end{itemize}
\end{frame}

\begin{frame}[fragile]
  \transwipe[direction=90]
  \frametitle{}
  \begin{itemize}
    \item 
  \end{itemize}
\end{frame}

\begin{frame}[fragile]
  \transwipe[direction=90]
  \frametitle{}
  \begin{itemize}
    \item 
  \end{itemize}
\end{frame}

\begin{frame}[fragile]
  \transwipe[direction=90]
  \frametitle{}
  \begin{itemize}
    \item 
  \end{itemize}
\end{frame}

\begin{frame}[fragile]
  \transwipe[direction=90]
  \frametitle{}
  \begin{itemize}
    \item 
  \end{itemize}
\end{frame}

\begin{frame}[fragile]
  \transwipe[direction=90]
  \frametitle{}
  \begin{itemize}
    \item 
  \end{itemize}
\end{frame}

\begin{frame}[fragile]
  \transwipe[direction=90]
  \frametitle{}
  \begin{itemize}
    \item 
  \end{itemize}
\end{frame}

\end{document}
