\documentclass{beamer}
\usepackage{beamerthemesplit}
\usepackage{wrapfig}
\usetheme{SPbGU}
\usepackage{pdfpages}
\usepackage{amsmath}
\usepackage{cmap} 
\usepackage[T2A]{fontenc} 
\usepackage[utf8]{inputenc}
\usepackage[english,russian]{babel}
\usepackage{indentfirst}
\usepackage{amsmath}
\usepackage{tikz}
\usepackage{multirow}
\usepackage[noend]{algpseudocode}
\usepackage{algorithm}
\usepackage{algorithmicx}
\usetikzlibrary{shapes,arrows}
\usepackage{fancyvrb}
\newtheorem{rutheorem}{Теорема}
\newtheorem{ruproof}{Доказательство}
\newtheorem{rudefinition}{Определение}
\newtheorem{rulemma}{Лемма}
\beamertemplatenavigationsymbolsempty

\title[]{Теория автоматов и формальных языков}
\subtitle[]{Организационные моменты}
\institute[]{
Санкт-Петербургский государственный электротехнический университет <<ЛЭТИ>>\\
}

\author[]{Екатерина Вербицкая}

\date{ сентября 2016г.}

\definecolor{orange}{RGB}{179,36,31}

\begin{document}
{
\begin{frame}
  \titlepage
\end{frame}

}

\begin{frame}[fragile]
  \transwipe[direction=90]
  \frametitle{Контакты}
  \begin{itemize}
    \item Екатерина [Андреевна] Вербицкая
    \item JetBrains Programming Languages and Tools Lab 
    \item email: \href{mailto:ekaterina.verbitskaya@jetbrains.com}{ekaterina.verbitskaya@jetbrains.com} 
    \item email: \href{mailto:kajigor@gmail.com}{kajigor@gmail.com} 
    \item Материалы: \href{https://goo.gl/0aRfjV}{https://goo.gl/0aRfjV}
  \end{itemize}
\end{frame}

\begin{frame}[fragile]
  \transwipe[direction=90]
  \frametitle{Занятия}
  \begin{itemize}
    \item Первая пара --- лекция
    \item Вторая пара --- практика
    \item Лабораторные выполняются дома, сдаются онлайн
    \begin{itemize}
      \item Сроки сдачи необходимо строго соблюдать
    \end{itemize}
    \item 2 контрольных
  \end{itemize}
\end{frame}

\begin{frame}[fragile]
  \transwipe[direction=90]
  \frametitle{Условия успешной сдачи}
  \begin{itemize}
    \item Балльная система
    \begin{itemize}
      \item $1$ балл за посещение лекции/практики
      \item $1$ балл за успешно решенную задачу у доски 
      \item $1$ балл за своевременно сданную (под)задачу
      \item $1.5$ балла за зачтенную (под)задачу на контрольных
      \item $10$ баллов за (отличный) доклад
    \end{itemize}
    \item Оценки в конце семестра выставляются в зависимости от суммы баллов 
--- конкретные пороги будут объявлены позже
    \pause
    \begin{itemize}
      \item $0  \% \le sum \le 55 \%$ --- неудовлетворительно
      \item $55 \% < sum \le 70 \%$ --- удовлетворительно
      \item $70 \% < sum \le 85 \%$ --- хорошо
      \item $85 \% < sum \le 100 \%$ --- отлично
    \end{itemize}
  \end{itemize}
  \begin{itemize}
    \item Оценку можно будет улучшить на очном экзамене
  \end{itemize}
\end{frame}

\begin{frame}[fragile]
  \transwipe[direction=90]
  \frametitle{Требования к оформлению домашних заданий}
  \begin{itemize}
    \item Задачи, не подразумевающие кодирование
    \begin{itemize}
      \item pdf-файл, сверстанный по 
шаблону (\href{https://goo.gl/PR2QwI}{https://goo.gl/PR2QwI})
      \item Прислать на почту с темой письма ``[ElTech\_FL] HW i''
    \end{itemize}
    \item Код сдается через систему pull request-ов вот тут:
  \end{itemize}
\end{frame}

\begin{frame}[fragile]
  \transwipe[direction=90]
  \frametitle{Литература и материалы}
  \begin{itemize}
    \item Хопкрофт Дж. Э., Мотвани Р., Ульман Дж. Д. Введение в теорию автоматов, языков и вычислений, 2-е изд./Пер. с англ. – М.: Вильямс, 2002.
    \item Ахо А., Ульман Дж. Теория синтаксического анализа, перевода и компиляции. Т.1/ Пер. с англ. – М.: Мир, 1978
    \item Вики-конспект ИТМО по аналогичному курсу: \href{http://goo.gl/we8kvB}{http://goo.gl/we8kvB}
  \end{itemize}
\end{frame}


\end{document}
