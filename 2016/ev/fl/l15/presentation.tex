\documentclass{beamer}
\usepackage{beamerthemesplit}
\usepackage{wrapfig}
\usetheme{SPbGU}
\usepackage{pdfpages}
\usepackage{amsmath}
\usepackage{mathtools}
\usepackage{cmap} 
\usepackage[T2A]{fontenc} 
\usepackage[utf8]{inputenc}
\usepackage[english,russian]{babel}
\usepackage{indentfirst}
\usepackage{amsmath}
\usepackage{tikz}
\usepackage{multirow}
\usepackage[noend]{algpseudocode}
\usepackage{algorithm}
\usepackage{algorithmicx}
\usepackage{ stmaryrd }
\usepackage{qtree}
\usepackage{ wasysym }
\usepackage{ dsfont }

\usetikzlibrary{shapes,arrows}
\usepackage{fancyvrb}
\usepackage{minted}
\newtheorem{rutheorem}{Теорема}
\newtheorem{ruproof}{Доказательство}
\newtheorem{rudefinition}{Определение}
\newtheorem{rulemma}{Лемма}
\beamertemplatenavigationsymbolsempty

\newcommand{\derive}[0]{\xRightarrow[]{*}}
\newcommand{\derives}[0]{\xRightarrow[]{}}
\newcommand{\derivek}[1]{\xRightarrow[]{#1}}
\newcommand{\deriveg}[1]{\xRightarrow[#1]{*}}
\newcommand{\derivegone}[1]{\xRightarrow[#1]{}}

\title[]{Теория автоматов и формальных языков}
\subtitle[]{Иерархия Хомского}
\institute[]{
Санкт-Петербургский государственный электротехнический университет <<ЛЭТИ>>\\
}

\author[]{Екатерина Вербицкая}

\date{13 декабря 2016г.}

\definecolor{orange}{RGB}{179,36,31}

\begin{document}
{
  \begin{frame}
    \titlepage
  \end{frame}
}

\begin{frame}[fragile]
  \transwipe[direction=90]
  \frametitle{В предыдущей серии}
  \begin{itemize}
    \item Регулярные языки и конечные автоматы
    \item Контекстно-свободные языки и магазинные автоматы
    \item Неограниченные языки и машины Тьюринга
  \end{itemize}
\end{frame}

\begin{frame}[fragile]
  \transwipe[direction=90]
  \frametitle{Контекстно-зависимые грамматики}
  КЗ грамматика: $(V_T, V_N, P, S)$
  
  \begin{itemize}
    \item $V_T$ --- алфавит терминалов
    \item $V_N$ --- алфавит нетерминалов, $V_T \cap V_N = \varnothing$
    \item $P$ --- конечное множество продукций грамматики вида $\alpha A \beta \rightarrow \alpha \gamma \beta: \, \alpha, \beta \in (V_T \cup V_N)^*, A \in V_N, \gamma \in (V_T \cup V_N)^+$
    \begin{itemize}
      \item $\alpha, \beta$ --- контекст
    \end{itemize}    
    \item $S \in V_N$ --- стартовый нетерминал
  \end{itemize} 
  
  Язык: $\{ \omega \in V_T^* \, | \, S \derive \omega\}$ 
\end{frame}


\begin{frame}[fragile]
  \transwipe[direction=90]
  \frametitle{Неукорачивающие грамматики}
  Неукорачивающая: $(V_T, V_N, P, S)$
  
  \begin{itemize}
    \item $V_T$ --- алфавит терминалов
    \item $V_N$ --- алфавит нетерминалов, $V_T \cap V_N = \varnothing$
    \item $P$ --- конечное множество продукций грамматики вида $\alpha \rightarrow \beta: \, \alpha, \beta \in (V_T \cup V_N)^+, 1 \leq |\alpha| \leq |\beta|$   
    \item $S \in V_N$ --- стартовый нетерминал
  \end{itemize}
  
  Язык: $\{ \omega \in V_T^* \, | \, S \derive \omega\}$   
\end{frame}


\begin{frame}[fragile]
  \transwipe[direction=90]
  \frametitle{Эквивалентность КЗ и неукорачивающих грамматик}
\begin{rutheorem}[]
  Контекстно-зависимые и неукорачивающие грамматики задают один и тот же класс языков
\end{rutheorem}

\begin{ruproof}
$\Rightarrow$: Любая КЗ-грамматика является неукорачивающей
\end{ruproof}

\end{frame}

\begin{frame}[fragile]
  \transwipe[direction=90]
  \frametitle{Эквивалентность КЗ и неукорачивающих грамматик}

\begin{ruproof}
$\Leftarrow$: Преобразуем правила неукорачивающей грам. к виду $\alpha A \beta \rightarrow \alpha \gamma \beta$

\begin{itemize}
  \item Заменим все вхождения терминалов в правило на нетерминалы, добавим соответствующие правила (вида $Z \rightarrow a, Z \in V_N, a \in V_T$)
  \item Теперь все правила имеют вид $X_1 X_2 \dots X_m \rightarrow Y_1 Y_2 \dots Y_{m+q}, m > 0, q \geq 0, X_i, Y_j \in V_N$
  \item Такие правила эквивалентны группе правил (требуемого вида):
  \begin{itemize}
    \item $X_1 X_2 \dots X_m \rightarrow A_1 X_2 \dots X_m$
    \item $A_1 X_2 \dots X_m \rightarrow A_1 A_2 \dots X_m$
    \item $\dots$
    \item $A_1 A_2 \dots X_m \rightarrow A_1 A_2 \dots A_m$
    \item $A_1 A_2 \dots A_m \rightarrow Y_1 A_2 \dots A_m$
    \item $Y_1 A_2 \dots A_m \rightarrow Y_1 Y_2 \dots A_m$
    \item $\dots$
    \item $Y_1 Y_2 \dots A_{m - 1} A_m \rightarrow Y_1 Y_2 \dots Y_{m-1} Y_m$
    \item $Y_1 Y_2 \dots Y_{m - 1} A_m \rightarrow Y_1 Y_2 \dots Y_{m-1} Y_m Y_{m+1} \dots Y_{m+q}$
  \end{itemize}
\end{itemize}
\end{ruproof}

\end{frame}

\begin{frame}[fragile]
  \transwipe[direction=90]
  \frametitle{Пример: язык $\{ a^n b^n c^n \, | \, n \geq 0 \}$}
  Неукорачивающая грамматика: 
  
$$
\begin{array}{crcl}
&S& \rightarrow & abc \, | \, a S Q \\
&bQc& \rightarrow & bbcc  \\
&cQ& \rightarrow & Qc 
\end{array}
$$   
\pause
$S \derives a S Q \derives a a b c Q \derives aabQc \derives aabbcc$
\pause
~\\

Преобразуем в КЗ-грамматику: \pause

\begin{itemize}
  \item $c Q \rightarrow Q c$
  \begin{itemize}
    \item $Z Q \rightarrow Q Z$
    \item $Z \rightarrow c$
  \end{itemize} \pause
  \item $Z Q \rightarrow Q Z$
  \begin{itemize}
    \item $Z Q \rightarrow A Q$
    \item $A Q \rightarrow A B$
    \item $A B \rightarrow Q B$
    \item $Q B \rightarrow Q Z$
  \end{itemize}
\end{itemize}
\end{frame}


\begin{frame}[fragile]
  \transwipe[direction=90]
  \frametitle{Рекурсивность КЗ-грамматик}
  Грамматика \textit{рекурсивна}, если существует алгоритм, определяющий, выводится ли данная строка в данной грамматике
  
  \begin{rutheorem}[]
  Контекстно-зависимые грамматики рекурсивны
  \end{rutheorem}
  
  \begin{ruproof}
  Действуем в предположении, что в грамматике нет правила $S \rightarrow \varepsilon$
  
  Строим алгоритм, проверяющий, выводится ли в грамматике $\omega \in V_T^+$
  
  Определим множества $T_m = \{ \alpha \in (V_T \cup V_N)^+\, | \, S \derivek{i} \alpha, i \leq m, | \alpha | \leq n \}$
  
  $T_0 =\{ S \}; T_m = T_{m-1} \cup \{ \alpha \, | \, \beta \derives \alpha, \beta \in T_{m-1}, | \alpha | \leq n \}; T_i \subseteq T_{i+1}$ 
  
  Если $S \derive \alpha, | \alpha | \leq n $, то $\exists m. \alpha \in T_m$
  
  Последовательно считаем множества $T_i$, пока не окажется $T_m = T_{m-1}$
  
  Количество всех возможных строк заданной длины ограничено, поэтому такая ситуация обязательно настанет
  
  Если $\omega \in T_m$, то она в языке; иначе --- нет. Алгоритм построен
  \end{ruproof}
  
\end{frame}

\begin{frame}[fragile]
  \transwipe[direction=90]
  \frametitle{Линейно-ограниченные автоматы}
  \textbf{Линейно-ограниченный автомат} --- недетерминированная одноленточная МТ, которая никогда не покидает те ячейки, в которых размещен ее вход. 
  
  Формально: $(Q, \Sigma, \Gamma, \delta, q_0, F)$
  \begin{itemize}
    \item $Q$ --- конечное множество состояний
    \item $q_0 \in Q$ --- стартовое состояние
    \item $F \subseteq Q$ --- множество конечных состояний
    \item $\Gamma$ --- алфавит допустимых символов ленты
    \item $\Sigma \subseteq \Gamma$ --- входной алфавит
    \begin{itemize}
      \item $\cent \in \Sigma$ --- маркер начала строки
      \item $\$ \in \Sigma$ --- маркер конца строки
      \item Маркеры нельзя перезаписывать или писать на место символов входной строки
    \end{itemize}
    \item $\delta: Q \times \Gamma \rightarrow 2^{Q \times \Gamma \times \{ -1, +1\} }$ --- отображение перехода
  \end{itemize}

\end{frame}

\begin{frame}[fragile]
  \transwipe[direction=90]
  \frametitle{Конфигурация и шаг}
\begin{itemize}
  \item Конфигурация: $(p, A_0, A_1, \dots, A_{n+1}, i)$
  \begin{itemize}
  	\item $p \in Q$ --- текущее состояние автомата
  	\item $A_0 = \cent$ --- маркер начала строки
  	\item $A_{n+1} = \$ $ --- маркер конца строки
  	\item $A_1 \dots A_n: A_j \in \Gamma$ --- содержимое ленты
  	\item $i \in \mathds{Z}, 0 \leq i \leq n+1$ --- позиция головки
  \end{itemize}    
  \item Отношение $\vdash$ на конфигурациях (шаг)
  \begin{itemize}
  	\item $\forall (p, A, -1) \in \delta(q, A_i), i > 0$, верно $(q, A_0, A_1, \dots, A_{n+1}, i) \vdash (p, A_0, A_1, \dots A_{i-1}, A, A_{i+1}, \dots, A_{n+1}, i - 1)$
  	\item $\forall (p, A, +1) \in \delta(q, A_i), i < n+1$, верно $(q, A_0, A_1, \dots, A_{n+1}, i) \vdash (p, A_0, A_1, \dots A_{i-1}, A, A_{i+1}, \dots, A_{n+1}, i + 1)$
  \end{itemize}
  \item Языком, принимаемым линейно-ограниченным автоматом, называется $\{ \omega \in (\Sigma \setminus \{ \cent, \$ \})^* \, | \, (q_0, \cent \omega \$, 0) \vdash^* (q, \cent \alpha \$, i), q \in F, \alpha \in \Gamma^*, 0 \leq i \leq n+1, |\omega| = n \}$
\end{itemize}

\end{frame}

\begin{frame}[fragile]
  \transwipe[direction=90]
  \frametitle{Линейно-ограниченные автоматы и КЗ-языки}
  \begin{rutheorem}[]
    Для любого КЗ-языка существует линейно-ограниченный автомат, принимающий его 
  \end{rutheorem}

  \begin{ruproof}
   Работаем с двудорожечной МТ: на первой ленте записано входное слово, вторая используется для вывода.
   
   Записываем на вторую дорожку символ $S$.
   
   На каждом шаге выбираем недетерминированно правило $\alpha \rightarrow \beta$, такое что $\alpha$ --- подстрока строки, записанной на второй дорожке. Заменяем $\alpha$ на $\beta$, сдвигая символы справа от $\alpha$, если необходимо.
   
   Если на каком-то шаге вышли за пределы длины слова --- провал. 
   
   Если удалось породить терминальную цепочку, сравниваем ее с входной. 
   
   Если совпала --- успех, иначе --- повторяем процесс
  \end{ruproof}

\end{frame}

\begin{frame}[fragile]
  \transwipe[direction=90]
  \frametitle{Линейно-ограниченные автоматы и КЗ-языки}
  \begin{rutheorem}[]
    Если язык принимается линейно-ограниченным автоматом, он является контекстно-зависимым
  \end{rutheorem}
\end{frame}

\begin{frame}[fragile]
  \transwipe[direction=90]
  \frametitle{КЗ-языки и рекурсивные множества}
  \begin{rutheorem}[]
    Существуют рекурсивные множества, не являющиеся КЗ-языками
  \end{rutheorem}
\end{frame}

\begin{frame}[fragile]
  \transwipe[direction=90]
  \frametitle{Иерархия Хомского: грамматики}
  Грамматика: $V_T, V_N, P, S$ (обозначим $V = V_T \cup V_N$). В зависимости от вида правил в $P$, выделяют разные классы грамматик:
  
  \begin{itemize}
    \item Типа 0: $\alpha \rightarrow \beta, \alpha = \gamma A \delta, A \in V_N, \gamma, \delta, \beta \in V^*$
    \item Типа 1: $\alpha A \beta \rightarrow \alpha \gamma \beta, \alpha, \beta \in V^*, A \in V_N, \gamma \in V^+$
    \begin{itemize}
      \item Или $\alpha \rightarrow \beta: 1 \leq |\alpha| \leq \beta $
    \end{itemize}
    \item Типа 2: $A \rightarrow \alpha, A \in V_N, \alpha \in V^*$
    \item Типа 3: $A \rightarrow a$, или $A \rightarrow a B$, или $A \rightarrow \varepsilon; A, B \in V_N, a \in V_T$
    \begin{itemize}
      \item Или $A \rightarrow a$, или $A \rightarrow B a$, или $A \rightarrow \varepsilon; A, B \in V_N, a \in V_T$
    \end{itemize}
  \end{itemize}
  
  Очевидным образом классы грамматик вкладываются друг в друга


\end{frame}


\begin{frame}[fragile]
  \transwipe[direction=90]
  \frametitle{Иерархия Хомского: грамматики, языки, распознаватели}
\begin{center}
\begin{tabular}{ l | c | c }
  Грамматики & Языки & Распознаватели \\ \hline \hline
  Типа 0 & неограниченные & машины Тьюринга \\ \hline
  Типа 1 & контекстно-зависимые & линейно-ограниченные автоматы \\ \hline
  Типа 2 & контекстно-свободные & магазинные автоматы \\ \hline
  Типа 3 & регулярные & конечные автоматы
  
\end{tabular}  
\end{center}

\end{frame}


\end{document}
