\section{Лекция k. Tree languages}

Вспомним определение языка. Вообще говоря, мы просто описываем правила построения некоторых множеств. Давайте констуировать множества деревьев.

Давайте взглянем на CYK. На тот CYK, который умеет строить деревья. Или просто на правила построения деревьев вывода. Вообще говоря, если не разделять шаги, то мы можем строить множества деревьев по каким-то правилам.

Эти множества могут обладать некоторыми интересными свойствами и их можно изучать как обобщённые структуры, а не страдать с каждым деревом отдельно. Вы наверняка знаете про бинарные деревья, вообще n-арные, сбалансированные и т.д.
\begin{mydef}[Tree grammar]
Tree grammar $G=(S,N,F,P)$
\begin{itemize}
\item $S$ --- аксиома. Сартовый нетерминал.
\item $N$ --- нетерминальные символы
\item $F$ --- терминальные символы
\item $P$ --- продукции вида $ \alpha \rightarrow \beta $ $\alpha, \beta$ --- деревья.
\end{itemize}
\end{mydef}

Грамматика над деревьями называется регулярной, если все нетерминальные символы имеют арность 0, и все продукции имеют вид $A \rightarrow \beta, A \in N, \beta is T(F \cup N)$


Пример --- списки натуральных чисел.


\begin{mytheorem}
\begin{enumerate}
\item $G$ --- CF grammar, тогда множество всех деревьев разбора $L(G)$ --- регулярный язык деревьев.
\item $L$ --- регулярный язык деревьев, тогда $Yield(L)$ --- контекстно свободный язык.
\item Существуют регулярные языки деревьев, не являющиеся множеством деревьев вывода никакого плоского КС.
\end{enumerate}
\end{mytheorem}

Пункт 1. $G=(S,N,\Sigma,P)$ --- КС грамматика. $G'=(S,N,F,P')$ --- регулярная грамматика деревьев.

\begin{itemize}
\item $F=\Sigma \cup \{\varepsilon\} \cup \{A_n | A \in N, \exists A \rightarrow \alpha \in P, |\alpha| = n\}$
\item Если $(A \rightarrow \varepsilon) \in P$, тогда $A \rightarrow A_0(\varepsilon) \in P'$
\item Если $(A \rightarrow a_1 \dots a_p) \in P$, тогда $A \rightarrow A_p(a_1, \dots, a_p) \in P'$
\end{itemize}


Интересен пункт 3.

$G=(S,N,F,P), F =(s(,); g(); a; b), N=(S;G';G''), P = (S \rightarrow s(G',G'');G' \rightarrow g(a); G'' \rightarrow g(b))$
В $L(G)$ есть единственное дерево $s(g(a),g(b))$.

Если попробуем построить КС-грамматику, то получим что-то вроде S->GG, G->a, G->b. В таком языке есть "aa" -> S(G(a),G(a))