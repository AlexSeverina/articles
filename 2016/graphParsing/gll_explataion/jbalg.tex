\documentclass[12pt]{article}  % standard LaTeX, 12 point type
\usepackage{amsfonts,latexsym}
\usepackage{amsthm}
\usepackage{amssymb}
\usepackage[utf8]{inputenc} % Кодировка
\usepackage[english,russian]{babel} % Многоязычность
\usepackage{algpseudocode}
\usepackage{algorithm}
\usepackage{algorithmicx}

\newtheorem{theorem}{Theorem}[section]
\newtheorem{proposition}[theorem]{Proposition}
\newtheorem{lemma}[theorem]{Lemma}
\newtheorem{corollary}[theorem]{Corollary}
\newtheorem{conjecture}[theorem]{Conjecture}

\theoremstyle{definition}
\newtheorem{definition}{Определение}[section]
\newtheorem{example}{Example}[section]

% unnumbered environments:

\theoremstyle{remark}
\newtheorem*{remark}{Remark}
\newtheorem*{notation}{Notation}
\newtheorem*{note}{Note}

\setlength{\parskip}{5pt plus 2pt minus 1pt}
%\setlength{\parindent}{0pt}

\usepackage{color}
\usepackage{listings}
\usepackage{caption}
\usepackage{graphicx}
\usepackage{ucs}

\graphicspath{{pics/}}



\title{Modification of modificated CYK}
\author{Semyon Grogorev}
\date{\today}

\begin{document}

\algnewcommand\algorithmicswitch{\textbf{switch}}
\algnewcommand\algorithmiccase{\textbf{case}}
\algnewcommand\algorithmicassert{\texttt{assert}}
\algnewcommand\Assert[1]{\State \algorithmicassert(#1)}
% New "environments"
\algdef{SE}[SWITCH]{Switch}{EndSwitch}[1]{\algorithmicswitch\ #1\ \algorithmicdo}{\algorithmicend\ \algorithmicswitch}
\algdef{SE}[CASE]{Case}{EndCase}[1]{\algorithmiccase\ #1}{\algorithmicend\ \algorithmiccase}

\algtext*{EndSwitch}
\algtext*{EndCase}
\algtext*{EndWhile}% Remove "end while" text
\algtext*{EndIf}% Remove "end if" text
\algtext*{EndFor}% Remove "end for" text
\algtext*{EndFunction}% Remove "end function" text



Выполняется обход грамматики, согласованный со входной строкой --- есть два указателя: в грамматику и позицию во воходе.
Указатель в грамматику --- слот: $X \rightarrow x_0 .. x_k \cdot x_{k+1} .. x_n$

Возможны несколько ситуаций. Предположим, что слот выглядит одним из следующих способов: $X \rightarrow \alpha \cdot x \beta$ или $X \rightarrow \alpha \cdot$. Указатель во входе находится на позиции $i$.
\begin{enumerate}
\item $x = \omega[i+1]$. Указатель в грамматике перемещается в позицию $X \rightarrow \alpha x \cdot \beta$ и указатель во входе сдвигается в позицию $i+1$.
\item $x$ --- это нетерминал $A$. Запоминаем точку возврата: кладём на стек слот $X \rightarrow \alpha x \cdot \beta$. Сдвигаем указатель в грамматике в позицию $A \rightarrow \cdot \gamma$. 
Здесь можно воспользоваться FIRST для детерминизации выбора. 
      Позиция во входе неизменна.
\item Указатель в грамматике в позиции $X \rightarrow \alpha \cdot$ и стек непуст. На вершине стека должен быть "адрес возврата" \  вида $Y \rightarrow \delta X \cdot \mu$. 
      Он извлекается из стека и становится текущим указателем в грамматике. 
\item Указатель в грамматике в позиции $S \rightarrow \alpha \cdot$ ($S$ --- стартоый нетерминал) и стек пуст. Успешное завершение разбора. Иначе провал. 
\end{enumerate}

На шаге 2 можно не продожать сразу все возможные варианты, а создать дескриптор --- сущность, позволяющую возобновить обход с места, которое он описывает.
Для каждого полученного слота вида $A \rightarrow \cdot \gamma$ создаём дескриптор, содержащий его, указатель на вход в позицию $i$ и указатель на стек, на адрес возврата, который только что был добавлен ($X \rightarrow \alpha A \cdot \beta$).
Для организации стека можно использовать GSS, а дескриптор будет указывыать на вершину в нём.

Возможны проблемы с бесконечным количеством дескрипторов и "потерей" \ точек возврата.

\begin{itemize}
\item $R$ --- мн-во дескрипторов для обработки
\item $U$ --- мн-во созданных дескрипторов
\item $P$ --- мн-во тех, для кого надо не забыть сделать $pop$
\end{itemize}

\begin{verbatim}
let _add L v i = 
    if (L, v, i) not in U 
    then 
      add (L, v, i) to U
      add (L, v, i) to R

let _pop v i =
    if v <> v_0
    then
      add (v,i) to P
      for each u in v.child do _add v.L u i

let _create L v i =
    if (L,i) not in GSS then add (L,i) to GSS
    let u = GSS.get (L,i)
    if there is not an edge from u to v
    then
      add edge from u to v
      for all (u,k) in P do _add L v k
    u
\end{verbatim}

Нам нужен табличный вариант.


\begin{verbatim}
let rec dispatcher () = 
  if R.Count <> 0 
  then 
    (L,v,i) := R.Get() 
    dispatch := false 
  else 
    stop := true 

and processing () =  
  dispatch := true 
  match L with
  | (X -> a . x b where x = input[i + 1]) ->
     i := i + 1
     L := (X -> a x . b)
     dispatch := false 
  | (X -> a . x b where x is nonterminal) ->
     v := _create (X -> a x . b) v i
     let slots = pTable[x][input[i]]  
     for L in slots do             
         _add L v i
  | (X -> a .) -> _pop v i
  | _ -> final result processing and error notification

let control () = 
    while not !stop do 
       if !dispatch then dispatcher() else processing() 

control() 

\end{verbatim}


Если граф детерминированный, то кроме использования в качестве позиции вершины не требуется иных модификаций, так как в первом случае ровно один вариант совпадения входного символа с ожидаемым в правиле.


Лес разбора ($L$ --- слот, $i,j,k$ --- координаты).
\begin{itemize}
\item Терминальные узлы $(T,i,j)$
\item Нетерминальные узлы $(N,i,j)$. Сыновья --- запакованные узлы вида $N \rightarrow \gamma \cdot , k$
\item Промежуточные узлы $(L,i,j)$. Сыновья --- запакованные узлы с меткой $L, k$, где $ i \leq k \leq j$ 
\item Запакованные узлы $(L,k)$. Один или два сына. Правый --- терминал или нетерминал с меткой $(S,k,i)$. Левый (если есть) --- терминал, нетерминал или промежуточный с меткой $(S, j, k)$ 
\end{itemize}

Теперь в дескрипторе надо запоминать ссылку на узел дерева: $(L, v, i, a)$
Дескрипторы с непустой ссылкой создаются в момент вызова \verb|_pop|. Правый сын создаваемого узла --- только что построенный узел. Левый сын хранится на ребре, исходящем из вершины, которую только что достали со стека.
Метка нового узла --- адрес возврата, полученный из достанного узла.

Теперь функции должны работать ещё и с узлами леса.

\begin{algorithm}[!ht]
\begin{algorithmic}[1]
\caption{Single vertex processing}
\label{processVertex}
\Function{add}{$L,v,i,a$}
  \If{$(L,v,i,a) \notin U$}  
      \State{$U.add(L,v,i,a)$}
      \State{$R.add(L,v,i,a)$}
  \EndIf
\EndFunction

\Function{pop}{$v,i,z$}
  \If{$v \neq v_0$}  
      \State{$P.add(v,z)$}
      \ForAll{$(a,u) \in v.outEdges$}
        \State{$y \gets$ \Call{getNodeP}{$v.L, a, z$}}
        \State{\Call{add}{$v.L,u,i,y$}}
      \EndFor
  \EndIf
\EndFunction

\Function{create}{$L,v,i,a$}
  \If{$(L,i) \notin GSS.nodes$}  
      \State{$GSS.nodes.add (L,i)$}
  \EndIf
  \State{$u \gets$ \Call{GSS.nodes.get}{$L, i$}}
  \If{$(u,a,v) \notin GSS.edges$}  
      \State{$GSS.edges.add(u,a,v)$}
      \ForAll{$(u,z) \in P$}
         \State{$y \gets$ \Call{getNodeP}{$L, a, z$}}
         \State{$(\_,\_, k) \gets z.lbl$}
         \State{\Call{add}{$L,v,k,y$}}
      \EndFor
  \EndIf
  \Return{$u$}
\EndFunction

\Function{getNodeT}{$x,i$}
  \If{$x = \varepsilon$}  
      \State{$h \gets i$}
  \Else
      \State{$h \gets i + 1$}
  \EndIf
  \If{$(x,i,h) \notin SPPF.nodes$}  
      \State{$SPPF.nodes.add(x,i,h)$}
  \EndIf
  \Return{$SPPF.nodes.get(x,i,h)$}
\EndFunction


%\Function{getNodeP}{$(X \rightarrow \omega_1 \cdot \omega_2), a ,z$}
%  \If{$\omega_1$ is terminal or non-nullable nonterminal and $\omega_2 \neq \varepsilon$}  
%      \Return{$z$}
%  \Else
%      \If{$\omega_2 = \varepsilon$}  
%          \State{$t \gets X$}
%       \Else
%          \State{$h \gets (\rightarrow \omega_1 \cdot \omega_2)$}
%       \EndIf
%       \State{$(q,k,i) \gets z.lbl$}
%      \If{$a \neq dummy$}  
%         \State{$(s,j,k) \gets a.lbl$}
%          \State{$y \gets $ find_or_create SPPF.nodes (n.lbl = (t,i,j))}
%      \EndIf

%  \EndIf
%  \Return{$SPPF.nodes.get(x,i,h)$}
%\EndFunction


\end{algorithmic}
\end{algorithm}

\begin{verbatim}
let getNodeP (X -> w1 . w2) a z =
    if w1 is terminal or non-nullable nonterminal and w2 <> \eps
    then return z
    else 
      let t = if w2 = \eps then X else (X -> w1 . w2)
      let (q,k,i) = z.lbl
      if a <> dummy
      then
       let (s,j,k) = a.lbl
       let y = find_or_create SPPF.nodes (n.lbl = (t,i,j))
       if y does not have a child with label (X -> w1 . w2)
       then
         let y' = new_packed_node(a,z)
         y.chld.add y'
       return y
      else
        let y = find_or_create SPPF.nodes (n.lbl = (t,k,i))   
        if y does not have a child with label (X -> w1 . w2)
        then
          let y' = new_packed_node(z)
          y.chld.add y'
        return y

\end{verbatim}


\begin{verbatim}
let rec dispatcher () = 
  if R.Count <> 0 
  then 
    (L,v,i,cN) := R.Get() 
    cR := dummy
    dispatch := false 
  else 
    stop := true 

and processing () =  
  dispatch := true 
  match L with
  | (X -> a . x b where x = input[i + 1]) ->
     if cN = dummyAST 
     then cN := getNodeT i
     else cR := getNodeT i
     i := i + 1
     L := (X -> a x . b)
     if !cR <> dummy
     then cN := getNodeP L cN cR 
     dispatch := false 
  | (X -> a . x b where x is nonterminal) ->
     v := _create (X -> a x . b) v i cN
     let slots = pTable[x][input[i]]  
     for L in slots do             
         _add L v i dummy
  | (X -> a .) -> 
     _pop v i cN
  | _ -> final result processing and error notification

let control () = 
    while not !stop do 
       if !dispatch then dispatcher() else processing() 

control() 

\end{verbatim}


\end{document}