\section{Preliminaries}

In this work we are focused on a parsing algorithm, and not on the data representation, and we assume that full input graph can be located in RAM memory in the optimal for our algorithm way.

Also we need to introduce some definitions.
\begin{itemize}
  \item Context-free grammar $G=(N, \Sigma, P, S)$ where $N$ is a set of nonterminal symbols, $\Sigma$ is a set of terminal symbols, $S \in N$ is a start nonterminal, and $P$ is a set of productions. 
  \item $\mathcal{L}(G)$ is a language specified by grammar $G$.
  \item Directed graph $M = (V,E,L)$ where $V$ --- vertices set, $L \subseteq \Sigma$ --- edge labels set, $E\subseteq V\times L\times V$. 
  We assume that there are no parallel edges with equal labels: for every $e_1=(v_1,l_1,v_2) \in E, e_2=(u_1,l_2,u_2) \in E$ if $v_1 = u_1$ and $v_2 = u_2$ then $l_1 \neq l_2$.
  \item $tag: E \rightarrow L$ is a helper function for edge's tag calculation . $$tag(e = (v_1,l,v_2), e \in E) = l$$
  \item $\oplus: L^+ \times L^+ \rightarrow L^+$ is a concatenation operation.
  \item Path $p$ in graph $M$ is a list of edges: 
  \begin{align*}
   p &= (v_0,l_0,v_1),(v_1,l_1,v_2),\dots,(v_{n-1},l_{n-1},v_n) \\
     &= e_0,e_1,\dots,e_{n-1}
  \end{align*}
  where $v_i \in V$,$e_i \in E$, $e_i=(v_i,l_i,v_{i+1})$, $l_i \in L$, $|p| = n, n \geq 1$. 
  \item Set of paths $P = \{p: p \text{ path in } M\}$ where $M$ is a directed graph.
  \item $\Omega: p \rightarrow L^+$ is a helper function for calculation string produced by path. 
  \begin{align*}
  & \Omega(p = e_{0},e_{1},\dots,e_{n-1}, p \in P) = \\
  & tag (e_{0}) \oplus \dots \oplus tag (e_{n-1}).
  \end{align*}
\end{itemize}

As a result we can define that context-free language constrained path querying means that we get query as grammar $G$ and result of this query is a set of paths $$P=\{p|\Omega(p) \in \mathcal{L}(G)\}.$$

For some graphs and some queries $P$ can be infinite set, and it can not be explicitly represented. 
In order to solve this problem, in this paper, we will construct compact data structure which stores all elements of $P$ in finite space and allows to extract every of them.
In this point our solution is slightly similar to subgraph querying proposed in article~\cite{GraphQueryWithEarley}, but we also construct derivation forest for result subgraph.
