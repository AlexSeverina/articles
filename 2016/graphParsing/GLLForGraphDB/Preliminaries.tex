\section{Preliminaries}

In this work we are focused on the parsing algorithm, and not on the data representation, and we assume that the whole input graph can be located in RAM memory in the optimal for our algorithm way.

We start by introduction of necessary definitions.
\begin{itemize}
  \item Context-free grammar is a quadruple $G=(N, \Sigma, P, S)$, where $N$ is a set of nonterminal symbols, $\Sigma$ is a set of terminal symbols, $S \in N$ is a start nonterminal, and $P$ is a set of productions. 
  \item $\mathcal{L}(G)$ denotes a language specified by grammar $G$, and is a set of terminal strings derived from start nonterminl of $G$: $L(G) = \{\omega | S \Rightarrow_{G}^{*} \omega\}$.
  \item Directed graph is a triple $M = (V,E,L)$, where $V$ is a set of vertices, $L \subseteq \Sigma$ is a set of edge's labels, and edges $E\subseteq V\times L\times V$. 
  We assume that there are no parallel edges with equal labels: for every $e_1=(v_1,l_1,v_2) \in E, e_2=(u_1,l_2,u_2) \in E$ if $v_1 = u_1$ and $v_2 = u_2$ then $l_1 \neq l_2$.
  \item $tag: E \rightarrow L$ is a helper function which allows to get tag of edge. $$tag(e = (v_1,l,v_2), e \in E) = l$$
  \item $\oplus: L^+ \times L^+ \rightarrow L^+$ denotes a tag concatenation operation.
  \item Path $p$ in graph $M$ is a list of incident edges: 
  \begin{align*}
   p &= (v_0,l_0,v_1),(v_1,l_1,v_2),\dots,(v_{n-1},l_{n-1},v_n) \\
     &= e_0,e_1,\dots,e_{n-1}
  \end{align*}
  where $v_i \in V$, $e_i \in E$, $e_i=(v_i,l_i,v_{i+1})$, $l_i \in L$, $|p| = n, n \geq 1$. 
  \item Set of paths $P = \{p: p \text{ path in } M\}$ where $M$ is a directed graph.
  \item $\Omega: p \rightarrow L^+$ is a helper function which constructs a string produced by the given path. 
  \begin{align*}
  & \Omega(p = e_{0},e_{1},\dots,e_{n-1}, p \in P) = \\
  & tag (e_{0}) \oplus \dots \oplus tag (e_{n-1}).
  \end{align*}
\end{itemize}

Using this definitions, we state the context-free language constrained path querying as, given a query in form of grammar $G$, construct the set of paths $$P=\{p|\Omega(p) \in \mathcal{L}(G)\}.$$

Note that, in some cases, $P$ can be an infinite set, and hence it cannot be explicitly represented. 
In order to solve this problem, in this paper, we construct compact data structure representation which stores all elements of $P$ in finite amount of space and allows to extract any of them.
%In this point our solution is slightly similar to subgraph querying proposed in article~\cite{GraphQueryWithEarley}, but we also construct derivation forest for result subgraph.
