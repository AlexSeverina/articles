\section{Evaluation}

In this section we show that performance of implemented algorithm is in good agreement with theoretical estimations, and that the worst-case of time and space complexity can be achieved.  

We use two grammars for balanced brackets --- ambiguous grammar $G_0$(fig.~\ref{grammarG0}) and unambiguous grammar $G_2$(fig.~\ref{grammarG2}) --- in order to investigate performance and grammar ambiguity correlation.

\begin{figure}[ht]
   \begin{center}
   \[
\begin{array}{rl}
   0: & S \rightarrow a \ S \ b \ S \\ 
   1: & S \rightarrow \varepsilon
\end{array}
\]
   \caption{Unambiguous grammar $G_2$ for balanced brackets}
   \label{grammarG2}        
   \end{center}
\end{figure}

As input we use complete graphs in which for each terminal symbol there is an edge labeled with it between every two vertices.
Note that we use only terminal symbols for edges labels.  
The task we solve in our experiments is to find all paths from all vertices to all vertices satisfied specified query.
Such designed input looks hard for querying in terms of required resources because there is a correct path between any two vertices and result set is infinite.

For complete graph $M=(V,E,L)$ $$\max\limits_{v \in V}\left(deg^+\left(v\right)\right) = (|V| - 1)*|\Sigma|$$, where $\Sigma$ is terminals of input grammar, hence we should get time complexity $O(|V|^4)$ and space complexity $O(|V|^3)$.

All tests were run on a PC with the following characteristics:
\begin{itemize}
\item OS: Microsoft Windows 10 Pro
\item System Type: x64-based PC
\item CPU: Intel(R) Core(TM) i7-4790 CPU @ 3.60GHz, 3601 Mhz, 4 Core(s), 4 Logical Processor(s)
\item RAM: 32 GB
\end{itemize}

Performance measurement results are presented in figure~\ref{pic:Perf}. 
For time measurement results we have that all two curves can be fit with polynomial function of degree 4 to a high level of confidence with $R^2$. 

%g(x) = m*x**3 + n*x**2 + o*x + p
%fit g(x) 'perf/2' using 1:4 via n,m,o,p

\begin{figure}[ht]
\centering
\begin{gnuplot}
set terminal epslatex color size 9cm,8cm
set yrange [0:]
set key box top left
set key width 2
set key opaque
set sample 1000
set xlabel 'Number of vertices in input graph'
set ylabel 'Time in milliseconds'

f1(x) = 0.000495989*x**4 + 0.001252184*x**3 + 0.068491746*x**2 - 0.306749160*x
f2(x) = 0.003368883*x**4 - 0.114919298*x**3 + 3.161793404*x**2 - 22.549491142*x

plot 'perf/2' using 1:3  pt 6 title '$G_2$',\
     'perf/2' using 1:4  pt 5 title '$G_0$',\
     f1(x)  with line lt -1 title '$f_1$',\
     f2(x)  lc rgb "black" dashtype 2 title '$f_2$'     

 \end{gnuplot}
\caption{Performance on complete graphs for grammar $G_0$ and $G_2$ \\ 
$f_1(x) = 0.000496*x^4 + 0.001252*x^3 + 0.068492*x^2 - 0.306749*x$; $R^2 = 0.99996$ \\
$f_2(x) = 0.003369*x^4 - 0.114919*x^3 + 3.161793*x^2 - 22.54949*x$; $R^2 = 0.99995$}
\label{pic:Perf}
\end{figure}

Also we present SPPF size in terms of nodes for both $G_0$ and $G_2$ grammars (fig.~\ref{pic:SPPFSize}).
As was expected, all two curves are cubic to a high level of confidence with $R^2 = 1$. 

\begin{figure}[ht]
\centering
\begin{gnuplot}
set terminal epslatex color size 9cm,8cm
set key box top left
set key width 2
set key opaque
set sample 1000
set xlabel 'Number of vertices in input graph'
set ylabel 'Number of SPPF nodes'

f1(x) = 3.000047*x**3 + 3.994579*x**2 + 4.191568*x
f2(x) = 3.000050*x**3 + 2.994338*x**2 + 4.196472*x


plot 'perf/2' using 1:6 pt 6 title '$G_2$',\
     'perf/2' using 1:7 pt 5 title '$G_0$',\
     f1(x)  with line lt -1 title '$f_1$',\
     f2(x)  lc rgb "black" dashtype 2 title '$f_2$'     

 \end{gnuplot}
\caption{SPPF size on complete graph for grammar $G_0$ and $G_2$ a complete graphs \\
$f_1(x) = 3.000047*x^3 + 3.994579*x^2 + 4.191568*x$; $R^2 = 1$\\
$f_2(x) = 3.000050*x^3 + 2.994338*x^2 + 4.196472*x$; $R^2 = 1$}
\label{pic:SPPFSize}
\end{figure}


%\begin{figure}[h]
%\centering
%\begin{gnuplot}
%set terminal epslatex color size 9cm,8cm
%set key box top left
%set logscale y
%set key width 2
%set key opaque
%set sample 1000
%set xlabel '$x$-label'
%set ylabel '$y$-label
%plot 'perf/3' using 1:2 with lines ls 2 ti '$Unamb$',\
%     'perf/3' using 1:3 with lines ls 3 ti '$amb$'
%
% \end{gnuplot}
%\caption{Performance on C graph for grmmars $G_0$ and $G_2$}
%\label{pic:DoubleCyclesPerf}
%\end{figure}




%To summarise we can say that performance for unambiguos grammars is better then for ambiguos. 

%Full graphs for balanced brackets.

%Full graph for highly unambiguos greammar $G_3$ (figure~\ref{grammarG3}).

%\begin{figure}[h]
%   \begin{center}
%\begin{verbatim}
%   0: s = s s s 
%   1: s = s s
%   2: s = A
%\end{verbatim}
%   \caption{Highly ambiguos grammar $G_3$}
%   \label{grammarG3}        
%   \end{center}
%\end{figure}
