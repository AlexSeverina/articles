\section{Evaluation}

One of classical graph querying problems is a navigation queries for ontologies, and we apply our algorithm to this problem in order to estimate its practical value.
We used dataset from paper~\cite{CFGonRDF}.
Our algorithm is aimed to process graphs, so RDF files were converted to edge-labeled directed graph.
For each triple $(o,p,s)$ form RDF we added two edges: $(o,p,s)$ and $(s,p^{-1},o)$.

We perform two classical \textit{same-generation queries}~\cite{FndDB}.

\textbf{Query 1} is based on the grammar for retrieving concepts on the same layer (presented in figure~\ref{grammarQ1}).
For this query our algorithm demonstrates up to 1000 times better performance and provides identical results as compared to the presented in~\cite{CFGonRDF} for $Q_1$. 

\textbf{Query 2} is based on the grammar for retrieving concepts on the adjacent layers (presented in figure~\ref{grammarQ2}). 
Note that this query differs from the original query $Q_2$ from article~\cite{CFGonRDF} in the following details.
First of all, we count only triples for nonterminal $S$ because only paths derived from it correspond to paths between concepts on adjacent layers.
Algorithm which is presented in~\cite{CFGonRDF} returns triples for all nonterminals.
Moreover, grammar $\mathcal{G}_2$, which is presented in~\cite{CFGonRDF}, describes paths not only between concepts on adjacent layers.
For example, path ``$\text{\textit{subClassOf} \textit{subClassOf}}^{-1}$'' can be derived in $\mathcal{G}_2$, but it is a path between concepts on the same layer, not adjacent.
We changed the grammar to fit a query to a description provided in paper~\cite{CFGonRDF}. 
Thus results of our query is different from results for $Q_2$ which provided in paper~\cite{CFGonRDF}.

All tests were run on a x64-based PC with Intel(R) Core(TM) i7-4790 CPU @ 3.60GHz and 32 GB RAM. 
Results of both queries are presented in table~\ref{tbl1}, where \#triples is a number of $(o,p,s)$ triples in RDF file, and \#results is a number of triples of form $(S,v_1,v_2)$.
In our approach result triples can be founded by filtering out all SPPF nonterminal nodes labeled by $(v_1,S,v_2)$.

\begin{figure*}[ht]
   \begin{center}
   \centering
   \begin{subfigure}[b]{0.4\textwidth}

   \[
\begin{array}{rl}
   0: & S \rightarrow \text{\textit{subClassOf}}^{-1} \ S \ \text{\textit{subClassOf}} \\ 
   1: & S \rightarrow \text{\textit{type}}^{-1} \ S \ \text{\textit{type}} \\ 
   2: & S \rightarrow \text{\textit{subClassOf}}^{-1} \ \text{\textit{subClassOf}} \\ 
   3: & S \rightarrow \text{\textit{type}}^{-1} \ \text{\textit{type}} \\ 
\end{array}
\]
   \caption{Grammar for query 1}
   \label{grammarQ1}
   \end{subfigure}
   \hspace{2em}
   \begin{subfigure}[b]{0.4\textwidth}
   \[
\begin{array}{rl}
   0: & S \rightarrow B \ \text{\textit{subClassOf}} \\ 
   1: & B \rightarrow \text{\textit{subClassOf}}^{-1} \ B \ \text{\textit{subClassOf}} \\
   2: & B \rightarrow \text{\textit{subClassOf}}^{-1} \ \text{\textit{subClassOf}} \\ 
\end{array}
\]
   \caption{Grammar for query 2}
   \label{grammarQ2}        
   \end{subfigure}
   \end{center}
   \caption{Grammars for evaluation}
    \label{GrammarsForEvaluation}
\end{figure*}



\begin{table*}
\centering
\caption{Evaluation results for Query 1 and Query 2}
\label{tbl1}

\begin{tabular}{ | c | c | c | c | c | c |}
\hline
Ontology & \#triples & \multicolumn{2}{|c|}{Query 1} & \multicolumn{2}{|c|}{Query 2} \\
\cline{3-6}
& & time(ms) & \#results & time(ms) & \#results \\
\hline 
\hline
skos        & 252 & 10 & 810 & 1 & 1 \\
generations & 273 & 19 & 2164 & 1 & 0 \\
travel      & 277 & 24 & 2499 & 1 & 63 \\
univ-bench  & 293 & 25 & 2540 & 11 & 81 \\
foaf        & 631 & 39 & 4118 & 2 & 10 \\
people-pets & 640 & 89 & 9472 & 3 & 37 \\
funding     & 1086 & 212 & 17634 & 23 & 1158 \\
atom-primitive & 425 & 255 & 15454 & 66 & 122 \\
biomedical-measure-primitive & 459 & 261 & 15156 & 45 & 2871 \\
pizza       & 1980 & 697 & 56195 & 29 & 1262 \\
wine        & 1839 & 819 & 66572 & 8 & 133 \\
\hline
\end{tabular}

\end{table*}

As a result, we conclude that our algorithm is fast enough to be applicable to some real-world problems.

