\subsection{Complexity}

Time complexity estimation in terms of input graph and grammar size is quite similar to the estimation of GLL complexity provided in~\cite{gllParsingTree}.

\begin{lemma}\label{lem:Descriptors}
For any descriptor $(L,u,i,w)$ either $w = \$$ or $w$ has extension $(j,i)$ where u has index $j$.
\end{lemma}
\begin{proof}
Proof of this lemma is the same as provided for original GLL in~\cite{gllParsingTree} because main function used for descriptors creation has not been changed.
\end{proof}


\begin{mytheorem}\label{thm:GSSSpace}
The GSS generated by GLL-based graph parsing algorithm for grammar $G$ and input graph $M=(V,E,L)$ has at most $O(|V|)$ vertices and $O(|V|^2)$ edges.
\end{mytheorem}

\begin{proof}

Proof is the same as the proof of \textbf{Theorem 2} from~\cite{gllParsingTree} because structure of GSS has not been changed. 

\end{proof}

\begin{mytheorem}\label{thm:SPPFSpace}
The SPPF generated by GLL-based graph parsing algorithm on input graph $M=(V, E, L)$ has at most $O(|V|^3 + |E|)$ vertices and edges.
\end{mytheorem}

\begin{proof}
Let us estimate the number of nodes of each type.
\begin{itemize}
\item \textbf{Terminal nodes} are labeled with $(v_0, T, v_1)$, and such label can only be created if there is such $e \in E$ that $e=(v_0, T,v_1)$. 
Note, that there are no duplicate edges. 
Hence there are at most $|E|$ terminal nodes.
\item \textbf{$\varepsilon$-nodes} are labeled with $(v, \varepsilon, v)$, hence there are at most $|V|$ of them. 
\item \textbf{Nonterminal nodes} have labels of form $(v_0, N, v_1)$, so there are at most $O(|V|^2)$ of them.
\item \textbf{Intermediate nodes} have labels of form $(v_0, t, v_1)$, where $t$ is a grammar slot, so there are at most $O(|V|^2)$ of them.
\item \textbf{Packed nodes} are children either of intermediate or nonterminal nodes and have label of form $(N \rightarrow \alpha \cdot \beta, v)$.
There are at most $O(|V|^2)$ parents for packed nodes and each of them can have at most $O(|V|)$ children.
\end{itemize}

As a result, there are at most $O(|V|^3 + |E|)$ nodes in SPPF.

The packed nodes have at most two children so there are at most $O(|V|^3 + |E|)$ edges which source is packed node. 
Nonterminal and intermediate nodes have at most $O(|V|)$ children and all of them are packed nodes.
Thus there are at most $O(|V|^3)$ edges with source in nonterminal or intermediate nodes. As a result there are at most $O(|V|^3 + |E|)$ edges in SPPF.


\end{proof}

\begin{mytheorem}
The worst-case space complexity of GLL-based graph parsing algorithm for graph $M=(V,E,L)$ is $O(|V|^3 + |E|)$.
\end{mytheorem}

%\begin{proof}

Immediately follows from theorems~\ref{thm:GSSSpace} and~\ref{thm:SPPFSpace}. 

%\end{proof}


\begin{mytheorem}\label{thm:complexity}
The worst-case runtime complexity of GLL-based graph parsing algorithm for graph $M=(V,E,L)$ is $$O\left(|V|^3*\max\limits_{v \in V}\left(deg^+\left(v\right)\right)\right).$$
\end{mytheorem}

\begin{proof}

From Lemma~\ref{lem:Descriptors}, there are at most $O(|V|^2)$ descriptors. 
Complexity of all functions which were used in algorithm is the same as in proof of \textbf{Theorem 4} from~\cite{gllParsingTree} except \textbf{Processing} function in which not a single next input token, but the whole set of outgoing edges, should be processed.
Thus, for each descriptor at most $$\max\limits_{v \in V}\left(deg^+\left(v\right)\right)$$ edges  are processed, where $deg^+(v)$ is outdegree of vertex $v$.

Thus, worst-case complexity of proposed algorithm is $$O\left(V^3*\max\limits_{v \in V}\left(deg^+\left(v\right)\right)\right).$$
\end{proof}

%Also we can get averege-case complexity by calculate averege outdegree:
%\begin{align} \label{eq:avg}
%  & O\left(|V|^3*\frac {\sum\limits_{v \in V} deg^+(v)}{|V|}\right) = \nonumber \\
%  & O\left(|V|^2*\sum\limits_{v \in V} deg^+(v)\right) = \nonumber \\
%  & O\left(|V|^2*|E|\right) 
%\end{align}

We can get estimations for linear input from theorem~\ref{thm:complexity}. $\text{For any } v \in V$, $deg^+(v) \leq 1$, thus $\max\limits_{v \in V}(deg^+(v))  = 1 $ and worst-case time complexity $O(|V|^3)$, as expected. 
For LL grammars and linear input complexity should be $O(|V|)$ for the same reason as for original GLL.
 
As discussed in~\cite{modellingGLL}, special data structures, which are required for the basic algorithm, can be not rational for practical implementation, and it is necessary to find balance between performance, software complexity, and hardware resources.
As a result, we can get slightly worse performance than theoretical estimation in practice.

Note that result SPPF contains only paths matched specified query, so result SPPF size is $O(|V'|^3 + |E'|)$ where $M'=(V',E',L')$ is a subgraph of input graph $M$ which contains only matched paths.
Also note that each specific path can be explored by linear SPPF traversal. 
