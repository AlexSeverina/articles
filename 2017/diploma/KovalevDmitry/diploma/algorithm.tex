\section{Алгоритм синтаксического анализа \\ контекстно-свободного представления}

Разработанный алгоритм расширяет подход к синтаксическому анализу графов на основе GLL. На вход алгоритм принимает два рекурсивных автомата, $RA_1$ и $RA_2$, построенных по исходным грамматикам $G_1$ и $G_2$ соответственно, где $G_1$ --- произвольная контекстно-свободная грамматика, а $G_2$, исходя из ограничений разрешимости задачи, --- грамматика, не содержащая непосредственной или скрытой рекурсии.

Алгоритм производит обход автоматов, рассматривая три типа ситуаций (финальное состояние, терминальный/нетерминальный переходы) для каждого из их состояний. Автомат, состояние которого обрабатывается в данный момент, будем называть \textit{основным}; другой автомат из пары назовем \textit{побочным}.

Оригинальный алгоритм синтаксического анализа графов использует один GSS, необходимый для правильного обхода рекурсивного автомата, который является представлением эталонной грамматики. Для работы с двумя рекурсивными автоматами требуется поддерживать одновременно два стека, $GSS_1$ и $GSS_2$. При этом используется модификация GSS с измененной структурой вершин: здесь они представляют собой тройку $(S, i, u)$, где: 
\begin{itemize}
	\setlength\itemsep{0em}
	\item $S$ --- нетерминал;
	\item $i$ --- состояние побочного автомата;
	\item $u$ --- вершина стека побочного автомата.
\end{itemize}
Пара $(i, u)$ описывает состояние обхода побочного автомата на момент начала разбора нетерминала $S$. В оригинальном алгоритме достаточно было сохранять в вершине GSS лишь состояние входного автомата, так как подразумевалось, что даный автомат относится к классу конечных и для его обхода не используется стек.

Дескрипторы, которые использует разработанный алгоритм, также отличаются от представленных ранее. Для описания процесса анализа в определенный момент времени теперь необходимо указывать вершины GSS для обоих автоматов. Получаем дескриптор вида $(S_1, S_2, u_1, u_2)$, где $S_1, S_2$ --- состояния автоматов, а $u_1, u_2$ --- вершины соответствующих стеков. Функции \textbf{add}, \textbf{create} и \textbf{pop} были изменены в соответствии с новыми определениями дескрипторов и вершин GSS.
\begin{algorithm}[H]
\begin{algorithmic}
\caption{Функции для работы с GSS и дескрипторами}
\Function{add}{$S_1, S_2, u_1, u_2$}
	\If{($(S_1, S_2, u_1, u_2) \notin U$)}
		\State $U.add(S_1, S_2, u_1, u_2)$
		\State $R.enqueue(S_1, S_2, u_1, u_2)$
	\EndIf
\EndFunction

\Function{create}{$S_{call}, S_{next}, u, S_i, w$}
	\State $A \gets \Delta(S_{call})$
	\If{($\exists$ GSS node labeled $(A, i, w)$)}  
		\State $v \gets$ GSS node labeled $(A, i)$
		\If{(there is no GSS edge from $v$ to $u$ labeled $S_{next}$)}
			\State add GSS edge from $v$ to $u$ labeled $S_{next}$
			\For{($(v, j) \in P$)}
				\If{($S_{next}$ is a final state)}
					\State \Call{pop}{$u, j, w$}
				\EndIf
				\State \Call{add}{$S_{next}, S_i, u, w$}
			\EndFor
		\EndIf
	\Else
		\State $v \gets$ \textbf{new} GSS node labeled $(A, i, w)$
		\State create GSS edge from $v$ to $u$ labeled $S_{next}$
		\State \Call{add}{$S_{call}, S_i, v, w$}
	\EndIf
\EndFunction
  
\Function{pop}{$u,S_i,w$}
	\If{($(u,i) \notin P$)}  
		\State $P.add(u,i)$
		\ForAll{GSS edges $(u,S,v)$}
			\If{($S$ is a final state)}
				\State \Call{pop}{$v, S_i, w$}
			\EndIf
			\State \Call{add}{$S, S_i, v, w$}
		\EndFor
	\EndIf
\EndFunction
\end{algorithmic}
\end{algorithm}

Обработка терминальных и нетерминальных переходов, а также финальных состояний автомата была вынесена из главного цикла алгоритма в отдельную функцию, которая вызывается для каждого из автоматов.

\begin{algorithm}[H]
\begin{algorithmic}
\caption{Функция для обработки состояния рекурсивного автомата}
\Function{handleState}{$RA, context$}
	\State $S_1, S_2, u_1, u_2 \gets$ get values for $RA$ from the $context$
		\If{($S_1$ is a final state)}
			\State \Call{pop}{$u_1,S_2,u_2$}
		\EndIf
	
		\For{\textbf{each} $transition (S_1,label,S_{next})$}
			\Switch{$label$}
			\Case{$Terminal(x)$}
				\For{\textbf{each} (input[i] $\xrightarrow[]{x}$ input[k])}
					\If{($S_{next}$ is a final state)}
						\State \Call{pop}{$u_1, k, u_2$}
					\EndIf
					\If{$S_{next}$ have multiple ingoing transitions}
						\State \Call{add}{$S_{next}, k, u_1, u_2$}
					\Else
						\State $R.enqueue(S_{next}, k, u_1, u_2)$
					\EndIf
				\EndFor
			\EndCase
			
			\Case{$Nonterminal(S_{call})$}
				\State \Call{create}{$S_{call}, S_{next}, u1, S_2, u_2$}
			\EndCase
			\EndSwitch
		\EndFor
\EndFunction
\end{algorithmic}
\end{algorithm}

Основная управляющая функция алгоритма, содержащая цикл обработки дескрипторов, представлена в листинге \ref{alg}. Необходимо отметить, что для поиска шаблонов-подстрок обход автомата $RA_2$ запускается из каждой его вершины. Соответствующие стартовые дескрипторы создаются перед входом в цикл.

\begin{algorithm}[H]
\begin{algorithmic}
\caption{Алгоритм синтаксического анализа КС-представления}
\Function{parse}{$RA_1, RA_2$}
	\For{\textbf{each} $q \in RA_2.States$}
		\State $v1, v2 \gets (RA_1.StartState, q, \$), (q, RA_1.StartState, \$)$
		\State add $v1$ to $GSS_1$, $v2$ to $GSS_2$
		\State $R.enqueue(RA_1.StartState, q, v1, v2)$
	\EndFor
	\While{$R \neq \emptyset$}
		\State $currentContext \gets R.dequeue()$
		\State \Call{handleState}{$RA_1, currentContext$}
		\State \Call{handleState}{$RA_2, currentContext$}
	\EndWhile
	\State $result \gets \emptyset$
	\For{\textbf{each} $(u, i) \in P$ where $u = GSS_1.Root$}
	\State $result.add(i)$
	\EndFor
	\If{$result \neq \emptyset$}
	\ return $result$
	\Else
	\ report failure
	\EndIf
\EndFunction
\label{alg}
\end{algorithmic}
\end{algorithm}

Результатом работы алгоритма являются пары вида $(n_1, n_2)$, где $n_1, n_2$ --- номера состояний автомата $RA_2$. 
Для каждой из таких пар выполняется следующее утверждение: $\exists \, \omega \in T^*$ такая, что $\omega \in L(G_1)$ и $\omega \in L(RA')$, где $RA'$ --- рекурсивный автомат, полученный из $RA_2$ путем замены начального и конечного состояний на $n_1$ и $n_2$ соответственно.

\pagebreak
