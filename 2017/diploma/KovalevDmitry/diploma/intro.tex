\section*{Введение}

Контекстно-свободные грамматики, наряду с регулярными выражениями, повсеместно используются для решения задач, связанных с разработкой формальных языков и синтаксических анализаторов. Одним из основных достоинств контекстно-свободных грамматик является возможность задания широкого класса языков при сохранении относительной компактности представления. Благодаря данному свойству, грамматики также представляют интерес в такой области информатики, как кодирование и сжатие данных. В частности, существует ряд алгоритмов, позволяющих производить сжатие текстовой информации, используя в качестве конечного \cite{Sequitur} или промежуточного \cite{Arimura} представления контекстно-свободную грамматику (grammar-based compression).

Использование компактного представления текста может быть актуально для некоторых задач биоинформатики. Примером такой задачи является анализ метагеномных сборок --- помеченных графов большого размера: порядка $10^6$ ребер и $10^6$ вершин. Метками на ребрах графа являются строки над алфавитом нуклеотидных символов. В магистерской диссертации Анастасии Рагозиной \cite{Nastya} был предложен алгоритм, позволяющий искать в метагеномных сборках шаблоны РНК, задаваемые при помощи контекстно-свободной грамматики. Производительность данного алгоритма, предположительно, может быть увеличена за счет контекстно-свободного сжатия меток на ребрах графа и трансформации его в грамматику, представленную в расширенной форме Бэкуса-Наура.

Задача поиска шаблонов при использовании такого представления метагеномной сборки формулируется следующим образом: необходимо найти все строки, принадлежащие перечению языков, задаваемых грамматикой шаблона и грамматикой сборки. В общем случае такая задача неразрешима, так как сводится к проверке пересечения двух произвольных контекстно-свободных языков на пустоту. Поэтому необходимо выставить ограничения и вообще...