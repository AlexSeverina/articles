\section*{Введение}

Контекстно-свободные грамматики, наряду с регулярными выражениями, активно используются для решения задач, связанных с разработкой формальных языков и синтаксических анализаторов. 
Одним из основных достоинств контекстно-свободных грамматик является возможность задания широкого класса языков при сохранении относительной компактности представления. 
Благодаря данному свойству, грамматики также представляют интерес в такой области информатики, как кодирование и сжатие данных. 
В частности, существует ряд алгоритмов, позволяющих производить сжатие текстовой информации, используя в качестве конечного \cite{Sequitur} или промежуточного \cite{Arimura} представления контекстно-свободную грамматику (grammar-based compression). 

Стандартной процедурой при работе с текстовыми данными является поиск в них определенных шаблонов, которые могут быть заданы строкой или регулярным выражением. 
В настоящее время большие объемы информации, как правило, хранятся и передаются по сети в сжатом виде, поэтому актуальной задачей становится поиск шаблонов непосредственно в компактном контекстно-свободном представлении текста.
Такой подход позволяет избежать дополнительных затрат памяти на восстановление исходной формы данных и, в некоторых случаях, увеличивает скорость выполнения запроса.
Шаблон здесь может быть, как и при поиске в обычном тексте, строкой (compressed pattern matching), сжатой строкой (fully compressed pattern matching) или регулярным выражением.

Известны ситуации, в которых для задания шаблона необходимо использовать более выразительные средства. 
Примером может служить одна из задач биоинформатики --- поиск определенных подпоследовательностей в геноме организма. Так, для классификации и исследования образцов, полученных в результате секвенирования, в них могут искать гены, описывающие специфические рРНК. Структура таких генов, как правило, задается при помощи контекстно-свободной грамматики \cite{Anderson2013}. Для уменьшения объемов памяти, необходимых для хранения большого количества геномов, используются различные алгоритмы сжатия, в том числе основанные на получении контекстно-свободной структуры исходных последовательностей \cite{galle2011dna}.

Задача поиска КС-шаблонов при использовании КС-представления данных формулируется следующим образом: необходимо найти все строки, принадлежащие пересечению двух языков, один из которых задается грамматикой шаблона, а второй представляет собой язык всех подстрок исходного множества строк, описываемого грамматикой, полученной в результате сжатия данных.
Назовем такой поиск \textit{синтаксическим анализом данных, представленных в виде КС-грамматики}.
В общем случае задача неразрешима, так как сводится к задаче о проверке пересечения двух языков, порождаемых произвольными КС-грамматиками, на пустоту. 
Для постановки экспериментов в области биоинформатики необходимо точнее исследовать возможность проведения синтаксического анализа КС-представления и разработать прототип алгоритма, позволяющего решить данную задачу.