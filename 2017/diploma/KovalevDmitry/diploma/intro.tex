\section*{Введение}

Контекстно-свободные грамматики, наряду с регулярными выражениями, активно используются для решения задач, связанных с разработкой формальных языков и синтаксических анализаторов. 
Одним из основных достоинств контекстно-свободных грамматик является возможность задания широкого класса языков при сохранении относительной компактности представления. 
Благодаря данному свойству, грамматики также представляют интерес в такой области информатики, как кодирование и сжатие данных. 
В частности, существует ряд алгоритмов, позволяющих производить сжатие текстовой информации, используя в качестве конечного \cite{Sequitur} или промежуточного \cite{Arimura} представления контекстно-свободную грамматику (grammar-based compression). 
%Грамматика в данном случае не содержит рекурсивных правил или скрытой рекурсии, а язык, порождаемый ей, состоит из одной строки --- исходных данных, к которым был применен алгоритм сжатия.

Стандартной процедурой при работе с текстовыми данными является поиск в них определенных шаблонов, которые могут быть заданы строкой или регулярным выражением. 
В настоящее время большие объемы информации, как правило, хранятся и передаются по сети в сжатом виде, поэтому актуальной задачей становится поиск шаблонов непосредственно в компактном контекстно-свободном представлении текста.
Такой подход позволяет избежать дополнительных затрат памяти на восстановление исходной формы данных и, в некоторых случаях, увеличивает скорость выполнения запроса.
Шаблон здесь может быть, как и при поиске в обычном тексте, строкой (compressed pattern matching), сжатой строкой (fully compressed pattern matching) или регулярным выражением.

Известны ситуации, в которых для задания шаблона необходимо использовать более выразительные средства. 
Примером может служить одна из задач биоинформатики --- поиск и классификация организмов в метагеномных сборках. 
Метагеномная сборка представляет собой помеченный граф большого размера, на ребрах которого хранятся строки над алфавитом нуклеотидных символов. 
Такой граф, по сути, описывает конечный автомат, который задает некоторое регулярное множество геномов. 
Шаблоном для поиска в данном случае является характерная для организма часть генома, структура которой описывается контекстно-свободной грамматикой \cite{Anderson2013}. Эта задача может быть решена при помощи алгоритма синтаксического анализа регулярных множеств \cite{Nastya}.

Для получения более компактного представления метагеномной сборки можно использовать контекстно-свободное сжатие.
Задача поиска шаблонов при использовании  КС-представления сборки формулируется следующим образом: необходимо найти все строки, принадлежащие пересечению языков, задаваемых грамматикой шаблона и грамматикой сборки. 
Или, иначе, --- найти все строки, выводимые в одной грамматике и порождаемые другой. Назовем эту задачу \textit{синтаксическим анализом данных, представленных в виде КС-грамматики}.
В общем случае такая задача неразрешима, так как она сводится к задаче о проверке пересечения двух языков, порождаемых произвольными КС-грамматиками, на пустоту. Для проведения экспериментов с метагеномными сборками необходимо точнее исследовать возможность проведения синтаксического анализа КС-представления и разработать прототип алгоритма, позволяющего решить данную задачу.