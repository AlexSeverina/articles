\section{Обзор}

\subsection{Подходы к анализу данных, представленных в виде КС-грамматики}
Пусть $G$ --- произвольная КС-грамматика, $M$ --- конечный автомат. Тогда задача проверки
\begin{itemize}
	\item включения языков ($L(M) \subseteq L(G)$) --- неразрешима
	\item пустоты пересечения ($L(M) \cap L(G) = \emptyset$) --- разрешима (т.к. в пересечении не более чем КС-язык) за полиномиальное время \cite{Hunt}
	\item регулярности языка $L(G)$ --- неразрешима \cite{Greibach1968}
\end{itemize} 

Если использовать представление регулярного языка $L(M)$ в виде КС-грамматики $G_r$, то задача проверки пустоты пересечения ($L(G_r) \, \cap \, L(G) = \emptyset$) становится немного интереснее: если $G_r$ 
\begin{itemize}
	\item нерекурсивная --- задача из PSPACE \cite{Nederhof} (точнее результата нет (я не нашел, по крайней мере))
	\item лево- или праволинейная --- ничего не известно (см. последний абзац заключения из \cite{Nederhof})
	\item принадлежит еще более широкому классу --- тем более ничего не известно
\end{itemize}

Еще немного про вложенную рекурсию и регулярность языка. Грамматика без вложенной рекурсии (NSE) порождает регулярный язык \cite{Chomsky} (обратное тоже верно, для регулярного языка можно построить NSE грамматику, т.к. праволинейная, например, --- частный случай NSE). Существует алгоритм, который позволяет проверять грамматику на наличие вложенной рекурсии за полином \cite{Anselmo}. Однако, грамматика с вложенной рекурсией тоже может порождать регулярный язык \cite{Andrei2004}, поэтому задача о проверке регулярности языка, порождаемого КС-грамматикой, остается неразрешимой. 

\subsection{Обобщенный синтаксический анализ}
\subsection{Алгоритм GLL и его модификации}
\subsection{Проект YaccConstructor}