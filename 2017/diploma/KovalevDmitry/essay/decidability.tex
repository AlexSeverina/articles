\section{Разрешимость задачи синтаксического \\ анализа контекстно-свободного представления}
Как было сказано ранее, задачу поиска шаблона, при условии, что и шаблон, и данные, в которых осуществляется поиск, представлены контекстно-свободными грамматиками, мы назовем синтаксическим анализом контекстно-свободного представления. В данном разделе определяются ограничения, при которых подобная задача разрешима.

Для доказательства предложений, сформулированных далее, используется следующая теорема \cite{Nederhof}.

\begin{theorem}[Nederhof, Satta]
	Пусть $G_1$ --- произвольная контекстно-свободная грамматика, $G_2$ --- грамматика, которая не содержит непосредственной или скрытой рекурсии. Тогда проблема проверки пустоты пересечения языков, порождаемых данными грамматиками, относится к классу PSPACE-complete.
\end{theorem}

Рассмотрим случай, когда грамматика данных задает ровно одну строку. Пусть $G_t$ --- произвольная КС-грамматика, задающая шаблоны для поиска, а $G_d$ --- КС-грамматика, которая не содержит непосредственной или скрытой рекурсии. $L(G_t)$ и $L(G_d)$ --- языки, порождаемые грамматиками, при этом $L(G_d) = \{\omega\}$, где $\omega$ --- исходные данные, к которым был применен алгоритм сжатия. 
Необходимо определить, существуют ли такие строки $\omega'$, что $\omega' \in L(G_t)$ и $\omega'$ --- подстрока $\omega$. 

\begin{prop}
	При выполнении описанных условий задача синтаксического анализа КС-представления разрешима.
\end{prop}

\begin{proof}
Пользуясь эквивалентностью представлений, можно записать грамматику $G_d$ в виде рекурсивного автомата $R_d$. Рассмотрим рекурсивный автомат $R_{i,\,j}$, полученный из $R_d$ путем замены стартового состояния на $i \in Q(R_d)$ и назначения терминирующего (финального, из которого не могут быть совершены переходы) состояния $j \in Q(R_d)$. Такой автомат описывает грамматику, которая является представлением некоторой подстроки $\omega$. 
Рассмотрев все возможные пары $i$ и $j$, получаем конечное множество грамматик, для каждой из которых необходимо проверить, содержится ли строка, порождаемая ей, в языке $L(G_t)$. 
Согласно теореме 1, такая проверка является разрешимой задачей и принадлежит к классу PSPACE-complete.
\end{proof}

Отдельно отметим, что для описанных процедур используется лишь исходный автомат, эквивалентный грамматике $G_d$. 
Условия задачи поиска шаблонов непосредственно в контекстно-свободном представлении, таким образом, выполняются. 
Верна также разрешимость более общей задачи.

\begin{prop}
	Пусть грамматика $G_d$ задает конечное множество строк $L(G_d) = \{\omega_1, \, \dots \, , \omega_n \}$. Необходимо определить, существуют ли строки $\omega'$, для которых верно: $\omega' \in L(G_t)$ и $\omega'$ --- подстрока одной из строк $\omega_i \in L(G_d)$. Данная задача разрешима и принадлежит классу PSPACE-complete.
\end{prop}

\begin{proof}
	Как и в предыдущем доказательстве, используем запись грамматики в виде рекурсивного автомата $R_d$ и рассмотрим автоматы $R_{i, j}$. В данном случае каждый из этих автоматов представляет собой грамматику, которая порождает некоторое конечное множество подстрок исходных строк из $L(G_d)$. Проверка пустоты пересечения такой грамматики с $G_t$ также соответствует условиям теоремы 1.
\end{proof}

В случае, когда грамматика $G_d$ представляет собой бесконечный регулярный язык (т.е. содержит левую и/или правую рекурсию), разрешимость задачи поиска шаблонов установить не удается. Подход, использованный ранее в доказательстве предложений, не может быть применен, так как части рекурсивного автомата, представляющего грамматику $G_d$, также могут содержать рекурсивные переходы, что выходит за рамки условия теоремы 1. Проверка разрешимости и определение класса сложности задачи проверки пустоты пересечения произвольной и регулярной КС-грамматик в настоящее время остаются открытыми проблемами \cite{Nederhof}.