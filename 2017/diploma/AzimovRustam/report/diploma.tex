% Тут используется класс, установленный на сервере Papeeria. На случай, если
% текст понадобится редактировать где-то в другом месте, рядом лежит файл matmex-diploma-custom.cls
% который в момент своего создания был идентичен классу, установленному на сервере.
% Для того, чтобы им воспользоваться, замените matmex-diploma на matmex-diploma-custom
% Если вы работаете исключительно в Papeeria то мы настоятельно рекомендуем пользоваться
% классом matmex-diploma, поскольку он будет автоматически обновляться по мере внесения корректив
%

% По умолчанию используется шрифт 14 размера. Если нужен 12-й шрифт, уберите опцию [14pt]
\documentclass{matmex-diploma}
%\documentclass[14pt]{matmex-diploma-custom}
\usepackage{algpseudocode}
\usepackage{algorithm}
\usepackage{caption}
\usepackage{algorithmicx}
\usepackage{amssymb}
\usepackage{listings}
\usepackage{graphicx} 
\usepackage{subcaption}
\usepackage[flushleft]{threeparttable}
\usepackage{longtable}
\usepackage{epstopdf}
\usepackage{chngcntr}
%\counterwithin{listing}{chapter}
%\counterwithout{figure}{chapter}
%\counterwithout{table}{chapter}

\usepackage{minted}
\usepackage{verbments}

\begin{document}

\algtext*{EndWhile}% Remove "end while" text
\algtext*{EndIf}% Remove "end if" text
\algtext*{EndFor}% Remove "end for" text
\algtext*{EndFunction}% Remove "end function" text

\renewcommand{\lstlistingname}{Листинг}
\renewcommand\listingscaption{Листинг}

% Год, город, название университета и факультета предопределены,
% но можно и поменять.
% Если англоязычная титульная страница не нужна, то ее можно просто удалить.
\filltitle{ru}{
    chair              = {Математическое обеспечение и администрирование \\ информационных систем \\ \vspace{5mm} Системное Программирование},
    title              = {Обзор задач синтаксического анализа графов},
    % Здесь указывается тип работы. Возможные значения:
    %   coursework - Курсовая работа
    %   diploma - Диплом специалиста
    %   master - Диплом магистра
    %   bachelor - Диплом бакалавра
    type               = {coursework},
    position           = {студента},
    group              = 546,
    author             = {Азимов Рустам Шухратуллович},
    supervisorPosition = {к.\,ф.-м.\,н., ст.\,преп.},
    supervisor         = {Григорьев C.\,В.},
%    reviewerPosition   = {},
%    reviewer           = {},
    chairHeadPosition  = {д.\,ф.-м.\,н., профессор},
    chairHead          = {Терехов А.\,Н.},
%   university         = {Санкт-Петербургский Государственный Университет},
%   faculty            = {Математико-механический факультет},
%   city               = {Санкт-Петербург},
%   year               = {2013}
}
\filltitle{en}{
    type               = {coursework},
    chair              = {Software and Administration of Information Systems \\ \vspace{5mm} Software Engineering},
    title              = {Survey of graph parsing problems},
    author             = {Rustam Azimov},
    supervisorPosition = {Senior Lecturer},
    supervisor         = {Semen Grigorev},
%    reviewerPosition   = {},
%    reviewer           = {},
    chairHeadPosition  = {Professor},
    chairHead          = {Andrey Terekhov},
}
\maketitle
\tableofcontents

\chapter*{Введение}                         % Заголовок
\addcontentsline{toc}{chapter}{Введение}    % Добавляем его в оглавление
\textbf{Актуальность работы}

Статический анализ исходного кода является известной техникой  получения знаний о программе без её исполнения ~\cite{StaticCodeAnalysis3,StaticCodeAnalysis2,StaticCodeAnalysis1}. Статический анализ является неотъемлемой частью многих процессов, связанных с разработкой программного обеспечения (ПО), и может использоваться, например, для упрощения работы с кодом с помощью подсветки синтаксиса языка в программах, навигации по коду, реализации контекстных подсказок. Более того, статический анализ используется для обнаружения ошибок на ранних стадиях разработки, до запуска программы, а также для поиска различных семантических ошибок, которые не могут быть определены обычным синтаксическим анализом.  Также, статический анализ используется при решении задач трансформации исходного кода и реинжиниринге~\cite{reengANT}. Однако во многих языках программирования имеются конструкции, которые существенно затрудняют статический анализ. 

Например, широко используются динамические встроенные языки --- приложение, созданное на одном языке, генерирует программу на другом языке и передаёт её на выполнение в соответствующее окружение. Примерами могут служить динамические SQL-запросы к базам данных из приложений на Java, С++, С\#, формирование HTML-страниц в PHP-приложениях~\cite{DSQLISO,JSP,PHPmySQL}. Генерируемый код собирается из строк таким образом, чтобы в момент выполнения результирующая строка представляла собой корректную программу. Примеры использования встроенных языков представлены в листингах~\ref{lst:dsql1},~\ref{lst:JsJava} и~\ref{lst:PhPSqlHtml}. Следует отметить, что одна программа может генерировать код на нескольких языках (см. листинг~\ref{lst:PhPSqlHtml}). При этом возможно получение частей кода из разных источников (например, учитывать текстовый ввод пользователя, что часто используется для задания фильтров при конструировании SQL-запросов). Использование динамически формируемых программ  позволяет избежать дополнительных накладных расходов, присущих таким технологиям, как ORM\footnote{ORM или Object-Relational Mapping --- технология программирования, которая связывает базы данных с концепциями объектно-ориентированных языков программирования~\cite{ORM}.}, и достичь высокой производительности. Благодаря этому использование динамически генерируемых программ получило широкое распространение и применяется до сих пор. Вместе с этим, несмотря на появление новых технологий, динамическая генерация SQL-запросов активно используется и в настоящее время~\cite{DSQLInActiveUse}.

\fvset{frame=lines,framesep=5pt,fontsize=\small}\

\begin{listing}
    \begin{pyglist}[language=sql,numbers=left,numbersep=5pt]

CREATE PROCEDURE [dbo].[MyProc]  @TABLERes   VarChar(30)
AS
    EXECUTE ('INSERT INTO ' + @TABLERes + ' (sText1)' +
             ' SELECT ''Additional condition: '' + sName' +
             ' from #tt where sAction = ''1000000''')
GO
    \end{pyglist}
\caption{Код с использованием динамического SQL}
\label{lst:dsql1}
\end{listing} 
 
\fvset{frame=lines,framesep=5pt}
\begin{listing}
    \begin{pyglist}[language=java,numbers=left,numbersep=5pt]
import javax.script.*;  
public class InvokeScriptFunction {  
    public static void main(String[] args) throws Exception {  
        ScriptEngineManager manager = new ScriptEngineManager();  
        ScriptEngine engine = manager.getEngineByName("JavaScript");  
        // JavaScript code in a String  
        String script = 
            "function hello(name) { print('Hello, ' + name); }";  
        // evaluate script  
        engine.eval(script);  
        // javax.script.Invocable is an optional interface.  
        // Check whether your script engine implements or not!  
        // Note that the JavaScript engine implements
        // Invocable interface.  
        Invocable inv = (Invocable) engine;  
        // invoke the global function named "hello"  
        inv.invokeFunction("hello", "Scripting!!" );  
    }  
}
    \end{pyglist}
\caption{Вызов JavaScript из Java}
\label{lst:JsJava}
\end{listing}


\fvset{frame=lines,framesep=5pt}
\begin{listing}
    \begin{pyglist}[language=php,numbers=left,numbersep=5pt]

<?php
    // Embedded SQL
    $query = 'SELECT * FROM ' . $my_table; 
    $result = mysql_query($query);
    
    // HTML markup generation
    echo "<table>\n";
    while ($line = mysql_fetch_array($result, MYSQL_ASSOC)) {
        echo "\t<tr>\n";    
        foreach ($line as $col_value) {
            echo "\t\t<td>$col_value</td>\n";
        }
        echo "\t</tr>\n";
    }
    echo "</table>\n";
?>
    \end{pyglist}
\caption{Использование нескольких встроенных в PHP языков (MySQL, HTML)}
\label{lst:PhPSqlHtml}
\end{listing}



Динамически формируемые выражения часто конструируются с помощью таких операций, как конкатенация в циклах или условных предложениях, или в рекурсивных процедурах. Это затрудняет статический анализ и приводит к получению множества возможных значений для каждого выражения в момент выполнения. Вследствие этого фрагменты динамически формируемого кода воспринимаются компилятором исходного языка как простые строки, не подлежащие дополнительному анализу, а это, в свою очередь, приводит к высокой вероятности возникновения ошибок во время выполнения программы. В худшем случае такая ошибка не приведёт к прекращению работы приложения, что указало бы на проблемы, однако целостность данных при этом может оказаться нарушена. Более того, использование динамически формируемых выражений затрудняет не только разработку информационных систем, так и также и реинжиниринг, поскольку в последнем случае важно автоматизировать перенос системы на новые зыки и платформы, что невозможно без качественного статического анализа. Например, при наличии в коде приложения динамически формируемых SQL-запросов нельзя точно ответить на вопрос о том, с какими элементами базы данных не взаимодействует система, и удалить их. При переносе такой системы на другую СУБД необходимо гарантировать, что для всех динамически формируемых выражений значение в момент выполнения будет корректным кодом на языке новой СУБД~\cite{JSquash}. Следует отметить, что отсутствие статического анализа динамически формируемых программ не позволяет реализовывать для них стандартную функциональность интегрированных сред разработки (Integrated Development Environment, IDE) --- подсветку синтаксиса и автодополнение, рефакторинг кода и т.д. Такая функциональность значительно упрощает процесс разработки и отладки приложений и полезна не только для основного языка, но и для встроенных языков. 

Для решения всех перечисленных выше задач необходимы инструменты, проводящие статический анализ динамически формируемых программ. Такой анализ может дать существенную информацию о таких программах, поскольку редко встречается ситуация полной динамической неопределённости (например, при создании динамических программ исключительно на основе пользовательского ввода). В большинстве случаев, имея программу, генерирующую динамические вставки, с помощью статического анализа можно получить достаточно информации для решения поставленных выше задач. Решению этой проблемы и посвящена данная диссертационная работа. 


\textbf{Степень разработанности темы}

Существуют классические исследования, посвященные разработке компиляторов --- работы А.~Ахо~\cite{Dragon}, А.~Брукера~\cite{CompilerCompiler}, С.~Джонсона~\cite{yaccBook},  Б.К.Мартыненко~\cite{Martinenko1, Martinenko2}  и др.  Однако содержащиеся там алгоритмы синтаксического анализа не могут быть применены к решению задачи анализа динамически формируемых программ, поскольку предназначены для обработки входных данных, представимых в видн линейной последовательности символов, а такое представление динамически формируемых программ не всегда возможно.

Методы обобщённого синтаксического анализа, лежащие в основе данной работы, изложены в трудах таких учёных как Масару Томита (Masaru Tomita)~\cite{Tomita}, Элизабет Скотт (Elizabeth Scott) и Адриан Джонстон (Adrian Johnstone)~\cite{RNGLR,RIGLR} из университета Royal Holloway (Великобритания), Ян Рекерс (Jan Rekers, University of Amsterdam)~\cite{SPPF}, Элко Виссер (Eelco Visser)~\cite{RNGLRSyntaxerror2,RNGLRSyntaxerror3} и других.

Анализу динамически формируемых строковых выражений посвящены работы таких зарубежных учёных как Кюнг-Гу Дох (Kyung-Goo Doh)~\cite{LrAbstract1,LrAbstract2,LRAbstractParsingSema}, Ясухико Минамиде (Minamide Yasuhiko)~\cite{PHPSA}, Андерс Мёллер (Anders M{\o}ller)~\cite{JSA} и отечественных учёных А.А.~Бреслава~\cite{Alvor1,Alvor2} и других. Хорошо изучены вопросы проверки корректности динамически формируемых выражений и поиска фрагментов кода, уязвимых для SQL-инъекций~\cite{SQLInjection,Dasgupta:2009:SAF:1546683.1547548}. Однако данные работы исследуют отдельные аспекты проблемы статического анализа динамически формируемых программ, оставляя в стороне создание готовых алгоритмов (в частности, не строят структурное представление анализируемых программ). В связи с этим возникают проблемы масштабируемости данных результатов, например, создание на их основе более сложных видов статического анализа.

Так же важным является предоставление компонентов, упрощающих создание новых инструментов для решения конкретных задач. Данных подход хорошо исследован в области разработки компиляторов, где широкое распространение получили генераторы анализаторов и пакеты стандартных библиотек (работы А.~Ахо~\cite{Dragon}, А.~Брукера~\cite{CompilerCompiler}, С.~Джонсона~\cite{yaccBook} и др.). 

В работах отечественных учёных М.Д.~Шапот и Э.В.~Попова~\cite{DynamicDSQLTranslation}, а так же зарубежных учёных Антони Клеви (Anthony Cleve), Жан-Люк Эно (Jean-Luc Hainaut)~\cite{DSQLReverseEngineering}, Йост Виссер (Joost Visser)~\cite{DSQLQualityMesure} и других рассматриваются различные аспекты реинжиниринга информационных систем, использующих встроенные SQL-запросы, однако не формулируется общего метода для решения таких задач. Этот вопрос также не затрагивается в классических работах, посвященных реинжиниригу~\cite{SoftwareReeng1, reengANT, SoftwareReeng2, SoftwareReeng3}. Однако разработка такого метода является актуальной задачей.

Таким образом, актуальной является задача дальнейшего исследования статического анализа динамически формируемых строковых выражений. Кроме этого важным является решение вопросов практического применения средств анализа динамически формируемого кода: упрощение разработки инструментов анализа и создание методов их применения в реинжиниринге программного обеспечения.
\textbf{Объект исследования}

Объектом исследования являются методы, алгоритмы и программные средства обработки динамически формируемых программ, а также задача реинжиниринга информационных систем.

\textbf{Цель и задачи диссертационной работы}

\textbf{Целью} данной работы является создание комплексного подхода к статическому синтаксическому анализу динамически формируемых программ.

Достижение поставленной цели обеспечивается решением следующих \textbf{задач}.
\begin{enumerate}
    \item Разработать универсальный алгоритм синтаксического анализа динамически формируемых программ, не зависящий от целевого языка программирования и допускающий реализацию различных видов статического анализа. 
    \item Создать архитектуру инструментария для автоматизации разработки программных средств статического анализа динамически формируемых программ.
    \item Создать метод реинжиниринга динамически формируемых программ.
\end{enumerate}

\textbf{Методология и методы исследования}

Методология исследования основана на подходе к спецификации и анализу формальных языков, активно развивающемуся с 50-х годов 20-го века (см., например, работы Н. Хомского~\cite{chomskyMethod ,chomskySyntactic}). В последствии этот подход получил широкое распространение в областях, связанных с обработкой языков программирования.
Основными элементами данного подхода являются алфавит и грамматика языка, разбиение автоматической обработки языка на выполнение таких шагов как лексический, синтаксический и семантический анализ. Решаемые в связи с этим задачи связаны с поиском эффективных алгоритмов, выполняющих эти шаги. 

В работе применяется алгоритм обобщённого восходящего синтаксического анализа RNGLR~\cite{RNGLR}, созданный Элизабет Скотт (Elizabeth Scott) и Адриан Джонстон (Adrian Johnstone) из университета Royal Holloway (Великобритания). Для компактного хранения леса вывода использовалась структура данных Shared Packed Parse Forest (SPPF), которую предложил Ян Рекерс (Jan Rekers, University of Amsterdam)~\cite{SPPF}.

Доказательство завершаемости и корректности предложенного алгоритма проводилось с применением теории формальных языков, теории графов и теории сложности алгоритмов. Приближение множества значений динамически формируемого выражения строилось в виде регулярного множества, описываемого с помощью конечного автомата.


\textbf{Положения, выносимые на защиту}
\begin{enumerate}
    \item Разработан алгоритм синтаксического анализа динамически формируемых программ, позволяющий обрабатывать произвольную регулярную аппроксимацию множества значений выражения в точке выполнения, реализующий эффективное управление стеком и гарантирующий конечность представления леса вывода. Доказана завершаемость и корректность предложенного алгоритма при обработке регулярной аппроксимации, представимой в виде произвольного конечного автомата без $\varepsilon$-переходов. 
    \item Создана архитектура инструментария для разработки программных средств статического анализа динамически формируемых программ.
    \item Разработан метод анализа и обработки динамически формируемых программ в проектах по реинжинирингу информационных систем. 
\end{enumerate}

\textbf{Научная новизна работы}

Научная новизна полученных в ходе исследования результатов заключается в следующем.

\begin{enumerate}

\item Алгоритм, предложенный в диссертации, отличается от аналогов (работы Андрея Бреслава~\cite{Alvor1, Alvor2}, Кюнг-Гу Дох~\cite{LrAbstract1, LrAbstract2}, Ясухико Минамиде~\cite{PHPSA}) возможностью построения компактной структуры данных, содержащей деревья вывода для всех корректных значений выражения. Это позволяет использовать результаты работы алгоритма для проведения более сложных видов анализа. Алгоритмы, представленные в (JSA~\cite{JSA}~\cite{Alvor1, Alvor2}, PHPSA~\cite{PHPSA}) предназначены только для проверки корректности выражений, основанной на решении задачи о включении одного языка в другой. Выполнение более сложных видов анализа, трансформаций или построения леса разбора не предполагается. 

\item Новизна представленной архитектуры заключается в том, что она позволяет создать платформу для разработки целевых инструментов, решающих широкий круг задач анализа динамически формируемого кода. Существующие архитектуры готовых инструментов (JSA, PHPSA, Alvor, Varis) предназначены для решения конкретных задач для определённых языков. Решение новых задач или поддержка других языков с помощью этих инструментов затруднены ввиду ограничений, накладываемых архитектурой и возможностями используемого алгоритма анализа. 

\item Метод анализа и обработки встроенного программного кода в проектах по реинжинирингу информационных систем предложен впервые. К.В.~Ахтырченко и Т.П.~Сорокваша отмечают~\cite{SoftwareReengMethods}, что существующие работы в области реинжиниринга программного обеспечения либо содержат высокоуровневые решения, не касающиеся деталей, важных при решении прикладных задач (например, работы К. Вагнера~\cite{SoftwareReeng3}, Х. Миллера~\cite{SoftwareReeng2}), либо являются набором подходов к решению конкретных задач (например, работы~\cite{SoftwareReeng1, reengANT, boulychev}). При этом, встроенный программный код часто не учитывается. С другой стороны, работы М.Д.~Шапот и Э.В.~Попова~\cite{DynamicDSQLTranslation}, С.Л.~Трошина~\cite{reengANT}, А.~Клеви~\cite{DSQLReverseEngineering}  посвящены решению конкретных задач обработки встроенного программного кода в контексте реинжиниринга информационных систем, но не предлагают обобщённого и масштабируемого метода.

\end{enumerate}


\textbf{Теоретическая и практическая значимость работы}

Теоретическая значимость диссертационного исследования заключается в разработке формального алгоритма синтаксического анализа динамически формируемого кода, решающего задачу построения конечного представления леса вывода, не решенную полностью ранее, а также в формальном доказательстве завершаемости и корректности разработанного алгоритма. 

На основе полученных в работе научных результатов был разработан инструментарий (Software Development Kit, SDK), предназначенный для создания средств статического анализа динамически формируемых программ. 
С использованием разработанного инструментария было реализовано расширение к инструменту ReSharper (ООО ``ИнтеллиДжей Лабс'', Россия), предоставляющее поддержку встроенного T-SQL в проектах на языке программирования C\# в среде разработки Microsoft Visual Studio. Так же было выполнено внедрение результатов работы в промышленный проект по переносу хранимого SQL-кода с MS-SQL Server 2005 на Oraclе 11gR2 (ЗАО ``Ланит-Терком'', Россия). 

\textbf{Степень достоверности и апробация результатов}

Достоверность и обоснованность результатов исследования опирается на использование формальных методов исследуемой области, проведенные доказательства, рассуждения и эксперименты.

Основные результаты работы были доложены на ряде научных конференций: SECR-2012, SECR-2013, SECR-2014, TMPA-2014, Parsing@SLE-2013, Рабочий семинар ``Наукоемкое программное обеспечение'' при конференции PSI-2014. Доклад на SECR-2014 награждён премией Бертрана Мейера за лучшую исследовательскую работу в области программной инженерии. Дополнительной апробацией является то, что разработка инструментальных средств на основе предложенного алгоритма была поддержана Фондом содействия развитию малых форм предприятий в технической сфере (программа УМНИК~\cite{UMNIC}, проекты \textnumero~162ГУ1/2013 и \textnumero~5609ГУ1/2014).

\textbf{Публикации по теме диссертации}

Все результаты диссертации изложены в 7 научных работах, из которых 3~\cite{YCArticle,SELforIDEru,AbstractGLL}, содержащие основные результаты, опубликованы в журналах из ‘’Перечня российских рецензируемых научных журналов, в которых должны быть опубликованы основные научные результаты диссертаций на соискание ученых степеней доктора и кандидата наук’’, рекомендовано ВАК. 
1 работа~\cite{GLRAbsPars} индексируются Scopus. В работах, написанных в соавторстве, вклад автора определяется следующим образом.  В~\cite{Syrcose} С. Григорьеву принадлежит реализация ядра платформы YaccConstructor. В~\cite{SELforIDEru, AbstractGLL} и~\cite{SELforIDE} С. Григорьеву принадлежит постановка задачи, формулирование требований к разрабатываемым инструментальным средствам, работа над текстом. 
В~\cite{GLRAbsPars} автору принадлежит идея, описание и реализация анализа встроенных языков на основе RNGLR алгоритма.  В~\cite{YCArticle} С. Григорьеву принадлежит реализация инструментальных средств, проведение экспериментов, работа над текстом. В работе~\cite{RelaxedARNGLR} автору принадлежит алгоритм синтаксического анализа динамически формируемого кода.


\textbf{Структура работы}

Диссертация состоит из введения, шести глав, заключения и построена следующим образом. В первой главе приводится обзор области исследования. Рассматриваются подходы к анализу динамически формируемых строковых выражений и соответствующие инструменты. Описывается алгоритм обобщённого восходящего синтаксического анализа RNGLR, взятый за основу в диссертационной работе. Также описываются проекты YaccConstructor и ReSharper SDK, использованные для реализации предложенного в работе инструментария. Во второй главе формализуется основная задача исследования и излагается решающий её алгоритм синтаксического анализа регулярных множеств. Приводится доказательство завершаемости и корректности представленного алгоритма, приводятся примеры. В третьей главе описывается инструментальный пакет YC.SEL.SDK, разработанного на основе алгоритма, описанного во второй главе и предназначеного для разработки инструментов анализа динамически формируемых программ. Описывается архитектура компонентов и особенности их реализации. Также описывается YC.SEL.SDK.ReSharper --- ``обёртка'' для YC.SEL.SDK, позволяющая создавать расширения к ReSharper для поддержки встроенных языков. В четвёртой главе описывается метод реинжиниринга встроенного программного кода.  В пятой главе приводятся результаты экспериментального исследования разработанного алгоритма и инструмента YC.SEL.SDK. Шестая глава содержит результаты сравнения и соотнесения полученных результатов с  существующими аналогами.

\textbf{Благодарности}

А.Н.Терехову, работкникам и администрации компании ЗАО ``Ланит-Терком'' за создания условий для изучения данной темы (организация проектов по реинжинирингу). Я.А.Кириленко за погружение в тему исследования и руководство на начальных этапах. Д.Ю.Булычеву за помощь в уточнении постановки задачи исследования и в написании статей. Студентам и аспирантам кафедры системного программирования Дмитрию Авдюхину, Анастасии Рагозиной, Екатерине Вербицкой, Марине Полубеловой, Иванову Андрею за помощь в реализации предложенных идей и проведение экспериментов. Отдельную благодарность  хочется выразить компании ООО ``ИнтеллиДжей Лабс'' и Андрею Иванову за поддержку исследований. Также хочется поблагодарить А.К.Петренко и В.М.Ицыксона, а также сотрудников ИСП РАН за ценные вопросы и комментарии к работе, позволившие уточнить ряд формулировок и улучшить изложение результатов. 

\section{Постановка задачи}
Целью данной работы является разработка алгоритма синтаксического анализа данных, представленных в виде контекстно-свободной грамматики. Для ее достижения были поставлены следующие задачи.
\begin{itemize}
	\item Определить ограничения, при которых синтаксический анализ \linebreak контекстно-свободного представления является разрешимой задачей.
	\item Разработать алгоритм синтаксического анализа КС-представления данных с учетом поставленных ограничений.
	\item Реализовать предложенный алгоритм.
	\item Провести экспериментальное исследование.
\end{itemize}
%\section{Related Works}

Our approach for syntax analysis of string-embedded languages borrows some common principles
from existing techniques in this area. In addition, we reuse RNGLR syntax analysis algorithm 
and some accompanying constructs. In this section we provide a review and recollect some important
notions which will be referred to later on. 

The analysis of string-embedded languages, as a rule, requires a set of \emph{hotspots} to
be indicated in the host application source code. Hotspot is considered as some ``point 
of interest'', where the analysis of the set of possible string values is desirable. This task can be
performed either in a user-assisted manner or automatically using some pragmatic 
considerations or knowledge of the framework being analyzed. The following logical steps 
include static analysis to construct an approximation for the set of all possible string values,
lexical, syntax, and, perhaps, some kind of semantic analysis. These steps not
necessarily performed separatedly; some of them may be omitted.

A rather natural idea of \emph{regular approximation} is to approximate the set of all possible 
strings by a regular expression. In recognition-centric formulation, this approach boils down to
the problem of inclusion of approximating regular language into context-free reference language which
is decidable for a number of practically significant cases~\cite{LangInclusion}.
Many approaches follow this route. In~\cite{Stranger}, forward reachability analysis is used to compute regular 
approximation for all string values in the program. Further analysis is based on patterns detection in approximation 
set or generation of some finite subset of strings for analysis by standalone tools. Regular approximation in~\cite{JSA} 
is acquired by widening context-free approximation, initially built as a result of program analysis. 
Our approach is partially inspired by Alvor~\cite{Alvor,ALVOR2} which utilizes GLR-based technique for syntax 
analysis of regular approximation; this framework implements abstract lexical analysis to convert a
regular language over characters into regular language over tokens which simplifies syntax analysis.

Kyung-Goo Doh et al. in a series of papers~\cite{AbstrParsing,LRAbstrParsing,LRAbstrParsingSema} introduced an
approach, based on implicit representation of the set of potential strings as a system of data-flow equations. 
Conventional LALR(1) is chosen for the basis of parsing algorithm; original control tables are reused. 
Syntax analysis is performed as the system of dataflow equations is being solved iteratively in the space of abstract stacks.
The problem of infinite stack growth, which appears in general case, is handled using abstract 
interpretation~\cite{AbstractInterpretation}. This approach later evolved to a certain kind of semantic processing
in terms of attribute grammars which made it possible to analyze a wider class of languages than
LALR(1).

RNGLR (Right-Nulled Genralized LR) is a modification of Generalized LR (GLR) algorithm, which
was developed by Masaru Tomita~\cite{Tomita} in the context of natural language processing. 
GLR was designed to handle ambiguous context-free grammars. Ambiguities in the grammar produce 
shift/reduce and reduce/reduce conflicts, speaking in terms of LR approach. The algorithm 
uses parser tables, similar to those in classical LR, each cell of which can contain multiple 
actions. The general approach is to carry out all possible actions during parsing 
using graph-based data structures to efficiently represent the set of stacks 
and derivation trees. Originally, Tomita's algorithm was unable to recognize all context-free languages.  
Elizabeth Scott and Adrian Johnstone presented RNGLR~\cite{RNGLR},
which extends GLR with a certain way of handling \emph{right nullable} 
rules (i.e. rules of the form $\mathrm{A} \rightarrow \alpha \beta$, where $\beta$ 
reduces to the empty string).

To efficiently represent the set of all stacks, produced during parsing,
RNGLR uses Graph Structured Stack (GSS). GSS is a directed graph,
whose vertices correspond to the elements of individual stacks and edges link succesive
stack elements. Each vertex can have multiple incoming and outgoing edges to merge 
multiple stacks together; thus stack element sharing is implemented. Each vertex is 
a pair $(s,l)$, where $s$ is a parser state and $l$ is a \emph{level} (position in the input string). 
Vertices in GSS are unique and there are no multi-edges. 

According to RNGLR, an input is read left-to-right, one token at a time, and 
the levels of GSS are constructed sequentially for each input position: first, all  
possible reductions are applied, then the next input terminal is shifted and
pushed to the GSS. When a reduction or pushing is performed, 
the algorithm modifies GSS in the following manner. Suppose an 
egde $(v_t,v_h)$ has to be added to the GSS. By construction, the head vertex
$v_h$ is always already in the GSS. If the tail vertex is also in the GSS, then
a new egde $(v_t,v_h)$ is added (provided it is not yet there); otherwise both 
new tail vertex and new edge are created and added to the GSS. Every time a new 
vertex $v=(s,l)$ is created, the algorithm calculates the new parser 
state $s'$ from $s$ and the next terminal of the input. The pair $(v,s')$, called 
\emph{push}, is added to the global collection $\mathcal{Q}$. The set of $\epsilon$-reductions 
is also calculated, when a new vertex is added to the GSS, and reductions from this set are added to the 
global queue $\mathcal{R}$. Reductions with length $l>0$ are calculated and added to $\mathcal{R}$ 
each time a new (non-$\epsilon$) edge is created. 

An input string can have several derivation trees and, as a rule, they can have 
numerous identical subtrees. Shared Packed Parse Forest (SPPF)~\cite{SPPF} is a directed graph
designed for a compact representation of all possible derivation trees.  
SPPF has the following structure: the \emph{root} (i.e. vertex with no incoming edges) corresponds 
to the starting nonterminal of the grammar; vertices with no outgoing edges correspond to the terminals 
or $\epsilon$-tree (which means for the nonterminal $A$ in parental site $\mathrm{A} \Rightarrow \epsilon$); 
the rest of the vertices are divided into two classes: \emph{nonterminal} and \emph{production}. 
Each nonterminal vertex keeps a collection of production nodes, each of which represents one  
possible derivation of that nonterminal. Production vertices represent a right-hand side of the 
production and keep an ordered list of terminal or nonterminal nodes. This list length lies
in the range $[l-k..l]$, where $l$ is the length of the production right-hand side, and $k$ is 
the number of final symbols which derive $\epsilon$ (nullable symbols are ignored to reduce memory consumption).

SPPF is constructed simultaneously with GSS. Each edge of the GSS is associated with either 
a terminal or nonterminal node. When a GSS edge is added with a push, 
a new terminal node is created and associated with the edge. Nonterminal nodes are associated
with edges which were added when reductions were performed: if the edge has already been in GSS, 
a production node is added to the family of nonterminal nodes, associated with the edge. All subgraphs 
from the edges of the reduction path are added as children to the production node. After the input 
is read to the end, all vertices with accepting states are searched and nodes associated with 
outgoing edges of such vertices are merged to form the resulting SPPF. All unreachable vertices 
are deleted from the SPPF graph, which leaves only the actual derivation trees for the input.

The detailed algorithm description in the form of pseudocode can be found in Appendix~\ref{RNGLRCode}.
\begin{tabular}{|P{1.4cm}|P{1.4cm}|P{1.4cm}|P{1.4cm}|P{1.4cm}|P{1.4cm}|}
\hline
Length & Samples & Alignment & Precision & Recall & F1 score \\ \hline \hline
\multirow{2}{*}{90} & \multirow{2}{*}{26511} & $\times$ & 67\% & 75\% & 68\% \\ \cline{3-6} 
 &  & \checkmark & 80\% & 66\% & 70\% \\ \hline \hline
\multirow{2}{*}{88-90} & \multirow{2}{*}{77976} & $\times$ & 66\% & 78\% & 69\% \\ \cline{3-6} 
 &  & \checkmark & 81\% & 62\% & 68\% \\ \hline \hline
\multirow{2}{*}{50-90} & \multirow{2}{*}{141835} & $\times$ & 60\% & 72\% & 63\% \\ \cline{3-6} 
 &  & \checkmark & 71\% & 61\% & 63\% \\ \hline
\end{tabular}

\clearpage
\setmonofont[Mapping=tex-text]{CMU Typewriter Text}
\bibliographystyle{ugost2008ls}
\bibliography{diploma.bib}
\end{document}
