% Тут используется класс, установленный на сервере Papeeria. На случай, если
% текст понадобится редактировать где-то в другом месте, рядом лежит файл matmex-diploma-custom.cls
% который в момент своего создания был идентичен классу, установленному на сервере.
% Для того, чтобы им воспользоваться, замените matmex-diploma на matmex-diploma-custom
% Если вы работаете исключительно в Papeeria то мы настоятельно рекомендуем пользоваться
% классом matmex-diploma, поскольку он будет автоматически обновляться по мере внесения корректив
%

% По умолчанию используется шрифт 14 размера. Если нужен 12-й шрифт, уберите опцию [14pt]
\documentclass[14pt]{matmex-diploma}
%\documentclass[14pt]{matmex-diploma-custom}

\begin{document}
% Год, город, название университета и факультета предопределены,
% но можно и поменять.
% Если англоязычная титульная страница не нужна, то ее можно просто удалить.
\filltitle{ru}{
    chair              = {Математическое обеспечение и администрирование информационных систем},
    title              = {Создание документации к библиотеке Brahma.FSharp},
    % Здесь указывается тип работы. Возможные значения:
    %   coursework - Курсовая работа
    %   diploma - Диплом специалиста
    %   master - Диплом магистра
    %   bachelor - Диплом бакалавра
    type               = {coursework},
    position           = {студента},
    group              = 243,
    author             = {Сусанина Юлия Алексеевна},
    supervisorPosition = {к.\,ф.-м.\,н., ст. преп.},
    supervisor         = {Григорьев С.\,В.},
%   university         = {Санкт-Петербургский Государственный Университет},
%   faculty            = {Математико-механический факультет},
%   city               = {Санкт-Петербург},
%   year               = {2013}
}
\maketitle
\tableofcontents
% У введения нет номера главы
\section*{Введение}
Хорошо написанная документация к программному обеспечению помогает как программистам и специалистам, занимающимся тестированием, так и пользователям данного продукта, понять его особенности и возможности. Она позволяет быстрее разобраться в основной функциональности ПО и значительно сэкономить время на начальном этапе работы с данным продуктом. 
 
Важной частью документации являются примеры использования данного ПО, именно поэтому важно хранить их в удобном систематизированном виде, так чтобы это не мешало пользователю или разработчику обращаться к исходному коду.
 
Однако часто документация к программному обеспечению не удовлетворяет указанным выше условиям, например, к библиотеке \linebreak Brahma.FSharp ~\cite{brahma}.
 
На данный момент к проекту Brahma.FSharp, библиотеке на F\# для интеграции вычислений на GPGPU, основана на транслировании F\# quotation в OpenCL ~\cite{opencl}, имеется некоторая документация, но существует необходимость в более подробном описании библиотеки и ее функций.
 
В данной работе разрабатывается более подробная и структурированная документация к Brahma.FSharp.

\section{Постановка задач}
Целью данной работы является создание более подробной и структурированной документации к библиотеке Brahma.FSharp.
 
Для достижения цели, поставленной в данной работе, были поставлены следующие задачи: 

\begin{itemize}
\item изучить особенности Brahma.FSharp;
\item изучить ProjectScaffold ~\cite{prscaffold};
\item добавить описание библиотеки Brahma.FSharp на сайты 
      \linebreak http://yaccconstructor.github.io/Brahma.FSharp и
      \linebreak http://fsharp.org/use/gpu;
\item создать новый репозиторий Brahma.FSharp.Examples с примерами из репозитория Brahma.FSharp;
\item написать сайт к созданному проекту Brahma.FSharp.Examples   на основе gh-pages;
\item очистить репозиторий Brahma.FSharp от перенесенных в новый репозиторий примеров.
\end{itemize}

\section {Обзор используемых технологий}
В рамках проекта Brahma.FSharp, к которому необходима документация,  разрабатывается средство для программирования гетерогенных систем вида "Многоядерный CPU + несколько GPGPU". Для интеграции GPGPU используется трансляция F\# quotation в OpenCL. 
 
За основу данной работы была взята уже существующая документация к библиотеке Brahma.FSharp : 
\begin{itemize}
\item https://sites.google.com/site/semathsrprojects/home/brahma-fsharp
\item http://yaccconstructor.github.io/Brahma.FSharp
\end{itemize}
и примеры использования библиотеки Brahma.FSharp, находящиеся в соответствующем репозитории.
 
Для создания репозитория был выбран ProjectScaffold \linebreak (https://fsprojects.github.io/ProjectScaffold). Это средство для разработчиков, позволяющее создать новое .NET/Mono решение со всем необходимым для удобной работы с кодом, инструментами и документацией: 
автоматическое и быстрое создание решения 
легкая работа с пакетами Nuget и другими зависимостями
автоматическая сборка решения и тестовых проектов
автоматически генерируемая документация.

\section {Основная часть}
\subsection{Создание репозитория Brahma.FSharp.Examples}

Brahma.FSharp  — библиотека на F# для интеграции вычислений на GPGPU, основана на транслировании F# quotation в OpenCL.
 
На начальном этапе работы в основном репозитории проекта был не только исходный код, но и примеры использования. Такой подход к хранению кода не всегда удобен, и  поэтому хотелось бы хранить примеры в другом месте. Так сделано уже во многих проектах, например, для YaccConstructor.
 
В ходе данной работы был создан новый репозиторий \linebreak Brahma.FSharp.Examples на github с целью хранения хранения в нем всех примеров и для написания более полной документации по \linebreak использованию библиотеки \linebreak Brahma.FSharp.
 
Для создания репозитория использовался ProjectScaffold. \linebreak ProjectScaffold - это средство для разработчиков, позволяющее создать новое .NET/Mono решение со всем необходимым для удобной работы с кодом, инструментами и документацией.

Основными причинами выбора именно этого инструмента были автоматическое создание и  последующая сборка .NET решения, удобная работа с подключаемыми через Nuget пакетами и генерируемая автоматически  документация, основанная на файлах на Markdown ~\cite{markdown} (облегчённом языке разметки, созданном с целью написания максимально читаемого и удобного для правки текста, но пригодного для преобразования в языки для продвинутых публикаций).

Также были подключены автоматические сборки на сервере, а именно Travis CI для Mono и AppVeyor для .NET. Были перенесены примеры использования из репозитория  Brahma.FSharp в репозиторий 
    \linebreak Brahma.FSharp.Examples.

\subsection{Добавление информации и документации к \\ библиотеке Brahma.FSharp}


К созданному проекту были написаны страницы с документацией, на которых выложены наглядные примеры работы Brahma.FSharp, и заведена новая колонка, в которой будут находиться различные статьи, связанные с использованием данной библиотеки.

Кроме того, было добавлено более развернутое описание о \linebreak Brahma.FSharp на сайты для информирования потенциальных пользователей и разработчиков о данном продукте. За основу было взято описание с сайта Лаборатории языковых инструментов JetBrains ~\cite{desc}, так как там указывались основные особенности работы данной библиотеки.






% У заключения нет номера главы
\section*{Заключение}
В ходе данной работы были получены следующие результаты:
\begin{itemize}
    \item изучены особенности Brahma.FSharp;
    \item изучен ProjectScaffold;
    \item добавлено описание библиотеки Brahma.FSharp на сайты \linebreak http://fsharp.org/use/gpu и 
          \linebreak http://yaccconstructor.github.io/Brahma.FSharp;
    \item создан новый репозиторий Brahma.FSharp.Examples (с примерами из репозитория Brahma.FSharp);
    \item написан сайт к созданному проекту Brahma.FSharp.Examples  (gh-pages);
    \item очищен репозиторий Brahma.FSharp от перенесенных в новый репозиторий примеров.
\end{itemize}


В дальнейшем потребуется дополнение уже существующей документации, связанное с развитием библиотеки, ее возможностей и области ее применения.


\setmonofont[Mapping=tex-text]{CMU Typewriter Text}
\bibliographystyle{ugost2008ls}
\bibliography{diploma.bib}
\end{document}
