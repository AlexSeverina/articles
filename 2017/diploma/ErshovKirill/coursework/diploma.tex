% Тут используется класс, установленный на сервере Papeeria. На случай, если
% текст понадобится редактировать где-то в другом месте, рядом лежит файл matmex-diploma-custom.cls
% который в момент своего создания был идентичен классу, установленному на сервере.
% Для того, чтобы им воспользоваться, замените matmex-diploma на matmex-diploma-custom
% Если вы работаете исключительно в Papeeria то мы настоятельно рекомендуем пользоваться
% классом matmex-diploma, поскольку он будет автоматически обновляться по мере внесения корректив
%

% По умолчанию используется шрифт 14 размера. Если нужен 12-й шрифт, уберите опцию [14pt]
\documentclass[14pt]{matmex-diploma}
%\documentclass[14pt]{matmex-diploma-custom}

\begin{document}
\filltitle{ru}{
    chair              = {Кафедра Системного программирования},
    title              = {Синтаксический анализ графов с помеченными вершинами и ребрами},
    type               = {coursework},
    position           = {студента},
    group              = 344,
    author             = {Ершов Кирилл Максимович},
    supervisorPosition = {ст. преп., к.\,ф.-м.\,н.},
    supervisor         = {Григорьев С.\,В.},
}
\maketitle
\tableofcontents
\section*{Введение}
    Помеченные графы являются удобным способом представления различных структурированных данных. Такие графы используются, например, в биоинформатике, логистике, графовых базах данных.
    
    Иногда для представления данных с использованием графов обходятся только метками на рёбрах. Но в некоторых случаях метки на вершинах позволяют более наглядно отображать зависимости между сущностями. К примеру, в биоинформатике существует большое количество данных, содержащих взаимосвязь между генами и белками. Такие данные удобно представлять в виде графа, вершины которого помечены определенными генами и белками, а ребра показывают их отношение (например, ген кодирует белок).
     
    Для эффективной работы с помеченными графами необходимо иметь возможность делать запросы, возвращающие нужную информацию из графа. Запросы можно представлять в виде грамматики. Тогда язык грамматики задает класс путей, удовлетворяющих запросу. Пути рассматриваются как строки, состоящие из меток на рёбрах и вершинах. Путь удовлетворяет запросу, если строка принадлежит соответствующему языку.  Для реализации запросов к помеченным графам широко используются регулярные грамматики. Однако с их помощью бывает невозможно описать нужные запросы. Поэтому актуальна задача организации более выразительных запросов, используя КС-грамматики.

    Для синтаксического анализа строки по произвольной КС-грамматике существуют различные алгоритмы. Например, Early parser \cite{Early} ,  CYK \cite{CYK},  GLR \cite{glr}, GLL \cite{gll}. Алгоритм GLL имеет оптимальное время работы ($O(n^3)$ в худшем случае) и основан на идее нисходящего анализа, а значит более удобен для реализации. Поэтому для поиска пути используется именно этот алгоритм.

    Таким образом, использование графов с метками на вершинах и рёбрах позволяет естественным образом представлять различные наборы данных, а обработка запросов необходима для эффективной работы с ними. КС-грамматики дают возможность писать выразительные запросы, при этом использование алгоритма GLL позволит быстро выполнять такие запросы.
\section{Постановка задачи}
Целью курсовой работы является реализация синтаксического анализа графов с помеченными вершинами и рёбрами. Для достижения этой цели поставлены перечисленные ниже задачи.
\begin{itemize}
    \item В рамках проекта YaccConstructor \cite{YaccConstructorPage} реализовать возможность поиска путей в графе с помеченными вершинами и рёбрами по заданной КС-грамматике.
    \item Реализовать удобный интерфейс для работы:
    \begin{itemize}
    \item создание и выполнение запросов 
    \item получение и обработка результатов
    \end{itemize}
    \item Провести апробацию и сравнить с существующими решениями.
\end{itemize}

\section{Обзор}
Для поиска путей в графе существует множество инструментов, позволяющих находить пути по регулярным грамматикам. Решений для поиска путей по КС-грамматике не так много, в особенности для графов с метками на вершинах и рёбрах.
 
В работе \cite{subgraph} решалась задача извлечения связного подграфа, состоящего из путей между двумя исходными вершинами, из графа с метками на вершинах и рёбрах. Класс подходящих путей описывается с помощью контекстно-свободной грамматики. Для синтаксического анализа используется алгоритм Earley, работающий в худшем случае за время $O(n^3)$. Однако, поиск путей производится не в исходном графе с метками на вершинах и рёбрах, а в преобразованном. Перед началом работы алгоритма из исходного получают новый двудольный граф с метками только на рёбрах. Новый граф имеет в 2 раза больше вершин и увеличивает число рёбер. Даже при небольших входных данных и для путей длины не больше 8 алгоритм работает 240 секунд, что делает его мало применимым на практике.

Одним из распространённых способов представлять данные в удобном для обработки виде является модель RDF. Данные, записанные в RDF, представляют собой набор триплетов субъект--предикат--объект. В совокупности они образуют помеченный ориентированный граф. Многие данные в биоинформатике представлены именно в таком формате.

Самым популярным языком для запросов к данным, представленным в формате RDF, является язык SPARQL \cite{prud2008sparql}. Однако, он позволяет описывать только регулярные выражения. В статье \cite{zhang2016context} авторы описали алгоритм для поиска путей в RDF-графе, принадлежащих КС-языку, а также предложили язык  csSPARQL, поддерживающий КС-грамматики. Показано, что сложность алгоритма $O((|N|*|G|)^3)$, где N --- нетерминалы входной грамматики, G --- RDF-граф.
\section{Реализация прототипа синтаксического анализа графа}
На кафедре Системного программирования в лаборатории языковых инструментов разрабатывается проект YaccConstructor. Это платформа для исследований в области синтаксического анализа, написанная на языке F\#. YaccConstructor позволяет создавать синтаксические анализаторы и имеет модульную архитектуру. Для построения анализатора  выбирается фронтенд для обработки грамматик, выполняются необходимые преобразования и по указанному генератору строится нужный результат.

В YaccConstructor есть абстрактная реализация алгоритма синтаксического анализа GLL. Исходная грамматика описывается на языке спецификации грамматик YARD. Затем генератором она преобразуется в файл на языке F\#, содержащий необходимую для алгоритма информацию о грамматике.

Во время выполнения алгоритм перемещается по входному объекту в зависимости от текущей позиции в грамматике. Объект, в котором требуется найти пути, удовлетворяющие исходной КС-грамматике, должен реализовывать интерфейс IParserInput. Мною реализован этот интерфейс для графов  с помеченными вершинами и рёбрами. Если текущая позиция --- вершина, следующими позициями в графе являются все исходящие рёбра. Если текущая позиция на ребре, следующим является конечная вершина. Таким образом, алгоритмом проверяются все возможные пути в графе. Для тех путей, которые удовлетворяют запросу, прототип пока что возвращает только начальную и конечную позиции в графе.

Прототип опробован на графах, содержащих от 630 до 640 рёбер. Время работы в среднем составило 34 мс. Однако, для тестов использовались графы с небольшим разнообразием меток и несложной грамматикой. В дальнейшем будет произведена апробация на реальных данных.

\section{Заключение}
Результаты, достигнутые на данный момент:
\begin{itemize}
    \item написан обзор предметной области
    \item реализован прототип, выполняющий поиск путей в графе с помеченными вершинами и рёбрами по заданной КС-грамматике
\end{itemize}
В дальнейшем планируется реализовать интерфейс для работы с запросами и обработки результатов, а также протестировать алгоритм на реальных данных и сравнить с существующими решениями.
\setmonofont[Mapping=tex-text]{CMU Typewriter Text}
\bibliographystyle{ugost2008ls}
\bibliography{diploma.bib}
\end{document}
