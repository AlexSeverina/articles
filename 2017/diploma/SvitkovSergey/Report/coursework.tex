\documentclass[14pt]{matmex-diploma-custom}
\usepackage{enumitem}
\usepackage{amsmath}

\begin{document}
\filltitle{ru}{
    chair              = {Кафедра Системного программирования},
    title              = {Реализация поиска путей с КС-ограничениями в рамках библиотеки YC.QuickGraph},
    type               = {coursework},
    position           = {студента},
    group              = 344,
    author             = {Свитков Сергей Андреевич},
    supervisorPosition = {ст. преп, к. ф-м. н.},
    supervisor         = {Григорьев С.\,В.},
}
\maketitle
\tableofcontents
\section*{Аннотация}
	Большинство промышленных языков для написания запросов к графовым базам данных являются регулярными. Но регулярные языки не применимы в ряде задач, поэтому актуальным является создание контекстно-свободного (далее --- КС) языка запросов. Существуют работы по этой теме, но они в основном теоретические. В данной работе рассматривается практическая реализация механизма контекстно-свободных запросов к ориентированным графам с помеченными ребрами для платформы .NET. Результатом работы является библиотека, предоставляющая набор функций для написания КС-запросов к графам. Полученные результаты могут быть применены в проектах, использующих C\# или F\#.

\section*{Введение}
	Модель представления данных в виде ориентированных графов с метками на ребрах, имеет широкую область применения и используется в биоинформатике, социальных исследованиях (например, при представлении социальных графов), semantic web, при реализации графовых баз данных. 
	
	При наличии представления данных в виде определенной структуры становится актуальным вопрос их обработки, а именно --- получения из всего набора только тех данных, которые представляют какой-либо конкретный интерес. Для этого используются языки запросов. Существует множество промышленных языков запросов к графам, например Gremlin\cite{Gremlin}, Cypher\cite{Cypher}, и т.д.. Но данные языки являются регулярными, а значит, не могут применяться в некоторых задачах. Например, при разборе генеалогического дерева, встречаются строки вида \(parent^nchild^n\). Такие строки нельзя распознать с помощью регулярной грамматики, но можно с помощью КС-грамматики с правилами вывода \(N \to parent\,child, \,N \to parentN\, child\). 
	
	Существуют работы, предлагающие различные подходы к реализации КС-запросов к графам, например \cite{sevon2008subgraph}, \cite{hellings2014conjunctive}. Но большая часть работ по данной теме представляет только теоретические сведения о возможных подходах к реализации, а те, что реализованы на практике, имеют довольно ограниченный функционал или же слишком узкую специализацию. Так, в работе \cite{hellings2014conjunctive} результатом запроса является КС-отношение --- тройка вида \((n, m, N)\), где \(n\) и \(m\) --- вершины, связанные путем, выводимым из нетерминала \(N\). 
	
	Поскольку класс задач, в которых могут быть применены КС-запросы к графам, является обширным, то возникает задача создания инструмента, с помощью которого можно было бы получать результат КС-запроса в нескольких формах. Для этого требуется язык запросов и средства синтаксического анализа. Одним из алгоритмов синтаксического анализа является GLL\cite{gll}. Как описано в работе \cite{ragRelaxedParsing}, данный алгоритм имеет широкое применение в задачах синтаксического анализа и хорошую асимптотику (\((O(n^3)\) --- на нелинейном входе, \(O(n)\)  --- на линейном). Кроме того, результатом вышеупомянутой статьи является реализация алгоритма в рамках проекта YaccConstructor \cite{YaccConstructorPage} (далее --- YC). Кроме того, YC имеет язык спецификаций грамматик YARD \cite{YARD}, который можно использовать в качестве языка запросов. Поэтому было принято решение использовать YC как инструмент для решения поставленной задачи. Средства для работы с графами собраны в библиотеке YC.QuickGraph \cite{YC.QuickGraph}.
	
	Поэтому механизм поиска путей с КС-ограничениями был реализован как расширение библиотеки YC.QuickGraph с использованием языка YARD для задания грамматик. Полученный результат --- библиотека, которая позволяет осуществлять КС-запросы к графам с помеченными ребрами, представляя результат в виде подграфа, пути или КС-отношения.
	
\section{Обзор предметной области}
1
\section{Постановка задачи}
1
\section{Основная часть}
1
\section{Эксперименты}
1
\section*{Заключение}
1
\setmonofont[Mapping=tex-text]{CMU Typewriter Text}
\bibliographystyle{ugost2008ls}
\bibliography{coursework}

\end{document}