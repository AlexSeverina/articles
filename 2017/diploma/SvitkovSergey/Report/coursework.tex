\documentclass[14pt]{matmex-diploma-custom}
\usepackage{enumitem}
\usepackage{amsmath}

\begin{document}
\filltitle{ru}{
    chair              = {Кафедра Системного программирования},
    title              = {Реализация поиска путей с КС-ограничениями в рамках библиотеки YC.QuickGraph},
    type               = {coursework},
    position           = {студента},
    group              = 344,
    author             = {Свитков Сергей Андреевич},
    supervisorPosition = {ст. преп, к. ф-м. н.},
    supervisor         = {Григорьев С.\,В.},
}
\maketitle
\tableofcontents
\section*{Аннотация}
	Большинство промышленных языков для написания запросов к графовым базам данных являются регулярными, например, Cypher, используемый в Neo4J. Но регулярные языки не применимы в ряде задач, поэтому актуальным является создание контекстно-свободного (в дальнейшем --- КС) языка запросов.		

	Предлагаемое в статье решение было реализовано как расширение бибилотеки YC.QuickGraph. Полученные результаты могут быть применены в проектах, реализованных с использованием языка C\# или F\#. В данной работе рассматривается реализация контекстно-свободных запросов к ориентированным графам с помеченными ребрами.

\section*{Введение}
	Модель представления данных в виде графов, в частности, в виде орграфов с метками на ребрах, имеет широкую область применения и используется в биоинформатике, социальных исследованиях (например, при представлении социальных графов), semantic web, при реализации графовых баз данных. 
	
	При наличии представления данных в виде определенной структуры становится актуальным вопрос их обработки, а именно --- получения из всего набора только тех данных, которые представляют какой-либо конкретный интерес. Для этого используются языки запросов. Из всего множества таких языков стоит выделить Cypher \cite{Cypher}, применяемый при работе с Neo4J (другие языки запросов к графовым базам данных во многом схожи с ним, поэтому их рассмотрение можно опустить). Но данный язык является регулярным, а значит, не может применяться в некоторых задачах. Например, при решении задачи о поиске пар \(n\) поколения потомков от общего предка, нужно найти строки вида \(parent^nchild^n\). Такую строку нельзя задать с помощью регулярной грамматики, но можно с помощью КС-грамматики с правилами вывода \(N \to parent\,child, \,N \to parentN\, child\). 
	
	Существуют работы, предлагающие различные подходы к реализации КС-запросов. Но данные решения в большинстве своем теоретические, а те, что реализованы на практике, предоставляют довольно бедный функционал, или же имеют слишком узкую специализацию. Поэтому было принято решение реализовать библиотеку, позволяющую использовать КС-грамматики в качестве языка запросов и поддерживающую возможность представления результата запроса в форме подграфа, множества путей или кратчейшего пути.
	
	Реализация выполнялась как расширение библиотеки YC.QuickGraph \cite{YC.QuickGraph} с использованием языка спецификаций грамматик YARD \cite{YARD} для запросов. Полученное расширение предоставляет набор функций для исполнения запросов и представления результата запроса в удобной для пользователя форме. 
	
\section{Обзор предметной области}
1
\section{Постановка задачи}
1
\section{Основная часть}
1
\section{Эксперименты}
1
\section*{Заключение}
1
\setmonofont[Mapping=tex-text]{CMU Typewriter Text}
\bibliographystyle{ugost2008ls}
\bibliography{coursework}

\end{document}