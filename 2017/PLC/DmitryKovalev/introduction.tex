\section*{Введение}
Современные языки программирования общего назначения поддерживают возможность работы со строковыми литералами, позволяя формировать из них выражения при помощи строковых операций. 
Строковые выражения могут создаваться динамически, с использованием таких конструкций языка, как циклы и условные операторы. 
Данный подход широко используется, например, при формировании SQL-запросов к базам данных из программ, написанных на Java, C$\#$ и других высокоуровневых языках (листинг \ref{lst:example}).

\begin{figure}[h]	
	\vspace{-10pt}
	\lstset{language=[Sharp]C,
		showstringspaces=false,
		basicstyle=\small,
		keywordstyle=\bfseries,,	
	}
	\begin{lstlisting}[caption={Динамически формируемый SQL-запрос}, label={lst:example}, captionpos=b]
private void Example (bool cond) {
    string columnName = cond ? "name" : "address";
    string queryString = 
        "SELECT id, " + columnName + " FROM users";
    Program.ExecuteImmediate(queryString);
}
	\end{lstlisting}
	\vspace{-10pt}
\end{figure}

Недостаток такого метода генерации кода заключается в том, что формируемые выражения, с точки зрения компилятора, являются обычными строками и не проходят статические проверки на корректность и безопасность, что приводит к ошибкам времени исполнения и усложняет разработку и сопровождение системы. 
Включение обработки динамически формируемых строковых выражений в фазу статического анализа осложняется тем, что такие выражения, в общем случае, невозможно представить в виде линейного потока, который принимают на вход традиционные алгоритмы лекcического/синтаксического анализа. 

Для решения данной проблемы были разработаны различные методы статического анализа множества значений формируемого выражения. 
Как правило, язык, на котором написана исходная программа, тьюринг-полон, что делает невозможным проведение точного анализа. Поэтому распространенным подходом является построение некоторой аппроксимации рассматриваемого множества. 
Ряд предложенных ранее решений использует для анализа \textit{регулярную аппроксимацию} --- множество строк, генерируемых программой, аппроксимируется сверху регулярным языком, и анализатор работает с его компактным представлением, таким как регулярное выражение или конечный автомат.

В магистерской диссертации \cite{gll_reg} был описан алгоритм, позволяющий проводить синтаксический анализ регулярной аппроксимации (конечного автомата, представленного в виде графа) множества значений динамически формируемого выражения. Основой для данного алгоритма служит алгоритм обобщенного синтаксического анализа Generalized LL (GLL, \cite{gll}).
Такой подход позволяет получать конечное представление леса разбора \cite{sppf} корректных строк, содержащихся в аппроксимации множества значений выражения. Это представление может быть использовано для проведения более сложных видов статического анализа и для целей реинжиниринга.

В данной статье будет представлен алгоритм синтаксического анализа, который основан на описанной выше модификации GLL, но работает с более точной, чем регулярная, контекстно-свободной аппроксимацией множества значений динамически формируемого выражения. 
Использование более точной аппроксимации позволяет снизить количество ложных синтаксических ошибок, возникающих в результате того, что аппроксимирующее множество может содержать строки, осутствующие среди значений искомого выражения.