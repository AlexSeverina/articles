\begin{abstract}
Многие программы в процессе работы формируют из строк исходный код на некотором языке программирования и передают его для исполнения в соответствующее окружение (пример --- dynamic SQL). 
Для статической проверки корректности динамически формируемого выражения используются различные методы, одним из которых является синтаксический анализ регулярной аппроксимации множества значений такого выражения. 
Аппроксимация может содержать строки, не принадлежащие исходному множеству значений, в том числе синтаксически некорректные. 
Анализатор в данном случае сообщит об ошибках, которые на самом деле отсутствуют в выражении, генерируемом программой. 
В данной статье будет описан алгоритм синтаксического анализа более точной, чем регулярная, контекстно-свободной аппроксимации динамически формируемого выражения.
\\
\\
\textbf{Ключевые слова:} синтаксический анализ, динамически формируемый код, контекстно-свободные грамматики, GLL, GFG, dynamic SQL, DSQL 
\end{abstract}
