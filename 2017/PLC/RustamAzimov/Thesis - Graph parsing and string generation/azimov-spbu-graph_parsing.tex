\documentclass[10pt]{article}

\newtheorem{mydef}{Определение}
\usepackage{mathtools}
\usepackage{cmap} % for serchable pdf's
\usepackage[utf8]{inputenc}
\usepackage[russian]{babel}
\usepackage{fontspec}
\usepackage{enumerate}
\setmainfont[Mapping=tex-text]{CMU Serif}



\begin{document}
\title{Синтаксический анализ графов и задача генерации строк с ограничениями}
\author{Рустам Азимов, Семён Григорьев \\ 
Лаборатория языковых инструментов JetBrains, \\
Санкт-Петербургский государственный университет, \\
Россия, 199034, Санкт-Петербург, Университетская наб. 7/9/ \\ 
\mail{rustam.azimov19021995@gmail.com}, \mail{Semen.Grigorev@jetbrains.com}
}

\maketitle

\begin{abstract}
Одной из задач, изучаемых в теории формальных языков, является задача генерации строк, удовлетворяющих заданной системе правил. С другой стороны, существует задача синтаксического анализа графов, то есть задача поиска путей в графе, метки на ребрах которых образуют строку, принадлежащую заданному формальному языку. В данной работе будет показана связь между этими двумя задачами.

\vspace{1em}
\textbf{Ключевые слова:} синтаксический анализ графов, генерация строк, формальные языки, конъюнктивные грамматики.

\end{abstract}

В таких областях, как графовые базы данных~\cite{azimov-spbu-graphDB, azimov-spbu-zhang}, биоинформатика~\cite{azimov-spbu-Anderson}, возникают задачи поиска путей в графах, удовлетворяющих определенным ограничениям. В качестве таких ограничений естественно выбрать формальный язык $L$~\cite{azimov-spbu-barrett} и искать пути в графе, соответствующие строкам из языка $L$. Задачи поиска путей в графе, которые используют такие ограничения с формальными языками, называются задачами \textit{синтаксического анализа графов}. Данная задача также возникает при статическом анализе динамически формируемого кода, например динамических SQL-запросов или генераторов Web-страниц. В данном случае графом является представление регулярной аппроксимации множества возможных значений динамически формируемых строк.

Кроме того, существует задача генерации строк, суть которой в построении строк, принадлежащих некоторому формальному языку. В работе~\cite{azimov-spbu-Okhotin} приведены формулировки задачи генерации строк с дополнительными ограничениями.

Некоторые вариации задач синтаксического анализа графов могут быть сведены к задаче генерации строк. Так, например, в большинстве задач синтаксического анализа графов недостаточно просто определить существование пути, соответствующего строке некоторого формального языка $L$, но также требуется предъявить такой путь. Так как все пути в графе соответствуют строкам из некоторого регулярного языка $R$, то в данной задаче требуется найти путь, соответствующий строке из языка $L \cap R$. Эта задача может быть решена с помощью генератора строк рассматриваемого пересечения языков. В рамках данной работы была поставлена задача исследования связей между задачей генерации строк~\cite{azimov-spbu-Okhotin} и некоторыми типами задач синтаксического анализа графов~\cite{azimov-spbu-hellings1, azimov-spbu-hellings2}, использующие контекстно-свободные и конъюнктивные~\cite{azimov-spbu-conj} языки.

Язык, который порождается графом $G$ и выделенными в нем вершинами $m, n$, обозначим $L(G, m, n)$. А язык, порождаемый грамматикой $C$, со стартовым нетерминалом $a$ обозначим $L(C,a)$.

В контексте задач синтаксического анализа графов бывает необходимо отвечать на различного рода вопросы, связанные с искомыми в графе путями. Тип вопросов, на которые отвечает задача принято называть \textit{семантикой запроса}.

Использование \textit{relational} семантики запроса означает, что для нетерминала $a$ и графа $G$ необходимо построить множество $\{(m, n)~|~L(C,a) \cap L(G,m,n) \neq \emptyset \}$. В случае использования КС-языка было выявлено отсутствие необходимости в применении генератора строк для поиска ответа на запрос с \textit{relational} семантикой, так как в работе~\cite{azimov-spbu-hellings2} используется аннотированная грамматика, которая порождает язык $L(C,a) \cap L(G,m,n)$ и ее построение автоматически решает поставленную задачу.

Использование \textit{all-path} семантики запроса означает, что для нетерминала $a$, графа $G$ и его вершин $m,n$, необходимо предъявить все пути из вершины $m$ в вершину $n$, такие что метки на ребрах этих путей образуют строку из языка $L(C,a)$. В случае использования КС-языка также было выявлено отсутствие необходимости в применении генератора строк для данной семантики, так как в работе~\cite{azimov-spbu-hellings2} аннотированную грамматику и предлагают в качестве ответа на запрос. Но также была выявлена возможность использования генератора строк для получения конкретных строк пользователем из полученной аннотированной грамматики.

Использование \textit{single-path} семантики запроса означает, что для нетерминала $a$, графа $G$ и его вершин $m,n$, необходимо предъявить какой-нибудь путь (если он существует) из вершины $m$ в вершину $n$, такой что метки на ребрах этого пути образуют строку из языка $L(C,a)$. Для КС-языков в работе~\cite{azimov-spbu-hellings2} строится аннотированная грамматика, и если она порождает непустой язык, то в ней ищется строка минимальной длины, которая и будет соответствовать искомому пути в графе $G$. Таким образом, было выявлено, что алгоритм решения задачи синтаксического анализа графов с использованием \textit{single-path} семантики запроса, предложенный в работе~\cite{azimov-spbu-hellings2}, и является примером использования генерации строки из КС-языка $L(C,a) \cap L(G,m,n)$.

Также была рассмотрена задача синтаксического анализа графов с использованием конъюнктивной грамматики. Из неразрешимости задачи определения пустоты конъюнктивных языков была получена неразрешимость задачи синтаксического анализа графов с использованием конъюнктивных языков и \textit{relational} семантики запроса, о чем также упоминается в работе~\cite{azimov-spbu-hellings1}. Кроме того, было выявлено, что при использовании конъюнктивных грамматик нельзя гарантировать нахождения хотя бы одной строки из конъюнктивного языка $L(C,a) \cap L(G,m,n)$. Предположим, что найдется хотя бы одна строка, удовлетворяющая рассматриваемым ограничениям. Тогда при использовании \textit{all-path} семантики запроса, применяя алгоритм генерации строки, происходил бы просто перебор всех возможных строк и проверка на принадлежность этих строк к языку $L(C,a) \cap L(G,m,n)$, что не соответствует практическому смыслу задачи. А для задачи синтаксического анализа графов с использованием \textit{single-path} семантики запроса есть возможность сгенерировать некоторую строку непустого языка $L(C,a) \cap L(G,m,n)$. Стоит отметить, что использование конъюнктивных языков в задачах синтаксического анализа графов мало изучено. Полученные результаты могут быть использованы в дальнейших исследованиях данной области. Одной из тем таких исследований, например, является применимость булевых ~\cite{azimov-spbu-bool} грамматик в синтаксическом анализе графов.

\setmonofont[Mapping=tex-text]{CMU Typewriter Text}
\bibliographystyle{ugost2008ls}
\bibliography{azimov-spbu-biblio.bib}
\end{document}
