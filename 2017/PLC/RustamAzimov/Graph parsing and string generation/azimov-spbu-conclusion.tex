\section{Заключение}

В данной работе была показана связь между задачей генерации строк, рассмотренной в работе~\cite{azimov-spbu-Okhotin}, и некоторыми вариациями задачи синтаксического анализа графов, предложенных в работах~\cite{azimov-spbu-hellings1, azimov-spbu-hellings2} и использующих контекстно-свободные и конъюнктивные грамматики.

Стоит отметить, что использование конъюнктивных языков в задачах синтаксического анализа графов мало изучено. Полученные результаты могут быть использованы в дальнейших исследованиях данной области. Одной из тем таких исследований, например, является применимость булевых ~\cite{azimov-spbu-bool, azimov-spbu-conjandbool} грамматик в синтаксическом анализе графов.