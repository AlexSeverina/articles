\section{Связь задачи генерации строк с синтаксическим анализом графов для конъюнктивных грамматик}

Рассмотрим задачу синтаксического анализа графа $G = (Q, \Sigma, \delta)$ с использованием конъюнктивной грамматики $C$. Аналогично случаю КС-грамматик будем обозначать $L(C,a)$ --- язык, порождаемый конъюнктивной грамматикой $C$ со стартовым нетерминалом $a$. Гарантировать завершаемость алгоритма генерации строк из бесконечного языка можно либо умея решать задачу определения пустоты этого языка, либо заранее зная его непустоту. Первая проблема, с которой мы сталкиваемся, рассматривая конъюнктивные грамматики вместо КС-грамматик --- это неразрешимость задачи определения пустоты языка. Поэтому далее будем предполагать, что пересечение рассматриваемых языков $L(C,a) \cap L(G,m,n) \ne \emptyset$ и если это предположение неверно, то мы не можем гарантировать завершаемость алгоритмов, использующих генератор строк данного языка. Отсюда следует неразрешимость задачи синтаксического анализа с использованием конъюнктивного языка и \textit{relational} семантики запроса, о чем также упоминается в работе~\cite{azimov-spbu-hellings1}.

Мы легко можем задать конъюнктивную грамматику, порождающую пересечение языков $L(C,a) \cap L(G,m,n)$, так как в данном формализме присутствует явная операция пересечения. Так как при всех видах ограничений, рассматриваемых в работе~\cite{azimov-spbu-Okhotin}, применяется ограничение сверху длины генерируемой строки, то, не имея оценки сверху на $minlen$ --- минимальную длину строки непустого языка $L(C,a) \cap L(G,m,n)$, нельзя гарантировать нахождения хотя бы одной строки. То есть для любого ограничения на генерируемую строку, накладывающего, в частности, ограничение сверху на длину строки $(|w| \le m)$, может оказаться, что $minlen > m$ и генератор строк не сможет сгенерировать ни одной строки. Но никакой оценки сверху на $minlen$ найти не удастся, так как в противном случае за конечное число проверок принадлежности строки к конъюнктивному языку (эта задача разрешима и P-полна) можно было бы решить задачу проверки пустоты этого языка (эта задача является неразрешимой). Таким образом, для любых ограничений заранее неизвестно, есть ли хотя бы одна строка, удовлетворяющая этим ограничениям. Но если известно для какого то ограничения, что такая строка существует, то задача генерации строк для конъюнктивных грамматик и этих ограничений, согласно работе~\cite{azimov-spbu-Okhotin}, NP-полна.

Предположим, что найдется хотя бы одна строка, удовлетворяющая рассматриваемым ограничениям. Тогда при использовании \textit{all-path} семантики запроса, применяя алгоритм генерации строки, происходил бы просто перебор всех возможных строк и проверка на принадлежность этих строк к языку $L(C,a) \cap L(G,m,n)$, что не соответствует практическому смыслу задачи. Для задачи синтаксического анализа графов с использованием \textit{single-path} семантики запроса есть возможность сгенерировать некоторую строку непустого языка $L(C,a) \cap L(G,m,n)$ (в работе~\cite{azimov-spbu-hellings2} подчеркнута логичность выбора строки минимальной длины, но это необязательно).
