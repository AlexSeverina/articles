\section*{Введение}

В таких областях, как графовые базы данных~\cite{azimov-spbu-graphDB}, биоинформатика~\cite{azimov-spbu-Anderson} и др., возникают задачи поиска путей в графах, удовлетворяющих определенным условиям. Например, на искомые пути могут быть наложены ограничения на длину, или производится поиск лишь простых путей. Но при работе со сложными системами зачастую таких ограничений бывает недостаточно. Поэтому широко распространено использование ограничений на метки ребер/вершин путей помеченного графа. В качестве таких ограничений естественно выбрать формальный язык $L$. Тогда при заданном алфавите $\Sigma$ и ориентированном графе $G$, ребра которого помечены символами из $\Sigma$, для искомых путей $p$ графа $G$ выполняется $l(p) \in L$, где $l(p)$ означает слово из $\Sigma^*$, полученное последовательной конкатенацией меток пути $p$. Задачи поиска путей в графе, которые используют такие ограничения с формальными языками, являются задачами синтаксического анализа графов.

Кроме того, существует задача генерации строк, суть которой в построении строк, принадлежащих некоторому формальному языку. В работе~\cite{azimov-spbu-Okhotin} получены оценки сложности задачи генерации строк с дополнительными ограничениями для различных классов формальных языков.

Некоторые вариации задач синтаксического анализа графов могут быть сведены к задаче генерации строк. Так, например, в большинстве задач синтаксического анализа графов требуется найти путь в графе, соответствующий строке некоторого формального языка $L$. Так как все пути в графе соответствуют строкам из некоторого регулярного языка $R$, то в данной задаче требуется найти путь, соответствующий строке из языка $L \cap R$. Эта задача может быть решена с помощью генератора строк рассматриваемого пересечения языков.

Таким образом, в данной работе исследуется связь задачи генерации строк и некоторых типов задач синтаксического анализа графов. В качестве формальных языков будут рассматриваться широко используемые в области синтаксического анализа графов контекстно-свободные языки. Также будут рассмотрены конъюнктивные~\cite{azimov-spbu-conjunctiveGrammar} языки, обладающие большей выразительной мощностью.
