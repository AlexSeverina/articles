\documentclass [a4paper] {article}

% ----------------------------------------------------------------
% Required packages

\usepackage [T2A] {fontenc}
\usepackage [utf8] {inputenc}
\usepackage [english, russian] {babel}

\usepackage {url}
\usepackage [style = gost-numeric] {biblatex}

% ----------------------------------------------------------------
% Optional packages

\usepackage {lipsum}

%\documentclass[10pt]{article}

%\newtheorem{mydef}{Определение}
%\usepackage{mathtools}
%\usepackage{cmap} % for serchable pdf's
%\usepackage[utf8]{inputenc}
%\usepackage[russian]{babel}
%\usepackage{fontspec}
%\usepackage{enumerate}
%\setmainfont[Mapping=tex-text]{CMU Serif}

\addbibresource {paper.bib}

\begin{document}
\title{Ostap: синтаксическое расширение OCaml для создания парсер-комбинаторов с поддержкой левой рекурсии}
\author{
  Вербицкая~Е.\,А., \\ \url {ekaterina.verbitskaya@jetbrains.com }, \\
  Санкт-Петербургский государственный университет, \\
  Лаборатория языковых инструментов JetBrains 
}

\maketitle

\begin{abstract}
Парсер-комбинаторы --- привлекательная техника реализации синтаксических анализаторов. Одним из недостатков подхода является невозможность использования левой рекурсии при объявлении парсер-комбинаторов. В докладе будет представлено синтаксическое расширение языка OCaml для создания парсер-комбинаторов, позволяющее реализовывать в том числе леворекурсивные парсер-комбинаторы.

\vspace{1em}
\textbf{Ключевые слова: парсер-комбинаторы, OCaml, левая рекурсия}  
\end{abstract}

Одним из способов реализации синтаксического анализа (проверки принадлежности предложения заданному языку и построения для него некоторого структурного представления) является техника парсер-комбинаторов~\cite{hutton1992higher}. Основная идея подхода заключается в моделировании синтаксических анализаторов функциями высшего порядка и комбинирования их при помощи небольшого множества комбинаторов. Существует множество разничных стилей реализации подхода, среди которых особо выделяются монадические парсер-комбинаторы~\cite{nott237}, позволяющие осуществлять синтаксический анализ, зависящий от контекста, что может быть полезно при разборе, например, двумерного синтаксиса. Помимо способности разбирать более широкий класс языков, чем класс контекстно-свободных, парсер-комбинаторы также предоставляют возможность вычислять семантические значения, не строя абстрактное синтаксическое дерево. Другим существенным преимуществом является возможность создавать парсеры высшего порядка, то есть синтаксические анализаторы, которые принимают аргументом другие парсеры. Данная особенность повышает композиционность и сокращает размер реализации синтаксического анализатора. 

Библиотека Ostap предоставляет набор парсер-комбинаторов и синтаксическое расширение языка OCaml для упрощения реализации синтаксических анализаторов. Библиотека реализует монадические парсер-комбинаторы с неограниченным предпросмотром, при этом коминатор выбора игнорирует вторую альтернативу, если анализ с помощью первой завершился успешно. Таким образом, результатом анализа всегда является единственный возможный вывод данного предложения (если он существует). Библиотека Ostap позволяет систематически создавать пользовательские парсеры высшего порядка из небольшого набора стандартных парсер-комбинаторов. Например, можно описать парсер выражений $expr \, ops \, opnd$, параметризуемый двумя парсерами: $ops$, разбирающим бинарные операции, и $opnd$, специфицирующим операнды. Далее парсер-комбинатор $expr$ можно использовать для спецификации пользовательского парсера арифметических выражений над натуральными числами, над числами с плавающей точкой или, например, лямбда-выражений: для этого достаточно применить $expr$ к соответствующим аргументам. 

Парсер-комбинаторы Ostap полиморфны относительно типа входного потока. С одной стороны, данная особенность позволяет унифицированно анализировать потоки символов из разных источников. С другой стороны, она значительно повышает модульность пользовательских парсер-комбинаторов: если входной поток конкретен, то комбинирование парсеров высшего порядка без их модификации затруднено. Помимо этого, становится возможным реализовывать дополнительную функциональность в объекте, моделирующем вход: например, производить лексический анализ (выделние лексем). В библиотеке есть стандартная реализация потока Matcher, инициализируемая строкой, производящая базовый лексический анализ и содержащая информацию о координатах анализируемой лексемы. 

Одним из недостатков нисходящего синтаксического анализа, реализацией идей которого являются парсер-комбинаторы, является невозможность обработки леворекурсивных определений парсеров. Леворекурсивные правила являются наиболее естественным способом описать левоассоциативные операции, поэтому их поддержка может значительно упростить создание синтаксических анализаторов. Существует несколько решений описанной проблемы~\cite{frost2008parser, warth2008packrat}, ни одно из которых не в состоянии обрабатывать леворекурсивные парсеры высшего порядка. Библиотека парсер-комбинаторов Meerkat позволяет использовать левую рекурсию при создании анализаторов, однако достигается это путем явного построения леса разбора для данной строки~\cite{Izmaylova}.

Для поддержки левой рекурсии в библиотеке Ostap мы использовали идеи, применённые для её добавления в формализм Parser Expression Grammar~\cite{Medeiros2012}. Данный подход оперирует понятием ограниченной рекурсии: такое использование нетерминала в выводе, что количество леворекурсивных вызовов в нем ограничено некоторой константой. Если строка имеет вывод в данной грамматике, то количество вызовов каждого леворекурсивного нетерминала при разборе конечно, поэтому для обработки леворекурсивного нетерминала достаточно найти оптимальное ограничение левой рекурсии, что и осуществляется динамически во время процесса вывода. Во время поиска промежуточные данные сохраняются в таблицу мемоизации, которая должна быть глобальна для всех используемых парсер-комбинаторов, поэтому эта функциональность реализована в абстракции входного потока, являющейся наследником класса Matcher. Для использования левой рекурсии в случае парсеров высшего порядка в библиотеке предусмотрен парсер-комбинатор $fix$. К сожалению, производительность решения на данный момент в случае использования взаимной рекурсии не является удовлетворительной и ее улучшение является задачей на будущее. 

\printbibliography

\end{document} 
