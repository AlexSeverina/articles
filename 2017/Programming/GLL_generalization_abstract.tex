\documentclass[12pt]{article}  % standard LaTeX, 12 point type
\usepackage{amsfonts,latexsym}
\usepackage{amsthm}
\usepackage{amssymb}
\usepackage[utf8x]{inputenc} % Кодировка
\usepackage[english]{babel} % Многоязычность

\newtheorem{theorem}{Theorem}[section]
\newtheorem{proposition}[theorem]{Proposition}
\newtheorem{lemma}[theorem]{Lemma}
\newtheorem{corollary}[theorem]{Corollary}
\newtheorem{conjecture}[theorem]{Conjecture}

\theoremstyle{definition}
\newtheorem{definition}{Определение}[section]
\newtheorem{example}{Example}[section]

% unnumbered environments:

\theoremstyle{remark}
\newtheorem*{remark}{Remark}
\newtheorem*{notation}{Notation}
\newtheorem*{note}{Note}

\setlength{\parskip}{5pt plus 2pt minus 1pt}
%\setlength{\parindent}{0pt}

\usepackage{color}
\usepackage{listings}
\usepackage{caption}
\usepackage{graphicx}
\usepackage{ucs}

\newcommand{\tab}[1][0.3cm]{\ensuremath{\hspace*{#1}}}
% A generalized view on parsing and translation
% http://dl.acm.org/citation.cfm?id=2206331
\title{Generalized LL Parsing Generalization}
\author{Semyon Grigorev
\\
       {Saint Petersburg State University}\\
       {7/9 Universitetskaya nab.}\\
       {St. Petersburg, 199034 Russia}\\
       {semen.grigorev@jetbrains.com}, 
       Anastasiya Ragozins, Artyom Gorokhov}
%\author{
%\alignauthor
%       Semyon Grigorev\\
%       \affaddr{Saint Petersburg State University}\\
%       \affaddr{7/9 Universitetskaya nab.}\\
%       \affaddr{St. Petersburg, 199034 Russia}\\
%       \email{semen.grigorev@jetbrains.com}
%\alignauthor
%       Anastasiya Ragozina\\
%       \affaddr{Saint Petersburg State University}\\
%       \affaddr{7/9 Universitetskaya nab.}\\
%       \affaddr{St. Petersburg, 199034 Russia}\\
%       \email{ragozina.anastasiya@gmail.com}
%}

\date{}

\begin{document}

\maketitle

Today data for parsing is not only linear string, and context-free grammar is not only programming language specification.
Classical example is a graph parsing where input is a graph and grammar is a paths constraints specification.
Also you can find such generalizations of parsing as Multi-string parsing presented at Parsing@SLE-2016, Abstract parsing~\cite{AbstractParsing}, ETC.
All of them are special cases of the Bar-Hillel~\cite{Bar-Hillel} theorem and can be generalized, but today many of them are separated solutions.

 Current the next tasks.
\begin{itemize}
\item Context-free path querying for graph data bases. 
\item Context-free pattern search in metagenomical assemblies. Not only regular but also CF-compressed input processing which is actual for metagenomic assembly precessing. 
Sequitur compression algorithm.
\item Multiple input parsing (Scott ... ). For lexing ambigueties solving.
\end{itemize}

We have some exerience in graph parsing and GLL. Our GLL-basd graph-aprsing algorithm is faster 
then presented at WWW~\cite{CFRDFParsing}.  End some ideas of application: Error recovery as a 
graph parsing. 
Thus, the goal of our work is an abstract framework for parsing based on geteralization of GLL parsing 
algorithm~\cite{GLL} which proposed by Scott and J.  Also we want to investigate practical areas of 
application.

\begin{thebibliography}{9}

\bibitem{Bar-Hillel}
  Bar-Hillel, Yehoshua, Micha Perles, and Eliahu Shamir.
  ``On formal properties of simple phrase structure grammars.''
   \emph{Sprachtypologie und Universalienforschung}
   14 (1961): 143-172.

\bibitem{CFRDFParsing}
  Zhang, Xiaowang, et al.
  ``Context-free path queries on RDF graphs.'' 
  \emph{International Semantic Web Conference.}
   Springer International Publishing, 2016.
   632--648.

\bibitem{GLL}
  Scott, Elizabeth, and Adrian Johnstone.   
  ``GLL parsing.'',
  \emph{Electronic Notes in Theoretical Computer Science},
  253.7 (2010): 177--189.

\bibitem{AbstractParsing}
  Doh, Kyung-Goo, Hyunha Kim, and David A. Schmidt.
  ``Abstract LR-parsing.'',
  \emph{Formal Modeling: Actors, Open Systems, Biological Systems.},
  Springer,
  2011.
  90--109.



\end{thebibliography}


\end{document}