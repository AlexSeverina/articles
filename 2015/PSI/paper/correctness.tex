\section{Correctness of the Algorithm}
In this section we prove the correctness of the algorithm proposed, namely \textsc{Theorem 1} proves its termination,
and \textsc{Theorem 2} and \textsc{Theorem 3} prove correctness of parse forest construction.
In \textsc{Definition}, we slightly redefine classical notion of correct parse tree to better suit our problem. 
SPPF construction is inherited from RNGLR-algorithm, so the proofs are presented
in terms of GSS construction correctness. 

\textsc{Theorem 1.}
\textit{Algorithm terminates for any input.}

\textsc{Proof.}
Each vertex of inner representation of the input finite automaton contains, at most, 
$N$ GSS vertices, where $N$ is a number of parser states. So, the total number of 
GSS vertices is, at most, $N\times n$, where $n$ is the number of vertices in the inner graph. 
Since GSS has no multi-edges, the number of its edges is $O((N\times n)^2)$. The algorithm 
dequeues vertex to be processed from $\mathcal Q$ in the each iteration of the 
main loop. Vertices are enqueued to $\mathcal Q$ only when a new edge is added to GSS. Since the number of 
GSS edges is finite, the algorithm always terminates. \qed

\textsc{Definition.} 
\emph{Correct tree} is an ordered tree with the following properties:
\begin{enumerate}
  \item The root is the start nonterminal of the grammar $G$.
  \item The leaf nodes are terminals of $G$. The sequence of the leaf nodes 
        corresponds to some path in the inner graph. 
  \item The interior nodes are nonterminals of $G$. All children of nonterminal 
        $N$ correspond to the symbols of the right-hand side of some production for $N$ in $G$.
\end{enumerate}

\textsc{Lemma.}
For every GSS edge $(v_{t}, v_{h})$, $v_{t} \in V_{t}.processed$, $v_{h} \in V_{h}.processed$, 
the terminals of the associated subtree correspond to some path in the inner graph $p$ 
from $V_{h}$ to $V_{t}$.

\textsc{Proof.}
The proof is by induction on the height of derivation tree. 
The base case is either some $\epsilon$-tree or a tree with the single leaf. An $\epsilon$-tree corresponds 
to a path of zero length; the tail and the head of the edge associated with $\epsilon$-tree are identical, 
thus the statement is true. A tree with the single leaf corresponds to a single terminal read from an edge 
($V_{h}$, $V_{t}$) of the inner graph, thus the statement is true.

A tree of height $k$ has a nonterminal $N$ as its root. By third statement of correct tree definition, 
there is a production $N \rightarrow A_{0}, A_{1}, \dots, A_{n}$ for children $A_{0}, A_{1}, \dots, A_{n}$ of the root node. 
A subtree $A_{i}$ is associated with GSS edge $(v_{t}^{i}, v_{h}^{i})$ and, as its height is $k-1$, by inductive hypothesis,
there is a path in the inner graph from $V_{h}^{i}$ to $V_{t}^{i}$. $V_{t}^i = V_{h}^{i+1}$, since $v_{t}^i = v_{h}^{i+1}$, 
thus there is a path in the inner graph from $V_{h}^{0}$ to $V_{t}^{n}$, corresponding to the tree under consideration.
\qed

\textsc{Theorem 2.} 
\textit{Every generated from SPPF tree is correct.}

\textsc{Proof.} Consider arbitrary tree, generated from SPPF, and prove that it is correct. The first and the third statements
of correctness definition immediately follow from SPPF definition. 

{\bf (did'not understand the following statement; Russian decryption is required:)}
\textsc{Lemma 1} proves the second item of the definition by consideration of all the edges from the GSS vertex
on the last level having accepting state to the vertex on the 0-level with the start parser state.

\qed

\textsc{Theorem 3.} 
\textit{For every path $p$ in the inner graph, a correct tree corresponding to $p$ can be generated from SPPF.}

\textsc{Proof.}
Consider arbitrary correct tree and show it can be generated from SPPF. The proof follows the proof of correctness 
for RNGLR-algorithm, except the following moment. RNGLR constructs GSS layer-by-layer: it is guaranteed, that $j$-th 
level of the GSS $\forall j \in [0..i-1]$ would be fixed by the time, when $i$-th level is processed. In our case, 
this property does not hold, which leads to possible generation of the paths for reductions already applied. 
The only possible way to actually add a new path is to add an edge $(v_{t}, v_{h})$, where $v_{t}$ is already in the GSS and 
it has incoming edges. Since the algorithm stores, which reductions have passed through each vertex, it is sufficient to 
{\bf (what does it mean: ``continue passing reductions?'')} continue passing reductions, stored in $v_{t}$ to overcome this problem, 
and this is exactly what \emph{applyPassingReductions} function does. 
\qed
