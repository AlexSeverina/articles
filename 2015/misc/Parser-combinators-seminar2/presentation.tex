\documentclass{beamer}
\usepackage{beamerthemesplit}
\usepackage{wrapfig}
\usetheme{SPbGU}
\usepackage{pdfpages}
\usepackage{amsmath}
\usepackage{cmap} 
\usepackage[T2A]{fontenc} 
\usepackage[utf8]{inputenc}
\usepackage[english,russian]{babel}
\usepackage{indentfirst}
\usepackage{amsmath}
\usepackage{tikz}
\usepackage{multirow}
\usepackage[noend]{algpseudocode}
\usepackage{algorithm}
\usepackage{algorithmicx}
\usetikzlibrary{shapes,arrows}
\usepackage{fancyvrb}
\newtheorem{rutheorem}{Теорема}
\newtheorem{ruproof}{Доказательство}
\newtheorem{rudefinition}{Определение}
\newtheorem{rulemma}{Лемма}
\beamertemplatenavigationsymbolsempty

\title[]{Парсер-комбинаторы и левая рекурсия}
\subtitle[]{продолжение}
% То, что в квадратных скобках, отображается в левом нижнем углу. 
\institute[]{
Лаборатория языковых инструментов JetBrains \\
Санкт-Петербургский государственный университет \\
Математико-механический факультет }

% То, что в квадратных скобках, отображается в левом нижнем углу.
\author[Екатерина Вербицкая]{Екатерина Вербицкая}

\date{9 ноября 2015г.}

\definecolor{orange}{RGB}{179,36,31}

\begin{document}
{
\begin{frame}[fragile]
  \titlepage
\end{frame}
}

\begin{frame}[fragile]
  \transwipe[direction=90]
  \frametitle{В предыдущей серии}
  Парсер-комбинатор: функция высшего порядка, которая конструирует парсер из набора других парсеров.

  \begin{itemize}
    \item Примитивные парсеры 
    \item Комбинаторы для последовательности, выбора, повторения и т.д.
  \end{itemize}
\end{frame}
            

\begin{frame}[fragile]
  \transwipe[direction=90]
  \frametitle{Монадические парсер-комбинаторы}
\begin{verbatim}
  -- result :: a -> Parser a
  result v = \inp -> [(v,inp)]

  -- bind :: Parser a -> (a -> Parser b) -> Parser b
  p ‘bind‘ f = \inp -> concat [f v out | (v,out) <- p inp]

  -- zero :: Parser a
  zero = \inp -> []
  
  -- (++) :: Parser a -> Parser a -> Parser a
  p ++ q = \inp -> (p inp ++ q inp)
\end{verbatim}
\end{frame}

\begin{frame}[fragile]
  \transwipe[direction=90]
  \frametitle{Преимущества монадических парсер-комбинаторов}
  \begin{itemize}
    \item Простота, гибкость, выразительность, модульность
    \item Возможность откатываться (backtracking)
    \item Лексический анализ не нужно выделять в отдельный шаг
    \item Можно считать семантику во время синтаксического анализа
  \end{itemize}

  \begin{itemize}
    \item Если использовать неграмотно, можно получить непредсказуемое время 
работы и легко исчерпать всю доступную память
    \begin{itemize}
      \item Есть набор практик, которые позволяют избежать проблем
    \end{itemize}
  \end{itemize}
\end{frame}

\begin{frame}[fragile]
  \transwipe[direction=90]
  \frametitle{Левая рекурсия в монадических парсер-комбинаторах}
В явном виде приводит к зацикливанию. Есть комбинатор для частного случая:

\begin{verbatim}
chainl1 :: Parser a -> Parser (a -> a -> a) -> Parser a
p ‘chainl1‘ op = [foldl (\x (f,y) -> f x y) x fys
                    | x <- p
                    , fys <- many [(f,y) | f <- op, y <- p]]
	
expr = factor ‘chainl1‘ addop
addop = [(+) | _ <- char ’+’] ++ [(-) | _ <- char ’-’]
factor = nat ++ bracket (char ’(’) expr (char ’)’)
\end{verbatim}
\end{frame}


\begin{frame}[fragile]
  \transwipe[direction=90]
  \frametitle{Что можно парсить монадическими парсер-комбинаторами?}
  \begin{itemize}
    \item Все контекстно-свободные + некоторое подмножество контекстно-зависимых
  \end{itemize}
  \begin{itemize}
    \item Parsec: "It can parse context-sensitive, infinite look-ahead grammars"
  \end{itemize}
\end{frame}

\begin{frame}[fragile]
  \transwipe[direction=90]
  \frametitle{Packrat-парсер}
  \begin{itemize}               
    \item Unlimited lookahed + backtracking 
    \item Гарантирует линейное время работы на LL(k) и LR(k) грамматиках
    \item Использует запоминание
  \end{itemize}
\end{frame}
         
\begin{frame}[fragile]
  \transwipe[direction=90]
  \frametitle{Борьба с левой рекурсией: Warth et al}
\begin{itemize}
  \item Первый раз, когда наткнулись на левую рекурсию, запомнить в таблице 
ошибку анализа --- посадить семечко
  \item Откатиться во входном потоке в начало правила и применить правило еще раз --- прорастить семечко
\end{itemize}

\begin{itemize}
  \item Определение левой рекурсии:
  \begin{itemize}
    \item В таблице наряду с вычисленным значением храним галочку: 
леворекурсивно ли это правило
    \item Перед применением правила, выставляем галочку в false
    \item Если в таблице нет вычисленного значения, значит попали в левую 
рекурсию: выставляем галочку в true, садим семечко
  \end{itemize}
\end{itemize}
\end{frame}

\begin{frame}[fragile]
  \transwipe[direction=90]
  \frametitle{Warth et al: неявная рекурсия}
\begin{itemize}
  \item Храним стек вызовов правил
  \item При проращивании семечка проращиваем все вовлеченные в левую рекурсию 
правила
\end{itemize}
\end{frame}


\begin{frame}[fragile]
  \transwipe[direction=90]
  \frametitle{Проблемы с подходом Warth et al}
Проблемы с ассоциативностью, если есть и левая, и правая рекурсия

\begin{verbatim}
E :: E - E | Num
\end{verbatim}

"$1-2-3$" разберется как "$1-(2-(3))$". Однако если переписать грамматику 
следующим образом, ассоциативность будет правильной

\begin{verbatim}
E :: E - E (- E)? | Num
\end{verbatim}

Частичное решение: ограничить правую рекурсию максимум одним шагом в глубину

\end{frame}

\begin{frame}[fragile]
  \transwipe[direction=90]
  \frametitle{Parser Expression Grammars}
\begin{itemize}
  \item Похоже на EBNF-форму, но выбор упорядочен
  \item Нет --- неоднозначностям, да --- правильному приоритету!
\end{itemize}

\begin{itemize}
  \item Любая PEG может быть проанализирована packrat-парсером
\end{itemize}

\begin{itemize}
  \item Класс принимаемых языков: не известен
  \begin{itemize}
    \item Все LR(k), LL(k) языки
    \item Какие-то еще КС-языки
    \item Но! Нельзя разобрать $S \rightarrow x \, S \, x \, | x$
    \item Зато! Можно описать не-КС язык ${a^n \, b^n \, c^n}$
  \end{itemize}
\end{itemize}

\begin{verbatim}
S ::= &(A c) a + B !(a/b/c)
A ::= a A? b
B ::= b B? c
\end{verbatim}

\end{frame}

\begin{frame}[fragile]
  \transwipe[direction=90]
  \frametitle{Борьба с левой рекурсией: Hafiz}

\end{frame}

\begin{frame}[fragile]
  \transwipe[direction=90]
  \frametitle{}

\end{frame}

\begin{frame}[fragile]
  \transwipe[direction=90]
  \frametitle{}

\end{frame}

\begin{frame}[fragile]
  \transwipe[direction=90]
  \frametitle{}

\end{frame}

\begin{frame}[fragile]
  \transwipe[direction=90]
  \frametitle{}

\end{frame}

\begin{frame}[fragile]
  \transwipe[direction=90]
  \frametitle{}

\end{frame}

\begin{frame}[fragile]
  \transwipe[direction=90]
  \frametitle{}

\end{frame}

\begin{frame}[fragile]
  \transwipe[direction=90]
  \frametitle{}

\end{frame}

\begin{frame}[fragile]
  \transwipe[direction=90]
  \frametitle{}

\end{frame}

\begin{frame}[fragile]
  \transwipe[direction=90]
  \frametitle{}

\end{frame}


\begin{frame}[fragile]
  \transwipe[direction=90]
  \frametitle{Литература}
\begin{itemize}
\end{itemize}

\end{frame}
\end{document}
