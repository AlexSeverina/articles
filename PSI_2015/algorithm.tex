\section{Relaxed Parsing of Regular Sets}

The input of the algorithm (see Algorithm~\ref{parsing}) is a reference grammar $G$ with alphabet of terminal symbols $T$ 
and a finite non-deterministic automaton $(Q, \Sigma, \delta, q_0, q_f)$ with a single start state $q_0$, single final state $q_f$ 
and no $\epsilon$-transitions, where $\Sigma \subseteq T$~--- alphabet of input symbols, $Q$~--- alphabet of states, 
$\delta$~--- transition relation. RNGLR parser tables and some accessory information (called $parserSource$ in pseudocode) 
are generated for the grammar $G$. 

The general idea of the algorithm is to traverse the automaton graph and sequentially construct GSS, similarly to RNGLR.
However, as we deal with a graph instead of a linear stream, the next symbol turns into the \emph{set of terminals} on the 
all outgoing edges of current vertex. This results in a a different semantics of pushing and reducing (see line~5, 
Algorithm~\ref{processVertex}, and lines~8 and~20, Algorithm~\ref{gss_construction}). We use queue $\mathcal Q$ to control the 
order of automaton graph vertices processing. Every time a new GSS vertex is added, all zero-reductions have to be performed 
and then new tokens have to be shifted, so a corresponging graph vertex has to be enqueueed for further processing. 
Addition of new GSS edge can produce reductions to handle, so the graph vertex at the tail of the added edge has 
also to be enqueueed (see Algorithm~\ref{gss_construction}). Reductions are applied along the paths in GSS, and if we add
a new edge to some tail vertex, which was already presented in GSS, we also have to recalculate all \emph{passing} reductions
(see \emph{applyPassingReductions} function in Algorithm~\ref{processVertex}).

Likewise RNGLR, we associate GSS vertices with positions in the input,
and, in our case, a position coinsides with some state of the input automaton. We construct some
inner data structure (referred to as inner graph) by copying input automaton graph and 
extending each its vertex with the following collections: 

\begin{itemize}
  \item \emph{processed}: GSS vertices, for which all the pushes were processed. 
   This set aggregates all GSS vertices, associated with inner graph vertex.
  \item \emph{unprocessed}: GSS vertices, for which all the pushes are to be processed. 
   This set is analogous to $\mathcal{Q}$ of original RNGLR.
  \item \emph{reductions}: a queue, which is analogous to $\mathcal{R}$ of original RNGLR: 
   all reductions to be processed.
  \item \emph{passingReductionsToHandle}: pairs of GSS vertex and GSS edge to apply 
   passing reductions along them.
\end{itemize}

Besides parser $state$ and $level$ (which is equal to the input automaton state), 
a collection of \emph{passing reductions} is stored in a GSS vertex. Passing reduction is a 
triplet $(startV, N, l)$, representing reductions, whose path contains given GSS vertex. 
This triplet is similar to one describing reduction, where $l$ is a remaining length of the path. 
Passing reductions are stored for every vertex of the path (except for the first and the last) 
during path search in \emph{makeReductions} function (see Algorithm~\ref{processVertex}).

We inherit SPPF construction from the original RNGLR; in our case, 
derivation trees for strings, accumulated along the paths of the input automaton 
graph, are merged. 

\begin{algorithm}[!ht]
\begin{algorithmic}[1]
\caption{Parsing algorithm}
\label{parsing}
\Function{parse}{$grammar, automaton$}
  \State{$inputGraph \gets$ construct inner graph representation of $automaton$}
  \State{$parserSource \gets$ generate RNGLR parser tables for $grammar$}
  \If{$inputGraph$ contains no edges}
    \If{$parserSource$ accepts empty input} {report success}
    \Else { report failure}
    \EndIf
  \Else
    \State{\Call{addVertex}{$inputGraph.startVertex, startState$}}
    \State{$\mathcal{Q}.Enqueue(inputGraph.startVertex)$}
    \While{$Q$ is not empty}
      \State{$v \gets \mathcal{Q}.Dequeue()$}
      \State{\Call{makeReductions}{$v$}}
      \State{\Call{push}{$v$}}
      \State{\Call{applyPassingReductions}{$v$}}
    \EndWhile
    \If{$v_f.level = q_f$ and $v_f.state$ is accepting} {report success}
    \Else { report failure}
    \EndIf
  \EndIf
\EndFunction
\end{algorithmic}
\end{algorithm}

\begin{algorithm}[!ht]
\begin{algorithmic}[1]
\caption{Single-vertex processing}
\label{processVertex}
\Function{push}{$innerGraphV$}
  \State{$\mathcal{U} \gets$ copy $innerGraphV.unprocessed$}
  \State{clear $innerGraphV.unprocessed$}
  \ForAll{$v_{h}$ in $\mathcal{U}$}  
    \ForAll{$e$ in outgoing edges of $innerGraphV$}
      \State{$push \gets$ calculate next state by $v_{h}.state$ and the token on $e$}
      \State{\Call{addEdge}{$v_{h}, e.Target, push, false$}}
      \State{add $v_{h}$ in $innerGraphV.processed$}
    \EndFor
  \EndFor
\EndFunction

\Function{makeReductions}{$innerGraphV$}
  \While{$innerGraphV.reductions$ is not empty}
    \State{$(startV, N, l) \gets innerGraphV.reductions.Dequeue()$}
    \State{find the set of vertices $\mathcal{X}$ reachable from $startV$}
    \State{along the path of length ($l-1$), or $0$ if $l=0$;}
    \State{add $(startV, N, l-i)$ in $v.passingReductions$,}
    \State{where $v$ is an $i$-th vertex of the path}
    \ForAll{$v_{h}$ in $\mathcal{X}$}
      \State{$state_{t} \gets$ calculate new state by $v_{h}.state$ and nonterminal $N$}
      \State{\Call{addEdge}{$v_{h}, startV, state_{t}, (l=0)$}}
    \EndFor
  \EndWhile
\EndFunction

\Function{applyPassingReductions}{$innerGraphV$}
  \ForAll{$(v, edge)$ in $innerGraphV.passingReductionsToHandle$}
    \ForAll{$(startV, N, l) \gets v.passingReductions.Dequeue()$}
      \State{find the set of vertices $\mathcal{X}$,}
      \State{reachable from $edge$ along the path of length ($l-1$)}
      \ForAll{$v_{h}$ in $\mathcal{X}$}
        \State{$state_{t} \gets$ calculate new state by $v_{h}.state$ and nonterminal $N$}
        \State{\Call{addEdge}{$v_{h}, startV, state_{t}, false$}}
      \EndFor
    \EndFor
  \EndFor
\EndFunction
\end{algorithmic}
\end{algorithm}
 
\begin{algorithm}[!ht]
\begin{algorithmic}[1]
\caption{GSS construction}
\label{gss_construction}
\Function{addVertex}{$innerGraphV, state$}
  \If{$innerGraphV.processed$ or $innerGraphV.unprocessed$ contains\\
    vertex $v$ with state = $state$ }
    \State{\Return{($v, false$)}}
  \Else
    \State{$v \gets$ create new vertex for $innerGraphV$ with state $state$}
    \State{add $v$ in $innerGraphV.unprocessed$}
    \ForAll{$e$ in outgoing edges of $innerGraphV$}
      \State{calculate the set of zero-reductions by $v$}
      \State{and the token on $e$ and add them in $innerGraphV.reductions$}
    \EndFor
    \State{\Return{$(v, true$)}}
  \EndIf
\EndFunction

\Function{addEdge}{$v_{h}, innerGraphV, state_{t}, isZeroReduction$}
  \State{$(v_{t}, isNew) \gets$ \Call{addVertex}{$innerGraphV, state_{t}$}}
  \If{GSS does not contain edge from $v_{t}$ to $v_{h}$}
    \State{$edge \gets$ create new edge from $v_{t}$ to $v_{h}$}
    \State{$\mathcal{Q}.Enqueue(innerGraphV)$}
    \If{not $isNew$ and $v_{t}.passingReductions.Count>0$}
      \State{add $(v_{t}, edge)$ in $innerGraphV.passingReductionsToHandle$}
    \EndIf
    \If{not $isZeroReduction$}
      \ForAll{$e$ in outgoing edges of $innerGraphV$}
        \State{calculate the set of reductions by $v$}
        \State{and the token on $e$ and add them in $innerGraphV.reductions$}
      \EndFor
    \EndIf
  \EndIf
\EndFunction
\end{algorithmic}
\end{algorithm}
