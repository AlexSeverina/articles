\section{Relaxed Parsing of Regular Sets}

The input of our algorithm (see Algorithm~\ref{parsing}) is a reference grammar $G$ with alphabet of terminal symbols $T$ 
and a finite non-deterministic automaton $(Q, \Sigma, \delta, q_0, q_f)$ with a single start state $q_0$, single final state $q_f$ 
and no $\epsilon$-transitions, where $\Sigma \subseteq T$~--- alphabet of input symbols, $Q$~--- alphabet of states, 
$\delta$~--- transition relation. RNGLR parse tables and some accessory information (called $parserSource$ in pseudocode) 
are generated for the grammar $G$. 

The general idea of the algorithm is to traverse the automaton graph and sequentially construct GSS, similarly as in RNGLR.
However, as we deal with a graph instead of a linear stream, the next symbol turns into the \emph{set of terminals} on 
all outgoing edges of current vertex. This results in a different semantics of pushing and reducing (see line~5, 
Algorithm~\ref{processVertex}, and lines~9 and~21, Algorithm~\ref{gss_construction}). We use queue $\mathcal Q$ to control the 
order of automaton graph vertices processing. Every time a new GSS vertex is added, all zero-reductions have to be performed 
and then new tokens have to be shifted, so a corresponding graph vertex has to be enqueueed for further processing. 
Addition of new GSS edge can produce reductions to handle, so the graph vertex at the tail of the added edge has 
also to be enqueueed (see Algorithm~\ref{gss_construction}). Reductions are applied along the paths in GSS, and if we add
a new edge to some tail vertex, which was already presented in GSS, we also have to recalculate all \emph{passing} reductions
(see \emph{applyPassingReductions} function in Algorithm~\ref{processVertex}).

Likewise RNGLR, we associate GSS vertices with positions in the input,
and, in our case, a position coincides with some state of input automaton. We construct some
inner data structure (referred to as \emph{inner graph}) by copying input automaton graph and 
extending each its vertex with the following collections: 

\begin{itemize}
  \item \emph{processed}: GSS vertices, for which all the pushes were processed. 
   This set aggregates all GSS vertices, associated with inner graph vertex.
  \item \emph{unprocessed}: GSS vertices, for which all the pushes are to be processed. 
   This set is analogous to $\mathcal{Q}$ of original RNGLR.
  \item \emph{reductions}: a queue, which is analogous to $\mathcal{R}$ of original RNGLR: 
   all reductions to be processed.
  \item \emph{passingReductionsToHandle}: pairs of GSS vertex and GSS edge to apply 
   passing reductions along them.
\end{itemize}

Besides parser $state$ and $level$ (which is equal to the input automaton state), 
a collection of \emph{passing reductions} is stored in a GSS vertex. Passing reduction is a 
triplet $(startV, N, l)$, representing reductions, whose path contains given GSS vertex. 
This triplet is similar to one describing reduction, where $l$ is a remaining length of the path. 
Passing reductions are stored for every vertex of the path (except for the first and the last) 
during path search in \emph{makeReductions} function (see Algorithm~\ref{processVertex}).

We inherit SPPF construction from the original RNGLR; in our case, 
derivation trees for strings, accumulated along the paths of the input automaton 
graph, are merged. 

\begin{algorithm}[!ht]
\begin{algorithmic}[1]
\caption{Parsing algorithm}
\label{parsing}
\Function{parse}{$grammar, automaton$}
  \State{$inputGraph \gets$ construct inner graph representation of $automaton$}
  \State{$parserSource \gets$ generate RNGLR parse tables for $grammar$}
  \If{$inputGraph$ contains no edges}
    \If{$parserSource$ accepts empty input} {report success}
    \Else { report failure}
    \EndIf
  \Else
    \State{\Call{addVertex}{$inputGraph.startVertex, startState$}}
    \State{$\mathcal{Q}.Enqueue(inputGraph.startVertex)$}
    \While{$Q$ is not empty}
      \State{$v \gets \mathcal{Q}.Dequeue()$}
      \State{\Call{makeReductions}{$v$}}
      \State{\Call{push}{$v$}}
      \State{\Call{applyPassingReductions}{$v$}}
    \EndWhile
    \If{$\exists v_f: v_f.level = q_f$ and $v_f.state$ is accepting} {report success}
    \Else { report failure}
    \EndIf
  \EndIf
\EndFunction
\end{algorithmic}
\end{algorithm}

\begin{algorithm}[!ht]
\begin{algorithmic}[1]
\caption{Single vertex processing}
\label{processVertex}
\Function{push}{$innerGraphV$}
  \State{$\mathcal{U} \gets$ copy $innerGraphV.unprocessed$}
  \State{clear $innerGraphV.unprocessed$}
  \ForAll{$v_{h}$ in $\mathcal{U}$}  
    \ForAll{$e$ in outgoing edges of $innerGraphV$}
      \State{$push \gets$ calculate next state by $v_{h}.state$ and the token on $e$}
      \State{\Call{addEdge}{$v_{h}, e.Head, push, false$}}
      \State{add $v_{h}$ in $innerGraphV.processed$}
    \EndFor
  \EndFor
\EndFunction

\Function{makeReductions}{$innerGraphV$}
  \While{$innerGraphV.reductions$ is not empty}
    \State{$(startV, N, l) \gets innerGraphV.reductions.Dequeue()$}
    \State{find the set of vertices $\mathcal{X}$ reachable from $startV$}
    \State{    along the path of length ($l-1$), or $0$ if $l=0$;}
    \State{add $(startV, N, l-i)$ in $v.passingReductions$,}
    \State{    where $v$ is an $i$-th vertex of the path}
    \ForAll{$v_{h}$ in $\mathcal{X}$}
      \State{$state_{t} \gets$ calculate new state by $v_{h}.state$ and nonterminal $N$}
      \State{\Call{addEdge}{$v_{h}, startV, state_{t}, (l=0)$}}
    \EndFor
  \EndWhile
\EndFunction

\Function{applyPassingReductions}{$innerGraphV$}
  \ForAll{$(v, edge)$ in $innerGraphV.passingReductionsToHandle$}
    \ForAll{$(startV, N, l) \gets v.passingReductions.Dequeue()$}
      \State{find the set of vertices $\mathcal{X}$,}
      \State{    reachable from $edge$ along the path of length ($l-1$)}
      \ForAll{$v_{h}$ in $\mathcal{X}$}
        \State{$state_{t} \gets$ calculate new state by $v_{h}.state$ and nonterminal $N$}
        \State{\Call{addEdge}{$v_{h}, startV, state_{t}, false$}}
      \EndFor
    \EndFor
  \EndFor
\EndFunction
\end{algorithmic}
\end{algorithm}
 
\begin{algorithm}[!ht]
\begin{algorithmic}[1]
\caption{GSS construction}
\label{gss_construction}
\Function{addVertex}{$innerGraphV, state$}
  \State{$v \gets$ find a vertex with state $=state$ in}
  \State{    $innerGraphV.processed \cup innerGraphV.unprocessed$}
  \If{$v$ is not $null$ } \Comment{The vertex have been found in GSS}
    \State{\Return{($v, false$)}} 
  \Else
    \State{$v \gets$ create new vertex for $innerGraphV$ with state $state$}
    \State{add $v$ in $innerGraphV.unprocessed$}
    \ForAll{$e$ in outgoing edges of $innerGraphV$}
      \State{calculate the set of zero-reductions by $v$}
      \State{    and the token on $e$ and add them in $innerGraphV.reductions$}
    \EndFor
    \State{\Return{$(v, true$)}}
  \EndIf
\EndFunction

\Function{addEdge}{$v_{h}, innerGraphV, state_{t}, isZeroReduction$}
  \State{$(v_{t}, isNew) \gets$ \Call{addVertex}{$innerGraphV, state_{t}$}}
  \If{GSS does not contain edge from $v_{t}$ to $v_{h}$}
    \State{$edge \gets$ create new edge from $v_{t}$ to $v_{h}$}
    \State{$\mathcal{Q}.Enqueue(innerGraphV)$}
    \If{not $isNew$ and $v_{t}.passingReductions.Count>0$}
      \State{add $(v_{t}, edge)$ in $innerGraphV.passingReductionsToHandle$}
    \EndIf
    \If{not $isZeroReduction$}
      \ForAll{$e$ in outgoing edges of $innerGraphV$}
        \State{calculate the set of reductions by $v$}
        \State{    and the token on $e$ and add them in $innerGraphV.reductions$}
      \EndFor
    \EndIf
  \EndIf
\EndFunction
\end{algorithmic}
\end{algorithm}

We conclude this section by justification of termination and correctness of our algorithm.

\textsc{Theorem 1.}
\textit{Algorithm terminates for any input.}

\textsc{Proof.}
Each vertex of inner graph contains, at most, 
$N$ GSS vertices, where $N$ is the number of parser states. So, the total number of 
GSS vertices is, at most, $N\times n$, where $n$ is the number of vertices in the inner graph. 
Since GSS has no multi-edges, the number of its edges is $O((N\times n)^2)$. The algorithm 
dequeues some vertex to process from $\mathcal Q$ in each iteration of the 
main loop. Vertices are enqueued to $\mathcal Q$ only when a new edge is added to GSS. Since the number of 
GSS edges is finite, the algorithm always terminates. \qed

To prove correctness, we first introduce the following definition:

\textsc{Definition.} 
\emph{Correct tree} is an ordered tree with the following properties:
\begin{enumerate}
  \item The root is the start nonterminal of the grammar $G$.
  \item The leaf nodes are terminals of $G$. The sequence of the leaf nodes 
        corresponds to some path in the inner graph. 
  \item The interior nodes are nonterminals of $G$. All children of nonterminal 
        $N$ correspond to the symbols of the right-hand side of some production for $N$ in $G$.
\end{enumerate}

Informally, a correct tree is a derivation tree (w.r.t. reference grammar) for some word in 
regular approximation. Now we have to prove that, first, SPPF contains only correct trees, 
and second, that for any recognized by reference grammar string there is some correct tree in
the SPPF.

\textsc{Lemma.}
For every GSS edge $(v_{t}, v_{h})$, $v_{t} \in V_{t}.processed$, $v_{h} \in V_{h}.processed$, 
the terminals of the associated subtree correspond to some path in the inner graph $p$ 
from $V_{h}$ to $V_{t}$.

\textsc{Proof.}
The proof is by induction on the height of derivation tree. 
The base case is either some $\epsilon$-tree or a tree with a single leaf. An $\epsilon$-tree corresponds 
to a path of zero length; the tail and the head of the edge associated with $\epsilon$-tree are identical, 
thus the statement is true. A tree with the single leaf corresponds to a single terminal read from an edge 
($V_{h}$, $V_{t}$) of the inner graph, thus the statement is true.

A tree of height $k$ has a nonterminal $N$ as its root. By third statement of correct tree definition, 
there is a production $N \rightarrow A_{0}, A_{1}, \dots, A_{n}$ for children $A_{0}, A_{1}, \dots, A_{n}$ of the root node. 
A subtree $A_{i}$ is associated with GSS edge $(v_{t}^{i}, v_{h}^{i})$ and, as its height is $k-1$, by inductive hypothesis,
there is a path in the inner graph from $V_{h}^{i}$ to $V_{t}^{i}$. $V_{t}^i = V_{h}^{i+1}$, since $v_{t}^i = v_{h}^{i+1}$, 
thus there is a path in the inner graph from $V_{h}^{0}$ to $V_{t}^{n}$, corresponding to the tree under consideration.
\qed

\textsc{Theorem 2.} 
\textit{Every tree, generated from SPPF, is correct.}

\textsc{Proof.} Consider arbitrary tree, generated from SPPF, and prove that it is correct. The first and the third statements
of correctness definition immediately follow from SPPF definition. The second statement of the definition follows from \textsc{Lemma 1} 
by considering all edges from GSS vertices on the last level, labeled by accepting state, to the vertices on level 0.
\qed

\textsc{Theorem 3.} 
\textit{For every path $p$ in the inner graph, recognized w.r.t. reference grammar, a correct tree corresponding to $p$ can 
be generated from SPPF.}

\textsc{Proof.}
Consider arbitrary correct tree and show it can be generated from SPPF. The proof follows the proof of correctness 
for RNGLR algorithm, except for the following moment. RNGLR constructs GSS layer-by-layer: it is guaranteed, that $\forall j \in [0..i-1]$ 
$j$-th level of the GSS would be fixed by the time, when $i$-th level is processed. In our case, 
this property does not hold, which leads to a possible generation of some paths for already applied reductions. 
The only possible way to actually add a new path is to add an edge $(v_{t}, v_{h})$, where $v_{t}$ is already in the GSS and 
it has some incoming edges. Since the algorithm stores which reductions have passed through each vertex, to overcome this problem 
it is sufficient to continue passing reductions, stored in $v_{t}$, and this is exactly what \emph{applyPassingReductions} 
function does. 
\qed
