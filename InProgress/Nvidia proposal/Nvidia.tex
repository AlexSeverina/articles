\documentclass[12pt]{article}  % standard LaTeX, 12 point type


\bibliographystyle{unsrt}
\usepackage{geometry}

\usepackage{amsmath}
\usepackage{amsfonts,latexsym}
\usepackage{amsthm}
\usepackage{amssymb}
\usepackage[utf8]{inputenc} % Кодировка
\usepackage[english]{babel} % Многоязычность
\usepackage{verbatim}
\usepackage{longtable}
\usepackage{csvsimple}
\usepackage[toc,page]{appendix}
\usepackage{booktabs}

\usepackage{float}
\usepackage{array}
\usepackage{multirow}
\usepackage{caption}
\usepackage{graphicx}
\usepackage{ucs}
\usepackage{rotating}
\usepackage{pdflscape}
\usepackage{afterpage}
\usepackage{capt-of}% or use the larger `caption` package
\usepackage{url}

% unnumbered environments:

\theoremstyle{remark}
\newtheorem*{remark}{Remark}
\newtheorem*{notation}{Notation}
\newtheorem*{note}{Note}

\setlength{\parskip}{5pt plus 2pt minus 1pt}
\newcolumntype{C}{>{\centering\arraybackslash}p{1.3cm}}
\graphicspath{{pics/}}

\title{Matrix-based path querying algorithms on the GPU}
\author{Rustam Azimov}
\date{}

\begin{document}

\newgeometry{left=0.8in,right=0.8in,top=1in,bottom=1in}

\maketitle

\section*{Keywords}
CFPQ, context-free path querying, graph databases, linear algebra, matrix operations, GPGPU, sparse matrices, formal language theory, context-free grammars.

\section{Introduction}
Graphs are used as a data structure to represent large volumes of information in a compact and convenient for analysis form in many areas: graph databases~\cite{redis,graphDB, kuijpers2019experimental}, bioinformatics~\cite{have2013graph, GraphQueryWithEarley}, static code analysis~\cite{kodumal2004set, zhang2013fast}, etc. In these areas, it is necessary to evaluate queries for large graphs in order to determine the dependencies between the nodes. 

The most popular queries are path queries which use regular or context-free grammars for constraints on the path. The use of context-free instead of regular grammars in path querying allows us to formulate more complex queries to the graph and solve a wider class of problems.

There is a number of algorithms for context-free path query evaluation (CFPQ)~\cite{GraphQueryWithEarley, GLL, hellingsRelational, RDF}. However, these algorithms demonstrate a poor performance when applied to large graphs.

One of the most popular techniques used to increase performance when working with large data is the use of a GPU to perform computation. We proposed the only known approach in this area that makes it possible to use the GPUs effectively~\cite{azimov2018context}. This is the matrix approach, in which the adjacency matrix of the input graph is built, the elements of which are sets of non-terminals of the input grammar. Next, the transitive closure of the constructed matrix is calculated using the derivation rules of the input grammar. In the process of computing, the operations of multiplication and addition of Boolean matrices are actively used and can be computed very efficiently using the GPUs. For example, such libraries as cuBLAS, cuSPARSE, or CUTLASS provide high-performance implementations of necessary matrix operations (GEMM, etc.). We are especially interested in the CUTLASS library because it provides the ability to customize matrix types. We already have some implementations~\cite{mishin2019evaluation} which show that GPGPU utilization for boolean matrices multiplication can significantly increase the performance of CFPQs evaluation.

The project is devoted to the study of our new matrix-based path querying approach and creating a high-performance implementations for analyzing large graphs. The active use of matrix operations in our approach makes it possible to efficiently apply a wide class of matrix optimizations and computing techniques (GPGPU, parallel processing, sparse matrix representation, distributed-memory computation, etc.).

\section{Motivation}
There are other projects which are interested in a high-performance context-free path querying implementations.

There is the GraphBLAS that provides standardized building blocks of graph algorithms, optimizes and simplifies computation of many different graph queries. We plan to provide a high-performance context-free path querying implementations as part of GraphBLAS. In these implementations we will use sparse matrix representation and with the help of Nvidia libraries we will apply such optimizations as GPGPU and distributed-memory computation.

Also, there is the industry’s most widely adopted graph database query language Cypher. The context-free path querying can be very useful when applied to the graph databases~\cite{kuijpers2019experimental}. Therefore, Cypher is interested in supporting context-free path querying and having a high-performance implementations for these queries.

In addition, there is the RedisGraph graph database which uses both GraphBLAS and Cypher query language. Therefore, we also plan to contribute a high-performance implementations of the expressive graph database queries to the RedisGraph. RedisGraph already uses linear algebra blocks from GraphBLAS, however, providing a GPU implementations for these blocks will significantly speed up graph database query evaluation.

\bibliography{Nvidia}

\end{document}
