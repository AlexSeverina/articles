% This is "sig-alternate.tex" V1.9 April 2009
% This file should be compiled with V2.4 of "sig-alternate.cls" April 2009
%
% This example file demonstrates the use of the 'sig-alternate.cls'
% V2.4 LaTeX2e document class file. It is for those submitting
% articles to ACM Conference Proceedings WHO DO NOT WISH TO
% STRICTLY ADHERE TO THE SIGS (PUBS-BOARD-ENDORSED) STYLE.
% The 'sig-alternate.cls' file will produce a similar-looking,
% albeit, 'tighter' paper resulting in, invariably, fewer pages.
%
% ----------------------------------------------------------------------------------------------------------------
% This .tex file (and associated .cls V2.4) produces:
%       1) The Permission Statement
%       2) The Conference (location) Info information
%       3) The Copyright Line with ACM data
%       4) NO page numbers
%
% as against the acm_proc_article-sp.cls file which
% DOES NOT produce 1) thru' 3) above.
%
% Using 'sig-alternate.cls' you have control, however, from within
% the source .tex file, over both the CopyrightYear
% (defaulted to 200X) and the ACM Copyright Data
% (defaulted to X-XXXXX-XX-X/XX/XX).
% e.g.
% \CopyrightYear{2007} will cause 2007 to appear in the copyright line.
% \crdata{0-12345-67-8/90/12} will cause 0-12345-67-8/90/12 to appear in the copyright line.
%
% ---------------------------------------------------------------------------------------------------------------
% This .tex source is an example which *does* use
% the .bib file (from which the .bbl file % is produced).
% REMEMBER HOWEVER: After having produced the .bbl file,
% and prior to final submission, you *NEED* to 'insert'
% your .bbl file into your source .tex file so as to provide
% ONE 'self-contained' source file.
%
% ================= IF YOU HAVE QUESTIONS =======================
% Questions regarding the SIGS styles, SIGS policies and
% procedures, Conferences etc. should be sent to
% Adrienne Griscti (griscti@acm.org)
%
% Technical questions _only_ to
% Gerald Murray (murray@hq.acm.org)
% ===============================================================
%
% For tracking purposes - this is V1.9 - April 2009

\documentclass{sig-alternate-05-2015}
  \pdfpagewidth=8.5truein
  \pdfpageheight=11truein

\usepackage{verbatim}
\usepackage{graphicx}
\usepackage{subcaption}
\usepackage{hyperref}
\usepackage{listings}
\usepackage{courier}
\usepackage{epstopdf}

% \lstset{language=[Sharp]C}
\lstset{numbers=left,xleftmargin=3em,numberstyle=\footnotesize\ttfamily,captionpos=b}
\lstset{basicstyle=\footnotesize\ttfamily}

\begin{document}
%\setcopyright{acmlicensed}
%
% --- Author Metadata here ---
% \conferenceinfo{SAC'15}{April 13-17, 2015, Salamanca, Spain.}
% \CopyrightYear{2015} % Allows default copyright year (2002) to be over-ridden - IF NEED BE.
% \crdata{978-1-4503-3196-8/15/04}  % Allows default copyright data (X-XXXXX-XX-X/XX/XX) to be over-ridden.
% --- End of Author Metadata ---

\title{On Development of Static Analysis Tools for String-Embedded Languages}
% \subtitle{[Extended Abstract]
% \titlenote{A full version of this paper is available as
% \textit{Author's Guide to Preparing ACM SIG Proceedings Using
% \LaTeX$2_\epsilon$\ and BibTeX} at
% \texttt{www.acm.org/eaddress.htm}}}
%
% You need the command \numberofauthors to handle the 'placement
% and alignment' of the authors beneath the title.
%
% For aesthetic reasons, we recommend 'three authors at a time'
% i.e. three 'name/affiliation blocks' be placed beneath the title.
%
% NOTE: You are NOT restricted in how many 'rows' of
% "name/affiliations" may appear. We just ask that you restrict
% the number of 'columns' to three.
%
% Because of the available 'opening page real-estate'
% we ask you to refrain from putting more than six authors
% (two rows with three columns) beneath the article title.
% More than six makes the first-page appear very cluttered indeed.
%
% Use the \alignauthor commands to handle the names
% and affiliations for an 'aesthetic maximum' of six authors.
% Add names, affiliations, addresses for
% the seventh etc. author(s) as the argument for the
% \additionalauthors command.
% These 'additional authors' will be output/set for you
% without further effort on your part as the last section in
% the body of your article BEFORE References or any Appendices.

\numberofauthors{3} %  in this sample file, there are a *total*
% of EIGHT authors. SIX appear on the 'first-page' (for formatting
% reasons) and the remaining two appear in the \additionalauthors section.
%
\author{
% You can go ahead and credit any number of authors here,
% e.g. one 'row of three' or two rows (consisting of one row of three
% and a second row of one, two or three).
%
% The command \alignauthor (no curly braces needed) should
% precede each author name, affiliation/snail-mail address and
% e-mail address. Additionally, tag each line of
% affiliation/address with \affaddr, and tag the
% e-mail address with \email.
%
% 1st. author
% 1st. author
\alignauthor Marat Khabibullin\\
       \affaddr{St. Petersburg Academic University}\\
       \affaddr{194021, Khlopina Str 8/3}\\
       \affaddr{St. Petersburg, Russia}\\
       \email{maratx387@gmail.com}
% 2nd. author       
\alignauthor Andrei Ivanov\\ 
       \affaddr{St. Petersburg State University}\\
       \affaddr{198504, Universitetsky prospekt 28}\\
       \affaddr{Peterhof, St. Petersburg, Russia}\\
       \email{ivanovandrew2004@gmail.com}
\and
% 3rd. author
\alignauthor Semyon Grigorev\\ 
       \affaddr{St. Petersburg State University}\\
       \affaddr{198504, Universitetsky prospekt 28}\\
       \affaddr{Peterhof, St. Petersburg, Russia}\\
       \email{rsdpisuy@gmail.com}
}
% There's nothing stopping you putting the seventh, eighth, etc.
% author on the opening page (as the 'third row') but we ask,
% for aesthetic reasons that you place these 'additional authors'
% in the \additional authors block, viz.
% \additionalauthors{Additional authors: John Smith (The
% Th{\o}rv{\"a}ld Group, email: {\texttt{jsmith@affiliation.org}})
% and Julius P.~Kumquat (The Kumquat Consortium, email:
% {\texttt{jpkumquat@consortium.net}}).}
\date{27 July 2015}
% Just remember to make sure that the TOTAL number of authors
% is the number that will appear on the first page PLUS the
% number that will appear in the \additionalauthors section.

\maketitle

\begin{abstract}

Some programs can produce string expressions with embedded code in other programming languages while running. This embedded code should be syntactically correct as it is typically executed by some subsystem. A program in Java language that builds and sends SQL queries to the database it works with can be considered as an example. In such scenarios, languages like SQL are called string-embedded and ones like Java -- host languages.

In spite of the fact such an approach of programs building is being replaced by alternative ones, for example by ORM and LINQ, string-embedding is still used in practice. Development and reengineering of the programs with string-embedded languages is complicated because the IDE and similar tools process the code embedded in strings as host language string literals and cannot provide the functionality to work with this code. To facilitate the development process, string-embedded code highlighting, completion, navigation and static errors checking would be useful. For the purposes of reengineering, embedded code metrics computation would be helpful.

Currently existing tools to string-embedded languages support only operate with one host language and a fixed set of string-embedded ones. Their functionality is often limited. Moreover, it is almost impossible or requires a substantial amount of work to add a support for both new host and string-embedded language. Attempts to extend their functionality often result in the same problem. 

In this paper we present the platform which can be used for relatively fast and easy building of endpoint tools that provide a support for different string-embedded languages inside different host languages. The tools built for T-SQL and arithmetic expressions language embedding in C\# are demonstrated as the examples of how the platform can be used.
\end{abstract}

\begin{CCSXML}
<ccs2012>
    <concept>
        <concept_id>10011007.10010940.10010992.10010998.10011000</concept_id>
        <concept_desc>Software and its engineering~Automated static analysis</concept_desc>
        <concept_significance>500</concept_significance>
    </concept>
    <concept>
        <concept_id>10011007.10011006.10011073</concept_id>
        <concept_desc>Software and its engineering~Software maintenance tools</concept_desc>
        <concept_significance>500</concept_significance>
    </concept>
    <concept>
        <concept_id>10003752.10010124.10010138.10010143</concept_id>
        <concept_desc>Theory of computation~Program analysis</concept_desc>
        <concept_significance>300</concept_significance>
    </concept>
    <concept>
        <concept_id>10003752.10010124.10010138.10010145</concept_id>
        <concept_desc>Theory of computation~Parsing</concept_desc>
        <concept_significance>300</concept_significance>
    </concept>
</ccs2012>
\end{CCSXML}

\ccsdesc[500]{Software and its engineering~Automated static analysis}
\ccsdesc[500]{Software and its engineering~Software maintenance tools}
\ccsdesc[300]{Theory of computation~Program analysis}
\ccsdesc[300]{Theory of computation~Parsing}

\printccsdesc

\keywords{String-embedded language, integrated development environment, IDE, approximation, control flow graph, CFG}

\section{Introduction}
\section{}
\section{}

\bibliographystyle{abbrv}
\bibliography{sigproc}

\balancecolumns

\end{document}