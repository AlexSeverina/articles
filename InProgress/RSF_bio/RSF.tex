\documentclass[12pt]{article}  % standard LaTeX, 12 point type

\usepackage{geometry}

\usepackage{amsmath}
\usepackage{amsfonts,latexsym}
\usepackage{amsthm}
\usepackage{amssymb}
\usepackage[utf8]{inputenc} % Кодировка
\usepackage[english,russian]{babel} % Многоязычность
\usepackage{verbatim}
\usepackage{longtable}
\usepackage{csvsimple}
\usepackage[toc,page]{appendix}
\usepackage{booktabs}

\usepackage{float}
\usepackage{array}
\usepackage{multirow}
\usepackage{caption}
\usepackage{graphicx}
\usepackage{ucs}
\usepackage{rotating}
\usepackage{pdflscape}
\usepackage{afterpage}
\usepackage{capt-of}% or use the larger `caption` package
\usepackage{url}

% unnumbered environments:

\theoremstyle{remark}
\newtheorem*{remark}{Remark}
\newtheorem*{notation}{Notation}
\newtheorem*{note}{Note}

\setlength{\parskip}{5pt plus 2pt minus 1pt}
\newcolumntype{C}{>{\centering\arraybackslash}p{1.3cm}}
\graphicspath{{pics/}}

\title{Использование формальных грамматик и искусственных нейронных сетей для анализа вторичной структуры геномных и протеомных  последовательностей}
\author{Семён Григорьев}
\date{\today}

\begin{document}

\newgeometry{left=0.8in,right=0.8in,top=1in,bottom=1in}

\maketitle

\section{Сведения о проекте}

\subsection{Название проекта}

\textbf{ru}\\
%+
Использование формальных грамматик и искусственных нейронных сетей для анализа вторичной структуры геномных и протеомных  последовательностей
\\
\\
\textbf{en}\\

\subsection{Направление из Стратегии НТР РФ}
%+
Н3 Переход к персонализированной медицине, высокотехнологичному здравоохранению и технологиям здоровьесбережения, в том числе за счет рационального применения лекарственных препаратов (прежде всего антибактериальных)

\subsection{Обоснование соответствия тематики проекта направлению из Стратегии НТР РФ: необходимо кратко сформулировать научную проблему (проблемы) и конкретные задачи в рамках выбранного направления, решению которых будет посвящен проект, обосновать соответствие проекта направлению}
\textbf{ru}\\
%+
Проект посвящён разработке методов анализа вторичной структуры цепочек с использованием формальных грамматик и искусственных нейронных сетей.

В рамках проекта ставятся следующие задачи.
Во-первых, необходимо сформулировать общие принципы построения формальных грамматик, описывающих вторичную структуру различных типов цепочек.
Во-вторых, разработать алгоритмы синтаксического анализа, пригодные для высокопроизводительной обработки реальных цепочек на основе построенных грамматик.
В-третьих, необходимо исследовать возможности совмещения алгоритмов синтаксического анализа с искусственными нейронными сетями (ИНС) для решения таких прикладных задач, как, на пример, поиск цепочек с аналогичными вторичными структурами.

Решение этих задач позволит создавать решения, применимые в таких областях, как анализ сообществ микроорганизмов, анализ генетической информации, поиск новых лекарственных препаратов.

При анализе сообществ микроорганизмов, что часто необходимо при диагностике различных заболеваний, один из подходов заключается в поиске маркерных цепочек, некоторые из которых обладают характерной вторичной структурой (например, 16s РНК), с последующей их классификацией.

Вместе с тем, анализ структурных особенностей белковых, геномных и других последовательностей необходим для эффективного поиска новых лекарственных препаратов, в том числе антибактериальных. Так, например, поиск новых антибактериальных препаратов часто основан на поиске соединений, структурно аналогичных уже известным, обладающим антибактериальными свойствами. В других же случаях требуется анализ структуры цепочки-мишени для более прицельного поиска препарата. Данные подходы могут совмещаться.
\\
\\
\textbf{en}\\


\subsection{Ключевые слова (приводится не более 15 терминов)}

\textbf{ru}\\
%+
Формальные грамматики, синтаксический анализ, параллельные алгоритмы, вторичная структура, РНК, геномные последовательности, белки, протеомные последовательности, метагеномная сборка, искусственные нейронные сети.
\\
\\
\textbf{en}\\


\subsection{Аннотация проекта}
%(объемом не более 2 стр.; в том числе кратко – актуальность решения указанной выше научной проблемы и научная новизна)
\textbf{ru}\\
%+
Различные молекулярные соединения, такие как белковые молекулы или ДНК/РНК-молекулы, часто рассматривают как цепочки, состоящие из последовательно соединённых более простых элементов-оснований (например, аминокислот или нуклеотидов).
При этом, кроме последовательных связей между основаниями образуются также дополнительные --- вторичные --- связи, которые задают вторичную структуру цепочки.
Известно, что вторичная структура некоторых цепочек обладает характерными особенностями.
Классический пример --- вторичная структура транспортной РНК: первичная структура (последовательность нуклеотидов) может сильно различаться даже у достаточно близких организмов, однако некоторые особенности вторичной структуры (характерный "крест") наблюдаются практически у всех организмов.

Также выяснено, что часто вторичная структура несёт существенную информацию о функциональной роли той или иной цепочки.

Всвязи с этим, важной задачей является разработка формальных методов для описания вторичной структуры и её особенностей.
При этом важно, чтобы полученные формальные модели позволяли создавать эффективные решения для прикладных задач, требующих анализа вторичной структуры.
Например, один из самых точных подходов к анализу вторичной структуры основан на анализе энергии межмолекулярного взаимодействия.
Однако данный подход трудно применим на практике при анализе больших объёмов данных ввиду его высокой вычислительной сложности.

Проект посвящён исследованию применимости формальных грамматик в качестве формальной модели для описания вторичной структуры различных типов цепочек, например, геномных или белковых, а также разработке соответствующих алгоритмов, позволяющих строить применимые на практике решения.

С одной стороны, применение результатов теории формальных языков для анализа биологических последовательностей исследуется достаточно давно.
В качестве примера можно привести результаты Шона Эдди (Sean Eddy) и инструмент Infernal.
С другой, грамматики, в основном, применялись для описания первичной структуры цепочек: предпринимались попытки анализировать цепочки как текст над некоторым алфавитом (набором оснований).
Применение формальных грамматик для описания вторичной структуры исследовано слабо.
Вместе с этим, появились новые результаты в области формальных языков, предложены новые типы грамматик (например, конъюнктивные), обладающие высокой выразительной силой и при этом позволяющие построение эффективных алгоритмов синтаксического анализа.
Применимость данных типов грамматик для описания вторичной структуры исследовано недостаточно.
Таким образом, планируется получение новых результатов, связанных с применением новых типов грамматик для описания вторичной структуры цепочек.

Также будет исследована возможность применения обыкновенных, не вероятностных, грамматик для описания вторичной структуры.
Современные подходы предполагают использование вероятностных грамматик для описания цепочек, что связано с тем, что реальные данные содержат большое количество мутаций и привнесённых шумов, что делает невозможным построение точных моделей.
В данном исследовании предлагается изучить вопрос использования обыкновенных грамматик, а в качестве вероятностной модели использовать искусственную нейронную сеть, что является новым подходом к использованию грамматик.
\\
\\
\textbf{en}\\


\subsection{Ожидаемые результаты и их значимость}
%(указываются результаты, их научная и общественная значимость (соответствие предполагаемых результатов мировому уровню исследований, возможность практического использования предполагаемых результатов проекта в экономике и социальной сфере))

\textbf{ru}\\
%+
В результате изучения применимости различных типов грамматик для описания вторичной структуры будут, во-первых, выявлены основные принципы построения грамматик для конкретных типов цепочек, а также предложены конкретные грамматики для некоторых типов цепочек и некоторых задач.

Предполагается, что будут разработаны новые алгоритмы синтаксического анализа, учитывающие особенности решаемой задачи, такие как, с одной стороны, свойства используемых грамматик (сильная неоднозначность), и с другой стороны возможности современного аппаратного обеспечения, такие как массовый параллелизм.

Также будет сформулирован метод совмещения обыкновенных грамматик и ИНС для решения задач анализа вторичной структуры цепочек.

В совокупности данные результаты должны позволить создавать прикладные решения, применимые как в исследовательских, так и в прикладных задачах биологии и медицины.
\\
\\
\textbf{en}\\


\section{Содержание проекта}

\subsection{Научная проблема, на решение которой направлен проект}

\textbf{ru}\\
%+
Проект посвящён разработке формальных моделей для описания вторичной структуры биологических цепочек и алгоритмов для решения задач анализа структуры, на них основанных. Таким образом, проблемы, решаемые в проекте, лежат в области биоинформатики.

Качественное решение прикладных задач невозможно без существования удачных формальных моделей, позволяющих не только успешно осуществлять поиск решения поставленной задачи, но и, с одной стороны, формализовать постановку задачи, с другой, получить механизмы оценки качества решения.
При этом, удачная формальная модель должна совмещать в себе два важных качества: с одной стороны быть достаточно выразительной, с другой, позволять эффективные реализации алгоритмов для решения прикладных задач.
Необходимо учитывать, что одной из ключевых особенностей прикладных задач в данной области является большой объём обрабатываемых данных.

Поиск такой модели для описания структуры биологических цепочек (например, белковых или геномных) ведётся на протяжении длительного времени.
С одной стороны, существуют модели, пытающиеся максимально полно учесть химические и физические законы взаимодействия между молекулами (например, модели, основанные на анализе энергии межмолекулярных связей).
Данные модели точны, однако громоздки как с точки зрения формальных рассуждений, так и с точки зрения эффективной реализации соответствующих алгоритмов, которые оказываются очень ресурсоёмкими и малопригодными для обработки реальных данных.
С другой стороны, существуют модели, рассматривающие такие цепочки как последовательный набор оснований и трактующие их, например, как строки в некотором алфавите (например, \{A,C,G,T\}).
Такие модели позволяют применять эффективные алгоритмы обработки строк, однако являются недостаточно точными для решения многих задач, так как не учитывают информацию о структурных особенностях цепочек (например, о вторичной структуре).

Один из подходов, активно исследуемых в настоящее время использует формальные грамматики для описания свойств цепочек.
Преимуществом является возможность привлечения обширных результатов теории формальных языков, развивающейся длительное время.
Теория формальных языков, с одной стороны, может предложить богатую теоретическую базу, с другой, эффективные алгоритмы.
При этом, выбирая класс грамматик можно подбирать баланс между выразительностью и возможность получить эффективную реализацию.

В рамках данного исследования планируется построение и изучение теоретических и практических свойств моделей, использующих формальные грамматики для описания вторичной структуры цепочек.
\\
\\
\textbf{en}\\



\subsection{Научная значимость и актуальность решения обозначенной проблемы}

\textbf{ru}\\
%+
Удачные с теоретической точки зрения модели позволяют эффективно рассуждать о свойствах изучаемых объектов, выдвигать и проверять новые научные гипотезы, прогнозировать границы разрешимости прикладных задач. Например, таких важных задач, как поиск маркерных последовательностей для обнаружения организмов, в том числе новых, ранее не изученных, или поиск лекарств (в том числе антибактериальных).

При этом, необходимо найти такие модели, которые при должной выразительности будут позволять реализовывать эффективные алгоритмические и прикладные решения.
Построение таких моделей востребовано ввиду большого объёма данных, требующих обработки при решении прикладных задач.

Кроме этого, в ходе исследования в данной области могут возникнуть новые задачи в области алгоритмов синтаксического анализа и теории формальных языков, что будет способствовать развитию данных областей.
\\
\\
\textbf{en}\\



\subsection{Конкретная задача (задачи) в рамках проблемы, на решение которой направлен проект, ее масштаб и комплексность}

\textbf{ru}\\
%+
В рамках изучения формальных грамматик в качестве средства описания вторичной структуры планируется изучение применимости обыкновенных (не вероятностных) контекстно-свободных и конъюнктивных грамматик для анализа вторичной структуры геномных и белковых цепочек.
В частности, планируется построение грамматик для конкретных задач, имеющих важное прикладное значение, таких как, например, поиск маркерных последовательностей.

Вместе с этим планируется построение алгоритмов, позволяющих проводить синтаксический анализ соответствующих классов языков и допускающих реализации, эффективно использующие возможности современного аппаратного обеспечения, такие как массовый параллелизм и распределённые вычисления.
Кроме того, разрабатываемые алгоритмы должны быть специализированы для работы с сильно неоднозначными грамматиками и решения специфичных задач, таких как поиск подстроки с заданной вторичной структурой.

Также планируется вести поиск новых подходов, позволяющих построить не только обозримые формальные модели, но и эффективные на практике решения по анализу вторичной структуры.
Одним из направлений будет совмещение методов теории формальных языков и синтаксического анализа с подходами машинного обучения.
\\
\\
\textbf{en}\\


\subsection{Научная новизна исследований, обоснование достижимости решения поставленной задачи (задач) и возможности получения запланированных результатов}

\textbf{ru}\\
%+
Поиск эффективных моделей для описания вторичной структуры цепочек активно ведётся в настоящее время.
Сформулированные задачи, с одной стороны, опираются на имеющиеся результаты, которые говорят о том, что формальные грамматики могут применяться для решения задач анализа биологических цепочек, с другой стороны, нацелены на улучшение существующих моделей.
Это, с одной стороны, позволяет говорить о возможности получения запланированных результатов, а с другой, о том, что любое разумное улучшение, как в выразительном плане, так и в смысле возможности построения эффективных реализаций, будет новым результатом.
Стоит отметить, что применение таких классов грамматик, как конъюнктивные, в данной области изучено крайне слабо, а у руководителя есть опыт применения таких грамматики и разработки алгоритмов синтаксического анализа для них.

Кроме того, часть задач сформулирована ранее, представлена и обсуждалась на международных конференциях, что говорит об их актуальности и возможности гарантировать новизну полученных результатов.

У руководителя проекта есть опыт исследований в области формальных грамматик и алгоритмов синтаксического анализа, что должно помочь решить поставленное задачи, так как они лежат в этой же области.
Кроме того, руководителем предложен метод совмещения формальных грамматик и методов машинного обучения для решения задач анализа вторичной структуры, который был успешно представлен на международной конференции.
\\
\\
\textbf{en}\\


\subsection{Современное состояние исследований по данной проблеме, основные направления исследований в мировой науке и научные конкуренты}

\textbf{ru}\\
%+
Применение формальных грамматик для анализа биологических цепочек --- одна из активно развивающихся областей в биоинформатике.

Большое количество результатов, многие из которых уже стали классическими, и воплощёны в широко распространённом инструменте Infernal, использующем вероятностные грамматики для анализа структуры РНК-последовательностей, получены группой под руководством Шона Эдди (Sean Eddy) в США.
Данная группа является ведущей в этой области и активно занимается исследованиями в этом направлении и в настоящее время.

Применение конъюнктивных грамматик для описания структуры геномных последовательностей исследовано крайне слабо.
В настоящий момент опубликована одна работа Райана Цир-Фогеля (Ryan Zier-Vogel, "RNA pseudoknot prediction through stochastic conjunctive grammars"), в которой для предсказания вторичной структуры РНК используются вероятностные конъюнктивные грамматики.

Исследование возможностей использования формальных грамматик и алгоритмов синтаксического анализа для изучения вторичной структуры белков в настоящее время активно исследуется группой под руководством Витольда Дирки (Witold Dyrka) в Польше.

Изучение применимости формальных грамматик в качестве теоретической модели для описания вторичной структуры РНК и исследование теоретических свойств этой модели активно ведётся Микелой Квадрини (Michela Quadrini) в Италии.

При этом, наиболее исследованным является применение вероятностных грамматик и алгоритмов их построения и работы с ними.
Использование обыкновенных грамматик в сочетании с ИНС в качестве вероятностной модели исследовано крайне слабо.
\\
\\
\textbf{en}\\



\subsection{Предлагаемые методы и подходы, общий план работы на весь срок выполнения проекта и ожидаемые результаты }
%(объемом не менее 2 стр.; в том числе указываются ожидаемые конкретные результаты по годам; общий план дается с разбивкой по годам)

\textbf{ru}\\
На первом этапе планируется выявить общие принципы построения формальных грамматик для описания вторичной структуры геномных и протеомных последовательностей.
Предстоит ответить на такие вопросы, как какие типы формальных граммтик необходимо использовать, какие особенности вторичной структуры необходимо учитывать при решении прикладных задач и, следовательно, описывать.
Для этого будут привлечены методы теории формальных языков, позволяющие рассуждать о выразительных свойствах грамматик.
При этом необходимо учитывать вычислительную сложность алгоритмов синтаксического анализа и возможность реализации параллельных алгоритмов.

Далее планируется разработать параллельный алгоритм синтаксического анализа для выбранного класса грамматик.
Предполагается, что будут исследованы возможности эффективного использования массово-параллельных архитекур.
При этом, алгоритм должен быть адаптирован к использованию сильно неоднозначных грамматики, и к решению задачи поиска подстроки в строке с заданной структурой.
Необходимо также будет учесть и другие особенности грамматики, такие как сравнительно небольшой терминальный алфавит и небольшое количество правил в самой грамматике, по сравнению с грамматиками, возникающими в лингвистике или при анализе языков программирования.
Работа на данном этапе потребует привлечения методов параллельного программирования, теории построения и анализа алгоритмов.
Построение алгоритма будет вестись на основе алгоритмов синтаксического анализа, использующих матричные операции, так как такие операции хорошо поддаются распараллеливанию на современном аппаратном обеспечении.
Ключевым является алгоритм Валианта, а также его модификация, предложенная Охотиным и адаптированная для работы с конъюнктивными грамматиками.

Следующий шаг будет посвящён развитию метода совмещения синтаксического анализа и искусственных нейронных сетей для анализа вторичной структуры, предложенный руководителем проекта.
Планируется изучить различные типы и архитектуры искусственных нейронных сетей, с целью выявления наиболее подходящей для рассматриваемого применения.
Среди типов ИНС особый интерес представляют свёрточные сети.
Предполагаемый алгоритм синтаксического анализа на выходе будет строить набор булевых матриц, которые можно трактовать как слои изображения.
Это позволит обрабатывать результат синтаксического анализа как изображения, что позволит, например, упросить решение задачи нормировки данных, применив стандартные решения из области цифровой обработки изображений.
Также необходимо изучить битовые сети, так как битовый вектор --- наиболее естественное представление результатов синтаксического анализа, а данный тип сетей предназначен для обработки таких данных.
Возможно, применение такого типа ИНС позволит уменьшить объём необходимой памяти и упростить процедуру обучения.

После этого планируется провести ряд экспериментов на реальных данных --- базах цепочек, имеющихся в открытом доступе.
Цель экспериментов --- проверить практическую применимость разработанных методов и алгоритмов.
Предполагается, что будут решаться задачи классификации цепочек по различным признакам.
Например, по функциям для белковых последовательностей, или по тому, является ли цепочка химерой, для маркерных РНК-последовательностей.
На данном шаге также будет вестись подбор грамматик для конкретных задач: несмотря на то, что предполагается наличие общих принципов построения таких грамматик, решение конкретной задачи может потребовать значительных уточнений грамматики для получения наилучшего результата.

2019-2020

Разработка грамматик для анализа вторичной структуры РНК-последовательностей.

Разработка параллельного алгоритма синтаксического анализа для решения задачи поиска подстроки с заданной в виде грамматики структурой.

Проведение экспериментов по обнаружению маркерных цепочек.

2020-2021

Исследование применимости различных типов ИНС для анализа результатов синтаксического анализа.

Доработка метода совмещения синтаксического анализа и ИНС для анализа вторичной структуры биологических цепочек с учётом результатов предыдущего шага.

Разработка граммтики для анализа вторичной структуры белков с использоваием предложенного метода, проведение соответствующих экспериментов на реальных данных.
\\
\\
\textbf{en}\\


\subsection{Имеющийся у руководителя проекта научный задел по проекту, наличие опыта совместной реализации проектов}

\textbf{ru}\\
%+
Руководитель проекта обладает опытом в разработке и исследовании алгоритмов синтаксического анализа, и их применении в различных областях, что подтверждается соответствующими статьями (Grigorev, Ragozina, "Context-free path querying with structural representation of result", SECR-2017; Azimov, Grigorev, "Context-free path querying by matrix multiplication", GRADES-NDA-2018; Verbitskaia, Kirillov, Nozkin, Grigorev, "Parser combinators for context-free path querying", Scala-2018)

В том числе, у руководителя имеется опыт применения формальных грамматик и алгоритмов синтаксического анализа для решения задач в области биологии (биоинформатики), что подтверждается выступлениями на тематических конференциях Biata-2017/2018, BIOINFORMATICS-2019.

Кроме того, руководителем был предложен метод совмещения формальных грамматик и ИНС для анализа вторичной структуры, который предполагается развивать в рамках данного исследования. Метод был изложен в статье "The Composition of Dense Neural Networks and Formal Grammars for Secondary Structure Analysis" и представлен на конференции BIOINFORMATICS-2019.

Руководитель принимал успешное участие в совместной работе над проектами в рамках грантов РФФИ (15-01-05431 и 18-01-00380), Фонда содействия развитию малых форм предприятий в технической сфере
(программа УМНИК, проекты N 162ГУ1/2013 и N 5609ГУ1/2014), а также является руководителем научной группы, в соавторстве с участниками которой опубликованы указанные выше и некоторые другие работы.
\\
\\
\textbf{en}\\

\subsection{Перечень оборудования, материалов, информационных и других ресурсов, имеющихся у руководителя проекта для выполнения проекта }
\textbf{ru}\\
%+
Ресурсы, необходимые для выполнения проекта, такие как базы данных с биологическими последовательностями (базы маркерных цепочек, базы белковых последовательностей) имеются в открытом доступе в сети Интернет.
Использование иных особых ресурсов не предполагается.
\\
\\
\textbf{en}\\


\subsection{План работы на первый год выполнения проекта}

\textbf{ru}\\
%+
Сопоставление особенностей вторичной структуры маркерных цепочек с классами формальных грамматик, необходимых для выражения таких особенностей. Построение обыкновенных грамматик, в том числе конъюнктивных, описывающих характерные особенности вторичной структуры маркерных цепочек.
Формулирование основных принципов построения таких грамматик.
Проведение ряда экспериментов на реальных данных для оценки практической значимости полученных грамматик. Для этого планируется провести анализ и предварительную подготовку данных, имеющихся в открытом доступе.

Разработка параллельного алгоритма синтаксического анализа, адаптированного для решения задачи поиска подстроки с заданной вторичной структурой в строке.
Оформление результатов и их публикация.
\\
\\
\textbf{en}\\

\subsection{Ожидаемые в конце первого года конкретные научные результаты}
%(форма изложения должна дать возможность провести экспертизу результатов и оценить степень выполнения заявленного в проекте плана работы)

\textbf{ru}\\
%+
Предложены обыкновенные (не вероятностные) грамматики для описания особенностей вторичной структуры маркерных цепочек.
Проведён ряд экспериментов по использованию данных грамматик для решения прикладных задач, таких как поиск и классификация цепочек.
По итогам данного этапа, две работы будут представлены на конференциях и опубликованы в сборниках докладов, индексируемом в scopus.

Разработан параллельный алгоритм синтаксического анализа, адаптированный для решения задачи поиска подстроки с заданной вторичной структурой в строке.
Разработанный алгоритм представлен на конференции и опубликован в сборнике материалов конференции, индексируемом в scopus.
\\
\\
\textbf{en}\\

\subsection{Перечень планируемых к приобретению руководителем проекта за счет гранта Фонда оборудования, материалов, информационных и других ресурсов для выполнения проекта}
%(в том числе – описывается необходимость их использования для реализации проекта)

\textbf{ru}\\
%+
Не более !!! тыс. рублей ежегодно будет тратиться на поездки с докладами на конференции. Расходов на оборудование не предполагается.
\\
\\
\textbf{en}\\

\end{document}
