\subsection{The algorithm}

In this section, we introduce the algorithm for the computation of 
context-free reachability in a graph $\mathcal{G}$. The algorithm determines 
the existence of a path, which forms a sentence of the language defined by 
the input RSM $R$, between each pair of vertices in the graph $\mathcal{G}$.
The algorithm is based on the generalization of the FSM intersection for an RSM, 
and an input graph. Since a graph can be interpreted as a FSM, in which 
transitions correspond to the labeled edges between vertices of the graph, 
and an RSM is composed of a set of FSMs, the intersection of such machines
can be computed using the classical algorithm for FSM intersection, presented 
in~\cite{automata:theory:10.5555/1177300}.

The intersection can be computed as a Kronecker product of the corresponding 
adjacency matrices for an RSM and a graph. Since we are only determining the
reachability of vertices, it is enough to represent intersection result as 
a Boolean matrix. It simplifies the algorithm implementation and allows 
one to express it in terms of basic matrix operations.

Listing~\ref{tensor:cfpq} shows main steps of the algorithm.
The algorithm accepts context-free grammar $G=(\Sigma,N,P$) and graph $\mathcal{G}=(V,E,L)$ as an input.
An RSM $R$ is created from the grammar $G$.
Note, that $R$ must have no $\varepsilon$-transitions.
$M_1$ and $M_2$ are the adjacency matrices for the machine $R$ and the graph $\mathcal{G}$ correspondingly.

Then for each vertex $i$ of the graph $\mathcal{G}$, the algorithm adds loops 
with non-terminals, which allows deriving $\varepsilon$-word.
Here the following rule is implied: each vertex of the graph is reachable 
by itself through an $\varepsilon$-transition. Since the machine $R$ does 
not have any $\varepsilon$-transitions, the $\varepsilon$-word could be 
derived only if a state $s$ in the box $B$ of the $R$ is both initial and final.
This data is queried by the $getNonterminals()$ function for each state $s$.

The algorithm terminates when the matrix $M_2$ stops changing.
Kronecker product of matrices $M_1$ and $M_2$ is evaluated for each iteration.
The result is stored in $M_3$ as a Boolean matrix.
For the given $M_3$ a $C_3$ matrix is evaluated by the $transitiveClosure()$ 
function call. The $M_3$ could be interpreted as an adjacency matrix for an 
directed graph with no labels, used to evaluate transitive closure in terms of 
classical graph definition of this operation.
Then the algorithm iterates over cells of the $C_3$.
For the pair of indices $(i,j)$, it computes $s$ and $f$ --- 
the initial and final states in the recursive automata $R$ which relate 
to the concrete $C_3[i,j]$ of the closure matrix.
If the given $s$ and $f$ belong to the same box $B$ of $R$, $s = q_B^0$, 
and $f \in F_B$, then $getNonterminals()$ returns the respective non-terminal.
If the the condition holds then the algorithm  adds the computed non-terminals 
to the respective cell of the adjacency matrix $M_2$ of the graph.

The functions $getStates$ and $getCoordinates$ (see listing~\ref{tensor:cfpq:help})
are used to map indices between Kronecker product arguments and the result matrix.
The Implementation appeals to the blocked structure of the matrix $C_3$, 
where each block corresponds to some automata and graph edge.

The algorithm returns the updated matrix $M_2$ which contains the initial 
graph $\mathcal{G}$ data as well as non-terminals from $N$.
If a cell $M_2[i,j]$ for any valid indices $i$ and $j$ contains symbol 
$S \in N$, then vertex $j$ is reachable from vertex $i$ in grammar $G$ for 
non-terminal $S$.

\begin{algorithm}[h]
\floatname{algorithm}{Listing}
\begin{algorithmic}[1]
\tiny
\caption{Kronecker product based CFPQ}
\label{tensor:cfpq}
\Function{contextFreePathQuerying}{G, $\mathcal{G}$}
    % Input data preparation
    \State{$R \gets$ Recursive automata for $G$}
    \State{$M_1 \gets$ Adjacency matrix for $R$}
    \State{$M_2 \gets$ Adjacency matrix for $\mathcal{G}$}
    % Eps-transition handling for graph
    \For{$s \in 0..dim(M_1)-1$}
        \For{$i \in 0..dim(M_2)-1$}
            \State{$M_2[i,i] \gets M_2[i,i] \cup \textit{getNonterminals}(R,s,s)$}
        \EndFor
    \EndFor
    \While{Matrix $M_2$ is changing}
        \State{$M_3 \gets M_1 \otimes M_2$}
        \Comment{Evaluate Kroncker product}
        \State{$C_3 \gets \textit{transitiveClosure}(M_3)$}
        \State{$n \gets$ dim($M_3)$}
        \Comment{Matrix $M_3$ size = $n \times n$}
        % Add non-terminals (possibly new)
        \For{$(i,j) \in [0..n-1] \times [0..n-1]$}
            \If{$C_3[i,j]$}
                \State{$s, f \gets \textit{getStates}(C_3,i,j)$}
                \If{$\textit{getNonterminals}(R,s,f) \neq \emptyset$}
                    \State{$x, y \gets \textit{getCoordinates}(C_3,i,j)$}
                    \State{$M_2[x,y] \gets M_2[x,y] \cup \textit{getNonterminals}(R,s,f)$}
                \EndIf
            \EndIf
        \EndFor
    \EndWhile
\State \Return $M_2$
\EndFunction
\end{algorithmic}
\end{algorithm}

\begin{algorithm}[h]
\floatname{algorithm}{Listing}
\begin{algorithmic}[1]
\tiny
\caption{Help functions for Kronecker product based CFPQ}
\label{tensor:cfpq:help}
\Function{getStates}{$C, i, j$}
    \State{$r \gets dim(M_1)$}
    \Comment{$M_1$ is adjacency matrix for automata $R$}
    \State \Return{$\left\lfloor{i / r}\right\rfloor, \left\lfloor{j / r}\right\rfloor$}
\EndFunction
\Function{getCoordinates}{$C, i, j$}
    \State{$n \gets dim(M_2)$}
    \Comment{$M_2$ is adjacency matrix for graph $\mathcal{G}$}
    \State \Return{$i \bmod n, j \bmod n$}
\EndFunction
\end{algorithmic}
\end{algorithm}

\begin{lemma}
    \label{lemma:algo:correctness}
    Let $\mathcal{G} = (V,E,L)$ be a graph and $G = (\Sigma, N, P)$ be a grammar.
    Let $\mathcal{G}_k = (V,E_k,L \cup N)$ be graph and $M_k$ its adjacency
    matrix of the execution some iteration $k \geq 0$ of the algorithm \ref{?}.
    Then for each edge $e = (m,S,n) \in E_k$, where $S \in N$,
    the following statement holds: $\exists m\pi n: S \to_{G} l(\pi)$.
\end{lemma}

\begin{proof}{(Proof by induction)}

    \textbf{Basis:} For $k = 0$ and the statement of the lemma holds, since
    $M_0 = M$, $M$ where is adjacency matrix of the graph $G$. Non-terminals,
    which allow to derive $\varepsilon$-word, are also added at algorithm
    preprocessing step, since each vertex of the graph is reachable by itself 
    through an $\varepsilon$-transition.
    
    \textbf{Inductive step:} Assume that the statement of the lemma holds for any
    $k \leq (p - 1)$ and show that it also holds for $k = p$, where $p \geq 1$.
    
    For the algorithm iteration $p$ the Kronecker product $K_p$ and transitive
    closure $C_p$ are evaluated as described in the algorithm. By the properties
    of this operations, some edge $e = ((s,m),(f,n))$ exists in the directed
    graph, represented by adjacency matrix $C_p$, if and only if $\exists s
    \pi ^{'} f$ in the RSM graph, represented by matrix $M_r$, and 
    $\exists m \pi n$ in graph, represented by $M_{p-1}$. Concatenated symbols 
    along the path $\pi^{'}$ form some derivation string v, composed from terminals
    and non-terminals, where $v \to_G l(\pi)$  by the inductive assumption.
    
    The new edge $e = (m,S,n)$ will be added to the $E_p$ only if $s$ and $f$ 
    are initial and final states of some box $B$ of the RSM corresponding to 
    the non-terminal $S_B$. In this case, the grammar $G$ has the derivation rule
    $S_B \to_G v$, by the inductive assumption $v \to_G l(\pi)$. Therefore, 
    $S_B \to_G l(\pi)$ and this completes the proof of the lemma.
    
\end{proof}

\begin{lemma}
    \label{lemma:algo:completeness}
    Let $\mathcal{G} = (V,E,L)$ be a graph and  $G = (\Sigma, N, P)$ be a grammar. 
    Let $\mathcal{G}_k = (V,E_k,L \cup N)$ be graph and $M_k$ its adjacency
    matrix of the execution some iteration $k \geq 1$ of the algorithm \ref{?}. 
    For any path $m \pi n$ in graph $\mathcal{G}$ with word $l = l(\pi)$ if 
    exists the derivation tree of $l$ for the grammar $G$ and starting non-terminal
    $S$ with the height $h \leq k$, then $\exists e = (m,S,n): e \in E_k$.

\end{lemma}

\begin{proof}{(Proof by induction)}

    \textbf{Basis:} Show that statement of the lemma holds for the $k = 1$. Matrix
    $M$ and edges of the graph $\mathcal{G}$ contains only labels from $L$. 
    Since the derivation tree of height $h = 1$ contains only one non-terminal 
    $S$ as a root and only symbols from $\Sigma \cup {\varepsilon}$ as leafs, 
    for all paths, which form a word with derivation tree of the height $h = 1$, 
    the corresponding nonterminals will be added to the $M_1$ via preprocessing step
    and first iteration of the algorithm. Thus, the lemma statement holds 
    for the $k = 1$.

    \textbf{Inductive step:} Assume that the statement of the lemma hold for any
    $k \leq (p - 1)$ and show that it also holds for $k = p$, where $p \geq 2$.
    
    For the algorithm iteration $p$ the Kronecker product $K_p$ and transitive
    closure $C_p$ are evaluated as described in the algorithm. By the properties
    of this operations, some edge $e = ((s,m),(f,n))$ exists in the directed
    graph, represented by adjacency matrix $C_p$, if and only if $\exists s
    \pi_1 f$ in the RSM graph, represented by matrix $M_{RSM}$, and 
    $\exists m \pi n$ in graph, represented by $M_{p-1}$. 
    
    For any path $m \pi n$, such that exist derivation tree of height $h < k$ 
    for the word $l(\pi)$ with root non-terminal $S$, there exists edge 
    $e = (m,S,n): e \in E_k$ by inductive assumption.
    
    Suppose, that exists derivation tree $T$ of height $h = p$ with the root 
    non-terminal $S$ for the path $m \pi n$. The tree $T$ is formed as
    $S \to a_1 .. a_d, d \geq 1$ where $\forall i \in [1..d]$ $a_i$ is sub-tree of
    height $h_i \leq p - 1$ for the sub-path $m_i \pi_i n_i$. 
    By inductive hypothesis, there exists path $\pi_i$ for each derivation sub-tree, 
    such that $m = m_1 \pi_1 m_2 .. m_{d} \pi_{d} m_{d+1} = n$ and concatenation 
    of these paths forms $m \pi n$, and the root non-terminals of 
    this sub-trees are included in the matrix $M_{p - 1}$. 
    
    Therefore, vertices $m_i ~\forall i \in [1..d]$ form path in the graph, 
    represented by matrix $M_{p-1}$, with complete set of labels.
    Thus, new edge between vertices $m$ and $n$ with the respective 
    non-terminal $S$ will be added to the matrix $M_p$ and this completes 
    the proof of the lemma.

\end{proof}

\begin{theorem}
    Let $\mathcal{G} = (V,E,L)$ be a graph and  $G = (\Sigma, N, P)$ be a grammar.
    Let $\mathcal{G}_R = (V, E_R, L)$  be a result graph for the execution 
    of the algorithm \ref{??}. The following statement holds: 
    $e = (m, S, n) \in E_R$, where $S \in N$, if and only if 
    $\exists m \pi n: S \to_G l(\pi)$. 
\end{theorem}{}
    
\begin{proof}
    
    This theorem is a consequence of the  
    Lemma~\ref{lemma:algo:correctness} and 
    Lemma~\ref{lemma:algo:completeness}.
    
\end{proof}{}

\begin{theorem}{}
    Let $\mathcal{G} = (V,E,L)$ be a graph and $G = (\Sigma, N, P)$ be a grammar.
    The algorithm \ref{?} terminates in finite number of steps.
\end{theorem}

\begin{proof}
    
    The main algorithm \textit{while-loop} is executed while graph adjacency 
    matrix $M$ is changing. Since the algorithm only adds the edges with 
    non-terminals from $N$, the maximum required number of iterations 
    is $|N| \times |V| \times |V|$, where each component has finite size. 
    This completes the proof of the theorem.
    
\end{proof}{}

% TODO: more accurate upper bound for the algorithm complexity 