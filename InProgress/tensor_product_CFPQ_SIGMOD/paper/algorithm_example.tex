\subsection{An example}
\label{example:section}
In this section we introduce detailed example to demonstrate steps of the 
proposed algorithms. Our example is based on the classical worst case scenario 
introduced by Jelle Hellings in~\cite{hellingsPathQuerying}. Namely, let we have 
a graph $\mathcal{G}$ presented in Figure~\ref{fig:example_input_graph} and the 
RSM $R$ presented in Figure~\ref{example:automata}.

First step we represent graph as a set of Boolean matrices as shown in 
Figure~\ref{fig:boolean_decomposition_of_graph}, and RSM as a set of Boolean 
matrices as presented in Figure~\ref{fig:boolean_decomposition_of_fsm}.
Notice that we should formally add new empty matrix $M_2^{S}$ to $\mathcal{M}_2$, 
where edges labeled by $S$ will be added in time of the computation. 

After the initialization, the algorithm handles $\varepsilon$-case.
The input RSM does not have $\varepsilon$-transitions and does not have states
that are both start and final, therefore, no edges added at this stage. 
After that we should iteratively compute $\mathcal{M}_2$ and $C_3$.
The loop iteration number of matrices evaluation is provided as 
the subscript in parentheses.

\textbf{First iteration.} Firstly, we compute Kronecker product of the
$\mathcal{M}_1$ and $\mathcal{M}_{2,(0)}$ matrices and store result in the
$\mathcal{M}_{3,(1)}$, and collapse this matrix to the single Boolean matrix
$M'_{3,(1)}$. For the sake of simplicity, we provide only
$M'_{3,(1)}$, which is evaluated as follows in the equivalent way.

{\tiny
    \renewcommand{\arraystretch}{0.5}
    \setlength\arraycolsep{0.1pt}
\begin{align*}
& M'_{3,(1)} = M_1^a \otimes M_{2,(0)}^a +  M_1^b \otimes M_{2,(0)}^b + M_1^S \otimes M_{2,(0)}^S = \\
& \kbordermatrix{
          & (0,0) & (0,1) & (0,2) & (0,3) & \vrule & (1,0) & (1,1) & (1,2) & (1,3) & \vrule &  (2,0) & (2,1) & (2,2) & (2,3) & \vrule &  (3,0) & (3,1) & (3,2) & (3,3) &\\ 
    (0,0) & . & . & . & . & \vrule & . & 1 & . & . & \vrule & . & . & . & . &  \vrule & . & . & . & . \\
    (0,1) & . & . & . & . & \vrule & . & . & 1 & . & \vrule & . & . & . & . &  \vrule & . & . & . & . \\
    (0,2) & . & . & . & . & \vrule & 1 & . & . & . & \vrule & . & . & . & . &  \vrule & . & . & . & . \\
    (0,3) & . & . & . & . & \vrule & . & . & . & . & \vrule & . & . & . & . &  \vrule & . & . & . & . \\
    \hline
    (1,0) & . & . & . & .  & \vrule & . & . & . & . & \vrule & . & . & . & . & \vrule & . & . & . & . \\
    (1,1) & . & . & . & .  & \vrule & . & . & . & . & \vrule & . & . & . & . & \vrule & . & . & . & . \\
    (1,2) & . & . & . & .  & \vrule & . & . & . & . & \vrule & . & . & . & . & \vrule & . & . & . & 1 \\
    (1,3) & . & . & . & .  & \vrule & . & . & . & . & \vrule & . & . & . & . & \vrule & . & . & 1 & . \\
    \hline
    (2,0) & . & . & . & .  & \vrule & . & . & . & . & \vrule & . & . & . & . & \vrule & . & . & . & . \\
    (2,1) & . & . & . & .  & \vrule & . & . & . & . & \vrule & . & . & . & . & \vrule & . & . & . & . \\
    (2,2) & . & . & . & .  & \vrule & . & . & . & . & \vrule & . & . & . & . & \vrule & . & . & . & 1 \\
    (2,3) & . & . & . & .  & \vrule & . & . & . & . & \vrule & . & . & . & . & \vrule & . & . & 1 & . \\
    \hline
    (3,0) & . & . & . & .  & \vrule & . & . & . & . & \vrule & . & . & . & . & \vrule & . & . & . & . \\
    (3,1) & . & . & . & .  & \vrule & . & . & . & . & \vrule & . & . & . & . & \vrule & . & . & . & . \\
    (3,2) & . & . & . & .  & \vrule & . & . & . & . & \vrule & . & . & . & . & \vrule & . & . & . & . \\
    (3,3) & . & . & . & .  & \vrule & . & . & . & . & \vrule & . & . & . & . & \vrule & . & . & . & . \\
}
\end{align*}
}

As far as the input graph has no edges with label $S$, therefore, the correspondent block of the Kronecker product will be empty. Then, the transitive closure evaluation result, stored in the matrix $C_{3,(1)}$, introduces one new path of length 2 (respective cell is filled with a grey colour). 

{\tiny
    \renewcommand{\arraystretch}{0.5}
    \setlength\arraycolsep{0.1pt}
\begin{align*}
& C_{3,(1)} = transitiveClosure(M'_{3,(1)}) = 
\\
& \kbordermatrix{
          & (0,0) & (0,1) & (0,2) & (0,3) & \vrule & (1,0) & (1,1) & (1,2) & (1,3) & \vrule &  (2,0) & (2,1) & (2,2) & (2,3) & \vrule &  (3,0) & (3,1) & (3,2) & (3,3) &\\ 
    (0,0) & . & . & . & . & \vrule & . & 1 & . & . & \vrule & . & . & . & . &  \vrule & . & . & . & . \\
    (0,1) & . & . & . & . & \vrule & . & . & 1 & . & \vrule & . & . & . & . &  \vrule & . & . & . & \mc \\
    (0,2) & . & . & . & . & \vrule & 1 & . & . & . & \vrule & . & . & . & . &  \vrule & . & . & . & . \\
    (0,3) & . & . & . & . & \vrule & . & . & . & . & \vrule & . & . & . & . &  \vrule & . & . & . & . \\
    \hline
    (1,0) & . & . & . & .  & \vrule & . & . & . & . & \vrule & . & . & . & . & \vrule & . & . & . & . \\
    (1,1) & . & . & . & .  & \vrule & . & . & . & . & \vrule & . & . & . & . & \vrule & . & . & . & . \\
    (1,2) & . & . & . & .  & \vrule & . & . & . & . & \vrule & . & . & . & . & \vrule & . & . & . & 1 \\
    (1,3) & . & . & . & .  & \vrule & . & . & . & . & \vrule & . & . & . & . & \vrule & . & . & 1 & . \\
    \hline
    (2,0) & . & . & . & .  & \vrule & . & . & . & . & \vrule & . & . & . & . & \vrule & . & . & . & . \\
    (2,1) & . & . & . & .  & \vrule & . & . & . & . & \vrule & . & . & . & . & \vrule & . & . & . & . \\
    (2,2) & . & . & . & .  & \vrule & . & . & . & . & \vrule & . & . & . & . & \vrule & . & . & . & 1 \\
    (2,3) & . & . & . & .  & \vrule & . & . & . & . & \vrule & . & . & . & . & \vrule & . & . & 1 & . \\
    \hline
    (3,0) & . & . & . & .  & \vrule & . & . & . & . & \vrule & . & . & . & . & \vrule & . & . & . & . \\
    (3,1) & . & . & . & .  & \vrule & . & . & . & . & \vrule & . & . & . & . & \vrule & . & . & . & . \\
    (3,2) & . & . & . & .  & \vrule & . & . & . & . & \vrule & . & . & . & . & \vrule & . & . & . & . \\
    (3,3) & . & . & . & .  & \vrule & . & . & . & . & \vrule & . & . & . & . & \vrule & . & . & . & . \\
}
\end{align*}
}

This path starts in the vertex $(0,1)$ and finishes in the vertex $(3,3)$. 
We can see, that 0 and 3 are a start and a final states of the some component 
state machine for label $S$ in $R$ respectively. Thus we can conclude that 
there exists a path between vertices 1 and 3 in the graph, such that 
respective word is derivable from $S$ in the $R$ execution flow.

As a result, we can add the edge $(1,S,3)$ to the result graph, what is 
formally done by the update of the matrix $M_2^S$.

\textbf{Second iteration.} Modified graph Boolean adjacency matrices contain 
now edge with label $S$. Therefore, this label contributes to the non-empty 
corresponding matrix block in the evaluated matrix $M'_{3,{2}}$. The transitive closure
evaluation introduces three new paths. Since only path between vertices $(0,0)$ and
$(3,2)$ connects start and final states in the automata, the edge $(0,S,2)$ is added
to the result graph.

{\tiny
    \renewcommand{\arraystretch}{0.5}
    \setlength\arraycolsep{0.1pt}
\begin{align*}
& M'_{3,(2)} = M_1^a \otimes M_{2,(2)}^a +  M_1^b \otimes M_{2,(2)}^b + M_1^S \otimes M_{2,(2)}^S = \\
& \kbordermatrix{
          & (0,0) & (0,1) & (0,2) & (0,3) & \vrule & (1,0) & (1,1) & (1,2) & (1,3) & \vrule &  (2,0) & (2,1) & (2,2) & (2,3) & \vrule &  (3,0) & (3,1) & (3,2) & (3,3) &\\ 
    (0,0) & . & . & . & . & \vrule & . & 1 & . & . & \vrule & . & . & . & . &  \vrule & . & . & . & . \\
    (0,1) & . & . & . & . & \vrule & . & . & 1 & . & \vrule & . & . & . & . &  \vrule & . & . & . & . \\
    (0,2) & . & . & . & . & \vrule & 1 & . & . & . & \vrule & . & . & . & . &  \vrule & . & . & . & . \\
    (0,3) & . & . & . & . & \vrule & . & . & . & . & \vrule & . & . & . & . &  \vrule & . & . & . & . \\
    \hline
    (1,0) & . & . & . & .  & \vrule & . & . & . & . & \vrule & . & . & . & . & \vrule & . & . & . & . \\
    (1,1) & . & . & . & .  & \vrule & . & . & . & . & \vrule & . & . & . &\mc& \vrule & . & . & . & . \\
    (1,2) & . & . & . & .  & \vrule & . & . & . & . & \vrule & . & . & . & . & \vrule & . & . & . & 1 \\
    (1,3) & . & . & . & .  & \vrule & . & . & . & . & \vrule & . & . & . & . & \vrule & . & . & 1 & . \\
    \hline
    (2,0) & . & . & . & .  & \vrule & . & . & . & . & \vrule & . & . & . & . & \vrule & . & . & . & . \\
    (2,1) & . & . & . & .  & \vrule & . & . & . & . & \vrule & . & . & . & . & \vrule & . & . & . & . \\
    (2,2) & . & . & . & .  & \vrule & . & . & . & . & \vrule & . & . & . & . & \vrule & . & . & . & 1 \\
    (2,3) & . & . & . & .  & \vrule & . & . & . & . & \vrule & . & . & . & . & \vrule & . & . & 1 & . \\
    \hline
    (3,0) & . & . & . & .  & \vrule & . & . & . & . & \vrule & . & . & . & . & \vrule & . & . & . & . \\
    (3,1) & . & . & . & .  & \vrule & . & . & . & . & \vrule & . & . & . & . & \vrule & . & . & . & . \\
    (3,2) & . & . & . & .  & \vrule & . & . & . & . & \vrule & . & . & . & . & \vrule & . & . & . & . \\
    (3,3) & . & . & . & .  & \vrule & . & . & . & . & \vrule & . & . & . & . & \vrule & . & . & . & . \\
} \\
& C_{3,(2)} = transitiveClosure(M'_{3,(2)}) = 
\\
& \kbordermatrix{
          & (0,0) & (0,1) & (0,2) & (0,3) & \vrule & (1,0) & (1,1) & (1,2) & (1,3) & \vrule &  (2,0) & (2,1) & (2,2) & (2,3) & \vrule &  (3,0) & (3,1) & (3,2) & (3,3) &\\ 
    (0,0) & . & . & . & . & \vrule & . & 1 & . & . & \vrule & . & . & . &\mc&  \vrule & . & . &\mc& . \\
    (0,1) & . & . & . & . & \vrule & . & . & 1 & . & \vrule & . & . & . & . &  \vrule & . & . & . & 1 \\
    (0,2) & . & . & . & . & \vrule & 1 & . & . & . & \vrule & . & . & . & . &  \vrule & . & . & . & . \\
    (0,3) & . & . & . & . & \vrule & . & . & . & . & \vrule & . & . & . & . &  \vrule & . & . & . & . \\
    \hline
    (1,0) & . & . & . & .  & \vrule & . & . & . & . & \vrule & . & . & . & . & \vrule & . & . & . & . \\
    (1,1) & . & . & . & .  & \vrule & . & . & . & . & \vrule & . & . & . & 1 & \vrule & . & . &\mc& . \\
    (1,2) & . & . & . & .  & \vrule & . & . & . & . & \vrule & . & . & . & . & \vrule & . & . & . & 1 \\
    (1,3) & . & . & . & .  & \vrule & . & . & . & . & \vrule & . & . & . & . & \vrule & . & . & 1 & . \\
    \hline
    (2,0) & . & . & . & .  & \vrule & . & . & . & . & \vrule & . & . & . & . & \vrule & . & . & . & . \\
    (2,1) & . & . & . & .  & \vrule & . & . & . & . & \vrule & . & . & . & . & \vrule & . & . & . & . \\
    (2,2) & . & . & . & .  & \vrule & . & . & . & . & \vrule & . & . & . & . & \vrule & . & . & . & 1 \\
    (2,3) & . & . & . & .  & \vrule & . & . & . & . & \vrule & . & . & . & . & \vrule & . & . & 1 & . \\
    \hline
    (3,0) & . & . & . & .  & \vrule & . & . & . & . & \vrule & . & . & . & . & \vrule & . & . & . & . \\
    (3,1) & . & . & . & .  & \vrule & . & . & . & . & \vrule & . & . & . & . & \vrule & . & . & . & . \\
    (3,2) & . & . & . & .  & \vrule & . & . & . & . & \vrule & . & . & . & . & \vrule & . & . & . & . \\
    (3,3) & . & . & . & .  & \vrule & . & . & . & . & \vrule & . & . & . & . & \vrule & . & . & . & . \\
}
\end{align*}
}

\begin{comment}
{\tiny
    \renewcommand{\arraystretch}{0.5}
    \setlength\arraycolsep{0.1pt}
\begin{align*}
& C_3^3 = 
\\
& \kbordermatrix{
          & (0,0) & (0,1) & (0,2) & (0,3) & \vrule & (1,0) & (1,1) & (1,2) & (1,3) & \vrule &  (2,0) & (2,1) & (2,2) & (2,3) & \vrule &  (3,0) & (3,1) & (3,2) & (3,3) &\\ 
    (0,0) & . & . & . & . & \vrule & . & 1 & . & . & \vrule & . & . & . & 1 &  \vrule & . & . & 1 & . \\
    (0,1) & . & . & . & . & \vrule & . & . & 1 & . & \vrule & . & . & . & . &  \vrule & . & . & . & 1 \\
    (0,2) & . & . & . & . & \vrule & 1 & . & . & . & \vrule & . & . &\mc& . &  \vrule & . & . & . &\mc\\
    (0,3) & . & . & . & . & \vrule & . & . & . & . & \vrule & . & . & . & . &  \vrule & . & . & . & . \\
    \hline
    (1,0) & . & . & . & .  & \vrule & . & . & . & . & \vrule & . & . & 1 & . & \vrule & . & . & . &\mc\\
    (1,1) & . & . & . & .  & \vrule & . & . & . & . & \vrule & . & . & . & 1 & \vrule & . & . & 1 & . \\
    (1,2) & . & . & . & .  & \vrule & . & . & . & . & \vrule & . & . & . & . & \vrule & . & . & . & 1 \\
    (1,3) & . & . & . & .  & \vrule & . & . & . & . & \vrule & . & . & . & . & \vrule & . & . & 1 & . \\
    \hline
    (2,0) & . & . & . & .  & \vrule & . & . & . & . & \vrule & . & . & . & . & \vrule & . & . & . & . \\
    (2,1) & . & . & . & .  & \vrule & . & . & . & . & \vrule & . & . & . & . & \vrule & . & . & . & . \\
    (2,2) & . & . & . & .  & \vrule & . & . & . & . & \vrule & . & . & . & . & \vrule & . & . & . & 1 \\
    (2,3) & . & . & . & .  & \vrule & . & . & . & . & \vrule & . & . & . & . & \vrule & . & . & 1 & . \\
    \hline
    (3,0) & . & . & . & .  & \vrule & . & . & . & . & \vrule & . & . & . & . & \vrule & . & . & . & . \\
    (3,1) & . & . & . & .  & \vrule & . & . & . & . & \vrule & . & . & . & . & \vrule & . & . & . & . \\
    (3,2) & . & . & . & .  & \vrule & . & . & . & . & \vrule & . & . & . & . & \vrule & . & . & . & . \\
    (3,3) & . & . & . & .  & \vrule & . & . & . & . & \vrule & . & . & . & . & \vrule & . & . & . & . \\
}
\end{align*}
}

{\tiny
    \renewcommand{\arraystretch}{0.5}
    \setlength\arraycolsep{0.1pt}
\begin{align*}
& C_3^4 = 
\\
& \kbordermatrix{
          & (0,0) & (0,1) & (0,2) & (0,3) & \vrule & (1,0) & (1,1) & (1,2) & (1,3) & \vrule &  (2,0) & (2,1) & (2,2) & (2,3) & \vrule &  (3,0) & (3,1) & (3,2) & (3,3) &\\ 
    (0,0) & . & . & . & . & \vrule & . & 1 & . & . & \vrule & . & . & . & 1 &  \vrule & . & . & 1 & . \\
    (0,1) & . & . & . & . & \vrule & . & . & 1 & . & \vrule & . & . & . &\mc&  \vrule & . & . &\mc& 1 \\
    (0,2) & . & . & . & . & \vrule & 1 & . & . & . & \vrule & . & . & 1 & . &  \vrule & . & . & . & 1\\
    (0,3) & . & . & . & . & \vrule & . & . & . & . & \vrule & . & . & . & . &  \vrule & . & . & . & . \\
    \hline
    (1,0) & . & . & . & .  & \vrule & . & . & . & . & \vrule & . & . & 1 & . & \vrule & . & . & . & 1\\
    (1,1) & . & . & . & .  & \vrule & . & . & . & . & \vrule & . & . & . & 1 & \vrule & . & . & 1 & . \\
    (1,2) & . & . & . & .  & \vrule & . & . & . & . & \vrule & . & . & . & 1 & \vrule & . & . &\mc& 1 \\
    (1,3) & . & . & . & .  & \vrule & . & . & . & . & \vrule & . & . & . & . & \vrule & . & . & 1 & . \\
    \hline
    (2,0) & . & . & . & .  & \vrule & . & . & . & . & \vrule & . & . & . & . & \vrule & . & . & . & . \\
    (2,1) & . & . & . & .  & \vrule & . & . & . & . & \vrule & . & . & . & . & \vrule & . & . & . & . \\
    (2,2) & . & . & . & .  & \vrule & . & . & . & . & \vrule & . & . & . & . & \vrule & . & . & . & 1 \\
    (2,3) & . & . & . & .  & \vrule & . & . & . & . & \vrule & . & . & . & . & \vrule & . & . & 1 & . \\
    \hline
    (3,0) & . & . & . & .  & \vrule & . & . & . & . & \vrule & . & . & . & . & \vrule & . & . & . & . \\
    (3,1) & . & . & . & .  & \vrule & . & . & . & . & \vrule & . & . & . & . & \vrule & . & . & . & . \\
    (3,2) & . & . & . & .  & \vrule & . & . & . & . & \vrule & . & . & . & . & \vrule & . & . & . & . \\
    (3,3) & . & . & . & .  & \vrule & . & . & . & . & \vrule & . & . & . & . & \vrule & . & . & . & . \\
}
\end{align*}
}

{\tiny
    \renewcommand{\arraystretch}{0.5}
    \setlength\arraycolsep{0.1pt}
\begin{align*}
& C_3^5 = 
\\
& \kbordermatrix{
          & (0,0) & (0,1) & (0,2) & (0,3) & \vrule & (1,0) & (1,1) & (1,2) & (1,3) & \vrule &  (2,0) & (2,1) & (2,2) & (2,3) & \vrule &  (3,0) & (3,1) & (3,2) & (3,3) &\\ 
    (0,0) & . & . & . & . & \vrule & . & 1 & . & . & \vrule & . & . &\mc& 1 &  \vrule & . & . & 1 &\mc\\
    (0,1) & . & . & . & . & \vrule & . & . & 1 & . & \vrule & . & . & . & 1 &  \vrule & . & . & 1 & 1 \\
    (0,2) & . & . & . & . & \vrule & 1 & . & . & . & \vrule & . & . & 1 & . &  \vrule & . & . & . & 1 \\
    (0,3) & . & . & . & . & \vrule & . & . & . & . & \vrule & . & . & . & . &  \vrule & . & . & . & . \\
    \hline
    (1,0) & . & . & . & .  & \vrule & . & . & . & . & \vrule & . & . & 1 & . & \vrule & . & . & . & 1\\
    (1,1) & . & . & . & .  & \vrule & . & . & . & . & \vrule & . & . & 1 & 1 & \vrule & . & . & 1 &\mc\\
    (1,2) & . & . & . & .  & \vrule & . & . & . & . & \vrule & . & . & . & 1 & \vrule & . & . & 1 & 1 \\
    (1,3) & . & . & . & .  & \vrule & . & . & . & . & \vrule & . & . & . & . & \vrule & . & . & 1 & . \\
    \hline
    (2,0) & . & . & . & .  & \vrule & . & . & . & . & \vrule & . & . & . & . & \vrule & . & . & . & . \\
    (2,1) & . & . & . & .  & \vrule & . & . & . & . & \vrule & . & . & . & . & \vrule & . & . & . & . \\
    (2,2) & . & . & . & .  & \vrule & . & . & . & . & \vrule & . & . & . & . & \vrule & . & . & . & 1 \\
    (2,3) & . & . & . & .  & \vrule & . & . & . & . & \vrule & . & . & . & . & \vrule & . & . & 1 & . \\
    \hline
    (3,0) & . & . & . & .  & \vrule & . & . & . & . & \vrule & . & . & . & . & \vrule & . & . & . & . \\
    (3,1) & . & . & . & .  & \vrule & . & . & . & . & \vrule & . & . & . & . & \vrule & . & . & . & . \\
    (3,2) & . & . & . & .  & \vrule & . & . & . & . & \vrule & . & . & . & . & \vrule & . & . & . & . \\
    (3,3) & . & . & . & .  & \vrule & . & . & . & . & \vrule & . & . & . & . & \vrule & . & . & . & . \\
}
\end{align*}
}
\end{comment}

The result transitive closure matrix $C_{3,(6)}$ of the remaining iterations evaluated 
as follows. The result graph is presented in Figure~\ref{fig:example_result}.

{\tiny
    \renewcommand{\arraystretch}{0.5}
    \setlength\arraycolsep{0.1pt}
\begin{align*}
& C_{3,(6)} = 
\\
& \kbordermatrix{
          & (0,0) & (0,1) & (0,2) & (0,3) & \vrule & (1,0) & (1,1) & (1,2) & (1,3) & \vrule &  (2,0) & (2,1) & (2,2) & (2,3) & \vrule &  (3,0) & (3,1) & (3,2) & (3,3) &\\ 
    (0,0) & . & . & . & . & \vrule & . & 1 & . & . & \vrule & . & . &\mc& 1 &  \vrule & . & . & 1 &\mc\\
    (0,1) & . & . & . & . & \vrule & . & . & 1 & . & \vrule & . & . & . &\mc&  \vrule & . & . &\mc& 1 \\
    (0,2) & . & . & . & . & \vrule & 1 & . & . & . & \vrule & . & . &\mc&\mc&  \vrule & . & . &\mc&\mc\\
    (0,3) & . & . & . & . & \vrule & . & . & . & . & \vrule & . & . & . & . &  \vrule & . & . & . & . \\
    \hline
    (1,0) & . & . & . & .  & \vrule & . & . & . & . & \vrule & . & . &\mc&\mc& \vrule & . & . &\mc&\mc\\
    (1,1) & . & . & . & .  & \vrule & . & . & . & . & \vrule & . & . &\mc& 1 & \vrule & . & . & 1 &\mc\\
    (1,2) & . & . & . & .  & \vrule & . & . & . & . & \vrule & . & . & . &\mc& \vrule & . & . &\mc& 1 \\
    (1,3) & . & . & . & .  & \vrule & . & . & . & . & \vrule & . & . & . & . & \vrule & . & . & 1 & . \\
    \hline
    (2,0) & . & . & . & .  & \vrule & . & . & . & . & \vrule & . & . & . & . & \vrule & . & . & . & . \\
    (2,1) & . & . & . & .  & \vrule & . & . & . & . & \vrule & . & . & . & . & \vrule & . & . & . & . \\
    (2,2) & . & . & . & .  & \vrule & . & . & . & . & \vrule & . & . & . & . & \vrule & . & . & . & 1 \\
    (2,3) & . & . & . & .  & \vrule & . & . & . & . & \vrule & . & . & . & . & \vrule & . & . & 1 & . \\
    \hline
    (3,0) & . & . & . & .  & \vrule & . & . & . & . & \vrule & . & . & . & . & \vrule & . & . & . & . \\
    (3,1) & . & . & . & .  & \vrule & . & . & . & . & \vrule & . & . & . & . & \vrule & . & . & . & . \\
    (3,2) & . & . & . & .  & \vrule & . & . & . & . & \vrule & . & . & . & . & \vrule & . & . & . & . \\
    (3,3) & . & . & . & .  & \vrule & . & . & . & . & \vrule & . & . & . & . & \vrule & . & . & . & . \\
}
\end{align*}
}

\begin{figure}[h]
    \centering         
    \begin{tikzpicture}[shorten >=1pt,auto]
    \node[state] (q_0)                      {$0$};
    \node[state] (q_1) [above right=of q_0] {$1$};
    \node[state] (q_2) [right=of q_0]       {$2$};
    \node[state] (q_3) [right=of q_2]       {$3$};
      \path[->]
        (q_0) edge node {a} (q_1)
        (q_1) edge[bend left] node {a} (q_2)
        (q_1) edge[bend right, color=red] node  {\textbf{S}} (q_2)
        (q_2) edge[bend right, above]  node {a} (q_0)
        (q_2) edge[loop right, color=red]  node {\textbf{S}} (q_2)
        (q_1) edge[bend left, above, color=red]  node {\textbf{S}} (q_3)
        (q_0) edge[bend right, above, color=red]  node {\textbf{S}} (q_2)
        (q_2) edge[bend left=20, above]  node {b} (q_3)
        (q_2) edge[bend left=55, above, color=red]  node {\textbf{S}} (q_3)
        (q_0) edge[bend right =  50, below, color=red]  node {\textbf{S}} (q_3)
        (q_3) edge[bend left, below]  node {b} (q_2);
    \end{tikzpicture}
    \caption{The result graph $\mathcal{G}$}
    \label{fig:example_result}    
\end{figure}


At this point the index creation is finished. 
One can use it to answer reachability queries, but for some problems it can be used 
to restore paths for some reachable vertices. The result transitive closure matrix
$C_3$ or so called \textit{index} could be used for that. For example, let we try to
restore paths from 2 to 2 derived from $S$ in the result graph.

To get these paths we should call \verb|getPaths(2, 2, s)| function.
Partial trace of this call is presented below in Figure~\ref{trc:exmaple}. 
First, we must query paths for all possible start and final states of the 
machine for the provided graph vertices. Since in the example RSM the component 
state machine with label $S$ has single final state, the function 
\verb|getPathsInner| is called with arguments $(0,2)$ and $(3,2)$.
Note, that in the path extraction algorithm passed values to the functions is 
pairs of the machine state and graph vertex, which uniquely identify cell of 
the index matrix $C_3$. Possible paths concatenation vertices are stored as \verb|parts={(1,0),(2,3)}|. Then we try to get parts of paths going through 
index vertex $(1,0)$. All possible concatenations variants of the paths are 
queried in the corresponding \verb|getSubpaths| function call. As the result,
we get the set of possible paths in the graph from $2$ to $2$.

\begin{figure}
\begin{minipage}[t]{0.48\textwidth}
{
\scriptsize
\setlength{\DTbaselineskip}{8pt}
\DTsetlength{0.2em}{0.5em}{0.2em}{0.4pt}{1.6pt}
\dirtree{%
.1 getPaths($2,2,S$).
.2 getPathsInner($(0,2),(3,2)$).
.3 parts$=\{(1,0),(2,3)\}$.
.3 getSubpaths($(0,2),(3,2),(1,0)$).
.4 l=$\{2 \xrightarrow{a} 0\}$.
.5 $\cdots$.
.6 getPathsInner($(0,0),(3,2)$).
.7 parts = $\{(1,1),(2,3)\}$.
.7 getSubpaths($(0,0), (3,2), (1,1)$).
.8 $\cdots$.
.9 getPaths(1, 3, S).
.10 $\cdots$.
.11 getSubpaths($(0,1),(3,3),(1,2)$).
.12 $l=\{ 1 \xrightarrow{a} 2 \}$.
.12 $r=\{ 2 \xrightarrow{b} 3 \}$.
.12 return $\{ 1\xrightarrow{a} 2 \xrightarrow{b} 3 \}$.
.7 getSubpaths($(0,0), (3,2), (2,3)$).
.8 $\cdots$.
.9 getPaths(1, 3, S) // \begin{minipage}[t]{4cm}An alternative way to get paths from 1 to 3 which leads to infinite set of paths \end{minipage}.
.10 return $r_\infty^{1\leadsto 3}$ // \begin{minipage}[t]{4.5cm} An infinite set of path from 1 to 3 \end{minipage}.
.8 $\cdots$.
.7 return $\{0 \xrightarrow{a} 1 \xrightarrow{a} 2 \xrightarrow{b} 3 \xrightarrow{b} 2\} \cup (\{0 \xrightarrow{a} 1\} \cdot r_\infty^{1\leadsto 3} \cdot \{3 \xrightarrow{b} 2\})$ .
.2 return $\{2 \xrightarrow{a} 0 \xrightarrow{a} 1 \xrightarrow{a} 2 \xrightarrow{a} 0 \xrightarrow{a} 1 \xrightarrow{a} 2 \xrightarrow{b} 3 \xrightarrow{b} 2 \xrightarrow{b} 3 \xrightarrow{b} 2 \xrightarrow{b} 3 \xrightarrow{b} 2\} \cup (\{2 \xrightarrow{a}  0 \xrightarrow{a} 1 \xrightarrow{a} 2 \xrightarrow{a} 0 \xrightarrow{a} 1\} \cdot r_\infty^{1\leadsto 3} \cdot \{3 \xrightarrow{b} 2 \xrightarrow{b} 3 \xrightarrow{b} 2 \xrightarrow{b} 3 \xrightarrow{b} 2\})$.
}
}
\caption{Example of call stack trace}
\label{trc:exmaple}
\end{minipage}
\end{figure}

Lazy evaluation is required here, since the result graph may possibly have an 
infinite number of path between some vertices pair. Another approach here is 
to try to query some fixed number of paths, or just single path. Eventually, 
the paths enumeration problem is actual here: how can we enumerate paths with small delay.