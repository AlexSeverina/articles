\subsection{Paths Extraction Algoritm}

After index created one can enumerate all paths between specified vertices.
Note, that the index stores information about all reachable pairs for all nonterminals.
Thus, the most natural way to use this index is to query paths between specified vertices derivable from specified nonterminal. 

To do it we provide a function \textsc{getPaths}($v_s, v_f, N$), where $v_s$ is a start vertex of the graph, $v_f$ --- the final vertex, and $N$ is a nonterminal. 
Implementation of this function is presented in Listing~\ref{tensor:pathsExtraction}.

\begin{algorithm}[h]
\floatname{algorithm}{Listing}
\begin{algorithmic}[1]
\footnotesize
\caption{Paths extraction algorithm}
\label{tensor:pathsExtraction}
\State{$C_3 \gets $ result of index creation algorithm: final transitive closure}
\State{$\mathcal{M}_1 \gets $  the set of adjacency matrices of the input RSM}
\State{$\mathcal{M}_2 \gets $ the set of adjacency matrices of the final graph}

\Function{getPaths}{$v_s, v_f, N$}
    \State{$q_N^0 \gets$ Start state of automata for $N$}
    \State{$F_N \gets$ Final states of automata for $N$}
    \State{$res \gets \bigcup\limits_{f \in F_N} \Call{getPathsInner}{(q_N,v_s),(f,v_f)}$}
\State \Return $res$
\EndFunction

% DONE: fixed vertices notation (s,v)
% note: the first index in the pair is the state of the RSM
% note: the second index in the pair is the vertex of the graph

\Function{getSubpaths}{$(s_i,v_i), (s_j,v_j), (s_k,v_k)$}
    %\State{$v_i \gets \Call{getGraphV}{i}$}
    %\State{$v_k \gets \Call{getGraphV}{k}$}
    %\State{$v_j \gets \Call{getGraphV}{j}$}

    %\State{$s_i \gets \Call{getRsmState}{i}$}
    %\State{$s_k \gets \Call{getRsmState}{k}$}
    %\State{$s_j \gets \Call{getRsmState}{j}$}

  \State{\begin{minipage}[t]{0.2\textwidth}
           \vspace{-13pt}
           \begin{align*}
              l \gets & \{(v_i,t,v_k) \mid M_2^t[s_i, s_k] \wedge M_1^t[v_i, v_k]\}\\
                & \cup \ \bigcup_{\{N \mid M_2^N[s_i, s_k]\}}\Call{getPaths}{v_i, v_k, N} \\
                & \cup \ \Call{getPathsInner}{(s_i,v_i), (s_k,v_k)}  
           \end{align*}
           \end{minipage}
          }
    \State{\begin{minipage}[t]{0.2\textwidth}
           \vspace{-13pt}
           \begin{align*}
              r \gets & \{(v_k,t,v_j) \mid M_2^t[s_k, s_j] \wedge M_1^t[v_k, v_j] \}\\
                      & \cup \ \bigcup_{\{N \mid M_2^N[s_k, s_j] \}}\Call{getPaths}{v_k, v_j, N} \\
                      & \cup \ \Call{getPathsInner}{(s_k,v_k), (s_j,v_j)}  
           \end{align*}
           \end{minipage}
          }
    \State \Return $l \cdot r$
\EndFunction

\Function{getPathsInner}{$(s_i,v_i), (s_j,v_j)$}
    \State{$parts \gets \{ (s_k,v_k) \mid C_3[(s_i,v_i),(s_k,v_k)] = 1 \wedge C_3[(s_k,v_k),(s_j,v_j)] = 1\}$}
    \State \Return $\bigcup_{(s_k,v_k) \in parts} \Call{getSubpaths}{(s_i,v_i), (s_j,v_j),(s_k,v_k)}$
\EndFunction
\end{algorithmic}
\end{algorithm}

Paths extraction is implemented as three mutually recursive functions.
The entry point is \textsc{getPaths}($v_s, v_f, N$). 
This function returns a set of paths between $v_s$ and $v_f$ such that the word formed by the path is derivable from nonterminal $N$.

To compute such paths it is necessary to compute paths from vertices of the form $(q_N^s,v_s)$ to vertices of the form $(q_N^f, v_f)$ in the result of transitive closure, where $q_N^s$ is an initial state of RSM for $N$ and $q_N^f$ is a final state.
To do it \textsc{getPathsInner}$((s_i,v_i),(s_j,v_j))$ is used. 
This function finds all possible vertices $(s_k,v_k)$  which split path from $(s_i,v_i)$ to $(s_j,v_j)$ into two subpaths.
After that, function \textsc{getSubpaths}$((s_i,v_i),(s_j,v_j),(s_k,v_k))$ is used to compute corresponding subpaths.
Each part of the path may be a single edge, or path with length more than one. 
In the second case \textsc{getPathsInner} is used to restore corresponding paths.
In the first case, the edge can be labeled by terminal or nonterminal. 
In the first case corresponding edge should be added to the result.
In the second case,  \textsc{getPaths} should be used to restore paths.

Note, that, first of all, we assume that sets are computed lazily.
It is necessary to work correctly in the case of an infinite number of paths.
Second, we use a set of path as a result, so we did not check duplicated paths manually. 
