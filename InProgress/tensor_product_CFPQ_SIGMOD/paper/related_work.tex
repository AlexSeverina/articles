\section{Related Work}

Language constrained path querying widely used in graph databases, static code analysis and other areas.
Both, RPQ and CFPQ (known as CFL reachability problem in static code analysis) actively studied last years.

Huge number of theoretical research on RPQ and it's specific cases.
A number of high-performance RPQ algorithms are: derivatives~\cite{10.1145/2949689.2949711}, Glushkov~\cite{Wang2019}, etc.!!!!~\cite{!!!} distributed, vertex-based parallelism (Pregel-like systems), not linear algebra.

A bunch of CFPQ algorithms based on different ideas and with different properties was proposed in recent years.
Hellings~\cite{hellingsRelational}, Bradford~\cite{8249039}, Azimov~\cite{Azimov:2018:CPQ:3210259.3210264}, Verbitskaya~\cite{10.1007/978-3-319-41579-6_22}, Martin Musicante~\cite{10.1007/978-3-319-91662-0_17,Medeiros:2018:EEC:3167132.3167265}, form static code analysis~\cite{!!!}, 
All of them have not better than cubic time complexity in terms of input graph size. 
Worth mentioning separately Azimov's algorithm~\cite{!!!}, which is first, in our knowledge, linear-algebra based algorithm for CFPQ. 
It was shown by Arseniy Terekhov et al.\cite{!!!} that this algorithm can be applied for real-world graph analysis problems, while J Kuipers et al. shows in~\cite{!!!} that other state-of-the-art CFPQ algorithm is not performant enough to handle real-world graphs.

Subcubic CFPQ is a long-standing problem which is actively studied in both graph database and static code analysis communities.
The question on existence of subcubic CFPQ algorithm was asked by Mikhalis Yannacackkis in 1997 in~\cite{!!!}. He notes that Valiant's algorithm, the first known subcubic algorithm for context-free parsing, can be generalized to direct acyclic graph querying, but it unlikely can be applied for general CFPQ. 
At the almost same time Thomas Reps formulate a problem of subcubic bottleneck of context-free language reachability~\cite{!!!}.
Since these problems were formulated,!!!!
The most general result is a slightly subcubic algorithm based on recursive state machine reachability, which was provided by !!! in~\cite{rsm:analysis:10.1007/3-540-44585-4_18}. This algorithm uses 4 Russians !!!! trick to achieve logarithmic speedup, and thus $O(n^3/\log{n})$ time complexity. 
The first truly subcubic algorithm (with $O(n^\omega)$ time complexity) for general graph and 1-Dyck language was provided by Phillip Bradford in~\cite{Bradford2017EfficientEP}. Unfortunately, this result cannot be generalized to general context-free languages.
The same result was provided by !!!! in ~\cite{zhang2020conditional,pavlogiannis2020finegrained}
Another partial case was investigated by Chatterjee et al. in~\cite{10.1145/3158118}.
The $O(!!!)$ algorithm for an arbitrary Dyck querying of bidirected graph was described.

Utilization of linear algebra for high-performance graph analysis.
GrpahBLAS~\cite{!!!} and SuiteSparse~\cite{!!!}.
Linear algebra based approaches to evaluate queries (Datalog !!!) SPARQL~\cite{10.1145/3302424.3303962,DBLP:journals/corr/MetzlerM15a}, etc !!!. 
Not focused on types of queries.
RedisGraph~\cite{!!!} is a linear-algebra powered graph database.

