\section{Related Work}

Language constrained path querying widely used in graph databases, static code analysis and other areas.
Both, RPQ and CFPQ (known as CFL reachability problem in static code analysis) actively studied last years.

There is a huge number of theoretical research on RPQ and it's specific cases.
RPQ with single-path semantics was investigated from theoretical point of view by Barrett et al. in~\cite{barrett2000formal}.
In order to research practical limits and restrictions of RPQ, a number of high-performance RPQ algorithms were provided.
For example, derivative-based solution provided by Maurizio Nol\'{e} and Carlo Sartiani which is implemented on the top of Pregel-based system~\cite{10.1145/2949689.2949711}, or solution of Andr\'{e} Koschmieder et al.~\cite{10.1007/978-3-642-31235-9_12}.
But only limited number of practical solutions provide ability to restore paths of interest. 
One of the resent work is a research of Xin Wang et al.~\cite{Wang2019} in which Pregel-based provenance-aware RPQ algorithm, which utilizes a Glushkov's construction~\cite{Glushkov1961}, is provided.
Applicability of linear algebra based RPQ algorithms with paths-providing semantics is not investigated.

On the other hand, a bunch of CFPQ algorithms based on different ideas and with different properties was proposed in recent years. 
All of them have not better than cubic time complexity in terms of input graph size, and exploit ideas of different parsing algorithms, such as CYK in works of Jelle Hellings~\cite{hellingsRelational} and Phillip Bradford~\cite{8249039}, (G)LR and (G)LL in works of Ekaterina Verbitskaia et al.~\cite{10.1007/978-3-319-41579-6_22}, Semyon Grigorev et al.~\cite{Grigorev:2017:CPQ:3166094.3166104}, Fred Santos et al.~\cite{10.1007/978-3-319-91662-0_17}, Ciro Medeiros et al.~\cite{Medeiros:2018:EEC:3167132.3167265}.
Worth mentioning separately Azimov's algorithm~\cite{Azimov:2018:CPQ:3210259.3210264}, which is first, in our knowledge, linear-algebra based algorithm for CFPQ. 
It was shown by Arseniy Terekhov et al.~\cite{10.1145/3398682.3399163} that this algorithm can be applied for real-world graph analysis problems, while Jochem Kuijpers et al. shows in~\cite{Kuijpers:2019:ESC:3335783.3335791} that other state-of-the-art CFPQ algorithms are not performant enough to handle real-world graphs.

One of the important properties of both RPQ and CFPQ algorithms is an ability to restore paths of interest. 
Some of mentioned algorithms can solve only reachability problem, while in some cases it is important to provide at least one path satisfies query. 
While Arseniy Terekhov et al.~\cite{10.1145/3398682.3399163} provide first linear algebra based CFPQ algorithm with single path semantics, Jelle Hellings in~\cite{!!!} provides first theoretical investigation of this problem. Also he provide overview of related researches and shows that the problem is related to strings generation problem and respective results from formal language theory.
Also he conclude that both theoretical and empirical investigation of CFPQ with single-path and all-path semantics are in early stage, and we agree with this point of view, because we only demonstrates applicability of our solution on paths extraction, without detailed investigation of its properties.

Subcubic CFPQ is a long-standing problem which is actively studied in both graph database and static code analysis communities.
The question on existence of subcubic CFPQ algorithm was asked by Mihalis Yannakakis in 1990 in~\cite{Yannakakis}.
He notes that Valiant's algorithm~\cite{10.1016/S0022-0000(75)80046-8}, the first known truly subcubic algorithm for context-free parsing, can be generalized to direct acyclic graph querying, but it unlikely can be applied for general CFPQ. 
At the almost same time Thomas Reps formulate a problem of subcubic bottleneck of context-free language reachability~\cite{!!!}.
Since these problems were formulated,!!!!
The most general result is a slightly subcubic algorithm based on recursive state machine reachability, which was provided by !!! in~\cite{rsm:analysis:10.1007/3-540-44585-4_18}. This algorithm uses 4 Russians !!!! trick to achieve logarithmic speedup, and thus $O(n^3/\log{n})$ time complexity. 
The first truly subcubic algorithm with $\widetilde{O}(n^\omega)$ time complexity ($\omega$ is the best exponent for matrix multiplication, $\widetilde{O}$ is the asymptotic upper-bound mod polylog factors) for general graph and 1-Dyck language was provided by Phillip Bradford in~\cite{Bradford2017EfficientEP}. Unfortunately, this result cannot be generalized to general context-free queries.
The same result was provided by !!!! in~\cite{zhang2020conditional,pavlogiannis2020finegrained}
Another partial case was investigated by Chatterjee et al. in~\cite{10.1145/3158118}.
The $O(!!!)$ algorithm for an arbitrary Dyck querying of bidirected graph was described.
Specific types of static code analysis related to CFL-r, especially Andersen's Pointer Analysis was studied recently. Reduction to BMM.
Other partial cases such as tree querying also were studied.

Utilization of linear algebra for high-performance graph analysis.
GrpahBLAS~\cite{!!!} and SuiteSparse~\cite{!!!}.
Linear algebra based approaches to evaluate queries (Datalog !!!) SPARQL~\cite{10.1145/3302424.3303962,DBLP:journals/corr/MetzlerM15a}, etc !!!. 
Not focused on types of queries.
RedisGraph~\cite{8778293} is a linear-algebra powered graph database.