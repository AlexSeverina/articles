\section{Introduction}


Language-constrained path querying~\cite{!!!} is one of techniques for graph navigation querying. 
This technique allows one to use formal languages as constraints on paths in edge-labeled graphs: path satisfies constraints if labels along it form a word from the specified language.

The utilization of regular languages as constraints, or \textit{Regular Path Querying} (RPQ), is most well-studied and widely spread.
RPQs are used for .... 
Support of RPQs s implemented in !!!
Even that, improvement of RPQ algorithm efficiency on huge graphs is an actual problem nowdays.
For example, !!!!

At the same time, utilization of more powerful languages, namely context-free languages, gain popularity in the last few years. 
\textit{Context-Free Path Querying} problem (CFPQ) was introduced by M Yannacacis in 1987 in~\cite{!!!}.
A number of different algoritms was proposed since that time, 
Context-free is more specific, but actively developing last years.


To make it usable... Integration with graph DB.
But recently, in~\cite{!!!} J Kujpers et al show that state-of-the-art CFPQ algorithms are not performant enoigh to be used in practice.
This fact motivates to finde new algorithms for CFPQ.

Integration with query languages. 
The problem. We cannot separate regular and context-free queries in general case.  

CFPQ as a separated algorithms. 
Matrix is the fastest. 

Moreover, grammar transformation for matrix-based (the fastest existing algorithm) is required, !!!

Linear algebra, GraphBLAS, !!!! is a right way.

Recently, an algortithm was proposed. In this work we improve it, blah-blah-blah

Subcubic CFPQ. Long-time open problem. 
The best known result is !!!, 
Also it is shown by Chattergee that !!!
For 1-Dyck language : Bradford~\cite{!!!}. Can not be generalized to arbitary CFPQ.s 
We find a way

Contribution
\begin{enumerate}
	\item New algorithm. Based on operation over Boolena matrices. All paths semantics. Previous matrix-based solution only single path. For both regualr and context-free path queries.
	\item Correctness and time complexity.
	%\item Subcubic for planar graps. This cretarion is output-sensitive, so it is not practical, but open a theoretical way to find more subclass with subcubic complexity.
	\item Interconnection between CFPQ and dynamic transitive closure. Conjecture on sublinear dynamic transitive closure and subcubic CFPQ. We show that dynamic transitive closure is a bottleneck on the way to get subcubic CFPQ algorithm.
	\item Evaluation on real-world data. RPQ, CFPQ. Results show that !!!
\end{enumerate}