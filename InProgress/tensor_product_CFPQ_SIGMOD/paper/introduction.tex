\section{Introduction}


Language-constrained path querying~\cite{barrett2000formal} is one of the techniques for graph navigation querying.
This technique allows one to use formal languages as constraints on paths in edge-labeled graphs: path satisfies constraints if labels along it form a word from the specified language.

The utilization of regular languages as constraints, or \textit{Regular Path Querying} (RPQ), is most well-studied and widely spread.
Different aspects of RPQs are actively studied in graph databases~\cite{10.1145/2463664.2465216, 10.1145/3104031,10.1145/2850413}, and support of regular constraints is implemented in such popular query languages as PGQL~\cite{10.1145/2960414.2960421}, or SPARQL\footnote{Specification of regular constraints in SPARQL property paths: \url{https://www.w3.org/TR/sparql11-property-paths/}. Access date: 07.07.2020.}~\cite{10.1007/978-3-319-25007-6_1} (property paths).
Even that, improvement of RPQ algorithms efficiency on huge graphs is an actual problem nowadays, and new algorithms and solutions are being created~\cite{Wang2019,10.1145/2949689.2949711}.

At the same time, the utilization of more powerful languages, namely context-free languages, gain popularity in the last few years. 
\textit{Context-Free Path Querying} problem (CFPQ) was introduced by Mihalis Yannakakis in 1990 in~\cite{Yannakakis}.
A number of different algorithms were proposed since that time, but recently, in~\cite{Kuijpers:2019:ESC:3335783.3335791} Jochem Kuijpers et al. show that state-of-the-art CFPQ algorithms are not performant enough to be used in practice.
This fact motivates us to find new algorithms for CFPQ.

One of the promising ways to achieve high-performance solutions for graph analysis problems is to reduce graph algorithms to linear algebra.
This way, the description of basic linear algebra primitives, the GraphBLAS~\cite{7761646} API, was proposed.
Solutions that use libraries that implement this API, such as SuiteSparce~\cite{10.1145/3322125} and CombBLAS~\cite{10.1177/1094342011403516}, show that reduction to linear algebra is a way to utilize high-performance parallel and distributed computations for graph analysis.

Rustam Azimov in~\cite{Azimov:2018:CPQ:3210259.3210264} shows how to reduce CFPQ to matrix multiplication.
Late, in~\cite{Mishin:2019:ECP:3327964.3328503} and~\cite{10.1145/3398682.3399163}, it was shown that utilization of appropriate libraries of linear algebra for Azimov's algorithm implementation allows one to get practical solution for CFPQ.
But Azimov's algorithm requires transforming the input grammar to Chomsky Normal Form, which leads to the grammar size increase, thus worsen performance especially for regular queries and complex context-free queries.

To solve these problems, recently, an algorithm based on automata intersection was proposed~\cite{10.1007/978-3-030-54832-2_6}.
This algorithm is based on linear algebra and does not require the input grammar transformation.
In this work, we improve it.
First of all, we reduce it to operations over Boolean matrices, thus simplify its description and implementation.
Also, we show that this algorithm is performant enough for regular queries, so it is a good candidate for integration with real-world query languages: we can use one algorithm to evaluate both regular and context-free queries. 

Moreover, we show that this algorithm is a way to attack a long-standing problem of subcubic CFPQ. 
The best-known result for the general case is an $O(n^3/\log{n})$ algorithm of Swarat Chaudhuri~\cite{10.1145/1328438.1328460}. 
Also, there are solutions for partial cases. 
For example, there is a truly subcubic algorithm for 1-Dyck language proposed by Phillip Bradford~\cite{8249039}.
But this solution cannot be generalized to arbitrary CFPQs. 
So, in our knowledge, there is no truly subcubic general algorithm for CFPQs. 
In this work, we show that incremental transitive closure is a bottleneck on the way to get subcubic time complexity for CFPQ.

To summarize, we make the following contributions in this paper.
\begin{enumerate}
	\item We rethink and improve the tensor-product-based algorithm for CFPQ. First of all, we reduce this algorithm to operations over Boolean matrices. We now can handle all-path query semantics. The previous matrix-based solution handles only single-path semantics. We can formulate our query using both regular and context-free grammars. We proof the correctness and time complexity for the proposed algorithm.
	\item We demonstrate the interconnection between CFPQ and incremental transitive closure. Conjecture on sublinear incremental transitive closure and subcubic CFPQ. We show that incremental transitive closure is a bottleneck on the way to get a subcubic CFPQ algorithm.
	\item By using existing results we show how to get a slightly subcubic algorithm for the general case, and a subcubic combinatorial algorithm for partial cases. This criterion is output-sensitive, so it is not practical, but open a theoretical way to find more subclass with subcubic complexity.
	\item We implement the described algorithm and evaluate it on real-world data. RPQ, CFPQ. Results show that !!!
\end{enumerate}