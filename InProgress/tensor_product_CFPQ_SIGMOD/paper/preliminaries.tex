\section{Preliminaries}

In this section we introduce basic notation and definitions from graph theory and formal language theory which are used in our work.

\subsection{Language-Constrained Path Querying Problem}

We use a directed edge-labeled graph as a data model. 
To introduce the \textit{Language-Constraint Path Querying Problem}~\cite{!!!} over directed edge-labeled graphs we should give both language and grammar definitions.

First of all, we introduce edge-labelled digraph $\mathcal{G} = \langle V,E,L \rangle$, where $V$ is a finite set of vertices, $E \in V \times L \times V$ is a finite set of edges, $L$ is a finite set of edge labels. 
Note that one can always introduce bijection between $V$ and $Q = \{0, \ldots, |V|-1\}$, thus in our work we guess that $V = \{0, \ldots, |V|-1\}$.

The example of a graph which we will use in further examples is presented in Figure~\ref{fig:example_input_graph}.

\begin{figure}[h]
    \centering        
    \begin{tikzpicture}[shorten >=1pt,auto]
       \node[state] (q_0)                      {$0$};
       \node[state] (q_1) [above right=of q_0] {$1$};
       \node[state] (q_2) [right=of q_0]       {$2$};
       \node[state] (q_3) [right=of q_2]       {$3$};
        \path[->]
        (q_0) edge  node {a} (q_1)
        (q_1) edge  node {a} (q_2)
        (q_2) edge  node {a} (q_0)
        (q_2) edge[bend left, above]  node {b} (q_3)
        (q_3) edge[bend left, below]  node {b} (q_2);
    \end{tikzpicture}
    \caption{The example of input graph $\mathcal{G}$}
    \label{fig:example_input_graph}
\end{figure}

Each edge-labeled graph can be represented as an adjacency matrix $M$: square $|V|\times|V|$ matrix, such that $M[i,j] = \{l \mid e = (i,l,j) \in E\}$.
Adjacency matrix $M_2$ of the graph $\mathcal{G}$ is 

$$
    M_2 =
    \begin{pmatrix}
    . & \{a\} & . & .     \\
    . & . & \{a\} & .     \\
    \{a\} & . & . & \{b\} \\
    . & . & \{b\} & .
    \end{pmatrix}.
$$

In our work we use decomposition of the adjacency matrix to a set of Boolean matrices: 
$$
\mathcal{M} = \{M^l \mid l \in L, M^l[i,j]=1 \iff l \in M[i,j]\}.
$$

Matrix $M_2$ can be represented as a set of two Boolean matrices $M_2^a$ and $M_2^b$ where
\begin{align}
M_2^{a} =
\begin{pmatrix}
    . & 1 & . & .   \\
    . & . & 1 & .   \\
    1 & . & . & .   \\
    . & . & . & .  
\end{pmatrix}, 
M_2^{b} =
\begin{pmatrix}      
    . & . & . & .   \\
    . & . & . & .   \\
    . & . & . & 1   \\
    . & . & 1 & . 
\end{pmatrix} \label{eq:boolean_decomposition_of_graph}
\end{align}


In this way we reduce operations which are necessary for our algorithm from operations over custom semiring (over edge labels) to operations over a Boolean semiring.

Also, we should define the path in the graph and the word formed by the path.

\begin{definition}
Path $\pi$ in the graph $\mathcal{G} = \langle V,E,L \rangle$ is a sequence $e_0,e_1,\ldots,e_{n-1}$, where $e_i = (v_i,l_i,u_i) \in E$ and for any $e_i, e_{i+1}$ $u_i = v_{i+1}$. We denote path from $v$ to $u$ as $v\pi u$.   
\end{definition}

\begin{definition}
The word formed by a path $$\pi = (v_0,l_0,v_1),(v_1,l_1,v_2),\ldots,(v_{n-1},l_{n-1},v_n)$$ is a concatenation of labels along the path: $\omega(\pi) = l_0 l_1 \ldots l_{n-1}$.
\end{definition}

The next part is a definitions from the formal language theory.
\begin{definition}
A language $\mathcal{L}$ over a finite alphabet $\Sigma$ is a subset of all possible sequences formed by symbols from the alphabet: $\mathcal{L}_{\Sigma} = \{\omega \mid \omega \in \Sigma^*\}$.
\end{definition}

Now we are ready to introduce CFPQ problem for the given graph  $\mathcal{G} = \langle V,E,L \rangle$ and the given language $\mathcal{L}$ with reachability and all paths semantics.

\begin{definition}
To evaluate context-free path query with reachability semantics is to construct a set of pairs of vertices $(v_i,v_j)$ such that there exists a path $v_i \pi v_j$ in $\mathcal{G}$ which forms the word from the given language:
$$
R = \{(v_i,v_j) \mid \exists \pi: v_i \pi v_j, \omega(\pi) \in \mathcal{L} \}
$$
\end{definition}

\begin{definition}
To evaluate context-free path query with all paths semantics is to construct a set of paths $\pi$ in $\mathcal{G}$ which form the word from the given language:
$$
\Pi = \{ \pi \mid \omega(\pi) \in \mathcal{L}\}
$$
\end{definition}

Note that $\Pi$ can be infinite, thus in practice, we should provide a way of enumerating such paths with reasonable complexity, instead of explicit construction of the $\Pi$.

\subsection{Regular Path Queries and Finite State Machine}

The first case of language-constrained path querying is \textit{Regular Path Querying} (RPQ): the language $L$ is a regular language. This case is widely spread in practice~\cite{!!!}. 

Usual way to specify regular languages is \textit{regular expressions}. 
We use the following definition of regular expressions.
\begin{definition}
Regular expression (and regular language) over alphabet $\Sigma$ can be inductively defined as follows.
\begin{itemize}
    \item $\varnothing$ (empty language) is regular expression
    \item $\varepsilon$ (empty string) is regular expression
    \item $a_i \in \Sigma$ is regular expression
    \item if $R_1$ and $R_2$ are regular expressions, then $R_1 \mid R_2$ (alternation), $R_1 \cdot R_2$ (concatenation), $R_1^*$ (Kleene star) are also regular expressions. 
\end{itemize}
\end{definition} 

For example, one can specify regular expression $R_1 = ab^*$ to find paths in the graph $\mathcal{G}$ (fig.~\ref{fig:example_input_graph}). Expected result is set of paths which start with $a$-labeled edge and contain zero or more $b$-labeled edges after that.

In this work we use the notion of \textit{Finite-State Machine} (FSM) or \textit{Finite-State Automaton} (FSA) for RPQs. 

\begin{definition}
Deterministic Finite-State Machine $T$ is a tuple $\langle \Sigma, Q, Q_s, Q_f, \delta \rangle$ where
\begin{itemize}
    \item $\Sigma$ is an input alphabet,
    \item $Q$ is a finite set of states,
    \item $Q_s \subseteq Q$ is a set of start (or initial) states,
    \item $Q_f \subseteq Q$ is a set of final states,
    \item $\delta: Q \times \Sigma \to Q$ is a transition function.
\end{itemize}
\end{definition}

It is well known, that every regular expression can be converted to deterministic FSM without $\varepsilon$-transitions.
To do it one can use~\cite{automata:theory:10.5555/1177300}.
In our work we use FSM as a representation of RPQ. 
FSM can be naturally represented by a directed edge-labeled graph: $V = Q$, $L = \Sigma$, $E = \{(q_i,l,q_j) \mid \delta(q_i,l) = q_j\}$, where some vertices have special markers to specify start and final states. Example of graph-style representation of FSM $T_1$ for the regular expression $R_1$ is presented in Figure~\ref{fig:example_fsm}.

\begin{figure}[h]
    \centering        
    \begin{tikzpicture}[shorten >=1pt,auto]
       \node[state, initial] (q_0)                      {$0$};
       \node[state, accepting] (q_1) [right=of q_0] {$1$};
       \path[->]
        (q_0) edge  node {a} (q_1)
        (q_1) edge[loop above]  node {b} (q_1)
        ;
    \end{tikzpicture}
    \caption{The example of graph representation of FSM for the regular expression $ab^*$}
    \label{fig:example_fsm}
\end{figure}

As a result, FSM also can be represented as a set of Boolean adjacency matrices $\mathcal{M}$ with additional information about start and final vertices. Such representation of $T_1$ is
$$
M^a =
\begin{pmatrix}
0&1 \\
0&0
\end{pmatrix},~
M^b =
\begin{pmatrix}
0&0 \\
0&1
\end{pmatrix}.
$$  

Note, that the edge-labeled graph is an FSM: edges are transitions, all vertices should be both start and final at the same time.
Thus RPQ evaluation is an intersection of two FSMs, and the result also can be represented as FSM, because regular languages are closed under intersection.

\subsection{Context-Free Path Querying and Recursive State Machines}

An even more general case, than RPQ, is a \textit{Context-Free Path Querying Problem (CFPQ)}, where one can use context-free languages as constraints. These constraints are more expressive than the regular ones, for example, one can express classical same-generation query using context-free language, but not a regular one.


\begin{definition} 
Context-free grammar $G = \langle\Sigma, N, S, P\rangle$ where $\Sigma$ is a finite set of terminals (or terminal alphabet), $N$ is a finite set of nonterminals (or nonterminal alphabet), $S \in N$ is a start nonterminal, and $P$ is a finite set of productions (grammar rules) of form $N_i \to \alpha$ where  $N_i \in N$, $\alpha \in (\Sigma \cup N)^*$.
\end{definition}

\begin{definition}
The sequence $\omega_2 \in (\Sigma \cup N)^*$ is derivable from $\omega_1 \in (\Sigma \cup N)^*$ in one derivation step, or $\omega_1 \to \omega_2$, in the grammar $G = \langle\Sigma, N, S, P\rangle$ iff $\omega_1=\alpha N_i \beta$, $\omega_2 = \alpha \gamma \beta$, and $N_i \to \gamma \in P$.
\end{definition}

\begin{definition}
Context-free grammar $G=\langle\Sigma, N, S, P\rangle$ specifies a \textit{context-free language}: $\mathcal{L}(G) = \{\omega \mid S \xrightarrow{*} \omega \}$, where $(\xrightarrow{*})$ denotes zero or more derivation steps $(\to)$.    
\end{definition}

Thus, one can use the grammar $G_1 = \langle \{a,b\}, \{S\}, S, \{S \to a \ b; \ S \to a \ S \ b\} \rangle$ to find paths which form words in the language $\mathcal{L}(G_1) = \{a^n b^n \mid n > 0\}$ in the graph $\mathcal{G}$ (fig.~\ref{fig:example_input_graph}).

Regular expressions can be transformed to a FSM, and a context free grammar can be transformed to \textit{Recursive State Machine} (RSM) (also known as recursive networks~\cite{!!!}, recursive automata~\cite{!!!}, !!!.) in the similar way.
In our work we use the following definition of RSM.

\begin{definition}
A recursive state machine $R$ over a finite alphabet $\Sigma$ is defined as a tuple of elements $(M,m,\{C_i\}_{i \in M})$, where:

\begin{itemize}
    \item $M$ is a finite set of labels of boxes.
    \item $m \in M$ is an initial box label.
    \item Set of \textit{component state machines} or \textit{boxes},
          where $C_i=(\Sigma \cup M, Q_i,q_i^0,F_i,\delta_i)$:
    \begin{itemize}
        \item $\Sigma \cup M$ is a set of symbols, $\Sigma \cap M = \emptyset$
        \item $Q_i$ is a finite set of states,
              where $Q_i \cap Q_j = \emptyset, \forall i \neq j$
        \item $q_i^0$ is an initial state for $C_i$
        \item $F_i$ is a set of final states for $C_i$, where $F_i \subseteq Q_i$
        \item $\delta_i: Q_i \times (\Sigma \cup M) \to Q_i$ is a transition function %for $C_i$
    \end{itemize}
\end{itemize}

\end{definition}

RSM behaves as a set of finite state machines (or FSM).
Each FSM is called a \textit{box} or a \textit{component state machine}~\cite{rsm:analysis:10.1007/3-540-44585-4_18}.
A box works almost the same way as a classical FSM, but it also handles additional \textit{recursive calls} and employs an implicit \textit{call stack} to \textit{call} one component from another and then return execution flow back.

The execution of an RSM could be defined as a sequence of the configuration transitions, which are done on input symbols reading. 
The pair $(q_i,S)$, where $q_i$ is current state for box $C_i$ and $S$ is stack of \textit{return states}, describes execution configurations. 

The RSM execution starts form configuration $(q_m^0, \langle\rangle)$. 
The following list of rules defines the machine transition from configuration $(q_i,S)$ to $(q',S')$ on some input symbol $a$ from input sequence, which is read as usual for FSA:

\begin{itemize}
    \item $(q_i^k,S) \leadsto (\delta_i (q_i^k, a),S)$
    \item $(q_i^k,S) \leadsto (q_j^0, \delta_i (q_i^k, j) \circ S)$
    \item $(q_j^k,q_i^t\circ S) \leadsto (q_i^t, S),$ where $q_j^k \in F_j$ 
    %\item $q' \gets q_j^0, S' \gets q_i^t \circ S$, where $q_i^t = \delta_i (q_i^k, j),  q_i^k = q, j \in M$
    %\item $q' \gets q_i^t, S' \gets S_{tail}$, where $S = q_i^t \circ S_{tail}, q_j^k = q, q_j^k \in F_j$
\end{itemize}

Some input sequence of the symbols $a_1 ... a_n$, which forms some input word, is accepted, if machine reaches configuration $(q,\langle\rangle)$, where $q \in F_m$. It is also worth noting that the  RSM makes nondeterministic transitions, without reading the input character when it \textit{calls} some component or  makes a \textit{return}.

According to~\cite{rsm:analysis:10.1007/3-540-44585-4_18}, recursive state machines are equivalent to pushdown systems.
Since pushdown systems are capable of accepting context-free languages~\cite{automata:theory:10.5555/1177300}, it is clear that RSMs are equivalent to context-free languages.
Thus RSMs suit to encode query grammars.
Any CFG can be easily converted to an RSM with one box per nonterminal.
The box which corresponds to a nonterminal $A$ is constructed using the right-hand side of each rule for $A$.

An example of such RSM $R$ constructed for the grammar $G$ with rules $S \to a S b \mid a b$ is provided in Figure~\ref{example:automata}. 
For a given example of the grammar and the RSM consider the following sequence of the machine configuration transitions, in case, where one want to determine, if input word $aabb$ belongs to the language $L(G)$. 
The RSM execution starts from configuration $(q_S^0,\langle \rangle)$, reads symbols $a$ and goes to $(q_S^1, \langle \rangle)$. 
Then, in the nondeterministic manner it tries to read $b$ but fails, and in the same time tries to derive $S$ and goes to configuration $(q_S^0, \langle q_S^2 \rangle)$, where $q_S^2$ is \textit{return} state. 
Then machine reads $a$ and goes to $(q_S^1, \langle q_S^2 \rangle)$. In this case, in the nondeterministic choice it fails to derive $S$, but successfully reads $b$ and goes to configuration $(q_S^3,\langle q_S^2 \rangle)$. 
Since $q_S^3$ is final state for the box $S$, the RSM tries to make $return$ and goes to $(q_S^2,\langle \rangle)$. 
Then it reads $b$ and transits to $(q_S^3,\langle \rangle)$. 
Since $q_S^3 \in F_S$ and the \textit{return} stack is empty, the machine accepts the input sequence $aabb$.

Since $R$ is a set of FSMs, it is useful to represent $R$ as an adjacency matrix for the graph where vertices are states from $\bigcup_{i \in M}Q_i$ and edges are transitions between $q_i^a$ and $q_i^b$ with label $l \in \Sigma \cup M$, if $\delta_i (q_i^a, l) = q_i^b$.
An example of such adjacency matrix $M_R$ for the machine $R$ is provided in section~\ref{example:section}.

\begin{figure}[h]
    \begin{tikzpicture}[shorten >=1pt,auto]
        \node[state, initial] (q_0)   {$q_S^0$};
        \node[state] (q_1) [right=of q_0] {$q_S^1$};
        \node[state] (q_2) [right=of q_1] {$q_S^2$};
        \node[state, accepting] (q_3) [right=of q_2] {$q_S^3$};
        \path[->]
            (q_0) edge node {a} (q_1)
            (q_1) edge node {S} (q_2)
            (q_2) edge node {b} (q_3)
            (q_1) edge [bend left, above]  node {b} (q_3);
        \node[draw=black, fit= (q_0) (q_1) (q_2) (q_3), xshift=-4.5ex,inner sep=0.75cm, label=Box S] {};
    \end{tikzpicture}
    \centering
    \caption{The recursive state machine $R$ for grammar $G$}
    \label{example:automata}
\end{figure}



Similarly to a FSM, an RSM can be representaed as a graph and, hence, as a set of Boolean adjacency matrices.
For our example, $M_1$ is:
    $$
    M_1 =
    \begin{pmatrix}
    . & . & \{a\} & .     \\
    . & . & \{S\} & \{b\} \\
    . & . & . & \{b\}     \\
    . & . & . & .
    \end{pmatrix}
    $$

Matrix $M_1$ can be represented as a set of Boolean matrices as follows:
{\small
\begin{align*}
M_1^S =
\begin{pmatrix}      
    . & . & . & .   \\
    . & . & 1 & .   \\
    . & . & . & .   \\
    . & . & . & .   
\end{pmatrix},~M_1^a =
\begin{pmatrix}       
   . & . & 1 & .   \\
   . & . & . & .   \\
   . & . & . & .   \\
   . & . & . & .   \\
\end{pmatrix},~M_1^b =
\begin{pmatrix}      
    . & . & . & .   \\
    . & . & . & 1   \\
    . & . & . & 1   \\
    . & . & . & .   \\
\end{pmatrix}
\end{align*}
}
Similarly to an RPQ, a CFPQ is the intersection of the given context-free language and a FSM specified by the given graph.
As far as every context-free language is closed under intersection with regular languages, such intersection can be represented as an RSM.
Also, one can look at the RSM as a FSM over $\Sigma \cup N$.
In this work we use this point of view to propose unified algorithm for evaluation both regular and context-free path queries with zero overhead for regular ones. 
 

\subsection{Graph Kronecker Product and Machines Intersection}

First of all, we introduce classical Kronecker product definition, describe graph Kronecker product and its relation to Boolean matrices algebra, RSM and FSM intersection.

\begin{definition}
Given two matrices $A$ and $B$ of sizes $m_1 \times n_1$ and $m_2 \times n_2$ respectively, with element-wise product operation $\cdot$. The Kronecker product of these two matrices is a new matrix $C = A \otimes B$, where: 
 \begin{itemize}
     \item $C$ has size $m_1 * m_2 \times n_1 * n_2$
     \item $C[u * m_1 + v,n_1 * p + q] = A[u,p] \cdot B[v,q]$
 \end{itemize}
\end{definition}

It is worth mention, that the Kronecker product produces blocked matrix $C$, with total number of the blocks $m_1 * n_1$ , where each block has size $m_2 * n_2$ and is defined as $A[i,j] \cdot B$ (scalar to matrix).

\begin{definition}
\label{def:graph:product}
Given two edge-labeled directed graphs $\mathcal{G}_1=\langle V_1, E_1, L_1 \rangle$ and $\mathcal{G}_2=\langle V_2, E_2, L_2 \rangle$ the Kronecker product of these two graphs is a edge-labeled directed graph $\mathcal{G}=\mathcal{G}_1 \otimes \mathcal{G}_2$, where $\mathcal{G}= \langle V, E, L \rangle$:
\begin{itemize}
    \item $V = V_1 \times V_2$
    \item $E = \{((u,v),l,(p,q)) \mid (u,l,p) \in E_1 \wedge (v,l,q) \in E_2 \}$
    \item $L = L_1 \cap L_2$
\end{itemize}
\end{definition}

The Kronecker product for graphs produces a new graph with a nice property, that if some path $(u,v)\pi(p,q)$ exists in the result graph then paths $u\pi_1p$ and $v\pi_2q$ exist in the input graphs, and $\omega(\pi) = \omega(\pi_1) = \omega(\pi_2)$. These paths $\pi_1$ and $\pi_2$ could be easily found from $\pi$ by its definition.

The Kronecker product for directed graphs can be easily described as the Kronecker product of the corresponding adjacency matrices of graphs, what gives us the following definition:

\begin{definition}
Given two adjacency matrices $M_1$ and $M_2$ of  sizes $m_1 \times n_1$ and $m_2 \times n_2$ respectively, for some directed graphs $\mathcal{G}_1$ and $\mathcal{G}_2$. 
The Kronecker product of these two adjacency matrices is the adjacency matrix $M$ of a some graph $\mathcal{G}$, where:
\begin{itemize}
    \item $M$ has size $m_1 * m_2 \times n_1 * n_2$
    \item $M[u * m_1 + v,n_1 * p + q] = M_1[u,p] \cap M_2[v,q]$
\end{itemize}
\end{definition}

By the definition, the Kronecker product for adjacency matrices gives an adjacency matrix with the same set of edges as in the resulting graph in the Def.~\ref{def:graph:product}.
Thus, $M(\mathcal{G}) = M(\mathcal{G}_1) \otimes M(\mathcal{G}_2)$, where $\mathcal{G} = \mathcal{G}_1 \otimes \mathcal{G}_2$.

\begin{definition}
\label{def:fsm:intersection}
Given two FSMs $T_1 = \langle \Sigma, Q^1, Q_S^1, S_F^1, \delta^1 \rangle$ and $T_2 = \langle \Sigma, Q^2, Q_S^2, S_F^2, \delta^2 \rangle$. The intersection of this two machines is a new FSM $T = \langle \Sigma, Q, Q_S, S_F, \delta \rangle$, where:
\begin{itemize}
    \item $Q = Q^1 \times Q^2$
    \item $Q_S = Q_S^1 \times Q_S^2$
    \item $Q_F = Q_F^1 \times Q_F^2$
    \item $\delta: Q \times \Sigma \to Q$, 
    $\delta (\langle q_1, q_2 \rangle, s) = \langle q_1', q_2' \rangle$, 
    \newline if $\delta(q_1,s)=q_1'$ and $\delta(q_2,s)=q_2'$
\end{itemize}
\end{definition}

According to~\cite{automata:theory:10.5555/1177300}, the above definition of the FSM intersection allows to construct the new machine with the following property: $L(T) = L(T_1) \cap L(T_2)$. 

The most computationally expensive part of such procedure is the $\delta$ function construction for the new machine $T$. Using adjacency matrices decomposition for FSMs we can reduce the intersection to the Kronecker product of such matrices over Boolean semiring at some extent, since the transition function $\delta$ of the machine $T$ in matrix form is exactly the same as the product result. More precisely:

\begin{definition}
\label{def:bool:product}
Given two adjacency matrices $\mathcal{M}_1$ and $\mathcal{M}_2$ over Boolean semiring. 
The Kronecker product of these matrices is a new matrix 
$\mathcal{M} = \mathcal{M}_1 \otimes \mathcal{M}_2$, defined as follows:
\begin{itemize}
    \item $\mathcal{M} = \{ M_1^a \otimes M_2^a~|~a \in \Sigma \}$
    \item The element-wise operation is \textit{and} over Boolean values
\end{itemize}
\end{definition}

Applying the Kronecker product theory for both the FSM and the edge-labeled directed graph, 
we can intersect this objects as shown in Def.~\ref{def:bool:product}, since the 
graph could be interpreted as an FSM with transitions matrix represented as 
the adjacency matrix.

In this work we show how to express RSM and FSM intersection in terms of 
Kronecker product and transitive closure over Boolean semiring.