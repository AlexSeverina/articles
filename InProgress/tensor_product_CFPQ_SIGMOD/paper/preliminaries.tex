\section{Preliminaries}

In this section we introduse basic notation and definitions from graph theory and formal language theory which are used in our work.

\subsection{Context-Free Path Querying Problem}

We introduce \textit{Context-Free Path Querying Problem (CFPQ)} over directed edge-labelled graphs.
Diraph $\mathcal{G} = \langle V,E,L \rangle$

\begin{figure}[h]
    \centering        
    \begin{tikzpicture}[shorten >=1pt,auto]
       \node[state] (q_0)                      {$0$};
       \node[state] (q_1) [above right=of q_0] {$1$};
       \node[state] (q_2) [right=of q_0]       {$2$};
       \node[state] (q_3) [right=of q_2]       {$3$};
        \path[->]
        (q_0) edge  node {a} (q_1)
        (q_1) edge  node {a} (q_2)
        (q_2) edge  node {a} (q_0)
        (q_2) edge[bend left, above]  node {b} (q_3)
        (q_3) edge[bend left, below]  node {b} (q_2);
    \end{tikzpicture}
    \caption{The example of input graph $\mathcal{G}$}
    \label{fig:example_input_graph}
\end{figure}

Each graph can be representad as adjacency matrix M.
We use decomposition to the matrix to the set of Boolean matrices 
$$
\mathcal{M} = \{M^l \mid l \in L, M^l[i,j]=1 \iff l \in M[i,j]\}.
$$

This way we reduce operations over custom semirings to Boolean semiring.

Grammar $G$

CFPQ with different semantics

Reachability semantisc:
$$
12
$$

All paths semantics:
$$
23
$$

\subsection{Recursive State Machines}

?Finite state machine. Regexp to FSM.?

Also known as recursive networks~\cite{!!!}, recursive automata~\cite{!!!}, !!!  

Definition

Properties.

Grammar to RSM convertion algorithm.
Eaxmple of convertion.

Boolean decomposition of adjacency matrix

\subsection{Graph Kronecker Product}

Kronecker product definition.

Tensor product definition.

Tensot ptoduct of adjacency matrices.

Tensor product for FSM intersection.

Tensor product for FSM intersection over Boolean semiring.

Rsm and FSM intersection classical theorem proof?

Dynamic graph problems? 