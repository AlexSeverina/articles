\section{Implementation Details}

In order to evaluate the proposed algorithm, we implement its na{\"i}ve version: transitive closure computes on each iteration from scratch, without incremental techniques utilization. 
For implementation we use PyGraphBLAS\footnote{GitHub repository of PyGraphBLAS, a Python wrapper for GraphBLAS API: \url{https://github.com/michelp/pygraphblas}. Access date: 07.07.2020.} --- a Python wrapper for SuiteSparse library~\cite{10.1145/3322125}\footnote{Web page of SuiteSparse:GraphBLAS library: \url{http://faculty.cse.tamu.edu/davis/GraphBLAS.html}. Access date: 07.07.2020.}. 
SuiteSparse is a C implementation of GraphBLAS~\cite{7761646} standard which introduces linear algebra building blocks for graph analysis algorithms implementation.
Thus we provide a highly-optimized parallel CPU implementation of the na{\"i}ve version of the proposed algorithm\footnote{!!!}. 

In the current version we do not provide integration with graph database and graph query language, because our goal is the algorithm applicability evaluation.
So, we suppose that graph is stored in file, and query is expressed in terms of context-free grammar and stored in file too.
As it was shown in~\cite{10.1145/3398682.3399163} it is possible to integrate SuiteSparse based implementation in the RedisGraph database. 
To provide integration with query language, it is necessary to extend the language first.
It is possible, for example one can use existing proposal\footnote{Cypher language extension proposal which introduces a syntax to express context-free queries: \url{https://github.com/thobe/openCypher/blob/rpq/cip/1.accepted/CIP2017-02-06-Path-Patterns.adoc}. Access date: 07.07.2020.} to extend Cypher language, but it requires a lot of technical effort, so it is an interesting challenge for future research to provide full-stack support for CFPQ.     

Paths extraction also is implemented in Python by using PyGraphBLAS. 
For evaluation we implement a version which has an additional parameter: a maximal number of paths to extract. 
This modification allows as to avoid lazy evaluation which is not natural for Python.
Note that one can provide other modifications of paths extraction algorithm based on the proposed idea.