\section{Conclusion and Future Work}

In this work we present an improved version of the tensor-based algorithm for CFPQ: we reduce the algorithm to operations over Boolean matrices, and we provide ability to extract all paths which satisfie the query.
Moreover, the provided algorithm can handle grammars in EBNF, thus it does not requires grammar to be in CNF transformation and avoids grammar explosion.
As a result, the algorithm demonstrates practical performance not only on CFPQ queries, but also on RPQ ones, which shown by our evaluation. 
Thus, we provide universal linear algebra based algorithm for RPQ and CFPQ evaluation with all-paths semantics.

The first important task for the future research is an detailed investigation of paths extraction algorithm.
Jelle Hellings in~\cite{!!!} provides theoretical investigation of single path extraction, and shows that the problem is related to formal language theory.
All paths extraction is more complicated and should be investigated carefully in order to provide optimal algorithm.

Also the algorithm open a way to attack long-standing problem on subcubic CFPQ by reducing it to incremental transitive closure: \textit{incremental transitive closure with $O(n^{3-\varepsilon})$ total update time for $n^2$ updates, such that each update returns all new reachable pairs, implies $O(n^{3-\varepsilon})$ CFPQ algorithm.}
In this work we prove $O(n^3/\log{n})$ time complexity by providing $O(n/\log{n})$ transitive closure algorithm.

Thus, the first task for the future is to find truly sublinear algorithm for incremental transitive closure or, as a first step, to improve logarithmic factor.
Also, it is interesting to get improved bounds in partial cases.
For example, fully dynamic transitive closure for planar graphs can be supported in $O(n^{2/3}\log{n})$ time per update~\cite{10.1007/3-540-57273-2_72}, and for undirected graph one can use \textit{disjoint sets} which provide operations with time complexity bounded by inverse Ackermann function.
Can we use these facts to provide better CFPQ algorithm for respective partial cases? 
In the case of planarity it is interesting to investigate properties of the input graph and grammar which allow to preserve planarity during query evaluation.

On the other hand, provided reduction open a way to investigate streaming graph querying.
This way we can formulate the following questions.
\begin{enumerate}
\item Can we provide more detailed analysis of dynamic CFPQ queries, than provided in~\cite{10.1007/978-3-662-54458-7_16}?
\item Can we provide practical solution for CFPQ querying of streaming graphs?
\item Can we improve existing solutions for RPQ of streaming graphs?
\end{enumerate}

From a practical perspective, it is necessary to analyze the usability of advanced algorithms for dynamic transitive closure.
In the current work we evaluate na{\"i}ve implementation in which transitive closure recalculated on each iteration from scratch.
In~\cite{cs6345} it is shown that some of advanced algorithms for dynamic transitive closure can be efficiently implemented.
Can one of these algorithms be efficiently parallelized and utilized in the proposed algorithm?

Also, it is necessary to evaluate GPGPU-based implementation.
Experience in Azimov's algorithm shows that the utilization of GPGPUs allows one to improve performance because operations of linear algebra can be efficiently implemented on GPGPU~\cite{Mishin:2019:ECP:3327964.3328503,10.1145/3398682.3399163}. 
Moreover, for practical reason, it is interesting to provide a multi-GPU version of the algorithm and to utilize unified memory, which is suitable for linear algebra based processing of out-of-GPGPU-memory data and traversing on large graphs~\cite{8946118,10.14778/3384345.3384358}.

In order to simplify the distributed processing of huge graphs, it may be necessary to investigate different formats for sparse matrices, such as HiCOO format~\cite{10.5555/3291656.3291682}. 
Another interesting question in this direction is about utilization of virtualization techniques: should we implement distributed version of algorithm manually or it can be better to use CPU and RAM virtualization to get a virtual machine with huge amount of RAM and big number of computational cores. 
The experience of the Trinity project team shows that it can make sense~\cite{10.1145/2463676.2467799}. 

Finally, it is necessary to provide a multiple-source version of the algorithm and integrate it with a graph database.
RedisGraph\footnote{RedisGraph is a graph database that is based on the Property Graph
Model. Project web page: \url{https://oss.redislabs.com/redisgraph/}. Access date:
07.07.2020.}~\cite{8778293} is a suitable candidate for this purpose.
This database uses SuiteSparse---an implementation of GraphBLAS standard---as a base for graph processing.
This fact allowed to Arseny Terkhov et.al.  to integrate Azimov's algorithm to RedisGraph with minimal effort~\cite{10.1145/3398682.3399163}.
