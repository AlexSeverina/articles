\section{Introduction}


Language-constrained path querying~\cite{barrett2000formal} is a technique for graph navigation querying.
This technique allows one to use formal languages as constraints on paths in edge-labeled graphs: path satisfies constraints if labels along it form a word from the specified language.

The utilization of regular languages as constraints, or \textit{Regular Path Querying} (RPQ), is most well-studied and widespread.
Different aspects of RPQs are actively studied in graph databases~\cite{10.1145/2463664.2465216, 10.1145/3104031,10.1145/2850413}, while regular constraints are supported in such popular query languages as PGQL~\cite{10.1145/2960414.2960421} and SPARQL\footnote{Specification of regular constraints in SPARQL property paths: \url{https://www.w3.org/TR/sparql11-property-paths/}. Access date: 07.07.2020.}~\cite{10.1007/978-3-319-25007-6_1} (property paths).
Nevertheless, there is certainly room for improvement of RPQ efficiency, and new solutions are being created~\cite{Wang2019,10.1145/2949689.2949711}.

At the same time, using more powerful languages, namely context-free languages, as constraints has gained popularity in the last few years.
\textit{Context-Free Path Querying} problem (CFPQ) was introduced by Mihalis Yannakakis in 1990 in~\cite{Yannakakis}.
Many algorithms were proposed since that time, but recently, Jochem Kuijpers et al. showed in~\cite{Kuijpers:2019:ESC:3335783.3335791} that state-of-the-art CFPQ algorithms are not performant enough for practical use.
This motivates us to develop new algorithms for CFPQ.

One promising way to achieve high-performance solutions for graph analysis problems is to reduce them to linear algebra operations.
This way, GraphBLAS~\cite{7761646} API, the description of basic linear algebra primitives, was proposed.
Solutions that use libraries that implement this API, such as SuiteSparce~\cite{10.1145/3322125} and CombBLAS~\cite{10.1177/1094342011403516}, show that reduction to linear algebra is a way to utilize high-performance parallel and distributed computations for graph analysis.

Rustam Azimov shows in~\cite{Azimov:2018:CPQ:3210259.3210264} how to reduce CFPQ to matrix multiplication.
Later, it was shown in~\cite{Mishin:2019:ECP:3327964.3328503} and~\cite{10.1145/3398682.3399163} that utilization of appropriate libraries for linear algebra for Azimov's algorithm implementation makes a practical solution for CFPQ.
However Azimov's algorithm requires transforming the input grammar to Chomsky Normal Form, which leads to the grammar size increase, and hence worsens performance, especially for regular queries and complex context-free queries.

To solve these problems, an algorithm based on automata intersection was proposed~\cite{10.1007/978-3-030-54832-2_6}.
This algorithm is based on linear algebra and does not require the transformation of the input grammar.
We improve the algorithm in this work.
We reduce the above mentioned solution to operations over Boolean matrices, thus simplifying its description and implementation.
Also, we show that this algorithm is performant enough for regular queries, so it is a good candidate for integration with real-world query languages: one algorithm can be used to evaluate both regular and context-free queries.

Moreover, we show that this algorithm opens the way to tackle a long-standing problem about the existence of truly-subcubic $O(n^{3-\epsilon})$ CFPQ algorithm ~\cite{10.1145/1328438.1328460, Yannakakis}.
Currently, the best result is an $O(n^3/\log{n})$ algorithm of Swarat Chaudhuri~\cite{10.1145/1328438.1328460}.
Also, there exist truly subcubic solutions which use fast matrix multiplication for some fixed subclasses of context-free languages~\cite{8249039}.
Unfortunately, this solutions cannot be generalized to arbitrary CFPQs.
In this work, we identify incremental transitive closure as a bottleneck on the way to achieve subcubic time complexity for CFPQ.

To sum up, we make the following contributions.
\begin{enumerate}
	\item We rethink and improve the CFPQ algorithm based on tensor-product proposed by Orachev et al. ~\cite{10.1007/978-3-030-54832-2_6}.
	We reduce this algorithm to operations over Boolean matrices.
	As a result, all-path query semantics is handled, as opposed to the previous matrix-based solution which handles only the single-path semantics.
	Also, both regular and context-free grammars can be used as queries.
	We prove the correctness and time complexity for the proposed algorithm.
	\item We demonstrate the interconnection between CFPQ and incremental transitive closure.
	We show that incremental transitive closure is a bottleneck on the way to achieve faster CFPQ algorithm for general case of arbitrary graphs as well as for special families of graphs, such as planar graphs.
	\item We implement the described algorithm and evaluate it on real-world data. RPQ, CFPQ. Results show that !!!
\end{enumerate}