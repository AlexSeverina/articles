\section{Implementation Details}

Currenly, our goal is to evaluate the applicability of the proposed algorithm, thus we implemented its naive version.
We compute the transitive closure from scratch on each iteration and do not use any incremental techniques.
In our implementation we use PyGraphBLAS\footnote{GitHub repository of PyGraphBLAS, a Python wrapper for GraphBLAS API: \url{https://github.com/michelp/pygraphblas}. Access date: 07.07.2020.} --- a Python wrapper for SuiteSparse library~\cite{10.1145/3322125}\footnote{Web page of SuiteSparse:GraphBLAS library: \url{http://faculty.cse.tamu.edu/davis/GraphBLAS.html}. Access date: 07.07.2020.}.
SuiteSparse is a C implementation of GraphBLAS~\cite{7761646} standard which introduces linear algebra building blocks for implementation of graph analysis algorithms.
Thus we provide a highly-optimized parallel CPU implementation of the naive version of the algorithm\footnote{Implementation of the described algorithm is published here: \url{anonimized/url/to/our/repo/with/sources}. Access date: 07.07.2020.}.%{https://github.com/JetBrains-Research/CFPQ_PyAlgo}. Access date: 07.07.2020.}.

At present, we do not integrate with a graph database and a graph query language.
We suppose that the input graph is stored in a file, while the query is expressed in terms of a context-free grammar and is also stored in file.
As it was shown in~\cite{10.1145/3398682.3399163}, it is possible to integrate SuiteSparse based implementation in the RedisGraph database.
Providing integration with a query language requires a lot of technical effort to extend the language.
There are existing proposals, for example to extend the Cypher language\footnote{Cypher language extension proposal which introduces a syntax to express context-free queries: \url{https://github.com/thobe/openCypher/blob/rpq/cip/1.accepted/CIP2017-02-06-Path-Patterns.adoc}. Access date: 07.07.2020.}.

Paths extraction is implemented in Python by using PyGraphBLAS.
Since lazy evaluation is not natural for Python, we cap the maximal number of paths to extract in the implementation.
