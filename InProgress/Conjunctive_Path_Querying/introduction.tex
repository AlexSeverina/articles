\section{Introduction}

Problems in many areas can be reduced to one of the formal-languages-constrained path problems~\cite{barrett2000formal}. For example, various problems of static code analysis~\cite{bastani2015specification,xu2009scaling} can be formulated in terms of the context-free language reachability~\cite{reps1998program} or in terms of the linear conjunctive language reachability~\cite{zhang2017context}.

The result of a context-free path query evaluation is usually a set of triples \mbox{$(A, m, n)$}, such that there is a path from the node $m$ to the node $n$, whose labeling is derived from a nonterminal $A$ of the given context-free grammar. This type of query is evaluated using the \emph{relational query semantics}~\cite{hellingsRelational}. There is a number of algorithms for context-free path query evaluation using this semantics~\cite{azimov2018context,GLL,hellingsRelational,GraphQueryWithEarley,RDF}.

In~\cite{azimov2018context}, the matrix-based algorithm for context-free path query evaluation w.r.t. relational query semantics is proposed. This algorithm computes the same parsing table as the CYK algorithm but does this by offloading the most intensive computations into calls to a Boolean matrix multiplication procedure. This approach allows us to efficiently apply \emph{GPGPU} (General-Purpose computing on Graphics Processing Units) computing techniques and other optimizations for matrix operations.

In static code analysis the expressiveness of context-free grammars is not enough and conjunctive grammars~\cite{okhotinConjAndBool} are better suited to do the task. For example, conjunctive grammars are used in static analysis to describe an interleaved matched-parentheses language~\cite{zhang2017context}, which is not context-free. Path querying with conjunctive grammars is undecidable~\cite{hellingsRelational}. There is an algorithm~\cite{zhang2017context} for path querying with linear conjunctive grammars~\cite{okhotinConjAndBool} which provides an over-approximation of the result. However, there is no algorithm for path querying with arbitrary conjunctive grammars.

The purpose of this work is to develop the algorithm
for path querying with arbitrary conjunctive grammars using the matrix-based approach from~\cite{azimov2018context}. It will also be shown that the proposed algorithm allows us to efficiently apply GPGPU computing techniques.

This paper is structured as follows. Section~\ref{section_preliminaries} defines some notions, used later on. In section~\ref{section_related} the overview of related works is presented. Section~\ref{section_main} discusses our matrix-based algorithm for path querying with arbitrary conjunctive grammar and provides a step-by-step demonstration for a small example. We evaluate the performance of our algorithm in section~\ref{section_evaluation}, and provide some concluding remarks in section~\ref{section_conclusion}.
