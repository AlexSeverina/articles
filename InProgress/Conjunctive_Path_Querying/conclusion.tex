\section{Conclusion and future work} \label{section_conclusion}
In this work, we have shown how the path query evaluation w.r.t. the conjunctive grammars and relational query semantics can be reduced to the calculation of matrix transitive closure. In addition, we introduced an algorithm for computing this transitive closure for an arbitrary conjunctive grammar. Also, we provided a formal proof of the correctness of the proposed algorithms. Finally, we have shown that the proposed algorithm allows us to efficiently apply GPGPU computing techniques by running different implementations of this algorithm on classical queries. 

We can identify several open problems for further research. In this work, we have considered only one semantics of path querying but there are other important semantics, such as single-path and all-path query semantics~\cite{hellingsPathQuerying}. Whether it is possible to generalize our approach for these semantics is an open question.

In our algorithm, we calculate the matrix transitive closure naively, but there are algorithms for the transitive closure calculation, which are asymptotically more efficient. Therefore, the question is whether it is possible to apply these algorithms for the matrix transitive closure calculation to the problem of conjunctive path querying.

Also, there are Boolean grammars~\cite{okhotinBoolean}, which have more expressive power than conjunctive grammars. Boolean path querying problems are undecidable~\cite{hellingsRelational} but our algorithm for path querying with conjunctive grammars can be trivially generalized to work on Boolean grammars because parsing with Boolean grammars can be expressed by matrix multiplication~\cite{okhotin_cyk}. It is not clear what a result of our algorithm applied to this grammars would look like. Our hypothesis is that it also would produce some over-approximation of a solution.
