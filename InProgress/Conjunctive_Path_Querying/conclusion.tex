\section{Conclusion and future work} \label{section_conclusion}
In this work, we have shown how the over-approximation of the result of path query evaluation w.r.t. the conjunctive grammars and relational query semantics can be found by the calculation of matrix transitive closure. In addition, we introduced an algorithm for computing this transitive closure for an arbitrary conjunctive grammar. Also, we provided a formal proof of the correctness of the proposed algorithms. Finally, we have shown that the proposed algorithm allows us to efficiently apply GPGPU computing techniques by running different implementations of this algorithm on classical queries. 

We can identify several open problems for further research. In this work, we have considered only one semantics of path querying but there are other important semantics, such as single-path and all-path query semantics~\cite{hellingsPathQuerying}. Whether it is possible to generalize our approach for these semantics is an open question.

In our algorithm, we calculate the matrix transitive closure naively, but there are algorithms for the transitive closure calculation, which are asymptotically more efficient. Therefore, the question is whether it is possible to apply these algorithms for the matrix transitive closure calculation to the problem of conjunctive path querying.

There are number of GPU implementations for graph algorithms which are used in static analysis~\cite{mendez2012gpu,su2013accelerating,su2016efficient}. Also, our algorithm  allows us to efficiently apply GPGPU computing techniques and can be used in static analysis, for example, using conjunctive grammar to describe an interleaved matched-parentheses language~\cite{zhang2017context}. The question is whether there are other conjunctive grammars which can be used in static analysis.
