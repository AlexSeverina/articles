\documentclass[12pt]{article}  % standard LaTeX, 12 point type

\usepackage{geometry}

\usepackage{amsmath}
\usepackage{amsfonts,latexsym}
\usepackage{amsthm}
\usepackage{amssymb}
\usepackage[utf8]{inputenc} % Кодировка
\usepackage[english,russian]{babel} % Многоязычность
\usepackage{verbatim}
\usepackage{longtable}
\usepackage{csvsimple}
\usepackage[toc,page]{appendix}
\usepackage{booktabs}

\usepackage{float}
\usepackage{array}
\usepackage{multirow}
\usepackage{caption}
\usepackage{graphicx}
\usepackage{ucs}
\usepackage{rotating}
\usepackage{pdflscape}
\usepackage{afterpage}
\usepackage{capt-of}% or use the larger `caption` package
\usepackage{url}

% unnumbered environments:

\theoremstyle{remark}
\newtheorem*{remark}{Remark}
\newtheorem*{notation}{Notation}
\newtheorem*{note}{Note}

\setlength{\parskip}{5pt plus 2pt minus 1pt}
\newcolumntype{C}{>{\centering\arraybackslash}p{1.3cm}}
\graphicspath{{pics/}}

\title{Использование формальных грамматик и искусственных нейронных сетей для анализа вторичной структуры геномных и протеомных  последовательностей}
\author{Семён Григорьев}
\date{\today}

\begin{document}

\newgeometry{left=0.8in,right=0.8in,top=1in,bottom=1in}

\maketitle

\section{Основные данные проекта}

\subsection{Название проекта}

\textbf{ru}\\
%
Использование формальных грамматик и искусственных нейронных сетей для анализа вторичной структуры геномных и протеомных  последовательностей
\\
\\
\textbf{en}\\

\subsection{Основной код (по классификатору РФФИ)}
%
Н3 Переход к персонализированной медицине, высокотехнологичному здравоохранению и технологиям здоровьесбережения, в том числе за счет рационального применения лекарственных препаратов (прежде всего антибактериальных)

\subsection{Дополнительные коды (по классификатору РФФИ)}

\subsection{Ключевые слова (указываются отдельные слова и словосочетания, наиболее полно отражающие содержание проекта: не более 15, строчными буквами, через запятые)}
\textbf{ru}\\
%

\textbf{en}\\


\subsection{Аннотация проекта (кратко, в том числе – актуальность, уровень значимости и научная новизна исследования; ожидаемые результаты и их значимость; аннотация будет опубликована на сайте РФФИ, если проект получит поддержку)}

\textbf{ru}\\
Графы широко используются для представления данных в таких областях, как социальные сети, графовые базы данных, верификация, semantic web, биоинформатика. Одной из часто решаемых задач является задача поиска путей в графе, удовлетворяющих заданным ограничениям на метки рёбер пути. Рассматривая метку одного ребра как символ в некотором алфавите, пути в графе можно сопоставить слово, а множеству путей — язык над некоторым алфавитом. Таким образом, искомое множество путей можно специфицировать с помощью грамматики. При этом выразительности регулярных грамматик часто оказывается недостаточно, поэтому при решении многих практических задач необходимо использовать контекстно-свободные грамматики (КС-грамматики). Например, такая задача биоинформатики, как поиск в метагеномной сборке цепочек с заданной вторичной структурой, которая описывается КС-грамматикой, тоже может быть сформулирована как поиск путей в графе. Также использование КС-грамматик актуально при обработке встроенных языков или динамически формируемого кода — ситуаций когда программа формирует в процессе своей работы код другой программы. Здесь граф представляет собой конечный автомат, порождающий все возможные значения динамически формируемого кода, а ограничения задаются грамматикой языка, код на котором должен генерироваться. Целью является проверка корректности исходной программы, а значит необходимо проверить, что все пути из стартовых в конечные состояния входного конечного автомата удовлетворяют заданным ограничениям, и выдать сообщение об ошибке, если это не так. При этом, в случае ошибки необходимо явно указать путь, приводящий к ней. В результате возникает задача поиска путей в графе, удовлетворяющих контекстно-свободным ограничениям: ограничениям, заданным в виде КС-грамматики. Исследования в данной области ведутся активно, однако ряд вопросов остаются открытыми, например, вопрос о возможности использования нисходящих алгоритмов синтаксического анализа для решения задачи поиска путей. В рамках данного исследования предполагается ответить на некоторые из этих вопросов.


\textbf{en}\\


\subsection{Сроки реализации проекта}

\textbf{ru}\\
3 года

\section{Содержание проекта}

\subsection{Цель и задачи проекта}

\textbf{ru}\\
Цель.
Разработка эффективного подхода к поиску путей в графе, удовлетворяющих контекстно-свободным ограничениям.

Задачи.
\begin{itemize}
\item Разработка алгоритма поиска путей в графах, удовлетворяющих ограничениям, заданным в виде контекстно-свободной грамматики, основанного на алгоритме обобщённого нисходящего синтаксического анализа (GLL).
\item Теоретическая оценка временной и пространственной сложности разработанного алгоритма.
\item Программная реализация предложенного алгоритма.
\item Экспериментальная проверка оценки сложности разработанного алгоритма.
\item Разработка механизма обнаружения и диагностики ошибок для разработанного алгоритма.
\item Разработка на основе предложенного алгоритма прототипа решения для поиска последовательностей с заданной вторичной структурой в метагеномных сборках.
\end{itemize}

\subsection{Актуальность исследования}

\textbf{ru}\\
Интерес к области растёт, работа показывает, что создание эффективного решения всё еще открытая проблема.

\subsection{Направление из Стратегии научно-технологического развития Российской Федерации (при наличии) (выбор из справочника)}

\textbf{ru}\\


\subsection{Анализ современного состояния исследований в данной области (приводится обзор исследований в данной области со ссылками на публикации в научной литературе)}

\textbf{ru}\\
Задача поиска путей с контекстно-свободными ограничениями разрешима за полиномиальное время от размера графа (Barrett C., Jacob R., Marathe M. Formal-language-constrained path problems. 2000 г.). В настоящий момент предложены алгоритмы со сложностью $O(|G|*n^5)$ (Lange M. Model checking propositional dynamic logic with all extras. 2006.), $O(n^3*m^2)$ (Sevon P., Eronen L. Subgraph queries by context-free grammars. 2008.), $O((|G|*m) + (|G|*n)^3)$ (Hellings J. Conjunctive context-free path queries. 2014.). Здесь $|G|$ — константа, зависящая от грамматики, n — количество вершин во входном графе, m — количество рёбер во входном графе. При этом в качестве алгоритма синтаксического анализа используются такие алгоритмы как CYK или Earley, вычислительная сложность которых в лучшем случае $O(n^3)$ и $O(n^2)$ соответственно, что хуже $O(n)$, достигаемого такими алгоритмами, как GLR, на однозначных грамматиках. Кроме этого, например, CYK требует преобразования грамматики к нормальной форме Хомского, что приводит к её значительному разрастанию и негативно влияет на производительность алгоритма (увеличивая константу |G|).

В работах Annamaa A. et al. “An interactive tool for analyzing embedded SQL queries” (2010г) и E. Verbitskaia, S. Grigorev, and D. Avdyukhin. “Relaxed parsing of regular approximations of string-embedded languages” (2015г), посвящённых обработке встроенных языков, используются алгоритмы семейства LR. Они не требуют преобразования грамматик в нормальную форму Хомского и демонстрируют линейную сложность на однозначных грамматиках. Однако известно, что GLR и RNGLR, используемые в указанных работах, в худшем случае могут демонстрировать более чем кубическую сложность. Кроме того, отсутствуют теоретические оценки сложности предложенных на их основе решений.

В рамках данного проекта предполагается разработка алгоритма поиска путей с КС-ограничениями на основе GLL-алгоритма, который для LL-грамматик показывает линейную сложность и кубическую в худшем случае. Также будет выполнена теоретическая оценка временной и пространственной сложности предложенного алгоритма.

Существенной проблемой также является сложность отладки запросов и анализа результатов их исполнения. Одно из возможных решений данной проблемы предложено в работе P. Hofman and W. Martens “Separability by short subsequences and subwords” (2015).

Предложенный в работе E. Verbitskaia, S. Grigorev, and D. Avdyukhin. Relaxed parsing of regular approximations of string-embedded languages (2015г) алгоритм строит лес вывода для всех корректных путей — структурное представление, которое может быть использовано для отладки и анализа структуры результата. В данной работе предполагается использовать эти результаты и адаптировать их к разрабатываемому алгоритму.

Для работы с метагеномными сборками существуют такие инструменты как Xander, EMIRGE и Reago, однако они не обладают достаточной производительностью и точностью. Кроме того, не все инструменты обрабатывают сборки, представленные в виде графа, и не используют грамматики для описания вторичных структур искомых цепочек. Использование грамматик для описания вторичной структуры достаточно изучено и распространено (Eddy S. R. “Homology searches for structural RNAs: from proof of principle to practical use”. 2015; Anderson J. W. J. et al. “Evolving stochastic contextfree grammars for RNA secondary structure prediction”. 2012) и используется, например, в инструменте Infernal, обладающем высокой точностью и скоростью работы. Однако данный инструмент не применим к метагеномным сборкам.

В рамках данного исследования предполагается разработать на основе предложенного алгоритма инструмент, применимый к метагеномным сборкам и использующий грамматики для задания вторичной структуры искомых цепочек.


\subsection{Научная новизна проекта (формулируется научная идея, постановка и решение заявленной проблемы)}

\textbf{ru}\\
!!!


\subsection{Предлагаемые подходы и методы, их обоснование для реализации цели и задачи проекта (Развернутое описание; форма изложения должна дать возможность эксперту оценить соответствие подходов и методов поставленным целям и задачам проекта)}

\textbf{ru}\\
— Алгоритм обобщённого нисходящего синтаксического анализа GLL. Ранее данный алгоритм не применялся для решения задачи поиска путей в графе.
— Построение конечной структуры данных для представления результатов запроса. Данный подход к представлению результата был впервые предложен участниками данной исследовательской группы. Планируется обобщение данного подхода для применения в целях отладки запросов и анализа их результатов.
— Применение разработанного алгоритма поиска путей с контекстно-свободными ограничениями для поиска цепочек с заданной с помощью грамматики вторичной структурой в метагеномных сборках. Использование грамматик для описания вторичной структуры широко распространено, однако алгоритм поиска путей с КС-ограничениями ранее не применялся.


\subsection{Ожидаемые результаты реализации проекта и их научная и прикладная значимость}

\textbf{ru}\\
— Алгоритм поиска путей в графе, удовлетворяющих КС-ограничениям. Результатом работы алгоритма является конечная структура данных, представляющая результат выполнения запроса.
— Теоретическая оценка временной и пространственной сложности предложенного алгоритма.
— Программная реализация разработанного алгоритма.
— Механизм диагностики ошибок для разработанного алгоритма.
— Прототипы программных средств, использующих разработанный алгоритм, для решения прикладных задач.
— Оценка применимости разработанного алгоритма к решению задачи биоинформатики по поиску цепочек в метагеномных сборках.
— Программные компоненты для решения прикладных задач, использующие разработанный алгоритм.

\subsection{Общий план реализации проекта (форма представления информации должна дать возможность эксперту оценить реализуемость заявленного исследования; общий план реализации проекта даётся с разбивкой по годам)}

\textbf{ru}\\
\begin{itemize}
  \item Первый год
  \begin{itemize}
    \item Собран и проанализирован набор данных для экспериментального исследования
    \item Проведён сравнительный анализ алгоритмов
    \item Разреженные матрицы
    \item Кратчайший путь
  \end{itemize}

  \item Второй год
  \begin{itemize}
    \item Собран и проанализирован набор данных для экспериментального исследования
    \item Проведён сравнительный анализ алгоритмов
    \item Разреженные матрицы
    \item Кратчайший путь
  \end{itemize}
  \item Третий год
  \begin{itemize}
    \item Собран и проанализирован набор данных для экспериментального исследования
    \item Проведён сравнительный анализ алгоритмов
    \item Разреженные матрицы
    \item Кратчайший путь
  \end{itemize}

\end{itemize}


\subsection{Ожидаемые научные результаты за первый год реализации проекта (форма изложения должна дать возможность провести экспертизу результатов)}

\textbf{ru}\\
\begin{itemize}
  \item Собран и проанализирован набор данных для экспериментального исследования
  \item Проведён сравнительный анализ алгоритмов
  \item Разреженные матрицы
  \item Кратчайший путь
  \item Подготовлены публикации. Результаты представлены на тематических конференциях
\end{itemize}


\subsection{Имеющийся у коллектива научный задел по проекту (указываются полученные результаты, разработанные программы и методы, экспериментальное оборудование, материалы и информационные ресурсы, имеющиеся в распоряжении коллектива для реализации проекта)}

\textbf{ru}\\
\begin{itemize}
  \item Разработан алгоритм ослабленного синтаксического анализа графов на основе обобщённого LR-анализа (RNGLR), решающий задачу поиска путей, удовлетворяющих контекстно-свободным ограничениям. Алгоритм RNGLR впервые применён для поиска путей.
  \item  Разработана модификация алгоритма обобщённого LL-анализа, использующая таблицы вместо явной генерации кода, что необходимо для его применения в задаче поиска путей. Ранее рассматривались версии обобщённого LL алгоритма, использующие явную генерацию кода.
  \item Разработан алгоритм поиска путей с контекстно-свободными ограничениями, основанный на алгритме обобщённого LL-анализа (GLL). Алгоритм GLL впервые применён для поиска путей.
  \item Разработано решение для поиска путей с контекстно-свободными ограничениями, основанное на парсер-комбинаторах. Особенностями данного решения являются: прозрачная интеграция запросов в язык программирования общего назначения, возможность снабжать запросы дополнительными семантическими действиями (напрмиер, дополнительная фильтрация путей). Решение с подобными свойствами предложено впервые.
  \item Разработан алгоритм поиска путей с контекстно-свободными ограничениями, основанный на опреациях над матрицами. Данное решение позволяет использовать для поиска путей с контекстно-свободными ограничениями массово-переллельные архитектуры (напрмиер, GPGPU). Ранее подобный подход к решению задачи представлен не был.
  \item  Предложена архитектура программного средства, позволяющая использовать разработанный алгоритм для создания инструментов синтаксического анализа встроенных языков. Разработана соответствующая программная платформа. Ранее предлагались инструменты, предназначенные для решения специфичных задач; платформа, позволяющая создавать собственные инструменты, предложена впервые.
  \item  Также нашим коллективом выполнены и успешно защищены кандидатская диссертация и ряд дипломных работ и магистерских диссертаций по тематике проекта.
\end{itemize}

\subsection{Публикации (не более 15) участников коллектива, включая руководителя коллектива, наиболее близко относящиеся к проекту за последние 5 лет (для каждой публикации, при наличии, указать ссылку в сети Интернет для доступа эксперта к аннотации или полному тексту публикации)}

\textbf{ru}\\
\begin{enumerate}
\item Nikita Mishin, Iaroslav Sokolov, Egor Spirin, Vladimir Kutuev, Egor Nemchinov, Sergey Gorbatyuk, and Semyon Grigorev. 2019. Evaluation of the Context-Free Path Querying Algorithm Based on Matrix Multiplication. In Proceedings of the 2nd Joint International Workshop on Graph Data Management Experiences \& Systems (GRADES) and Network Data Analytics (NDA) (GRADES-NDA'19), Akhil Arora, Arnab Bhattacharya, and George Fletcher (Eds.).
Ссылка: https://dl.acm.org/citation.cfm?id=3328503

\item Ekaterina Verbitskaia, Ilya Kirillov, Ilya Nozkin, and Semyon Grigorev. 2018. Parser combinators for context-free path querying. In Proceedings of the 9th ACM SIGPLAN International Symposium on Scala (Scala 2018).
Ссылка: https://dl.acm.org/citation.cfm?id=3241655

\item Rustam Azimov and Semyon Grigorev. 2018. Context-free path querying by matrix multiplication. In Proceedings of the 1st ACM SIGMOD Joint International Workshop on Graph Data Management Experiences \& Systems (GRADES) and Network Data Analytics (NDA) (GRADES-NDA '18), Akhil Arora, Arnab Bhattacharya, George Fletcher, Josep Lluis Larriba Pey, Shourya Roy, and Robert West (Eds.).
Ссылка: https://dl.acm.org/citation.cfm?id=3210264

\item Semyon Grigorev and Anastasiya Ragozina. 2017. Context-free path querying with structural representation of result. In Proceedings of the 13th Central \& Eastern European Software Engineering Conference in Russia (CEE-SECR '17).
Ссылка: https://dl.acm.org/citation.cfm?id=3166104

\item Marina Polubelova, Sergey Bozhko, Semyon Grigorev. 2016. Certified Grammar Transformation to Chomsky Normal Form in F*. Proceedings of the Institute for System Programming.
Ссылка: \url{http://www.ispras.ru/en/proceedings/isp_28_2016_2/isp_28_2016_2_127/}

\item Polubelova M. I., Grigor'ev S. V. Lexical analysis of dynamically generated string expressions //Sistemy i Sredstva Informatiki [Systems and Means of Informatics]. – 2016. – Т. 26. – №. 2. – С. 43-62.
Ссылка: \url{http://www.mathnet.ru/php/archive.phtml?wshow=paper&jrnid=ssi&paperid=461&option_lang=eng}

\item Verbitskaia E., Grigorev S., Avdyukhin D. Relaxed Parsing of Regular Approximations of String-Embedded Languages //International Andrei Ershov Memorial Conference on Perspectives of System Informatics. – Springer International Publishing, 2015. – С. 291-302.
Ссылка: \url{http://link.springer.com/chapter/10.1007/978-3-319-41579-6_22}

\item Semen Grigorev, Ekaterina Verbitskaia, Andrei Ivanov, Marina Polubelova, and Ekaterina Mavchun. 2014. String-embedded language support in integrated development environment. In Proceedings of the 10th Central and Eastern European Software Engineering Conference in Russia (CEE-SECR '14). ACM, New York, NY, USA, , Article 21 , 11 pages.
Ссылка: \url{http://dl.acm.org/citation.cfm?id=2687247&CFID=498663671&CFTOKEN=45521518}

\item Grigor'ev S. V., Ragozina A. K. Generalized table-based LL-parsing //Sistemy i Sredstva Informatiki [Systems and Means of Informatics]. – 2015. – Т. 25. – №. 1. – С. 89-107.
Ссылка: \url{http://www.mathnet.ru/php/archive.phtml?wshow=paper&jrnid=ssi&paperid=395&option_lang=eng}

\end{enumerate}

\end{document}
