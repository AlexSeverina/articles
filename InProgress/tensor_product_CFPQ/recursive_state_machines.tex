\section{Recursive State Machines}

In this section, we introduce the recursive state machine (RSM). This kind of computational machine extends the definition of finite state machines and increases the computational capabilities of this formalism.

A recursive state machine $R$ over a finite alphabet $\Sigma$ is defined as tuple of elements $(M,m,\{C_i\}_{i \in M})$, where:

\begin{itemize}
    \item $M$ is a finite set of boxes' labels
    \item $m$ is an initial box label
    \item Set of \textit{component state machines} or \textit{boxes}, 
          where $C_i=(\Sigma \cup M, Q_i,q_i^0,F_i,\delta_i)$:
    \begin{itemize}
        \item $\Sigma \cup M$ is set of symbols, $\Sigma \cap M = \emptyset$
        \item $Q_i$ is finite set of states, 
              where $Q_i \cap Q_j = \emptyset, \forall i \neq j$
        \item $q_i^0$ is an initial state for component state machine $C_i$
        \item $F_i$ is set of final states for $C_i$, where $F_i \subseteq Q_i$
        \item $\delta_i$ is transition function for $C_i$, 
              where $\delta_i: Q_i \times (\Sigma \cup M) 
              \to Q_i$
    \end{itemize}
\end{itemize}

RSM behaves as set of finite state machines (or FSM), so called \textit{boxes} or \textit{component state machines}~\cite{rsm:analysis:10.1007/3-540-44585-4_18}, which are executed in classical definition of FSM with additional \textit{recursive calls} and implicit \textit{call stack}, what allows to \textit{call} one component from another, and then return execution flow back.


%The execution of RSM $R$ is a sequance of transitions between configurations of form $(q,S)$, where $S$ is global stack of \textit{return states} from $\bigcup_{i \in M}Q_i$ such as $S=\langle ...q_r \rangle$, and $q \in \bigcup_{i \in M}Q_i$ is a current state of the machine. The execution starts from box $m$ initial state $q_m^0$ and empty stack as follows $(q_m^0,\langle \rangle)$. Transitions for a given global machine state $(q,S)$ to some new state $(q',S')$ are defined as follows:

Accordingly to~\cite{rsm:analysis:10.1007/3-540-44585-4_18}, recursive state machines are equivalent to pushdown systems. Since pushdown systems are capable of accepting context-free languages~\cite{automata:theory:10.5555/1177300}, it is clear that RSMs are equals to context-free languages.
Thus we can use an RSMs to encode query grammar. 
Any CFG can be easily converted to the corresponding RSM with one box per nonterminal. The box corresponding to the nonterminal $A$ constructed using all right hand sides of the rules with $A$ as left hand side. 
An example of such RSM $R$ for the grammar $G$ with rule $S \to a S b \mid a b$ is provided in figure~\ref{example:automata}.

%We escape detailed discussion of the conversion of a context free grammar $G$ to a RSM $R$. In the basic case, an algorithm for constructing such RSM could be composed from several stages, where each stage involves building of finite state machine for each non-terminal $N$ from grammar $G$.

Since $R$ is a set of FSMs, it is useful for computational tasks to represent $R$ as a adjacency matrix, where vertices are states from $\bigcup_{i \in M}Q_i$ and edges are transitions between $q_i^a$ and $q_i^b$ with label $l \in \Sigma \cup M$, if $\delta_i (q_i^a, l) = q_i^b$. An example of such adjacency matrix $M_R$ for our machine $R$ is be provided in section~\ref{example:section}.

%\begin{itemize}
%    \item $q':=q_b', S':=S$, 
%    where $\delta_b (q_b,s) \to q_b', s \in \Sigma, 
%    q=q_b, S = \langle ...q_r \rangle$
%    \item $q':=q_t^0, S':=\langle ...q_r,q_b'\rangle$, 
%    where $\delta_b (q_b,t) \to q_b', t \in M, q=q_b, S = \langle ...q_r \rangle$
%    \item $q':=q_b', S':=\langle ...q_r \rangle$, 
%    where $q=q_t^i, q_t^i \in F_t, S=\langle ...q_r,q_b' \rangle$
%\end{itemize}

\begin{figure}[h]
    \begin{tikzpicture}[shorten >=1pt,auto]
        \node[state, initial] (q_0)   {$q_S^0$};
        \node[state] (q_1) [right=of q_0] {$q_S^1$};
        \node[state] (q_2) [right=of q_1] {$q_S^2$};
        \node[state, accepting] (q_3) [right=of q_2] {$q_S^3$};
        \path[->]
            (q_0) edge node {a} (q_1)
            (q_1) edge node {S} (q_2)
            (q_2) edge node {b} (q_3)
            (q_1) edge [bend left, above]  node {b} (q_3);
        \node[draw=black, fit= (q_0) (q_1) (q_2) (q_3), xshift=-4.5ex,inner sep=0.75cm, label=Box S] {};
    \end{tikzpicture}
    \centering
    \caption{The recursive state machine $R$ for grammar $G$}
    \label{example:automata}
\end{figure}

