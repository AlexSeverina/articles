\section{Conclusion}

We presented a new algorithm for CFPQ which is based on Kronecker product and transitive closure.
Thus it can be implemented by using high-performance libraries for linear algebra. Also, our algorithm handles queries represented as recursive state machines, thus it avoids grammar growth.

We implement the proposed algorithm by using SuiteSparse and evaluate it on several graphs and queries.
We show that in some cases our algorithm outperforms the matrix-based algorithm, but in the future, we should improve our implementation to be applicable for real-world graphs analysis. 

Also in the future, we should investigate such formal properties of the proposed algorithm as time and space complexity.
Moreover, we should analyze behavior dependency on query type and its form. Namely, we should analyze regular path queries evaluation and context-free path queries in the form of extended context-free grammars (ECFG)~\cite{10.1007/978-3-642-00982-2_35}.
Utilization of ECFGs may provide a way to optimize queries by minimization of both the right parts of productions and whole result RSM.

Finally, it is necessary to compare our algorithm with the matrix-based one in cases when the size difference between Chomsky Normal Form and ECFG representation of the query is significant.