\section{Evaluation}

We implement the proposed algorithm by using SuiteSparse\footnote{SuteSparse is a sparse matrix software which incudes GraphBLAS API implementation. Project web page: \url{http://faculty.cse.tamu.edu/davis/suitesparse.html}. Access date: 12.03.2020}~\cite{Davis2018Algorithm9S}: the implementation of GraphBlas API~\cite{7761646}. GraphBlas API specifies a set of linear algenbra promitives aand operation which allows one to formulate graph algorithms using linear algebra over custom semirings. 

We compare our implementation with results provided in~\cite{10.1145/3327964.3328503}, accordingly we use dataset described in this article which consists of \textbf{RDF}, \textbf{Worst case}, and \textbf{Full} subsets. For RDF querying we use same-generatoin query $G_4$ from~\cite{10.1145/3327964.3328503}.

For evaluation, we use a PC with Ubuntu 18.04 installed. It has Intel(R) Core(TM) i7-4790 CPU @ 3.60GHz CPU, DDR4 32 Gb RAM.

The results of the evaluation are summarized in the table~\ref{tbl:tableRDF}.
Time is measured in seconds, $t_1$ is an execution time for the proopsed algortihm, and $t_2$ is a time for M4RI-based implementation --- the best CPU version form~\cite{10.1145/3327964.3328503}. The result for algorithm is averaged over 10 runs. We exclude the time required to load data from file. The time required for data transfer and its conversion is included.

{\setlength{\tabcolsep}{0.4em}
\begin{table}[ht]
\centering
\caption{Evaluation results}
\label{tbl:tableRDF}
%\rowcolors{1}{}{lightgray}
\begin{tabular}{| c | p{1.6cm} | c | c | c | c || c | p{0.8cm} | c | c | c | c |}
    \hline
      &  Graph              & \#V & \#E  & $t_1$  & $t_2$ &  & Graph & \#V & \#E     & $t_1$    & $t_2$ \\
       \hline
       \hline
    \parbox[t]{2mm}{\multirow{11}{*}{\rotatebox[origin=c]{90}{RDF}}}
      & \small{atm-prim}                    & 291 & 685     & 0.24   & 0.02 & 
     \parbox[t]{2mm}{\multirow{2}{*}{\rotatebox[origin=c]{90}{RDF}}} & \small{core}                        & 1323 & 8684   & 0.28  &  0.12   \\
      & \small{biomed}                      & 341 & 711     & 0.24  & 0.05 & & \small{wine}                        & 733 & 2450    & 1.71  & 0.06      \\\cline{7-12}
      & \small{foaf}                        & 256 & 815     & 0.07  & 0.02 & 
      \parbox[t]{2mm}{\multirow{5}{*}{\rotatebox[origin=c]{90}{Worst case}}} & $WC_1$& 64 & 65 & 0.03 & 0.04      \\
      & \small{funding}                     & 778 & 1480    & 0.43  & 0.07 & & $WC_2$ & 128 & 129 & 0.16 & 0.23      \\
      & \small{generations}                 & 129 & 351     & 0.04  & 0.03 & & $WC_3$ & 256 & 257 & 0.96 & 1.94    \\
      & \small{people\_pets}                & 337 & 834     & 0.18  & 0.03 & & $WC_4$ & 512 & 513 & 7.14 & 23.21      \\
      & \small{pizza}                       & 671 & 2604    & 1.14  & 0.08 & & $WC_5$ & 1024& 1025&  121.99 & 528.52      \\ \cline{7-12}
      & \small{skos}                        & 144 & 323     & 0.02  & 0.04 & 
      \parbox[t]{2mm}{\multirow{4}{*}{\rotatebox[origin=c]{90}{Full}}} & $F_1$ & 100 & 100 & 0.17 &  0.02     \\
      & \small{travel}                      & 131 & 397     & 0.05  & 0.05 & & $F_2$ & 200 & 200 & 1.04 & 0.03        \\
      & \small{unv-bnch}                    & 179 & 413     & 0.05  & 0.04 & & $F_3$ & 500 & 500 & 18.86  & 0.03   \\
      & \small{pathways}                    & 6238 & 37196  & 4.88 &   0.18 & & $F_4$ & 1000 & 1000& 554.22 & 0.07       \\
      \hline
  \end{tabular}
\end{table}
}

We can see, that while RDF querying time is better for M4RI in general, in some cases execution times are comparable. For example, for graphs \textit{generations}, \textit{travel},  \textit{unv-bnch}, \textit{skos}.
For \textbf{Full} data set performance of our algorithm is bad because SuiteSparse is based on sparse matrix representattion, and in this case matrices density changes agressively from very sparse to full.
At the same time we can see, that in \textbf{Worst case} our algorithms up to 4 times faster tat M4RI (graph $WC_5$).

To sum up, our prototype implementation of the described algorithm is not performant enough to be used for real-world applications but it outperforms a matrix-based algorithm on \textbf{Worst case} dataset and comparable with it on some graphs from the \textbf{RDF} dataset. Thus we can conclude that we should improve our implementation to achieve better performance.

