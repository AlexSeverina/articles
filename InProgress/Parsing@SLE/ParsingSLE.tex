\documentclass{vldb}
\usepackage{graphicx}
%\usepackage{balance}  % for  \balance command ON LAST PAGE  (only there!)

%\usepackage{caption}
%\usepackage{subcaption}
%\usepackage{gnuplottex}
%\usepackage{tikz}
\usepackage{mathtools}

%\usepackage{graphicx}
\usepackage{hyperref}
%\usepackage{textcomp}

%\usepackage{algpseudocode}
%\usepackage{algorithm}
%\usepackage{algorithmicx}

\begin{document}

%\newtheorem{mytheorem}{Theorem}
%\newtheorem{lemma}{Lemma}
%\newtheorem{mydef}{Definition}

%\algrenewcommand\algorithmicindent{0.5em}
%\algnewcommand\algorithmicswitch{\textbf{switch}}
%\algnewcommand\algorithmiccase{\textbf{case}}
%\algnewcommand\algorithmicassert{\texttt{assert}}
%\algnewcommand\Assert[1]{\State \algorithmicassert(#1)}
% New "environments"
%\algdef{SE}[SWITCH]{Switch}{EndSwitch}[1]{\algorithmicswitch\ #1\ \algorithmicdo}{\algorithmicend\ \algorithmicswitch}
%\algdef{SE}[CASE]{Case}{EndCase}[1]{\algorithmiccase\ #1}{\algorithmicend\ \algorithmiccase}

%\algtext*{EndSwitch}
%\algtext*{EndCase}
%\algtext*{EndWhile}% Remove "end while" text
%\algtext*{EndIf}% Remove "end if" text
%\algtext*{EndFor}% Remove "end for" text
%\algtext*{EndFunction}% Remove "end function" text

%\newif\ifboldnumber
%\newcommand{\boldnext}{\global\boldnumbertrue}

% Default definition is \footnotesize#1:
%\algrenewcommand\alglinenumber[1]{%
%  \footnotesize\ifboldnumber\bfseries\fi\global\boldnumberfalse#1:}


\makeatletter
\def\@copyrightspace{\relax}
\makeatother

% ****************** TITLE ****************************************

\title{Graph parsing: parsing techniques for graph analysis}

% ****************** AUTHORS **************************************

\numberofauthors{1}

\author{
\alignauthor
       Semyon Grigorev\\
       \affaddr{Saint Petersburg State University}\\
       \affaddr{7/9 Universitetskaya nab.}\\
       \affaddr{St. Petersburg, 199034 Russia}\\
       \email{Semen.Grigorev@jetbrains.com}
}

\maketitle

%\title{Generalized LL Parsing Generalization}

%\begin{document}

\maketitle

Nowadays input data for parsing algorithms are not limited to be linear strings, and context-free grammars are used not only for programming languages specification.
One of classical examples is a context-free path querying for graph data bases where input is a graph and path constraints are specified by a grammar.
Solutions are CYK-based~\cite{...}.
There are some open questions.
We are working on it.

Applications: software engeneering (dynamically generated strings analysis), graph data bases, bioinformatics.
An idea, proposed at past parsing workshop is an partial of our case.

Parser-combinators (meercat based), GLL, matrix multiplication (Okhotin-Valiant inspired), conjunctive grammars, mechanisation in coq.

There are also other generalizations of parsing, such as multiple input GLL parsing presented at Parsing@SLE-2016 by Elizabeth Scott and Adrian Johnstone, 
Abstract LR parsing~\cite{AbstractParsing} and other techniques for parsing of dynamically generated strings.

We have some experience in the areas mentioned above~\cite{GraphGLL, RelaxedRNGLR}.
GLL-based context-free path querying algorithm~\cite{GraphGLL} implemented by the authors is faster than solution which was presented at ISWC-2016~\cite{CFRDFParsing}. 

We have some ideas of graph parsing applications.
For example, context-free pattern search in metagenomical assemblies. 
Secondary structure can be specified in terms of grammar (context-free or conjunctive).
Assembley is a graph.

All existing applications seem to be special cases of the Bar-Hillel~\cite{Bar-Hillel} theorem for context-free and regular language intersection, and can be generalized, but today many of them are developed as stand alone solutions.
Thus, the goal of our work is to create an abstract framework for parsing based on generalization of GLL parsing algorithm~\cite{GLL} proposed by Elizabeth Scott and Adrian Johnstone. 

Our plans are 
Advanced matrix multiplication techniques: approximated matrix multiplication, sparse matrix multiplication.
Boolean grammars --- problems with monotonic
Intersection of other types of grammar (start point is Nederhof)
We also want to investigate practical areas of application and to create solutions based on our framework to demonstrate its practical value.

\begin{thebibliography}{9}

\bibitem{Bar-Hillel}
  Bar-Hillel, Yehoshua, Micha Perles, and Eliahu Shamir.
  ``On formal properties of simple phrase structure grammars.''
   \emph{Sprachtypologie und Universalienforschung}
   14 (1961): 143-172.

\bibitem{AbstractParsing}
  Doh, Kyung-Goo, Hyunha Kim, and David A. Schmidt.
  ``Abstract LR-parsing.'',
  \emph{Formal Modeling: Actors, Open Systems, Biological Systems.},
  Springer,
  2011.
  90--109.

\bibitem{GraphGLL}
  Grigorev, Semyon, and Anastasiya Ragozina. 
  ``Context-Free Path Querying with Structural Representation of Result.''
   \emph{arXiv preprint arXiv:1612.08872}
    (2016).

\bibitem{GLL}
  Scott, Elizabeth, and Adrian Johnstone.   
  ``GLL parsing.'',
  \emph{Electronic Notes in Theoretical Computer Science},
  253.7 (2010): 177--189.

\bibitem{RelaxedRNGLR}
  Verbitskaia, Ekaterina, Semyon Grigorev, and Dmitry Avdyukhin.
  ``Relaxed Parsing of Regular Approximations of String-Embedded Languages.''
  \emph{International Andrei Ershov Memorial Conference on Perspectives of System Informatics.}
  Springer International Publishing, 2015.

\bibitem{CFRDFParsing}
  Zhang, Xiaowang, et al.
  ``Context-free path queries on RDF graphs.'' 
  \emph{International Semantic Web Conference.}
   Springer International Publishing, 2016.
   632--648.

\end{thebibliography}

\end{document}
