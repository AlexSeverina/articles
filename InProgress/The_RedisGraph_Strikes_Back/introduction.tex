\section{Introduction}

Language-constrained path querying~\cite{doi:10.1137/S0097539798337716} is a way to find paths in edge-labeled graphs with constraints are formulated in terms of language which restrict words formed by paths: the word formed by path's labels concatenation should be in the specified language.
This way is very natural for navigational queries in graph databases, and one of the most popular languages which are used for such constraints is a regular language. 
But in some cases, regular languages are not expressive enough, as a result, context-free languages gain popularity.
Constraints in the form of context-free languages, or context-free path querying (CFPQ), can be used for RDF analysis~\cite{10.1007/978-3-319-46523-4_38}, biological data analysis~\cite{SubgraphQueriesbyContextfreeGrammars}, static code analysis~\cite{Zheng,10.1145/373243.360208}, and in other areas.

Big amount of research done on CFPQ, a number of CFPQ algorithms were proposed, but the application of context-free constraints for real-world data analysis faced with some problems.
The first problem is a bad performance of proposed algorithms on real-world data, as was shown by Jochem Kuijpers et al.~\cite{Kuijpers:2019:ESC:3335783.3335791}.
The second problem is that there are no graph databases with full-stack support of CFPQ, the main effort was made in algorithms and their theoretical properties research.
This fact hinders research of problems reducible to CFPQ, thus it hinders the development of new solutions for some problems.
For example, recently graph segmentation in data provenance analysis was reduced to CFPQ~\cite{8731467}, but authors faced the problem during the evaluation of the proposed approach: no one graph database support CFPQ.

In~\cite{Azimov:2018:CPQ:3210259.3210264} Rustam Azimov proposed a matrix-based algorithm for CFPQ.
This algorithm is one of promising way to solve the first problem and provide performant solution for real-world data analysis, as was shown by Nikita Mishim et al. in~\cite{Mishin:2019:ECP:3327964.3328503} and Arseniy Terekhov et al. in~\cite{10.1145/3398682.3399163}. 
But this algorithm always computes information (reachability facts or single path which satisfies constraints) for all pairs of vertices in the graph, namely it solves \textit{all-pairs} context-free path querying problem.
Handling of all possible pairs is unreasonable in some real-world scenarios when one can provide a relatively small set of start vertices or even single start vertex. 

While all-pairs context-free path querying is a classical problem that investigates in a number of works, there is no, in our knowledge, solutions for single-source and multiple-source CFPQ.
In this work we propose a matrix-based \textit{multiple-source} (and \textit{single-source} as a partial case) CFPQ algorithm.

To solve the second problem, we provide full-stack support of CFPQ for the RedisGraph\footnote{RedisGraph graph database Web-page: \url{https://redislabs.com/redis-enterprise/redis-graph/}. Access date: 19.07.2020.}~\cite{8778293} graph database.
We implement a Cypher query language extension\footnote{Proposal which describes path patterns specification syntax for Cypher query language: \url{https://github.com/thobe/openCypher/blob/rpq/cip/1.accepted/CIP2017-02-06-Path-Patterns.adoc}. The proposed syntax allows one to specify context-free constraints. Access date: 19.07.2020.} that allows one to express context-free constraints, and extend the RedisGraph to support this extension.
In our knowledge, it is the first full-stack implementation of CFPQ.

To summarize, we make the following contribution in this paper.
\begin{enumerate}
	\item We modify Azimov's matrix-based CFPQ algorithm and provide a multiple-source matrix-based CFPQ algorithm.
	As a partial case, it is possible to use our algorithm in a single-source scenario.
	Our modification still based on linear algebra, hence it is simple to implementation and allows one to use high-performance libraries and utilize modern parallel hardware for queries evaluation. 
	\item We evaluate two version the proposed algorithm: with caching of results which should helps to reduce multiple calculation of the same data, and without caching (na\"{i}ve).
	Our evaluation shows that the na\"{i}ve version is performant than the version with results caching in almost the all cases. Moreover, this version is more memory-efficient. Thus it is good choice for implementation in real-world graph database.
	\item We provide full-stack support of CFPQ by extending the RedisGraph graph database.
	To do it, we extend Cypher with syntax allows one to express context-free constraints, implement the proposed algorithm in a RedisGraph backend, and support new syntax in the RedisGraph query execution engine. Finally, we evaluate the proposed solution and show that it is performant and memory-efficient enough to be applicable for real-world graph querying.
\end{enumerate}