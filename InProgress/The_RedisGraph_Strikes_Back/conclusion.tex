\section{Conclusion}

In this paper we propose a number of multiple-source modifications of Azimov's CFPQ algorithm.
Evaluation of the proposed modifications on the real-world examples shows that !!!!
Finally, we provide the full-stack support of CFPQ.
For our solution we implement corresponding Cypher extension as a part of \texttt{libcypher-parser}, integrate the proposed algorithm into RedisGraph, and extend RedisGraph execution plan builder to support extended Cypher queres.
We demonstrate, that our solution allows one evaluate not only context-free queryes, but also regular one.

In the future, it is necessary to provide formal translation of Cypher to linear algebra, or find a maximal subset of Cypher which can be translated to linear algebra.
There is a number of work on a subset of SPARQL to linear algebra translation, such as~\cite{!!!}, but they are very limited. 
Deep investigation of this topic helps one to realize limits and restrictions of linear algebra utilization for graph databases.
Moreover, it helps to improve existing solutions.

We show that evaluation of regular queryes is possible in practice by using CFPQ algorithm, as far as regular queries is a partial cas of the context-free one.
But it seems, that the proposed solution is not optimal. 
For real-world solutions it is important to provide an optimal unified algorithm for both RPQ and CFPQ.
One of possible way to solve this provlem is to use tensor-based algorithm~\cite{!!!}.

Another imprtant task is to compare non-linear-algebra-based approaches to multiple-source CFPQ with the proposed solution. 
In~\cite{!!!} Johem Kujpers et.al. shows that all-pairs CFPQ algorithms implemented in Neo4j demonstrate unreasonable performance on real-world data for Neo4j.
At the same time, Arseniy Terekhov et.al. shows that matrix-based all-pairs CFPQ algortihm implemented in appropriate linear algebra based graph database (RedisGraph) demonstrates good performance.
But in the case of multiple-source scenario, when a number of sources is relatively small, non-linear-algebra-based solutions can be better, because such solutions naturally handle small reqired subgraph.