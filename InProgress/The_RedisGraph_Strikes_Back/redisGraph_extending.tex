\section{CFPQ Full-Stack Support}

In order to provide full-stack support of CFPQ it is necessatry to choose an appropriatr graph database.
It was shown by Arseniy Terekhov et al. in~\cite{10.1145/3398682.3399163} that matrix-based algorithm can be naturelly integrated into RedisGraph graph database because both, the algorithm and the database, operates over matrix representation of graphs.
Moreover, RedisGraph supports Cypher as a query language and there is a proposal which describes Cypher expetsion which allows one to specify context-free constraints.
Thus we choose RedisGraph as a base for our solution.  


\subsection{Cypher Extending}

The first what we should do is to extend Cypher to be able to express context-free constraints.
There is a description of the respective Cypher syntax extension\footnote{Formal syntax specification: \url{https://github.com/thobe/openCypher/blob/rpq/cip/1.accepted/CIP2017-02-06-Path-Patterns.adoc\#11-syntax}. Access date: 19.07.2020.}, proposed by Tobias Lindaaker, but this syntax does not implement yet in Cypher parsers.

RedisGraph database supports subset of Cypher language and useses \texttt{libcypher-parser}\footnote{The \texttt{libcypher-parser} is an open-source parser library for Cypher query language. GitHub repository of the project: \url{https://github.com/cleishm/libcypher-parser}. Access date: 19.07.2020.} library to parse queries.
We extend this library by intoducing new syntax proposed !!! 
We implement\footnote{The modified libsypher-pareser library with support of syntax for path patterns: \url{https://github.com/YaccConstructor/libcypher-parser}. Access date: 19.07.2020.} full extension, not only part which is necessary for simple CFPQ. 


Main feature which allows one to specify cintext-free constraints is a \textit{named path patterns}: one can specify a name for pattern and after that use it in other patterns, or in the same pattern.
Using this feature, structure of query is pretty similar to context-free grammar.
For exmaple !!!


\subsection{RedisGraph Extending}

CFPQ to matrix expressions, etc. 

Limits, restrictions, etc.

\subsection{Evaluation}

Small basic evalustion on real-world graph (geo?).
In order to show, that performance is reasonable.