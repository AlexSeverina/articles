\section{Introduction}

Language-constrained path querying~\cite{FLCpathProblem}, and particularly Context-Free Path Querying (CFPQ)~\cite{Yannakakis}, allows one to use formal grammars as constraints for paths: concatenation of the labels along the path is treated as a word, and a constraint on the path is a specification of the language which should contain specific words.
CFPQ is widely used for graph-structured data analysis in such domains as biological data analysis, RDF, network analysis.
Huge amount of the real-world data makes performance of CFPQ critical for practical tasks.
Several algorithms for CFPQ based on such parsing techniques as (G)LL, (G)LR, and CYK are proposed recently~\cite{hellingsPathQuerying,Grigorev:2017:CPQ:3166094.3166104,Verbitskaia:2018:PCC:3241653.3241655,RDF,10.1007/978-3-319-91662-0_17,Medeiros:2018:EEC:3167132.3167265}.

Different algorithms provide different functions/abilities for users.
For example, matrix-based and CYK-based algortithms requre grammar to be in CNF.
More over, different behaviour in different cases: different semantics, different query types, etc.
It is necessary to investigate the effect of the specific algorithms and implementation techniques on the performance of CFPQ.


In this work, we do an empirical performance comparison of several implementations of different algorithms for CFPQ on both real-world data and synthetic data for the worst cases.
We make the following contributions in this paper.

\begin{enumerate}
\item We describe and compare !!!
\item We extend a dataset forr CFPQ evaluation.
\item We provide evaluation !!!.
\end{enumerate}
