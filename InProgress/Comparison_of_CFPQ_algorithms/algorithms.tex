\section{CFPQ algorithms}

Algorithms to evaluate and compare


\subsection{LL-based}

Description~\cite{Medeiros:2018:EEC:3167132.3167265}

%Ciro
\subsection{LR-based}

Description~\cite{10.1007/978-3-319-91662-0_17}

\subsection{GLL-based}

Description~\cite{Grigorev:2017:CPQ:3166094.3166104}

\subsection{Combinators-based}

Trails~\cite{Kroni:2013:PGA:2489837.2489844}, Meerkat-based~\cite{Verbitskaia:2018:PCC:3241653.3241655}

%(???)
\subsection{RNGLR-based}

Description~\cite{10.1007/978-3-319-41579-6_22}


\subsection{Matrix-based Algorithm for CFPQ}

Matrix-based algorithm for CFPQ was proposed by Rustam Azimov~\cite{Azimov:2018:CPQ:3210259.3210264}.
This algorithm can be expressed in terms of operations over matrices (see listing~\ref{lst:algo1}), and it is a sufficient advantage for implementation.
It was shown that the utilization of GPGPU improves the context-free path querying performance significantly in comparison to other implementations~\cite{Azimov:2018:CPQ:3210259.3210264} even if float matrices are used instead of boolean matrices.

\begin{algorithm}
  \floatname{algorithm}{Listing}
\begin{algorithmic}[1]
\caption{Context-free path quering algorithm}
\label{lst:algo1}
\Function{contextFreePathQuerying}{D, G}

    \State{$n \gets$ the number of nodes in $D$}
    \State{$E \gets$ the directed edge-relation from $D$}
    \State{$P \gets$ the set of production rules in $G$}
    \State{$T \gets$ the matrix $n \times n$ in which each element is $\varnothing$}
    \ForAll{$(i,x,j) \in E$}
    \Comment{Matrix initialization}
        \State{$T_{i,j} \gets T_{i,j} \cup \{A~|~(A \rightarrow x) \in P \}$}
    \EndFor
    \While{matrix $T$ is changing}

        \State{$T \gets T \cup (T \times T)$}
        \Comment{Transitive closure calculation}
    \EndWhile
\State \Return $T$
\EndFunction
\end{algorithmic}
\end{algorithm}

Here $D = (V, E)$ is the input graph and $G = (N,\Sigma,P)$ is the input grammar.
Each cell of the matrix $T$ contains the set of nonterminals such that $N_k \in T[i,j] \iff \exists p = v_i \ldots v_j $---path in $D$, such that $N_k \xRightarrow[G]{*} \omega(p) $, where $\omega(p)$ is a word formed by the labels along the path $p$.
Thus, this algorithm solves reachability problem, or, according to Hellings~\cite{hellingsRelational}, processes CFPQs by using relational query semantics.

The performance-critical part of the algorithm is matrix multiplication.
Note, that the set of nonterminals is finite, and we can represent the matrix $T$ as a set of boolean matrices: one for each nonterminal.
In this case the operation of matrix update is $T_{N_i} \leftarrow T_{N_i} + (T_{N_j} \times T_{N_k})$ for each production $N_i \rightarrow N_j \ N_k$ in $P$.
Thus we can reduce CFPQ to boolean matrices multiplication.
After such transformation, we can apply the next optimization: we can skip update if the matrices $T_{N_j}$ and $T_{N_k}$ have not been changed at the previous iteration.

Thus, the most important part is the efficient implementation of operations over boolean matrices.
In this paper, we compare the effects of different approaches to matrices multiplication.
All our implementations are based on the optimized version of the algorithm.


\subsection{Other solutions on CFPQ}

Other algorithms which are described by other groups, but not evaluated in our work.
OpenCypher, Bradford, RDF-CYK, Hellings, Earley stc. 