% Use:
% pdflatex conj_path_querying
% biber conj_path_querying
% pdflatex conj_path_querying
\documentclass [a4paper] {article}

% ----------------------------------------------------------------
% Required packages

\usepackage [T2A] {fontenc}
\usepackage [utf8] {inputenc}
\usepackage [english, russian] {babel}

\usepackage {url}
\usepackage [style = gost-numeric,backend=biber] {biblatex}

% ----------------------------------------------------------------
% Optional packages

\usepackage {lipsum}

\addbibresource {conj_path_querying.bib}

\begin{document}
	\title{Синтаксический анализ графов с использованием конъюнктивных грамматик}
	\author{
		Азимов~Р.\,Ш., \\ \url {rustam.azimov19021995@gmail.com}, \\
		Санкт-Петербургский государственный университет, \\
		Лаборатория языковых инструментов JetBrains 
	}
	
	\maketitle
	
	\begin{abstract}
		Графы используются в качестве структуры данных во многих областях, например, биоинформатика, графовые базы данных. В этих областях часто необходимо вычислять некоторые запросы к большим графам. Один из наиболее распространенных запросам к графам являются навигационные запросы. Результатом вычисления таких запросов является множество неявных отношений между вершинами графа, то есть путей в графе. Естественно выделять такие отношения --- пометив ребра графа символами из некоторого конечного алфавита и выделив необходимые пути в графе с помощью формальных грамматик над тем же алфавитом. Наиболее популярны запросы, использующие контекстно-свободные грамматики. Ответом на такие запросы обычно является множество всех троек $(A, m, n)$, для которых существует путь в графе от вершины $m$ до вершины $n$ такой, что метки на ребрах этого пути образуют строку, выводимою из нетерминала данной контекстно-свободной грамматики $A$. Говорят, что такой тип запросов вычислен с использованием \textit{реляционной семантики запросов}. Кроме того, существуют \textit{конъюнктивные грамматики}, образующие более широкий класс грамматик, чем контекстно-свободные. Использование конъюнктивных грамматик в задаче синтаксического анализа графов позволит формулировать более сложные запросы к графу и решать более широкий круг задач. Известно, что задача вычисления запросов к графу с использованием реляционной семантики и конъюнктивных грамматик --- неразрешима. В данной работе будет предложен алгоритм, вычисляющий приближенное решение данной задачи, а именно аппроксимацию сверху множества троек $(A, m, n)$. Предложенный алгоритм основан на матричных операциях, что позволяет повысить производительность, используя вычисления на графическом процессоре.
		
		\vspace{1em}
		\textbf{Ключевые слова: синтаксический анализ графов, конъюнктивные грамматики, транзитивное замыкание, матричные операции, вычисления на GPU}  
	\end{abstract}
	
	\section{Введение}
	Графы используются в качестве структуры данных во многих областях, например, биоинформатика~\cite{Bio}, графовые базы данных~\cite{graphDB}. В этих областях часто необходимо вычислять некоторые запросы к большим графам. Один из наиболее распространенных запросам к графам являются навигационные запросы. Результатом вычисления таких запросов является множество неявных отношений между вершинами графа, то есть путей в графе. Естественно выделять такие отношения --- пометив ребра графа символами из некоторого конечного алфавита и выделив необходимые пути в графе с помощью формальных грамматик (регулярные выражения, контекстно-свободные грамматики) над тем же алфавитом. Наиболее популярны запросы, использующие контекстно-свободные грамматики, так как КС-языки обладают большей выразительной мощностью, чем регулярные.
	
	Ответом на запросы с использованием КС-грамматик обычно является множество всех троек $(A, m, n)$, для которых существует путь в графе от вершины $m$ до вершины $n$ такой, что метки на ребрах этого пути образуют строку, выводимою из нетерминала $A$ данной КС-грамматики. Говорят, что такой тип запросов вычислен с использованием \textit{реляционной семантики запросов}~\cite{hellingsRelational}. 
	
	Существует ряд алгоритмов синтаксического анализа графов с использование реляционной семантики запросов и КС-грамматик~\cite{hellingsRelational, RDF, GraphQueryWithEarley}. Данные алгоритмы демонстрируют низкую производительность на больших графах. Одной из самых популярных техник, используемых для увеличения производительности при работе с большими объемами данных, является использование графического процессора для вычислений, но перечисленные алгоритмы не позволяют эффективно применить данную технику.
	
	Кроме того, существует алгоритм синтаксического анализа графов с использование реляционной семантики запросов и КС-грамматик, вычисляющий матричное транзитивное замыкание. Активное использование матричных операций в данном алгоритме позволяет эффективно использовать вычисления на графическом процессоре~\cite{matricesOnGPGPU}. 
	
	Также существуют \textit{конъюнктивные грамматики}~\cite{okhotin2001conjunctive}, образующие более широкий класс грамматик, чем контекстно-свободные. Использование конъюнктивных грамматик в задаче синтаксического анализа графов позволит формулировать более сложные запросы к графу и решать более широкий круг задач. Известно, что задача вычисления запросов к графу с использованием реляционной семантики и конъюнктивных грамматик --- неразрешима~\cite{hellingsRelational}. Один из распространенных способов найти приближенное решение неразрешимой задачи --- найти аппроксимацию решения (сверху или снизу).
	
	В данной работе будет предложен алгоритм, вычисляющий приближенное решение задачи синтаксического анализа графов с использованием реляционной семантики запросов и конъюнктивных грамматик, а именно аппроксимацию сверху множества троек $(A, m, n)$. Предложенный алгоритм основан на матричных операциях, что позволяет повысить производительность, используя вычисления на графическом процессоре.
	\section{Обзор}
	\section{Существующие работы}
	\section{Определения}
	\section{Сведение синтаксического анализа графов к поиску транзитивного замыкания}
	\section{Алгоритм}
	\section{Апробация}
	\section{Заключение}
	
	
	\printbibliography
	
\end{document} 
