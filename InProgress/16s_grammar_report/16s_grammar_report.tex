\documentclass[12pt]{article}  % standard LaTeX, 12 point type
\usepackage{amsfonts,latexsym}
\usepackage{amsthm}
\usepackage{amssymb}
\usepackage[utf8]{inputenc} % Кодировка
\usepackage[english,russian]{babel} % Многоязычность
\usepackage{algpseudocode}
\usepackage{algorithm}
\usepackage{algorithmicx}

\newtheorem{theorem}{Theorem}[section]
\newtheorem{proposition}[theorem]{Proposition}
\newtheorem{lemma}[theorem]{Lemma}
\newtheorem{corollary}[theorem]{Corollary}
\newtheorem{conjecture}[theorem]{Conjecture}

\theoremstyle{definition}
\newtheorem{definition}{Определение}[section]
\newtheorem{example}{Example}[section]

% unnumbered environments:

\theoremstyle{remark}
\newtheorem*{remark}{Remark}
\newtheorem*{notation}{Notation}
\newtheorem*{note}{Note}

\setlength{\parskip}{5pt plus 2pt minus 1pt}
%\setlength{\parindent}{0pt}

\usepackage{color}
\usepackage{listings}
\usepackage{caption}
\usepackage{graphicx}
\usepackage{ucs}

\graphicspath{{pics/}}



\title{Использование КС-грамматика для распознования центрального домена 16s}
\author{Semyon Grogorev}
\date{\today}

\begin{document}

\algnewcommand\algorithmicswitch{\textbf{switch}}
\algnewcommand\algorithmiccase{\textbf{case}}
\algnewcommand\algorithmicassert{\texttt{assert}}
\algnewcommand\Assert[1]{\State \algorithmicassert(#1)}
% New "environments"
\algdef{SE}[SWITCH]{Switch}{EndSwitch}[1]{\algorithmicswitch\ #1\ \algorithmicdo}{\algorithmicend\ \algorithmicswitch}
\algdef{SE}[CASE]{Case}{EndCase}[1]{\algorithmiccase\ #1}{\algorithmicend\ \algorithmiccase}

\algtext*{EndSwitch}
\algtext*{EndCase}
\algtext*{EndWhile}% Remove "end while" text
\algtext*{EndIf}% Remove "end if" text
\algtext*{EndFor}% Remove "end for" text
\algtext*{EndFunction}% Remove "end function" text

\maketitle 

\section{Введение}

Вторичная структура достаточно богата.
Более того, известно, что некоторые участки обладают достаточно консервативной вторичной структурой.

Грамматика позволяет минимизировать знания о первичной структуре.
Поиск структурного шаблона.
Грамматика задаётся вручную, но возможен и вывод грамматика, но это тема для отдельного исследования.

\section{Грамматика}

Язык YARD.
Примеры конструкций, которые используются при описании грамматики 16s.

Используемая грамматика.

Огромный листинг с грамматикой.

\section{Эксперименты}

Два эксперимента: обработка баз известных 16s, обработка полноразмерных геномов.

Базы 16s: все нитки должны быть распознаны.

Базы размеченных полноразмерных геномов с информацией о 16s: оценить точность, полноту и т.д. (сколько из отмеченных найдено, сколько из отмеченных не найдено, сколько найдено неотмеченных).
Проанализировать ложные срабатывания и пропущенных кандидатов.

\end{document}