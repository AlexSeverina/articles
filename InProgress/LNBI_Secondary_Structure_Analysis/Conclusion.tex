\section{Conclusion}
We describe the modifications of the proposed approach~\cite{grigorevcomposition} for biological sequences analysis using the combination of formal grammars and neural networks.
We show that it is possible to improve the quality of the solution by representing parsing result as an image and handleing it by using convolutional layers while processing it with a neural network.
Also, we provide a technique that removes the parsing step from the trained model use and allows to run models on the original RNA sequences.
As a result, the performance of the solution is significantly improved.
We demonstrate the applicability of the proposed modifications for real-world problems\footnote{
Project description is available at the project page: \url{https://research.jetbrains.org/groups/plt\_lab/projects?project\_id=43}.
Source code and documentation are published at GitHub: \url{https://github.com/LuninaPolina/SecondaryStructureAnalyzer}. Access date: 07.03.2020}.

We can provide several directions for future research.
First of all, it is necessary to investigate the applicability of the proposed approach for other sequences processing tasks such as 16s rRNA processing and chimeric sequences filtration.

Another possible application is a secondary structure prediction.
We plan to investigate the possibility of creating network which generates the most possible contact map for the given sequence.
It is necessary to compare this approach with both calssical aproaches and tools to secondary structure prediction, such as IPKnot~\cite{10.1093/bioinformatics/btr215}, and artificial neural network based ones~\cite{Lu2019,Singh2019}.

The image-based model demonstrates a higher quality.
We believe that it is caused by a better locality of features.
If so, it should be possible to create a deep convolutional network for secondary structure analysis: further investigation is needed.

Finally, it is important to find a theoretical base for grammar tuning.
It is important to adopt the theoretical results on secondary structure description by using formal grammar, such as~\cite{MQbioinformatics19} to find the optimal grammar for our approach.
