\section{\bf Introduction}

Secondary structure of genomic sequences prediction plays important role in classification and recognition problems. It comes from the idea that secondary structure is a powerful source of information  related with different biological functions of various organisms.

Some approaches connected with secondary structure analysis based on using formal grammars, as they are successful in modeling strings with correlated symbols~\cite{knudsen1999rna, dowell2004evaluation}.
That means the specific features of secondary structure can be described by some context-free grammar (CFG) and the prediction problem can be reduced to parsing---verification if some sequence can be derived in this grammar.
But checking the derivability is not frequently the main problem, sometimes all the derivable subsequences must be found~\cite{durbin1996biological}.

The main disadvantage of CFG-based approaches is considerable problems with computational complexity.
Traditional parsing method which is used in these approaches is CYK~\cite{kasami1966efficient, Younger:1966:CLP:1441427.1442019} with cubic-time complexity, but this algorithm demonstrates poor performance on long strings or big grammars~\cite{liu2005parallel}.
And so, as such field of application as bioinformatics requires working with a large amount of data, it is necessary to find more efficient parsing algorithms.

Still asymptotically most efficient parsing algorithm is based on matrix multiplication Valiant's algorithm~\cite{Valiant:1975:GCR:1739932.1740048}.
Moreover, Okhotin generalized this algorithm to conjunctive and Boolean grammars which are the natural extensions of CFG with more expressive power~\cite{Okhotin:2014:PMM:2565359.2565379}.
For example, it is possible to express pseudoknots by using conjunctive grammars~\cite{zier2013rna}, while it is impossible by using context-free one.
Valiant’s algorithm allows to simply utilize parallel techniques to improve performance by offloading critical computations onto matrices multiplication.
However, this algorithm is not appropriate for finding substrings problem.

In this paper we present the modification of Valiant's algorithm, which increase the power of using GPGPU and parallel computations by computing some matrices products concurrently.
Also proposed algorithm can be easily adopted for the string-matching, or substring finding, problem.