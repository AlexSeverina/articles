\section{\bf Introduction}

The secondary structure of such genomic sequences as RNAs is related to biological functions of various organisms.
Thus, the analysis of the secondary structure of genomic sequences plays an important role in organisms classification and recognition problems.

Specific features of secondary structure can be described by some context-free grammar (CFG).
As a result, there is a number of approaches to sequences analysis which utilize parsing: verification whether some sequence can be derived in the specified grammar.
For example, some approaches to secondary structure analysis model string with correlated symbols with probabilistic formal grammars~\cite{knudsen1999rna, dowell2004evaluation}.
For some problems, it is necessary to find all derivable substrings of the given string~\cite{durbin1996biological}.
This case is similar to the string-matching problem, but template, or pattern, is described by using formal grammar.

Most CFG-based approaches suffer the same issue: the computational complexity is bad.
Traditionally used CYK~\cite{kasami1966efficient, Younger:1966:CLP:1441427.1442019} runs with a cubic time complexity, demonstrates poor performance on long strings or big grammars~\cite{liu2005parallel}.
We argue that in such field as bioinformatics where a large amount of data is usual, more efficient algorithms are needed.
Moreover, in some cases, context-free grammars are not enough, and more expressive grammars are required.
For example, it is possible to express pseudoknots by using conjunctive grammars~\cite{zier2013rna}, while it is impossible by using context-free one.

Asymptotically most efficient parsing algorithm is Valiant's algorithm~\cite{Valiant:1975:GCR:1739932.1740048} which is based on matrix multiplication.
Moreover, Okhotin generalized this algorithm to conjunctive and Boolean grammars which are the natural extensions of CFG with more expressive power~\cite{Okhotin:2014:PMM:2565359.2565379}.
Valiant’s algorithm allows to simply utilize parallel techniques to improve performance by offloading critical computations onto matrices multiplication.
However, this algorithm is not appropriate for the string-matching problem.

In this paper we present the modification of Valiant's algorithm, which increases the power of using GPGPU and parallel computations by computing some matrices products concurrently.
Also, the proposed algorithm can be easily utilized for the string-matching, or string-searching, problem.
