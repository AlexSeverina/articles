\section{Related works} \label{section_related}
Our work is inspired by Valiant~\cite{valiant}, who proposed an algorithm for general context-free recognition in less than cubic time. This algorithm computes the same parsing table as the Cocke-Kasami-Younger algorithm~\cite{kasami, younger} but does this by offloading the most intensive computations into calls to a Boolean matrix multiplication procedure. This approach not only provides an asymptotically more efficient algorithm but it also allows to effectively apply GPGPU computing techniques. Valiant used the following definition of a transitive closure.

\begin{mydef}
The transitive closure of a square matrix $a$ is a matrix $a^+ = a^{(1)} \cup a^{(2)} \cup \cdots$ where:
\begin{itemize}
    \item $a^{(i)} = \bigcup^{i-1}_{j=1}{a^{(j)} \cdot a^{(i - j)}}$, $i \ge 2$;
    \item $a^{(1)} = a$.
\end{itemize}
\end{mydef}

Valiant also showed that the matrix multiplication operation used in this approach is computationally no more difficult than a Boolean matrix multiplication. Denote the number of elementary operations executed by the algorithm of multiplying $n \times n$ Boolean matrices as $BMM(n)$. Valiant showed that the matrix multiplication operation used in this approach is essentially the same as $|N|^2$ Boolean matrix multiplications, where $|N|$ is the number of non-terminals of the given context-free grammar in Chomsky normal form.

Hellings~\cite{hellingsRelational} presented an algorithm for query evaluation using the relational query semantics and context-free grammars. According to Hellings, for a given graph $D = (V, E)$ and a grammar $G = (N, \Sigma, P)$ the query evaluation using the relational query semantics reduces to a calculation of the relations $R_A$. Thus, in this paper, we focus on the calculation of these relations.

Yannakakis~\cite{transitive-closure} analyzed the reducibility of various graph parsing problems to the calculation of the transitive closure. He formulated a problem of generalization Valiant's technique to the query evaluation using the relational query semantics and context-free grammars. Also, he assumed that this technique can not be generalized for arbitrary graphs, though it does for acyclic graphs.

Thus, the possibility of reducing the query evaluation using the relational query semantics and context-free grammars to the calculation of the transitive closure is an open problem. In this paper, we do not generalize Valiant's approach. We use a different definition of the transitive closure. The possibility of using Valiant's transitive closure in graph parsing is an open problem.