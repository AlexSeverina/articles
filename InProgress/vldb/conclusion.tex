\section{Conclusion and future work}
In this paper, we presented the algorithm for reducing graph query evaluation using the relational query semantics to the calculation of matrix transitive closure. Also, we provide a formal proof of the correctness of the proposed reduction. In addition, we introduce an algorithm for computing this transitive closure, which allows to effectively apply GPGPU computing techniques. Finally, we show the practical applicability of the proposed algorithm by running different implementations of our algorithm on real-world data.

We identify several open problems for further research. In this paper we have considered only one semantics of graph querying but there are other important semantics, such as \textit{single-path} and \textit{all-path} semantics~\cite{hellingsPathQuerying}, which require to present paths, not only check reachability. Graph parsing implemented with algorithm~\cite{GLL} can answer the queries in these semantics by parsing a forest construction. It is possible to construct a parsing forest for a linear input parsing by the matrix multiplication~\cite{okhotin_cyk}. Whether it is possible to generalize this approach for a graph input is an open question.

In our algorithm, we calculate the matrix transitive closure naively, but there are algorithms for the transitive closure calculation, which are asymptotically more efficient. Therefore, the question is whether it is possible to apply these algorithms for the matrix transitive closure calculation to the problem of graph parsing. One way to answer this question is to generalize the Valiant's technique to arbitrary graphs. Yannakakis~\cite{transitive-closure} formulated this problem and assumed that Valiant's technique does not seem to generalize to arbitrary graphs.

Also, there are Boolean grammars~\cite{okhotinBoolean}, which have more expressive power than context-free grammars. Graph parsing with boolean grammars is undecidable problem~\cite{hellingsRelational} but our algorithm can be trivially generalized to work on boolean grammars because parsing with boolean grammars can be expressed by matrix multiplication~\cite{okhotin_cyk}. It is not clear, what will be a result of our algorithm applied to Boolean grammars. Our hypothesis is that it will produce the upper approximation of a solution.

Matrix multiplication in the main loop of the proposed algorithm may be performed on different GPGPU independently. It can help to utilize the power of multi-GPU systems and increase the performance of graph parsing.