\section{Preliminaries} \label{section_preliminaries}
In this section, we introduce the basic notions used throughout the paper.

Let $\Sigma$ be a finite set of edge labels. Define an \textit{edge-labeled directed graph} as a tuple $D = (V, E)$ with $V$ is a set of nodes and $E \subseteq V \times \Sigma \times V$ is a directed edge-relation.  For a path $\pi$ in graph $D$ we denote $l(\pi)$ --- the unique word obtained by concatenating the labels of the edges along the path $\pi$. Also, we write $n \pi m$ to indicate that a path $\pi$ starts at node $n \in V$ and ends at node $m \in V$.

According to Hellings~\cite{hellingsRelational}, we deviate from the usual definition of a context-free grammar in \textit{Chomsky Normal Form}~\cite{chomsky} by not including a special start non-terminal, which will be specified in the queries to the graph. Since every context-free grammar can be transformed into an equivalent one in Chomsky Normal Form and checking that an empty string is in the language is trivial, then it is sufficient to only consider grammars of the following type. A \textit{context-free grammar} is 3-tuple $G = (N, \Sigma, P)$ where $N$ is a finite set of non-terminals, $\Sigma$ is a finite set of terminals, and $P$ is a finite set of productions of the following forms:

\begin{itemize}
    \item $A \rightarrow B C$, for $A,B,C \in N$,
    \item $A \rightarrow x$, for $A \in N$ and $x \in \Sigma$.   
\end{itemize}

Note that we omit rules of the form $A \rightarrow \varepsilon$, where $\varepsilon$ denotes an empty string.  This does not limit the applicability of further algorithms because checking that an empty string belongs to the context-free language in Chomsky normal form is trivial.

We use the conventional notation $A \xrightarrow{*} w$ to denote that the string $w \in \Sigma^*$ can be derived from a non-terminal $A$ by some sequence of applying the production rules from $P$. The \textit{language} of a grammar $G = (N,\Sigma,P)$ with respect to a start non-terminal $S \in N$ is defined by $L(G_S) = \{w \in \Sigma^*~|~S \xrightarrow{*} w\}$.

For a given graph $D = (V, E)$ and a context-free grammar $G = (N, \Sigma, P)$, we define \textit{context-free relations} $R_A \subseteq V \times V$, for every $A \in N$, such that $R_A = \{(n,m)~|~\exists n \pi m~(l(\pi) \in L(G_A))\}$.

We define a binary operation on arbitrary subsets $N_1 , N_2$ of $N$ with respect to a context-free grammar $G = (N, \Sigma, P)$ as $N_1 \cdot N_2 = \{A~|~\exists B \in N_1, \exists C \in N_2 \text{ such that }(A \rightarrow B C) \in P\}.$

Using this binary operation as a multiplication on arbitrary subsets of $N$ and union of sets as an addition, we can define \textit{matrix multiplication}, $a \cdot b = c$, where $a$ and $b$ are matrices of suitable size that have subsets of $N$ as elements, as $c_{i,k} = \bigcup^{n}_{j=1}{a_{i,j} \cdot b_{j,k}}$.

We define the \textit{transitive closure} of a square matrix $a$ as $a^+ = a^{(1)} \cup a^{(2)} \cup \cdots$ where $a^{(i)} = a^{(i-1)} \cup (a^{(i-1)} \cdot a^{(i-1)})$, $i \ge 2$ and $a^{(1)} = a$.
