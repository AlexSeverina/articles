\section{Introduction}

The secondary structure of RNA's is tightly related to biological functions of organisms.
So its analysis plays an important role in organisms classification and recognition problems.

Specific features of secondary structure can be described by some context-free grammar (CFG).
There is a number of approaches to sequences analysis based on parsing: verification whether the given sequence can be derived in the specified grammar.
Some approaches to secondary structure analysis model a string with correlated symbols with probabilistic formal grammars~\cite{knudsen1999rna,dowell2004evaluation}.
For some problems, it is necessary to find all derivable substrings of the given string~\cite{durbin1996biological}.
This case is the string-matching problem also known as string-searching problem.

Most CFG-based approaches suffer the same issue: the computational complexity is poor.
Traditionally used CYK~\cite{kasami1966efficient,Younger:1966:CLP:1441427.1442019} runs with a cubic time complexity and demonstrates poor performance on long strings or big grammars~\cite{liu2005parallel}.
We argue that more efficient algorithms are needed in such field as bioinformatics where a large amount of data is common.
In some cases, context-free grammars are not expressive enough.
Fortunately, some features can be expressed with more exotic grammar classes:  for example, pseudoknots can be described by using conjunctive grammars~\cite{zier2013rna}, while it is impossible by using context-free one.

Asymptotically most efficient parsing algorithm is Valiant's algorithm~\cite{Valiant:1975:GCR:1739932.1740048} which is based on matrix multiplication.
Okhotin generalized this algorithm to conjunctive and Boolean grammars which are the natural extensions of CFG which have more expressive power~\cite{Okhotin:2014:PMM:2565359.2565379}.
Valiant’s algorithm simplifies the utilization of parallel techniques to improve performance by offloading critical computations onto matrices multiplication.
However, this algorithm is not suitable for the string-matching problem.

In this paper we present the modification of Valiant's algorithm, which improves utilization of GPGPU and parallel computations by computing some matrices products concurrently.
Also, the proposed algorithm can be easily utilized for the string-matching problem.
We also prove the correctness of our algorithm and analyze its time complexity.
The performance of the proposed solution was evaluated using fast matrix multiplication algorithms and parallel techniques.
