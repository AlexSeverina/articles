\section{Introduction}

The secondary structure of RNA's is tightly related to the biological functions of organisms and plays an important role in classification and recognition problems.
One of the approaches to analyze RNA's secondary structure is based on formal language methods.
Namely, one can process RNA sequence as a string over 4-letter alphabet $\{G,A,C,U\}$ and use formal language methods to describe properties of this string and parsing methods to analyze strings w.r.t described properties. 

One of the most popular languages in this area is a context-free languags (CFL), and related probabilistic context-free grammars which widely used for secondary structure description and related tasks~\cite{knudsen1999rna,dowell2004evaluation}.
But more powerful languages are required to describe some important features of secondary structure. For example, pseudoknots can not be expressed in terms of context-free grammar, but can be expressed by using conjunctive grammars~\cite{zier2013rna} which proposed by Alexander Okhotin~\cite{10.5555/543313.543323} and are the natural extensions of CFG. 

For some problems, it is necessary to find all derivable substrings of the given string~\cite{durbin1996biological}.
This case is the string-matching problem also known as a string-searching problem.
The classical example of such a problem is to find substrings satisfied with the given regular expression (or regular template). 
But if one tries to find substring with a specific secondary structure, then it is necessary to use at least context-free template (context-free grammar) and, as a result, utilize respective parsing algorithm. 

Most CFG-based approaches suffer the same issue: the computational complexity is poor.
Traditionally used CYK~\cite{kasami1966efficient,Younger:1966:CLP:1441427.1442019} runs with a cubic time complexity and demonstrates poor performance on long strings or big grammars~\cite{liu2005parallel}.
We argue that more efficient algorithms are needed in such a field as bioinformatics where a large amount of data is common.

Asymptotically most efficient parsing algorithm is Valiant's algorithm~\cite{Valiant:1975:GCR:1739932.1740048} which is based on matrix multiplication.
Okhotin generalized this algorithm to conjunctive and Boolean grammars~\cite{Okhotin:2014:PMM:2565359.2565379}. 
Moreover, in comparison to CYK, Valiant’s algorithm simplifies the utilization of parallel techniques to improve performance by offloading critical computations onto matrices multiplication.
However, this algorithm is not suitable for the string-matching problem.

In this paper we present the modification of Valiant's algorithm, which improves the utilization of GPGPU and parallel computations by processing some submatrices products concurrently.
Also, the proposed algorithm can be easily utilized for the string-matching problem.
We also prove the correctness of our algorithm and analyze its time complexity.
The performance of the proposed solution was evaluated using fast matrix multiplication algorithms and parallel techniques.
