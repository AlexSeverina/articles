\section{introduction}

Modern tools for development (IDEs, language services, etc) provide much more information ``on the fly''---during code typing.
The problem is that it is necessary to build parsing tree to get such information.
But code not contimiously correct.
To be able to provide all possible information even for parlially incorrect code syntax error recovery mechanism is requred.

General goal of error recovery is to find syntactically correc string which a close to the given string as mach as possible, and provide parsing result of this string.
In this terms error recovery problem is a language editing distance problem: for the given language and the given string it is necessary to find string which is in language such that editing distance between this atring and the given is minimal.

On the other hand, one can treet input as a linear graph where token is an edge.
Similar view, for exmaple, was proposed by Jonstone at Parsing@SLE !!! on multivariant tokenization parsing.
After thet one can precompute all possible editing steps statically and add appropriate edges into input graph: epsilon edges for delition, parallel edges with tokens for repacement, and loop edges for insertion.
Additional edges should be weighed alike in classical editing problems.
After that the problem is to find shortest path from start vertex to final such that string along this path is in the given language.

Williams on cliques --- CFL distance to clique.
Static graph.
Can we do it in dynamic?

GLL~\cite{scott2010gll} for CFPQ with structural representation of result~\cite{GrigorevR16}.

\begin{enumerate}
\item bridging the gap between ....
\item We propose the way to utilize CFPQ algorithms
\item We evaluate ...
\end{enumerate}
