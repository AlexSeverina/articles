% This is "sig-alternate.tex" V1.9 April 2009
% This file should be compiled with V2.4 of "sig-alternate.cls" April 2009
%
% This example file demonstrates the use of the 'sig-alternate.cls'
% V2.4 LaTeX2e document class file. It is for those submitting
% articles to ACM Conference Proceedings WHO DO NOT WISH TO
% STRICTLY ADHERE TO THE SIGS (PUBS-BOARD-ENDORSED) STYLE.
% The 'sig-alternate.cls' file will produce a similar-looking,
% albeit, 'tighter' paper resulting in, invariably, fewer pages.
%
% ----------------------------------------------------------------------------------------------------------------
% This .tex file (and associated .cls V2.4) produces:
%       1) The Permission Statement
%       2) The Conference (location) Info information
%       3) The Copyright Line with ACM data
%       4) NO page numbers
%
% as against the acm_proc_article-sp.cls file which
% DOES NOT produce 1) thru' 3) above.
%
% Using 'sig-alternate.cls' you have control, however, from within
% the source .tex file, over both the CopyrightYear
% (defaulted to 200X) and the ACM Copyright Data
% (defaulted to X-XXXXX-XX-X/XX/XX).
% e.g.
% \CopyrightYear{2007} will cause 2007 to appear in the copyright line.
% \crdata{0-12345-67-8/90/12} will cause 0-12345-67-8/90/12 to appear in the copyright line.
%
% ---------------------------------------------------------------------------------------------------------------
% This .tex source is an example which *does* use
% the .bib file (from which the .bbl file % is produced).
% REMEMBER HOWEVER: After having produced the .bbl file,
% and prior to final submission, you *NEED* to 'insert'
% your .bbl file into your source .tex file so as to provide
% ONE 'self-contained' source file.
%
% ================= IF YOU HAVE QUESTIONS =======================
% Questions regarding the SIGS styles, SIGS policies and
% procedures, Conferences etc. should be sent to
% Adrienne Griscti (griscti@acm.org)
%
% Technical questions _only_ to
% Gerald Murray (murray@hq.acm.org)
% ===============================================================
%
% For tracking purposes - this is V1.9 - April 2009

\documentclass{sig-alternate-05-2015}
  \pdfpagewidth=8.5truein
  \pdfpageheight=11truein

\usepackage{verbatim}
\usepackage{graphicx}
\usepackage{subcaption}
\usepackage{hyperref}
\usepackage{listings}
\usepackage{courier}
\usepackage{epstopdf}

% \lstset{language=[Sharp]C}
\lstset{numbers=left,xleftmargin=3em,numberstyle=\footnotesize\ttfamily,captionpos=b}
\lstset{basicstyle=\footnotesize\ttfamily}

\begin{document}
\setcopyright{acmlicensed}
%
% --- Author Metadata here ---
% \conferenceinfo{SAC'15}{April 13-17, 2015, Salamanca, Spain.}
% \CopyrightYear{2015} % Allows default copyright year (2002) to be over-ridden - IF NEED BE.
% \crdata{978-1-4503-3196-8/15/04}  % Allows default copyright data (X-XXXXX-XX-X/XX/XX) to be over-ridden.
% --- End of Author Metadata ---

\title{Error Recovery By Using Graph Parsing}
% \subtitle{[Extended Abstract]
% \titlenote{A full version of this paper is available as
% \textit{Author's Guide to Preparing ACM SIG Proceedings Using
% \LaTeX$2_\epsilon$\ and BibTeX} at
% \texttt{www.acm.org/eaddress.htm}}}
%
% You need the command \numberofauthors to handle the 'placement
% and alignment' of the authors beneath the title.
%
% For aesthetic reasons, we recommend 'three authors at a time'
% i.e. three 'name/affiliation blocks' be placed beneath the title.
%
% NOTE: You are NOT restricted in how many 'rows' of
% "name/affiliations" may appear. We just ask that you restrict
% the number of 'columns' to three.
%
% Because of the available 'opening page real-estate'
% we ask you to refrain from putting more than six authors
% (two rows with three columns) beneath the article title.
% More than six makes the first-page appear very cluttered indeed.
%
% Use the \alignauthor commands to handle the names
% and affiliations for an 'aesthetic maximum' of six authors.
% Add names, affiliations, addresses for
% the seventh etc. author(s) as the argument for the
% \additionalauthors command.
% These 'additional authors' will be output/set for you
% without further effort on your part as the last section in
% the body of your article BEFORE References or any Appendices.

\numberofauthors{3} %  in this sample file, there are a *total*
% of EIGHT authors. SIX appear on the 'first-page' (for formatting
% reasons) and the remaining two appear in the \additionalauthors section.
%
\author{
% You can go ahead and credit any number of authors here,
% e.g. one 'row of three' or two rows (consisting of one row of three
% and a second row of one, two or three).
%
% The command \alignauthor (no curly braces needed) should
% precede each author name, affiliation/snail-mail address and
% e-mail address. Additionally, tag each line of
% affiliation/address with \affaddr, and tag the
% e-mail address with \email.
%
% 1st. author
% 1st. author
\alignauthor Marat Khabibullin\\
       \affaddr{St. Petersburg Academic University}\\
       \affaddr{194021, Khlopina Str 8/3}\\
       \affaddr{St. Petersburg, Russia}\\
       \email{maratx387@gmail.com}
% 2nd. author       
\alignauthor Andrei Ivanov\\ 
       \affaddr{St. Petersburg State University}\\
       \affaddr{198504, Universitetsky prospekt 28}\\
       \affaddr{Peterhof, St. Petersburg, Russia}\\
       \email{ivanovandrew2004@gmail.com}
\and
% 3rd. author
\alignauthor Semyon Grigorev\\ 
       \affaddr{St. Petersburg State University}\\
       \affaddr{198504, Universitetsky prospekt 28}\\
       \affaddr{Peterhof, St. Petersburg, Russia}\\
       \email{rsdpisuy@gmail.com}
}
% There's nothing stopping you putting the seventh, eighth, etc.
% author on the opening page (as the 'third row') but we ask,
% for aesthetic reasons that you place these 'additional authors'
% in the \additional authors block, viz.
% \additionalauthors{Additional authors: John Smith (The
% Th{\o}rv{\"a}ld Group, email: {\texttt{jsmith@affiliation.org}})
% and Julius P.~Kumquat (The Kumquat Consortium, email:
% {\texttt{jpkumquat@consortium.net}}).}
\date{27 July 2018}
% Just remember to make sure that the TOTAL number of authors
% is the number that will appear on the first page PLUS the
% number that will appear in the \additionalauthors section.

\maketitle

\begin{abstract}

Abstract is very abstract. Abstract is very abstract. Abstract is very abstract. Abstract is very abstract.
Abstract is very abstract. Abstract is very abstract. Abstract is very abstract. Abstract is very abstract.
Abstract is very abstract. Abstract is very abstract. Abstract is very abstract. Abstract is very abstract.
Abstract is very abstract. Abstract is very abstract. Abstract is very abstract. Abstract is very abstract.
Abstract is very abstract. Abstract is very abstract. Abstract is very abstract. Abstract is very abstract.
Abstract is very abstract. Abstract is very abstract. Abstract is very abstract. Abstract is very abstract.
Abstract is very abstract. Abstract is very abstract. Abstract is very abstract. Abstract is very abstract. 
Abstract is very abstract. Abstract is very abstract. Abstract is very abstract. Abstract is very abstract.
Abstract is very abstract. Abstract is very abstract. Abstract is very abstract. Abstract is very abstract.

\end{abstract}

\printccsdesc

\keywords{!!!!!}

\section{Introduction}

Foundation in some areas: graphs, code analysis, etc.
Why is it important to proof B-H in Coq?
Bar-Hillel theorem is a main on �.
Short overview of current results.

\section{Error Recovery Algorithm}


Additional edges with error markers goes forward and with all tokens, goes in the its start vertex
(as a result we have loops).

\begin{tikzpicture}[shorten >=1pt,node distance=2cm,on grid,auto]
\node[state] (q_1)   {$1$};
\node[state] (q_2) [above=of q_1] {$2$};
\node[state] (q_3) [above right=of q_1, below right=of q_2] {$0$};
\node[state] (q_4) [right=of q_3] {$3$};
\path[->]
(q_1) edge  node {A} (q_2)
(q_2) edge  node {A} (q_3)
(q_3) edge  node {A} (q_1)
(q_3) edge[bend left, above]  node {B} (q_4)
(q_4) edge[bend left, below]  node {B} (q_3);
\end{tikzpicture}


Number of edges may be optimized by filtering with FIRST/REST and other functions

Select the best tree from SPPF after parsing finish.

Priority queue for descriptors.
How to choose priority function? --- ordered tuples!

Priority is a number of additional edges (not from the original input) in processed prefix.
Suffix length.


\bibliographystyle{abbrv}
\bibliography{sigproc}

\balancecolumns

\end{document}