\section{Практика 6}

\subsection{Григорьев С.В.}

Алгоритм CYK и алгоритм Хеллингса.

Формат входа для CYK:
\begin{enumerate}
\item Грамматика: смотри предыдущее ДЗ.
\item Входная строка: терминалы разделены пробелами.
\end{enumerate}

Пример входа, описывающего граммтику $S \to a S b S \mid \varepsilon$:

\begin{verbatim}
S a S b S
S eps 
\end{verbatim}

Пример входной строки:
\begin{verbatim}
a a b a a b b b 
\end{verbatim}


Формат входа алгоритма Хеллингса:
\begin{enumerate}
\item Грамматика: тот же формат, что и для CYK. НО! ИСпользуем преобразование а ослабленную НФХ.
\item Входной граф: файл в ктором на каждой строке записано ребро в виде тройки 
$$
\langle\textit{вершина } \textit{метка\_ребра} \textit{ вершина}\rangle.
$$
Элементы тройки разделены пробелами.
\item Можно считать, что все вершины графа --- числа от нуля, идущие подряд.
\end{enumerate}

Пример входного графа:

\begin{center}
    \begin{tikzpicture}[node distance=3cm,shorten >=1pt,on grid,auto]
    \node[state] (q_0)   {$0$};
    \node[state] (q_1) [above right=of q_0] {$1$};
    \node[state] (q_2) [right=of q_0] {$2$};
    \node[state] (q_3) [right=of q_2] {$3$};
    \path[->]
    (q_0) edge  node {$a$} (q_1)
    (q_1) edge  node {$a$} (q_2)
    (q_2) edge  node {$a$} (q_0)
    (q_2) edge[bend left, above]  node {$b$} (q_3)
    (q_3) edge[bend left, below]  node {$b$} (q_2);
    \end{tikzpicture}
\end{center}

Пример описания входного графа:
\begin{verbatim}
0 a 1
1 a 2 
2 a 0
2 b 3
3 b 2
\end{verbatim}

Домашнее задание.
\begin{enumerate}
    \item Реализовать алгоритм CYK для линейного входа. На вход принимаются два файла: с граммтикой и входной строкой. Результат (выводится ли входная цепочка в грамматике) печатается в консоль.
    \item Реализовать алгоритм Хеллингса. На вход принимается файл с граммтикой и файл с графом. В результирующий файл печатается граммтика в ослабленной НФХ (с которой непосредсвенно работал алгоритм) и множество пар достижимых вершин для стартового нетерминала (одна пара на строку, две вершины через пробел)  
\end{enumerate} 
