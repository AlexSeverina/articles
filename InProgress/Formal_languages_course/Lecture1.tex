\section{Лекция 1: Введение}

Алфавит, язык. Операции над строками. Операции над языками.

Какие вопросы можно задавать о языках: о пустоте, универсальности, о построении пересечения, о пустоте пересечения, о вложенности, о эквивалентности.

Базовые способы задания: перечисление, генератор, распознаватель.

Грамматики. Иерархия Хомского. Проблемы с ней. Классы языков.

Взаимосвязь теории формальных языков с другими областями, области её применения.
\begin{itemize}
  \item Синтаксический анализ языков программирования: в компиляторах, интерпертаторах, средах разработки, других инстументах.
  \item Анализ естественных языков.
  Активность в этой области несколько спала, так как на передний план сейчас вышли различные методы машинного обучения.
  \item Статический анализ кода.
  \begin{itemize}
    \item Репс и компания
    \item Типы в Java
    \item Шафл и потоки

  \end{itemize}
  \item Графовые базы данных
  \item Биоинформатика
  \item Языки --- это не только про строки.
  \begin{itemize}
    \item Языки деревьев
    \item Языки графов
    \item !!!
  \end{itemize}
  \item Теория групп.
  \item Прочая забавная математика.
  \begin{itemize}
    \item Салвати
    \item Клики
    \item Уравнения 
  \end{itemize}
\end{itemize}
