\section{Лекция 1: Введение}

Алфавит, язык. Операции над строками. Операции над языками.

Какие вопросы можно задавать о языках: о пустоте, универсальности, о построении пересечения, о пустоте пересечения, о вложенности, об эквивалентности.

Базовые способы задания: перечисление, генератор, распознаватель.

Грамматики. Иерархия Хомского. Проблемы с ней. Классы языков.

Взаимосвязь теории формальных языков с другими областями, области её применения.
\begin{itemize}
  \item Синтаксический анализ языков программирования: в компиляторах, интерпертаторах, средах разработки, других инстументах.
  \item Анализ естественных языков.
  Активность в этой области несколько спала, так как на передний план сейчас вышли различные методы машинного обучения.
  Однако и в этой области ведуться работы. 
  Примеры конференций: 
  \begin{itemize}
    \item \href{http://www.wikicfp.com/cfp/servlet/event.showcfp?eventid=98626&copyownerid=320}{International Conference on Parsing Technologies} (\href{https://iwpt20.sigparse.org/callforpapers.html}{IWPT-2020})
    \item \href{http://www.wikicfp.com/cfp/program?id=1029&s=FG&f=Formal%20Grammar}{FG: Formal Grammar} (\href{http://fg.phil.hhu.de/2020/}{FG-2020})
  \end{itemize}

  \item Статический анализ кода.
  \begin{itemize}
    \item Различные задачи межпроцедурного анализа. Основной подход --- language reachability. Основоположник --- Томас Репс. Примеры работ.
    \begin{itemize}
      \item Thomas Reps. 1997. Program analysis via graph reachability. In Proceedings of the 1997 international symposium on Logic programming (ILPS ’97). MIT Press, Cambridge, MA, USA, 5–19.
      \item Qirun Zhang and Zhendong Su. 2017. Context-sensitive data-dependence analysis via linear conjunctive language reachability. In Proceedings of the 44th ACM SIGPLAN Symposium on Principles of Programming Languages (POPL 2017). Association for Computing Machinery, New York, NY, USA, 344–358. DOI:https://doi.org/10.1145/3009837.3009848
      \item Kai Wang, Aftab Hussain, Zhiqiang Zuo, Guoqing Xu, and Ardalan Amiri Sani. 2017. Graspan: A Single-machine Disk-based Graph System for Interprocedural Static Analyses of Large-scale Systems Code. In Proceedings of the Twenty-Second International Conference on Architectural Support for Programming Languages and Operating Systems (ASPLOS ’17). Association for Computing Machinery, New York, NY, USA, 389–404. DOI:https://doi.org/10.1145/3037697.3037744
      \item Lu Y., Shang L., Xie X., Xue J. (2013) An Incremental Points-to Analysis with CFL-Reachability. In: Jhala R., De Bosschere K. (eds) Compiler Construction. CC 2013. Lecture Notes in Computer Science, vol 7791. Springer, Berlin, Heidelberg
    \end{itemize}
    \item Интерливинг (или шафл) языков для верификаци многопоточных программ.
    \begin{itemize}
      \item \href{http://uu.diva-portal.org/smash/get/diva2:442518/FULLTEXT01.pdf}{Approximating the Shuffle of Context-free Languages to Find Bugs in Concurrent Recursive Programs}
      \item Flick N.E. (2015) Quotients of Unbounded Parallelism. In: Leucker M., Rueda C., Valencia F. (eds) Theoretical Aspects of Computing - ICTAC 2015. ICTAC 2015. Lecture Notes in Computer Science, vol 9399. Springer, Cham
    \end{itemize}

    \item Система типов Java: \href{https://arxiv.org/abs/1605.05274}{Radu Grigore, Java Generics are Turing Complete}.
  \end{itemize}

  \item Графовые базы данных
  \item Биоинформатика
  \item Машинное обучение.
   \begin{itemize}
      \item \href{https://arxiv.org/abs/1703.01925}{Matt J. Kusner, Brooks Paige, José Miguel Hernández-Lobato. Grammar Variational Autoencoder}. Опубликована в 2017 году и уже \href{https://scholar.google.com/scholar?cites=4080460899049502885&as_sdt=2005&sciodt=0,5&hl=ru}{больше 200 цитирований.}
      \item \href{https://www.aclweb.org/anthology/D17-1180.pdf}{TAG Parsing with Neural Networks and Vector Representations of Supertags}. К разговору об оброаботке естественных языков.
      \item \href{https://arxiv.org/abs/1804.06610}{Jungo Kasai, Robert Frank, Pauli Xu, William Merrill, Owen Rambow. End-to-end Graph-based TAG Parsing with Neural Networks.}
    \end{itemize}

  \item Языки --- это не только про строки.
  \begin{itemize}
    \item Языки деревьев: \href{http://tata.gforge.inria.fr/}{Tree Automata Techniques and Applications}.
    \item Языки графов: 
    \begin{itemize}
      \item \href{http://www.its.caltech.edu/~matilde/GraphGrammarsLing.pdf}{Graph Grammars}
      \item \href{https://people.cs.umu.se/drewes/biblio/ps-files/hrg.pdf}{HYPEREDGE REPLACEMENT GRAPH GRAMMARS}
      \item \href{https://www.aclweb.org/anthology/W17-3410.pdf}{(Re)introducing Regular Graph Languages}
      \item \href{https://www.springer.com/gp/book/9783540560050}{Hyperedge Replacement: Grammars and Languages}
    \end{itemize}
    \item $\ldots$
  \end{itemize}
  \item Теория групп. Как правило, это проблема слов группы или дополнение к ней.
  \begin{itemize}
    \item Anisimov, A.V. Group languages. Cybern Syst Anal (1971) 7: 594.
    \item David E. Muller, Paul E. Schupp, Groups, the Theory of ends, and context-free languages, Journal of Computer and System Sciences, Volume 26, Issue 3, 1983, Pages 295-310, ISSN 0022-0000
    \item HOLT, D., REES, S., ROVER, C., \& THOMAS, R. (2005). GROUPS WITH CONTEXT-FREE CO-WORD PROBLEM. Journal of the London Mathematical Society, 71(3), 643-657. doi:10.1112/S002461070500654X
    \item \href{https://arxiv.org/abs/1407.7745}{Groups with Context-Free Co-Word Problem and Embeddings into Thompson's Group V} 
    \item \href{https://www.degruyter.com/view/j/gcc.2019.11.issue-1/gcc-2019-2004/gcc-2019-2004.xml}{Kropholler, R. \& Spriano, D. (2019). Closure properties in the class of multiple context-free groups. Groups Complexity Cryptology, 11(1), pp. 1-15. Retrieved 13 Feb. 2020, from doi:10.1515/gcc-2019-2004}
    \item \href{https://personalpages.manchester.ac.uk/staff/Mark.Kambites/events/nbsan/nbsan17_thomas.pdf}{Word problems of groups, formal languages and decidability}
  \end{itemize}

  \item Прочая забавная математика.
  \begin{itemize}
    \item Немного топологии в теории формальных языков: \href{https://hal.archives-ouvertes.fr/hal-01771670/}{Salvati S. On is an n-MCFL. – 2018.}
    \item Salvati S. MIX is a 2-MCFL and the word problem in Z2 is captured by the IO and the OI hierarchies //Journal of Computer and System Sciences. -- 2015. -- Т. 81. -- \textnumero. 7. -- С. 1252-1277.
    \item О том, как задачи из теории графов связаны с теорией формальных языков: Abboud, Amir \& Backurs, Arturs \& Williams, Virginia. (2015). If the Current Clique Algorithms are Optimal, So is Valiant's Parser. 98-117. 10.1109/FOCS.2015.16.
    \item \href{https://www.sciencedirect.com/science/article/abs/pii/S0196885819300739}{A context-free grammar for the Ramanujan-Shor polynomials}
  \end{itemize}
\end{itemize}
