\section{Практика 8}

\subsection{Григорьев С.В.}

Разработка синтаксических анализаторов: лексический и синтаксический анализы, абстрактный и конкретный синтаксис.

\subsubsection{Лексический анализ}

$$
E \to n \mid E \ + \ E \mid E \ * \ E 
$$

Что такое $n$ (число)? Это абстракция.

$$
n = [1-9][0-9]^*
$$

$$
n = 0 \mid ((-)?[1-9][0-9]^*)
$$

$$
n = 0 \mid ((-)?[1-9][0-9]^*(.)[0-9]^*[1-9])
$$

То же самое верно и для других терминалов.

$$
E \to n \mid E \ op\_plus \ E \mid E \ op\_pow \ E \mid E \ op\_mult \ E 
$$

$$
op\_pow = \verb|^|
$$


$$
op\_pow = **
$$

Для введени этой абстракции анализ языков разделяют на две стадии: 
\begin{enumerate}
	\item Лексический анализ: переводим последовательность ``символов с клавиатуры'' в последовательность токенов/терминальных символов. Работает на основе регулярных выражений (как правило).

	\item Синтаксичексий анализ: переводим последовательность терминальных символов в структурное представление (дерево разбора). Работает на основе КС грамматик.
\end{enumerate}


Да, можно и не разделять: scanerless parsing.


\subsubsection{Абстрактный и конкретный синтаксис}

Структурное представление текста, удобное для решения тех или иных задач, не содержит многие детали исходного текста.

Абстрактное описание синтаксиса:
\begin{verbatim}
type stmt =
   ...
   | IfStmt of expr*List<stmt>*List<stmt>
   ...
\end{verbatim}

Примеры конкретного синтаксиса:
\hrule
\begin{verbatim}
...
if ( a + b > 0 )
then
  ...
else
  ...
...
\end{verbatim}
\hrule
\begin{verbatim}
...
if ( a + b > 0 )
{
  ...
}
else
{
  ...
}
...
\end{verbatim}
\hrule
\begin{verbatim}
...
cond_branch ( a + b > 0 )( ... )( ... )
...
\end{verbatim}
\hrule

Ещё пример абстрактного синтаксиса:

\begin{verbatim}
type arithExpr =
   | Num of int
   | OpPlus of arithExpr * arithExpr
   | OpMult of arithExpr * arithExpr
   | OpPow of arithExpr * arithExpr
\end{verbatim}

Важно то, что здесь нет ни слова про скобки, приоритеты, ассоциотивность. 

\hrule
\begin{verbatim}
12 + 3 * 4 ^ 5 + 2
\end{verbatim}
\hrule
\begin{verbatim}
(12 + 3) * 4 ^ 5 + 2
\end{verbatim}
\hrule
\begin{verbatim}
12 + 3 * 4 ^ (5 + 2)
\end{verbatim}
\hrule
\begin{verbatim}
(12 + 3 * 4) ^ (5 + 2)
\end{verbatim}
\hrule

Потому что многие свойства естественным образом выражается структкрой дерева и, соответственно, детали исходного текста теряются.

\hrule
\begin{verbatim}
12 + 3 * 4 ^ 5 + 2

OpPlus 
   (Num 12) 
   (OpPlus 
      (OpMult 
         (Num 3) 
         (OpPow 
            (Num 4) 
            (Num 5))) 
      (Num 2)
\end{verbatim}
\hrule
\begin{verbatim}
(12 + 3) * 4 ^ 5 + 2

OpPlus
   (OpMult
      (OpPlus
         (Num 12)
         (Num 3))
      (OpPow
         (Num 4)
         (Num 5))
    (Num 2))
\end{verbatim}
\hrule
\begin{verbatim}
12 + 3 * 4 ^ (5 + 2)

OpPlus 
   (Num 12)
   (OpMult
      (Num 3)
      (OpPow
         (Num 4)
         (OpPlus
            (Num 5)
            (Num 2))))
\end{verbatim}
\hrule
\begin{verbatim}
(12 + 3 * 4) ^ (5 + 2)

OpPow
   (OpPlus
      (Num 12)
      (OpMult
         (Num 3)
         (Num 4))
   )
   (OpPlus
      (Num 5)
      (Num 2))
\end{verbatim}
\hrule


Но важно помнить, что для разных задач нужна разная информация.


Можно начинать знакомство с \href{https://www.antlr.org/download.html}{ANTLR}, так как он будет использоваться в следующих домашних работах.

