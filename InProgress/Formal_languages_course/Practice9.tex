\section{Практика 9}

\subsection{Григорьев С.В.}

Абстрактный и конкретный синтаксис языка запросов к графам.

Абстрактный синтаксис.

\begin{verbatim}
script: List<stmt>

nt_name: String

stmt:
  | connect of String
  | list
  | select of obj_expr * from_expr * where_expr
  | named_pattern of nt_name * regexp

from_expr: String

obj_expr:
  | vertices of vs_info  
  | conut of vs_info
  | exists of vs_info

vs_info:
  | pair of v_info * v_info
  | single of v_info

v_info:
  | name of String
  | any
  
where_expr:
  | path of v_expr * pattern * v_expr

v_expr: 
  | name of String
  | any 
  | v_pattern

v_pattern:
  | id_pattern of Int   

ident: 
  | nt_name of String 
  | t_name of String

pattern: 
  | smb of ident
  | star of regexp
  | plus of regexp
  | alt of regexp * regexp
  | seq of regexp * regexp
  | option of regexp

\end{verbatim}

Пример конкретного синтаксиса.

\begin{align*}
\textit{script} \to & \varepsilon  \\
                    & \mid \textit{stmt \ SEMI \ script} \\
\textit{stmt} \to & \textit{KW\_CONNECT \ KW\_TO \ STRING} \\
                  & \mid KW\_LIST \ KW\_ALL \ KW\_GRAPHS \\
                  & \mid select\_stmt \\
                  & \mid named\_pattern\_stmt \\
named\_pattern \to & NT\_NAME \ OP\_EQ  \ pattern \\
select\_stmt \to & \textit{KW\_SELECT} \ obj\_expr \ \textit{KW\_FROM} \ STRING \ \textit{KW\_WHERE} \ where\_expr \\
obj\_expr \to & vs\_info \\
              & \mid \textit{KW\_COUNT} \ vs\_info \\
              & \mid \textit{KW\_EXISTS} \ vs\_info \\
vs\_info \to & LBR \ IDENT \ COMMA \ IDENT \ RBR \\
             & \mid IDENT \\
where\_expr \to & LBR \ v\_expr RBR \ \\
                & OP\_MINUS \ pattern \ OP\_MINUS \ OP\_GR \\
                & LBR \ v\_expr \ RBR \\
v\_expr \to & IDENT \mid UNDERSCORE \mid IDENT \ DOT \ KW_ID \ OP\_EQ \ INT \\
pattern \to & alt\_elem \\
            & \mid alt\_elem \ MID \ pattern \\
alt\_elem \to & seq \\
             & \mid LBR \ RBR \\
seq \to & seq\_elem \\
        & \mid seq\_elem \ seq \\
seq\_elem \to & prim\_pattern \\
              & \mid prim\_pattern \ OP\_STAR \\
              & \mid prim\_pattern \ OP\_PLUS \\
              & \mid prim\_pattern \ OP\_Q \\
prim\_pattern \to & IDENT \\
                 & \mid NT\_NAME \\
                 & \mid LBR \ pattern \ RBR \\
\end{align*}

Токены (терминальный алфавит).
\begin{align*}
LBR = & '(' \\
RBR = & ')' \\
COMMA = & ',' \\
SEMI = & ';' \\
MID = & '|' \\
DOT = & '.' \\
OP\_STAR = & '*' \\ 
OP\_PLUS = & '+' \\
OP\_Q = & '?' \\
OP\_MINUS = & '-' \\
OP\_GR = & '>' \\
OP\_EQ = & '=' \\
KW\_ID = & 'ID' \\
KW\_COUNT = & 'COUNT' \\
KW\_EXISTS = & 'EXISTS' \\
KW\_FROM = & 'FROM' \\
KW\_WHERE = & 'WHERE' \\
KW\_LIST = & 'LIST' \\
KW\_ALL = & 'ALL' \\
KW\_GRAPHS = & 'GRAPHS' \\
KW\_CONNECT = & 'CONNECT' \\
KW\_TO = & 'TO' \\
IDENT = & [a-z][a-z]^* \\
INT = & 0 \mid [1-9][0-9]^* \\
NT\_NAME = & [A-Z][a-z]^* \\
STRING = & {'['} ([aA-zZ] \mid [0-9] \mid ({'-'} \mid {' \ '} \mid {'\_'} \mid {'/'} \mid {'.'}))^* {']'}
\end{align*}

Пример скрипта:
\begin{verbatim}
CONNECT TO [\home\user\graph_db];
S = a S b S | () ;
SELECT COUNT(u) FROM [g1.txt] where (v.id = 10) - S -> (u);
\end{verbatim}

Задача на дом.
\begin{enumerate}
  \item Описать конкретный синтаксис языка запросов к графам, согралованный с приведённым абстрактным синтаксисом. Результат: грамматика и лексика в маркдаун (*.md) файле в репозитории (например, описание языка в readme.md). С примерами выражений на языка и текстовым описанием его особенностей.
  \item Синтаксический анализатор предложенного языка на основе имеющейся реализации CYK. На вход принимается строка (с консоли или из файла), в которой токены разделены пробелами и переносами строк. Для приведённого выше скрипта вход выглядит следующим образом:
  \begin{verbatim}
  KW_CONNECT KW_TO STRING SEMI 
  NT_NAME OP_EQ IDENT NT_NAME IDENT NT_NAME MID LBR RBR SEMI 
  KW_SELECT KW_COUNT LBR IDENT RBR 
  KW_FROM STRING 
  KW_WHERE LBR IDENT DOT KW_ID OP_EQ INT RBR 
  OP_MINUS NT_NAME OP_MINUS OP_GR LBR IDENT RBR SEMI
  \end{verbatim}
\end{enumerate}