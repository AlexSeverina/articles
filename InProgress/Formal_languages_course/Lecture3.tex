\section{Лекция3. Контекстно-свободные грамматики}

Лево- и право-линейные грамматики и регулярные языки. Неразрешимость задачи проверки того, что граммтика задаёт регулярный язык. Статья на эту тему: \href{https://link.springer.com/article/10.1007/s00236-003-0133-8}{Self-embedded context-free grammars with regular counterparts}. Грамматика $\to$ регулярка и регулярка $\to$ грамматика.

Выовд цепочки в грамматике, левосторонний, правосторонний вывод, неоднозначные и однозначные грамматики. Примеры.
Существенно неоднозначные языки.

Дерево вывода. Соотношение между деревьями и выводами. Примеры.

Расширенные контекстно-свободные грамматики.