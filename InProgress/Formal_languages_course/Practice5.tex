\section{Практика 5}

\subsection{Григорьев С.В.}

Преобразование в нормальную форму Хомского.

Формат входа:
\begin{enumerate}
\item Одна продукция на строку.
\item Продукция --- это список терминалов и нетерминалов через пробел, начинающийся с нереминала (левая часть продукции).
\item Нетерминалы --- заглавные буквы с опциональным числовым суффиксом.
\item Терминалы --- строчные буквы с опциональным числовым суффиксом.
\item Специальный символ \verb|eps| для обобзначения $\varepsilon$.
\end{enumerate}

Пример входя, описывающего граммтику $S \to a S b S \mid \varepsilon$:

\begin{verbatim}
S a S b S
S eps 
\end{verbatim}

Домашнее задание.
\begin{enumerate}
    \item Реализовать преобразование в нормальную форму Хомского. На входе файл с граммтикой, на выходе --- файл с граммтикой в НФХ в том же формате, что и вход.
\end{enumerate} 
