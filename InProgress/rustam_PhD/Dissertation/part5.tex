\chapter{Эксперименты, ограничения, обсуждение}\label{ch:ch5}
Завершаемость и корректность предложенных алгоритмов формально доказаны выше, однако их производительность требует экспериментальной оценки. При этом основной интерес представляет оценка на входных данных, близких к реальным. Вместе с этим необходимо рассмотреть границы применимости полученных в работе результатов. Таким образом, в данной главе решаются следующие задачи.

Оценка производительности проводилась на реальных и на синтетических данных. Краткое описание экспериментов приведено в таблице.

Таблица эксперимент и что исследовалось.
1) Сравнение между матричными реализациями. 2) Сравнение лучших матричных с GLL и бразильцами.

Далее приводяся детальные описания экспериментов.

\section{Экспериментальное исследование}\label{sec:ch5/sect1}

\subsection{Постановка экспериментов}
Характеристики системы.

Описание реализаций.

Использованный датасет CFPQ\_data.

\subsection{Сравнение предложенных реализаций}
Таблица, обсуждение.

\subsection{Сравнение с существующими реализациями}
Сравнение лучших матричных с GLL и бразильцами. Таблица. Обсуждение.
 
\subsection{Выводы}
Выводы

\section{Ограничения}\label{sec:ch5/sect3}
Используемые подходы и алгоритмы имеют ряд ограничений.

Пользуемся разреженностью реальных графов.

В случае когда грамматика нужна в нормальной форме, зависим от её разрастания.

Зависим от реализованных матричных операций. Функционала библиотек.

Большая временная сложность при определнном графе и грамматике.

\clearpage