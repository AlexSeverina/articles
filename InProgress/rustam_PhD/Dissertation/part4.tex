\chapter{Семейство алгоритмов вычисления КС-запросов не требующих трансформации входной грамматики}\label{ch:ch4}
В данной главе изложены алгоритмы вычисления КС-запросов к графам, который не требует трансформации входной КС-грамматики, а также использует операции линейной алгебры и различные семантики запросов. Также сформулированы и доказаны утверждения о корректности изложенных алгоритмов. Кроме того, работа данных алгоритмов продемонстрирована на примере. 
\section{Алгоритмы}\label{sec:ch4/sect1}
В данном разделе изложены алгоритмы вычисления КС-запросов к графам, который не требует трансформации входной КС-грамматики, а также использует операции линейной алгебры и различные семантики запросов.

Строим по грамматике рекурсивный автомат.

Определяем полукольцо. Показываем переход к булевым матрицам. Произведение Кронекера.

Приводим алгоритм для реляционной семантики запросов.

Приводим алгоритм для одногу пути.

И алгоритм получения этого пути.

Приводим все тоже самое для семнтики всех путей.

\section{Корректность алгоритмов}\label{sec:ch4/sect2}
В данном разделе сформулированы и доказаны утверждения о корректности и завершаемости изложенных алгоритмов. Также представлены оценки временной сложности данных алгоритмов.

Теорема. Завершаемость всех алгортимов.

Теорема. Корректность алгоритма для реляционной семантики запросов.

Теорема. Оценка временной сложности алгоритма для реляционной семантики запросов.

Теорема. Корректность алгоритма одного пути.

Теорема. Корректность алгоритма получения пути.

Теорема. Оценка временной сложности алгоритма одного пути и алгоритма получения пути.

Теорема. Корректность алгоритма всех путей.

Теорема. Корректность алгоритма получения всех путей.

Теорема. Оценка временной сложности алгоритма всех путей и алгоритма получения путей.

\section{Пример}\label{sec:ch4/sect3}
В данном разделе работа изложенных алгоритмов продемонстрирована на примере, основанном на КС-языке правильных вложенных скобочных последовательностей.

Грамматика, граф.

Достигается худший случай.

Работа базового алгортима, картинки матриц. Результат.

Работа алгоритма одного пути. Картинки матриц. Результат.

Работа алгоритма всех путей. Картинки матриц. Результат.

\clearpage
