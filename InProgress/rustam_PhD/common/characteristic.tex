
{\actuality} 
Графы используются в качестве структуры данных для представления больших объемов информации в компактной и удобной для анализа форме во многих областях, например, в биоинформатике, в графовых базах данных, при статическом анализе программ. При этом оказывается необходимым выявлять сложные зависимости между вершинами графа, которые выражаются путями между ними. Для этого необходимо решать задачу достижимости с определенными ограничениями на пути в графе, которая отвечает на вопрос о существовании путей, удовлетворяющих данным ограничениям между каждой парой вершин. Кроме того, в некоторых областях, в качестве доказательства существования таких путей необходимо предъявить все или хотя бы один из них. 


Для описания ограничений на пути в помеченном графе естественно выделять пути с помощью формальных грамматик (регулярные выражения, контекстно-свободные грамматики) над алфавитом, содержащим метки на ребрах этого графа. В настоящее время активно исследуются ограничения в виде контекстно-свободных (КС) грамматик, так как они позволяют описывать более широкий класс запросов, чем регулярные выражения.

Однако большинство существующих алгоритмов в данной области имеют низкую производительность на больших графах, что затрудняет их применение на практике. Хорошую производительность показывают алгоритмы анализа графов, использующие операции линейной алгебры. В данном подходе, при активном использовании матричных операций в процессе анализа графов, может быть применен широкий класс оптимизаций, например, разреженное представление матриц и параллельное вычисление. Таким образом, использование подобного подхода для поиска путей в графе с КС-ограничениями требует изучения. На текущий момент существующий матричный алгоритм в данной области позволяет лишь решить задачу достижимости, т.е. установить факт наличия между двумя вершинами пути определенного вида, при этом сам путь не предоставляется. 

Кроме того, в данной области не исследована возможность использования такой матричной операции, как произведение Кронекера. Данное направление интересно тем, что позволяет уменьшить общее количество операций над матрицами и оперировать матрицами большего размера, чем при использовании обычного матричного произведения. Это может позволить получить еще больший прирост производительности после применения матричных оптимизаций.

Таким образом, для большинства типов ограничений необходима разработка алгоритмов поиска путей, использующих различные операции линейной алгебры и имеющих высокую производительность на больших графах.


%Обзор, введение в тему, обозначение места данной работы в
%мировых исследованиях и~т.\:п., можно использовать ссылки на~другие
%работы~\autocite{Gosele1999161}
%(если их~нет, то~в~автореферате
%автоматически пропадёт раздел <<Список литературы>>). Внимание! Ссылки
%на~другие работы в~разделе общей характеристики работы можно
%использовать только при использовании \verb!biblatex! (из-за технических
%ограничений \verb!bibtex8!. Это связано с тем, что одна
%и~та~же~характеристика используются и~в~тексте диссертации, и в
%автореферате. В~последнем, согласно ГОСТ, должен присутствовать список
%работ автора по~теме диссертации, а~\verb!bibtex8! не~умеет выводить в~одном
%файле два списка литературы).
%При использовании \verb!biblatex! возможно использование исключительно
%в~автореферате подстрочных ссылок
%для других работ командой \verb!\autocite!, а~также цитирование
%собственных работ командой \verb!\cite!. Для этого в~файле
%\verb!common/setup.tex! необходимо присвоить положительное значение
%счётчику \verb!\setcounter{usefootcite}{1}!.


{\progress}
TODO
%Этот раздел должен быть отдельным структурным элементом по
% ГОСТ, но он, как правило, включается в описание актуальности
% темы. Нужен он отдельным структурынм элемементом или нет ---
% смотрите другие диссертации вашего совета, скорее всего не нужен.

{\aim} данной работы является разработка алгоритмов поиска путей в графе с КС-ограничениями, использующих операции линейной алгебры и имеющих высокую производительность на больших графах.

Достижение поставленной цели обеспечивается решением следующих {\tasks}:
\begin{enumerate}[beginpenalty=10000] % https://tex.stackexchange.com/a/476052/104425
  \item Разработать подход к вычислению КС-запросов к графам, использующий операции линейной алгебры.
  \item Разработать семейство алгоритмов вычисления КС-запросов к графам, использующих предложенный подход и позволяющих предоставлять искомые пути.
  \item Разработать алгоритм вычисления КС-запросов к графам, использующий предложенный подход и произведение Кронекера в качестве основной операции линейной алгебры.
\end{enumerate}


{\novelty}
\begin{enumerate}[beginpenalty=10000] % https://tex.stackexchange.com/a/476052/104425
	
	\item Подход, предложенный в диссертации, отличается от аналогов (работы Семёна Григорьева, Джелле Хеллингса) активным использованием матричных операций в процессе вычисления запросов. Это позволяет применять широкий класс оптимизаций для вычисления матричных операций и получать высокую производительность при работе с большими графами.
	
	\item Семейство алгоритмов, предложенное в диссертации, отличается от аналогов (работы Семёна Григорьева, Джелле Хеллингса, Сяованга Чжана) более высокой производительностью на больших графах и (формулировка улучшения собственного результата?) отличается от матричного алгоритма Азимова Рустама возможностью построения искомых путей. Это позволяет предъявлять пути в качестве доказательства отношения определенного вида между парами вершин, что является важным результатом анализа во многих областях.
	
	\item Алгоритм, предложенный в диссертации, отличается от аналогов (работы Семёна Григорьева, Джелле Хеллингса, Сяованга Чжана) более высокой производительностью на больших графах и (формулировка улучшения собственного результата?) отличается от матричного алгоритма Азимова Рустама использованием в процессе вычисления запросов произведения Кронекера и представлением КС-грамматики запроса в виде рекурсивного автомата. Это позволяет оперировать в процессе вычисления матрицами большего размера и уменьшенить общее количество операций над ними.
	
\end{enumerate}

{\influence} 

Теоретическая значимость диссертационного исследования заключается в разработке подхода к вычислению КС-запросов к графам, использующего операции линейной алгебры, в разработке формальных алгоритмов, использующих данный подход, а также в формальном доказательстве завершаемости и корректности разработанных алгоритмов.

(Про практическую значимость обязательна ли интеграция? Или можно написать, что получены реализации, которые производительнее аналогов?)


{\methods} TODO

{\defpositions}
\begin{enumerate}[beginpenalty=10000] % https://tex.stackexchange.com/a/476052/104425
	\item Разработан подход к вычислению КС-запросов к графам, использующий операции линейной алгебры.
	\item Разработано семейство алгоритмов вычисления КС-запросов к графам, использующих предложенный подход и позволяющих предоставлять искомые пути. Доказана завершаемость и корректность предложенных алгоритмов. Проведено экспериментальное исследование.
	\item Разработан алгоритм вычисления КС-запросов к графам, использующий предложенный подход и произведение Кронекера в качестве основной операции линейной алгебры. Доказана завершаемость и корректность предложенного алгоритма. Проведено экспериментальное исследование.
\end{enumerate}

{\reliability}

Достоверность и обоснованность результатов исследования опирается на использование формальных методов исследуемой области, выполнение формальных доказательств и инженерные эксперименты.

Основные результаты работы были доложены на ряде международных научных конференций: GRADES'18, GRADES'20, ADBIS’20, SIGMOD'21(еще не приняли). Дополнительной апробацией является то, что разработка предложенных алгоритмов была поддержана грантом РНФ \textnumero 18-11-00100 и грантом РФФИ \textnumero 19-37-90101.

%{\contribution} Автор принимал активное участие \ldots

{\publications} Все результаты диссертации изложены в 5~\cite{1,2,3,4,5} научных работах, которые содержат основные результаты работы и индексируются Scopus. Работы~\cite{1,2,3,4}написаны в соавторстве. В~\cite{1} Р.~Азимову принадлежит разработка алгоритма, доказательство его корректности и завершаемости, реализация алгоритма, работа над текстом. В~\cite{2} Р.~Азимову принадлежит разработка алгоритма, доказательство его корректности и завершаемости, работа над текстом. В~\cite{3,4} Р.~Азимову принадлежит работа над доказательствами корректности и завершаемости алгоритма, работа над текстом.
    
Работа~\cite{5} еще не написана.
    
Работы, которые можно включить в список: Публикация~\cite{6} ВАК и SCOPUS матричный алгоритм для конъюнктивных грамматик. Статья~\cite{7} ВАК конъюнктивные труды ИСП РАН. Статья~\cite{8} РИНЦ про конъюнктивные и выступление на конференции PLC'17.

\newcounter{firstbib}

\begin{thebibliography}{99}
	\bibitem{1} Azimov R. Context-Free Path Querying by
	Matrix Multiplication / Azimov R., Grigorev S. // In Proceedings of the
	1st Joint International Workshop on Graph Data Management Experiences \&
	Systems (GRADES) and Network Data Analytics (NDA) (GRADES-NDA’18)
	\bibitem{2} Azimov R. Context-Free Path Querying with Single-Path Semantics by
	Matrix Multiplication / Terekhov A., Khoroshev A., Azimov R., Grigorev S. // In Proceedings of the
	3rd Joint International Workshop on Graph Data Management Experiences \&
	Systems (GRADES) and Network Data Analytics (NDA) (GRADES-NDA’20)
	\bibitem{3} Azimov R. Context-Free Path Querying by Kronecker
	Product / Orachev E., Epelbaum I., Azimov R., Grigorev S. // In Proceedings of the
	24th European Conference on Advances in Databases and Information Systems (ADBIS’20)
	\bibitem{4} Azimov R. Context-Free Path Querying by Kronecker
	Product большая версия / Orachev E., Epelbaum I., Azimov R., Grigorev S. // In Proceedings of the (SIGMOD’21)
	
	\bibitem{5} Azimov R. Ненаписанная работа матричный алгоритм по всем путям
	
	\bibitem{6} Azimov R. Path Querying with Conjunctive Grammars by Matrix Multiplication / Azimov R., Grigorev S. //Programming and Computer Software. – 2019. – Vol. 45. – №. 7. – pp. 357-364.
	\setcounter{firstbib}{\value{enumiv}}
	
	
	
	\bibitem{7} Азимов Р. Ш. Синтаксический анализ графов с использованием конъюнктивных грамматик / Азимов Р., Григорьев С. // Труды ИСП РАН, 2018, том 1 вып. 2, с. 3-4.
	
	\bibitem{8} Азимов Р. Ш. Синтаксический анализ графов и задача генерации строк с ограничениями / Азимов Р., Григорьев С. // Сборник трудов конференции PLC 2017, с. 24-27.
\end{thebibliography}