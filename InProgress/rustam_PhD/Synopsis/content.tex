%Перечня рецензируемых научных изданий
\subsection*{\Large Общая характеристика работы}
\fontsize{14pt}{15pt}\selectfont
\subsubsection*{\large{Актуальность темы исследования}}
Графы используются в качестве структуры данных для представления больших объемов информации в компактной и удобной для анализа форме во многих областях, например, в биоинформатике, в графовых базах данных, при статическом анализе программ. При этом оказывается необходимым выявлять сложные зависимости между вершинами графа, которые выражаются путями между ними. Для этого необходимо решать задачу достижимости с определенными ограничениями на пути в графе, которая отвечает на вопрос о существовании путей, удовлетворяющих данным ограничениям между каждой парой вершин. Кроме того, в некоторых областях, в качестве доказательства существования таких путей необходимо предъявить все или хотя бы один из них. 


Для описания ограничений на пути в помеченном графе естественно выделять пути с помощью формальных грамматик (регулярные выражения, контекстно-свободные грамматики) над алфавитом, содержащим метки на ребрах этого графа. В настоящее время активно исследуются ограничения в виде контекстно-свободных (КС) грамматик, так как они позволяют описывать более широкий класс запросов, чем регулярные выражения.

Однако большинство существующих алгоритмов в данной области имеют низкую производительность на больших графах, что затрудняет их применение на практике. Хорошую производительность показывают алгоритмы анализа графов, использующие операции линейной алгебры. В данном подходе, при активном использовании матричных операций в процессе анализа графов, может быть применен широкий класс оптимизаций, например, разреженное представление матриц и параллельное вычисление. Таким образом, использование подобного подхода для поиска путей в графе с КС-ограничениями требует изучения. На текущий момент существующий матричный алгоритм в данной области позволяет лишь решить задачу достижимости, т.е. установить факт наличия между двумя вершинами пути определенного вида, при этом сам путь не предоставляется. 

Кроме того, в данной области не исследована возможность использования такой матричной операции, как произведение Кронекера. Данное направление интересно тем, что позволяет уменьшить общее количество операций над матрицами и оперировать матрицами большего размера, чем при использовании обычного матричного произведения. Это может позволить получить еще больший прирост производительности после применения матричных оптимизаций.

Таким образом, для большинства типов ограничений необходима разработка алгоритмов поиска путей, использующих различные операции линейной алгебры и имеющих высокую производительность на больших графах.



%Взаимодействие различных компонент приложений часто реализуется с помощью встроенных языков, то есть приложение, созданное на одном языке, генерирует код на другом языке и передаёт этот код на выполнение в соответствующее окружение. Примерами могут служить динамические SQL-запросы к базам данных в Java-коде или формирование HTML-страниц в PHP-приложениях. Генерируемая программа строится таким образом, чтобы в момент выполнения результирующий фрагмент кода (строка) представлял собой корректное выражение на соответствующем языке. Такой подход весьма гибок, так как позволяет использовать для формирования фрагментов кода различные строковые операции (replace, substring и т.д.) и комбинировать код из различных источников (например, учитывать текстовый ввод пользователя, что часто используется для задания фильтров при конструировании SQL-запросов). Необходимо отметить, что такой подход не имеет дополнительных накладных расходов, присущих, например, ORM-технологиям, и это позволяет достигать высокой производительности. 

%Однако динамическое формирование программ часто происходит с помощью операций конкатенации в циклах, условных операторах или рекурсивных процедурах, что приводит к множеству возможных вариантов значений для каждого выражения, что затрудняет их анализ. Поэтому фрагменты кода на встроенных языках воспринимаются компилятором исходного языка как простые строки, что приводит к высокой вероятности возникновения ошибок во время выполнения программы. В худшем случае такие ошибки не приведут к прекращению работы приложения, что явно указало бы на проблему, но целостность данных при этом может оказаться нарушенной. Более того, например, при наличии в коде приложения встроенных SQL-запросов нельзя, не проанализировав все динамически формируемые выражения, точно ответить на вопрос о том, с какими элементами базы данных не взаимодействует система, и  удалить их. При переносе такой системы на другую СУБД необходимо гарантировать, что для всех динамически формируемых выражений значение в момент выполнения будет корректным кодом на языке новой СУБД. Кроме того, при создании приложений распространённой практикой является использование интегрированных сред разработки, выполняющих подсветку синтаксиса и автодополнение, сигнализирующих о синтаксических ошибках, предоставляющих инструменты рефакторинга. Эти возможности значительно упрощают процесс разработки и отладки приложений и полезны не только для основного языка, но и для встроенных языков. Таким образом, для решения данных задач необходимы инструменты, проводящие статический анализ динамически формируемых программ.  

\subsubsection*{\large{Степень разработанности темы исследования}}
TODO
%Существуют классические исследования, посвященные разработке компиляторов --- работы А.~Ахо, А.~Брукера, С.~Джонсона, А.~П.~Ершова,  А.~Н.~Терехова, В.~О.~Сафонова, Б.~К.~Мартыненко и др.  Однако изложенные там алгоритмы синтаксического анализа, как и методы обобщённого синтаксического анализа,  исследованные такими учёными как Масару Томита (Masaru Tomita), Элизабет Скотт (Elizabeth Scott) и Адриан Джонстон (Adrian Johnstone) из университета Royal Holloway (Великобритания), Ян Рекерс (Jan Rekers, University of Amsterdam), Элко Виссер (Eelco Visser), не могут быть применены к решению задачи анализа динамически формируемых программ, поскольку предназначены для обработки входных данных, представимых в виде линейной последовательности символов, а такое представление динамически формируемых программ не возможно.

%Анализу динамически формируемых строковых выражений посвящены работы таких зарубежных учёных как Кюнг-Гу Дох (Kyung-Goo Doh), Ясухико Минамиде (Minamide Yasuhiko), Андерс Мёллер (Anders M{\o}ller) и отечественных учёных А.~А.~Бреслава, М.~Д.~Шапот. Хорошо изучены вопросы проверки корректности динамически формируемых выражений и поиска фрагментов кода, уязвимых для SQL-инъекций. Однако данные работы исследуют отдельные проблемы статического анализа динамически формируемых программ, оставляя в стороне создание готовых алгоритмов (в частности, не строят структурное представление анализируемых программ). В связи с этим возникают проблемы масштабируемости данных результатов, например, создание на их основе более сложных видов статического анализа.

%Также важным является предоставление компонентов, упрощающих создание новых инструментов для решения конкретных задач. Данный подход хорошо исследован в области разработки компиляторов, где широкое распространение получили генераторы анализаторов и пакеты стандартных библиотек (работы А.~Ахо, А.~Брукера, С.~Джонсона и др.), однако его применение в области анализа динамически формируемого кода не исследовано. 

%В работах отечественных учёных М.~Д.~Шапот и Э.~В.~Попова, а так же зарубежных учёных Антони Клеви (Anthony Cleve), Жан-Люк Эно (Jean-Luc Hainaut), Йост Виссер (Joost Visser) рассматривается реинжиниринг информационных систем, использующих встроенные SQL-запросы, однако не формулируется общего метода для решения таких задач. Этот вопрос также не затрагивается в классических работах, посвященных реинжиниригу (Х.~Миллера, А.~Н.~Терехова, Р.~С.~Арнольда и др.).

%Таким образом, актуальной является задача дальнейшего исследования статического анализа динамически формируемых строковых выражений. Кроме этого важным является решение вопросов практического применения средств анализа динамически формируемого кода: упрощение разработки инструментов анализа и создание методов их применения в реинжиниринге программного обеспечения.

\subsubsection*{\large{Объект исследования}}

 Объектом исследования являются алгоритмы поиска путей в графе с КС-ограничениями, использующие операции линейной алгебры.

\subsubsection*{\large{Цель и задачи диссертационной работы}}

\textbf{Целью} данной работы является разработка алгоритмов поиска путей в графе с КС-ограничениями, использующих операции линейной алгебры и имеющих высокую производительность на больших графах.

Достижение поставленной цели обеспечивается решением следующих \textbf{задач}.
\begin{enumerate}
	\item Разработать подход к вычислению КС-запросов к графам, использующий операции линейной алгебры.
	\item Разработать семейство алгоритмов вычисления КС-запросов к графам, использующих предложенный подход и позволяющих предоставлять искомые пути.
	\item Разработать алгоритм вычисления КС-запросов к графам, использующий предложенный подход и произведение Кронекера в качестве основной операции линейной алгебры.
\end{enumerate}

%Цели и задачи диссертационной работы соответствуют области исследований паспорта специальности 05.13.11 ``Математическое и программное обеспечение вычислительных машин, комплексов и компьютерных сетей'' --- пункту 
%1 (модели, методы и алгоритмы проектирования и анализа программ и программных систем, их эквивалентных преобразований, верификации и тестирования),
%пункту 2 (языки программирования и системы программирования, семантика программ) и пункту 10 (оценка качества, стандартизация и сопровождение программных систем).

\subsubsection*{\large{Методология и методы исследования}}
TODO

%Методология исследования основана на подходе к спецификации и анализу формальных языков, который начал активно развиваться в 50-х годах 20-го века в связи с изучением естественных языков (работы Н.~Хомского). В последствии этот подход получил широкое распространение в областях, связанных с обработкой языков программирования.
%При этом основными элементами данного подхода являются алфавит и грамматика исследуемого языка, разбиение автоматической обработки языка на выполнение таких шагов, как лексический, синтаксический и семантический анализ. Решаемые в связи с этим задачи связаны с поиском эффективных алгоритмов, выполняющих эти шаги. 

%В работе применяется алгоритм обобщённого восходящего синтаксического анализа RNGLR, созданный Элизабет Скотт (Elizabeth Scott) и Адриан Джонстон (Adrian Johnstone) из университета Royal Holloway (Великобритания). Для компактного хранения леса вывода используется структура данных Shared Packed Parse Forest (SPPF), которую предложил Ян Рекерс (Jan Rekers, University of Amsterdam). Доказательство завершаемости и корректности предложенного алгоритма проводится с применением теории формальных языков, теории графов и теории сложности алгоритмов. Приближение множества значений динамически формируемого выражения строилось в виде регулярного множества, описываемого с помощью конечного автомата.

\subsubsection*{\large{Положения, выносимые на защиту}}
\begin{enumerate}
	\item Разработан подход к вычислению КС-запросов к графам, использующий операции линейной алгебры.
	\item Разработано семейство алгоритмов вычисления КС-запросов к графам, использующих предложенный подход и позволяющих предоставлять искомые пути. Доказана завершаемость и корректность предложенных алгоритмов. Проведено экспериментальное исследование.
	\item Разработан алгоритм вычисления КС-запросов к графам, использующий предложенный подход и произведение Кронекера в качестве основной операции линейной алгебры. Доказана завершаемость и корректность предложенного алгоритма. Проведено экспериментальное исследование.
\end{enumerate}


\subsubsection*{\large{Научная новизна}}

Научная новизна полученных в ходе исследования результатов заключается в следующем.

\begin{enumerate}

\item Подход, предложенный в диссертации, отличается от аналогов (работы Семёна Григорьева, Джелле Хеллингса) активным использованием матричных операций в процессе вычисления запросов. Это позволяет применять широкий класс оптимизаций для вычисления матричных операций и получать высокую производительность при работе с большими графами.

\item Семейство алгоритмов, предложенное в диссертации, отличается от аналогов (работы Семёна Григорьева, Джелле Хеллингса, Сяованга Чжана) более высокой производительностью на больших графах и (формулировка улучшения собственного результата?) отличается от матричного алгоритма Азимова Рустама возможностью построения искомых путей. Это позволяет предъявлять пути в качестве доказательства отношения определенного вида между парами вершин, что является важным результатом анализа во многих областях.

\item Алгоритм, предложенный в диссертации, отличается от аналогов (работы Семёна Григорьева, Джелле Хеллингса, Сяованга Чжана) более высокой производительностью на больших графах и (формулировка улучшения собственного результата?) отличается от матричного алгоритма Азимова Рустама использованием в процессе вычисления запросов произведения Кронекера и представлением КС-грамматики запроса в виде рекурсивного автомата. Это позволяет оперировать в процессе вычисления матрицами большего размера и уменьшенить общее количество операций над ними.

\end{enumerate}

\subsubsection*{\large{Теоретическая и практическая значимость работы}}
Теоретическая значимость диссертационного исследования заключается в разработке подхода к вычислению КС-запросов к графам, использующего операции линейной алгебры, в разработке формальных алгоритмов, использующих данный подход, а также в формальном доказательстве завершаемости и корректности разработанных алгоритмов.

(Про практическую значимость обязательна ли интеграция? Или можно написать, что получены реализации, которые производительнее аналогов?)


%Теоретическая значимость диссертационного исследования заключается в разработке формального алгоритма синтаксического анализа динамически формируемого кода, решающего задачу построения конечного представления леса вывода, не решаемую ранее, а также в формальном доказательстве завершаемости и корректности разработанного алгоритма. 

%На основе полученных в работе научных результатов был разработан инструментарий (Software Development Kit, SDK), предназначенный для создания средств статического анализа динамически формируемых выражений. Данный инструментарий позволяет автоматизировать создание лексических и синтаксических анализаторов при решении задач реинжиниринга --- изучения и инвентаризации систем, автоматизации трансформации выражений на встроенных языках. Предложенный инструмент также может использоваться при реализации поддержки встроенных языков в интегрированных средах разработки.

%С использованием разработанного инструментария было реализовано расширение к инструменту ReSharper (ООО ``ИнтеллиДжей Лабс'', Россия), предоставляющее поддержку встроенного T-SQL в проектах на языке программирования C\# в среде разработки Microsoft Visual Studio. Так же было выполнено внедрение результатов работы в промышленный проект по переносу хранимого SQL-кода с MS-SQL Server 2005 на Oraclе 11gR2 (ЗАО ``Ланит-Терком'', Россия). 


\subsubsection*{\large{Степень достоверности и апробация результатов}}
Достоверность и обоснованность результатов исследования опирается на использование формальных методов исследуемой области, выполнение формальных доказательств и инженерные эксперименты.

Основные результаты работы были доложены на ряде международных научных конференций: GRADES'18, GRADES'20, ADBIS’20, SIGMOD'21(еще не приняли). Дополнительной апробацией является то, что разработка предложенных алгоритмов была поддержана грантом РНФ \textnumero 18-11-00100 и грантом РФФИ \textnumero 19-37-90101.

\subsubsection*{\large{Публикации по теме диссертации}}
 Все результаты диссертации изложены в 5~\cite{1,2,3,4,5} научных работах, которые содержат основные результаты работы и индексируются Scopus. Работы~\cite{1,2,3,4}написаны в соавторстве. В~\cite{1} Р.~Азимову принадлежит разработка алгоритма, доказательство его корректности и завершаемости, реализация алгоритма, работа над текстом. В~\cite{2} Р.~Азимову принадлежит разработка алгоритма, доказательство его корректности и завершаемости, работа над текстом. В~\cite{3,4} Р.~Азимову принадлежит работа над доказательствами корректности и завершаемости алгоритма, работа над текстом.
 
 Работа~\cite{5} еще не написана.
 
 Работы, которые можно включить в список: Публикация~\cite{6} ВАК и SCOPUS матричный алгоритм для конъюнктивных грамматик. Статья~\cite{7} ВАК конъюнктивные труды ИСП РАН. Статья~\cite{8} РИНЦ про конъюнктивные и выступление на конференции PLC'17.


%\subsubsection*{\large{Объем и структура работы}}
%Диссертация состоит из~введения, шести глав, заключения и~списка литературы. Полный объем диссертации \textbf{125}~страниц текста с~\textbf{26}~рисунками и~\textbf{8}~таблицами. Список литературы содержит \textbf{106}~наименований.

%\subsection*{\Large Содержание работы}


\newcounter{firstbib}

\begin{thebibliography}{99}
	\bibitem{1} Азимов Р. Ш. Context-Free Path Querying by
	Matrix Multiplication / Азимов Р., Григорьев С. // In Proceedings of the
	1st Joint International Workshop on Graph Data Management Experiences \&
	Systems (GRADES) and Network Data Analytics (NDA) (GRADES-NDA’18)
	\bibitem{2} Азимов Р. Ш. Context-Free Path Querying with Single-Path Semantics by
	Matrix Multiplication / Терехов А., Хорошев А., Азимов Р., Григорьев С. // In Proceedings of the
	3rd Joint International Workshop on Graph Data Management Experiences \&
	Systems (GRADES) and Network Data Analytics (NDA) (GRADES-NDA’20)
	\bibitem{3} Азимов Р. Ш. Context-Free Path Querying by Kronecker
	Product / Орачев Е., Эпельбаум И., Азимов Р., Григорьев С. // In Proceedings of the
	24th European Conference on Advances in Databases and Information Systems (ADBIS’20)
	\bibitem{4} Азимов Р. Ш. Context-Free Path Querying by Kronecker
	Product большая версия / Орачев Е., Эпельбаум И., Азимов Р., Григорьев С. // In Proceedings of the (SIGMOD’21)
	
	\bibitem{5} Азимов Р. Ш. Ненаписанная работа матричный алгоритм по всем путям
	
	\bibitem{6} Азимов Р. Ш. Path Querying with Conjunctive Grammars by Matrix Multiplication / Азимов Р., Григорьев С. //Programming and Computer Software. – 2019. – Т. 45. – №. 7. – С. 357-364.
	\setcounter{firstbib}{\value{enumiv}}
	
	
	
	\bibitem{7} Азимов Р. Ш. Синтаксический анализ графов с использованием конъюнктивных грамматик / Азимов Р., Григорьев С. // Труды ИСП РАН, 2018, том 1 вып. 2, с. 3-4.
	
	\bibitem{8} Азимов Р. Ш. Синтаксический анализ графов и задача генерации строк с ограничениями / Азимов Р., Григорьев С. // Сборник трудов конференции PLC 2017, с. 24-27.
\end{thebibliography}

%\vfill
%\small
%\centering
%\hrule
%\vspace{2.5pt}
%Печать\\
%Печать\\
%Печать\\
%\vspace{2.5pt}
%\hrule
%\vspace{2.5pt}
%Печать\\
%Печать\\
%Печать\\
