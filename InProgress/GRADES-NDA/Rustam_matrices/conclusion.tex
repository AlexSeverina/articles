\section{Conclusion and future work}
In this paper, we have shown how the context-free path query evaluation w.r.t. the relational query semantics can be reduced to the calculation of matrix transitive closure. Also, we provided a formal proof of the correctness of the proposed reduction. In addition, we introduced an algorithm for computing this transitive closure, which allows us to efficiently apply GPGPU computing techniques. Finally, we have shown the practical applicability of the proposed algorithm by running different implementations of our algorithm on some popular ontologies.

We can identify several open problems for further research. In this paper, we have considered only one semantics of context-free path querying but there are other important semantics, such as all-path query semantics~\cite{hellingsPathQuerying} which requires presenting all paths for all triples $(A,m,n)$. Context-free path querying implemented with the algorithm~\cite{GLL} can answer the queries in the all-path query semantics by constructing a parse forest. It is possible to construct a parse forest for a linear input by matrix multiplication~\cite{okhotin_cyk}. Whether it is possible to generalize this approach for a graph input is an open question.

In our algorithm, we calculate the matrix transitive closure naively, but there are algorithms for the transitive closure calculation, which are asymptotically more efficient. Therefore, the question is whether it is possible to apply these algorithms for the matrix transitive closure calculation to the problem of context-free path querying.

Also, there are conjunctive~\cite{okhotinConjAndBool} and Boolean grammars~\cite{okhotinBoolean}, which have more expressive power than context-free grammars. Conjunctive language and Boolean path querying problems are undecidable~\cite{hellingsRelational} but our algorithm can be trivially generalized to work on this grammars because parsing with conjunctive and Boolean grammars can be expressed by matrix multiplication~\cite{okhotin_cyk}. It is not clear what a result of our algorithm applied to this grammars would look like. Our hypothesis is that it would produce the upper approximation of a solution. Also, path querying problem w.r.t. the conjunctive grammars can be applied to static code analysis~\cite{zhang2017context}.

From a practical point of view, matrix multiplication in the main loop of the proposed algorithm may be performed on different GPGPU independently. It can help to utilize the power of multi-GPU systems and increase the performance of the context-free path querying.

There is an algorithm~\cite{apspGPU} for transitive closure calculation on directed graphs which generalized to handle graph sizes inherently larger than the DRAM memory available on the GPU. Therefore, the question is whether it is possible to apply this approach to the matrix transitive closure calculation in the problem of context-free path querying.
