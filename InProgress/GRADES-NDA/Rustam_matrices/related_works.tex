\section{Related works} \label{section_related}
Traditionally query languages for graph databases use the regular expressions to describe the required paths~\cite{reutter2017regular, fan2011adding, abiteboul1997regular, nole2016regular, graphDB} but there are some useful queries that cannot be expressed by regular expressions. For example, there are classical \textit{same-generation queries}~\cite{FndDB}, that can be used for finding all the nodes at the same level of a hierarchy, and are useful for discovering vertex similarity. The context-free path querying algorithms can be used to evaluate such types of queries since this queries can be expressed by the context-free grammars.  

There are a number of solutions~\cite{hellingsRelational, GraphQueryWithEarley, RDF} for context-free path query evaluation w.r.t. the relational query semantics, which employ such parsing algorithms as CYK~\cite{kasami, younger} or Earley~\cite{Grune}.

Hellings~\cite{hellingsRelational} presented an algorithm for the context-free path query evaluation using the relational query semantics. According to Hellings, for a given graph $D = (V, E)$ and a grammar $G = (N, \Sigma, P)$ the context-free path query evaluation w.r.t. the relational query semantics reduces to a calculation of the context-free relations $R_A$. Thus, in this paper, we focus on the calculation of these context-free relations. Also, the algorithm in~\cite{hellingsRelational} was implemented by~\cite{RDF}, in the context of RDF graphs.

Other examples of path query semantics are \textit{single-path} and \textit{all-path query semantics}~\cite{hellingsPathQuerying}. The all-path query semantics requires presenting all possible paths from the node $m$ to the node $n$ whose labeling is derived from a non-terminal $A$ for all triples $(A, m, n)$ evaluated using the relational query semantics. The single-path query semantics requires presenting only one such path for each triple $(A, m, n)$. Hellings~\cite{hellingsPathQuerying} presented algorithms for the context-free path query evaluation using the single-path and the all-path query semantics. If a context-free path query w.r.t. the all-path query semantics is evaluated on cyclic graphs, then the query result can be an infinite set of paths. For this reason, in~\cite{hellingsPathQuerying}, annotated grammars are proposed as a possible solution.

In~\cite{GLL}, the algorithm for context-free path query evaluation w.r.t. the all-path query semantics is proposed. This algorithm is based on the generalized top-down parsing algorithm~---~GLL~\cite{scott2010gll}. This solution uses derivation trees for the result representation which is more native for grammar-based analysis. The algorithms in~\cite{GLL, hellingsPathQuerying} for the context-free path query evaluation w.r.t. the all-path query semantics can also be used for query evaluation using the relational and the single-path semantics.

Our work is inspired by Valiant~\cite{valiant}, who proposed an algorithm for general context-free recognition in less than cubic time. This algorithm computes the same parsing table as the CYK algorithm but does this by offloading the most intensive computations into calls to a Boolean matrix multiplication procedure. This approach not only provides an asymptotically more efficient algorithm but it also allows us to effectively apply GPGPU computing techniques. Valiant's algorithm computes the transitive closure $a^+$ of a square upper triangular matrix $a$. Valiant also showed that the matrix multiplication operation $(\times)$ is essentially the same as $|N|^2$ Boolean matrix multiplications, where $|N|$ is the number of non-terminals of the given context-free grammar in Chomsky normal form.

Yannakakis~\cite{transitive-closure} analyzed the reducibility of various path querying problems to the calculation of the transitive closure. He formulated a problem of Valiant's technique generalization to the context-free path query evaluation w.r.t. the relational query semantics. Also, he assumed that this technique cannot be generalized for arbitrary graphs, though it does for acyclic graphs.

Thus, the possibility of reducing the context-free path query evaluation using the relational query semantics to the calculation of the matrix transitive closure is an open problem.
