\section{Introduction}
Graph data models are widely used in many areas, for example, graph databases~\cite{graphDB}, bioinformatics~\cite{Bio}. In these areas, it is often required to process queries for large graphs. The most common among graph queries are navigational queries. The result of query evaluation is a set of implicit relations between nodes of the graph, i.e. paths in the graph. A natural way to specify these relations is by specifying paths using formal grammars (regular expressions, context-free grammars) over the alphabet of edge labels. Context-free grammars are actively used in graphs queries because of the limited expressive power of regular expressions. For example, classical \textit{same-generation queries}~\cite{FndDB} cannot be expressed by regular expressions.

The result of context-free path query evaluation is usually a set of triples $(A, m, n)$ such that there is a path from the node $m$ to the node $n$, whose labeling is derived from a non-terminal $A$ of the given context-free grammar. This type of query is evaluated using the \textit{relational query semantics}~\cite{hellingsRelational}. There is a number of algorithms for context-free path query evaluation using this semantics~\cite{GLL, hellingsRelational, RDF, GraphQueryWithEarley}.

The existing algorithms for context-free path query evaluation w.r.t. this semantics demonstrate poor performance when applied to big data. One of the open problems is to create a matrix-based algorithm for context-free path query evaluation using the relational query semantics. The active use of matrix operations (such as matrix multiplication) in the process of the context-free path query evaluation makes it possible to efficiently apply such computing techniques as \textit{GPGPU} (General-Purpose computing on Graphics Processing Units) and parallel computation~\cite{matricesOnGPGPU}. From a practical point of view, matrix multiplication may be performed on different GPU independently. It can help to utilize the power of multi-GPU systems and increase the performance of the context-free path querying. Also, the algorithms for distributed-memory matrix
multiplication allow us to handle graph sizes inherently larger than the DRAM memory available on the GPU~\cite{MM_on_multi-GPU, hetero_multi-GPU, choi1994pumma}.

The algorithms for context-free language recognition had a similar problem until Valiant~\cite{valiant} proposed a parsing algorithm which computes a recognition table by computing matrix transitive closure. This algorithm works with a linear input and has the complexity which essentially the same as a Boolean matrix multiplication. One of the hard open problems is to generalize Valiant's matrix-based approach to the context-free path query evaluation. We do not achieve the same worst-case time complexity for graph input as Valiant's algorithm for linear input but we generalize this approach to graphs and provide the first matrix-based algorithm for context-free path query evaluation using the relational query semantics. Valiant's algorithm computes the transitive closure of an upper triangular matrix by increasing the length of the paths considered. We use another definition of the matrix transitive closure which does not depend explicitly on the path length since the cyclic graphs contain paths of infinite length. Additionally, we compute the transitive closure of an arbitrary matrix, since the context-free path query evaluation requires to process arbitrary graphs.

We address the problem of creating a matrix-based algorithm for context-free path query evaluation using the relational query semantics which allows us to speed up computations with GPGPU.

The main contribution of this paper can be summarized as follows:
\begin{itemize}
	\item We show how the context-free path query evaluation w.r.t. the relational query semantics can be reduced to the calculation of matrix transitive closure.
	\item We introduce a matrix-based algorithm for context-free path query evaluation w.r.t. the relational query semantics which is based on matrix operations that make it possible to speed up computations by means of GPGPU.
	\item We provide a formal proof of correctness of the proposed algorithm.
	\item We show the practical applicability of the proposed algorithm by running different implementations of our algorithm on some popular ontologies.
\end{itemize}

This paper is structured as follows: Section~\ref{section_motivating} provides a small motivating example; Section~\ref{section_preliminaries} defines some concepts used in this research; Section~\ref{section_related} presents some works closely related to our; Section~\ref{section_main} discusses our matrix-based algorithm for context-free path query evaluation and provides a step-by-step demonstration of this algorithm on a small example;  we evaluate the performance of our algorithm in Section~\ref{section_evaluation}, and we provide some concluding remarks in Section~\ref{section_conclusion}.
