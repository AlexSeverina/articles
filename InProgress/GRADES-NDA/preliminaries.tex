\section{Preliminaries} \label{section_preliminaries}
In this section, we introduce the basic notions used throughout the paper.

Let $\Sigma$ be a finite set of edge labels. Define an \textit{edge-labeled directed graph} as a tuple $D = (V, E)$ with a set of nodes $V$ and a directed edge-relation $E \subseteq V \times \Sigma \times V$.  For a path $\pi$ in a graph $D$, we denote the unique word obtained by concatenating the labels of the edges along the path $\pi$ as $l(\pi)$. Also, we write $n \pi m$ to indicate that a path $\pi$ starts at the node $n \in V$ and ends at the node $m \in V$.

Following Hellings~\cite{hellingsRelational}, we deviate from the usual definition of a context-free grammar in \textit{Chomsky Normal Form}~\cite{chomsky} by not including a special starting non-terminal, which will be specified in the path queries to the graph. Since every context-free grammar can be transformed into an equivalent one in Chomsky Normal Form and checking that an empty string is in the language is trivial it is sufficient to consider only grammars of the following type. A \textit{context-free grammar} is a triple $G = (N, \Sigma, P)$, where $N$ is a finite set of non-terminals, $\Sigma$ is a finite set of terminals, and $P$ is a finite set of productions of the following forms:

\begin{itemize}
    \item $A \rightarrow B C$, for $A,B,C \in N$,
    \item $A \rightarrow x$, for $A \in N$ and $x \in \Sigma$.   
\end{itemize}

Note that we omit the rules of the form $A \rightarrow \varepsilon$, where $\varepsilon$ denotes an empty string. This does not restrict the applicability of our algorithm because only the empty paths $m \pi m$ correspond to an empty string $\varepsilon$.

We use the conventional notation $A \xrightarrow{*} w$ to denote that a string $w \in \Sigma^*$ can be derived from a non-terminal $A$ by some sequence of applications of the production rules from $P$. The \textit{language} of a grammar $G = (N,\Sigma,P)$ with respect to a start non-terminal $S \in N$ is defined by $$L(G_S) = \{w \in \Sigma^*~|~S \xrightarrow{*} w\}.$$

For a given graph $D = (V, E)$ and a context-free grammar $G = (N, \Sigma, P)$, we define \textit{context-free relations} $R_A \subseteq V \times V$, for every $A \in N$, such that $$R_A = \{(n,m)~|~\exists n \pi m~(l(\pi) \in L(G_A))\}.$$

We define a binary operation $(~\cdot~)$ on arbitrary subsets $N_1 , N_2$ of $N$ with respect to a context-free grammar $G = (N, \Sigma, P)$ as $$N_1 \cdot N_2 = \{A~|~\exists B \in N_1, \exists C \in N_2 \text{ such that }(A \rightarrow B C) \in P\}.$$

Using this binary operation as a multiplication of subsets of $N$ and union of sets as an addition, we can define a \textit{matrix multiplication}, $a \times b = c$, where $a$ and $b$ are matrices of a suitable size that have subsets of $N$ as elements, as $$c_{i,j} = \bigcup^{n}_{k=1}{a_{i,k} \cdot b_{k,j}}.$$

According to Valiant~\cite{valiant}, we define the \textit{transitive closure} of a square matrix $a$ as $a^+ = a^{(1)}_+ \cup a^{(2)}_+ \cup \cdots$ where $a^{(1)}_+ = a$ and $$a^{(i)}_+ = \bigcup^{i-1}_{j=1}{a^{(j)}_+ \times a^{(i-j)}_+}, ~i \ge 2.$$

We enumerate the positions in the input string $s$ of Valiant's algorithm from 0 to the length of $s$. Valiant proposes the algorithm for computing this transitive closure only for upper triangular matrices, which is sufficient since for Valiant's algorithm the input is essentially a directed chain and for all possible paths $n \pi m$ in a directed chain $n < m$. In the context-free path querying input graphs can be arbitrary. For this reason, we introduce an algorithm for computing the transitive closure of an arbitrary square matrix.

For the convenience of further reasoning, we introduce another definition of the transitive closure of an arbitrary square matrix $a$ as $a^{cf} = a^{(1)} \cup a^{(2)} \cup \cdots$ where $a^{(1)} = a$ and $$a^{(i)} = a^{(i-1)} \cup (a^{(i-1)} \times a^{(i-1)}), ~i \ge 2.$$

To show the equivalence of these two definitions of transitive closure, we introduce the partial order $\succeq$ on matrices with the fixed size which have subsets of $N$ as elements. For square matrices $a, b$ of the same size, we denote $a \succeq b$ iff $a_{i,j} \supseteq b_{i,j}$, for every $i, j$. For these two definitions of transitive closure, the following lemmas and theorem hold.

\begin{lemma}\label{lemma:cf_geq_valiant}
	Let $G =(N,\Sigma,P)$ be a grammar, let $a$ be a square matrix. Then $a^{(k)} \succeq a^{(k)}_+$ for any $k \geq 1$.
\end{lemma}
\begin{proof}(Proof by Induction)
	
	\textbf{Basis}: The statement of the lemma holds for $k = 1$, since $$a^{(1)} = a^{(1)}_+ = a.$$
	
	\textbf{Inductive step}: Assume that the statement of the lemma holds for any $k \leq (p - 1)$ and show that it also holds for $k = p$ where $p \geq 2$. For any $i \geq 2$ $$a^{(i)} = a^{(i-1)} \cup (a^{(i-1)} \times a^{(i-1)}) \Rightarrow a^{(i)} \succeq a^{(i-1)}.$$ Hence, by the inductive hypothesis, for any $i \leq (p-1)$ $$a^{(p-1)} \succeq a^{(i)} \succeq a^{(i)}_+.$$ Let $1 \leq j \leq (p - 1)$. The following holds $$(a^{(p-1)} \times a^{(p-1)}) \succeq (a^{(j)}_+ \times a^{(p-j)}_+),$$ since $a^{(p-1)} \succeq a^{(j)}_+$ and $a^{(p-1)} \succeq a^{(p-j)}_+$. By the definition, $$a^{(p)}_+ = \bigcup^{p-1}_{j=1}{a^{(j)}_+ \times a^{(p-j)}_+}$$ and from this it follows that $$(a^{(p-1)} \times a^{(p-1)}) \succeq a^{(p)}_+.$$ By the definition, $$a^{(p)} = a^{(p-1)} \cup (a^{(p-1)} \times a^{(p-1)}) \Rightarrow a^{(p)} \succeq (a^{(p-1)} \times a^{(p-1)}) \succeq a^{(p)}_+$$ and this completes the proof of the lemma.
\end{proof}

\begin{lemma}\label{lemma:valiant_geq_cf}
	Let $G =(N,\Sigma,P)$ be a grammar, let $a$ be a square matrix. Then for any $k \geq 1$ there is $j \geq 1$, such that $(\bigcup^{j}_{i=1}{a^{(i)}_+}) \succeq a^{(k)}$.
\end{lemma}
\begin{proof}(Proof by Induction)
	
	\textbf{Basis}: For $k = 1$ there is $j = 1$, such that $$a^{(1)}_+ = a^{(1)} = a.$$ Thus, the statement of the lemma holds for $k = 1$.
	
	\textbf{Inductive step}: Assume that the statement of the lemma holds for any $k \leq (p - 1)$ and show that it also holds for $k = p$ where $p \geq 2$. By the inductive hypothesis, there is $j \geq 1$, such that $$(\bigcup^{j}_{i=1}{a^{(i)}_+}) \succeq a^{(p-1)}.$$ By the definition, $$a^{(2j)}_+ = \bigcup^{2j-1}_{i=1}{a^{(i)}_+ \times a^{(2j-i)}_+}$$ and from this it follows that $$(\bigcup^{2j}_{i=1}{a^{(i)}_+}) \succeq (\bigcup^{j}_{i=1}{a^{(i)}_+}) \times (\bigcup^{j}_{i=1}{a^{(i)}_+}) \succeq (a^{(p-1)} \times a^{(p-1)}).$$ The following holds $$(\bigcup^{2j}_{i=1}{a^{(i)}_+}) \succeq a^{(p)} = a^{(p-1)} \cup (a^{(p-1)} \times a^{(p-1)}),$$ since $$(\bigcup^{2j}_{i=1}{a^{(i)}_+}) \succeq (\bigcup^{j}_{i=1}{a^{(i)}_+}) \succeq a^{(p-1)}$$ and $$(\bigcup^{2j}_{i=1}{a^{(i)}_+}) \succeq (a^{(p-1)} \times a^{(p-1)}).$$ Therefore there is $2j$, such that $$(\bigcup^{2j}_{i=1}{a^{(i)}_+}) \succeq a^{(p)}$$ and this completes the proof of the lemma.	
\end{proof}

\begin{mytheorem}\label{thm:closures}
	Let $G =(N,\Sigma,P)$ be a grammar, let $a$ be a square matrix. Then $a^+ = a^{cf}$.
\end{mytheorem}
\begin{proof}
	
	By the lemma~\ref{lemma:cf_geq_valiant}, for any $k \geq 1$, $a^{(k)} \succeq a^{(k)}_+$. Therefore $$a^{cf} = a^{(1)} \cup a^{(2)} \cup \cdots \succeq a^{(1)}_+ \cup a^{(2)}_+ \cup \cdots = a^+.$$ By the lemma~\ref{lemma:valiant_geq_cf}, for any $k \geq 1$ there is $j \geq 1$, such that $$(\bigcup^{j}_{i=1}{a^{(i)}_+}) \succeq a^{(k)}.$$ Hence $$a^+ = (\bigcup^{\infty}_{i=1}{a^{(i)}_+}) \succeq a^{(k)},$$ for any $k \geq 1$. Therefore $$a^+ \succeq a^{(1)} \cup a^{(2)} \cup \cdots = a^{cf}.$$ Since $a^{cf} \succeq a^+$ and $a^+ \succeq a^{cf}$, $$a^+ = a^{cf}$$ and this completes the proof of the theorem.
\end{proof}

Further, in this paper, we use the transitive closure $a^{cf}$ instead of $a^+$ and, by the theorem~\ref{thm:closures}, an algorithm for computing $a^{cf}$ also computes Valiant's transitive closure $a^+$.
