\section{Introduction}

One of useful type of graph queries is language-constarined path queries~\cite{FLCpathProblem}.
There is a number of languages for graph traversing/querying which support constrains formulated in terms of regular languages.
For example SPARQL~\cite{sparql}, Cypher~\footnote{Cypher languge web page: \url{https://neo4j.com/developer/cypher-query-language/}. Access date: 16.01.2018}, and Gremlin~\cite{gremlin}.
In this work we are focused on context-free path queries (CFPQ) which use context-free languages for constrains specification and used in such areas as bioinformatics~\cite{GraphQueryWithEarley}, static code analysis~\cite{Reps, Zheng, LabelFlowCFLReachability, specificationCFLReachability}, RDF processing~\cite{CFGonRDF}. 
There are a lot of theoretical research and problem-specific solutions on CFPQ~\cite{Yannakakis, ConjCFPathQuery, Hellings16, QueryGraphWithData, RegularDBQuery, GraphQueryWithEarley, graphDB}, but there is only one known graph query languege with support of CF constraints: cfSPARQL~\cite{CFGonRDF}.

When you develop a data-centric application, you want to use general purpose programming language and have an transparent and native access to data sources.
The naiv approach, named string-embedded DSLs, which means that you have a driver which can execute a string with query and return (posibly untyped) result have some serious problems.
First of all, it is necessary to use special DSL which may require additional knolages from developer.
Moreover, string-embedded languahe is a source of errors and vulnerabilities which static detection is very hard~\cite{stringEmbeddedLanguagesProblem}.  
These leeds to creation such special techniques as Object Relationship Mapping (ORM) or Language Integrated Query~(LINQ)~\cite{LINQ1, LINQ2} which solve some problems, but have some difficulties with flexibility, for example with query decomposition and subquery reusing.
In this paper we propose solution for naural transperent integration of CFPQs into general-purpose language. 

One of the natural way to specify any language is specify its formal grammar which can be done by using special DSL based, for example, on EBNF-like notation~\cite{EBNFISO}.
The classical alternative way is a pasrer combinators technique which provide !!!Ekaterina, we need your help!!.

Unfortunately, classical combinators are based on LL(k) top-down parsing technique and can not handle left recursive and umbigues grammars~\cite{!!!}.
In~\cite{Meerkat} authors show that it is possible to use ideas of Generalized LL~\cite{scott2010gll} (GLL) for creation parser combinators which can handle arbitrary context-free grammars.
Meerkat~\footnote{Meerkat project repository:~\url{https://github.com/meerkat-parser/Meerkat}. Access date: 16.01.2018} parser combinators library is bsed on~\cite{Meerkat} and provide Shared Packed Parse Forest~\cite{SPPF}~(SPPF) --- compact representation of parsing result. 
Furthermore, there is a solution which use GLL for CFPQ~\cite{GrigorevR16} and it is showed that SPPF can be used is finite structural representation of query result even when set of paths is infinite.
It is important feature because such tasks as paths extraction, queries debuggibg and result processing~\cite{structForDebugging} require appropriate representation of query result.

On the other hand, an idea to use combinators for graph traversing proposed in~\cite{ScalaGraphParsing}, but presented solution provide approximated handling of cycles in input graph and have a problem with left-recursive grammars. 
Authors pointed out that described idea is very close to classical parser combinators technique, but supported language class and restrictions are not discussed.

In this paper we show how to compose these ideas and present parser combinators for CFPQ which can handle arbitrary context-free grammar and provide structural representation of result.
We make the following contributions in this paper.

\begin{enumerate}
\item We show that it is possible to create parser combinators for context-free path querying which supports both arbitrary contex-free grammars and arbitrary graphs and provide finite structural representation of query result.
\item We provide the implementation of parser combinators library in Scala. This library provide integration with Neo4J graph data base. Source code available on GitHub:\url{https://github.com/YaccConstructor/Meerkat}.
\item We perform an evaluation on realistic data. 
Also we compare performance of our library with other GLL-bsed CFPQ tool and with Trails library.
As a result we conclude that our solution provide good expressivity and performance to be applied for real-world problems. 
\end{enumerate}