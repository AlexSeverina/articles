\newcommand{\CS}{C\nolinebreak\hspace{-.05em}\raisebox{.4ex}{\scriptsize\bf \#}}

Static code analysis is an important part of modern software development tools.
It takes care of verifying the correctness of some program's behavior freeing a programmer from this duty.
Static analysis can be classified as intraprocedural or interprocedural depending on which scope of a program it uses. Intraprocedural analysis considers only one procedure whereas interprocedural inspects a program as a whole.
Interprocedural analyses, in theory, are more precise due to the amount of available information.

\begin{figure}[h]
	\includegraphics[width=\linewidth]{pictures/{SampleCode.dia}.png}
	\caption{Sample code in C\# which represents a case for interprocedural taint analysis}
	\label{fig:SampleCode}
\end{figure}

One classical problem which requires interprocedural analysis is taint tracking problem (e.g. described in \cite{TaintAnalysis}).
Input data can have an inappropriate format or contain an exploit such as SQL injection.
Such data is called \textit{tainted}.
Tainted data can lead to incorrect behavior in case of incorrect format, or can cause a security issue.

A simple example of C\# code which may require taint analysis is presented in fig.~\ref{fig:SampleCode}.
We use this example to illustrate our solution.
First of all, we introduce a number of entities which play important roles in analysis.
The first is \textit{source}: a field that potentially contains the tainted data, for example, the field \textit{Source} of the class \textit{A}.
The second is \textit{sink}: a method which is vulnerable to tainted data.
In our example, the method \textit{Sink} of the class \textit{C} has this property.
The third important type of entities is \textit{filter} (or \textit{sanitizer} in some other definitions).
It is a method that checks the correctness of data passed into it.
If data is incorrect, \textit{filter} throws an exception or modifies data to ensure the correctness of the result.
The method \textit{Filter} of the class \textit{B} in the given snippet is a filter.

We assume that each entity is annotated by a programmer with an appropriate attribute: \textit{[Tainted]} for sources, \textit{[Filter]} for filters and \textit{[Sink]} for sinks.
The taint analysis problem is to find all traces such that data can flow into a sink bypassing any filter.

In our example the class \textit{D} extracts data using the class \textit{A} as a source than validates it using class \textit{B}, after that performs some computations involving the class \textit{C}.
Invokation of the method \textit{Consume} leads to using the data which have not been properly validated.
To identify such issues we need to run the taint analysis.

One of the well-known frameworks for interprocedural static code analysis is a CFL-reachability framework proposed by Thomas Reps~\cite{Reps}.
This framework is generic: it provides the abstraction which allows one to implement many different types of interprocedural static code analysis, such as pointer analysis~\cite{Zheng, JavaCFL}, taint analysis~\cite{Huang:2015:SPT:2771783.2771803}, label flow analysis~\cite{10.1007/11823230_7,CFLr}, library summarization~\cite{10.1007/978-3-662-54434-1_33}.
Moreover, CFL-reachability is a long-time studied framework and thus there is a number of specific solutions which demonstrate reasonable performance in practice~\cite{Wang:2017:GSD:3093315.3037744}.

It is important for static code analysis to be flow-sensitive that means correct handling call-return pairs: when we have an interprocedural graph, we have no explicit synchronization of call and return edges.
As a result, one can find a path which contains incorrect sequences of calls and returns.
Also, it is important to provide correct handling of read-write, get-set and other pairs of edges (operations).
Note, that CFL-reachability framework allows one to implement analysis which can be flow-sensitive or context-sensitive, but not both~\cite{Reps:2000:UCD:345099.345137}.
In order to provide both flow- and context-sensitivity one can use more powerful framework~\cite{Zhang:2017:CDA:3093333.3009848}.
In our examples, we implement flow-sensitive analysis, but one can modify it to be context-sensitive.

The main idea of CFL-reachability is to find paths in a graph that satisfy constraints defined by a context-free grammar.
In particular, a path is accepted if the concatenation of labels on its edges forms a word which can be derived in the grammar.
However, in practice, there are few drawbacks of such a grammar-based definition.
Firstly, grammar-driven parsing is based on exact matching of terminals which forces to generate a very large grammar in the case when edges contain some unique attributes.
For example, brackets matching described in~\cite{CFLr} or~\cite{Zheng, JavaCFL} requires to generate as many rules as there are call sites in the source code.
Secondly, grammar provides an additional level of abstraction.
This level is useful for algorithms formalization, but it is not as flexible as direct manipulation with correspondent push-down automaton.

CFL-reachibility requires a graph representing the whole program.
It is not feasible to construct such a graph from scratch each time a programmer modifies a program.
We need to ensure that the graph is modified with minial effort.
To create a generic framework we also need to make sure the graph representation is generic.

The main goal of our work is to implement a tool solving all mentioned problems and thus allowing to use CFL-reachability in practice.
We make the following contributions in the paper.
\begin{itemize}
	\item We describe the scalable representation of a graph which allows containing as much information about the program as necessary.
	\item We introduce the approach for the definition of constraints based on pushdown automata instead of grammars which eases formulation of analyses.
	\item We present the implementation of the proposed approach which allows one to create new types of analysis using introduced abstractions.
	Also we implement\footnote{Source code and executables can be downloaded here: \url{github.com/gsvgit/CoFRA}.} a plugin which uses the solution to provide analysis results to ReSharper, Rider, and InspectCode.
	\item We evaluate the implementation on both synthetic tests and large open source projects.
\end{itemize}

%Another approach is abstract interpretation which main idea is to define the semantics abstracting the concrete one.
%We propose to combine these two approaches to achieve acceptable performance and expressive power.
%I.e. the program is translated into a graph as it is in CFL-r, but constraints that specifiy what paths is needed to be accepted are set by pushdown automaton which transition relation simulates the semantics of original program.
%Let's take a closer look at these two components that define an analysis in conjunction.
