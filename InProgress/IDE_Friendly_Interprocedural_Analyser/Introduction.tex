\newcommand{\CS}{C\nolinebreak\hspace{-.05em}\raisebox{.4ex}{\scriptsize\bf \#}}

Static analyses are important part of modern development tools. 
They take care of verifying correctness of some program's behaviour freeing a programmer from this duty.
By used scope of program, an analyze can be classified as intraprocedural or interprocedural, i.e. as those which make decisions based on only one current procedure or based on the whole program respectively.
And interprocedural analyses, in theory, can be more precise due to amount of available information.

\begin{figure}[h]
	\includegraphics[width=\linewidth]{pictures/{SampleCode.dia}.png}
	\caption{Sample code}
	\label{fig:SampleCode}
\end{figure}

For example, let's consider the listing in \CS at fig.~\ref{fig:SampleCode}.
It is known that method \textit{Sink} is vulnerable to invalid arguments and so, it is marked by appropriate attribute.
The method \textit{Filter} validates its argument and possibly modifies it to ensure that returned data is definitely valid. Methods with such behaviour are marked with attribute \textit{Filter}. The most common example of such method is the one that adds escape characters into the string to avoid affecting internal behaviour of the system by user input.
And the field \textit{Source} is known as potentially tainted that is also indicated by attribute.
Class \textit{D} extracts data using class \textit{A} as a source, then validates it using class \textit{B} then performs some computations involving class \textit{C}.
So, there might appear an issue that leads to usage of data that are not been validated yet as it happens during the invocation of method \textit{Consume}.
The problem is to find all such issues, i.e. we want an analysis that finds all possible ways how data from source can reach sink bypassing filters.

This problem is a special case of label-flow analysis, so there are several approaches of solving such problems.
One of them is CFL-reachability. (TODO: ADV: performance, DISADV: expressive power, non-obvious structure, CITATIONS)
Another is abstract interpretation. (TODO: vice versa)
We propose to combine these two approaches to achieve acceptable performance and expressive power.
I.e. the program is translated into a graph as it is in CFL-r, but constraints that specifiy what paths is needed to be accepted are set by pushdown automaton which transition relation simulates the semantics of original program.
Let's take a closer look at these two components that define an analysis in conjunction.
