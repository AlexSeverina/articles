\section{Conclusion}

We propose and implement in C\# programming language the generic framework for interprocedural static code analysis implementation.
This framework allows one to implement arbitrary interprocedural analysis in terms of CFL-reachability.
By using the proposed framework, we implement a plugin upon ReSharper infrastructure which provides simple taint analysis and demonstrate that our solution can handle important real-world cases.
Also we show that the proposed framework can be used for real-world solutions analysis.

One of the directions for future work is a creation of analysis and its evaluation on real-world projects.
By this way, we want to get information which helps to improve the usability of our framework: tune performance, improve API, etc.
Also we should improve documentation and create more examples of usage.

Another direction is a practical evaluation of automatic fix location prediction by using minimum cuts method~\cite{10.1007/978-3-319-63390-9_27}.

Also we want to compare the proposed approach with other generic CFL-reachability based approaches for interprocedural code analysis cretion. For example, fith generation-based approach~\cite{LPAR-21:Cauliflower_Solver_Generator_for}, which idea is similar to parser generators.
