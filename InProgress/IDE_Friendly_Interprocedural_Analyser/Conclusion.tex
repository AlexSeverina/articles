\section{Conclusion}

We propose and implement in C\# the generic framework for interprocedural static code analysis.
This framework allows one to implement arbitrary interprocedural analysis in terms of CFL-reachability.
With the proposed framework, we implement a plugin upon ReSharper infrastructure which provides simple taint analysis and demonstrate that our solution can handle important real-world cases.
We show that the proposed framework can be used for real-world analysis.

One possible direction for future work is improving the implemented framework: tuning performance and improving APIs. 
We believe, that the best way to do this is by employing the framework for more analyses and real-world projects. 

Another direction is a practical evaluation of automatic fix location prediction by using minimum cuts method~\cite{10.1007/978-3-319-63390-9_27}.
Such function may be helpful for end-users: tool can propose possible fixes of the detected problem, not only report on it.

We want to compare the proposed approach with other generic CFL-reachability based approaches for interprocedural code analysis such as the generation-based approach~\cite{LPAR-21:Cauliflower_Solver_Generator_for}.
We should compare both frameworks and specific tools created with frameworks.
