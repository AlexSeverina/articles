\section{Conclusion}

We propose and implement in C\# the generic framework for interprocedural static code analysis.
This framework allows one to implement arbitrary interprocedural analysis in terms of CFL-reachability.
With the proposed framework, we implement a plugin upon ReSharper infrastructure which provides simple taint analysis and demonstrate that our solution can handle important real-world cases.
We show that the proposed framework can be used for real-world analysis.

One possible direction for future work is improving the implemented framework: tuning performance and improving APIs. 
We believe, that the best way to do this is by employing the framework for more analyses and real-world projects. 

Another direction is a practical evaluation of automatic fix location prediction by using minimum cuts method~\cite{10.1007/978-3-319-63390-9_27}.
Such function may be helpful for end-users: tool can propose possible fixes of the detected problem, not only report on it.

CFL-reachability based approaches and their practical applications to interprocedural code analysis are widely developed in a number of modern tools.
Cauliflower~\cite{LPAR-21:Cauliflower_Solver_Generator_for} generates analyzers by grammar based specification and produces a fixed solver implemented in C++ for each type of analysis.
Although it reaches an outstanding performance, the resulting analyzers is still faced with the problem of flexibility of grammars.
It requires the graph to be labeled with terminals which are already defined in the grammar and because of this it is hard to use the same graph for two different types of analysis or use the result of first type to modify behaviour of the second analyzer.
Graspan~\cite{Wang:2017:GSD:3093315.3037744} uses the same grammar-driven approach and successfully analyses large systems such as Linux and PostgreSQL.
It also proves that CFL-reachability framework is practically applicable.

However, one of our goals is to alter CFL-reachability based analysis to make it closer to orginal execution semantics and also avoid problems inherited from grammars.
The similar idea is described in the article~\cite{Facchinetti:2019:HDP:3343145.3310340}.
There suggested to approximate the operational semantics of a language with a pushdown system to make the interpretation decidable.

So, our solution unites the ideas of all listed developments to provide flexible and intuitive abstraction solving existing problems of CFL-reachability framework.
