
%I.e. there are a graph which is image of the source code and constraints that describe the sort of paths in the graph we want to select.

To apply the CFL-reachability framework, the first thing to do is to define the representation of a graph and path restrictions.
We assume that a graph is an image of a program and it stores the information about the source code that can be mapped back to locate issues in the sources which are found by the analysis of the graph.
Then restrictions on paths define the sequences of operations leading to an issue.

In order to keep the expressive power of the CFL-reachability approach but make the implementation easier, we propose to use pushdown automata instead of grammars which still have equivalent power~\cite{AutomataTheory}.
Further, we take a closer look at each component and consider the construction of them in application to our example.

\subsection{Graph extraction}

The graph that is explored during the analysis is the union of control flow graphs each of which is extracted from one distinct method.
The graph which corresponds to our example is shown in fig.~\ref{fig:SampleGraph}.

\begin{figure}[h]
	\includegraphics[width=\linewidth]{pictures/{SampleGraph.dia}.png}
	\caption{Extracted graph for the code presented in fig.~\ref{fig:SampleCode}}
	\label{fig:SampleGraph}
\end{figure}

Each edge contains an operation that represents a statement in the source code and at the same time, the target of the edge indicates the target to jump after the execution of the operation.
We classify operations into a few types and operations which have the same type differs only by unique attributes.
However, the number of types is not fixed and some analysis-specific operations can be added if necessary.
In our example, we consider three different types of operations: invocations, assignments, and returns.
Each of them is an image of some source code instruction.
Invocations are produced from call sites and have the same information as ones in the original code.
Their notation has the following form:
\begin{equation}
\begin{split}
	& \text{invoke:} \\
	& o.m \rightarrow v \\
	& (f_1 := c_1; \ldots; f_k := c_k)
\end{split}
\end{equation}
Where $o$ is an object or a class which method is called, $m$ is the name of a method, $v$ is a variable where the result is stored and $f_i, c_i$ are pairs of the formal and the actual parameter respectively.

Each assignment corresponds to a real assignment and is written in the following way:
\begin{equation}
\begin{split}
	& \text{assign:} \\
	& s \rightarrow t
\end{split}
\end{equation}
Where $s$ is a source of data and $t$ is the target variable.

And a return statement just indicates the end of a method. It can be not present explicitly in the source code but is still needed to be added to inform the analyzer about the return point.

Nodes have no data and correspond to positions between instructions in the source code.

So, we have a bunch of graphs each of which represents the content of one method.
Next step is to interconnect them to have an opportunity to perform interprocedural jumps during invocations.
There are several possible ways to do that.
First of them is to expand invocations statically, i.e. add a pair of edges for each target of each invocation.
One to represent a jump from the call site to the entry point of target and one to emulate return from the final node of the method to the caller.
But this approach leads to the need to update all these additional connections if some method is removed or its body is changed.
So, we propose to resolve invocations dynamically right during an analysis.
It allows us to use the graph which is composed right of graphs of methods and has no additional edges.
Also, it does not require to modify any other method when someone is updated.
Nevertheless, it is still needed to have a mechanism that can collect all targets of any invocation using references stored there.

To implement such a mechanism, called resolver, we offer to accumulate some meta-information about the program besides graphs themselves.
The relations in required data is shown at fig.~\ref{fig:Metadata}.

\begin{figure}[h]
	\includegraphics[width=\linewidth]{pictures/{TopLevelEntitiesHierarchy.dia}.png}
	\caption{Structure of the metadata for extracted graph}
	\label{fig:Metadata}
\end{figure}

Firstly, it is important to keep the hierarchy of inheritance to support polymorphic calls and invocations of methods of a basic class.
Secondly, it is needed to know which methods are contained in each class to find the method by its name and its location.
Thirdly, methods can have local functions and it is necessary to keep their hierarchy too to support, for example, anonymous function invocations, delegates passing and so on.
And finally, methods themselves has references to nodes they own which is used to find the entry point and update the graph when the body of method is changed.
This structure also contains such data as class fields and local variables of a method which can be referenced by operations.
So, the resolver takes the class name and the identifier of a method or a field and walks through the hierarchy trying to find all suitable entities.

\subsection{Push-down automata construction}

Further, we need to define restrictions on paths in the graph in terms of pushdown automata.
Formally, nondeterministic pushdown automaton~\cite{AutomataTheory} is a tuple $(Q, \Sigma, \Gamma, \delta, q_0, Z_0, F)$, where $Q$, $\Sigma$ and $\Gamma$ are finite sets of states, input symbols and stack symbols respectively, $q_0 \in Q$ and $Z_0 \in \Gamma$ are initial state and stack symbol, $F \subseteq Q$ is a set of final states and $\delta: Q \times \Sigma \cup \{\epsilon\} \times \Gamma \rightarrow \mathcal P (Q \times \Gamma^*)$ is a transition relation which takes the current state, an input symbol, the top of the stack and computes a new state and the sequence of stack symbols which must replace the top one.
We also add the following restriction on the transition relation.
The resulting stack must differ from the source one by no more than one top symbol. I.e. only one symbol can be pushed or popped during the transition.
It does not affect the expressive power of the resulting abstraction because push of a sequence of symbols can be emulated using the chain of states connected by $\varepsilon$-transitions each of which pushes one symbol.

Next, we propose to take the set of all edges in the control flow graph as $\Sigma$.
%So, the transition relation can be understood as a structural operational semantics that defines how a configuration is changed during the execution of a statement.
All other sets can be taken arbitrary.

However, there is one more problem.
Semantics of invocation contains the need to make a jump from the current position to the entry point of the target instead of just going to the node that is pointed by the current edge.
To support such behaviour we propose to change the codomain of $\delta$ such that $\delta: Q \times \Sigma \cup \{\epsilon\} \times \Gamma \rightarrow \mathcal P (Q \times \Gamma^* \times N \cup \{\nu\})$, where $N$ is the set of nodes of the graph and $\nu$ is the dummy value that means that there is no need to jump and PDA just goes to the next node.

As an example, we construct the PDA performing the taint tracking analysis described in the introduction.
Let $Q := V \cup \{q_0, q_f\}$, where $V$ is set of all local variables of all methods and $q_0$ and $q_f$ are dummy initial and final state, so $F \coloneqq \{q_f\}$.
$\Gamma \coloneqq I \cup \{Z_0\}$, where $I \subset \Sigma$ is set of all edges containing an invocation and $Z_0$ is dummy initial stack symbol.
And $\delta$ is defined by the case analysis~(\ref{fig:TransitionRelation}).
\begin{equation}
	\label{fig:TransitionRelation}
	\begin{split}
		&\begin{split}
			1)\ \delta(q_0, & i@\textrm{invocation}, \gamma) \coloneqq \\
			\{& (q_0, i\gamma, s_0), \ldots, (q_0, i\gamma, s_n), (q_0, \gamma, \nu): \\
			& s_0, \ldots, s_n \in R(i)\} \\
		\end{split} \\
		&\begin{split}
			2)\ \delta(q_0, & a@\textrm{assignment} (v_s, v_t), \gamma) \coloneqq \\
			&\begin{cases}
				\{(v_t, \gamma, \nu), (q_0, \gamma, \nu)\},& \textrm{if}\ \textit{source}(v_s) \\
				\{(q_0, \gamma, \nu)\},& \textrm{otherwise}
			\end{cases}
		\end{split} \\
		&\begin{split}
			3)\ \delta(v, a@\textrm{assignment}(v, v_t), & \gamma) \coloneqq {(v_t, \gamma, \nu)}
		\end{split} \\
		&\begin{split}
			4)\ \delta(v, & i@\textrm{invocation}, \gamma) \coloneqq \\
			&\begin{split}
				\bigcup_{j=0}^{n} \{&(v_{j0}, i\gamma, s_j), \ldots, (v_{jm}, i\gamma, s_j), (v, \gamma, \nu): \\
				& v_{jk} \in A(i, j, v)\}, s_j \in R(i)
			\end{split}
		\end{split} \\
		&\begin{split}
			5)\ \delta(v, & r@return, i@invocation) \coloneqq \\
			&\begin{cases}
				\{(RV(i), \epsilon, T(i))\},& \textrm{if}\ \textit{returned(v)} \\
				\varnothing, & \textrm{otherwise}
			\end{cases}
		\end{split} \\
		& 6)\ \delta(q, \_, \gamma) := \{(q, \gamma, \nu)\}
	\end{split}
\end{equation}
Where notation $v@\textit{pattern}$ means that $v$ must be an object which is constructed by \textit{pattern}, \textit{source} checks if a variable is a source, \textit{returned} checks if current variable is a return value of some method, $T$ returns target node of an edge, $RV$ returns the variable where the result of an invocation must be put, $R$ is the resolver returning entry points of all possible targets of an invocation and $A$ is defined by equation~(\ref{fig:FunctionA}).
\begin{equation}
	\label{fig:FunctionA}
	A(i, j, v) \coloneqq
	\begin{cases}
		\{q_f\},  \begin{split}&\textrm{if $j$-th target of invocation $i$} \\ & \textrm{is sink and $v$ is its argument}\end{split} \\
			\{v_k: v \mapsto v_k\}, \begin{split}&\textrm{if $j$-th target of i} \\ & \textrm{is not filter}\end{split} \\
		\varnothing, \textrm{otherwise}
	\end{cases}
\end{equation}
Where $v \mapsto v_k$ means that $v$ is passed as $k$-th parameter and becomes local variable $v_k$ of the target after passing.

\subsection{Analysis execution}

In this section we describe how the constructed automaton can be used to find an issue in the sample source code.
The goal is to find the sequence of operations which starts in the entry point and ends at the invocation which uses the data from an unfiltered source.
Since the automaton accepts such sequences, we can simulate the switching of its configurations according to the input statements taken from the source graph.
We use the following notation for the PDA steps description.
$(q, \gamma_1 \ldots \gamma_k, n)$ is the current configuration of the simulation where $q$ is the current state, $\gamma_j$ is a symbol on the stack and $n$ is the current position in the input graph.
$c_1 \xrightarrow[k]{r} c_2$ is the $k$-th transition which switches configuration $c_1$ to $c_2$ using rule $r$ from equation~(\ref{fig:TransitionRelation}).
Local variables are written as $\text{\textlangle{}Name of the containing method\textrangle{}}.\text{\textlangle{}Variable identifier\textrangle{}}$.
Since the automaton is non-deterministic, it can produce a graph of configurations, so, in this example, we explore only the branch where the final state is reached.

According to our example, we start with the configuration $(q_0, Z_0, 1)$.
Configurations which appear after the first step are produced by accepting the first rule to the current configuration and the input symbol located at the edge between nodes 1 and 2.
First of them is the one corresponding to the performed invocation of the \textit{Read} method, the invocation statement is pushed onto the stack and the position is changed to 7. Second of them is the branch where invocation is just skipped.
Since the second configuration does not lead to an error, we continue with the first configuration $(q_0, i_1 Z_0, 7)$ where $i_1$ is the invocation statement.
Next step changes the state to the local \textlangle{}result\textrangle{} variable according to the rule 2 because the right part of the assignment is a source. Configuration switches to $(\text{Read.\textlangle{}result\textrangle{}}, i_1 Z_0, 8)$.
Further step processes the return statement using rule 5 and performs two important actions.
Firstly, it pops the invocation stored on the top of the stack and jumps to the return point.
Secondly, there is performed a change of currently tracked variable from the local one of the method \textit{Read} to the local variable \textit{d} of the method \textit{Process} because it stores a result of the invocation.
So, next configuration is $(\text{Process.d}, Z_0, 2)$.
Processing of the edge between nodes 2 and 3 uses rule 4 and produces only one branch which just skips the invocation because the target is a filter and there is no need to enter this method.
However, the invocation of \textit{Consume} is processed fairly and there is produced the configuration $(\text{Consume.d}, i_2 Z_0, 11)$ which, further, iterates until the \textit{Sink} invocation and reaches the final state $q_f$.

The full chain of configurations is shown in equation~\ref{fig:Configurations}.

\begin{equation}
	\label{fig:Configurations}
	\begin{split}
		& (q_0, Z_0, 1) \xrightarrow[1]{1} (q_0, i_1 Z_0, 7) \xrightarrow[2]{2} \\
		& (\text{Read.\textlangle{}result\textrangle{}}, i_1 Z_0, 8) \xrightarrow[3]{5} \\
		& (\text{Process.d}, Z_0, 2) \xrightarrow[4]{4} (\text{Process.d}, Z_0, 3) \xrightarrow[5]{4} \\
		& (\text{Process.d}, i_2 Z_0, 10) \xrightarrow[6]{4} (\text{Consume.d}, i_2 Z_0, 11) \xrightarrow[7]{4} \\
		& (q_f, i_3 i_2 Z_0, \text{\textlangle{}\textit{Sink} entry point\textrangle{}})
	\end{split}
\end{equation}

So, the path in the graph which is present in equation~(\ref{fig:Path}) is accepted by the automaton and contains an issue.

\begin{equation}
	\label{fig:Path}
	\begin{split}
		& 1 \xdashrightarrow{\begin{subarray}{l} \text{invoke:} \\ \text{this.Read $\rightarrow$ local d} \\ \text{(a := local a)} \end{subarray}} 7
			\xrightarrow{\begin{subarray}{l} \text{assign:} \\ \text{(local a).Source $\rightarrow$ \textlangle{}result\textrangle{}} \end{subarray}} \\
		& 8 \xdashrightarrow{\text{return}} 2 \xrightarrow{\begin{subarray}{l} \text{invoke:} \\ \text{B.Filter $\rightarrow$ local f} \\ \text{(d := local d)} \end{subarray}} 3
		 	\xdashrightarrow{\begin{subarray}{l} \text{invoke:} \\ \text{this.Consume} \\ \text{(d := local d)} \end{subarray}} \\
		& 10 \xrightarrow{\begin{subarray}{l} \text{invoke:} \\ \text{C..ctor $\rightarrow$ local c} \end{subarray}} 11
			\xdashrightarrow{\begin{subarray}{l} \text{invoke:} \\ \text{(local c).Sink} \\ \text{(d := local d)} \end{subarray}} \text{\textlangle{}\textit{Sink} entry point\textrangle{}}
	\end{split}
\end{equation}

Where dashed arrows indicate jumps and solid ones correspond to straightforward transitions.
