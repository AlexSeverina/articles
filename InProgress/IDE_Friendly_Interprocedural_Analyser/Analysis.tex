\subsection{Graph extraction}

The graph that is explored during analysis is an aggregate of control-flow graphs of each method.
The one that corresponds to our example is shown at fig.~\ref{fig:SampleGraph}. 

\begin{figure}[h]
	\includegraphics[width=\linewidth]{pictures/{SampleGraph.dia}.png}
	\caption{Sample graph}
	\label{fig:SampleGraph}
\end{figure}

Each edge contains an operation that represents a statement in the source code and in the same time the target of the edge indicates where to jump after execution of the operation.
For example, there exist three different types of operations: invocations, assignments and returns.
Each of them is an image of some source instruction. 
Invocations are produced from call sites and have the same information as ones in original code.
I.e. it contains a reference to the entity which method is called, the name of target, a set of arguments and the information about returned values.
Each assignment corresponds to real assignment and has references to the source entity and the target one.
And return just indicates the end of a method. It can be not present directly in the source code but is needed to be added to 

\subsection{PDA construction}
