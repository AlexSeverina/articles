\section{Evaluation}
\label{sec:evaluation}

In this section, we present an evaluation of our graph querying library.
We measure its performance on a classical set of ontology graphs~\cite{CFGonRDF}: both when the graph is loaded into the RAM and for the integration with the Neo4j database and compare it with the solution based on the GLL parsing algorithm~\cite{GrigorevR16} and the Trails~\cite{ScalaGraphParsing} library for graph traversals. We also show how may-alias static code analysis can be done by the means of the library. 

All tests have been performed on a computer running Fedora 27 with quad-core Intel Core i7 2.5 GHz CPU and 8 GB of RAM.


\begin{table*}[t]
\centering
\begin{tabular}{|l|ll|lllll|lllll|}
\hline
\multirow{2}{*}{Ontology} & 
\multirow{2}{*}{\#nodes} & 
\multirow{2}{*}{\#edges} & 

\multicolumn{5}{c|}{Query 1}  &
\multicolumn{5}{c|}{Query 2}\\ 
\cline{4-13} 
& 
&
&
{\begin{tabular}[c]{@{}c@{}}\#results\end{tabular}} & 
{\begin{tabular}[c]{@{}c@{}}In memory \\ graph\\ (ms)\end{tabular}} & 
{\begin{tabular}[c]{@{}c@{}}DB query \\ (ms)\end{tabular}} & 
{\begin{tabular}[c]{@{}c@{}}Trails\\ (ms)\end{tabular}} & 
{\begin{tabular}[c]{@{}c@{}}GLL \\ (ms)\end{tabular}} & 
{\begin{tabular}[c]{@{}c@{}}\#results\end{tabular}} &
{\begin{tabular}[c]{@{}c@{}}In memory \\ graph\\ (ms)\end{tabular}} &
{\begin{tabular}[c]{@{}c@{}}DB query \\ (ms)\end{tabular}} &
{\begin{tabular}[c]{@{}c@{}}Trails \\ (ms)\end{tabular}} &
{\begin{tabular}[c]{@{}c@{}}GLL \\ (ms)\end{tabular}} \\



\hline\hline
atom-primitive              & 291 & 685  & 15454 & 64  & 67 & 2849 & 232 & 122  &  29 &  79 & 453 & 19 \\
\shortstack[l]{biomedical- \\
 measure-primitive}         & 341 & 711  & 15156 & 112 & 108 & 3715 & 482 & 2871 &  18 & 18 & 60  & 26 \\
foaf                        & 256 & 815  & 4118  & 11  & 11  & 432  & 29  & 10   &  1  & 1  & 1   & 1 \\
funding                     & 778 & 1480 & 17634 & 69  & 68  & 367  & 179 & 1158 &  9  & 9   & 76  & 13 \\
generations                 & 129 & 351  & 2164  & 4   & 4   & 9    & 12  & 0    &  1  & 0  & 0   & 0 \\
people\_pets                & 337 & 834  & 9472  & 37  & 37  & 75   & 80  & 37   &  1  & 1  & 2   & 1 \\
pizza                       & 671 & 2604 & 56195 & 333 & 325 & 7764 & 793 & 1262 &  17 & 18 & 905 & 50 \\
skos                        & 144 & 323  & 810   & 1   & 2   & 6    & 6   & 1    &  1  & 0  & 0   & 0 \\
travel                      & 131 & 397  & 2499  & 11  & 13  & 34   & 21  & 63   &  1  & 1  & 1   & 2 \\
univ-bench                  & 179 & 413  & 2540  & 10  & 10  & 31   & 24  & 81   &  1  & 1  & 2   & 1 \\
wine                        & 733 & 2450 & 66572 & 405 & 401 & 3156 & 606 & 133  &  2  & 3  & 4   & 5 \\
\hline
\end{tabular}
\caption{Comparation of Meerkat, Trails and GLL performance on ontologies}
\label{table:rdfs}
\end{table*}


\subsection{Ontology querying}
\label{sec:ontology}

Querying for ontologies is a well-known graph querying problem. We evaluate our library on some popular ontologies in form of RDF files from the paper~\cite{CFGonRDF}. 
First, we convert RDF files to a labelled directed graph in the following manner: we create two edges $(subject,\ predicate,\ object)$ and $(object,\ predicate^{-1},\ subject)$ for every RDF triple $(subject,\ predicate,\ object)$. Then the graph is either loaded into the Neo4j database or is loaded directly into the memory. 
Then we run two queries from the paper~\cite{GrigorevR16} for these graphs. The grammars for the queries are presented in Fig.~\ref{fig:query1Gen} and Fig.~\ref{fig:query2Gen}.

The performance results are shown in the table~\ref{table:rdfs} where $\#results$ is a number of pairs of the nodes between which exists at least one S-path. 

The Meerkat-based and the GLL-based~\cite{GrigorevR16} solutions show the same results (column $\#results$) and the Query 1 runs up to two times faster on the Meerkat-based solution than on the GLL-based, meanwhile the GLL-based solution is faster for the Query 2. Querying the database is naturally $2-4$ times slower than querying the graphs located in the RAM.


\subsection{Static code analysis}

Alias analysis is a fundamental static analysis problem~\cite{Marlowe}: it checks may-alias relations between code expressions and can be formulated as a context-free language (CFL) reachability problem~\cite{Reps}, which is closely related to the context-free path querying problem.
In this analysis, a program is represented as a Program Expression Graph (PEG)~\cite{Zheng}.
Vertices in a PEG correspond to program expressions and edges are relations between them.
There are two types of edges possible while analyzing C source code: \textbf{D}-edge and \textbf{A}-edge.

\begin{itemize}
    \item Pointer dereference edge (\textbf{D}-edge). For each pointer dereference $*e$ there is a directed \textbf{D}-edge from $e$ to $*e$.
    \item Pointer assignment edge (\textbf{A}-edge). For each assignment $*e_1=e_2$ there is a directed \textbf{A}-edge from $e_2$ to $*e_1$
\end{itemize}

For the sake of simplicity, we add edges labelled by $\overline{D}$ and $\overline{A}$ which correspond to the reversed \textbf{D}-edge and \textbf{A}-edge, respectively.

The grammar for may-alias problem from~\cite{Zheng} is presented in fig.~\ref{lst:aliasGrammar}.
It contains two nonterminals \textbf{M} and \textbf{V}, which we can query for.

\begin{itemize}
    \item \textbf{M}-production means that two l-value expressions are memory aliases i.e. may refer to the same memory location.
    \item \textbf{V}-production means that two expressions are value aliases i.e. may evaluate to the same pointer value.
\end{itemize}

We run the \textbf{M} and \textbf{V} queries on some open-source C projects: the results are in the table~\ref{table:staticAnalysis}.
We conclude that our solution is expressive and performant enough to be used for static analysis problems, which can be expressed as CFPQs.

\begin{figure}[t]
\begin{align*}
& M \rightarrow\ \overline{D}\ V\ D\\
& V \rightarrow (M ?\ \overline{A})^{*} \ M?\ (A\ M?)^{*}
\end{align*}
\caption{Context-Free grammar for the may-alias problem}
\label{lst:aliasGrammar}
\end{figure}

\begin{figure}[h]
\begin{lstlisting}
val M  = syn("nd" ~ V ~ "d")
val V  = syn((M.? ~ "na").* ~ M.? ~ ("a" ~ M.?).*)
\end{lstlisting}
\caption{Meerkat representation of may-alias problem grammar}
\label{fig:aliasMeerkat}
\end{figure}


\begin{table*}[t]
\centering
\begin{tabular}{|l|lll|lll|ll|}
\hline
%\multirow{2}{*}{Program} &
%\multirow{2}{*}{\#edges} &
%\multirow{2}{*}{\#nodes} &
%\multirow{2}{*}{Code Size (KLOC)} & 
{Program} &
{\#edges} &
{\#nodes} &
{Code Size (KLOC)} & 

\multicolumn{1}{c|}{In memory  graph (ms)} &
\multicolumn{1}{c|}{Neo4j graph (ms)} &
\multicolumn{1}{c|}{Trails graph (ms)} &
\multicolumn{1}{c|}{M aliases} &
\multicolumn{1}{c|}{V aliases}  \\ 
\hline
\hline
wc-5.0      & 332  & 770  & 0.5 & 0   & 2   & 3   & 174  & 107 \\
pr-5.0      & 815  & 2062 & 1.7 & 11  & 12  & 14  & 1131 & 63  \\
ls-5.0      & 1687 & 4734 & 2.8 & 43  & 51  & 170 & 5682 & 253 \\
bzip2-1.0.6 & 632  & 1508 & 5.8 & 8   & 13  & 21  & 813  & 71  \\
gzip-1.8    & 2687 & 7510 & 31  & 111 & 120 & 537 & 4567 & 227 \\
\hline
\end{tabular}
\caption{Running may-alias queries on Meerkat on some C open-source projects}
\label{table:staticAnalysis}
\end{table*}

\subsection{Classical Movies Queries}

In order to examine the applicability of our library for the real data processing we evaluate queries presented in the section~\ref{sec:examples} on the classical movie database which contains more than 60000 nodes and more than 100000 edges. 

All queries are performed using the Neo4j database connected to Meerkat.
During evaluation of these queries, caching in Neo4j was disabled to simplify time measurement.

\begin{table*}[t]
\centering
\begin{tabular}{|l|c|c|}
\hline
{Query} &
{Neo4j + Meerkat (ms)} &
{Neo4j + Cypher (ms)} \\
\hline
\hline
Mutual friend recommendations & 106.86 & 17.26 \\
Directed more than 2 films, Acted in more than 10 & 348.11 & 44.82 \\
Find the most profilic actors & 2333.37 & 189.32 \\
Actors who played in some film & 89.78 & 11.31 \\
\hline
\end{tabular}
\caption{Running regular queries using Meerkat and Cypher}
\label{table:movies}
\end{table*}

Our library implements a general parsing algorithm which supports arbitrary context-free queries. 
Unfortunately, when the queries are regular, there is a big overhead for processing them.
The queries mentioned in this section are all regular, thus it is not a surprise that our implementation is dramatically slower that the native Neo4j solution. Besides the overhead itself, our library does not take any advantages from the internal Neo4j optimizations. 

%However, when our library is used to execute regular query there appears another serious problem.
%Meerkat has more complicated and general parsing algorithm than regular path query engines.
%It allows to parse context-free grammars but gives an overhead on regular ones.


%It is not surprise, that our implementation is drammatically slower than native Neo4j solution.
%Our library has more complicated and general parsing algorithm than regular path query engines.
%It allows to handle context-free constarined paths queries, which can not be expressed in Cypher, but gives an overhead on regular ones.
%Moreover we do not use internal Neo4j optimizations. 

Despite the seeming failure of this test, we still can conclude that the combinators can process quite big datasets in a reasonable time: 6 times bigger in terms of the number of vertices than the other tests.
It also shows that further optimization of the realization is justified. 

%But these results are interested in some reason.
%First of all, we show that combinators can process big datasets in reasonable time: datasets in RDFs and static code analysis has less then 10000 nodes, movie dataset has more than 60000.
%Secondly, we show that optimizations of proposed solution is requred may be reasonable (our 
%solution is not drammatically slow, so we can hope that optimizations can help us).


\subsection{Comparison with Trails}

Trails~\cite{ScalaGraphParsing} is a Scala graph combinator library.
It provides traversers for describing paths in graphs which resemble parser combinators and calculates possibly infinite stream of all possible paths described by the composition of the basic traversers.
Both Trails and Meerkat-based solution support parsing of the graphs located in RAM, so we compare the performance of the Trails and the Meerkat-based solution on the ontology queries described above: the results are presented in the table~\ref{table:rdfs}.
Trails and Meerkat-based solution compute the same results, but Trails is up to 10 times slower than the Meerkat-based solution.

To summarize, we demonstrated that parser combinators suit for implementing real queries. 
Our implementation is as performant as the other existing combinators library and is comparable to the GLL-based solution.
