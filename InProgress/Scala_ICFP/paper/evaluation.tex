\section{Evaluation}

In this section, we present an evaluation of our graph querying library.
We measure its performance on a classical set of ontology graphs~\cite{CFGonRDF}: both when the graph is loaded into the RAM and for the integration with the Neo4j database and compare it with the solution based on the GLL parsing algorithm~\cite{GrigorevR16} and the Trails~\cite{ScalaGraphParsing} library for graph traversals. We also show how may-alias static code analysis can be done by means of the developed library. 

All tests have been performed on a computer running Fedora 27 with quad-core Intel Core i7 2.5 GHz CPU and 8 GB of RAM.


\begin{table*}[t]
\centering
\begin{tabular}{|l|lll|lllll|lllll|}
\hline
\multicolumn{1}{|c|}{\multirow{2}{*}{Ontology}} & 
\multicolumn{1}{c|}{\multirow{2}{*}{\#tripples}} & 
\multicolumn{1}{c|}{\multirow{2}{*}{\#nodes}} & 
\multicolumn{1}{c|}{\multirow{2}{*}{\#edges}} & 

\multicolumn{5}{c|}{Query 1}  &
\multicolumn{5}{c|}{Query 2}\\ 
\cline{3-13} 
\multicolumn{1}{|c|}{} & 
\multicolumn{1}{c|}{}  &
\multicolumn{1}{c|}{\begin{tabular}[c]{@{}c@{}}\#results\end{tabular}} & 
\multicolumn{1}{c|}{\begin{tabular}[c]{@{}c@{}}In memory \\ graph\\ (ms)\end{tabular}} & 
\multicolumn{1}{c|}{\begin{tabular}[c]{@{}c@{}}DB query \\ (ms)\end{tabular}} & 
\multicolumn{1}{c|}{\begin{tabular}[c]{@{}c@{}}Trails\\ (ms)\end{tabular}} & 
\multicolumn{1}{c|}{\begin{tabular}[c]{@{}c@{}}GLL \\ (ms)\end{tabular}} & 
\multicolumn{1}{c|}{\begin{tabular}[c]{@{}c@{}}\#results\end{tabular}} &
\multicolumn{1}{c|}{\begin{tabular}[c]{@{}c@{}}In memory \\ graph\\ (ms)\end{tabular}} &
\multicolumn{1}{c|}{\begin{tabular}[c]{@{}c@{}}DB query \\ (ms)\end{tabular}} &
\multicolumn{1}{c|}{\begin{tabular}[c]{@{}c@{}}Trails \\ (ms)\end{tabular}} &
\multicolumn{1}{c|}{\begin{tabular}[c]{@{}c@{}}GLL \\ (ms)\end{tabular}} \\
\hline\hline
atom-primitive              & 425  & 291 & 685  & 15454 & 112 & 167 & 2849 & 232 & 122  & 49 & 52 & 453 & 19 \\
\shortstack[l]{biomedical- \\
 measure-primitive}         & 459  & 341 & 711  & 15156 & 226 & 247 & 3715 & 482 & 2871 & 34 & 42 & 60  & 26 \\
foaf                        & 631  & 256 & 815  & 4118  & 16  & 25  & 432  & 29  & 10   & 1  & 2  & 1   & 1 \\
funding                     & 1086 & 778 & 1480 & 17634 & 123 & 152 & 367  & 179 & 1158 & 18 & 23 & 76  & 13 \\
generations                 & 273  & 129 & 351  & 2164  & 6   & 21  & 9    & 12  & 0    & 0  & 0  & 0   & 0 \\
people\_pets                & 640  & 337 & 834  & 9472  & 63  & 84  & 75   & 80  & 37   & 2  & 3  & 2   & 1 \\
pizza                       & 1980 & 671 & 2604 & 56195 & 544 & 650 & 7764 & 793 & 1262 & 44 & 47 & 905 & 50 \\
skos                        & 252  & 144 & 323  & 810   & 4   & 9   & 6    & 6   & 1    & 0  & 1  & 0   & 0 \\
travel                      & 277  & 131 & 397  & 2499  & 21  & 55  & 34   & 21  & 63   & 2  & 2  & 1   & 2 \\
univ-bench                  & 293  & 179 & 413  & 2540  & 15  & 43  & 31   & 24  & 81   & 2  & 2  & 2   & 1 \\
wine                        & 1839 & 733 & 2450 & 66572 & 543 & 727 & 3156 & 606 & 133  & 5  & 7   & 4   & 5 \\
\hline
\end{tabular}
\caption{Comparation of Meerkat, Trails and GLL performance on ontologies}
\label{table:rdfs}
\end{table*}


\subsection{Ontology querying}
\label{sec:ontology}

Querying for ontologies is a well-known graph querying problem. We evaluate our library on some popular ontologies which are presented as RDF files in the paper~\cite{CFGonRDF}. 
First, we convert RDF files to a labelled directed graph in the following manner: we create two edges $(subject,\ predicate,\ object)$ and $(object,\ predicate^{-1},\ subject)$ for every RDF triple $(subject,\ predicate,\ object)$. Then the graph is either loaded into the Neo4j database or is loaded directly into the memory. 
Then we run two queries from the paper~\cite{GrigorevR16} for these graphs. The grammars for the queries are presented in Fig.~\ref{fig:query1Meerkat} and Fig.~\ref{fig:query2Meerkat}.


\begin{figure}[h]
\begin{lstlisting}
val query1 = syn(
   "subclassof-1" ~ S.? ~ "subclassof" |
   "type-1" ~ S.? ~ "type")
\end{lstlisting}
\caption{The same generation query (Query 1) in Meerkat}
\label{fig:query1Meerkat}
\end{figure}

\begin{figure}[h]
\begin{lstlisting}
val S = syn(
  "subclassof-1" ~ S ~ "subclassof")
val query2 = syn(S ~"subclassof")
\end{lstlisting}
\caption{Query 2 grammar}
\label{fig:query1Meerkat}
\end{figure}

The performance results are shown in the table~\ref{table:rdfs}. $\#triples$ reflects the size of the RDF file with the corresponding ontology, $\#results$ is a number of pairs of the nodes between which at least one S-path exists. 

The Meerkat-based and the GLL-based~\cite{GrigorevR16} solutions show the same results (column $\#results$) and the Query 1 runs up to two times faster on the Meerkat-based solution than on the GLL-based, meanwhile the GLL-based solution is faster for the Query 2. Querying the database is naturally $2-4$ times slower than querying the graphs located in the RAM.


\subsection{Static code analysis}

Alias analysis is a fundamental static analysis problem~\cite{Marlowe}: it checks may-alias relations between code expressions and can be formulated as a context-free language (CFL) reachability problem~\cite{Reps} which is closely related to context-free path querying problem.
In this analysis, a program is represented as a Program Expression Graph (PEG)~\cite{Zheng}.
Vertices in a PEG correspond to program expressions and edges are relations between them.
There are two types of edges possible while analyzing C source code: \textbf{D}-edge and \textbf{A}-edge.

\begin{itemize}
    \item Pointer dereference edge (\textbf{D}-edge). For each pointer dereference $*e$ there is a directed \textbf{D}-edge from $e$ to $*e$.
    \item Pointer assignment edge (\textbf{A}-edge). For each assignment $*e_1=e_2$ there is a directed \textbf{A}-edge from $e_2$ to $*e_1$
\end{itemize}

For the sake of simplicity, we add edges labelled by $\overline{D}$ and $\overline{A}$ which correspond to the reversed \textbf{D}-edge and \textbf{A}-edge, respectively.

The grammar for may-alias problem from~\cite{Zheng} is presented in Fig.~\ref{lst:aliasGrammar}.
It contains two nonterminals \textbf{M} and \textbf{V} and allows us to make two kinds of queries for each of them.

\begin{itemize}
    \item \textbf{M}-production means that two l-value expressions are memory aliases i.e. may refer to the same memory location.
    \item \textbf{V}-production means that two expressions are value aliases i.e. may evaluate to the same pointer value.
\end{itemize}

We run the \textbf{M} and \textbf{V} queries on some open-source C projects: the results are in table~\ref{table:staticAnalysis}.
We can conclude that our solution is expressive and performant enough to be used for static analysis problems which can be expressed as CFPQs.

\begin{figure}[t]
\begin{align*}
& M \rightarrow\ \overline{D}\ V\ D\\
& V \rightarrow (M ?\ \overline{A})^{*} \ M?\ (A\ M?)^{*}
\end{align*}
\caption{Context-Free grammar for the may-alias problem}
\label{lst:aliasGrammar}
\end{figure}

\begin{figure}[h]
\begin{lstlisting}
val M  = syn("nd" ~ V ~ "d")
val V  = syn((M.? ~ "na").* ~ M.? ~ ("a" ~ M.?).*)
\end{lstlisting}
\caption{Meerkat representation of may-alias problem grammar}
\label{fig:aliasMeerkat}
\end{figure}


\begin{table}[t]
\centering
\begin{tabular}{|llll|l|llll|}
\hline
\multirow{2}{*}{Program} &
\multirow{2}{*}{\#edges} &
\multirow{2}{*}{\#nodes} &
\multirow{2}{*}{Code Size (KLOC)} & 
\multicolumn{1}{c|}{In memory  graph (ms)} &
\multicolumn{1}{c|}{Neo4j  graph (ms)} &
\multicolumn{1}{c|}{M aliases} &
\multicolumn{1}{c|}{V aliases} &  \\ 
\hline
\hline
wc-5.0      & 332  & 770  & 0.5 & 0   & 4   & 174  & 107 \\
pr-5.0      & 815  & 2062 & 1.7 & 13  & 17  & 1131 & 63  \\
ls-5.0      & 1687 & 4734 & 2.8 & 52  & 76  & 5682 & 253 \\
bzip2-1.0.6 & 632  & 1508 & 5.8 & 9   & 19  & 813  & 71  \\
gzip-1.8    & 2687 & 7510 & 31  & 120 & 182 & 4567 & 227 \\
\hline
\end{tabular}
\caption{Running may-alias queries on Meerkat on some C open-source projects}
\label{table:staticAnalysis}
\end{table}

\subsection{Comparison with Trails}

Trails~\cite{ScalaGraphParsing} is a Scala graph combinator library.
It provides traversers for describing paths in graphs which resemble parser combinators and calculates possibly infinite stream of all possible paths described by the composition of basic traversers.
Both Trails and Meerkat-based solution support parsing of the graphs located in RAM, so we compare the performance of Trails and Meerkat-based solution on the ontology queries described above: the results are presented in table~\ref{table:rdfs}.
Trails and Meerkat-based solution compute the same results, but Trails is up to 10 times slower than Meerkat-based solution.

To summarize, we demonstrated that parser combinators are expressive enough to be used for implementing real queries. 
Our implementation is as performant as the other existing combinators library and is comparable to the GLL-based solution.
