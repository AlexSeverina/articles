\section{Contex-Free Path Querying Problem}
\label{sec:CFPQ}

In this section we formally describe the context-free path querying problem (or context-free reachability problem).

%In this section we intrduce context-free path querying problem (or context-free reachability problem) formal description.

First, we introduce the necessary definitions.
%First of all we introduce necessary definitions.
\begin{itemize}
  \item Context-free grammar is a quadruple $G=(N, \Sigma, P, S)$, where $N$ is a set of nonterminal symbols, $\Sigma$ is a set of terminal symbols, $S \in N$ is a start nonterminal, and $P$ is a set of productions. 
  \item $\mathcal{L}(G)$ denotes a language specified by the grammar $G$, and is a set of terminal strings derived from the start nonterminal of $G$: $L(G) = \{\omega | S \Rightarrow_{G}^{*} \omega\}$.
  \item Directed graph is a triple $M = (V,E,L)$, where $V$ is a set of vertices, $L \subseteq \Sigma$ is a set of labels, and a set of edges $E\subseteq V\times L\times V$. 
  We assume that there are no parallel edges with equal labels: for every $e_1=(v_1,l_1,v_2) \in E, e_2=(u_1,l_2,u_2) \in E$ if $v_1 = u_1$ and $v_2 = u_2$ then $l_1 \neq l_2$.
  \item $tag: E \rightarrow L$ is a helper function which returns a tag of a given edge. $$tag(e = (v_1,l,v_2), e \in E) = l$$
  \item $\oplus: L^+ \times L^+ \rightarrow L^+$ denotes a tag concatenation operation.
  \item $\Omega$ is a helper function which constructs a string produced by the given path. For every $p \text{ path in } M$
  $$ \Omega(p = e_{0},e_{1},\dots,e_{n-1}) = tag (e_{0}) \oplus \dots \oplus tag (e_{n-1}).$$
\end{itemize}

We define the context-free language constrained path querying as, given a query in the form of a grammar $G$, to construct the set of the paths $$Q(M,G)=\{p|p \text{ is path in } M, \Omega(p) \in \mathcal{L}(G)\}.$$
The CFL reachability problem is pretty similar and is formulated as follows: $$Q(M,G) =\{ (v_0,v_n) \ | \ \exists p \text{ is path in } M, p = v_0 \xrightarrow{l_0} \cdots \xrightarrow{l_{n-1}}v_n,$$
$$ \Omega(p) \in   \mathcal{L}(G))\}$$.

%Using these definitions, we define the context-free language constrained path querying as, given a query in form of grammar $G$, to construct the set of paths $$Q(M,G)=\{p|p \text{ is path in } M, \Omega(p) \in \mathcal{L}(G)\}.$$
%Reachability problem is a pretty similar and may be formulated as follows: $$Q(M,G) =\{ (v_0,v_n) \ | \ \exists p \text{ is path in } M, p = v_0 \xrightarrow{l_0} \cdots \xrightarrow{l_{n-1}}v_n,$$
%$$ \Omega(p) \in   \mathcal{L}(G))\}$$.

Note that $Q(M, G)$ can be an infinite set, hence it cannot be represented explicitly. 
We show how to construct a compact data structure which stores all the elements of $Q(M,G)$ in a finite space; every path can be extracted from this representation. 

%Note that $Q(M, G)$ can be an infinite set, hence it cannot be represented explicitly. 
%In order to solve this problem, we construct compact data structure representation which stores all elements of $Q(M,G)$ in finite space and from which one can extract any of them.

%The main idea of our solution is that language may be specified in terms of parser combinators.
%Further we describe some detailes of parsing techniques which is necessary to provide generic parer combinators based solution for CFPQs evaluation.
