\section{Conclusion}
\label{sec:conclusion}

We propose a native way to integrate a language for context-free path querying into a general-purpose programming language.
Our solution handles arbitrary context-free grammars and arbitrary input graphs.
The proposed approach is language-independent and may be implemented in nearly every general-purpose programming language.
We implement it in Scala and show that our approach can be applied to the real world problems

%We propose a native way to integrate a language for context-free path querying into a general-purpose programming language.
%Our solution can handle arbitrary context-free grammars and arbitrary input graphs.
%The proposed approach is language-independent and may be implemented for closely all general-purpose programming languages.
%We implement it in the Scala programming language and show that our implementation can be applied to the real world problems.

We can propose some possible directions for the future work.
%First of all, it is desirable to improve the interface for the SPPF processing. 
%Currently, we can extract only reachibility information from the SPPF, but, being a parse forest, SPPF also offers data about the structure of the paths which has its own value.
%A lazy stream of paths accompanied by their structural representation can simplify debugging and further processing.

%First of all, it is necessary to formulate the creation of a user-friendly interface for SPPF processing.
%We can just extract reachability information, but SPPF contains mach more useful information.
%One such representation may be a set of paths with additional information about their structure.
%This may simplify debugging and query result processing.
%
In order to improve the performance and investigate the scalability of the solution, we plan to implement a parallel single machine and distributed GLL. 
It is a challenge from both the theoretical and the practical standpoint.

%In order to improve performance and investigate scalability of proposer solution it is necessary to try to implement parallel single machne and distributed GLL.
%It is not only algorithmic problem: to get practical solution we should choose appropriate tools, libraryes for parallel and distributet computing for Scala.

%Another direction is improvements of semantic actions computation, otherwise known as the attribute grammars handling. 
%Although it is impossible in general, techniques such as lazy evaluation can provide a technically adequate solution which is demonstarted in our solution.
%But our implmentation is very naive and should be improved.
%One of possible direction is utilizatuin of relational programming (minikanren) which is aimed to search~\cite{DB}{Ekaterina, we need your help!!!}.
%For what class of semantic actions it is possible to provide a precise general solution is a theoretical question to be answered. 

Some important problems in the realm of the static code analysis cannot be expressed in terms of context-free path querying. 
For example, the context-sensitive data-dependence analysis may be precisely expressed in terms of the linear-conjunctive language~\cite{Okhotin2003LCL} reachability but not context-free~\cite{LCLReachability}.
How to support the arbitrary conjunctive grammars is also worth research. 
This technique can be employed as a static analysis framework. 

%Some important problems in static code analysis reqwuire languages more expressive than contex-free one.
%For example, context-sensitive data-dependence analysis may be precisely expressed in terms of linear-conjunctive language~\cite{Okhotin2003LCL} reachability, but not context-free~\cite{LCLReachability}.
%While problem formulation is precise, it is possible to get only approximated solution, because emptyness problem for linear-conjunctive languages is undersidable.
%It would be an interesting task to support not only linear-cinjunctive grammars, but arbitrary conjunctive grammars~\cite{okhotin2001conjunctive} in te library and investigate nature of approximation.
%Finally it would be interesting to create a core for static analysis framework based on language reachcbility and compare it with the respective part of Flix or other related tools.

%Improved version of OpenCypher~\cite{OpenCypherPR}, which is the one of the most popular graph query languages, provides context-free path querying mechanism.
%Detailed comparison with it may provide more information for direcion of future work.
