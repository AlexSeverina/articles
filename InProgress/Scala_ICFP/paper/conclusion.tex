\section{Conclusion}

We propose a native way to integrate a language for context-free path querying into a general-purpose programming language.
Our solution can handle arbitrary context-free grammars and arbitrary input graphs.
The proposed approach is language-independent and may be implemented for closely all general-purpose programming languages.
We implement it in the Scala programming language and show that our implementation can be applied to the real world problems.

We can propose some possible directions for the future work.
First of all, it is necessary to extend the library with combinators for vertices information processing.
The next one is in the area of technical improvements: the creation of a user-friendly interface for SPPF processing.
One such representation may be a set of paths with additional information about their structure.
This may simplify debugging and query result processing.

Another direction is a semantic actions computation, otherwise known as attributed grammars handling.
It increases the expressiveness of queries by means of the specification of user-defined actions, such as filters, over subqueries result. 
Although it is impossible in general, techniques such as lazy evaluation can provide a technically adequate solution.
For what class of semantic actions it is possible to provide a precise general solution is a theoretical question to be answered. 

Conjunction support. Approximated result.

Parallel and distributed GLL for hard queryies on huge DB execution.