\section{Examples}
\label{sec:examples}

In this section, we describe some examples of queries written with the library proposed.
We show that the combinators are expressive enough for realistic queries and also ease their creation.

%In this section we introduce and describe some examples of our library usage.
%We show that combinators are expressive enough for realistic queries and allows to create generic queries easely.

\subsection{Complicated Query to Map}

We would like to search for all the routes which pass through a fixed city $a$ such that the countries of the cities on the route form a palindrome. 
This means that if a route starts at the city of the country $X$, then the last visited city should also belong to the country $X$. 
We also demand that the fixed city $a$ is in the middle of the route. 
This query can be written as shown in fig.~\ref{fig:pathQuery}.
A combinator \lstinline{reduceChoice} reduces a list of queries to a single query by means of the combinator \lstinline{|}.
The implementation of the \lstinline{reduceChoice} is straightforward and can be found in fig.~{fig:reduceChoice}. 
A subquery \lstinline{middleCity} matches the fixed city $a$ and a query \lstinline{roadTo} matches a \emph{roadTo} edge.

%
%Let us capture one city, let's say city with name $a$.
%Now having a city graph and captured vertex we would like to know all paths such if as $i$ city from begining of our path we visit country $X$ then as $i$ city from end of our path we visit country $X$ too.
%And also the middle city in our path is our captured city $a$.
%In a terms of combinators we can define our path as shown on fig.~\ref{fig:pathQuery}.
%Here \lstinline{reduceChoice} is a function which transforms a list of queries to one query which is formed by reducing given list with \lstinline{|} combinator.
%The \lstinline{pathPart} query recursively defines a path of our way.
%Also, \lstinline{middleCity} is a vertex query which parses our captured city $a$ and \lstinline{roadTo} query parses a \emph{roadTo} edge.

\begin{figure}[h]
\begin{lstlisting}
val countriesList = List("X", "Y")
val path =
  (reduceChoice(countriesList.map(pathPart)) |
    middleCity)
def pathPart(country: String) =
  syn(city(country) ~ roadTo ~ path ~
    roadTo ~ city(country))

val middleCity = V(_.value() == "a")
val roadTo = E(_.value() == "road_to")
def city(country: String) =
  V(_.country == country)
\end{lstlisting}
\caption{Path query}
\label{fig:pathQuery}
\end{figure}

\begin{figure}[h]
\begin{lstlisting}
def reduceChoice(xs: List[Nonterminal]) =
  xs match {
   case x :: Nil => x
   case x :: y :: xs =>
      syn(xs.foldLeft(x | y)(_ | _))
 }
\end{lstlisting}
\caption{Reduce choice function implementation}
\label{fig:reduceChoice}
\end{figure}

To filter out all the data but the list of the cities on the route, we can add the semantic actions as shown in fig.~\ref{fig:fixedPathQ}

%Now we would like, to get from our query only \lstinline{city} combinator result.
%For that purpose let us modify it to make return result.
%In our library we have a \lstinline{^} and \lstinline{&} functions for that.
%Then we will have definition of our combinators as presented in fig.~\ref{fig:fixedPathQ}.

\begin{figure}[h]
\begin{lstlisting}
val middleCity =
  syn(V(_.value() == "a") ^^) & (List(_))
def pathPart(country: String) = syn(
  (city(country) ~ roadTo ~
    path ~ roadTo ~ city(country) & {
     case a ~ (b: List[_]) ~ c =>
        a +: b :+ c})
\end{lstlisting}
\caption{Fixed queries}
\label{fig:fixedPathQ}
\end{figure}

Executing the query for the graph in fig.~\ref{fig:graph} returns the only three routes, which satisfy our restrictions. 

%%Now we execute our query. It is evident that for the graph presented on fig.~\ref{fig:graph} we can get only three paths which satisfies given criteria:
\begin{itemize}
\item single-vertex path $a$;
\item $b \rightarrow a \rightarrow d$
\item $c \rightarrow b \rightarrow a \rightarrow d \rightarrow e$
\end{itemize}

A simplified SPPF for this query is presented in fig.~\ref{fig:sppf}. Rounded rectangles represent nonterminals and other rectangles represent productions.
Every rectangle vertex contains a nonterminal name or a production rule, as well as the start and the end nodes in the input graph for the path derived from the corresponding SPPF vertex.
The start nonterminals are drawn in grey.

\subsection{Same Generation Query}

The generalisation of the classical same generation query benefits from utilising the first-order functions for querying. 
Such a query can be used for the hierarchy analysis in RDF storages~\cite{CFGonRDF}.
Let's consider RDF graphs which have two pairs of relations between objects: (\emph{subClassOf}; $\text{\emph{subClassOf}}^{-1}$) and (\emph{type}; $\text{\emph{type}}^{-1}$). Each relation has its reverse denoted by the $-1$ superscript. 
To search for the vertices which are on the same level of hierarchy one can use the grammars $G_1$~(fig.~\ref{grammarQ1}) and $G_2$~(fig.~\ref{grammarQ2}).

%Yet another example of first order functions usage is generalisation of classical same generation query which is one of basic context-free path queries.
%One of application of such queries is hierarchy analysing in RDF storages~\cite{CFGonRDF}.
%Let suppose that we have RDF graphs with two pairs of relation (each pair is relation and its revers): (\emph{subClassOf}; $\text{\emph{subClassOf}}^{-1}$) and (\emph{type}; $\text{\emph{type}}^{-1}$).
%We want to evaluate two queries which detect all pairs of nodes which are connected by path derivable in grammars $G_1$~(Fig.~\ref{grammarQ1}) and $G_2$ respectively~(Fig.~\ref{grammarQ2}).
%

\begin{figure}[h]
   \centering
   \[
\begin{array}{rl}
   0: & S \rightarrow \text{\textit{subClassOf}}^{-1} \ S \ \text{\textit{subClassOf}} \\
   1: & S \rightarrow \text{\textit{type}}^{-1} \ S \ \text{\textit{type}} \\
   2: & S \rightarrow \text{\textit{subClassOf}}^{-1} \ \text{\textit{subClassOf}} \\
   3: & S \rightarrow \text{\textit{type}}^{-1} \ \text{\textit{type}} \\
\end{array}
\]
   \caption{Context-free grammar $G_1$ for query 1}
   \label{grammarQ1}
   \end{figure}

\begin{figure}[h]
   \centering
   \[
\begin{array}{rl}
   0: & S \rightarrow B \ \text{\textit{subClassOf}} \\
   0: & S \rightarrow \text{\textit{subClassOf}} \\
   1: & B \rightarrow \text{\textit{subClassOf}}^{-1} \ B \ \text{\textit{subClassOf}} \\
   2: & B \rightarrow \text{\textit{subClassOf}}^{-1} \ \text{\textit{subClassOf}} \\
\end{array}
\]
   \caption{Context-free grammar $G_2$ for query 2}
   \label{grammarQ2}
   \end{figure}

   These two queries are context-free, so they can be easily written in Meerkat: the code is presented in fig.~\ref{fig:query1Meerkat} and fig.~\ref{fig:query2Meerkat}.

\begin{figure}[h]
\begin{lstlisting}
val query1: Nonterminal = syn(
   "subclassof-1" ~ query1.? ~ "subclassof" |
   "type-1" ~ query1.? ~ "type")
\end{lstlisting}
\caption{The same generation query (Query 1) in Meerkat}
\label{fig:query1Meerkat}
\end{figure}


\begin{figure}[h]
\begin{lstlisting}
val S = syn(
  "subclassof-1" ~ S ~ "subclassof")
val query2 = syn(S ~"subclassof")
\end{lstlisting}
\caption{The same generation query (Query 2) in Meerkat}
\label{fig:query2Meerkat}
\end{figure}

The implementations of the queries are similar and we can get rid of the code duplication by generalizing them. 
The function \lstinline{sameGen} presented in Fig~\ref{fig:gen} generalize the same generation query and is independent of the environment such as the input graph structure or other parsers.
Other queries can be written with this function, including the one presented in Fig~\ref{fig:query1Meerkat}: it is the result of the application of \lstinline{sameGen} to the appropriate relations (which can be treated as opening and closing brackets).
Another application of the \lstinline{sameGen} is the Query 2, presented in fig.~\ref{fig:query2Gen}.

%As you can see, grammars and code representations for these two queries looks pretty similar.
%May we avoid code duplication and generalize them?
%Yes, we can and not only for these two queries.
%The function \lstinline{sameGen} presented in Fig~\ref{fig:gen} is a generalization of the same generation query and is independent of the environment such as the input graph structure or other parsers and also uses a function \lstinline{reduceChoice} presented in \ref{fig:reduceChoice}.
%It can be used for the creation of other queries, including the one presented in Fig~\ref{fig:query1Meerkat}: it is the result of the application of \lstinline{sameGen} to the appropriate relations (which can be treated as opening and closing brackets).
%Another application of the \lstinline{sameGen} is a Query 2, which can be founded in Fig.~\ref{fig:query2Gen}.

\begin{figure}[h]
\begin{lstlisting}
def sameGen(brs) =
  reduceChoice(
    bs.map {case (lbr, rbr) =>
      lbr ~ syn(sameGen(bs).?) ~ rbr})
\end{lstlisting}
\caption{Generic function for the same generations query}
\label{fig:gen}
\end{figure}


\begin{figure}[h]
\begin{lstlisting}
val query1 = syn(sameGen(List(
    ("subclassof-1", "subclassof"),
    ("type-1", "type"))))
\end{lstlisting}
\caption{Query 1 as an application of \lstinline{sameGen}}
\label{fig:query1Gen}
\end{figure}


\begin{figure}[h]
\begin{lstlisting}
val query2 = syn(
  sameGen(List(("subclassof-1", "subclassof"))) ~
   "subclassof")
\end{lstlisting}
\caption{Query 2 as an application of \lstinline{sameGen}}
\label{fig:query2Gen}
\end{figure}


We illustrated that the generic queries are easily written by means of the parser combinators.
It is possible to create a library of \ ``standard templates'' for most popular generic queries, such as the same generation query or other domain specific queries (for example, for specific static code analysis problem).


\subsection{Classical Movies Queries}

We also implemented some queries to the movie database used in the Neo4j tutorial~\footnote{The set of classical queries to movie dataset in Cypher language: \url{https://neo4j.com/developer/movie-database/}.}
The database contains the data about movies, actors, directors and the relations between them.
In this set of queries we demonstrate more semantic actions for the results processing. 

%In order to demonstarte expressive power of our solution and to demonstrate more scenarios for
%semantic actions usage we preovide some examples of classical queries to movie database which
%represents movies, actors, directors, users and relationships between them.
%All of queries can be found on a Neo4j tutorial page~\footnote{The set of classical queries to movie dataset in Cypher language: \url{https://neo4j.com/developer/movie-database/}. Access date: 16.01.2018.}.

Several helper functions were implemented to simplify the processing of the nodes and the edges. 
They are built upon the basic combinators and are presented in fig.~\ref{fig:helpers}.

%First of all, we need to introduce some helpers for nodes and edges porcessing simplification.
%They are builded upon basic combinators from our library and presented in figure~\ref{fig:helpers}.

\begin{figure}[h]
\begin{lstlisting}
def LV(labels: String*) = 
  V(e => labels.forall(e.hasLabel))
def outLE(label: String) = 
  outE(_.label() == label)
def inLE(label: String) = 
  inE(_.label() == label)
\end{lstlisting}
\caption{Helpers for edges and nodes processing}
\label{fig:helpers}
\end{figure}

%Now we can start to build queries. 
Having the helper functions implemented, we can start building queries. 
The first query is selects \emph{actors who played in some film}.
Let us compare the Cypher (figure~\ref{fig:Q1_C}) and the Meerkat versions (figure~\ref{fig:Q1_M}) of this query.
The structure of them both is similar: the first part is a path specification and the second part is a specification of the value to return; in the Meerkat version we use semantic actions to calculate it.

\begin{figure}[h]
\begin{lstlisting}
MATCH (m:Movie {title: 'Forrest Gump'})
      <-[:ACTS_IN]-(a:Actor)
RETURN a.name, a.birthplace;
\end{lstlisting}
\caption{The query \emph{actors who played in some film} in Cypher}
\label{fig:Q1_C}
\end{figure}


\begin{figure}[h]
\begin{lstlisting}
val query = 
  syn((
    (LV("Movie")::V(_.title == "Forrest Gump")) ~ 
    inLE("ACTS_IN") ~
    syn(
      LV("Actor") ^ (e => 
        (e.name,
          if (e.hasProperty("birthplace")) 
            e.birthplace 
          else "")))) &&)
 
executeQuery(query, input).foreach(println)
\end{lstlisting}
\caption{The query \emph{actors who played in some film} in Meerkat}
\label{fig:Q1_M}
\end{figure}


In the \emph{most prolific actors} query (figure~\ref{fig:Q2_C}) we need to use the postprocessing of the paths set to express ordering: \lstinline{executeQuery} returns a lazy set of paths which can be processed by using standard Scala functions as shown in fig.~\ref{fig:Q2_C}.

\begin{figure}[h]
\begin{lstlisting}
MATCH (a:Actor)-[:ACTS_IN]->(m:Movie)
RETURN a, count(*)
ORDER BY count(*) DESC LIMIT 10
\end{lstlisting}
\caption{The query \emph{most prolific actors} in Cypher}
\label{fig:Q2_C}
\end{figure}


\begin{figure}[h]
\begin{lstlisting}
val query = 
  syn((
    syn(LV("Actor") ^^) ~ 
    outLE("ACTS_IN") ~ 
    LV("Movie")) & (a => (a.name, a.toInt)))
 
executeQuery(query, input)
  .groupBy {case (a, i) => i}
  .toIndexedSeq
  .map {case (i, ms) => (ms.head._1, ms.length)}
  .sortBy {case (a, mc) => -mc}}
  .take(10)
  .foreach(println)
  
\end{lstlisting}
\caption{The query \emph{most prolific actors} in Meerkat}
\label{fig:Q2_C}
\end{figure}

For query presented in fig.~\ref{fig:Q4_C} it is necessary to use subqueries.
First of all, we specify the \lstinline{user} subquery to find the user with the specific login.
Then we specify the \lstinline{friendsWith} subquery.
Finally, we combine these subqueries to form the resulting query.

\begin{figure}[h]
\begin{lstlisting}
MATCH (u:User {login: 'adilfulara'})-
      [:FRIEND]-(f:Person)-[r:RATED]->(m:Movie)
WHERE r.stars > 3
RETURN f.name, m.title, r.stars, r.comment;
\end{lstlisting}
\caption{The query \emph{mutual friend recommendations} in Cypher}
\label{fig:Q4_C}
\end{figure}


\begin{figure}[h]
\begin{lstlisting}
val user = syn(
  LV("User")::V(_.login == "adilfulara"))

val friendsWith = 
  syn(inLE("FRIEND") | outLE("FRIEND"))

val query = syn((
  user ~ friendsWith ~ syn(LV("Person") ^^) ~
  syn(outLE("RATED") ^^) ~ syn(LV("Movie") ^^)) &
    {case p ~ r ~ m => 
       (p.name, 
        m.title, 
        r.stars.toInt,
        if (r.hasProperty("comment")) r.comment 
        else "")})
 
executeQuery(query, input)
  .filter {case (_, _, s, _) => s > 3} 
  .foreach(println)

\end{lstlisting}
\caption{The query \emph{mutual friend recommendations} in Meerkat}
\label{fig:Q4_M}
\end{figure}

%Finally we present composition of all abilities.
We also can compose all of the features used in the last two queries. 
To express query presented in fig.~\ref{fig:Q3_C}, we need to use not only postprocessing but also subquerying with postprocessing.
First, we evaluate the \lstinline{directors} subquery and build \lstinline{directorsMap} which is further used in the semantic actions for the \lstinline{actor_prof_director} subquery.
In the posprocessing step of the \lstinline{directors} subquery, we implement the filtering which relates to \lstinline{WHERE length(directed) >= 2} condition of the Cypher query.
In order to get the final result, we also need to use postprocessing as far as it is necessary to implement the filtering and sorting.

\begin{figure}[h]
\begin{lstlisting}
MATCH (a:Actor:Director)-[:ACTS_IN]->(m:Movie)
WITH a, count(1) AS acted WHERE acted >= 10
WITH a, acted 
   MATCH (a:Actor:Director)-[:DIRECTED]->(m:Movie)
WITH a, acted, collect(m.title) 
   AS directed WHERE length(directed) >= 2
RETURN a.name, acted, directed
ORDER BY length(directed) DESC, acted DESC;
\end{lstlisting}
\caption{The query \emph{Directed more than 2 films, acted in more than 10} in Cypher}
\label{fig:Q3_C}
\end{figure}


\begin{figure}[h]
\begin{lstlisting}
val directors = syn((
  syn(LV("Actor", "Director") ^^) ~ 
  outLE("DIRECTED") ~ LV("Movie")) & (_.id.toInt))
 
val directorsMap = executeQuery(directors, input)
  .groupBy(i => i)
  .map {case (i, ms) => (i, ms.length)}
  .filter {case (_, ms) => ms >= 2}
 
val actor_prof_director =  syn(
  LV("Actor", "Director") ::
    V(e => directorsMap.contains(e.id.toInt)) ^^)
 
val acts = syn(
  (actor_prof_director ~ outLE("ACTS_IN") ~ 
    LV("Movie")) & (a => 
      (a.name, a.id.toInt)))
 
executeQuery(acts, input)
  .groupBy {case (a, i) => i} 
  .toStream
  .map {case (i, ms) => (i, ms.head._1, ms.length)} 
  .filter {case (i, a, mc) => mc >= 10}
  .map {case (i, a, mc) => (a, mc, directorsMap(i))}
  .sortBy {case (a, mc, dc) => (-dc, -mc)}
  .foreach(println)
    
\end{lstlisting}
\caption{The query \emph{Directed more than 2 films, acted in more than 10} in Meerkat}
\label{fig:helpers}
\end{figure}


This shows that our library is expressive enough to formulate realistic queries.
The main drawback of our library as compared to the Cypher language is that all additional logic such as
filtering, sorting or grouping has to be implemented manually as a separate step.

