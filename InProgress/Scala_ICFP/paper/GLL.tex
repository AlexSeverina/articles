\section{Generalized Parser Combinators}
\label{sec:GLL}

Combinators techniques are shown to be applicable for graph traversing~\cite{ScalaGraphParsing}, but it still suffers the common issue with left-recursive definitions.
A general parser combinators library Meerkat~\cite{Meerkat}, implemented in the Scala programming language, removes this restrction by using memoization, continuation-passing style, and the ideas of Juhnson~\cite{Johnson}.
It supports the arbitrary (left-recursive and ambiguous) context-free specifications, but it also supports the specification of action code, and provide a \lstinline{syn} macro for custom handling of the recursive nonterminal descriptions.
Meerkat constructs the compact representation of the parse forest in the form of SPPF, which can be used for CFPQs results representation~\cite{GrigorevR16}.
The worst case time and space complexity of the solution is cubic.

%In~\cite{ScalaGraphParsing} showed that combinators techniques is applicable for graph travesing, but well-known problem of combinator with left recursion is not solved.
%But this problew was solved in classical parser-combinators~\cite{Meerkat} and implemented in Meerkat library.
%Meerkat library is a general parser combinators library implemented in Scala programming language; by using memoization, continuation-passing style and the ideas of Johnson~\cite{Johnson}, it supports arbitrary (left-recursive and ambigues) context-free specifications.
%This library implements classical set of combinators, supports action code specification, and provide \lstinline{syn} macro for custom handling of recursive nonterminal description.
%Meercat creates compressed representation of parse forest (SPPF) which is useful for CFPQs results representation~\cite{GrigorevR16}.
%Time and space complexity of this soluton are cubic in worst case.

A Meerkat specification of the language $\{a^n b^n \mid n \geq 1\}$ is presented in fig.~\ref{fig:anbnMeerkat}.

%An example Meerkat specification of grammar~\ref{fig:anbnGrammar} is presented in~\ref{fig:anbnMeerkat}.

\begin{figure}[h]
\begin{lstlisting}
val S = syn("a" ~ S.? ~ "b")
\end{lstlisting}
\caption{Meercat specification of $\{a^n b^n \mid n \geq 1\}$}
\label{fig:anbnMeerkat}
\end{figure}

\cite{GrigorevR16} showed that Generalized LL parsing algorithm~\cite{scott2010gll} can be generalized to effectively process CFPQs and the query result can be finitely represented.
As the Meerkat library is closely related to the Generalized LL algorithm and since GLL can be generalized for context-free path querying, it is also possible to adapt the Meerkat library for graph querying.
It can be done by providing a function for retrieving the symbols which follow the specified position and utilizing it in the basic set of combinators.
Details described below.



%In~\cite{GrigorevR16} showed that generalized LL parsing algorithm~\cite{scott2010gll} can be generalized to CFPQs processing and this generalization prodices efficient solution which can provide structured represenataion of query result.
%As the Meerkat library is closely related to the Generalized LL algorithm and since GLL can be generalized for context-free path querying, it is also possible to adapt Meerkat library for graph querying.
%It can be done by providing a function for retrieving the symbols which follow the specified position and utilizing it in the basic set of combinators.
%Details described below.


\subsection{SPPF}

Parsing of a string with respect to an ambiguous grammar can result in several derivation trees for a single string.
The set of derivation trees is named a \emph{derivation forest}.
To store a derivation forest efficiently, the generalized parsing algorithms utilize a \emph{Shared Packed Parse Forest} proposed by Joan Rekers~\cite{SPPF}.
The most efficient compact representation of derivation forests is a Binarized Shared Packed Parse Forest (we will abbreviate it to SPPF)~\cite{brnglr}.
GLL algorithm which utilizes this structure achieves the worst-case cubic space complexity~\cite{gllParsingTree}.

%Generalized parsing algorithms can handle ambiguose grammars and in this case they produce all posiible derivation trees named dervation forest which can has an exponential size.
%To solve the problem of compact forest representation Shared Packed Parse Forest (SPPF) was proposed by Joan Rekers~\cite{SPPF}.
%Binarized Shared Packed Parse Forest (SPPF)~\cite{brnglr} compresses derivation trees optimally reusing common nodes and subtrees.
%Version of GLL which utilizes this structure for parsing forest representation achieves worst-case cubic space complexity~\cite{gllParsingTree}.

Binarized SPPF is a directed graph, each node of which has one of the four types described below.
If $i$ is a start position of a substring and $j$---its end position, then $(i,j)$ is called an \emph{extension} of a node.

%Binarized SPPF can be represented as a graph in which each node has one of four types described below.
%We denote the start and the end positions of substring as $i$ and $j$ respectively, and we call tuple $(i,j)$ an \textit{extension} of a node.

\begin{itemize}
    \item \textbf{Terminal node} labeled $(T, i, j)$.
    \item \textbf{Nonterminal node} labeled $(N, i, j)$.
    This node denotes that there is at least one derivation $N \Rightarrow^*_G \omega[i \dots j-1]$---a substring of the input from $i$-th to $j$-th position.
    Every derivation tree for the given substring and nonterminal can be extracted by left-to-right top-down traversal of SPPF started from the respective node.
    \item \textbf{Intermediate node}: a special kind of node used for the binarization of the SPPF. These nodes are labeled with $(t,i,j)$, where $t$ is a grammar slot.
    \item \textbf{Packed node} labeled $(N \rightarrow \alpha, k)$, where $k$ is a position in input of the right end of leftmost subtree of this node.
    A subgraph with the ``root'' in such node is one derivation from the nonterminal $N$.

\end{itemize}


An example of SPPF is presented in figure~\ref{fig:sppf}.
We removed redundant intermediate and packed nodes for simplicity and to decrease the size of the figure.

SPPF can finitely represent a possibly infinite set of paths in the context of the language-constrained graph querying~\cite{GrigorevR16}.
Since the SPPF stores derivation trees for all paths, it can be useful for the postprocessing and further understanding of the query results.
In static code analysis, for example, it is possible to map paths back onto the source code thus providing a human-readable result.


%In the context of language-constarined graph querying SPPF can be used as finite represenataion of possibly infinite paths set as showed~\cite{Grigorev16}.
%As far as SPPF stores not only paths but also its derivation trees, this structure may be useful
%for query results understending and postprocessing.
%For example, for static code analysis may be interesting to map path structure to source code in
%order to make results more human-readable.
