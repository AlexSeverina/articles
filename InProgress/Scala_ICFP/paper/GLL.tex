\section{Generalized Parser Combinators}
\label{sec:GLL}

Combinators techniques are shown to be applicable for graph traversing~\cite{ScalaGraphParsing}, but it still suffers from the common issue with left-recursive definitions.
A general parser combinators library Meerkat~\cite{Meerkat}, implemented in the Scala programming language, removes this restriction by using memoization, continuation-passing style, and the ideas of Johnson~\cite{Johnson}.
Meerkat converts parsers into a memoized versions of themselves.
The memoization routine stores parsing results the first time they are computed and reuses them every time they are needed again.
Every time a new parsing result for some combinator is calculated, all the combinators, which are dependent on it, are recomputed.
This way Meerkat employs left-recursive parser combinators, and it is also crucial for the handling of the cycles in the input graphs.

Meerkat supports the arbitrary (left-recursive and ambiguous) context-free specifications; it also supports the specification of action codes and provides a \lstinline{syn} macro for custom handling of the recursive nonterminal descriptions.
Meerkat constructs the compact representation of the parse forest in the form of SPPF, which can be used for CFPQs results representation~\cite{GrigorevR16}.
The worst case time and space complexity of the solution are cubic.

A Meerkat specification of the language $\{a^n b^n \mid n \geq 1\}$ is \lstinline{val S = syn("a" ~ S.? ~ "b")}. Here \lstinline{syn} is a macro which creates a parser: for example, it transforms string literals into parsers for those strings. The tilde \lstinline{~} stands for a sequential parser combinator, and the question mark \lstinline{?} describes optional parsing. Other combinators available in the Meerkat library are shown in Table~\ref{table:combinators}.

It is shown in~\cite{GrigorevR16} that the Generalized LL parsing algorithm~\cite{scott2010gll} can be generalized to process CFPQs effectively and the query result can be finitely represented.
As the Meerkat library is closely related to the Generalized LL algorithm, it is also possible to adapt the Meerkat library for graph querying.
It can be done by providing a function for retrieving the symbols which follow the specified position and utilising it in the basic set of combinators.
Details are described in the next section.

\subsection{SPPF}

Parsing of a string with respect to an ambiguous grammar can result in several derivation trees for a single string.
The set of derivation trees is named a \emph{derivation forest}.
To store a derivation forest efficiently, the generalized parsing algorithms utilize a \emph{Shared Packed Parse Forest} proposed by Joan Rekers~\cite{SPPF}.
The most efficient compact representation of derivation forests is a Binarized Shared Packed Parse Forest (we will abbreviate it to SPPF)~\cite{brnglr}.
The GLL algorithm, which employs this structure, achieves the worst-case cubic space complexity~\cite{gllParsingTree}.

Binarized SPPF is a directed graph, each node of which has one of the four types described below.
Almost every node of the SPPF is decorated with the \emph{extension}: a pair $(i, j)$ where $i$ is a start position of a substring derivable from the node and $j$---its end position.

\begin{itemize}
    \item A \textbf{terminal node} is labelled $(T, i, j)$.
    \item A \textbf{nonterminal node} is labelled $(N, i, j)$.
    This node denotes that there is at least one derivation $N \Rightarrow^*_G \omega[i \dots j-1]$---a substring of the input from the $i$-th to the $j$-th position.
    Every derivation tree for the given substring and nonterminal can be extracted by left-to-right top-down traversal of SPPF started from the respective node.
    \item An \textbf{intermediate node}: a special kind of node used for the binarization of the SPPF. These nodes are labelled with $(t,i,j)$, where $t$ is a grammar slot.
    \item A \textbf{packed node} is labelled $(N \rightarrow \alpha, k)$, where $k$ is a position in the input of the right end of the leftmost subtree of this node.
    A subgraph with the root in such node is, in turn, a parse forest for which the first production is $N \rightarrow \alpha$.
\end{itemize}


An example of SPPF is presented in Fig.~\ref{fig:sppf}.
We removed redundant intermediate and packed nodes for simplicity and to decrease the size of the figure.

SPPF finitely represents a possibly infinite set of paths in the context of the language-constrained graph querying~\cite{GrigorevR16}.
Since the SPPF stores derivation trees for all paths, it is useful for postprocessing and further understanding of the query results.
In static code analysis, for example, it is possible to map paths back onto the source code thus providing a human-readable result.