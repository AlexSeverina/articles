\section{Introduction}

One useful type of graph queries is language-constrained path queries~\cite{FLCpathProblem}.
There are several languages for graph traversing/querying which support constraints formulated in terms of regular languages.
For example SPARQL~\cite{sparql}, Cypher~\footnote{Cypher language web page: \url{https://neo4j.com/developer/cypher-query-language/}. Access date: 16.01.2018}, and Gremlin~\cite{gremlin}.
In this work, we are focused on context-free path queries (CFPQs) which use context-free languages for constraints specification and are used in bioinformatics~\cite{GraphQueryWithEarley}, static code analysis~\cite{Reps, Zheng, LabelFlowCFLReachability, specificationCFLReachability, JavaCFL}, and RDF processing~\cite{CFGonRDF}. 
There are a lot of problem-specific solutions and theoretical research on CFPQs~\cite{Yannakakis, ConjCFPathQuery, Hellings16, GrigorevR16, QueryGraphWithData, RegularDBQuery, GraphQueryWithEarley, graphDB}.
Among them, cfSPARQL~\cite{CFGonRDF} is a single known graph query language to support CF constraints. Generic solution for the integration of CFPQs into general-purpose languages is not discussed enough.

When one develops a data-centric application, one wants to use a general purpose programming language and have a transparent and native access to data sources.
String-embedded DSLs is one way to do it. 
It utilizes a driver to execute a query written as a string and to return a possibly untyped result. 
This approach has serious drawbacks.
First of all, a DSL may require additional knowledge from a developer.
Moreover, a string-embedded language itself is a source of possible errors and vulnerabilities, static detection of which is very difficult~\cite{stringEmbeddedLanguagesProblem}.  
In trying to solve these issues, such special techniques as Object Relationship Mapping (ORM) or Language Integrated Query~(LINQ)~\cite{LINQ1, LINQ2, LinqRDF} were created. Unfortunately, they still experience difficulties with flexibility: for example with the query decomposition and the reusing of subqueries.
In this paper, we propose a transparent and natural integration of CFPQs into a general-purpose language. 

It is necessary to find an appropriate technique for integration of context-free language specification into general-purpose programming languages.
One natural way to specify a language is to specify its formal grammar which can be done by using a special DSL based, for example, on EBNF-like notation~\cite{EBNFISO}.
The classical alternative way is parser combinators~\cite{MonadicPArserCombinators} which provide all required features, including transparent integration, compile-time checks of correctness, high-level techniques for generalization.

An idea to use combinators for graph traversing has already been proposed in~\cite{ScalaGraphParsing}, but the solution presented provides only approximated handling of cycles in the input graph and does not support left-recursive grammars. 
Authors pointed out that the idea described is very similar to the classical parser combinators, but the language class supported or restrictions are not discussed.
This point is very important, because conventional combinators implement top-down parsing and cannot handle left-recursive and ambiguous grammars.

In~\cite{Meerkat}, authors demonstrate a set of parser combinators which can handle arbitrary context-free grammars by using ideas of Generalized LL~\cite{scott2010gll} (GLL).
Meerkat~\footnote{Meerkat project repository:~\url{https://github.com/meerkat-parser/Meerkat}. Access date: 16.01.2018} parser combinators library is based on~\cite{Meerkat}
 and provides result of parsing in a compact form as Shared Packed Parse Forest~\cite{SPPF}~(SPPF).
It is showed that SPPF is a suitable finite structural representation of a CFPQ query result, even if the set of paths is infinite~\cite{GrigorevR16}, which can be used for paths extraction, queries debugging and processing of result.

In this paper, we show how to compose these ideas and present the parser combinators for CFPQ which can handle arbitrary context-free grammars and provide a structural representation of the result.
We make the following contributions in this paper.

\begin{enumerate}
\item We show that it is possible to create parser combinators for context-free path querying which work on both arbitrary context-free grammars and arbitrary graphs and provide a finite structural representation of the query result.
\item We provide the implementation of the parser combinators library in Scala. This library provides an integration to Neo4j~\footnote{Neo4j graph database site: \url{https://neo4j.com/}. Access date: 16.01.2018} graph database. The source code is available on GitHub:\url{https://github.com/YaccConstructor/Meerkat}.
\item We perform an evaluation on realistic data. 
Also, we compare the performance of our library with another GLL-based CFPQ tool and with the Trails library.
We conclude that our solution is expressive and performant enough to be applied to the real-world problems. 
\end{enumerate}