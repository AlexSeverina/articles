\section{Introduction}

Graph querying is finding all paths in the graph which satisfy some constraints.
If the constraints are specified with some language formalism, i.e. a grammar, it is called a language-constrained path query. The simplest query described with a grammar $S \rightarrow a \ b$ being run against the graph in the Fig.~\ref{fig:exmplInputGraph} returns the only path $2, 3, 4$ (shown in red).

A grammar $S \rightarrow a \ S \ b \mid a \ b$ is a query for the paths of the form $a^n b^n$, where $n \geq 1$.
Querying the graph in the Fig.~\ref{fig:exmplInputGraph} returns the infinite set of paths one of which starts and ends in the vertex $3$ and goes around the cycles in the graph the appropriate number of times: $3,1,2,3,1,2,3,4,3,4,3,4,3$.

\begin{figure}[h]
\begin{tikzpicture}[shorten >=1pt,node distance=2cm,on grid,auto]
   \node[state] (q_1)   {$1$};
   \node[state] (q_2) [above=of q_1] {$2$};
   \node[state] (q_3) [above right=of q_1, below right=of q_2] {$3$};
   \node[state] (q_4) [right=of q_3] {$4$};
    \path[->]
    (q_1) edge  node {a} (q_2)
    (q_2) edge[red]  node {a} (q_3)
    (q_3) edge  node {a} (q_1)
    (q_3) edge[bend left, above, red]  node {b} (q_4)
    (q_4) edge[bend left, below]  node {b} (q_3);
\end{tikzpicture}
\caption{An example of input graph}
\label{fig:exmplInputGraph}
\end{figure}

Most existing graph traversing/querying languages, including SPARQL~\cite{sparql}, Cypher~\footnote{Cypher language web page: \url{https://neo4j.com/developer/cypher-query-language/}. Access date: 16.01.2018}, and Gremlin~\cite{gremlin} support only regular languages as constrains.
For some applications, regular languages are not expressive enough.
Context-free path queries (CFPQ), on which we focus this paper, employ context-free languages for constraints specification.
CFPQs are used in bioinformatics~\cite{GraphQueryWithEarley}, static code analysis~\cite{Reps, Zheng, LabelFlowCFLReachability, specificationCFLReachability, JavaCFL}, and RDF processing~\cite{CFGonRDF}.
Although there is a lot of problem-specific solutions and theoretical research on CFPQs~\cite{Yannakakis, ConjCFPathQuery, Hellings16, GrigorevR16, QueryGraphWithData, RegularDBQuery, GraphQueryWithEarley, graphDB}, cfSPARQL~\cite{CFGonRDF} is the single known graph query language to support CF constraints.
Generic solution for the integration of CFPQs into general-purpose languages is not discussed enough.

When developing a data-centric application, one wants to use a general-purpose programming language and also to have transparent and native access to data sources.
One way to achieve this is to use string-embedded DSLs.
In this approach, a query is written as a string, then passed on to a dedicated driver which executes it and returns a possibly untyped result.
Despite the simplicity, string-embedded DSLs have serious drawbacks.
First of all, they require the developer to learn the language itself, its features, runtime, and how the integration between the languages is implemented.
DSLs are also a source of possible errors and vulnerabilities, static detection of which is a serious challenge~\cite{stringEmbeddedLanguagesProblem}.
Such techniques as the Object Relationship Mapping (ORM) or Language Integrated Query~(LINQ)~\cite{LINQ1, LINQ2, LinqRDF} partially solve these problems, but they still have issues with flexibility: both query decomposition and  reusing of subqueries are a struggle.
In this paper, we propose a transparent and natural integration of CFPQs into a general-purpose programming language.

Context-free path queries are known in various domains under different names. The \emph{context-free language reachability framework} or  \emph{IFDS framework} is how they are called in the area of static code analysis.
In~\cite{Reps:1995, Reps} Thomas Reps shows that the wide range of static code analysis problems can be formulated in terms of CFL-reachability in the graph.
This framework is used for such problems as the taint analysis~\cite{CFLTaint}, the alias analysis~\cite{JavaCFL, Zheng, CFLGraspan}, the label flow analysis~\cite{LabelFlowCFLReachability}, and the fix locations problem~\cite{CFLfinding}.
What we propose in the paper can be viewed as a core of such framework since it provides both problem and domain independent mechanism for CFPQ evaluation.

We view parser combinators as the best way to integrate context-free language specifications into a general-purpose programming language. Parser combinators provide not only a transparent integration but also compile-time checks of correctness and high-level techniques for generalization. An idea to use combinators for graph traversing has already been proposed in~\cite{ScalaGraphParsing}. Unfortunately, the solution presented processes cycles in the input graph only approximately and is unable to handle left-recursive combinators, which is the most common issue of the approach. Authors pointed out that the idea described is similar to the parser combinators, but the language class supported or restrictions are not discussed.

Parser combinators are known to handle only a subset of context-free grammars: left recursion and ambiguity of the grammars are problematic.
In~\cite{Meerkat}, the authors demonstrate a set of parser combinators which handles arbitrary context-free grammars by using ideas of the Generalized LL~\cite{scott2010gll} algorithm (GLL).
Meerkat~\footnote{Meerkat project repository:~\url{https://github.com/meerkat-parser/Meerkat}. Access date: 16.01.2018} parser combinators library implements the ideas from the paper~\cite{Meerkat} and provides the parsing result in a compact form as a Shared Packed Parse Forest~\cite{SPPF}~(SPPF).
SPPF is a suitable finite structural representation of a CFPQ result, even when the set of paths is infinite~\cite{GrigorevR16}.
All the paths can be extracted from the SPPF---in the form of the corresponding derivation trees---and further analysis can be done.
It is also possible to run some further processing over the SPPF itself---not extracting the paths explicitly.

In this paper, we compose these ideas and present a set of parser combinators for context-free path querying which handles arbitrary context-free grammars and provides a structural representation of the result.
We make the following contributions in the paper.

\begin{enumerate}
\item We show that it is possible to create a set of parser combinators for context-free path querying which works on both arbitrary context-free grammars and arbitrary graphs and provides a finite structural representation of the query result.
\item We implement the parser combinators library in Scala. This library provides integration to Neo4j~\footnote{Neo4j graph database site: \url{https://neo4j.com/}. Access date: 16.01.2018} graph database. The source code is available on GitHub: \url{https://github.com/YaccConstructor/Meerkat}.
\item We perform an evaluation on realistic data and compare the performance of our library with another GLL-based CFPQ tool and with the Trails library.
We conclude that our solution is expressive and performant enough to be applied to the real-world problems.
\end{enumerate}

This paper is organized as follows.
We introduce a formal definition of the CFPQ problem in section~\ref{sec:CFPQ}, and we provide a basic description of the Meerkat library and SPPF data structure in section~\ref{sec:GLL}.
We describe our solution in section~\ref{sec:combinators}.
In section~\ref{sec:examples} we present and discuss a set of classical queries (the same generation query, the queries to a movie dataset~\footnote{The movie database is a traditional dataset for graph databases. Detailed description is available here: \url{https://neo4j.com/developer/movie-database/}. Access date: 16.01.2018})
formulated in terms of our library.
Evaluation of the library is described in section~\ref{sec:evaluation}.
Finally, In section~\ref{sec:conclusion} we conclude and discuss possible directions for further research.
