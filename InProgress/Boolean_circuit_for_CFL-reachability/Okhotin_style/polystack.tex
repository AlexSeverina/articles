\section{Languages with poly-slender storage languages}
\label{sec:poly}
In the previous section restriction of PDA in terms of variability of stack height was described. But this is not the case for $D_1$, which is not $k$-oscillating CFL for any $k$, but has the polynomial rational index. In this section another kind of stack restriction is considered --- poly-slenderness of a pushdown storage language as a measure of how stack contents vary along accepting computations of PDA.


For a PDA $M$, its \textit{pushdown store language} $P(M)$ consists of all words
occurring on the stack along accepting computations of $M$. It is well-known that the store language of any PDA is regular. The language $D_1$ is a one-counter language, so its pushdown store language is $Z^*Z_0$, where $Z$ is a single pushdown symbol and $Z_0$ is a bottom symbol $Z_0$.


Afrati et al. \cite{ChainQ} define the notion of \textit{polynomial stack property} and show that if a PDA has the polynomial stack property, then corresponding query has the polynomial fringe property (and hence, lies in NC). A PDA has the polynomial stack property iff the largest possible number of different contents of the same height $k$ along the any accepting computation of $M$ is bounded by polynomial $O(k^d)$ for $d \ge 0$.  For example, the usual PDA for $D_1$ has the polynomial stack property, because there is only one possible variant of contents for every stack height. 


Generalizing an example of the family of one-counter languages, we can define the family of languages whose PDAs have the polynomial stack property --- languages with a \textit{poly-slender} pushdown store language (or storage language with polynomial density). The density of a language is a function $f(n)$ that shows the number of words of length $n$ in language. A language $L \subseteq \Sigma^*$ is called \textit{poly-slender language (or with the polynomial density)} if the function $f(n)$ is bounded by $O(n^k)$ for some $k \ge 0$. For example, the language $Z^*Z_0$ is of polynomial density (even of a constant density), whereas the language ${(Z_1 + Z_2)}^*Z_0$ is of exponential density.


Thus we have the following corollary.
\begin{corollary}
Let $L$ be a context-free language and $M$ be a PDA recognizing it. If the pushdown storage language $P(M)$ is poly-slender, then the CFL-reachability problem for $L$ and an arbitrary given graph is in NC complexity class.
\end{corollary}
The property of having a poly-slender storage language implies the polynomial stack property, but the converse is not true: there are PDAs with a storage language of an exponential density, which have the polynomial stack property. Consider the language of even-length palindromes $L = \{ ww^R \ | w \in {\{0, 1\}}^*\}$. It is easy to see that usual PDA $M$ for this language has the storage language $P(M) = {(0 + 1)}^*$, which is of an exponential density. But the language $L$ is linear and  the PDA $M$ is a finite-turn automaton, therefore $M$ has bounded stack heights during every accepting computation, and, hence, has the polynomial stack property. 


Whereas the polynomial fringe property of a query is undecidable \cite{Ullman}, it is decidable in polynomial time whether a given PDA has a poly-slender storage language. At first, for a given NFA it is decidable whether its language has a polynomial or exponential density \cite*{sparseness, poldens} . The Lemma~\ref{polyslender} gives a useful condition of poly-slenderness of a language using the notion of commutativity. A language $L \subseteq \Sigma^*$ is said to be \textit{commutative} if there exists $\omega \in \Sigma^*$  such that $L \subseteq \omega^*$.
\begin{lemma}[\cite*{Gawrychowski, ginsburg1966bounded}]
\label{polyslender}
Let $M = (Q,\Sigma,\delta ,q_{0},F)$ be an NFA. For every $q \in Q$ define an NFA $M_q = (Q,\Sigma,\delta,q, \{q\})$ with $L_q = L(M_q )$. Then the language $L(M)$ has polynomial density if and only if for every $q \in Q$, $L_q$ is commutative.
\end{lemma}

Gawrychowski et al. \cite{Gawrychowski} give an algorithm for testing whether $L(M)$ is of polynomial or exponential density in $O(|Q| + |\delta|)$ time for an NFA $M = (Q,\Sigma,\delta ,q_{0},F)$. An NFA for pushdown store language of a given PDA $\mathcal{A} = (Q', \Sigma', \Gamma, \delta', q_0', Z_0, F')$ can be constructed directly in $O({|Q'|}^5{|\Gamma|}^2|\delta'|)$ time \cite{Malcher}. This construction uses the notion of meaningful triples, which form the states of NFA. A triple $[p, Z, q] \in Q' \times \Gamma \times Q'$ is \textit{meaningful} if there exists a computation of $\mathcal{A}$ starting from state $p$ with the sole symbol $Z$ in the pushdown, and ending in $q$ with the empty pushdown. By definition, there are at most $|\Gamma|{|Q'|}^2$ meaningful triples, and, hence, states of NFA. Thus, we immediately deduce the following.
\begin{corollary}
Given a pushdown automaton $M$, it can be decided whether $P(M)$ is poly-slender in polynomial time.
\end{corollary}

We show a criteria for having a poly-slender storage language for a given PDA in normal form. A PDA is said to be in the \textit{normal form} if any transition
$(q, \omega) \in \delta(p, a, Z)$ satisfies $|\omega| \le 2$. It is known that each PDA $M$ accepting a non-emplty language can be converted into an equivalent PDA $M'$ in normal form, preserving the same pushdown store language, with a state set $|Q'| \le |Q|$, where $Q$ and $Q'$ are state sets of $M$ amd $M'$ respectively \cite*{Ginsburgbook, Malcher}.
\begin{theorem}
Let $M$ be a pushdown automaton in normal form with state set $Q$. Poly-slenderness of a pushdown storage language $P(M)$ can be tested as follows:
\begin{enumerate}
\item Find the set $S$ of all meaningful triples of $M$;
\item For every pair $(q, Z)$ of $s \in S$ find whether there is a computation started in the state $q$ with a simbol $Z$ on the top of the stack and ended in the same state with the same symbol on top, while the pushdown height does not decrease during the computation;
\item All states envolved in computation should form meaningful triples;
\item If such computations for every $[q, Z, p]$ leads to only variant of the stack contents, then $P(M)$ is of polynomial density, if there are at least two variants of contens for some meaningful triple --- P(M) is of exponential density.
\end{enumerate}
\end{theorem}
\begin{figure}
\begin{tikzpicture}
\draw (0, 0) -- (0.5, 0) -- (0.5, 0.5) -- (0, 0.5) -- (0, 0);
\draw [dashed] (0, 0) -- (0, -0.5) -- (5.5, -0.5) -- (5.5, 0);
\draw (0.5, 0) -- (0.5, 1) -- (1, 1) -- (1, 0) -- (0.5, 0);
\draw (1, 0) -- (1, 1) -- (1.5, 1) -- (1.5, 0) -- (1, 0);
\draw (1.5, 0) -- (1.5, 1.5) -- (2, 1.5) -- (2, 0) -- (1.5, 0);
\draw [dashed] (2, 0) -- (2, -0.5);
\draw (3, 0.5) -- (3, 1) -- (3.5, 1) -- (3.5, 0.5) -- (3, 0.5);
\draw [dashed](3, 0.5) -- (3, -0.5);
\draw [dashed](3.5, 0.5) -- (3.5, -0.5);
\draw [dashed](4, 0) -- (4, -0.5);
\draw [dashed](4.5, 0) -- (4.5, -0.5);
\draw (4, 0) -- (4.5, 0) -- (4.5, 0.5) -- (4, 0.5) -- (4, 0);
\draw (5, 0) -- (5.5, 0);
\draw [dashed] (5, 0) -- (5, -0.5);
\draw[thick, snake, densely dotted, rounded corners=0.5mm] (2, 1.5) -- (3, 1);
\draw[thick, snake, densely dotted, rounded corners=0.5mm] (3.5, 1) -- (4, 0.5);
\draw[thick, snake, densely dotted, rounded corners=0.5mm] (4.5, 0.5) -- (5, 0);

\node (z) at (0.25, 0.25) {$Z$}; 
\node (z) at (0.75, 0.25) {$Y$}; 
\node (z) at (0.75, 0.75) {$X$}; 
\node (z) at (1.25, 0.25) {$Y$}; 
\node (z) at (1.25, 0.75) {$W$}; 
\node [blue] (z) at (1.75, 0.25) {$Y$}; 
\node[blue]  (z) at (1.75, 0.75) {$V$}; 
\node (z) at (1.75, 1.25) {$Z$}; 
\node  (z) at (3.25, 0.75) {$V$}; 
\node at (4.25, 0.25) {$Y$}; 


\node (z) at (0.25, -0.7) {$q$}; 
\node (z) at (0.75, -0.7) {$r$}; 
\node (z) at (1.25, -0.7) {$l$}; 
\node (z) at (1.75, -0.7) {$q$}; 
\node (z) at (3.25, -0.7) {$t$}; 
\node (z) at (4.25, -0.7) {$s$}; 
\node (z) at (5.25, -0.7) {$p$}; 


%%%%%%%%%%%%%%%%%%%%%%%%%
\draw [dashed] (0, 3) -- (0, 2.5) -- (5.5,2.5) -- (5.5, 3);
\draw (0, 3) -- (0, 3.5) -- (0.5, 3.5) -- (0.5, 3) -- (0, 3);
\draw (0.5, 3.5) -- (0.5, 4) -- (1, 4) -- (1, 3) -- (0.5, 3);
\draw (1, 4) -- (1, 4.5) -- (1.5, 4.5) -- (1.5, 3) -- (1, 3);
\draw [dashed] (1.5, 3) -- (1.5, 2.5);

\draw (3, 3.5) -- (3, 4) -- (3.5, 4) -- (3.5, 3.5) -- (3, 3.5);
\draw (4, 3) -- (4, 3.5) -- (4.5, 3.5) -- (4.5, 3) -- (4, 3);
\draw (5, 3) -- (5.5, 3);


\node (z) at (0.25, 2.3) {$q$};
\node (z) at (0.75, 2.3) {$k$};
\node (z) at (1.25, 2.3) {$q$};
\node (z) at (3.25, 2.3) {$l$};
 \node (z) at (4.25, 2.3) {$m$};
\node (z) at (5.25, 2.3) {$p$};   

\draw[dashed] (3, 3.5) -- (3, 2.5);
\draw[dashed] (3.5, 3.5) -- (3.5, 2.5);

\draw[dashed] (4, 3) -- (4, 2.5);
\draw[dashed] (4.5, 3) -- (4.5, 2.5);
\draw[dashed] (5, 3) -- (5, 2.5);

\draw[thick, snake, densely dotted, rounded corners=0.5mm] (1.5, 4.5) -- (3, 4);
\draw[thick, snake, densely dotted, rounded corners=0.5mm] (3.5, 4) -- (4, 3.5);
\draw[thick, snake, densely dotted, rounded corners=0.5mm] (4.5, 3.5) -- (5, 3);

\node (z) at (0.25, 3.25) {$Z$}; 
\node (z) at (0.75, 3.25) {$N$}; 
\node (z) at (0.75, 3.75) {$M$}; 
\node[red] (z) at (1.25, 3.25) {$N$}; 
\node[red] (z) at (1.25, 3.75) {$X$}; 
\node (z) at (1.25, 4.25) {$Z$}; 

\node  (z) at (3.25, 3.75) {$X$}; 
\node at (4.25, 3.25) {$N$}; 

\node at (6, 3) {}; 
\end{tikzpicture}
\begin{tikzpicture}[shorten >=1pt,node distance=2cm,on grid,auto] 
   \node[state,accepting] (i) [initial]  {$[q, Z, p]$}; 
     \node[state] (u) [below left=of i ]  {$[r, X, s]$}; 
  \node[state] (w) [below right=of u ]  {$[l, W, s]$}; 
     \node[state] (v) [right=of i]  {$[k, M, m]$}; 
    \path[->]          
    (i) edge [blue] node {$Y$} (u)
     (i) edge  [red, bend left, above] node {$N$} (v)
     (v) edge [red, bend left, below] node {$X$} (i)
    (u) edge [blue] node {$\varepsilon$} (w)
      (w) edge [blue]  node {$V$} (i);
\end{tikzpicture}
\\
	\caption{A possible computations of the PDA $M$ leading to the two different stack contents (left) and a corresponding part of NFA recognizing $L_{[q, Z, p]}$ (right).}
\label{densex}
\end{figure}
\begin{proof}
Consider an example of possible computation of a PDA $M$ in Figure~\ref{densex}. Without loss of generality suppose that $M$ has the following transitions from state $q$ to $q$, satisfying the criteria from the theorem:
\begin{itemize}
\item $\delta(q, \sigma, Z) \in (r, YX), (k, NM)$;
\item $\delta(r, \sigma', X) \in (l, W)$;
\item $\delta(l, \sigma'', W) \in (q, VZ)$;
\item $\delta(k, \sigma''', M) \in (q, XZ)$.
\end{itemize}
According to the construction of an NFA $\mathcal{A} = (S,\Gamma,\delta',[q_I, Z_0, q_f], t_f)$ recognizing pushdown store language of $M$ \cite{Malcher}, the NFA $\mathcal{A}$ has a state $[q, Z, p]$. Consider a new NFA $\mathcal{A}_{[q, Z, p]} = (S,\Gamma,\delta',[q, Z, p], \{[q, Z, p]\})$ with $L_{[q, Z, p]} = L(\mathcal{A}_{[q, Z, p]})$, which is illustrated in Figure~\ref{densex} (right). If there are at least two computations of PDA resulting in different stack contents for the same state $q$ and top symbol $Z$, where $[q, Z, p]$ is meaningful triple, then there are two paths $a$ and $b$ from $[q, Z, p]$ to $[q, Z, p]$ in NFA $\mathcal{A}_{[q, Z, p]}$ such that $ab \neq ba$. Thus, $L(\mathcal{A}_{[q, Z, p]})$ is not commutative and by Lemma~\ref{polyslender}, $P(M)$ has exponential density. If there is only variant of the stack contents, than NFA $\mathcal{A}_{[q, Z, p]}$ has only one path from $[q, Z, p]$ to $[q, Z, p]$ or several paths satisfying $ab = ba$ for each pair $a, b$ of these paths, therefore $L(\mathcal{A}_{[q, Z, p]})$ is commutative. If it holds for every meaningful triple, then $P(M)$ has polynomial density.
\end{proof}





