\section{Boolean circuit for CFL-reachability}
\label{sec:circuit}
In this section we build generalization of classical Boolean circuit for context-free recognition from the Brent-Goldschlager-Rytter algorithm \cite*{Brent, Rytter} in order to evaluate CFL-reachability problem.
\subsection{Circuit description}
\label{Bdesc}
The idea of the Brent-Goldschlager-Rytter parallel algorithm is based on replacement of ordinary parse trees with a equivalent system of logical dependencies. We can adapt this system for graph input as follows (context-free grammar is in the Chomsky normal form):
\begin{enumerate}
\item $\frac{i \xrightarrow{a} j}{A(i , j)}$  --- if edge from the node $i$ to the node $j$ has label ``$a$'' and $A \rightarrow a \in P$, then there is a parse tree with nonterminal $A$ at the root deriving a path from $i$ to $j$.
\\
\item $\frac{B(i , j)}{\frac{A}{C}(i , j :: z)}$ --- creating a gap on the right: if a parse tree with the nonterminal $B$ at the root deriving a path from $i$ to $j$ exists and $A \rightarrow BC \in P$, then a parse tree with nonterminal $A$ at the root containing a smaller subtree with the nonterminal $C$ deriving path from $j$ to $z$ (``gap''), where $1 \le z \le n$, may exist.
\\
\item $\frac{C(j  , z)}{\frac{A}{B}(i :: j  , z)}$ --- creating a gap on the left: if a parse tree with the nonterminal $C$ at the root deriving a path from $j$ to $z$ exists and $A \rightarrow BC \in P$,  then a parse tree with nonterminal $A$ at the root containing a smaller subtree with the nonterminal $B$ deriving a path from $i$ to $j$ (``gap''), where $1 \le i \le n$, may exist.
\\
\item  $\frac{B(i, j), \frac{A}{B}(i :: j  , z)}{A(i, z)}$ --- filling the gap: if a parse tree with the nonterminal $A$ at the root deriving a path from $i$ to $z$, which cointains the gap represented by a node labelled $B$ deriving a path from $i$ to $j$ and a parse tree with the nonterminal $B$ at the root deriving a path from $i$ to $j$ exist, then the whole parse tree with the nonterminal $A$ at the root deriving a path from $i$ to $z$ exists.
\\
\item $\frac{\frac{A}{E}(i , j :: w, z), \frac{E}{D}( j , u :: v , w)}{\frac{A}{D}(i, u :: v , z)}$ --- combination of conditional propositions: if there is a parse tree with the nonterminal $A$ at the root deriving a path from $i$ to $z$ with the gap represented by a node labelled $E$ deriving a path from $j$ to $w$ and there is a parse tree with the nonterminal $E$ at the root deriving a path from $j$ to $w$, which contains gap represented by some node $D$, then there is a parse tree with the nonterminal $A$ at the root deriving a path from $i$ to $z$, which contains the gap represented by the above mentioned node $D$.
\end{enumerate}


Using logical dependencies above, elements of the circuit can be described with two types of elements: $x_{A, i,  j}$, which is true when a parse tree with the nonterminal $A$ at the root deriving a path from $i$ to $j$ exists, and $y_{A, i,  j, B, k, l}$ for conditional propositions, which is true when there is a parse tree with the nonterminal $A$ at the root deriving a path from $i$ to $j$ containg a hole instead of a subtree with the nonterminal $B$ at the root deriving a path from $k$ to $l$.


Therefore the circuit contains the following gates:
\begin{itemize}
\item \textit{Input}: input gates, one gate for every edge of the input graph and alphabet symbol; evaluates to true if the corresponding edge has an appropriate label
\item internal AND-gates for evaluating the combinations of conditional propositions and filling the gaps
\item internal OR-gates for creating gaps: conditional proposition $y_{A, i,  j, E, k, l}$ can be obtained by taking the gap from the right subtree (if $x_{B, i,  z}$ is true and $y_{C, z,  j, E, k, l}$ is true) or by taking the gap from the left subtree ($y_{B, i,  z, E, k, l}$ is true and $x_{C, z,  j}$ is true) for rule $A \rightarrow BC \in P$ . Notice that if start and end nodes of the tree with the gap coincide with start and end nodes of the gap ($i = k, z = l$ for the left case, $z = k, j = l$ for the right case), then the gap just added to the left or right side of another tree.
\item  \textit{Output}: output gates, one gate for every pair of graph nodes and nonterminal; evaluates to true if appropriate parse tree exists.
\end{itemize}

\subsection{Bounds on the circuit size and depth}
\paragraph{Circuit size}
\begin{lemma}
Let $G = (\Sigma, N, P)$ be a context-free grammar in Chomky normal form and let $n$ be a number of nodes in the input graph. Then the circuit described in \ref{Bdesc} has $O(n^6)$ elements.
\end{lemma}
\begin{proof} Let's count circuit gates by type. The number of input gates is $|\Sigma|n^2$ and the number of output gates is $|N|n^2$. Also there are $|N|^2n^5$ OR-gates for creating the gaps and $|N|^2n^4$ AND-gates for filling gaps. It is easy to see that the maximal number of elements is needed for combinations of conditional propositions: there are $|N|^3n^6$ such elements ( $|N|^2n^4 \times |N|n^2$ ). Therefore the circuit contains at most $O(n^6)$ elements.
\end{proof}
\paragraph{Circuit depth}
Before we analyze the depth of the circuit we shall prove the following statement.
\begin{lemma}
\label{bigsubtree}
Let $G = (\Sigma, N, P)$ be a context-free grammar in Chomky normal form. Then every parse tree with $k$ leaves contains a middle node (``critical'' node) that spans over more than $\frac{1}{3}k$ and at most $\frac{2}{3}k$ leaves.
\end{lemma}
\begin{proof} Because the grammar is in the Chomsky normal form, every node of a parse tree has two children. Going from the root to leaves, we can choose the largest of two subtrees at each node while the current subtree has more than $\frac{2}{3}k$ leaves. Obviously, the first node that has at most $\frac{2}{3}k$ leaves is bound to have more than $\frac{1}{3}k$ leaves due to binary branching. 
\end{proof}
\begin{lemma}
\label{depthproof}
Let $l$ be a length of the shortest path from node $i$  to node $j$ labelled by string from $L(G_A)$. Then proposition $A(i, j)$ or $\frac{A}{D}(i , u :: v, j)$ has a proof of height $O(\log l)$.
\end{lemma}
\begin{proof} Proof by induction on $l$.
\\ \textit{Induction hypotesis}: every proposition with no more than $\frac{2}{3}l$ leaves has a proof of logarithmic height ($O(\log l)$).


At first, consider the proposition $A(i, j)$. By Lemma \ref{bigsubtree}, a corresponding tree has a critical node $E$ with two children $B$ and $C$ (for $E \rightarrow BC \in P$). Then we can write the proof tree as follows.
\begin{prooftree}
\AxiomC{$B(u, l)$}
\AxiomC{$C(l, v)$}
\BinaryInfC{$E(u, v)$}
\AxiomC{$\frac{A}{E}(i , u :: v , j)$}
\BinaryInfC{$A (i, j)$}
\end{prooftree}
Because $E$ is the critical node, each of the propositions $B(u, l), C(l, v)$, $\frac{A}{E}(i , u :: v , j)$ has no more than $\frac{2}{3}l$ leaves and therefore has a proof of logarithmic height by inductive hypothesis.


There are two cases for the second proposition $\frac{A}{D}(i , u :: v, j)$.

\begin{figure}
\begin{tikzpicture}
\draw (0,0) -- (1.5,0) ;
\draw (0,0) -- (2,4) ;
\draw (2,4) -- (4,0) ;
\draw (1.5,0) -- (2,1) ;
\draw (2,1) -- (2.5,0) ;
\draw (2.5,0) -- (4,0) ;
\draw (0.75,0) -- (2,2.5) ;
\draw (2,2.5) -- (3.25,0) ;
\draw [thick,densely dotted, rounded corners=1mm](2,4) --  (1.9, 3.5) -- (2.2, 3) -- (2,2.5);
\draw [thick,densely dotted, rounded corners=1mm] (2,2.5) -- (1.9, 1.8) -- (2.2, 1.4) -- (2,1);
\node (start) {}; 
\node [below=0.05cm of start] (i1) {$i$};
\node [right=3.7cm of i1] (j1) {$j$}; 
\node [right=0.3cm of i1] (k1) {$k$}; 
\node [right=1cm of i1] (u1) {$u$}; 
\node [right=2.2cm of i1] (v1) {$v$}; 
\node [right=2.9cm of i1] (m1) {$m$}; 
\node [right=2.0cm of start] (middle1) {}; 
\node (A1) at (2, 4.3) {$A$}; 
\node (D1) at (1.9, 1.2) {$D$};
\node [circle, fill=red, inner sep = 1.5pt, minimum size=1pt](cr) at (2, 2.5) {};
\draw[red, semitransparent] (2, 2.5) circle[radius=3.5pt];
\node (E1) at (1.97, 2.7) {$E$};
\node [circle, fill=black, inner sep = 1.5pt, minimum size=1pt](pD1) at (2,1) {};
\node [circle, fill=black, inner sep = 1.5pt, minimum size=1pt](pA1) at (2,4) {};

%second picture
\draw (6,0) -- (8.5,0) ;
\draw (9.5,0) -- (11.5,0) ;
\draw (6,0) -- (8.75,4) ;
\draw (8.75,4) -- (11.5,0) ;
%\draw (7,0) -- (8.8,3) ;
%\draw (8.8,3) -- (10.5,0) ;
\draw (8.5,0) -- (9,1) ;
\draw (9,1) -- (9.5,0) ;
\draw (6.5, 0) -- (7.8,2) ;
\draw (7.8,2) -- (8,0) ;
\draw (8,0) -- (9,2.5) ;
\draw (9,2.5) -- (10.5,0) ;
\draw (8.6,3) -- (9,2.5) ;
\draw (8.6,3) -- (7.8,2) ;
\node (start2) at (6, 0) {}; 
\node [below=0.05cm of start2] (i2) {$i$};
\node [right=0.2cm of i2] (k2) {$k$}; 
\node [right=1.5cm of i2] (m2) {$m$}; 
\node [right=2.1cm of i2] (u2) {$u$}; 
\node [right=3.2cm of i2] (v2) {$v$}; 
\node [right=4.2cm of i2] (w) {$w$}; 
\node [right=5.2cm of i2] (j2) {$j$}; 
\node (A2) at (8.75, 4.3) {$A$}; 
\node (E2) at (8.79, 3.1) {$E$}; 
\node (B) at (7.98, 2) {$B$}; 
\node (C) at (9.2,2.5) {$C$}; 
\node (D2) at (9.2,1) {$D$}; 
\node [circle, fill=black, inner sep = 1.5pt, minimum size=1pt](pB) at (7.8,2) {};
\node [circle, fill=red, inner sep = 1.5pt, minimum size=1pt](crU) at (7.6,1) {};
\draw[red, semitransparent] (7.6, 1) circle[radius=3.5pt];
\node [circle, fill=black, inner sep = 1.5pt, minimum size=1pt](pA2) at (8.75,4) {};
\node [circle, fill=black, inner sep = 1.5pt, minimum size=1pt](pC) at (9,2.5) {};
\node [circle, fill=black, inner sep = 1.5pt, minimum size=1pt](pE2) at (8.6,3) {};
\node [circle, fill=black, inner sep = 1.5pt, minimum size=1pt](pD2) at (9,1) {};
\draw [thick,densely dotted, rounded corners=1mm](9,2.5) -- (9.1, 2) -- (8.9, 1.3) -- (9,1) ;
\draw [thick,densely dotted, rounded corners=1mm](8.75,4) -- (8.8, 3.5) -- (8.6,3);
\draw [thick,densely dotted, rounded corners=1mm]  (7.8,2) -- (7.6, 1.5) -- (7.6,1.05);
\end{tikzpicture}
% Use the relevant command to insert your figure file.
% For example, with the graphicx package use
% figure caption is below the figure
\caption{ Cases for conditional proposition $\frac{A}{D}(i , u :: v, j)$: (left) a subtree with the critical node at the root contains the hole; (right) a subtree with the critical node at the root contains only leaves corresponding to the path $i \rightarrow u$.}
\label{proofcases}       % Give a unique label
\end{figure}



\begin{enumerate}
\item \textit{A subtree with the critical node at the root contains the hole $u :: v$.} 
\\ Let $E$ be a label of the critical node. Then a corresponding subtree splits the path from the vertex $i$ to the vertex $u$ onto two paths: $i \rightarrow k$ and $k \rightarrow u$ and the path from the vertex $v$ to the vertex $j$ onto $v \rightarrow m$ and $m \rightarrow j$ respectively (as illustrated in Figure \ref{proofcases} (left)). The next proof tree represents the proof of the proposition $\frac{A}{D}(i , u ::v, j)$ in this case. 
\begin{prooftree}
\AxiomC{$\frac{A}{E}(i , k :: m , j)$}
\AxiomC{$\frac{E}{D}(k , u :: v , m)$}
\BinaryInfC{ $\frac{A}{E}(i , u :: v , j)$}
\end{prooftree}
Conditional proposition $\frac{A}{E}(i , k :: m , j)$ and $\frac{E}{D}(k , u :: v , m)$ have no more than $\frac{2}{3}l$ leaves, so $\frac{A}{E}(i , u :: v , j)$ has a proof of logarithmic height.

\item \textit{A subtree with the critical node at the root contains only leaves corresponding to the path from the vertex $i$ to the vertex $u$ in the input graph.}


This case is illustrated in Figure \ref{proofcases} (right). Consider the biggest subtree which contains a subtree with the critical node at the root and has leaves corresponding only to the path from the vertex $i$ to the vertex $u$ of the input graph. Let $B$ be a label at the root of this tree and $k \rightarrow m$  be a corresponding path in the input graph. Let $E$ be a parent node of node $B$ and let $C$ be a second child of $E$ (for $E \rightarrow BC \in P$). A parse tree with the root labelled by $C$ has some leaves corresponding to the path $v \rightarrow j$, so let $w$ be a vertex which splits this path on two. Now we are able to write the proof of $\frac{A}{E}(i , u :: v , j)$.
\begin{prooftree}
\AxiomC{$\frac{A}{E}(i , k :: w , j)$}
\AxiomC{$B(k, m)$}
\UnaryInfC{$\frac{E}{C}(k , m :: w)$}
\AxiomC{$\frac{C}{D}(m , u :: v , w)$}
\BinaryInfC{$\frac{E}{D}(k , u ::v , w)$}
\BinaryInfC{ $\frac{A}{D}(i , u :: v , j)$}
\end{prooftree}
The subtree with the root labelled $B$ obviously contains at least $\frac{2}{3}l$ leaves, so each of the other propositions $\frac{C}{D}(m , u :: v , w)$ and $\frac{A}{E}(i , k :: w , j)$ has no more then  $\frac{2}{3}l$ leaves. By induction hypothesis they have a proof of logarithmic height. Thus, proposition  $\frac{A}{D}(i , u :: v , j)$ can be proved via 4 steps using propositions with the proofs of logarithmic depth.
\\The case when the subtree with the critical node at the root contains only leaves corresponding to the path from the vertex $v$ to the vertex $j$ in the input graph is held symmetrically.
\end{enumerate}
Before we will make a conclusion, the following definition should be introduced.
\begin{definition}
\label{L}
Let $G = (\Sigma, N, P)$ be a context-free grammar and $D$ be a directed labelled graph. Then $\mathzapf{L}$ is the length of longest shortest path $i \pi j$ for all triples $(A, i, j)$, where $j$ is reachable from $i$ via path labelled by $l(i\pi j) \in L(G_A)$ and $A \in N$.
\end{definition}
\end{proof}
Using lemmas above, we can conclude the following.
\begin{corollary}
\label{coldepth}
Let $G = (\Sigma, N, P)$ be a context-free grammar in Chomky normal form and let $\mathzapf{L}$ be the length of the maximum in all nonterminals and all pairs of graph nodes shortest path labelled by a string from $L(G)$ (see Definition \ref{L}). Then there is a uniform family of circuits for solving CFL-reachability problem on the graphs with $n$ nodes, which are of depth $O(\log n \log \mathzapf{L})$ and contain $O(n^6)$ elements.
\end{corollary}


The depth and therefore the efficiency of the circuit depends on the value of parameter $\mathzapf{L}$. We investigate bounds for $\mathzapf{L}$ in the next two sections.
