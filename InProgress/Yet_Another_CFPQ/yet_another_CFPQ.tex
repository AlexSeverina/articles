\documentclass{article}

\usepackage[utf8]{inputenc}
\usepackage[T2A]{fontenc}
\usepackage[utf8]{inputenc}
\usepackage[russian]{babel}
\usepackage{amsmath}

\title{Yet Another CFPQ: $O(n^3)$}
\author{Arseniy Terekhov}
\date{October 2019}


\begin{document}

\maketitle

\section{Идея}
Если посмотреть на исходный алгоритм, то можно понять, что мы делаем настолько много лишних операций, что это выливается в ненужный $O(n^2)$. Если немного аккуратнее вычислять произведения матриц и более точно считать асимтотику, все итерации алгоритма "свернуться" в одно перемножение матриц.

\section{Алгоритм}
Пускай у нас есть граф $(V, E, L)$, где $L$ - отображение, сопостовляющее рёбрам терминалы. Так же у нас есть грамматика $(\Sigma, N, P, S)$. Я предполагаю то, что в процессе будет строится матрица с той самой экзотической операцией умножения двух нетерминальных множеств.

Давайте за $A_i$ обозначим обозначим следующие матрицы:
\begin{equation*}
    \begin{cases}
        $A_1 = \{a_{ij} = \{n \in N: n \rightarrow L(E_{ij})  \in P\}\}$ \\
        $A_i = (\sum_{j=1}^{i-1}A_{j})^2$ & $i > 1$
    \end{cases}
\end{equation*}

Обратите внимание на то, что $A_i$ не стремятся к транзитивному замыканию, к нему стремятся их частичные суммы. То есть критерий остановки следующий:
\[\sum_{j=1}^{i-1}{A_j} = \sum_{j=1}^i{A_j}\]

Давайте распишем итерации получения $A_i$:

\begin{itemize}
    \item $A_1$ база
    \item $A_2 = A_1^2$
    \item $A_3 = (A_1 + A_2)^2 = A_1^2 + A_1*A_2 + A_2*A_1 + A_2^2$
    \item $A_4 = (A_1 + A_2 + A_3)^2 = (A_1 + A_2)^2 + (A_1 + A_2)*A_3 + A_3*(A_1 + A_2) + A_3^2$
    \item $A_n = (A_1 + A_2 +...+ A_{n-2} + A_{n-1})^2 = (A_1 + A_2 +...+ A_{n-2})^2 + (A_1 + A_2 +...+ A_{n-2})*A_{n-1} + A_{n-1}*(A_1 + A_2 +...+ A_{n-2}) + A_{n-1}^2$
\end{itemize}

Теперь заметим, что первые квадраты в левой части повторяются и на самом деле уже посчитаны на предыдущем шаге. Так же за $B_i$ обозначим частичные суммы $A_i$, которые можно считать итеративно:
\[B_i = \sum_{j=1}^i A_i = B_{i-1} + A_i\]

С этими наблюдениями сумма превращается в красотень:

\begin{itemize}
    \item $A_1$ база
    \item $A_2 = A_1^2$
    \item $A_3 = (A_1 + A_2)^2 = A_2 + B_1*A_2 + A_2*B_1 + A_2^2$
    \item $A_4 = (A_1 + A_2 + A_3)^2 = A_3 + B_2*A_3 + A_3*B_2 + A_3^2$
    \item $A_n = (A_1 + A_2 +...+ A_{n-2} + A_{n-1})^2 = A_{n-1} + B_{n-2}*A_{n-1} + A_{n-1}*B_{n-2} + A_{n-1}^2$
\end{itemize}

Ну и собственно ответ на задачу - последняя $B_n$.

\section{Асимптотика}
Мы знаем, что в худшем случае кол-во итераций может быть $n^2$. Давай-те посчитаем асимптотику получения матрицы $A_{n^2}$. Понятно, что если мы посчитем $A_{n^2}$, то и посчитаем все предыдущие.

Давайте за $SMM(A, B)$ обозначим функцию асимптотики перемножения двух разряженных матриц. Есть простое утверждение, что
\[O(SMM(A, B)) = O(|A|*n) = O(|B|*n)\]
где $|A|$ - кол-во элементов в матрице A. Оно следует из того, что каждый элемент левой матрицы может принимать участие в O(n) бинарных операциях.
При этом это грубая оценка.

На самом деле мы хотим, чтобы $A_i$ попарно не пересекались (поэлементно). Поэтому предыдущее определение нам не совсем подходит. Давайте немного переопределим $A_i$:

\[A_i = (\sum_{j=1}^{i-1}A_{j})^2 / \sum_{j=1}^{i-1} A_i\]

Таким образом $A_i$ попарно не пересекаются, то есть
\[\forall k, l \forall i, j {A_k}_{ij} \cap {A_k}_{ij} = \emptyset \]
Так же $B_i$ не меняются, а все прошлые выводы остаются верными (просто тогда итерации выглядят уж очень страшно, поэтому я не стал добавлять это определение в начало).

За $O(A_i)$ я обозначаю функцию кол-ва итераций, необходимых для получения $A_i$.

\[O(A_{n^2}) = O(A_{n^2-1}) + O(SMM(B_{n^2-2}, A_{n^2-1})) + O(SMM(A_{n^2-1}, B_{n^2-2})) + O(SMM(A_{n^2-1}, A_{n^2-1}))\]
\[= O(A_1) + O(SMM(A_1, A_1))\ +\]
\[\sum_{i=3}^{n^2}(O(SMM(B_{i-2}, A_{i-1})) + O(SMM(A_{i-1}, B_{i-2})) + O(SMM(A_{i-1}, A_{i-1})))\]
\[= O(n^2) + |A_1|*O(n) + \sum_{i=3}^{n^2}{|A_{i-1}|*O(n) +|A_{i-1}|*O(n) + |A_{i-1}|*O(n)}\]
\[= O(n^2) + |A_1|*O(n) + \sum_{i=2}^{n^2-1}{|A_{i}|*O(n)}\]
\[= O(n^2) + \sum_{i=1}^{n^2-1}{|A_{i}|*O(n)}\]
\[= O(n^2) + O(n)*\sum_{i=1}^{n^2-1}{|A_{i}|}\]
\[= O(n^2) + O(n)*O(n^2)*|N|\]
\[= O(n^3)\]

В предпоследнем переходе использовался факт того, что все $A_i$ попарно не пересекаются, а значит, сумма сумм их элементов меньше, чем $n^2*|N|$.
Единственное, стоит сказать, что операция разности множеств может быть сделана во время вычисления $A_i$:

\[A_n = A_{n-1} + B_{n-2}*A_{n-1} + A_{n-1}*B_{n-2} + A_{n-1}^2\]

В момент вычисления $A_n$ мы уже знаем $B_{n-1}$. Так что можно просто не записывать те результирующие нетерминалы в $A_n$, если они уже есть в $B_{n-1}$.

Ну и так же совсем очевидно, что на вычисление всех $B_n$ тратиься хотя бы $O(n^3)$ времни.

\section{Заключение}
Это очень черный черновик, но основные моменты раскрыты.

\end{document}
