\chapter{Сравнение и соотнесение} \label{chaptComp}

В данной главе представлено сравнение полученных результатов с основными существующими решениями в области анализа динамически формируемых строковых выражений. Описание существующих решений представлено 
в разделе~\ref{SELToolsDescr} данной работы.

В качестве инструментов, с которыми производилось сравнение, выбраны следующие: Alvor~\cite{AlvorUrl}, JSA~\cite{JSAUrl}, PHPSA~\cite{PHPSAUrl}, IntelliLang~\cite{IntelliLang}, Varis~\cite{Varis}. Так же проводилось сравнение с инструментом, названным нами условно AbsPars, реализованным авторами работ по абстрактному синтаксическому анализу~\cite{LrAbstract1, LrAbstract2, LRAbstractParsingSema}. Несмотря на то что в свободном доступе реализации 
алгоритма, изложенного в указанных статьях, не обнаружено, самими авторами приводятся достаточно подробные результаты апробации реализации алгоритма, что позволяет сделать некоторые выводы о его 
основных возможностях. Кроме того, стоит отметить, что Varis является очень молодым инструментом: впервые он был представлен в 2015 году на конференции ICSE\footnote{Международная конференция по 
разработке программного обеспечения ICSE (International Conference on Software Engineering). Сайт конференции (дата обращения: 29.07.2015): \url{http://2015.icse-conferences.org/}.}. По этой причине детальная оценка его возможностей затруднена. 

Для сравнения инструментов были выбраны критерии, представленные в таблице~\ref{tbl:metricsForComparison}. Данные критерии позволяют оценить два основных результаты данной работы: алгоритм синтаксического анализа динамически 
формируемых выражений и архитектуру инструмента. 

\begin{table} [h]
  \centering
  \parbox{14cm}{\caption{Критерии сравнения инструментов анализа динамически формируемых строковых выражений}\label{tbl:metricsForComparison}}
  \begin{tabular}{| p{4.5cm} | p{3cm} | p{8.5cm} |}
  \hline                               
  \hline
  Критерии & Название колонки в таблице с результатами сравнения~\ref{tbl:comparison} &Описание \\
  \hline
  \multicolumn{3}{|c|}{Алгоритм синтаксического анализа динамически формируемых выражений}\\
  \hline 
  Построение леса разбора     & Лес разбора  & Предоставляет ли алгоритм и соответствующий инструмент функциональность по построению леса разбора динамически формируемого кода.\\
  Поиск синтаксических ошибок & Синт. ошибки & Реализует ли алгоритм и соответствующий инструмент поиск синтаксических ошибок в динамически формируемом коде.\\
  Поиск семантических ошибок  & Сем. ошибки  & Предоставляет ли алгоритм возможности для поиска семантических ошибок в динамически формируемом коде.\\
  \hline 
  \multicolumn{3}{|c|}{Архитектура}\\
  \hline 
  Платформа для разработки & Платформа & Предоставляется ли в явном виде платформа для создания новых инструментов статического анализа динамически формируемых выражений.  \\ 
  Модульность обработки & Модульность & Выделены ли отдельные независимые шаги обработки или же анализ является монолитным. Реализованы ли соответствующие выделенным шагам независимые компоненты.\\
  \hline
  \hline
  \end{tabular}
\end{table}

В таблице~\ref{tbl:comparison} приведены основные результаты сравнени. Были использованы следующие обозначения.
\begin{itemize}
    \item '$+$' --- функциональность, соответствующая критерию, полностью реализована.
    \item '$-$' --- функциональность, соответствующая критерию, полностью не реализована.
    \item '$+-$' --- соответствующая функциональность реализована частично.   
\end{itemize}

\begin{table} [h]
  \centering
\parbox{14cm}{\caption{Сравнение инструментов анализа динамически формируемых строковых выражений}\label{tbl:comparison}}
\begin{threeparttable}
  
  \begin{tabular}{| p{2.7cm} || p{2.4cm} | p{2.4cm} | p{2.4cm} | p{2.4cm} | p{2.4cm}l |}
  \hline                               
  \hline
  {Инструмент}   &\centering {Лес разбора}           &\centering {Синт. ошибки}        &\centering {Сем. ошибки}    &\centering {Платформа}         &\centering {Модульность} & \\
  \hline                                                                                                                                       
  AbsPars        &\centering  $+-$\tnote{*}          &\centering  $+$                  &\centering  $+$             &\centering  $-$                &\centering  $-$        & \\
  Alvor          &\centering  $-$                    &\centering  $+$                  &\centering  $-$             &\centering  $-$                &\centering  $+$        &\\
  JSA            &\centering  $-$                    &\centering  $+$                  &\centering  $-$             &\centering  $-$                &\centering  $-$        &\\
  PHPSA          &\centering  $-$                    &\centering  $+$                  &\centering  $-$             &\centering  $-$                &\centering  $-$        &\\
  IntelliLang    &\centering  $-$                    &\centering  $+$                  &\centering  $+$             &\centering  $+-$\tnote{**}     &\centering  $+$        &\\
  Varis          &\centering  $+$\tnote{***}         &\centering  $+$                  &\centering  $-$             &\centering  $-$                &\centering  $-$        &\\
  YC.SEL.SDK     &\centering  $+$                    &\centering  $-$\tnote{****}      &\centering  $+$             &\centering  $+$                &\centering  $+$        &\\
  \hline
  \hline
  \end{tabular}\small{
  \begin{tablenotes}
            \item[*] В работе~\cite{LRAbstractParsingSema} утверждается, что поддерживаются атрибутные грамматики для описания языка. Это даёт возможность описать семантику построения леса. При этом в работе обсуждаются проблема уменьшения точности анализа при использовании семантики, а построение леса не обсуждается. Возможность построения корректного конечного представления леса разбора требует отдельного исследования.
            \item[**] IntelliLang использует возможности платформы IntelliJ IDEA~\cite{IDEA}, поэтому средства для расширения некоторых возможностей унифицировано, однако самостоятельной платформы для анализа встроенных языков не предоставляется.
            \item[***] Такие свойства конструируемой структуры данных, как конечность, а так же возможности её дальнейшей обработки в общем виде не обсуждаются.
            \item[****] Возможность диагностики синтаксических ошибок не реализована на данный момент в рамках платформы. Однако данная задача может быть решена другими инструментами, так как обработка производится по шагам и возможно добавить новый без существенных затрат.
  \end{tablenotes}    }
  \end{threeparttable}
\end{table}

Метод реинжиниринга встроенного программного кода сформулирован впервые. Большинство существующих работ и инструментов посвящены решению конкретных задач, часто входящих в состав мероприятий по реинжинирингу. 
Вместе с этим работа~\cite{DSQLQualityMesure} рассматривает вопрос учёта встроенного SQL-кода при оценке качества информационных систем и описывает метод решения данной задачи. Однако, метод
затрагивает только часть области и является более специальным и узно направленным, чем предложенный в данной работе. 

Проведённое сравнение позволяет выявить несколько аспектов.
\begin{itemize}
    \item Многие инструменты могут быть расширены, однако ранее не предоставлялось полноценного самостоятельного инструментария специально предназначенного для создания новых инструментов для обработки динамически формируемых выражений.
    \item Основная цель большинства инструментов --- это проверка корректности динамически формируемого кода. Прежде всего осуществляется поиск синтаксических ошибок.
    \item Инструменты реализованы на основе разных подходов и предназначены для решения разных задач, поэтому детальное сравнение их возможностей, производительности и других аспектов не представляется возможным. Например, инструмент PHPSA изначально создавался для решения задачи проверки корректности динамически формируемых выражений и отсутствие подсветки синтаксиса не является его недостатком.
    \item Существование многочисленных инструментов для решения различных задач и создание новых говорит о том, что обработка встроенных языков является актуальной задачей.
\end{itemize}

В результате можно утверждать, что YC.SEL.SDK является единственной полноценной платформой для создания различных инструментов статического анализа динамически формируемых выражений, применимых в разных областях и обладающих широкими функциональными возможностями.


