\section{Conclusion and Future Work}

We provide an CPU and GPGPU based context-free path querying implementations for RedisGraph and show that CFPQ can be fast enough to analize real-world data.
But our implementations are on prototype stage and we should provide full integration of CFPQ to RedisGraph. 
First of all it is necessary to extend Cupher graph query language, which uses in RedisGraph, to support respective syntax for context-free constraints specification.
There is a proposal whish describes such syntax extension~\footnote{Proposal with path pattern syntax for openCypher: \url{https://github.com/thobe/openCypher/blob/rpq/cip/1.accepted/CIP2017-02-06-Path-Patterns.adoc}. It is shown that context-free constraints are expressible in proposrd syntax. Access date: 12.11.2019} and we plan to support proposed sintax in libcypher-parser~\footnote{Web page of libcypher-parser project: \url{http://cleishm.github.io/libcypher-parser/}. Access date: 12.11.2019} which uses in RedisGraph.

In current version we use CUSP matrix multiplication library for GPGPU utilization, but it may be better to use GraphBLAST\footnote{GraphBLAST project: \url{https://github.com/gunrock/graphblast}. Access date: 12.11.2019.}~\cite{yang2019graphblast} --- Gunrock\footnote{Gunrock project web page: \url{https://gunrock.github.io/docs/}. Access date: 12.11.2019.}~\cite{Wang:2017:GGG:3131890.3108140} based implementation of GraphBLAS API for GPGPU.
Firs of all, we should evaluate GraphBlast based implementation of CFPQ. 
Also, we should investigate to implement multi-GPU support for GraphBlast, because it should improve performance of huge real-world data processing.

Our implementations calculate queryes in respect to relational semantics, but in sume cases it is necessary to provide a path which satisfied constraints.
As we know, matrix based algorithm for single path or all paths semantix is not prvided yet, and it is a direction for future research.

Another important question for future research is how to update query result dynamically when data changes. 
Mechanism for result updating allows one to recalculate qury faster and use result as an index for other queryes.

Also, futer improvements of the dataset are requred.
For example, it is necessary to include real-world cases from static code analysis~\cite{!!!}.