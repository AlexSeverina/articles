\section{Conclusion and Future Work}

We implemented a CPU and GPGPU based context-free path querying for RedisGraph and showed that CFPQ can be performant enough to analyze real-world data.
However, our implementations are prototypes and we plan to provide full integration of CFPQ to RedisGraph.
First of all, it is necessary to extend Cypher graph query language, which is used in RedisGraph, to support syntax for specification of context-free constraints.
There is a proposal which describes such syntax extension\footnote{Proposal with path pattern syntax for openCypher: \url{https://github.com/thobe/openCypher/blob/rpq/cip/1.accepted/CIP2017-02-06-Path-Patterns.adoc}. It is shown that context-free constraints are expressible in proposed syntax. Access date: 12.11.2019} and we plan to support proposed the syntax in libcypher-parser\footnote{Web page of libcypher-parser project: \url{http://cleishm.github.io/libcypher-parser/}. Access date: 12.11.2019} which is used in RedisGraph.

Current version uses CUSP matrix multiplication library for GPGPU utilization, but it may be better to use GraphBLAST\footnote{GraphBLAST project: \url{https://github.com/gunrock/graphblast}. Access date: 12.11.2019.}~\cite{yang2019graphblast} --- Gunrock\footnote{Gunrock project web page: \url{https://gunrock.github.io/docs/}. Access date: 12.11.2019.}~\cite{Wang:2017:GGG:3131890.3108140} based implementation of GraphBLAS API for GPGPU.
First of all, we plan to evaluate GraphBLAST based implementation of CFPQ.
Also, we plan to investigate how multi-GPU support for GraphBLAST influences the performance of CFPQ in the case of processing huge real-world data.

Our implementations compute relational semantics of a query, but some problems require to find a path which satifies the constraints.
Best to our knowledge, there is no matrix-based algorithm for single path or all path semantics, thus we see it as a direction for future research.

Another important open question is how to update the query results dynamically when data changes.
The mechanism for result updating allows one to recalculate query faster and use the result as an index for other queries.

Also, further improvements of the dataset are required.
For example, it is necessary to include real-world cases from the area of static code analysis~\cite{Zheng:2008:DAA:1328897.1328464,veduradabatch,LPAR-21:Cauliflower_Solver_Generator_for}.
