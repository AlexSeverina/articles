\section{Evaluation and Discussion}

We evaluate all the described implementations on all the datasets and the queries presented.
We compare our implementations with~\cite{Mishin:2019:ECP:3327964.3328503} and~\cite{Kuijpers:2019:ESC:3335783.3335791}.
We measure the full time of query execution including all overhead on data preparation.
This way we can estimate the applicability of the matrix-based algorithm to real-world problems.

For evaluation, we use a PC with Ubuntu 18.04 installed.
It has Intel core i7-6700 CPU, 3.4GHz, DDR4 32Gb RAM, and Geforce GTX 1070 GPGPU with 8Gb RAM.

The results of the evaluation are summarized in the tables below.
We provide results only for a part of the collected dataset because of the page limit.
Running time is measured in seconds, RAM memory consumption is measured in megabytes unless specified otherwise.
Note that we provide results from the corresponding papers for all implementations except our own.
The cell is left blank if the time limit is exceeded, or if there is not enough memory to allocate the data.

The results of the first dataset \textbf{[RDF]} are presented in table~\ref{tbl:tableRDF}.
We can see that the running time of both CPU and GPGPU versions is small even for graphs with a big number of vertices and edges.
The relatively small number of edges of interest may be the reason for such behavior.
We believe it is necessary to extend the dataset with new queries which involve more different types of edges.
Also, we can see, that \textit{m4ri} version which uses dense bit matrices requires more memory.
Thus we recommend to use sparse matrices on GPGPU.

Geospecies dataset currently can be processed only by using CPU version and we compare our matrix-based CPU implementation with the result form~\cite{Kuijpers:2019:ESC:3335783.3335791} for \textit{AnnGram$_{\textit{rel}}$} algorithm\footnote{Only \textit{AnnGram} works correctly and fits limits, other implementations are faster, but either return an incorrect result or do not fit the memory .}.
Fortunately, both algorithms calculate queries under relational semantics.
The result is provided in the table~\ref{tbl:geo}.

As we can see, the matrix-based algorithm implemented for RedisGraph is more than 1000 times faster than the one based on annotated grammar implemented for Neo4j and uses more than 4 times less memory.
We can conclude that the matrix-based algorithm is better than other CFPQ algorithms for query evaluation under a relational semantics for real-world data processing.
CFPQ evaluation under other semantics (single path, all paths, etc) by using a matrix-based algorithm is a direction for future research.


The next is the \textbf{[FreeScale]} datatset.
We compare our implementations with two implementations form~\cite{Kuijpers:2019:ESC:3335783.3335791} which evaluate queries under relatonal semantics: \textit{Neo4j\_AnnGram${_\textit{rel}}$} and \textit{Neo4j\_Matrix}. The results are presented in table~\ref{tbl:tableFreeScale}.
The evaluation shows that sparsity of graphs (value of parameter \texttt{p}) is important both for implementations which use sparse matrices and for implementations which use dense matrices.
Note that the results for implementations for Neo4j are restored from graphics provided in~\cite{Kuijpers:2019:ESC:3335783.3335791}.
So, values are not precise, but it is possible to compare implementations.

Evaluation shows that our CPU version is comparable with \textit{Neo4j\_AnnGram$_{\textit{rel}}$} and for relatively dense graphs (each vertex has 10 connections) our implementation is faster. Moreover, while \textit{Neo4j\_Matrix} exceeded limits on the biggest graph, our implementation works fine.
This demonstrantes the importance of using of appropriate libraries for matrix-based algorithm implementation.
Also, we can see, that GPGPU version which utilizes sparse matrices is significantly faster than the other implementations.
Note, that for GPGPU versions we include time required for data transferring and formats convertion.

Finally, we conclude that the matrix-based algorithm paired with a suitable database and employing appropriate libraries for linear algebra is a promising way to make CFPQ applicable for real-world data analysis.
We show that SuiteSparse-based CPU implementation is performant enough to be comparable with GPGPU-based implementations on real-world data.
It means that we can handle more complex data.
We can also see, that more complex queries should be added to the dataset to make it more representable.


{\setlength{\tabcolsep}{0.4em}
\begin{table*}[h]
\caption{RDFs relation semantics query $G_1$}
\label{tbl:tableRDFRelationalPathIndexQ1}
\rowcolors{3}{}{lightgray}
\begin{tabular}{| l | c  c | c  c | c  c |}
    \hline
    \multirow{2}{*}{Name}	&	\multicolumn{2}{|c|}{RG\_CPU\textsubscript{rel}}	&	\multicolumn{2}{|c|}{RG\_CUSP\textsubscript{rel}}	&	\multicolumn{2}{|c|}{RG\_SPARSE\textsubscript{rel}}	 \\
    		& Time & Mem &  Time     & Mem & Time     & Mem  \\    
    \hline
    \hline
    atom-primitive              & 0.016 & 1.2  & 0.016 & 0.1 & 0.005 & 0.1 \\
biomedical-mesure-primitive & 0.016 & 0.6  & 0.012 & 0.1 & 0.005 & 0.1     \\
core                        & 0.004 & 0.3  & 0.022 & 2   & 0.01  & 0.1     \\
eclass\_514en                 & 0.067 & 13.8 & 0.075 & 14  & 0.166 & 16    \\
enzyme                      & 0.018 & 5.9  & 0.021 & 0.1 & 0.018 & 4       \\
foaf                        & 0.002 & 0.4  & 0.013 & 0.1 & 0.006 & 0.1     \\
funding                     & 0.006 & 0.5  & 0.019 & 0.1 & 0.006 & 0.1     \\
generations                 & 0.002 & 0.3  & 0.01  & 0.1 & 0.004 & 0.1     \\
go-hierarchy                & 0.091 & 16.3 & 0.433 & 650 & 0.108 & 121.2   \\
go                          & 0.604 & 28.8 & 0.59  & 70  & 0.365 & 30.2    \\
pathways                    & 0.011 & 0.1  & 0.019 & 0.1 & 0.007 & 0.1     \\
people\_pets                & 0.017 & 0.4  & 0.025 & 0.1 & 0.007 & 0.1     \\
pizza                       & 0.03  & 1.8  & 0.021 & 4   & 0.006 & 0.1     \\
skos                        & 0.001 & 0.1  & 0.008 & 0.1 & 0.004 & 0.1     \\
travel                      & 0.004 & 0.3  & 0.022 & 2   & 0.007 & 0.1     \\
univ-bench                  & 0.002 & 0.3  & 0.01  & 0.1 & 0.005 & 0.1     \\
wine                        & 0.017 & 3.5  & 0.032 & 6   & 0.009 & 0.1     \\
    \hline
  \end{tabular}
\end{table*}
}

{\setlength{\tabcolsep}{0.4em}
\begin{table*}[h]
\caption{RDFs relation semantics query $G_2$}
\label{tbl:tableRDFRelationalPathIndexQ2}
\rowcolors{3}{}{lightgray}
\begin{tabular}{| l | c  c | c  c | c  c |}
    \hline
    \multirow{2}{*}{Name}	&	\multicolumn{2}{|c|}{RG\_CPU\textsubscript{rel}}	&	\multicolumn{2}{|c|}{RG\_CUSP\textsubscript{rel}}	&	\multicolumn{2}{|c|}{RG\_SPARSE\textsubscript{rel}}	 \\
    		& Time & Mem &  Time     & Mem & Time     & Mem  \\    
    \hline
    \hline
    atom-primitive          & 0.001 & 0.3  & 0.001 & 0.1 & 0.002 & 0.1   \\
biomedical-mesure-primitive & 0.002 & 0.1  & 0.022 & 2   & 0.009 & 0.1   \\
core                        & 0.001 & 0.3  & 0.006 & 0.1 & 0.004 & 0.1   \\
eclass\_514en               & 0.035 & 6.5  & 0.339 & 16  & 0.1   & 12    \\
enzyme                      & 0.006 & 3.9  & 0.076 & 0.6 & 0.01  & 0.1   \\
foaf                        & 0.001 & 0.1  & 0.004 & 0.1 & 0.002 & 0.1   \\
funding                     & 0.002 & 0.1  & 0.015 & 0.4 & 0.007 & 0.1   \\
generations                 & 0.001 & 0.1  & 0.001 & 0.1 & 0.001 & 0.1   \\
go-hierarchy                & 0.095 & 17.8 & 2.025 & 528 & 0.175 & 130.4 \\
go                          & 0.306 & 25.8 & 0.633 & 84  & 0.181 & 25.4  \\
pathways                    & 0.005 & 0.2  & 0.016 & 0.4 & 0.004 & 0.1   \\
people\_pets                & 0.001 & 0.1  & 0.007 & 0.1 & 0.004 & 0.1   \\
pizza                       & 0.002 & 0.3  & 0.012 & 0.2 & 0.008 & 0.1   \\
skos                        & 0.001 & 0.1  & 0.001 & 0.1 & 0.001 & 0.1   \\
travel                      & 0.001 & 0.1  & 0.007 & 0.1 & 0.005 & 0.1   \\
univ-bench                  & 0.001 & 0.1  & 0.007 & 0.1 & 0.005 & 0.1   \\
wine                        & 0.001 & 0.3  & 0.006 & 0.1 & 0.004 & 0.1  \\
    \hline
  \end{tabular}
\end{table*}
}

{\setlength{\tabcolsep}{0.4em}
\begin{table*}[h]
\caption{RDFs single path semantics query $G_1$}
\label{tbl:tableRDFSinglePathIndexQ1}
\rowcolors{3}{}{lightgray}
\begin{tabular}{| l | c  c | c  c |}
    \hline
    \multirow{2}{*}{Name}	&	\multicolumn{2}{|c|}{RG\_CPU\textsubscript{path}}	&	\multicolumn{2}{|c|}{RG\_SPARSE\textsubscript{path}}	 \\
    		& Time & Mem &  Time     & Mem   \\    
    \hline
    \hline
    atom-primitive          & 0.033 & 1.5  & 0.008 & 0.1   \\
biomedical-mesure-primitive & 0.027 & 1.3  & 0.007 & 0.1   \\
core                        & 0.002 & 0.3  & 0.016 & 0.1   \\
eclass\_514en               & 0.195 & 31.2 & 0.496 & 26    \\
enzyme                      & 0.029 & 8.1  & 0.043 & 6     \\
foaf                        & 0.024 & 0.4  & 0.009 & 0.1   \\
funding                     & 0.057 & 0.5  & 0.009 & 0.1   \\
generations                 & 0.013 & 0.3  & 0.005 & 0.1   \\
go-hierarchy                & 0.976 & 92   & 0.336 & 125   \\
go                          & 1.286 & 75.7 & 0.739 & 45.4  \\
pathways                    & 0.021 & 0.5  & 0.021 & 2     \\
people\_pets                & 0.031 & 0.6  & 0.011 & 0.1   \\
pizza                       & 0.075 & 5.5  & 0.009 & 0.1   \\
skos                        & 0.005 & 0.3  & 0.006 & 0.1   \\
travel                      & 0.008 & 0.3  & 0.01  & 0.1   \\
univ-bench                  & 0.013 & 0.3  & 0.007 & 0.1   \\
wine                        & 0.117 & 7.1  & 0.015 & 0.2   \\
    \hline
  \end{tabular}
\end{table*}
}

{\setlength{\tabcolsep}{0.4em}
\begin{table*}[h]
\caption{RDFs single path semantics query $G_2$}
\label{tbl:tableRDFSinglePathIndexQ2}
\rowcolors{3}{}{lightgray}
\begin{tabular}{| l | c  c | c  c |}
    \hline
    \multirow{2}{*}{Name}	&	\multicolumn{2}{|c|}{RG\_CPU\textsubscript{path}}	&	\multicolumn{2}{|c|}{RG\_SPARSE\textsubscript{path}}	 \\
    		& Time & Mem &  Time     & Mem   \\    
    \hline
    \hline
    atom-primitive          & 0.001 & 0.3  & 0.002 & 0.1   \\
biomedical-mesure-primitive & 0.006 & 0.1  & 0.012 & 0.1   \\
core                        & 0.003 & 0.3  & 0.005 & 0.1   \\
eclass\_514en               & 0.123 & 17.7 & 0.127 & 18    \\
enzyme                      & 0.012 & 5.3  & 0.008 & 0.4   \\
foaf                        & 0.001 & 0.1  & 0.003 & 0.1   \\
funding                     & 0.009 & 0.1  & 0.008 & 0.1   \\
generations                 & 0.001 & 0.1  & 0.001 & 0.1   \\
go-hierarchy                & 0.884 & 88.8 & 0.306 & 138.8 \\
go                          & 0.918 & 78.1 & 0.219 & 34.2  \\
pathways                    & 0.017 & 0.5  & 0.003 & 0.1   \\
people\_pets                & 0.001 & 0.1  & 0.005 & 0.1   \\
pizza                       & 0.01  & 0.3  & 0.009 & 0.1   \\
skos                        & 0.001 & 0.1  & 0.002 & 0.1   \\
travel                      & 0.001 & 0.1  & 0.005 & 0.1   \\
univ-bench                  & 0.001 & 0.1  & 0.005 & 0.1   \\
wine                        & 0.002 & 0.3  & 0.004 & 0.1   \\
    \hline
  \end{tabular}
\end{table*}
}

{\setlength{\tabcolsep}{0.4em}
\begin{table*}[h]
\caption{Geospecies querying results}
\label{tbl:tableGeospeciesResults}
\rowcolors{3}{}{lightgray}
\begin{tabular}{| c  c | c  c | c  c | c  c |}
    \hline
    
    \multicolumn{4}{|c|}{Relational semantics index}	&	\multicolumn{4}{|c|}{Single path semantics index} \\    
    
    \hline
    
    
    \multicolumn{2}{|c|}{RG\_CPU\textsubscript{rel}}	&	\multicolumn{2}{|c|}{RG\_SPARSE\textsubscript{rel}} & \multicolumn{2}{|c|}{RG\_CPU\textsubscript{path}}	&	\multicolumn{2}{|c|}{RG\_SPARSE\textsubscript{path}}	 \\
     Time & Mem & Time & Mem & Time & Mem & Time & Mem \\    
    \hline
    \hline
    7.146 & 16934.2 & 0.856 & 5274 & 15.134 & 35803.6 & 1.935 & 5282   \\
    \hline
  \end{tabular}
\end{table*}
}

{\setlength{\tabcolsep}{0.4em}
\begin{table*}[h]
\caption{RDFs properties}
\label{tbl:propRDF}
\rowcolors{3}{}{lightgray}
\begin{tabular}{| l | c | c | c | c |}
    \hline
    Name                  & \#V    & \#E     & \#type &\#subClassOf \\
    \hline
    \hline
    atom-primitive				& 291		& 685		& 138	& 122	\\
    univ-bench					& 179		& 413		& 84		& 36		\\
    travel						& 131		& 397		& 90		& 30		\\
    skos							& 144		& 323		& 70		& 1		\\
    people\_pets					& 337		& 834		& 161	& 33		\\
    generations					& 129		& 351		& 78		& 0		\\
    foaf							& 256		& 815		& 174	& 10		\\
    biomedical-mesure-primitive	& 341		& 711		& 130	& 122	\\
    funding						& 778		& 1480		& 304	& 90               \\
    pizza						& 671		& 2604		& 365	& 259              \\
    wine							& 733		& 2450		& 485	& 126              \\
    core							& 1323		& 8684		& 1412	& 178              \\
    pathways						& 6238		& 37196		& 3118 	& 3117             \\
    go-hierarchy					& 45007		& 1960436	& 0		& 490109           \\
    enzyme						& 48815		& 219390		& 14989	& 8163             \\
    eclass\_514en				& 239111		& 1047454	& 72517	& 90962            \\
    go							& 272770		& 1068622	& 58483	& 90512            \\
    \hline
  \end{tabular}
\end{table*}
}


{\setlength{\tabcolsep}{0.4em}
\begin{table*}[h]
\caption{Geospecies properties}
\label{tbl:propGeo}
\rowcolors{3}{}{lightgray}
\begin{tabular}{| c | c | c |}
    \hline
    \#V    & \#E     & \#broaderTransitive \\
    \hline
    \hline
	450609 & 2311461 & 20867 \\
    \hline
  \end{tabular}
\end{table*}
}

