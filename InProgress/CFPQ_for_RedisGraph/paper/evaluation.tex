\section{Evaluation and Discussion}

We evaluate all the described implementations on all the datasets and the queries presented.
We compare our implementations with~\cite{Mishin:2019:ECP:3327964.3328503} and~\cite{Kuijpers:2019:ESC:3335783.3335791}.
We measure full time of query execution, so all overhead on data preparing is incuded.
Thus we can estimate applicability of matrix-based algrithm for real-world solutions.

For evaluation, we use a PC with Ubuntu 18.04 installed.
It has Intel core i7-6700 CPU, 3.4GHz, DDR4 32Gb RAM, and Geforce GTX 1070 GPGPU with 8Gb RAM.

The results of the evaluation are summarized in the tables below.
We provide results only for part of the collected data set because of the page limit.
Time is measured in seconds, memory is measured in megabites unless specified otherwise.
Note that for all implementations except our own we provide results form related paper.
The cell is left blank if the time limit is exceeded, or if there is not enough memory to allocate the data. %, or information is not available.

{\setlength{\tabcolsep}{0.4em}
\begin{table*}
\caption{RDFs querying results}
\label{tbl:tableRDF}
\rowcolors{3}{}{lightgray}
\begin{tabular}{| p{1.5cm} | c | c |c | c | c | c | c | c | c | c |}
    \hline
    \multicolumn{5}{|c|}{RDF}        & \multicolumn{3}{|c|}{Query $G_1$}                               & \multicolumn{3}{|c|}{Query $G_2$} \\
    \hline
    Name                  & \#V    & \#E     & \#type &\#subClassOf & RG\_CPU & RG\_M4RI  & RG\_CUSP   & RG\_CPU & RG\_M4RI  & RG\_CUSP \\
    \hline
    \hline
    \small{funding}       & 778    & 1480    & 304   & 90           & 0.01      & <0.01   & 0.02       & < 0.01  & < 0.01    & < 0.01    \\
    \small{pizza}         & 671    & 2604    & 365   & 259          & 0.01      & <0.01   & 0.02       & < 0.01  & < 0.01    & < 0.01    \\
    \small{wine}          & 733    & 2450    & 485   & 126          & 0.01      & <0.01   & 0.02       & < 0.01  & < 0.01    & < 0.01    \\
    \small{core}          & 1323   & 8684    & 1412  & 178          & < 0.01    & 0.12    & 0.02       & < 0.01  & < 0.01    & < 0.01    \\
    \small{pathways}      & 6238   & 37196   & 3118  & 3117         & 0.01      & 0.18    & 0.03       & < 0.01  & 0.06      & < 0.01    \\
    \small{go-hierarchy}  & 45007  & 1960436 & 0     & 490109       & 0.09      & -       & 1.50       & < 0.01  & -         & 0.55      \\
    \small{enzyme}        & 48815  & 219390  & 14989 & 8163         & 0.02      & 61.23   & 0.10       & < 0.01  & 6.97      & 0.02      \\
    \small{eclass\_514en} & 239111 & 1047454 & 72517 & 90962        & 0.06      & -       & 0.39       & 0.01    & -         & 0.10      \\
    \small{go}            & 272770 & 1068622 & 58483 & 90512        & 0.49      & -       & 0.83       & 0.01    & -         & 0.11      \\
    \hline
  \end{tabular}
\end{table*}
}


The results of the first dataset \textbf{[RDF]} are presented in table~\ref{tbl:tableRDF}.
We can see, that in this case the running time of all our implementations is smaller than of the reference implementation, and all implementations but \textbf{[CuSprs]} demonstrate similar performance.
It is obvious that performance improvement in comparison with the first implementation is huge and it is necessary to extend the dataset with new RDFs of the significantly bigger size.

Geospecies dataset currently can be processed only by using CPU version.
So, we can compare our matrix-based CPU implentation with the result form~\cite{Kuijpers:2019:ESC:3335783.3335791} for \textit{AnnGram$_{\textit{rel}}$} algorithm\footnote{Only \textit{AnnGram} works correctly and fits to limits, other implementations are faster, but return incorrect result, or do not fit to memory limits.}.
Fortunately both algorithms calulate queryes under relational semantics.
Result is provided in the table~\ref{tbl:geo}.

{\setlength{\tabcolsep}{0.4em}
\begin{table}[H]
\caption{Evaluation results geospecies data}
\label{tbl:geo}
\begin{tabular}{| c | c | c | c | }
    \hline
     \multicolumn{2}{|c|}{RG\_CPU}     & \multicolumn{2}{|c|}{Neo4j\_AnnGram$_{rel}$}          \\
     \hline
     Time  & Memory (Gb)  & Time  & Memory (Gb)   \\
    \hline
    \hline
    6.8   & 6.83    & 6 953.9  & 29.17   \\
    \hline
\end{tabular}
\end{table}
}

We can see, that the matrix-based algorithm implemented for RedisGraph is more than 1000 times faster than based on annotated grammar implemented for Neo4j and use more than 4 times less memory.
Thus we can conclude that the matrix-based algorithm is better than other CFPQ algorithms for query evaluation under a relational semantics for real-world data processing.
CFPQ evaluation under other semantics (single path, all paths, etc) by using a matrix-based algorithm is a direction for future research.

The next is the \textbf{[FreeScale]} datatset.
We compare our implementations with two implwmwntations form~\cite{Kuijpers:2019:ESC:3335783.3335791} which evaluate queryes under relatonal semantics: \textit{Neo4j\_AnnGram${_\textit{rel}}$} and \textit{Neo4j\_Matrix}. Results are presented in table~\ref{tbl:tableFreeScale}.
The evaluation shows that sparsity of graphs (value of parameter \texttt{p}) is important both for implementations which use sparse matrices and for implementations which use dense matrices.
Note that results for implemenattions for Neo4j are restored from graphics provided in~\cite{Kuijpers:2019:ESC:3335783.3335791}.
So, values are not precize, but it is possible to compare implementations.

Evaluation shouws that our CPU version is comparable with \textit{Neo4j\_AnnGram$_{\textit{rel}}$} and for relatively dense graphs (each vertex has 10 connections) our implementation is faster. Moreover, while \textit{Neo4j\_Matrix} is out of limits on biggest graph, our implementattion works fine.
So, it s important ot use an appropriate libraries for matrix-based algorithm implementation.
Also we can see, that GPGPU version which utilizes sparse matrices is significantly faster then other implementations.
Note, that for GPGPU versions we include time requred for data transferring and formans convertion.

\begin{table*}
\caption{Free scale graphs querying results}
\label{tbl:tableFreeScale}
\rowcolors{3}{}{lightgray}
\begin{tabular}{| l | c  c | c  c | c  c | c  c | c  c |}
    \hline
    \multirow{2}{*}{Graph} & \multicolumn{2}{|c|}{RG\_CPU} & \multicolumn{2}{|c|}{RG\_m4ri} & \multicolumn{2}{|c|}{RG\_CUSP} &\multicolumn{2}{|c|}{Neo4j\_AnnGram$_{\text{rel}}$} &\multicolumn{2}{|c|}{Neo4j\_Matrix} \\
    \cline{2-11}
                  & Time     & Mem             & Time   & Mem          & Time     & Mem         & Time    & Mem         & Time    & Mem\\
    \hline
    \hline
    G(100,1)      & < 0.01  & < 0.01           & < 0.01  & 0.10        & 0.01   & 2.00          & < 0.02  & 0.08      & 0.20     & 0.03  \\
    G(100,3)      & < 0.01  & < 0.01           & < 0.01  & 0.10        & 0.04   & 2.00          & 0.02    & 0.15      & 0.40     & 0.03  \\
    G(100,5)      & < 0.01  & < 0.01           & < 0.01  & 0.10        & 0.05   & 2.00          & 0.03    & 0.21      & 0.40     & 0.03  \\
    G(100,10)     & < 0.01  & < 0.01           & 0.01    & 0.10        & 0.07   & 2.00          & 0.09    & 0.60      & 0.60     & 0.03  \\
    \hline
    G(500,1)      & < 0.01  & < 0.01           & < 0.01  & 2.00        & 0.01   & 2.00          & < 0.02  & 0.20      & 20.00    & 0.60   \\
    G(500,3)      & < 0.01  & < 0.01           & < 0.01  & 2.00        & 0.07   & 2.00          & 0.03    & 0.50      & 40.00    & 0.60   \\
    G(500,5)      & < 0.01  & 0.17             & < 0.01  & 2.00        & 0.10   & 2.00          & 0.10    & 1.10      & 50.00    & 0.60   \\
    G(500,10)     & 1.24    & 0.78             & 0.01    & 2.00        & 0.11  & 4.00           & 0.50    & 4.00      & 55.00    & 0.60   \\
    \hline
    G(2500,1)     & < 0.01  & 0.11             & 0.07    & 30.00       & 0.03   & 2.00          & 0.03    & 0.70       & 0.023   & 14.00  \\
    G(2500,3)     & 0.01    & 0.11             & 0.11    & 30.00       & 0.10   & 2.00          & 0.15    & 2.50       & 0.105   & 14.00  \\
    G(2500,5)     & 2.06    & 0.11             & 0.11    & 30.00       & 0.12   & 4.00          & 0.70    & 8.00       & 1.636   & 14.00  \\
    G(2500,10)    & 3.25    & 3.77             & 0.13    & 30.00       & 0.31  & 31.20          & 5.00    & 20.00      & 13.071  & 14.00  \\
    \hline
    \Cline{6-7}
    G(10000,1)    & < 0.01  & 0.47             & 1.55    & 200.00      & \Thickvrulel{0.04} & \Thickvruler{2.0}           & 0.10    & 2.50       & -       & -  \\
    G(10000,3)    & 5.439   & 1.15             & 3.60    & 200.00      & \Thickvrulel{0.20} & \Thickvruler{3.20}          & 0.40    & 10.00      & -       & -  \\
    G(10000,5)    & 7.978   & 2.64             & 3.32    & 200.00      & \Thickvrulel{0.25} & \Thickvruler{13.20}         & 3.00    & 35.00      & -       & -  \\
    G(10000,10)   & 13.180  & 21.08            & 3.60    & 200.00      & \Thickvrulel{1.23} & \Thickvruler{198.00}        & 40.00   & 240.00     & -       & -  \\
    \hline
    \Cline{6-7}
  \end{tabular}
\end{table*}


Finally, we can conclude that GPGPU utilization for CFPQ can significantly improve performance, but more research on advanced optimization techniques should be done.
On the other hand, the high-level implementation (\textbf{[GPU\_Py]}) is comparable with other GPGPU-based implementations.
So, it may be a balance between implementation complexity and performance.
Highly optimized existing libraries can be of some use: the implementation based on m4ri is faster than the reference implementation and the other CPU-based implementation.
Moreover, it is comparable with some GPGPU-based implementations in some cases.
Sparse matrices utilization demands more thorough investigation.
The main question is if we can create an efficient implementation for sparse boolean matrices multiplication.
