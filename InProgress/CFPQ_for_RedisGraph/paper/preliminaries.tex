\section{Preliminaries}
\label{section_preliminaries}

Let $\Sigma$ be a finite set of edge labels. Define an \emph{edge-labeled directed graph} as a tuple \mbox{$D = (V, E)$} with a set of nodes $V$ and a directed edge relation \mbox{$E \subseteq V \times \Sigma \times V$}.  For a path $\pi$ in a graph $D$, we denote the unique word, obtained by concatenating the labels of the edges along the path $\pi$ as \mbox{$l(\pi)$}. Also, we write \mbox{$n \pi m$} to indicate, that the path $\pi$ starts at the node \mbox{$n \in V$} and ends at the node \mbox{$m \in V$}.

Following Hellings~\cite{hellingsRelational}, we deviate from the usual definition of a context-free grammar in \emph{Chomsky Normal Form}~\cite{chomsky} by not including a special starting non-terminal, which will be specified in the path queries for the graph. Since every context-free grammar can be transformed into an equivalent one in Chomsky Normal Form and checking, that empty string belongs to the language is trivial, it is sufficient to consider only grammars of the following type. A \emph{context-free grammar} is a triple \mbox{$G = (N, \Sigma, P)$}, where $N$ is a finite set of non-terminals, $\Sigma$ is a finite set of terminals, and $P$ is a finite set of productions of the following forms:

\begin{itemize}
    \item $A \rightarrow B C$, for $A,B,C \in N$,
    \item $A \rightarrow x$, for $A \in N$ and $x \in \Sigma$.   
\end{itemize}

Note that we omit the rules of the form \mbox{$A \rightarrow \varepsilon$}, where $\varepsilon$ denotes empty string. This does not restrict the applicability of our algorithm since only the empty paths \mbox{$m \pi m$} correspond to empty string $\varepsilon$.

We use the conventional notation \mbox{$A \xrightarrow{*} w$} to denote, that a string \mbox{$w \in \Sigma^*$} can be derived from a non-terminal $A$ by some sequence of production rule applications from $P$. The \emph{language} of a grammar \mbox{$G = (N,\Sigma,P)$} with respect to a start non-terminal \mbox{$S \in N$} is defined by $$L(G_S) = \{w \in \Sigma^*~|~S \xrightarrow{*} w\}.$$

For a given graph \mbox{$D = (V, E)$} and a context-free grammar $G = (N, \Sigma, P)$, we define \emph{context-free relations} \mbox{$R_A \subseteq V \times V$} for every \mbox{$A \in N$}, such that $$R_A = \{(n,m)~|~\exists n \pi m~(l(\pi) \in L(G_A))\}.$$

For the context-free path query evaluation w.r.t. the relational query semantics we use a binary operation $(~\cdot~)$ defined in~\cite{Azimov:2018:CPQ:3210259.3210264} for arbitrary subsets \mbox{$N_1, N_2$} of $N$ with respect to a context-free grammar \mbox{$G = (N, \Sigma, P)$} as $$N_1 \cdot N_2 = \{A~|~\exists B \in N_1, \exists C \in N_2 \text{ such that }(A \rightarrow B C) \in P\}.$$

Using this binary operation as subset multiplication, and union as an addition, we can define a \emph{matrix multiplication}, \mbox{$a \times b = c$}, where $a$ and $b$ are matrices of a suitable size, that have subsets of $N$ as elements, as $$c_{i,j} = \bigcup^{n}_{k=1}{a_{i,k} \cdot b_{k,j}}.$$

Also, we use the element-wise union operation on matrices $a$ and $b$ with the same size: \mbox{$a \cup b = c$}, where $c_{i,j} = a_{i,j} \cup b_{i,j}.$

According to Azimov~\cite{Azimov:2018:CPQ:3210259.3210264}, we define the \emph{transitive closure} of a square matrix $a$ as \mbox{$a^{cf} = a^{(1)} \cup a^{(2)} \cup \cdots$}, where \mbox{$a^{(1)} = a$} and $$a^{(i)} = a^{(i-1)} \cup (a^{(i-1)} \times a^{(i-1)}), ~i \ge 2.$$

For the context-free path query evaluation w.r.t. the single-path query semantics, we must provide such a path for each node pair if it exists. In order to do this, we introduce the $$PathIndex = (left,right,middle,height,length)$$ --- the elements of matrices which describe the found paths as concatenations of two smaller paths and help to restore each path and derivation tree for it at the end of evaluation. Here $left$ and $right$ stand for the indexes of starting and ending node in the founded path, $middle$ --- the index of intermediate node used in the concatenation of two smaller paths, $height$ --- the height of the derivation tree of the string corresponding to the founded path, and $length$ is a length of founded path. When we dont find the path for some node pair $i,j$, we use the $PathIndex = \bot = (0,0,0,0,0)$.

Also, we will use the notation $proper matrix$ which means that for every element of the matrix with indexes $i,j$ we either $PathIndex = (i,j,\_,\_,\_)$ or $\bot$.

For proper matrices we use a binary operation $\otimes$ defined for PathIndexes \mbox{$PI_1, PI_2$} with $PI_1.right = PI_2.left$ as 

$PI_1 \otimes PI_2 = (PI_1.left, PI_2.right, PI_1.right, \\ max(PI_1.height, PI_2.height)+1,PI_1.length + PI_2.lenght).$

For proper matrices we also use a binary operation $\oplus$ defined for PathIndexes \mbox{$PI_1, PI_2$} with $PI_1.left = PI_2.left$ and $PI_1.right = PI_2.right$ as $PI_1$ if $PI_1.height \leq PI_2.height$ and $PI_2$ otherwise.

Using $\otimes$ as multiplication of PathIndexes, and $\oplus$ as an addition, we can define a \emph{matrix multiplication}, \mbox{$a \odot b = c$}, where $a$ and $b$ are matrices of a suitable size, that have PathIndexes as elements, as $$c_{i,j} = \bigoplus^{n}_{k=1}{a_{i,k} \otimes b_{k,j}}.$$

Also, we use the element-wise $\oplus$ operation on matrices $a$ and $b$ with the same size: \mbox{$a + b = c$}, where $c_{i,j} = a_{i,j} \oplus b_{i,j}.$
