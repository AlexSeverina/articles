\section{Introduction}

Formal language constrained path querying, or formal language constrained path problem~\cite{FLCpathProblem}, is a graph analysis problem when formal languages are used as a constraints for navigational path queryes.
Namely, in this approach concatenation of the labels along the path is treated as a word, and a specification of the language which should contain specific words is a constraint.
While regular path querying (RPQ) is actively used in many different graph query languages and graph analysis engines, more expressive one, contxt-free path querying (CFPQ)~\cite{Yannakakis} at the start stage and activly developed.
Context-free constraints allow one to express such important class of queryes as \textit{same generation quey}~\cite{FndDB} which can not be expressed in terms of regular constraints. 


%CFPQ is widely used for graph-structured data analysis in such domains as biological data analysis, RDF, network analysis.
%Huge amount of the real-world data makes performance of CFPQ critical for practical tasks.

Several algorithms for CFPQ based on such parsing techniques as (G)LL, (G)LR, and CYK are proposed recently~\cite{bradford2007quickest,ward2008distributed,bradford2016fast,hellingsPathQuerying,Grigorev:2017:CPQ:3166094.3166104,Verbitskaia:2018:PCC:3241653.3241655,RDF,10.1007/978-3-319-91662-0_17,Medeiros:2018:EEC:3167132.3167265}.
But recent research by Jochem Kuijpers et.al.~\cite{Kuijpers:2019:ESC:3335783.3335791} shows that existing solutions are not applicable for real-world graph analysis because running time and memory consumption are poor.
At the same time, Nikita Mishin et.al in~\cite{Mishin:2019:ECP:3327964.3328503} shows that matrix-based CFPQ algorithm can demonstrate good performance on real-world data.
Matrix-based algorithm is proposed by Rustam Azimov~\cite{Azimov:2018:CPQ:3210259.3210264} and offloads the most critical computations onto boolean matrices multiplication.
As a result, it is easy to implement, and allows one to utilize modern massive-parallel hardware for CFPQ.
But only algorithm performance without integration with graph storage is provided in the paper while J.~Kuijpers provides evaluation of integrated with Neo4j\footnote{!!!!} algorithms.
Also, in~\cite{Kuijpers:2019:ESC:3335783.3335791} matrix-based algorithm is evaluated in simple single-thread Java implementation, while N.~Mishin shows that the most efficient implementation should utilize high-performance matrix multioplication libraries which are highly parallel, or even utilize GPGPU.


%The implementation provided by the authors utilizes GPGPU by using cuSPARSE\footnote{cuSparse is a library for sparse matrices multiplication on GPGPU. Official documentation: \url{https://docs.nvidia.com/cuda/cusparse/index.html}. Access date: 12.11.2019} library which is a floating point sparse matrices multiplication library.
%Although it does not use advanced algorithms for boolean matrices, it outperforms existing algorithms.


In this work we show that CFPQ in relational semantics (according Hellings~\cite{hellingsRelational}) can be fast enough to be applicable to real-world graph analysis.
We use RedisGraph\footnote{RedisGraph is a graph data base which is based on Property Graph Model. Project web page: \url{https://oss.redislabs.com/redisgraph/}. Access data: 12.11.2019.}~\cite{8778293} garph data base as a storage.
This data base use adjacency matrices as arepresenataion of graph and GraphBLAS~\cite{7761646} for matrices manipualtion. 
These facts allow us to integrate matrix-based CFPQ algorithm to RedisGraph with minimal effort.
We make the following contributions in this paper.
\begin{enumerate}
\item We provide a number of implementations of the matrix multiplication based CFPQ algorithm which uses RedisGraph as a graph storage.
The first implementation is CPU-based and utilize SuteSparse~\cite{!!!} implemenation of GraphBLAS API for matrices manipulation.
The second implementation is GPGPU-based and include both existing implementation from~\cite{!!!} and our own CUSP\footnote{CUSP is an open source library for sparse matrix multiplication on GPGPU. \url{!!!}. Access date: 12.11.2019.}-based implenetation.
The source code is available on GitHub: \url{https://github.com/YaccConstructor/RedisGraph}.
\item We extend the dataset !!!
\item We provide evaluation which shows that !!!.
\end{enumerate}
