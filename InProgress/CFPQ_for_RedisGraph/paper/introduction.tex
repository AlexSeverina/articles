\section{Introduction}

Formal language constrained path querying, or formal language constrained path problem~\cite{FLCpathProblem}, is a graph analysis problem in which formal languages are used as constraints for navigational path queries.
In this approsch a path is viewed as a word constructed by concatenation of edge labels.
Paths of interest are constrained with some formal language: a query should find only paths labeled by words from the language.
The class of language constraints which is most widely spread is regular: it is used in various graph query languages and engines.
Context-free path queruing (CFPQ)~\cite{Yannakakis}, while being more expressive, is still at the early stage of development.
Context-free constraints allow one to express such important class of queries as \textit{same generation queies}~\cite{FndDB} which cannot be expressed in terms of regular constraints.

Several algorithms for CFPQ based on such parsing techniques as (G)LL, (G)LR, and CYK were proposed recently~\cite{bradford2007quickest,ward2008distributed,bradford2016fast,hellingsPathQuerying,Grigorev:2017:CPQ:3166094.3166104,Verbitskaia:2018:PCC:3241653.3241655,RDF,10.1007/978-3-319-91662-0_17,Medeiros:2018:EEC:3167132.3167265}.
Yet recent research by Jochem Kuijpers et.al.~\cite{Kuijpers:2019:ESC:3335783.3335791} shows that existing solutions are not applicable for real-world graph analysis because of significant running time and memory consumption.
At the same time, Nikita Mishin et.al show in~\cite{Mishin:2019:ECP:3327964.3328503} that the matrix-based CFPQ algorithm demonstrates good performance on real-world data.
A matrix-based algorithm proposed by Rustam Azimov~\cite{Azimov:2018:CPQ:3210259.3210264} offloads the most critical computations onto boolean matrices multiplication.
This algorithm is easy to implement and to employ modern massive-parallel hardware for CFPQ.
The paper measures the performance of the algorithm in isolation while J.~Kuijpers provides the evaluation of the algorithms which are integrated with Neo4j\footnote{Neo4j graph database web page: \url{https://neo4j.com/}. Access date: 12.11.2019.} graph database.
Also, in~\cite{Kuijpers:2019:ESC:3335783.3335791} the matrix-based algorithm is implemented as a simple single-thread Java program, while N.~Mishin shows that to achieve the best performance, one should utilize high-performance matrix multiplication libraries which are highly parallel or utilize GPGPU better.
Thus, it is required to evaluate a matrix-based algorithm which is integrated with a graph storage and makes use of performant libraries and hardware.

In this work we show that CFPQ in relational semantics (according to Hellings~\cite{hellingsRelational}) can be performant enough to be applicable to real-world graph analysis.
We use RedisGraph\footnote{RedisGraph is a graph database which is based on Property Graph Model. Project web page: \url{https://oss.redislabs.com/redisgraph/}. Access date: 12.11.2019.}~\cite{8778293} graph database as a storage.
This database uses adjacency matrices as a representation of a graph and GraphBLAS~\cite{7761646} for matrices manipulation.
These facts allow us to integrate matrix-based CFPQ algorithm with RedisGraph with minimal effort.
We make the following contributions in this paper.
\begin{enumerate}
\item We provide a number of implementations of the CFPQ algorithm which is based on matrix multiplication and uses RedisGraph as graph storage.
The first implementation is CPU-based and utilizes SuteSparse\footnote{SuteSparse is a sparse matrix software which incudes GraphBLAS API implementation. Project web page: \url{http://faculty.cse.tamu.edu/davis/suitesparse.html}. Access date: 12.11.2019.}~\cite{Davis2018Algorithm9S} implementation of GraphBLAS API for matrices manipulation.
The second implementation is GPGPU-based and includes both the existing implementation from~\cite{Mishin:2019:ECP:3327964.3328503} and our own CUSP\footnote{CUSP is an open source library for sparse matrix multiplication on GPGPU. Project site: \url{https://cusplibrary.github.io/}. Access date: 12.11.2019.}-based implementation.
The source code is available on GitHub\footnote{Sources of matrix-based CFPQ algorithm for RedisGraph database: \url{https://github.com/YaccConstructor/RedisGraph}. Access date: 12.11.2019.}.
\item We extend the dataset presented in~\cite{Mishin:2019:ECP:3327964.3328503} with new real-world and synthetic cases of CFPQ\footnote{The CFPQ\_Data data set fro CFPQ algorithms evaluation and comparison. GitHub page: \url{https://github.com/JetBrains-Research/CFPQ_Data}. Access date: 12.11.2019.}.
\item We provide evaluation which shows that matrix-based CFPQ implementation for RedisGraph database is performant enough for real-world data analysis.
\end{enumerate}
