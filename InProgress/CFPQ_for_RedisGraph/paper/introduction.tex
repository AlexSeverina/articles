\section{Introduction}

Formal language constrained path querying, or formal language constrained path problem~\cite{FLCpathProblem}, is a graph analysis problem in which formal languages are used as constraints for navigational path queries.
In this approach, a path is viewed as a word constructed by concatenation of edge labels.
Paths of interest are constrained with some formal language: a query should find only paths labeled by words from the language.
The class of language constraints which is most widely spread is regular: it is used in various graph query languages and engines.
Context-free path querying (CFPQ)~\cite{Yannakakis}, while being more expressive, is still at the early stage of development.
Context-free constraints allow one to express such important class of queries as \textit{same generation queies}~\cite{FndDB} which cannot be expressed in terms of regular constraints.

Several algorithms for CFPQ based on such parsing techniques as (G)LL, (G)LR, and CYK were proposed recently~\cite{bradford2007quickest,ward2008distributed,bradford2016fast,hellingsPathQuerying,Grigorev:2017:CPQ:3166094.3166104,Verbitskaia:2018:PCC:3241653.3241655,RDF,10.1007/978-3-319-91662-0_17,Medeiros:2018:EEC:3167132.3167265}.
Yet recent research by Jochem Kuijpers et.al.~\cite{Kuijpers:2019:ESC:3335783.3335791} shows that existing solutions are not applicable for real-world graph analysis because of significant running time and memory consumption.
At the same time, Nikita Mishin et.al show in~\cite{Mishin:2019:ECP:3327964.3328503} that the matrix-based CFPQ algorithm demonstrates good performance on real-world data.
A matrix-based algorithm proposed by Rustam Azimov~\cite{Azimov:2018:CPQ:3210259.3210264} offloads the most critical computations onto Boolean matrices multiplication.
This algorithm is easy to implement and to employ modern massive-parallel hardware for CFPQ.
The paper measures the performance of the algorithm in isolation while J.~Kuijpers provides the evaluation of the algorithms which are integrated with Neo4j\footnote{Neo4j graph database web page: \url{https://neo4j.com/}. Access date: 12.11.2019.} graph database.
Also, in~\cite{Kuijpers:2019:ESC:3335783.3335791} the matrix-based algorithm is implemented as a simple single-thread Java program, while N.~Mishin shows that to achieve the best performance, one should utilize high-performance matrix multiplication libraries which are highly parallel or utilize GPGPU better.
Thus, it is required to evaluate a matrix-based algorithm which is integrated with graph storage and makes use of performant libraries and hardware.

All discussed matrix-based algorithms correspond to the CFPQ with relational query semantics (according to Hellings~\cite{hellingsRelational}) and solve the reachability problem. However, in some areas, it is important to have a proof of existence of certain paths. This problem can be solved using CFPQ algorithms with single-path query semantics (according to Hellings~\cite{hellingsPathQuerying}), which provide some path for each node pair if one exists. There are many results on the CFPQ with single-path query semantics which use the shortest paths to return~\cite{hellingsPathQuerying,barrett2000formal,bradford2007quickest,ward2010complexity}. We provide the algorithm for CFPQ with single-path query semantics which, for performance reasons, returns a path corresponding to the string with derivation tree of minimal height. However, our algorithm can be easily modified to return the shortest paths. 

In this work, we show that CFPQ with relational and single-path query semantics can be performant enough to be applicable to real-world graph analysis.
We use RedisGraph\footnote{RedisGraph is a graph database that is based on the Property Graph Model. Project web page: \url{https://oss.redislabs.com/redisgraph/}. Access date: 12.11.2019.}~\cite{8778293} graph database as a storage.
This database uses adjacency matrices as a representation of a graph and GraphBLAS~\cite{7761646} for matrices manipulation.
These facts allow us to integrate a matrix-based CFPQ algorithm with RedisGraph with minimal effort.
We make the following contributions in this paper.
\begin{enumerate}
\item We provide the first matrix-based algorithm for CFPQ with single-path query semantics and prove the correctness of this algorithm.
\item We provide several implementations of the CFPQ algorithms for relational and single-path query semantics which are based on matrix multiplication and uses RedisGraph as graph storage.
The first implementation for relational query semantics is CPU-based and utilizes SuteSparse\footnote{SuteSparse is a sparse matrix software which incudes GraphBLAS API implementation. Project web page: \url{http://faculty.cse.tamu.edu/davis/suitesparse.html}. Access date: 12.11.2019.}~\cite{Davis2018Algorithm9S} implementation of GraphBLAS API for matrices manipulation.
Also, we provide GPGPU-based implementation for relational query semantics: the CUSP\footnote{CUSP is an open source library for sparse matrix multiplication on GPGPU. Project site: \url{https://cusplibrary.github.io/}. Access date: 12.11.2019.}-based implementation and the implementation based on the idea from the paper~\cite{NsparsePaper}. Finally, we provide the GPGPU-based implementation for the proposed CFPQ algorithm with single-path query semantics. 
The source code is available on GitHub\footnote{Sources of matrix-based CFPQ algorithm for the RedisGraph database: \url{https://github.com/YaccConstructor/RedisGraph}. Access date: 12.11.2019.}.
\item We extend the dataset presented in~\cite{Mishin:2019:ECP:3327964.3328503} with new real-world and synthetic cases of CFPQ\footnote{The CFPQ\_Data dataset fro CFPQ algorithms evaluation and comparison. GitHub page: \url{https://github.com/JetBrains-Research/CFPQ_Data}. Access date: 12.11.2019.}.
\item We provide evaluation which shows that matrix-based CFPQ implementation for the RedisGraph database is performant enough for real-world data analysis.
\end{enumerate}
