\section{Dataset description}

In our evaluation we use combined dataset which contains the following parts.
\begin{itemize}
\item CFPQ\_Data dataset which is provided in\footnote{CFPQ\_Data data set GitHub repository: \url{https://github.com/JetBrains-Research/CFPQ_Data}. Access date: 12.11.2019.}~\cite{Mishin:2019:ECP:3327964.3328503} and contains both synthetic and real-world graphs and queries.
Real-world data includes RDFs, synthetic cases include theoretical worst-case and random graphs.
\item Dataset which is provided in~\cite{Kuijpers:2019:ESC:3335783.3335791}. Both Geospecies --- RDF which contains information about biological hierrarchy\footnote{\url{https://old.datahub.io/dataset/geospecies}. Access date: 12.11.2019.} and same generation query over \textit{broaderTransitive} relation, and Synthetic --- the set of graphs generated by using the Barab\'asi-Albert model~\cite{Albert_statisticalmechanics} of scale-free networks and same generation quey, are integrated to CFPQ\_Data and used it in our evaluation.
\item In~\cite{Mishin:2019:ECP:3327964.3328503} was shown that matrix-based algorithm is performant enough to handle bigger RDFs than those used in initial data sets, such as~\cite{RDF}.
So, we add a number of big RDFs to CFPQ\_Data and use them in our evaluation.
New RDFs: \textit{go-hierarchy, go, enzime, core, pathways} are from UniProt database\footnote{Protein sequences data base: \url{https://www.uniprot.org/}. RDFs with data are avalable here: \url{ftp://ftp.uniprot.org/pub/databases/uniprot/current_release/rdf}. Access date: 12.11.2019}, and \textit{eclass-514en} is from eClassOWL project\footnote{eClassOWL project: \url{http://www.heppnetz.de/projects/eclassowl/}. eclass-514en file is available here: \url{http://www.ebusiness-unibw.org/ontologies/eclass/5.1.4/eclass_514en.owl}. Access date: 12.11.2019.}.
\end{itemize}

The variants of the \textit{same generation query}~\cite{FndDB} are used in almost all cases because it is an important example of real-world queries that are context-free but not regular.
So, variations of the same generation query are used in our evaluation.
All queries are added to the CFPQ\_Data data set.

For RDFs (\textbf{[RDF]} dataset) we use two queries over \textit{subClassOf} and \textit{type} relations.
The first query is the grammar $G_1$:
\[
 \begin{array}{lcl}
   s  \rightarrow \textit{subClassOf}^{\ -1} \ s \ \textit{subClassOf}   & \quad & s  \rightarrow \textit{type}^{\ -1} \ s \ \textit{type}     \\
   s  \rightarrow \textit{subClassOf}^{\ -1} \ \textit{subClassOf}       & \quad & s  \rightarrow  \textit{type}^{\ -1}  \ \textit{type}

 \end{array}
 \]
The second one is the grammar $G_2$: $$s \rightarrow \textit{subClassOf}^{\ -1} \ s \ \textit{subClassOf} \mid  \textit{subClassOf}.$$

For geospecies and free scale graphs querying we use same-generation queries form the original paper.
