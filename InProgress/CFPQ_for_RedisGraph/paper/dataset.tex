\section{Dataset description}

We created and published a dataset for CFPQ algorithms evaluation.
This dataset contains both the real-world data and synthetic data for different specific cases, such as the theoretical worst case, or the worst cases specific to matrices representations.

Our goal is to evaluate querying algorithms, not graph storages or graph databases, so all data is presented in a text-based format to simplify usage in different environments.
Grammars are in Chomsky Normal Form, and graphs are represented as a list of triples (edges).
Some details of the data representation can be found in the Appendix.

The variants of the \textit{same generation query}~\cite{FndDB} are an important example of queries that are context-free but not regular, so we use this type of queries in our evaluation.
The dataset includes data for the following cases.
Each case is a pair of a set of graphs and a set of grammars: each query (grammar) should be applied to each graph.

\textbf{[RDF]} The set of the real-world RDF files (ontologies) from~\cite{RDF} and two variants of the same generation query which describes hierarchy analysis.
The first query is the grammar $G_4$:
\[
 \begin{array}{lcl}
   s  \rightarrow \textit{SCOR} \ s \ \textit{SCO}   & \quad & s  \rightarrow \textit{TR} \ s \ \textit{T}     \\
   s  \rightarrow \textit{SCOR} \ \textit{SCO}       & \quad & s  \rightarrow  \textit{TR}  \ \textit{T}

 \end{array}
 \]
The second one is the grammar $G_5$: $s \rightarrow \textit{SCOR} \ s \ \textit{SCO} \ | \  \textit{SCO}$.

\textbf{[Worst]} The theoretical worst case for CFPQ time complexity proposed by Hellings~\cite{hellingsPathQuerying}: the graph is two cycles of coprime lengths with a single common vertex.
The first cycle is labeled by the open bracket and the second cycle is labeled by the close bracket.
Query is a grammar for the $A^nB^n$ language.
The example of such graph and grammar is presented in figure~\ref{fig:grammar_example}.

\begin{figure}[h]
    \centering
    \begin{subfigure}[b]{0.20\textwidth}
        \centering
        \[
         \begin{array}{l}
           s \rightarrow A \ s \ B \\
           s \rightarrow A \ B
         \end{array}
         \]
        \caption{Grammar $G_1$ for $\{A^n B^n\}$}
    \end{subfigure}%
    ~\quad
    \begin{subfigure}[b]{0.24\textwidth}
        \centering
        \begin{tikzpicture}[shorten >=1pt,node distance=2cm,on grid,auto]
   \node[state] (q_1)   {$1$};
   \node[state] (q_2) [above=of q_1] {$2$};
   \node[state] (q_3) [above right=of q_1, below right=of q_2] {$0$};
   \node[state] (q_4) [right=of q_3] {$3$};
    \path[->]
    (q_1) edge  node {A} (q_2)
    (q_2) edge  node {A} (q_3)
    (q_3) edge  node {A} (q_1)
    (q_3) edge[bend left, above]  node {B} (q_4)
    (q_4) edge[bend left, below]  node {B} (q_3);
\end{tikzpicture}

        \caption{Input graph $D_1$ for the worst case with \texttt{A} as the open bracket, and \texttt{B} as the close bracket}
        \label{fig:worstCaseGraph}
    \end{subfigure}%
    \caption{Graph and grammar for the worst case}
    \label{fig:grammar_example}
\end{figure}


\textbf{[Full]} The case when the input graph is sparse, but the result is a full graph.
Such a case may be hard for sparse matrices representations.
As an input graph, we use a cycle, all edges of which are labeled by the same token.
As a query we use two grammars which describe the sequence of tokens of arbitrary length: the simple ambiguous grammar $G_2$: $s \rightarrow  s \ s \ | \ A$,  and the highly ambiguous grammar $G_3$: $s \rightarrow s \ s \ s \ | \ A$.

\textbf{[Sparse]} Sparse graphs from~\cite{fan2018scaling} are generated by the GTgraph graph generator, and emulate realistic sparse data.
Names of these graphs have the form \texttt{Gn-p}, where \texttt{n} corresponds to the total number of vertices, and \texttt{p} is the probability that some pair of vertices is connected.
The query is the same generation query represented by the grammar $G_1$ (figure~\ref{fig:grammar_example}).
