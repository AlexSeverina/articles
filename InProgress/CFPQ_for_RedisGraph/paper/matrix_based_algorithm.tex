\section{Matrix-based Algorithm for CFPQ}

The matrix-based algorithm for CFPQ was proposed by Rustam Azimov~\cite{Azimov:2018:CPQ:3210259.3210264}.
This algorithm can be expressed in terms of operations over boolean matrices (see listing~\ref{lst:algo1}) which is an advantage for implementation.

\begin{algorithm}
  \floatname{algorithm}{Listing}
\begin{algorithmic}[1]
\caption{Context-free path quering algorithm}
\label{lst:algo1}
\Function{contextFreePathQuerying}{D, G}

    \State{$n \gets$ the number of nodes in $D$}
    \State{$E \gets$ the directed edge-relation from $D$}
    \State{$P \gets$ the set of production rules in $G$}
    \State{$N \gets$ the set of nonterminals in $G$}
    \State{$T \gets \{T^{A_i} \mid A_i \in N, T^{A_i}$ is a matrix $n \times n$ in which each element is \texttt{false}\}}
    \ForAll{$(i,x,j) \in E$}
        \Comment{Matrix initialization}
        \For{$A_k \mid A_k \to x \in P$}
          {$T^{A_k}_{i,j} \gets \texttt{true}$}
        \EndFor
    \EndFor
    \For{$A_k \mid A_k \to \varepsilon \in P$}
       {$T^{A_k}_{i,i} \gets \texttt{true}$}
    \EndFor

    \While{any matrix in $T$ is changing}
        \Comment{Transitive closure calculation}
        \For{$A_i \to A_j A_k \in P$}
          {$T^{A_i} \gets T^{A_i} + (T^{A_j} \times T^{A_k})$}
        \EndFor
    \EndWhile
\State \Return $T$
\EndFunction
\end{algorithmic}
\end{algorithm}

Here $D = (V, E)$ is the input graph and $G = (N,\Sigma,P)$ is the input grammar.
For each matrix $T^{A_k}$, $T^{A_k}[i,j] = \texttt{true} \iff \exists \pi = v_i \ldots v_j $---path in $D$, such that $A_k \xRightarrow[G]{*} \omega(\pi) $, where $\omega(\pi)$ is a word formed by the labels along the path $\pi$.
Thus, this algorithm solves the reachability problem, or, according to Hellings~\cite{hellingsRelational}, implements relational query semantics.

The performance-critical part of the algorithm is boolean matrix multiplication.
Note, that if the matrices $T_{N_j}$ and $T_{N_k}$ have not been changed at the previous iteration, then we can skip update operation.
Such optimization can improve performance.
Also, it is important for applications that real-world data is often sparse, so it should be a better solution to use libraries which manipulate with sparse matrices.
