\section{Matrix-based Algorithm for CFPQ}

Matrix-based algorithm for CFPQ was proposed by Rustam Azimov~\cite{Azimov:2018:CPQ:3210259.3210264}.
This algorithm can be expressed in terms of operations over matrices boolean (see listing~\ref{lst:algo1}), and it is a sufficient advantage for implementation.
%It was shown that the utilization of GPGPU improves the context-free path querying performance significantly in comparison to other implementations~\cite{Azimov:2018:CPQ:3210259.3210264} even if float matrices are used instead of boolean matrices.

\begin{algorithm}
  \floatname{algorithm}{Listing}
\begin{algorithmic}[1]
\caption{Context-free path quering algorithm}
\label{lst:algo1}
\Function{contextFreePathQuerying}{D, G}

    \State{$n \gets$ the number of nodes in $D$}
    \State{$E \gets$ the directed edge-relation from $D$}
    \State{$P \gets$ the set of production rules in $G$}
    \State{$N \gets$ the set of nonterminals in $G$}
    \State{$T \gets \{T^{A_i} \mid A_i \in N, T^{A_i}$ is a matrix $n \times n$ in which each element is \texttt{false}\}}
    \ForAll{$(i,x,j) \in E$}
        \Comment{Matrix initialization}
        \For{$A_k \mid A_k \to x \in P$}
          \State{$T^{A_k}_{i,j} \gets \texttt{true}$}
        \EndFor
    \EndFor
    \For{$A_i \mid A_i \to \varepsilon \in P$}
       \State{$T^{A_i}_{i,i} \gets \texttt{true}$}
    \EndFor

    \While{any matrix in $T$ is changing}
        \Comment{Transitive closure calculation}
        \For{$A_i \to A_j A_k \in P$}
          \State{$T^{A_i} \gets T^{A_i} + (T^{A_j} \times T^{A_k})$}
        \EndFor
    \EndWhile
\State \Return $T$
\EndFunction
\end{algorithmic}
\end{algorithm}

Here $D = (V, E)$ is the input graph and $G = (N,\Sigma,P)$ is the input grammar.
For each matrix $T^{A_i}$, $T^{A_i}[i,j] = \texttt{true} \iff \exists \pi = v_i \ldots v_j $---path in $D$, such that $A_i \xRightarrow[G]{*} \omega(\pi) $, where $\omega(\pi)$ is a word formed by the labels along the path $\pi$.
Thus, this algorithm solves reachability problem, or, according to Hellings~\cite{hellingsRelational}, processes CFPQs by using relational query semantics.

The performance-critical part of the algorithm is matrix boolean multiplication.
Note, that we can apply the next optimization: we can skip update if the matrices $T_{N_j}$ and $T_{N_k}$ have not been changed at the previous iteration.
Also, it is important for applications that real-world data is almost sparse, so it should be better solution to use linraries which manipulates with sparse matrices.

As we can see, CFPQ can be naturally reduced to linead algebra. 
Linear algebra for graph problems is an actively developed area. 
One of the most important result is a GraphBLAS API which provides a way to operate over matrices and vectors over user-defined semirings.
In this paper we use SuteSparse implementation of GraphBLAS and boolena semiring.
All our implementations are based on the optimized version of the algorithm.