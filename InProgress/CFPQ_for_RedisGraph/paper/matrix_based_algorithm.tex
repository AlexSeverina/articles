\section{Matrix-based Algorithm for CFPQ}

The matrix-based algorithm for CFPQ w.r.t. the relational query semantics was proposed by Rustam Azimov~\cite{Azimov:2018:CPQ:3210259.3210264}.
This algorithm can be expressed in terms of operations over Boolean matrices (see listing~\ref{lst:algo1}) which is an advantage for implementation.
{\small
\begin{algorithm}
  \floatname{algorithm}{Listing}
\begin{algorithmic}[1]
\caption{Context-free path quering algorithm}
\label{lst:algo1}
\Function{evalCFPQ}{$D=(V,E), G=(N,\Sigma,P)$}
    \State{$n \gets$ |V|}
    \State{$T \gets \{T^{A_i} \mid A_i \in N, T^{A_i}$ is a matrix $n \times n$, $T^{A_i}_{k,l} \gets$ \texttt{false}\} }
    \ForAll{$(i,x,j) \in E$, $A_k \mid A_k \to x \in P$}
        %\Comment{Matrices initialization}
        %\For{$A_k \mid A_k \to x \in P$}
          {$T^{A_k}_{i,j} \gets \texttt{true}$}
        %\EndFor
    \EndFor
    \For{$A_k \mid A_k \to \varepsilon \in P$}
       {$T^{A_k}_{i,i} \gets \texttt{true}$}
    \EndFor

    \While{any matrix in $T$ is changing}
        %\Comment{Transitive closure calculation}
        \For{$A_i \to A_j A_k \in P$}
          { $T^{A_i} \gets T^{A_i} + (T^{A_j} \times T^{A_k})$ } 
        \EndFor
    \EndWhile
\State \Return $T$
\EndFunction
\end{algorithmic}
\end{algorithm}
}

Here $D = (V, E)$ is the input graph and $G = (N,\Sigma,P)$ is the input grammar.
For each matrix $T^{A_k}$ indexed with a nonterminal $A_k \in N$, a cell holds a true value ($T^{A_k}_{i,j} = \texttt{true}$) if and only if there exists $i \pi j$~--- a path in $D$ such that $A_k \xRightarrow[G]{*} l(\pi) $, where $l(\pi)$ is a word formed by the labels along the path $\pi$.
Thus, this algorithm solves the reachability problem, or, according to Hellings~\cite{hellingsRelational}, implements relational query semantics.

The performance-critical part of the algorithm is Boolean matrix multiplication, thus one can achieve better performance by using libraries that efficiently multiply boolean matrices. 
There is also the following optimization: if the matrices $T^{A_j}$ and $T^{A_k}$ have not changed at the previous iteration, then we can skip the update operation in line 7.
Data in real-world problems is often sparse, thus employing libraries that manipulate sparse matrices improves running time even more. 
