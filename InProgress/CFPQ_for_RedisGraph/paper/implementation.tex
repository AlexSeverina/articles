\section{Implementation}

We showed that CFPQ can be naturally reduced to linear algebra.
Linear algebra for graph problems is an actively developed area.
One of the most important results is a GraphBLAS API which provides a way to operate over matrices and vectors over user-defined semirings.

Previous works show~\cite{Mishin:2019:ECP:3327964.3328503, Azimov:2018:CPQ:3210259.3210264} that existing linear algebra libraries utilization is the right way to achieve high-performance CFPQ implementation with minimal effort.
But neither of these works provide an evaluation with data storage: algorithm execution time has been measured in isolation.

We provide a number of implementations of the matrix-based CFPQ algorithm.
We use RedisGraph as storage and implement CFPQ as an extension by using the mechanism provided.
Note that currently, we do not provide complete integration with the querying mechanism: one cannot use Cypher~--- a query language used in RedisGraph.
Instead, a query should be provided explicitly as a file with grammar in Chomsky normal form.
This is enough to evaluate querying algorithms and we plan to improve integration in the future to make our solution easier to use. 

\textbf{CPU-based implementation (RG\_CPU)} uses SuteSparse implementation of GraphBLAS, which is also used in RedisGraph, and a predefined boolean semiring.
Thus we avoid data format issues: we use native RedisGraph representation of the adjacency matrix in our algorithm.

\textbf{GPGPU-based implementation} has two versions.
The first one (\textbf{RG\_M4RI}) uses the Method of Four Russians implemented in~\cite{Mishin:2019:ECP:3327964.3328503}, and the second one (\textbf{RG\_CUSP}) utilizes a modified CUSP library for matrix operations.
Both implementations require matrix format conversion.
