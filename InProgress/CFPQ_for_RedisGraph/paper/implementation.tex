\section{Implementation}

As we can see, CFPQ can be naturally reduced to linear algebra.
Linear algebra for graph problems is an actively developed area.
One of the most important results is a GraphBLAS API which provides a way to operate over matrices and vectors over user-defined semirings.

Previous works show~\cite{Mishin:2019:ECP:3327964.3328503, Azimov:2018:CPQ:3210259.3210264} that existing linear algebra libraries utilization is the right way to get high-performance CFPQ implementation in minimal effort.
But none of these works do not provide an evaluation with data storage, only pure time of algorithm execution was measured.

We provide a number of implementations of matrix-based CFPQ algorithm.
All of them a based on RedisGraph~---~we use RedisGraph as a storage and implement CFPQ as an extension by using provided mechanism.
Note that currently, we do not provide full integration with querying mechanism: one can not use Cypher, which uses in RedisGraph as a query language.
Instead, in the current implementation query provided explicitly as a file with grammar in Chomsky normal form.
So, we can evaluate querying algorithms, but we should improve integration to make our solution applicable.

\textbf{CPU-based implementation} uses SuteSparse implementation of GraphBLAS, which is used in RedisGraph, and predefined boolean semiring.
Thus we avoid data format problems: we use native RedisGraph representation of the adjacency matrix in our algorithm.

\textbf{GPGPU-based implementation} is provided in two versions.
The first one uses m4ri method implemented in~\cite{Mishin:2019:ECP:3327964.3328503}, and the second one utilizes a modified CUSP library for matrix operations.
Both these implementations require matrix format conversion.
