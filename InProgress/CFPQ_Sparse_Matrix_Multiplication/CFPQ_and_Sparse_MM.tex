\documentclass[sigconf]{acmart}
\usepackage{algorithm}
\usepackage{caption}
\usepackage{algorithmicx}
\usepackage[noend]{algpseudocode}

%%
%% \BibTeX command to typeset BibTeX logo in the docs
\AtBeginDocument{%
  \providecommand\BibTeX{{%
    \normalfont B\kern-0.5em{\scshape i\kern-0.25em b}\kern-0.8em\TeX}}}

%% Rights management information.  This information is sent to you
%% when you complete the rights form.  These commands have SAMPLE
%% values in them; it is your responsibility as an author to replace
%% the commands and values with those provided to you when you
%% complete the rights form.
\setcopyright{acmcopyright}
\copyrightyear{2018}
\acmYear{2018}
\acmDOI{10.1145/1122445.1122456}

%% These commands are for a PROCEEDINGS abstract or paper.
\acmConference[Woodstock '18]{Woodstock '18: ACM Symposium on Neural
  Gaze Detection}{June 03--05, 2018}{Woodstock, NY}
\acmBooktitle{Woodstock '18: ACM Symposium on Neural Gaze Detection,
  June 03--05, 2018, Woodstock, NY}
\acmPrice{15.00}
\acmISBN{978-1-4503-XXXX-X/18/06}


%%
%% Submission ID.
%% Use this when submitting an article to a sponsored event. You'll
%% receive a unique submission ID from the organizers
%% of the event, and this ID should be used as the parameter to this command.
%%\acmSubmissionID{123-A56-BU3}

%%
%% The majority of ACM publications use numbered citations and
%% references.  The command \citestyle{authoryear} switches to the
%% "author year" style.
%%
%% If you are preparing content for an event
%% sponsored by ACM SIGGRAPH, you must use the "author year" style of
%% citations and references.
%% Uncommenting
%% the next command will enable that style.
%%\citestyle{acmauthoryear}

%%
%% end of the preamble, start of the body of the document source.
\begin{document}

%%
%% The "title" command has an optional parameter,
%% allowing the author to define a "short title" to be used in page headers.
\title{Context-Free Path Queries and Sparse Matrix Multiplication}

%%
%% The "author" command and its associated commands are used to define
%% the authors and their affiliations.
%% Of note is the shared affiliation of the first two authors, and the
%% "authornote" and "authornotemark" commands
%% used to denote shared contribution to the research.

\author{Ekaterina Shemetova}
\affiliation{%
  \institution{The Th{\o}rv{\"a}ld Group}
  \streetaddress{1 Th{\o}rv{\"a}ld Circle}
  \city{Hekla}
  \country{Iceland}}
\email{larst@affiliation.org}

\author{Arseniy Terekhov}
\affiliation{%
  \institution{Inria Paris-Rocquencourt}
  \city{Rocquencourt}
  \country{France}
}

\author{Semyon Grigorev}
\affiliation{%
 \institution{Rajiv Gandhi University}
 \streetaddress{Rono-Hills}
 \city{Doimukh}
 \state{Arunachal Pradesh}
 \country{India}}



%%
%% By default, the full list of authors will be used in the page
%% headers. Often, this list is too long, and will overlap
%% other information printed in the page headers. This command allows
%% the author to define a more concise list
%% of authors' names for this purpose.

%%
%% The abstract is a short summary of the work to be presented in the
%% article.
\begin{abstract}

\end{abstract}

%%
%% The code below is generated by the tool at http://dl.acm.org/ccs.cfm.
%% Please copy and paste the code instead of the example below.
%%
\begin{CCSXML}
<ccs2012>
 <concept>
  <concept_id>10010520.10010553.10010562</concept_id>
  <concept_desc>Computer systems organization~Embedded systems</concept_desc>
  <concept_significance>500</concept_significance>
 </concept>
 <concept>
  <concept_id>10010520.10010575.10010755</concept_id>
  <concept_desc>Computer systems organization~Redundancy</concept_desc>
  <concept_significance>300</concept_significance>
 </concept>
 <concept>
  <concept_id>10010520.10010553.10010554</concept_id>
  <concept_desc>Computer systems organization~Robotics</concept_desc>
  <concept_significance>100</concept_significance>
 </concept>
 <concept>
  <concept_id>10003033.10003083.10003095</concept_id>
  <concept_desc>Networks~Network reliability</concept_desc>
  <concept_significance>100</concept_significance>
 </concept>
</ccs2012>
\end{CCSXML}

\ccsdesc[500]{Computer systems organization~Embedded systems}
\ccsdesc[300]{Computer systems organization~Redundancy}
\ccsdesc{Computer systems organization~Robotics}
\ccsdesc[100]{Networks~Network reliability}

%%
%% Keywords. The author(s) should pick words that accurately describe
%% the work being presented. Separate the keywords with commas.
\keywords{datasets, neural networks, gaze detection, text tagging}


%%
%% This command processes the author and affiliation and title
%% information and builds the first part of the formatted document.
\maketitle
\chapter*{Введение}                         % Заголовок
\addcontentsline{toc}{chapter}{Введение}    % Добавляем его в оглавление
\textbf{Актуальность работы}

Статический анализ исходного кода является известной техникой  получения знаний о программе без её исполнения ~\cite{StaticCodeAnalysis3,StaticCodeAnalysis2,StaticCodeAnalysis1}. Статический анализ является неотъемлемой частью многих процессов, связанных с разработкой программного обеспечения (ПО), и может использоваться, например, для упрощения работы с кодом с помощью подсветки синтаксиса языка в программах, навигации по коду, реализации контекстных подсказок. Более того, статический анализ используется для обнаружения ошибок на ранних стадиях разработки, до запуска программы, а также для поиска различных семантических ошибок, которые не могут быть определены обычным синтаксическим анализом.  Также, статический анализ используется при решении задач трансформации исходного кода и реинжиниринге~\cite{reengANT}. Однако во многих языках программирования имеются конструкции, которые существенно затрудняют статический анализ. 

Например, широко используются динамические встроенные языки --- приложение, созданное на одном языке, генерирует программу на другом языке и передаёт её на выполнение в соответствующее окружение. Примерами могут служить динамические SQL-запросы к базам данных из приложений на Java, С++, С\#, формирование HTML-страниц в PHP-приложениях~\cite{DSQLISO,JSP,PHPmySQL}. Генерируемый код собирается из строк таким образом, чтобы в момент выполнения результирующая строка представляла собой корректную программу. Примеры использования встроенных языков представлены в листингах~\ref{lst:dsql1},~\ref{lst:JsJava} и~\ref{lst:PhPSqlHtml}. Следует отметить, что одна программа может генерировать код на нескольких языках (см. листинг~\ref{lst:PhPSqlHtml}). При этом возможно получение частей кода из разных источников (например, учитывать текстовый ввод пользователя, что часто используется для задания фильтров при конструировании SQL-запросов). Использование динамически формируемых программ  позволяет избежать дополнительных накладных расходов, присущих таким технологиям, как ORM\footnote{ORM или Object-Relational Mapping --- технология программирования, которая связывает базы данных с концепциями объектно-ориентированных языков программирования~\cite{ORM}.}, и достичь высокой производительности. Благодаря этому использование динамически генерируемых программ получило широкое распространение и применяется до сих пор. Вместе с этим, несмотря на появление новых технологий, динамическая генерация SQL-запросов активно используется и в настоящее время~\cite{DSQLInActiveUse}.

\fvset{frame=lines,framesep=5pt,fontsize=\small}\

\begin{listing}
    \begin{pyglist}[language=sql,numbers=left,numbersep=5pt]

CREATE PROCEDURE [dbo].[MyProc]  @TABLERes   VarChar(30)
AS
    EXECUTE ('INSERT INTO ' + @TABLERes + ' (sText1)' +
             ' SELECT ''Additional condition: '' + sName' +
             ' from #tt where sAction = ''1000000''')
GO
    \end{pyglist}
\caption{Код с использованием динамического SQL}
\label{lst:dsql1}
\end{listing} 
 
\fvset{frame=lines,framesep=5pt}
\begin{listing}
    \begin{pyglist}[language=java,numbers=left,numbersep=5pt]
import javax.script.*;  
public class InvokeScriptFunction {  
    public static void main(String[] args) throws Exception {  
        ScriptEngineManager manager = new ScriptEngineManager();  
        ScriptEngine engine = manager.getEngineByName("JavaScript");  
        // JavaScript code in a String  
        String script = 
            "function hello(name) { print('Hello, ' + name); }";  
        // evaluate script  
        engine.eval(script);  
        // javax.script.Invocable is an optional interface.  
        // Check whether your script engine implements or not!  
        // Note that the JavaScript engine implements
        // Invocable interface.  
        Invocable inv = (Invocable) engine;  
        // invoke the global function named "hello"  
        inv.invokeFunction("hello", "Scripting!!" );  
    }  
}
    \end{pyglist}
\caption{Вызов JavaScript из Java}
\label{lst:JsJava}
\end{listing}


\fvset{frame=lines,framesep=5pt}
\begin{listing}
    \begin{pyglist}[language=php,numbers=left,numbersep=5pt]

<?php
    // Embedded SQL
    $query = 'SELECT * FROM ' . $my_table; 
    $result = mysql_query($query);
    
    // HTML markup generation
    echo "<table>\n";
    while ($line = mysql_fetch_array($result, MYSQL_ASSOC)) {
        echo "\t<tr>\n";    
        foreach ($line as $col_value) {
            echo "\t\t<td>$col_value</td>\n";
        }
        echo "\t</tr>\n";
    }
    echo "</table>\n";
?>
    \end{pyglist}
\caption{Использование нескольких встроенных в PHP языков (MySQL, HTML)}
\label{lst:PhPSqlHtml}
\end{listing}



Динамически формируемые выражения часто конструируются с помощью таких операций, как конкатенация в циклах или условных предложениях, или в рекурсивных процедурах. Это затрудняет статический анализ и приводит к получению множества возможных значений для каждого выражения в момент выполнения. Вследствие этого фрагменты динамически формируемого кода воспринимаются компилятором исходного языка как простые строки, не подлежащие дополнительному анализу, а это, в свою очередь, приводит к высокой вероятности возникновения ошибок во время выполнения программы. В худшем случае такая ошибка не приведёт к прекращению работы приложения, что указало бы на проблемы, однако целостность данных при этом может оказаться нарушена. Более того, использование динамически формируемых выражений затрудняет не только разработку информационных систем, так и также и реинжиниринг, поскольку в последнем случае важно автоматизировать перенос системы на новые зыки и платформы, что невозможно без качественного статического анализа. Например, при наличии в коде приложения динамически формируемых SQL-запросов нельзя точно ответить на вопрос о том, с какими элементами базы данных не взаимодействует система, и удалить их. При переносе такой системы на другую СУБД необходимо гарантировать, что для всех динамически формируемых выражений значение в момент выполнения будет корректным кодом на языке новой СУБД~\cite{JSquash}. Следует отметить, что отсутствие статического анализа динамически формируемых программ не позволяет реализовывать для них стандартную функциональность интегрированных сред разработки (Integrated Development Environment, IDE) --- подсветку синтаксиса и автодополнение, рефакторинг кода и т.д. Такая функциональность значительно упрощает процесс разработки и отладки приложений и полезна не только для основного языка, но и для встроенных языков. 

Для решения всех перечисленных выше задач необходимы инструменты, проводящие статический анализ динамически формируемых программ. Такой анализ может дать существенную информацию о таких программах, поскольку редко встречается ситуация полной динамической неопределённости (например, при создании динамических программ исключительно на основе пользовательского ввода). В большинстве случаев, имея программу, генерирующую динамические вставки, с помощью статического анализа можно получить достаточно информации для решения поставленных выше задач. Решению этой проблемы и посвящена данная диссертационная работа. 


\textbf{Степень разработанности темы}

Существуют классические исследования, посвященные разработке компиляторов --- работы А.~Ахо~\cite{Dragon}, А.~Брукера~\cite{CompilerCompiler}, С.~Джонсона~\cite{yaccBook},  Б.К.Мартыненко~\cite{Martinenko1, Martinenko2}  и др.  Однако содержащиеся там алгоритмы синтаксического анализа не могут быть применены к решению задачи анализа динамически формируемых программ, поскольку предназначены для обработки входных данных, представимых в видн линейной последовательности символов, а такое представление динамически формируемых программ не всегда возможно.

Методы обобщённого синтаксического анализа, лежащие в основе данной работы, изложены в трудах таких учёных как Масару Томита (Masaru Tomita)~\cite{Tomita}, Элизабет Скотт (Elizabeth Scott) и Адриан Джонстон (Adrian Johnstone)~\cite{RNGLR,RIGLR} из университета Royal Holloway (Великобритания), Ян Рекерс (Jan Rekers, University of Amsterdam)~\cite{SPPF}, Элко Виссер (Eelco Visser)~\cite{RNGLRSyntaxerror2,RNGLRSyntaxerror3} и других.

Анализу динамически формируемых строковых выражений посвящены работы таких зарубежных учёных как Кюнг-Гу Дох (Kyung-Goo Doh)~\cite{LrAbstract1,LrAbstract2,LRAbstractParsingSema}, Ясухико Минамиде (Minamide Yasuhiko)~\cite{PHPSA}, Андерс Мёллер (Anders M{\o}ller)~\cite{JSA} и отечественных учёных А.А.~Бреслава~\cite{Alvor1,Alvor2} и других. Хорошо изучены вопросы проверки корректности динамически формируемых выражений и поиска фрагментов кода, уязвимых для SQL-инъекций~\cite{SQLInjection,Dasgupta:2009:SAF:1546683.1547548}. Однако данные работы исследуют отдельные аспекты проблемы статического анализа динамически формируемых программ, оставляя в стороне создание готовых алгоритмов (в частности, не строят структурное представление анализируемых программ). В связи с этим возникают проблемы масштабируемости данных результатов, например, создание на их основе более сложных видов статического анализа.

Так же важным является предоставление компонентов, упрощающих создание новых инструментов для решения конкретных задач. Данных подход хорошо исследован в области разработки компиляторов, где широкое распространение получили генераторы анализаторов и пакеты стандартных библиотек (работы А.~Ахо~\cite{Dragon}, А.~Брукера~\cite{CompilerCompiler}, С.~Джонсона~\cite{yaccBook} и др.). 

В работах отечественных учёных М.Д.~Шапот и Э.В.~Попова~\cite{DynamicDSQLTranslation}, а так же зарубежных учёных Антони Клеви (Anthony Cleve), Жан-Люк Эно (Jean-Luc Hainaut)~\cite{DSQLReverseEngineering}, Йост Виссер (Joost Visser)~\cite{DSQLQualityMesure} и других рассматриваются различные аспекты реинжиниринга информационных систем, использующих встроенные SQL-запросы, однако не формулируется общего метода для решения таких задач. Этот вопрос также не затрагивается в классических работах, посвященных реинжиниригу~\cite{SoftwareReeng1, reengANT, SoftwareReeng2, SoftwareReeng3}. Однако разработка такого метода является актуальной задачей.

Таким образом, актуальной является задача дальнейшего исследования статического анализа динамически формируемых строковых выражений. Кроме этого важным является решение вопросов практического применения средств анализа динамически формируемого кода: упрощение разработки инструментов анализа и создание методов их применения в реинжиниринге программного обеспечения.
\textbf{Объект исследования}

Объектом исследования являются методы, алгоритмы и программные средства обработки динамически формируемых программ, а также задача реинжиниринга информационных систем.

\textbf{Цель и задачи диссертационной работы}

\textbf{Целью} данной работы является создание комплексного подхода к статическому синтаксическому анализу динамически формируемых программ.

Достижение поставленной цели обеспечивается решением следующих \textbf{задач}.
\begin{enumerate}
    \item Разработать универсальный алгоритм синтаксического анализа динамически формируемых программ, не зависящий от целевого языка программирования и допускающий реализацию различных видов статического анализа. 
    \item Создать архитектуру инструментария для автоматизации разработки программных средств статического анализа динамически формируемых программ.
    \item Создать метод реинжиниринга динамически формируемых программ.
\end{enumerate}

\textbf{Методология и методы исследования}

Методология исследования основана на подходе к спецификации и анализу формальных языков, активно развивающемуся с 50-х годов 20-го века (см., например, работы Н. Хомского~\cite{chomskyMethod ,chomskySyntactic}). В последствии этот подход получил широкое распространение в областях, связанных с обработкой языков программирования.
Основными элементами данного подхода являются алфавит и грамматика языка, разбиение автоматической обработки языка на выполнение таких шагов как лексический, синтаксический и семантический анализ. Решаемые в связи с этим задачи связаны с поиском эффективных алгоритмов, выполняющих эти шаги. 

В работе применяется алгоритм обобщённого восходящего синтаксического анализа RNGLR~\cite{RNGLR}, созданный Элизабет Скотт (Elizabeth Scott) и Адриан Джонстон (Adrian Johnstone) из университета Royal Holloway (Великобритания). Для компактного хранения леса вывода использовалась структура данных Shared Packed Parse Forest (SPPF), которую предложил Ян Рекерс (Jan Rekers, University of Amsterdam)~\cite{SPPF}.

Доказательство завершаемости и корректности предложенного алгоритма проводилось с применением теории формальных языков, теории графов и теории сложности алгоритмов. Приближение множества значений динамически формируемого выражения строилось в виде регулярного множества, описываемого с помощью конечного автомата.


\textbf{Положения, выносимые на защиту}
\begin{enumerate}
    \item Разработан алгоритм синтаксического анализа динамически формируемых программ, позволяющий обрабатывать произвольную регулярную аппроксимацию множества значений выражения в точке выполнения, реализующий эффективное управление стеком и гарантирующий конечность представления леса вывода. Доказана завершаемость и корректность предложенного алгоритма при обработке регулярной аппроксимации, представимой в виде произвольного конечного автомата без $\varepsilon$-переходов. 
    \item Создана архитектура инструментария для разработки программных средств статического анализа динамически формируемых программ.
    \item Разработан метод анализа и обработки динамически формируемых программ в проектах по реинжинирингу информационных систем. 
\end{enumerate}

\textbf{Научная новизна работы}

Научная новизна полученных в ходе исследования результатов заключается в следующем.

\begin{enumerate}

\item Алгоритм, предложенный в диссертации, отличается от аналогов (работы Андрея Бреслава~\cite{Alvor1, Alvor2}, Кюнг-Гу Дох~\cite{LrAbstract1, LrAbstract2}, Ясухико Минамиде~\cite{PHPSA}) возможностью построения компактной структуры данных, содержащей деревья вывода для всех корректных значений выражения. Это позволяет использовать результаты работы алгоритма для проведения более сложных видов анализа. Алгоритмы, представленные в (JSA~\cite{JSA}~\cite{Alvor1, Alvor2}, PHPSA~\cite{PHPSA}) предназначены только для проверки корректности выражений, основанной на решении задачи о включении одного языка в другой. Выполнение более сложных видов анализа, трансформаций или построения леса разбора не предполагается. 

\item Новизна представленной архитектуры заключается в том, что она позволяет создать платформу для разработки целевых инструментов, решающих широкий круг задач анализа динамически формируемого кода. Существующие архитектуры готовых инструментов (JSA, PHPSA, Alvor, Varis) предназначены для решения конкретных задач для определённых языков. Решение новых задач или поддержка других языков с помощью этих инструментов затруднены ввиду ограничений, накладываемых архитектурой и возможностями используемого алгоритма анализа. 

\item Метод анализа и обработки встроенного программного кода в проектах по реинжинирингу информационных систем предложен впервые. К.В.~Ахтырченко и Т.П.~Сорокваша отмечают~\cite{SoftwareReengMethods}, что существующие работы в области реинжиниринга программного обеспечения либо содержат высокоуровневые решения, не касающиеся деталей, важных при решении прикладных задач (например, работы К. Вагнера~\cite{SoftwareReeng3}, Х. Миллера~\cite{SoftwareReeng2}), либо являются набором подходов к решению конкретных задач (например, работы~\cite{SoftwareReeng1, reengANT, boulychev}). При этом, встроенный программный код часто не учитывается. С другой стороны, работы М.Д.~Шапот и Э.В.~Попова~\cite{DynamicDSQLTranslation}, С.Л.~Трошина~\cite{reengANT}, А.~Клеви~\cite{DSQLReverseEngineering}  посвящены решению конкретных задач обработки встроенного программного кода в контексте реинжиниринга информационных систем, но не предлагают обобщённого и масштабируемого метода.

\end{enumerate}


\textbf{Теоретическая и практическая значимость работы}

Теоретическая значимость диссертационного исследования заключается в разработке формального алгоритма синтаксического анализа динамически формируемого кода, решающего задачу построения конечного представления леса вывода, не решенную полностью ранее, а также в формальном доказательстве завершаемости и корректности разработанного алгоритма. 

На основе полученных в работе научных результатов был разработан инструментарий (Software Development Kit, SDK), предназначенный для создания средств статического анализа динамически формируемых программ. 
С использованием разработанного инструментария было реализовано расширение к инструменту ReSharper (ООО ``ИнтеллиДжей Лабс'', Россия), предоставляющее поддержку встроенного T-SQL в проектах на языке программирования C\# в среде разработки Microsoft Visual Studio. Так же было выполнено внедрение результатов работы в промышленный проект по переносу хранимого SQL-кода с MS-SQL Server 2005 на Oraclе 11gR2 (ЗАО ``Ланит-Терком'', Россия). 

\textbf{Степень достоверности и апробация результатов}

Достоверность и обоснованность результатов исследования опирается на использование формальных методов исследуемой области, проведенные доказательства, рассуждения и эксперименты.

Основные результаты работы были доложены на ряде научных конференций: SECR-2012, SECR-2013, SECR-2014, TMPA-2014, Parsing@SLE-2013, Рабочий семинар ``Наукоемкое программное обеспечение'' при конференции PSI-2014. Доклад на SECR-2014 награждён премией Бертрана Мейера за лучшую исследовательскую работу в области программной инженерии. Дополнительной апробацией является то, что разработка инструментальных средств на основе предложенного алгоритма была поддержана Фондом содействия развитию малых форм предприятий в технической сфере (программа УМНИК~\cite{UMNIC}, проекты \textnumero~162ГУ1/2013 и \textnumero~5609ГУ1/2014).

\textbf{Публикации по теме диссертации}

Все результаты диссертации изложены в 7 научных работах, из которых 3~\cite{YCArticle,SELforIDEru,AbstractGLL}, содержащие основные результаты, опубликованы в журналах из ‘’Перечня российских рецензируемых научных журналов, в которых должны быть опубликованы основные научные результаты диссертаций на соискание ученых степеней доктора и кандидата наук’’, рекомендовано ВАК. 
1 работа~\cite{GLRAbsPars} индексируются Scopus. В работах, написанных в соавторстве, вклад автора определяется следующим образом.  В~\cite{Syrcose} С. Григорьеву принадлежит реализация ядра платформы YaccConstructor. В~\cite{SELforIDEru, AbstractGLL} и~\cite{SELforIDE} С. Григорьеву принадлежит постановка задачи, формулирование требований к разрабатываемым инструментальным средствам, работа над текстом. 
В~\cite{GLRAbsPars} автору принадлежит идея, описание и реализация анализа встроенных языков на основе RNGLR алгоритма.  В~\cite{YCArticle} С. Григорьеву принадлежит реализация инструментальных средств, проведение экспериментов, работа над текстом. В работе~\cite{RelaxedARNGLR} автору принадлежит алгоритм синтаксического анализа динамически формируемого кода.


\textbf{Структура работы}

Диссертация состоит из введения, шести глав, заключения и построена следующим образом. В первой главе приводится обзор области исследования. Рассматриваются подходы к анализу динамически формируемых строковых выражений и соответствующие инструменты. Описывается алгоритм обобщённого восходящего синтаксического анализа RNGLR, взятый за основу в диссертационной работе. Также описываются проекты YaccConstructor и ReSharper SDK, использованные для реализации предложенного в работе инструментария. Во второй главе формализуется основная задача исследования и излагается решающий её алгоритм синтаксического анализа регулярных множеств. Приводится доказательство завершаемости и корректности представленного алгоритма, приводятся примеры. В третьей главе описывается инструментальный пакет YC.SEL.SDK, разработанного на основе алгоритма, описанного во второй главе и предназначеного для разработки инструментов анализа динамически формируемых программ. Описывается архитектура компонентов и особенности их реализации. Также описывается YC.SEL.SDK.ReSharper --- ``обёртка'' для YC.SEL.SDK, позволяющая создавать расширения к ReSharper для поддержки встроенных языков. В четвёртой главе описывается метод реинжиниринга встроенного программного кода.  В пятой главе приводятся результаты экспериментального исследования разработанного алгоритма и инструмента YC.SEL.SDK. Шестая глава содержит результаты сравнения и соотнесения полученных результатов с  существующими аналогами.

\textbf{Благодарности}

А.Н.Терехову, работкникам и администрации компании ЗАО ``Ланит-Терком'' за создания условий для изучения данной темы (организация проектов по реинжинирингу). Я.А.Кириленко за погружение в тему исследования и руководство на начальных этапах. Д.Ю.Булычеву за помощь в уточнении постановки задачи исследования и в написании статей. Студентам и аспирантам кафедры системного программирования Дмитрию Авдюхину, Анастасии Рагозиной, Екатерине Вербицкой, Марине Полубеловой, Иванову Андрею за помощь в реализации предложенных идей и проведение экспериментов. Отдельную благодарность  хочется выразить компании ООО ``ИнтеллиДжей Лабс'' и Андрею Иванову за поддержку исследований. Также хочется поблагодарить А.К.Петренко и В.М.Ицыксона, а также сотрудников ИСП РАН за ценные вопросы и комментарии к работе, позволившие уточнить ряд формулировок и улучшить изложение результатов. 

\section{Preliminaries} \label{section_preliminaries}
In this section, we introduce the basic notions used throughout the paper.

Let $\Sigma$ be a finite set of edge labels. Define an \textit{edge-labeled directed graph} as a tuple $D = (V, E)$ with $V$ is a set of nodes and $E \subseteq V \times \Sigma \times V$ is a directed edge-relation.  For a path $\pi$ in a graph $D$ we denote $l(\pi)$ --- the unique word obtained by concatenating the labels of the edges along the path $\pi$. Also, we write $n \pi m$ to indicate that a path $\pi$ starts at node $n \in V$ and ends at node $m \in V$.

According to Hellings~\cite{hellingsRelational}, we deviate from the usual definition of a context-free grammar in \textit{Chomsky Normal Form}~\cite{chomsky} by not including a special start non-terminal, which will be specified in the queries to the graph. Since every context-free grammar can be transformed into an equivalent one in Chomsky Normal Form and checking that an empty string is in the language is trivial, then it is sufficient to only consider grammars of the following type. A \textit{context-free grammar} is 3-tuple $G = (N, \Sigma, P)$ where $N$ is a finite set of non-terminals, $\Sigma$ is a finite set of terminals, and $P$ is a finite set of productions of the following forms:

\begin{itemize}
    \item $A \rightarrow B C$, for $A,B,C \in N$,
    \item $A \rightarrow x$, for $A \in N$ and $x \in \Sigma$.   
\end{itemize}

Note that we omit the rules of the form $A \rightarrow \varepsilon$, where $\varepsilon$ denotes an empty string. This does not limit the applicability of further algorithms because checking that an empty string belongs to the context-free language in Chomsky normal form is trivial and only the empty paths $m \pi m$ correspond to an empty string $\varepsilon$.

We use the conventional notation $A \xrightarrow{*} w$ to denote that the string $w \in \Sigma^*$ can be derived from a non-terminal $A$ by some sequence of applying the production rules from $P$. The \textit{language} of a grammar $G = (N,\Sigma,P)$ with respect to a start non-terminal $S \in N$ is defined by $L(G_S) = \{w \in \Sigma^*~|~S \xrightarrow{*} w\}$.

For a given graph $D = (V, E)$ and a context-free grammar $G = (N, \Sigma, P)$, we define \textit{context-free relations} $R_A \subseteq V \times V$, for every $A \in N$, such that $R_A = \{(n,m)~|~\exists n \pi m~(l(\pi) \in L(G_A))\}$.

We define a binary operation on arbitrary subsets $N_1 , N_2$ of $N$ with respect to a context-free grammar $G = (N, \Sigma, P)$ as $N_1 \cdot N_2 = \{A~|~\exists B \in N_1, \exists C \in N_2 \text{ such that }(A \rightarrow B C) \in P\}.$

Using this binary operation as a multiplication on arbitrary subsets of $N$ and union of sets as an addition, we can define a \textit{matrix multiplication}, $a \cdot b = c$, where $a$ and $b$ are matrices of the suitable size that have subsets of $N$ as elements, as $c_{i,j} = \bigcup^{n}_{k=1}{a_{i,k} \cdot b_{k,j}}$.

We define the \textit{transitive closure} of a square matrix $a$ as $a^+ = a^{(1)} \cup a^{(2)} \cup \cdots$ where $a^{(i)} = a^{(i-1)} \cup (a^{(i-1)} \cdot a^{(i-1)})$, $i \ge 2$ and $a^{(1)} = a$.
\section{Cubic upper bound using sparse matrix multiplication}
In this section we show that the Algorithm ~\ref{CFPQMM} has complexity $O(n^3)$ if sparse matrix multiplication is used. The reason is that one needs to compute at most $|N|n^2$ realizable triples in total during all iterations of the algorithm, whereas the number of new triples found on each $i$-th iteration can be relatively small. The number of operations of the $i$-th iteration can be reduced by multiplying the matrix $B_{(i-2)}$ containing all the previously found triples, on the matrix $A_{(i-1)}$ which has only those triples firstly obtained in the $(i-1)$-th iteraition. In other words, we have:
\begin{equation}
\label{Bi}
B_i = B_{i-1} + A_i, 
\end{equation}
where
\begin{equation}
\label{Ai}
A_i = (B_{i-2}A_{i-1} + A_{i-1}B_{i-2} + A_{i-1}A_{i-1}) - B_{i-2}.
\end{equation}
So the following two conditions hold:
\begin{enumerate}
\item $\forall i, B_{(i-2)} \cap A_{(i-1)} = \emptyset$;
\item $\forall i, j,  A_i \cap A_j = \emptyset$.
\end{enumerate}
Notice that the first condition implies that one of the two multiplied matrices should be sparse, because $nnz(B_{(i-2)}) + nnz(A_{(i-1)}) \le |N|n^2$. Also, by the second condition matrices $A$ are pairwise disjoint, therefore $nnz(\sum\limits_{i=1}^{n^2}{A_i}) \le |N|n^2$. 


The Algorithm \ref{CFPQMM} can be modified using Equations \ref{Bi} and \ref{Ai} instead of the naive calculation of transitive closure on every iteration. It is important that the modified agorithm has the same number of iterations in the worst case as the original one --- $|N|n^2 = O(n^2)$. This is because the height of the parse tree does not exceed this value. As in the Algorithm~\ref{CFPQMM}, modified version derives new triples, going from leaves to the root of the parse tree.
\begin{theorem}
The matrix $B_{n^2}$ containing all possible realizable triples can be calculated in $O(n^3)$ time.
\end{theorem}
\begin{proof}
The correctness of the algorithm can be easily deduced from the correctness of the Algorithm \ref{CFPQMM} \cite{Azimov}. Now we show the cubic time complexity of the modified algorithm. Consider the equation for calculating $B_{n^2}$:
$$B_{n^2} = B_{n^2-1} + B_{i-2}A_{i-1} + A_{i-1}B_{i-2} + A_{i-1}A_{i-1} = $$
$$ = B_{n^2-2} + B_{i-3}A_{i-2} + A_{i-2}B_{i-3} + A_{i-2}A_{i-2} +$$ 
$$+ B_{i-2}A_{i-1} + A_{i-1}B_{i-2} + A_{i-1}A_{i-1} = ... =$$
$$= B_1 + B_1B_1 + \sum\limits_{i=2}^{n^2-1}B_{i-2}A_{i-1}+ \sum\limits_{i=2}^{n^2-1}A_{i-1}B_{i-2} + \sum\limits_{i=2}^{n^2-1}A_{i-1}A_{i-1}.$$
Suppose, without loss of generality, that the matrix $B_{i-2}$ is dense, than the matrix $A_{i-1}$ is sparse. Using naive sparse matrix multiplication algorithm, we have that $O(nnz(A_{i-1})n)$ operations are needed for multiplication of the matrix $B_{i-2}$ and the matrix $A_{i-1}$. Let $T(AB)$ be the number of operations that need to be performed to multiply matrices $A$ and $B$. Thus, the total number of operation for obtaining $B_{n^2}$ is in
$$ T(B_1B_1) + \sum\limits_{i=2}^{n^2-1}T(B_{i-2}A_{i-1})+ \sum\limits_{i=2}^{n^2-1}T(A_{i-1}B_{i-2}) + \sum\limits_{i=2}^{n^2-1}T(A_{i-1}A_{i-1})=$$
$$= O(n^{\omega}) +  O(\sum\limits_{i=2}^{n^2-1}nnz(A_{i-1})n) =$$
$$= O(n^{\omega}) + O(nnz(\sum\limits_{i=2}^{n^2-1}A_{i-1})n)  = O(|N|n^2n) = O(n^3).$$
\end{proof}



%%
%% The next two lines define the bibliography style to be used, and
%% the bibliography file.
\bibliographystyle{ACM-Reference-Format}
\bibliography{references}

\end{document}
\endinput
%%
%% End of file `sample-sigconf.tex'.
