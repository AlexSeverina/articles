\section{Cubic upper bound using sparse matrix multiplication}
In this section we show that the Algorithm ~\ref{CFPQMM} has complexity $O(n^3)$ if sparse matrix multiplication is used. The reason is that one needs to compute at most $|N|n^2$ realizable triples in total during all iterations of the algorithm, whereas the number of new triples found on each $i$-th iteration can be relatively small. The number of operations of the $i$-th iteration can be reduced by multiplying the matrix $B_{(i-2)}$ containing all the previously found triples, on the matrix $A_{(i-1)}$ which has only those triples firstly obtained in the $(i-1)$-th iteraition. In other words, we have:
\begin{equation}
\label{Bi}
B_i = B_{i-1} + A_i, 
\end{equation}
where
\begin{equation}
\label{Ai}
A_i = (B_{i-2}A_{i-1} + A_{i-1}B_{i-2} + A_{i-1}A_{i-1}) - B_{i-2}.
\end{equation}
So the following two conditions hold:
\begin{enumerate}
\item $\forall i, B_{(i-2)} \cap A_{(i-1)} = \emptyset$;
\item $\forall i, j,  A_i \cap A_j = \emptyset$.
\end{enumerate}
Notice that the first condition implies that one of the two multiplied matrices should be sparse, because $nnz(B_{(i-2)}) + nnz(A_{(i-1)}) \le |N|n^2$. Also, by the second condition matrices $A$ are pairwise disjoint, therefore $nnz(\sum\limits_{i=1}^{n^2}{A_i}) \le |N|n^2$. 


The Algorithm \ref{CFPQMM} can be modified using Equations \ref{Bi} and \ref{Ai} instead of the naive calculation of transitive closure on every iteration. It is important that the modified agorithm has the same number of iterations in the worst case as the original one --- $|N|n^2 = O(n^2)$. This is because the height of the parse tree does not exceed this value. As in the Algorithm~\ref{CFPQMM}, modified version derives new triples, going from leaves to the root of the parse tree.
\begin{theorem}
The matrix $B_{n^2}$ containing all possible realizable triples can be calculated in $O(n^3)$ time.
\end{theorem}
\begin{proof}
The correctness of the algorithm can be easily deduced from the correctness of the Algorithm \ref{CFPQMM} \cite{Azimov}. Now we show the cubic time complexity of the modified algorithm. Consider the equation for calculating $B_{n^2}$:
$$B_{n^2} = B_{n^2-1} + B_{i-2}A_{i-1} + A_{i-1}B_{i-2} + A_{i-1}A_{i-1} = $$
$$ = B_{n^2-2} + B_{i-3}A_{i-2} + A_{i-2}B_{i-3} + A_{i-2}A_{i-2} +$$ 
$$+ B_{i-2}A_{i-1} + A_{i-1}B_{i-2} + A_{i-1}A_{i-1} = ... =$$
$$= B_1 + B_1B_1 + \sum\limits_{i=2}^{n^2-1}B_{i-2}A_{i-1}+ \sum\limits_{i=2}^{n^2-1}A_{i-1}B_{i-2} + \sum\limits_{i=2}^{n^2-1}A_{i-1}A_{i-1}.$$
Suppose, without loss of generality, that the matrix $B_{i-2}$ is dense, than the matrix $A_{i-1}$ is sparse. Using naive sparse matrix multiplication algorithm, we have that $O(nnz(A_{i-1})n)$ operations are needed for multiplication of the matrix $B_{i-2}$ and the matrix $A_{i-1}$. Let $T(AB)$ be the number of operations that need to be performed to multiply matrices $A$ and $B$. Thus, the total number of operation for obtaining $B_{n^2}$ is in
$$ T(B_1B_1) + \sum\limits_{i=2}^{n^2-1}T(B_{i-2}A_{i-1})+ \sum\limits_{i=2}^{n^2-1}T(A_{i-1}B_{i-2}) + \sum\limits_{i=2}^{n^2-1}T(A_{i-1}A_{i-1})=$$
$$= O(n^{\omega}) +  O(\sum\limits_{i=2}^{n^2-1}nnz(A_{i-1})n) =$$
$$= O(n^{\omega}) + O(nnz(\sum\limits_{i=2}^{n^2-1}A_{i-1})n)  = O(|N|n^2n) = O(n^3).$$
\end{proof}