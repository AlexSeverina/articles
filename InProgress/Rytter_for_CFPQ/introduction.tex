\section{Introduction}

%The plan.
%\begin{itemize}
%\item Reduction of arbitrary CFPQ to Dyck query.
%\item Strongly-connected components handling
%\begin{itemize}
%  \item From all pairs reachcbility to single source reacability
%  \item Rytter for graph
%\end{itemize}
%
%\item Full graph processing
%\end{itemize}
%
%We provide an idea of two steps reduction of CFPQs to Boolean matrix multiplication.
%First step is reduction of arbitrary CFPQ to Dyck query.
%The next step is strongly connected components handling.
%This step is based on
%Second step is adaptation Rytter's results from~\cite{Rytter} for graph.
%We hope that such reduction helps to get algorithm for CFPQ with $\widetilde{O}(BMM(n))$ time complexity %where $\widetilde{O}$ meens polylog factors.
%
%Additionally we discuss ``fully algebraic'' view on CFPQ complexity which requires investigation of %noncommutative structures and matrix spaces over them.

The problem is to check the emptyness of the intersection of the regular language $R$ mhich is represented as FSA $A$ with number of states $n$, and context-free language $L$ in less then $O(n^3)$. The equvalent problem is a context-free reachability problem~\cite{Reps:1997:PAV:271338.271343}.

First step is a reduction of the given problem to BMM($n$) (with possibly polylogarithmic factors).
We hope that such reduction helps us to get algorithm for CFL reachability with $\widetilde{O}(BMM(n))$ time complexity where $\widetilde{O}$ meens polylog factors.
