\section{Algebraic View}

Steps for reduction of our problem to purely algebraic problem.
\begin{enumerate}
  \item Utilize Rytter's~\cite{Rytter} ideas to construct a grid graph $\mathcal{G}$. All are similar to the linear input parsing, with some detales.
  \begin{enumerate}
    \item We use states numbers instead of positions.
    \item To do it we should guarantee that state numbers are in $[0..n-1]$.
    \item As a result, grid graph can has cycles.
    \item Edges congruation property still holds.
  \end{enumerate}
\item We can see, that $\mathcal{G}$ is a Cartezian product of two graphs: $\mathcal{G}_H$ (a horisontal row of $\mathcal{G}$) and $\mathcal{G}_V$ (a vertical row of $\mathcal{G}$) with respective adjacency matrices.
Adjacency matrix of $\mathcal{G}$ is $M(\mathcal{G}) = M(\mathcal{G}_V) \otimes I + I \otimes M(\mathcal{G}_H)$ where $I$ is identity matrix of size $n \times n$ and $\otimes$ is a Kronecker product.
\item Instead of SSSP in the Rytter's algorithm we should compue APSP as a atomic step. We should to do it beacause there is now start position in the graph (FSA). Then we should proof that the number of such steps is $O(\log n)$.
Thus we want to compute $\text{vec}(X)*M(\mathcal{G})^k = \text{vec}(X)*[M(\mathcal{G}_V) \otimes I + I \otimes M(\mathcal{G}_H)]^k$. Where $X$ is a matrix of already proved facts, and $M(\mathcal{G})^k$ is a transitive closure of the adjacency matrix of the $\mathcal{G}$. Is it possible to do it in $\widetilde{O}(BMM(n))$?
\item Note that instead of $(B^T \otimes A) * \text{vec}(X) = \text{vec}(C)$ we can solve $A*X*B = C$ (one of fundamental properties of equitations with Kronecker product~\cite{schacke2004kronecker}).
The idea is to use this property.
In our case it helps to reduce multiplication of $n^2 \times n^2$ matrices to multiplication of $n \times n$ matrices.
\textbf{But} multiplication in our semiring is noncommutative.
Namely, weights are from noncommutative idempotatnt semiring.
So we need to investigate properties of Kronecker product over such semiring.
Related research by Thomas Reps: ``Newtonian Program Analysis via Tensor Product''~\cite{Reps:2017:NPA:3062396.3024084}
\end{enumerate}
