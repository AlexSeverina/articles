\section{Context-free Path Querying by Transitive Closure Calculation}
\label{section_main}

In this section, we show, how the context-free path query evaluation using relational query semantics can be reduced to the calculation of matrix transitive closure $a^{cf}$, prove the correctness of this reduction, introduce an algorithm for computing the transitive closure $a^{cf}$, and provide a step-by-step demonstration of this algorithm on a small example.

\subsection{Reducing Context-Free Path Querying to the Calculation of Transitive Closure} \label{section_reducing}

In this section, we show, how the context-free relations $R_A$ can be calculated by computing the transitive closure $a^{cf}$.

Let $G = (N,\Sigma,P)$ be a grammar and $D = (V, E)$ be a graph. We enumerate the nodes of the graph $D$ from 0 to $(|V| - 1)$. We initialize the elements of the $|V| \times |V|$ matrix $a$ with $\varnothing$. Further, for every $i$ and $j$ we set $$a_{i,j} = \{A_k~|~((i,x,j) \in E) \wedge ((A_k \rightarrow x) \in P)\}.$$ Finally, we compute the transitive closure $$a^{cf} = a^{(1)} \cup a^{(2)} \cup \cdots$$ where $$a^{(i)} = a^{(i-1)} \cup (a^{(i-1)} \times a^{(i-1)}),$$ for $i \ge 2$ and $a^{(1)} = a$. For the transitive closure $a^{cf}$, the following statements hold.

\begin{lemma}\label{lemma:cf}
Let $D = (V,E)$ be a graph, let $G =(N,\Sigma,P)$ be a grammar. Then for any $i, j$ and for any non-terminal $A \in N$, $A \in a^{(k)}_{i,j}$ iff $(i,j) \in R_A$ and $i \pi j$, such that there is a derivation tree of the height $h \leq k$ for the string $l(\pi)$ and a context-free grammar $G_A = (N,\Sigma,P,A)$.
\end{lemma}
\begin{proof}(Proof by Induction)

\textbf{Base case}: Show that the lemma holds for $k = 1$. For any $i, j$ and for any non-terminal $A \in N$, $A \in a^{(1)}_{i,j}$ iff there is $i \pi j$ that consists of a unique edge $e$ from the node $i$ to the node $j$ and $(A \rightarrow x) \in P$, where $x = l(\pi)$. Therefore $(i,j) \in R_A$ and there is a derivation tree of the height $h = 1$, shown on Figure~\ref{tree1}, for the string $x$ and a context-free grammar $G_A = (N,\Sigma,P,A)$. Thus, it has been shown that the lemma holds for $k = 1$.

\begin{figure}[h!]
	\centering
	\begin{tikzpicture}
	\begin{scope}[every node/.style={circle,thick,draw,minimum size=0.8cm}]
	\node (0) at (0,1) {A};
	\node (1) at (0,-1) {x};
	\end{scope}
	
	\begin{scope}[>={Stealth[black]},
	every node/.style={fill=white,circle},
	every edge/.style={draw=black,very thick}]
	\path [->] (0) edge (1);
	\end{scope}
	\end{tikzpicture}
	\caption{The derivation tree of the height $h = 1$ for the string $x = l(\pi)$.}
	\label{tree1}
\end{figure}

\textbf{Inductive step}: Assume that the lemma holds for any $k \leq (p - 1)$ and show that it also holds for $k = p$, where $p \geq 2$. For any $i, j$ and for any non-terminal $A \in N$, $$A \in a^{(p)}_{i,j} \text{ iff } A \in a^{(p-1)}_{i,j} \text{ or } A \in (a^{(p-1)} \times a^{(p-1)})_{i,j},$$ since $$a^{(p)} = a^{(p-1)} \cup (a^{(p-1)} \times a^{(p-1)}).$$

Let $A \in a^{(p-1)}_{i,j}$. By the inductive hypothesis, $A \in a^{(p-1)}_{i,j}$ iff $(i,j) \in R_A$ and there exists $i \pi j$, such that there is a derivation tree of the height $h \leq (p-1)$ for the string $l(\pi)$ and a context-free grammar $G_A = (N,\Sigma,P,A)$. The statement of the lemma holds for $k = p$ since the height $h$ of this tree is also less than or equal to $p$.

Let $A \in (a^{(p-1)} \times a^{(p-1)})_{i,j}$. By the definition of the binary operation $(\cdot)$ on arbitrary subsets, $A \in (a^{(p-1)} \times a^{(p-1)})_{i,j}$ iff there are $r$, $B \in a^{(p-1)}_{i,r}$ and $C \in a^{(p-1)}_{r,j}$, such that $(A \rightarrow B C) \in P$. Hence, by the inductive hypothesis, there are $i \pi_1 r$ and $r \pi_2 j$, such that $(i,r) \in R_B$ and $(r,j) \in R_C$, and there are the derivation trees $T_B$ and $T_C$ of heights $h_1 \leq (p-1)$ and $h_2 \leq (p-1)$ for the strings $w_1 = l(\pi_1)$, $w_2 = l(\pi_2)$ and the context-free grammars $G_B$, $G_C$ respectively. Thus, the concatenation of paths $\pi_1$ and $\pi_2$ is $i \pi j$, where $(i,j) \in R_A$ and there is a derivation tree of the height $h = 1 + max(h_1, h_2)$, shown on Figure~\ref{tree2}, for the string $w = l(\pi)$ and a context-free grammar $G_A$.

\begin{figure}[h!]
	\centering
	\begin{tikzpicture}
	\begin{scope}[every node/.style={circle,thick,draw,minimum size=0.8cm}]
	\node (0) at (0,1) {A};
	\end{scope}
	
	\begin{scope}[every node/.style={regular polygon, regular polygon sides=3,thick,draw}]
	\node (1) at (-1,-1) {$T_B$};
	\node (2) at (1,-1) {$T_C$};
	\end{scope}
	
	\begin{scope}[>={Stealth[black]},
	every node/.style={fill=white,circle},
	every edge/.style={draw=black,very thick}]
	\path [->] (0) edge (1);
	\path [->] (0) edge (2);
	\end{scope}
	\end{tikzpicture}
	\caption{The derivation tree of the height $h = 1 + max(h_1, h_2)$ for the string $w = l(\pi)$, where $T_B$ and $T_C$ are the derivation trees for strings $w_1$ and $w_2$ respectively.}
	\label{tree2}
\end{figure}

The statement of the lemma holds for $k = p$ since the height $h = 1 + max(h_1, h_2) \leq p$. This completes the proof of the lemma.
\end{proof}

\begin{mytheorem}\label{thm:correct}
 Let $D = (V,E)$ be a graph and let $G =(N,\Sigma,P)$ be a grammar. Then for any $i, j$ and for any non-terminal $A \in N$, $A \in a^{cf}_{i,j}$ iff $(i,j) \in R_A$.
\end{mytheorem}
\begin{proof}

Since the matrix $a^{cf} = a^{(1)} \cup a^{(2)} \cup \cdots,$ for any $i, j$ and for any non-terminal $A \in N$, $A \in a^{cf}_{i,j}$ iff there is $k \geq 1$, such that $A \in a^{(k)}_{i,j}$. By the lemma~\ref{lemma:cf}, $A \in a^{(k)}_{i,j}$ iff $(i,j) \in R_A$ and there is $i \pi j$, such that there is a derivation tree of the height $h \leq k$ for the string $l(\pi)$ and a context-free grammar $G_A = (N,\Sigma,P,A)$. This completes the proof of the theorem.
\end{proof}

We can, therefore, determine whether $(i,j) \in R_A$ by asking whether $A \in a^{cf}_{i,j}$. Thus, we show how the context-free relations $R_A$ can be calculated by computing the transitive closure $a^{cf}$ of the matrix $a$.



\subsection{The algorithm} \label{section_algorithm}
In this section, we introduce an algorithm for calculating the transitive closure $a^{cf}$ which was discussed in section~\ref{section_reducing}.

Let $D = (V, E)$ be the input graph and $G = (N,\Sigma,P)$ be the input grammar.

\begin{algorithm}[H]
\begin{algorithmic}[1]
\caption{Context-free recognizer for graphs}
\label{alg:graphParse}
\Function{contextFreePathQuerying}{D, G}
    
    \State{$n \gets$ the number of nodes in $D$}
    \State{$E \gets$ the directed edge-relation from $D$}
    \State{$P \gets$ the set of production rules in $G$}
    \State{$T \gets$ the matrix $n \times n$ in which each element is $\varnothing$}
    \ForAll{$(i,x,j) \in E$}
    \Comment{Matrix initialization}
        \State{$T_{i,j} \gets T_{i,j} \cup \{A~|~(A \rightarrow x) \in P \}$}
    \EndFor    
    \While{matrix $T$ is changing}
       
        \State{$T \gets T \cup (T \times T)$}
        \Comment{Transitive closure $T^{cf}$ calculation} 
    \EndWhile
\State \Return $T$
\EndFunction
\end{algorithmic}
\end{algorithm}

Note, the matrix initialization in lines \textbf{6-7} of the Algorithm~\ref{alg:graphParse} can handle arbitrary graph $D$. For example, if a graph $D$ contains multiple edges $(i,x_1,j)$ and $(i,x_2,j)$ then both the elements of the set $\{A_1~|~(A_1 \rightarrow x_1) \in P \}$ and the elements of the set $\{A_2~|~(A_2 \rightarrow x_2) \in P \}$ will be added to $T_{i,j}$.

We need to show that the Algorithm~\ref{alg:graphParse} terminates in a finite number of steps. Since each element of the matrix $T$ contains no more than $|N|$ non-terminals, the total number of non-terminals in the matrix $T$ does not exceed $|V|^2|N|$. Therefore, the following theorem holds.

\begin{mytheorem}\label{thm:finite}
 Let $D = (V,E)$ be a graph and let $G =(N,\Sigma,P)$ be a grammar. The Algorithm~\ref{alg:graphParse} terminates in a finite number of steps. 
\end{mytheorem}
\begin{proof}
It is sufficient to show, that the operation in the line \textbf{9} of the Algorithm~\ref{alg:graphParse} changes the matrix $T$ only finite number of times. Since this operation can only add non-terminals to some elements of the matrix $T$, but not remove them, it can change the matrix $T$ no more than $|V|^2|N|$ times.
\end{proof}

Denote the number of elementary operations executed by the algorithm of multiplying two $n \times n$ Boolean matrices as $\text{\it{BMM}}(n)$. According to Valiant, the matrix multiplication operation in the line \textbf{9} of the Algorithm~\ref{alg:graphParse} can be calculated in $O(|N|^2 \text{\it{BMM}}(n)(|V|))$. Denote the number of elementary operations, executed by the matrix union operation of two $n \times n$ Boolean matrices as $\text{\it{BMU}}(n)$. Similarly, it can be shown that the matrix union operation in the line \textbf{9} of the Algorithm~\ref{alg:graphParse} can be calculated in $O(|N|^2 \text{\it{BMU}}(n))$. Since the line \textbf{9} of the Algorithm~\ref{alg:graphParse} is executed no more than $|V|^2|N|$ times, the following theorem holds.

\begin{mytheorem}\label{thm:time}
Let $D = (V,E)$ be a graph and let $G =(N,\Sigma,P)$ be a grammar. Then the Algorithm~\ref{alg:graphParse} calculates the transitive closure $T^{cf}$ in \\ $O(|V|^2|N|^3(\text{\it{BMM}}(|V|) + \text{\it{BMU}}(|V|))).$
\end{mytheorem}

We also find the worst-case example taken from~\cite{hellingsPathQuerying}, for which the time complexity in terms of the graph size provided by Theorem~\ref{thm:time} cannot be improved. This example is based on the context-free grammar $G = (N, \Sigma, P)$ where:
\begin{itemize}
	\item the set of non-terminals $N = \{S\}$;
	\item the set of terminals $\Sigma = \{a, b\}$;
	\item the set of production rules $P$ is presented on Figure~\ref{ProductionRulesWorsCaseExample}.
\end{itemize}

\begin{figure}[h]
	\[
	\begin{array}{rccl}
	0: & S & \rightarrow & \text{\emph{a}} \ S \ \text{\emph{b}} \\
	1: & S & \rightarrow & \text{\emph{a}} \ \text{\emph{b}} \\ 
	\end{array}
	\]
	\caption{Production rules for the worst-case example.}
	\label{ProductionRulesWorsCaseExample}
\end{figure}

Let the size $|N|$ of the grammar $G$ be a constant. The worst-case time complexity is reached by running this query on the double-cyclic graph where:
\begin{itemize}
	\item one of the cycles having $u = 2^k + 1$ edges labeled with $a$;
	\item another cycle having $v = 2^k$ edges labeled with $b$;
	\item the two cycles are connected via a shared node $m$.
\end{itemize}

A small example of such graph with $k = 1$, $u = 3$, $v = 2$, and $m = 0$ is presented on Figure~\ref{Example_Graph}.

The shortest path $\pi$ from the node $m$ to the node $m$, whose labeling forms a string from the language $L(G,S)=\{a^n b^n| n \geq 1\}$, has a length $l = 2*u*v$, since $u = 2^k + 1$ and $v = 2^k$ are coprime, and string $s$, formed by this path, consists of $u*v$ labels $a$ and $u*v$ labels $b$. The string $s = l(\pi)$ has a derivation tree, according to a context-free grammar $G$ and starting nonterminal $S$, of the minimal height $h = 2*u*v$ among all the paths from the node $m$ to the node $m$ in this double-cyclic graph. Therefore, if we run the worst-case example query on this graph, then the operation in the line \textbf{9} of the Algorithm~\ref{alg:graphParse} changes the matrix $T$ at least $h = 2*u*v$ times. Hence, the Algorithm~\ref{alg:graphParse} computes this query in $O(|V|^2(\text{\it{BMM}}(|V|) + \text{\it{BMU}}(|V|)))$, since $|V| = (u + v - 1) = 2*v$ and $h = 2*u*v > 2*v*v = |V|^2 / 4 = O(|V|^2)$.

\subsection{An Example}
\label{section_example}

In this section, we provide a step-by-step demonstration of the proposed algorithm. For this, we consider the example with the worst-case time complexity.

The \textbf{example query} is based on the context-free grammar $G = (N, \Sigma, P)$ of the worst-case example query which was discussed in section~\ref{section_algorithm}. The set of production rules for this grammar is shown on Figure~\ref{ProductionRulesWorsCaseExample}.

Since the proposed algorithm processes only grammars in Chomsky normal form, we first transform the grammar $G$ into an equivalent grammar $G' = (N', \Sigma', P')$ in normal form, where:
\begin{itemize}
    \item the set of non-terminals $N' = \{S, S_1, A, B\}$;
    \item the set of terminals $\Sigma' = \{a, b\}$;
    \item the set of production rules $P'$ is presented on Figure~\ref{ProductionRulesExampleQueryCNF}.
\end{itemize}

\begin{figure}[h]
   \[
\begin{array}{rccl}
   0: & S & \rightarrow & A \ B \\
   1: & S & \rightarrow & A \ S_1 \\
   2: & S_1 & \rightarrow & S \ B \\
   3: & A & \rightarrow & \text{\emph{a}} \\ 
   4: & B & \rightarrow & \text{\emph{b}} \\ 
\end{array}
\]
\caption{Production rules for the example query grammar in normal form.}
\label{ProductionRulesExampleQueryCNF}
\end{figure}

We run the query on a graph, presented on Figure~\ref{Example_Graph}. We provide a step-by-step demonstration of the work with the given graph $D$ and grammar $G'$ of the Algorithm~\ref{alg:graphParse}. After the matrix initialization in lines \textbf{6-7} of the Algorithm~\ref{alg:graphParse}, we have a matrix $T_0$, presented on Figure~\ref{ExampleQueryInitMatrix}.

\begin{figure}[h]
\[
T_0 = \begin{pmatrix}
    \varnothing & \{A\}       & \varnothing & \{B\}       \\
    \varnothing & \varnothing & \{A\}       & \varnothing \\
    \{A\}       & \varnothing & \varnothing & \varnothing \\
    \{B\}       & \varnothing & \varnothing & \varnothing \\
\end{pmatrix}
\]
\caption{The initial matrix for the example query.}
\label{ExampleQueryInitMatrix}
\end{figure}

Let $T_i$ be the matrix $T$, obtained after executing the loop in the lines \textbf{8-9} of the Algorithm~\ref{alg:graphParse} $i$ times. The calculation of the matrix $T_1$ is shown on Figure~\ref{ExampleQueryFirstIteration}.

\begin{figure}[h]
\[
T_0 \times T_0 = \begin{pmatrix}
	\varnothing & \varnothing & \varnothing & \varnothing \\
	\varnothing & \varnothing & \varnothing & \varnothing \\
	\varnothing & \varnothing & \varnothing & \{S\}       \\
	\varnothing & \varnothing & \varnothing & \varnothing \\
\end{pmatrix}
\]

\[
T_1 = T_0 \cup (T_0 \times T_0) = \begin{pmatrix}
	\varnothing & \{A\}       & \varnothing & \{B\}       \\
	\varnothing & \varnothing & \{A\}       & \varnothing \\
	\{A\}       & \varnothing & \varnothing & \{\pmb{S}\}       \\
	\{B\}       & \varnothing & \varnothing & \varnothing \\
\end{pmatrix}
\]
\caption{The first iteration of computing the transitive closure for the example query.}
\label{ExampleQueryFirstIteration}
\end{figure}

When the algorithm at some iteration finds new paths in the graph $D$, then it adds corresponding nonterminals to the matrix $T$. For example, after the first loop iteration, non-terminal $S$ is added to the matrix $T$. This non-terminal is added to the element with a row index $i = 2$ and a column index $j = 3$. This means, that there is $i\pi j$ (a path $\pi$ from the node 2 to the node 3), such that $S \xrightarrow{*} l(\pi)$. For example, such a path consists of two edges with labels $a$ and $b$, and $S \xrightarrow{0} A \ B \xrightarrow{3} a \ B \xrightarrow{4} a \ b$.

The calculation of the transitive closure is completed after $k$ iterations, when a fixpoint is reached: $T_{k-1} = T_k$. For the example query, $k = 13$ since $T_{13} = T_{12}$. The remaining iterations of computing the transitive closure are presented on Figure~\ref{ExampleQueryFinalIterations} (new matrix elements on each iteration are shown in bold).

\begin{figure*}[ht]
\begin{alignat*}{5}
& &&T_2 &&= \begin{pmatrix}
\varnothing & \{A\}       & \varnothing & \{B\}       \\
\varnothing & \varnothing & \{A\}       & \varnothing \\
\{A, \pmb{S_1}\}  & \varnothing & \varnothing & \{S\}       \\
\{B\}       & \varnothing & \varnothing & \varnothing \\
\end{pmatrix} \ \ \ \ &&T_3 &&= \begin{pmatrix}
\varnothing & \{A\}       & \varnothing & \{B\}       \\
\{\pmb{S}\}       & \varnothing & \{A\}       & \varnothing \\
\{A, S_1\}  & \varnothing & \varnothing & \{S\}       \\
\{B\}       & \varnothing & \varnothing & \varnothing \\
\end{pmatrix} \\
& &&T_4 &&= \begin{pmatrix}
\varnothing & \{A\}       & \varnothing & \{B\}       \\
\{S\}       & \varnothing & \{A\}       & \{\pmb{S_1}\}     \\
\{A, S_1\}  & \varnothing & \varnothing & \{S\}       \\
\{B\}       & \varnothing & \varnothing & \varnothing \\
\end{pmatrix} \ \ \ \ &&T_5 &&= \begin{pmatrix}
\varnothing & \{A\}       & \varnothing & \{B, \pmb{S}\}    \\
\{S\}       & \varnothing & \{A\}       & \{S_1\}     \\
\{A, S_1\}  & \varnothing & \varnothing & \{S\}       \\
\{B\}       & \varnothing & \varnothing & \varnothing \\
\end{pmatrix} \\
& &&T_6 &&= \begin{pmatrix}
\{\pmb{S_1}\}     & \{A\}       & \varnothing & \{B, S\}    \\
\{S\}       & \varnothing & \{A\}       & \{S_1\}     \\
\{A, S_1\}  & \varnothing & \varnothing & \{S\}       \\
\{B\}       & \varnothing & \varnothing & \varnothing \\
\end{pmatrix} \ \ \ \ &&T_7 &&= \begin{pmatrix}
\{S_1\}     & \{A\}       & \varnothing & \{B, S\}    \\
\{S\}       & \varnothing & \{A\}       & \{S_1\}     \\
\{A, S_1, \pmb{S}\}  & \varnothing & \varnothing & \{S\}    \\
\{B\}       & \varnothing & \varnothing & \varnothing \\
\end{pmatrix}  \\
& &&T_8 &&= \begin{pmatrix}
\{S_1\}     & \{A\}       & \varnothing & \{B, S\}    \\
\{S\}       & \varnothing & \{A\}       & \{S_1\}     \\
\{A, S_1, S\}  & \varnothing & \varnothing & \{S, \pmb{S_1}\} \\
\{B\}       & \varnothing & \varnothing & \varnothing \\
\end{pmatrix} \ \ \ \ &&T_9 &&= \begin{pmatrix}
\{S_1\}     & \{A\}       & \varnothing & \{B, S\}    \\
\{S\}       & \varnothing & \{A\}       & \{S_1, \pmb{S}\}     \\
\{A, S_1, S\}  & \varnothing & \varnothing & \{S, S_1\} \\
\{B\}       & \varnothing & \varnothing & \varnothing \\
\end{pmatrix} \\
& &&T_{10} &&= \begin{pmatrix}
\{S_1\}     & \{A\}       & \varnothing & \{B, S\}    \\
\{S, \pmb{S_1}\}       & \varnothing & \{A\}       & \{S_1, S\}     \\
\{A, S_1, S\}  & \varnothing & \varnothing & \{S, S_1\} \\
\{B\}       & \varnothing & \varnothing & \varnothing \\
\end{pmatrix} \ \ \ \ &&T_{11} &&= \begin{pmatrix}
\{S_1, \pmb{S}\}     & \{A\}       & \varnothing & \{B, S\}    \\
\{S, S_1\}       & \varnothing & \{A\}       & \{S_1, S\}     \\
\{A, S_1, S\}  & \varnothing & \varnothing & \{S, S_1\} \\
\{B\}       & \varnothing & \varnothing & \varnothing \\
\end{pmatrix} \\
& &&T_{12} &&= \begin{pmatrix}
\{S_1, S\}     & \{A\}       & \varnothing & \{B, S, \pmb{S_1}\}    \\
\{S, S_1\}       & \varnothing & \{A\}       & \{S_1, S\}     \\
\{A, S_1, S\}  & \varnothing & \varnothing & \{S, S_1\} \\
\{B\}       & \varnothing & \varnothing & \varnothing \\
\end{pmatrix} \ \ \ \ &&T_{13} &&= \begin{pmatrix}
\{S_1, S\}     & \{A\}       & \varnothing & \{B, S, S_1\}    \\
\{S, S_1\}       & \varnothing & \{A\}       & \{S_1, S\}     \\
\{A, S_1, S\}  & \varnothing & \varnothing & \{S, S_1\} \\
\{B\}       & \varnothing & \varnothing & \varnothing \\
\end{pmatrix}
\end{alignat*}
\caption{Remaining states of the matrix $T$.}
\label{ExampleQueryFinalIterations}
\end{figure*}

Thus, the result of the Algorithm~\ref{alg:graphParse} for the example query is the matrix $T_{13} = T_{12}$. Now, after constructing the transitive closure, we can construct the context-free relations $R_A$. These relations for each non-terminal of the grammar $G'$ are presented on Figure~\ref{ExampleQueryCFRelations}. The example graph with the result of the context-free path querying is presented on Figure~\ref{FinalExampleGraph}.

\begin{figure}[h]
\begin{eqnarray*}
R_S&=&\{(0,0),(0,3),(1,0),(1,3),(2,0),(2,3)\},\\
R_{S_1}&=&\{(0,0),(0,3),(1,0),(1,3),(2,0),(2,3)\},\\
R_{A}&=&\{(0,1),(1,2),(2,0)\}, \\
R_{B}&=&\{(0,3), (3,0)\}.
\end{eqnarray*}
\caption{Context-free relations for the example query.}
\label{ExampleQueryCFRelations}
\end{figure}

\begin{figure}[h]
	\centering
	\begin{tikzpicture}
	\begin{scope}[every node/.style={circle,thick,draw,minimum size=0.8cm}]
	\node (0) at (0,0) {0};
	\node (1) at (-7,0) {1};
	\node (2) at (-3.5,1.5) {2};
	\node (3) at (4,0) {3};
	\end{scope}
	
	\begin{scope}[>={Stealth[black]},
	every node/.style={fill=white,circle},
	every edge/.style={draw=black,very thick}]
	\path [->] (0) edge[bend left=30] node {$a$} (1);
	\path [->] (1) edge node {$a$} (2);
	\path [->] (2) edge node {$a$} (0);
	\path [->] (0) edge[bend left=60] node {$b$} (3);
	\path [->] (3) edge[bend left=60] node {$b$} (0);
	\end{scope}
	
	\begin{scope}[>={Stealth[black]},
	every node/.style={fill=white,circle},
	every edge/.style={draw=black,very thick,dashed}]
	\path [->] (0) edge[loop below] node {$S,S_1$} (0);
	\path [->] (0) edge[bend left=60] node {$A$} (1);
	\path [->] (0) edge[bend left=10] node {$S,S_1,B$} (3);
	\path [->] (1) edge node {$S,S_1$} (0);
	\path [->] (1) edge[bend left=30] node {$A$} (2);
	\path [->] (1) edge[bend left=90] node {$S,S_1$} (3);
	\path [->] (2) edge[bend left=45] node {$S,S_1,A$} (0);
	\path [->] (2) edge[bend left=60] node {$S,S_1$} (3);
	\path [->] (3) edge[bend left=10] node {$B$} (0);
	\end{scope}
	\end{tikzpicture}
	\caption{The example graph with the result of CFPQ.}
	\label{FinalExampleGraph}
\end{figure}

In the context-free relation $R_S$, we have all node pairs corresponding to paths, whose labeling is in the language $L(G,S) = \{a^n b^n| n \geq 1\}$. This conclusion is based on the fact, that the grammar $G'$ with the starting nonterminal $S$ is equivalent to the grammar $G$ with the same starting nonterminal and $L(G',S) = L(G,S)$.