\section{Related Work}
\label{section_related}
Traditionally, query languages for graph databases use regular expressions to describe paths to find~\cite{reutter2017regular,fan2011adding,abiteboul1997regular,nole2016regular,graphDB}, but there are some other useful queries, which cannot be expressed by regular expressions. For example, there are classical \emph{same-generation queries}~\cite{FndDB}, which can be used for finding all the nodes at the same level in some hierarchy, and are useful for discovering vertex similarity. The context-free path querying algorithms can be used to evaluate such types of queries since this queries can be represented by context-free grammars.  

In~\cite{kuijpers2019experimental}, the comparative experimental investigation of several state of the art context-free path query evaluation methods w.r.t. relational query semantics is provided.

There are a number of solutions~\cite{hellingsRelational,GraphQueryWithEarley,RDF} for context-free path query evaluation w.r.t. relational query semantics, which make use of such parsing algorithms as CYK~\cite{kasami,younger} or Earley~\cite{Grune}.

Hellings~\cite{hellingsRelational} presented an algorithm for context-free path query evaluation using relational query semantics. According to Hellings, for a given graph \mbox{$D = (V, E)$} and a grammar $G = (N, \Sigma, P)$ the context-free path query evaluation w.r.t. relational query semantics reduces to a calculation of a set of context-free relations $R_A$. Thus, in this paper, we focus on the calculation of these context-free relations. The worst-case time complexity of this algorithm is $O(|N||E| + (|N||V|)^3)$. Additionally, the algorithm of Hellings was implemented~\cite{RDF} in the context of RDF processing.

In~\cite{santos2018bottom}, a bottom-up algorithm for a context-free path querying over graph databases is presented. This algorithm is inspired by the LR~\cite{aho2007compilers} parsing technique and uses a variant of the GSS introduced in~\cite{tomita1987efficient} to encompass several derivations at a time. Also, this algorithm was implemented in the context of RDF processing.

An algorithm for the context-free path query evaluation and its implementation over RDF graph databases are proposed in~\cite{medeiros2019ll}. This algorithm is inspired by the LL~\cite{Grune} parsing technique, has $O(|V|^3|P|)$ worst-case time complexity, and has $O(|V|^2|N|)$ worst-case space complexity, where $V$ is the set of vertices, $N$ is a set of non-terminal symbols, and $P$ is a set of production (or derivation) rules of the context-free grammar.

In~\cite{bradford2017efficient} an algorithm for context-free path query evaluation with Dyck and semi-Dyck grammars is presented. Dyck and semi-Dyck context-free languages are important due to the close relationship between transitive closure, Boolean and algebraic matrix multiplication, and context-free grammar recognition. This algorithm has $O(|V|^{\omega}log^3(|V|))$ time complexity where $\omega$ is the best exponent for matrix multiplication.

Given any two vertices $s$ and $t$, and the output of Nyk\"anen and Ukkonen's~\cite{nykanen2002exact} exact integer path length algorithm that costs $O(n^3)$. An algorithm in~\cite{bradford2016fast} gives a minimal-cost point-to-point Dyck shortest path result in $O(n^2 log(n))$ additional operations. This paper assumes the graphs have no cycles and the Dyck languages have a single parenthesis type.

Firstly, the paper~\cite{chatterjee2017optimal} addresses the problem of context-free path querying w.r.t. relation semantics with Dyck languages on bidirected graphs. Given a bidirected graph with $|V|$ nodes and $|E|$ edges: (i) an algorithm with worst-case running time $O(|E| + |V| \cdot \alpha(|V|))$ is presented, where $\alpha(|V|)$ is the inverse Ackermann function; (ii) provided a matching lower bound that shows that this algorithm is optimal w.r.t. worst-case complexity; and (iii) presented an optimal average-case upper bound of $O(|E|)$ time. Secondly, presented the proof that combinatorial algorithms for context-free path querying with Dyck languages on general graphs with truly sub-cubic bounds cannot be obtained without obtaining sub-cubic combinatorial algorithms for Boolean Matrix Multiplication, which is a long-standing open problem. This means that the existing combinatorial algorithms for context-free path querying with Dyck languages are (conditionally) optimal for general graphs.

Other examples of path query semantics are \emph{single-path} and \emph{all-path query semantics}~\cite{hellingsPathQuerying}. The all-path query semantics requires a finding of all possible paths from a node $m$ to a node $n$ whose labelings are derived from a non-terminal $A$. The single-path query semantics requires presenting only one such path (usually, the shortest one). 

Hellings~\cite{hellingsPathQuerying} presented some algorithms for context-free path query evaluation using single-path and all-path query semantics. If a context-free path query w.r.t. all-path query semantics is evaluated for cyclic graphs, then the query result can be an infinite set of paths. For this reason, in~\cite{hellingsPathQuerying} annotated grammars were proposed as a way to represent the results.

In~\cite{ward2008distributed} a distributed context-free path query evaluation algorithm w.r.t. the single-path query semantics is proposed. This algorithm provides a shortest path between each pair of nodes according to the context-free grammar and a weight function on graph edges. This algorithm can be efficiently distributed on up to $O(|V||N|)$ compute nodes.

An algorithm for context-free path query evaluation w.r.t. all-path query semantics is proposed in~\cite{GLL}. This algorithm is based on the generalized top-down parsing algorithm (GLL)~\cite{scott2010gll}. For the result representation, this solution uses derivation trees, which is more native for grammar-based analysis. The worst-case time complexity of this algorithm is $O(|V|^3 max_{v\in V} (deg^+(v)))$, where $deg^+(v)$ is the outdegree of vertex $v$. For complete graphs, the time complexity of this algorithm can reach $O(|V|^4)$.

The algorithms~\cite{GLL,hellingsPathQuerying} for context-free path query evaluation w.r.t. single-path and all-path query semantics can also be used for query evaluation using relational semantics.

Our work is inspired by Valiant~\cite{valiant}, who proposed an algorithm for general context-free recognition in less than cubic time. This algorithm computes the same parsing table as CYK algorithm but does this by offloading the most intensive computations into calls to the Boolean matrix multiplication procedure. This approach not only provides an asymptotically more efficient algorithm but also allows us to effectively apply GPGPU computing techniques. Valiant's algorithm computes the transitive closure $a^+$ of a square upper-triangular matrix $a$. Valiant also has shown, that the matrix multiplication operation $(\times)$ is essentially the same as $|N|^2$ Boolean matrix multiplications, where $|N|$ is the number of non-terminals in the given context-free grammar in Chomsky normal form.

Yannakakis~\cite{transitive-closure} analyzed the reducibility of various path querying problems to the calculation of transitive closure. He formulated a problem of Valiant's technique generalization for the context-free path query evaluation w.r.t. relational query semantics. Also, he conjectured, that this technique cannot be generalized for arbitrary graphs, though it does for acyclic graphs.
