\section{Preliminaries}
\label{section_preliminaries}

\subsection{Context-Free Path Querying}
Let $\Sigma$ be a finite set of edge labels. Define an \emph{edge-labeled directed graph} as a tuple \mbox{$D = (V, E)$} with a set of nodes $V$ and a directed edge relation \mbox{$E \subseteq V \times \Sigma \times V$}.  For a path $\pi$ in a graph $D$, we denote the unique word, obtained by concatenating the labels of the edges along the path $\pi$ as \mbox{$l(\pi)$}. Also, we write \mbox{$n \pi m$} to indicate, that the path $\pi$ starts at the node \mbox{$n \in V$} and ends at the node \mbox{$m \in V$}.

Following Hellings~\cite{hellingsRelational}, we deviate from the usual definition of a context-free grammar in \emph{Chomsky Normal Form}~\cite{chomsky} by not including a special starting non-terminal, which will be specified in the path queries for the graph. Since every context-free grammar can be transformed into an equivalent one in Chomsky Normal Form and checking, that empty string belongs to the language is trivial, it is sufficient to consider only grammars of the following type.

A \emph{context-free grammar} is a triple \mbox{$G = (N, \Sigma, P)$}, where $N$ is a finite set of non-terminals, $\Sigma$ is a finite set of terminals, and $P$ is a finite set of productions of the following forms:

\begin{itemize}
    \item $A \rightarrow B C$, for $A,B,C \in N$,
    \item $A \rightarrow x$, for $A \in N$ and $x \in \Sigma$.   
\end{itemize}

Note that we omit the rules of the form \mbox{$A \rightarrow \varepsilon$}, where $\varepsilon$ denotes empty string. This does not restrict the applicability of our algorithm since only the empty paths \mbox{$m \pi m$} correspond to empty string $\varepsilon$.

We use the conventional notation \mbox{$A \xrightarrow{*} w$} to denote, that a string \mbox{$w \in \Sigma^*$} can be derived from a non-terminal $A$ by some sequence of production rule applications from $P$. The \emph{language} of a grammar \mbox{$G = (N,\Sigma,P)$} with respect to a start non-terminal \mbox{$S \in N$} is defined by $$L(G,S) = \{w \in \Sigma^*~|~S \xrightarrow{*} w\}.$$

Rules in our grammars will be enumerated. When we want to specify the number $i$ of the applied rule, we use the conventional notation \mbox{$w_1 \xrightarrow{i} w_2$} where \mbox{$w_1,w_2 \in (N \cup \Sigma)^*$}.

For a given graph \mbox{$D = (V, E)$} and a context-free grammar $G = (N, \Sigma, P)$, we define \emph{context-free relations} \mbox{$R_A \subseteq V \times V$} for every \mbox{$A \in N$}, such that $$R_A = \{(n,m)~|~\exists n \pi m~(l(\pi) \in L(G,A))\}.$$

In order to find these context-free relations, we use a semiring without associativity of the product over the subsets of the set of nonterminals $N$. It consists of a product operation $(~\cdot~) : 2^N \times 2^N \rightarrow 2^N$ and the union $(\cup) : 2^N \times 2^N \rightarrow 2^N$ as an associative and commutative sum operation. Also, the sum is distributive over the product.

We define a product operation $(~\cdot~)$ for arbitrary subsets \mbox{$N_1, N_2$} of $N$ with respect to a context-free grammar \mbox{$G = (N, \Sigma, P)$} as $$N_1 \cdot N_2 = \{A~|~\exists B \in N_1, \exists C \in N_2 \text{ such that }(A \rightarrow B C) \in P\}.$$

Using this semiring, we can define a \emph{matrix multiplication}, \mbox{$a \times b = c$}, where $a$ and $b$ are matrices of size $n\times n$, that have subsets of $N$ as elements, as $$c_{i,j} = \bigcup^{n}_{k=1}{a_{i,k} \cdot b_{k,j}}.$$

Also, we use the element-wise union operation on matrices $a$ and $b$ with the same size: \mbox{$a \cup b = c$}, where $c_{i,j} = a_{i,j} \cup b_{i,j}.$

According to Valiant~\cite{valiant}, we define the \emph{transitive closure} of a square matrix $a$ as \mbox{$a_+ = a^{(1)}_+ \cup a^{(2)}_+ \cup \cdots$}, where \mbox{$a^{(1)}_+ = a$} and $$a^{(i)}_+ = \bigcup^{i-1}_{j=1}{a^{(j)}_+ \times a^{(i-j)}_+}, ~i \ge 2.$$

Valiant proposed an algorithm for a context-free recognition for a linear input, which in graph terms corresponds to a directed chain of nodes. The algorithm enumerates the positions in the input string $s$ from 0 to the length of $s$, constructs an upper-triangular matrix, and computes its transtive closure. In the context-free path querying input graphs can be arbitrary, which removes the upper-triangularity  limitation. For this reason, we introduce another definition of transitive closure for arbitrary square matrix $a$ as \mbox{$a^{cf} = a^{(1)} \cup a^{(2)} \cup \cdots$}, where \mbox{$a^{(1)} = a$} and $$a^{(i)} = a^{(i-1)} \cup (a^{(i-1)} \times a^{(i-1)}), ~i \ge 2.$$

These two transitive closure definitions are in fact equivalent. To show the equivalence of $a^{cf}$ and $a_+$ definitions of transitive closure, we introduce the partial order $\succeq$ on matrices with a fixed size that have subsets of $N$ as elements. For square matrices $a, b$ of the same size we denote $a \succeq b$ iff $a_{i,j} \supseteq b_{i,j}$, for every $i, j$. For these two definitions of transitive closure, the following lemmas and theorem hold.

\begin{lemma}\label{lemma:cf_geq_valiant}
	Let $G =(N,\Sigma,P)$ be a grammar, let $a$ be a square matrix. Then $a^{(k)} \succeq a^{(k)}_+$ for any $k \geq 1$.
\end{lemma}
\begin{proof}(Proof by Induction)
	
	\textbf{Base case}: The statement of the lemma holds for $k = 1$, since $$a^{(1)} = a^{(1)}_+ = a.$$
	
	\textbf{Inductive step}: Assume that the statement of the lemma holds for any $k \leq (p - 1)$ and show that it also holds for $k = p$ where $p \geq 2$. For any $i \geq 2$ $$a^{(i)} = a^{(i-1)} \cup (a^{(i-1)} \times a^{(i-1)}) \Rightarrow a^{(i)} \succeq a^{(i-1)}.$$ Hence, by the inductive hypothesis, for any $i \leq (p-1)$ $$a^{(p-1)} \succeq a^{(i)} \succeq a^{(i)}_+.$$ Let $1 \leq j \leq (p - 1)$. The following holds $$(a^{(p-1)} \times a^{(p-1)}) \succeq (a^{(j)}_+ \times a^{(p-j)}_+),$$ since $a^{(p-1)} \succeq a^{(j)}_+$ and $a^{(p-1)} \succeq a^{(p-j)}_+$. By the definition, $$a^{(p)}_+ = \bigcup^{p-1}_{j=1}{a^{(j)}_+ \times a^{(p-j)}_+}$$ and from this it follows that $$(a^{(p-1)} \times a^{(p-1)}) \succeq a^{(p)}_+.$$ By the definition, $$a^{(p)} = a^{(p-1)} \cup (a^{(p-1)} \times a^{(p-1)}) \Rightarrow a^{(p)} \succeq (a^{(p-1)} \times a^{(p-1)}) \succeq a^{(p)}_+$$ and this completes the proof of the lemma.
\end{proof}

\begin{lemma}\label{lemma:valiant_geq_cf}
	Let $G =(N,\Sigma,P)$ be a grammar, let $a$ be a square matrix. Then for any $k \geq 1$ there is $j \geq 1$, such that $(\bigcup^{j}_{i=1}{a^{(i)}_+}) \succeq a^{(k)}$.
\end{lemma}
\begin{proof}(Proof by Induction)
	
	\textbf{Base case}: For $k = 1$ there is $j = 1$, such that $$a^{(1)}_+ = a^{(1)} = a.$$ Thus, the statement of the lemma holds for $k = 1$.
	
	\textbf{Inductive step}: Assume that the statement of the lemma holds for any $k \leq (p - 1)$ and show that it also holds for $k = p$ where $p \geq 2$. By the inductive hypothesis, there is $j \geq 1$, such that $$(\bigcup^{j}_{i=1}{a^{(i)}_+}) \succeq a^{(p-1)}.$$ By the definition, $$a^{(2j)}_+ = \bigcup^{2j-1}_{i=1}{a^{(i)}_+ \times a^{(2j-i)}_+}$$ and from this it follows that $$(\bigcup^{2j}_{i=1}{a^{(i)}_+}) \succeq (\bigcup^{j}_{i=1}{a^{(i)}_+}) \times (\bigcup^{j}_{i=1}{a^{(i)}_+}) \succeq (a^{(p-1)} \times a^{(p-1)}).$$ The following holds $$(\bigcup^{2j}_{i=1}{a^{(i)}_+}) \succeq a^{(p)} = a^{(p-1)} \cup (a^{(p-1)} \times a^{(p-1)}),$$ since $$(\bigcup^{2j}_{i=1}{a^{(i)}_+}) \succeq (\bigcup^{j}_{i=1}{a^{(i)}_+}) \succeq a^{(p-1)}$$ and $$(\bigcup^{2j}_{i=1}{a^{(i)}_+}) \succeq (a^{(p-1)} \times a^{(p-1)}).$$ Therefore there is $2j$, such that $$(\bigcup^{2j}_{i=1}{a^{(i)}_+}) \succeq a^{(p)}$$ and this completes the proof of the lemma.	
\end{proof}

\begin{mytheorem}\label{thm:closures}
	Let $G =(N,\Sigma,P)$ be a grammar, let $a$ be a square matrix. Then $a_+ = a^{cf}$.
\end{mytheorem}
\begin{proof}
	
	By the lemma~\ref{lemma:cf_geq_valiant}, for any $k \geq 1$, $a^{(k)} \succeq a^{(k)}_+$. Therefore $$a^{cf} = a^{(1)} \cup a^{(2)} \cup \cdots \succeq a^{(1)}_+ \cup a^{(2)}_+ \cup \cdots = a_+.$$ By the lemma~\ref{lemma:valiant_geq_cf}, for any $k \geq 1$ there is $j \geq 1$, such that $$(\bigcup^{j}_{i=1}{a^{(i)}_+}) \succeq a^{(k)}.$$ Hence $$a_+ = (\bigcup^{\infty}_{i=1}{a^{(i)}_+}) \succeq a^{(k)},$$ for any $k \geq 1$. Therefore $$a_+ \succeq a^{(1)} \cup a^{(2)} \cup \cdots = a^{cf}.$$ Since $a^{cf} \succeq a_+$ and $a_+ \succeq a^{cf}$, $$a_+ = a^{cf}$$ and this completes the proof of the theorem.
\end{proof}

Furthermore, in this paper we use the transitive closure $a^{cf}$ instead of $a_+$, and the algorithm for computing $a^{cf}$ also computes Valiant's transitive closure $a_+$.

\subsection{GPU}