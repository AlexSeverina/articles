\section{Preliminaries}
\label{section_preliminaries}

Let $\Sigma$ be a finite set of edge labels. Define an \emph{edge-labeled directed graph} as a tuple \mbox{$D = (V, E)$} with a set of nodes $V$ and a directed edge relation \mbox{$E \subseteq V \times \Sigma \times V$}.  For a path $\pi$ in a graph $D$, we denote the unique word, obtained by concatenating the labels of the edges along the path $\pi$ as \mbox{$l(\pi)$}. Also, we write \mbox{$n \pi m$} to indicate, that the path $\pi$ starts at the node \mbox{$n \in V$} and ends at the node \mbox{$m \in V$}.

Following Hellings~\cite{hellingsRelational}, we deviate from the usual definition of a context-free grammar in \emph{Chomsky Normal Form}~\cite{chomsky} by not including a special starting non-terminal, which will be specified in the path queries for the graph. Since every context-free grammar can be transformed into an equivalent one in Chomsky Normal Form and checking, that empty string belongs to the language is trivial, it is sufficient to consider only grammars of the following type. A \emph{context-free grammar} is a triple \mbox{$G = (N, \Sigma, P)$}, where $N$ is a finite set of non-terminals, $\Sigma$ is a finite set of terminals, and $P$ is a finite set of productions of the following forms:

\begin{itemize}
    \item $A \rightarrow B C$, for $A,B,C \in N$,
    \item $A \rightarrow x$, for $A \in N$ and $x \in \Sigma$.   
\end{itemize}

Note that we omit the rules of the form \mbox{$A \rightarrow \varepsilon$}, where $\varepsilon$ denotes empty string. This does not restrict the applicability of our algorithm since only the empty paths \mbox{$m \pi m$} correspond to empty string $\varepsilon$.

We use the conventional notation \mbox{$A \xrightarrow{*} w$} to denote, that a string \mbox{$w \in \Sigma^*$} can be derived from a non-terminal $A$ by some sequence of production rule applications from $P$. The \emph{language} of a grammar \mbox{$G = (N,\Sigma,P)$} with respect to a start non-terminal \mbox{$S \in N$} is defined by $$L(G_S) = \{w \in \Sigma^*~|~S \xrightarrow{*} w\}.$$

For a given graph \mbox{$D = (V, E)$} and a context-free grammar $G = (N, \Sigma, P)$, we define \emph{context-free relations} \mbox{$R_A \subseteq V \times V$} for every \mbox{$A \in N$}, such that $$R_A = \{(n,m)~|~\exists n \pi m~(l(\pi) \in L(G_A))\}.$$

We define a binary operation $(~\cdot~)$ for arbitrary subsets \mbox{$N_1, N_2$} of $N$ with respect to a context-free grammar \mbox{$G = (N, \Sigma, P)$} as $$N_1 \cdot N_2 = \{A~|~\exists B \in N_1, \exists C \in N_2 \text{ such that }(A \rightarrow B C) \in P\}.$$

Using this binary operation as subset multiplication, and union as an addition, we can define a \emph{matrix multiplication}, \mbox{$a \times b = c$}, where $a$ and $b$ are matrices of a suitable size, that have subsets of $N$ as elements, as $$c_{i,j} = \bigcup^{n}_{k=1}{a_{i,k} \cdot b_{k,j}}.$$

Also, we use the element-wise union operation on matrices $a$ and $b$ with the same size: \mbox{$a \cup b = c$}, where $c_{i,j} = a_{i,j} \cup b_{i,j}.$

According to Valiant~\cite{valiant}, we define the \emph{transitive closure} of a square matrix $a$ as \mbox{$a^+ = a^{(1)}_+ \cup a^{(2)}_+ \cup \cdots$}, where \mbox{$a^{(1)}_+ = a$} and $$a^{(i)}_+ = \bigcup^{i-1}_{j=1}{a^{(j)}_+ \times a^{(i-j)}_+}, ~i \ge 2.$$

Valiant proposed an algorithm for a context-free recognition for a linear input, which in graph terms corresponds to a directed chain of nodes. The algorithm enumerates the positions in the input string $s$ from 0 to the length of $s$, constructs an upper-triangular matrix, and computes its transtive closure. In the context-free path querying input graphs can be arbitrary, which removes the upper-triangularity  limitation. For this reason, we introduce another definition of transitive closure for arbitrary square matrix $a$ as \mbox{$a^{cf} = a^{(1)} \cup a^{(2)} \cup \cdots$}, where \mbox{$a^{(1)} = a$} and $$a^{(i)} = a^{(i-1)} \cup (a^{(i-1)} \times a^{(i-1)}), ~i \ge 2.$$

These two transitive closure definitions are in fact equivalent (a formal proof can be found in Appendix~\ref{def_eq}). Furthermore, in this paper we use the transitive closure $a^{cf}$ instead of $a^+$, and the algorithm for computing $a^{cf}$ also computes Valiant's transitive closure $a^+$.