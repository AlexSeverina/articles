\section{The reduction}

Suppose we have $\Phi$ --- an instance of 3-SAT problem contains $m$ clauses over $k$ variables.

First of all, we should to constract a graph. 
To do it we follow the next steps.
\begin{enumerate}
	\item 
	Let $\gamma_i = \{v_1 \leftarrow b_1, v_2 \leftarrow b_2, \cdots, v_k \leftarrow b_k\}$ where $b_k \in \{0,1\}$.
	For each substitution $\gamma_i$ a vertex $V_{\gamma_i}$ should be created.
	\item For each $V_{\gamma_i}$ the following edges should be added: $\{ (V_{\gamma_i}, [\gamma_i: v_j \leftarrow b_l]^+ ,V_{\gamma_i}) \mid v_j \leftarrow b_l \in \gamma_i \}$.
	\item For each clause $(l_1 \vee l_2 \vee l_3)$ the following subgraph should be created.
	First, two new vertices are edded: $c_1$ and $c_2$.
	After that, the following edges for each $l_p$ and for each $\gamma_i$ should be added $$\{(c_1, [\gamma_i: v_j \leftarrow b_l]^- ,c_2) \mid b_l = \systeme*{1 \text{ if } l_p = v_j, 0  \text{ if } l_p = \neg v_j} \}.$$
	\item Subgraph for all clauses should be connected sequencially. 
	Suppose we have seqence of subgraps with vertices $$\{(c_1^1,c_2^1),(c_1^2,c_2^2),\cdots,(c_1^m,c_2^m)\}.$$ To connect them we should merge vertices $c_2^{i}$ and $c_1^{i+1}$ for all $i$ except $i=m$.
	After that we fix $c_1^1$ as a start vertex of formula subgraph, and $c_2^m$ as a final vertex of formula subgraph.
    \item Finally, for all $V_{\gamma_i}$ we should add the following set of edges
    $$
    \{(V_{\gamma_i}, [\gamma_i: v_j \leftarrow b_l]^+ ,c_1^1) \mid v_j \leftarrow b_l \in \gamma_i\}
    $$
\end{enumerate}

The second part is a query.
Suppose, we have p different substitutions.
The gramamr is following
\begin{align*}
S & \to S_{\gamma_1} \mid S_{\gamma_2} \mid \cdots \mid S_{\gamma_p} \\
S_{\gamma_i} & \to \varepsilon \\
S_{\gamma_i} & \to [\gamma_i: v_1 \leftarrow b_1]^+ \ S_{\gamma_i} \ [\gamma_i: v_1 \leftarrow b_1]^-  \\
             & \mid \cdots \\
             & \mid  [\gamma_i: v_k \leftarrow b_k]^+ \ S_{\gamma_i} \ [\gamma_i: v_k \leftarrow b_k]^- \\ 
\end{align*}

After that we should applay transformation which is described in the section~\ref{sec:cfpq_to_dyck}. 
As a result we get h-Dyck reachability problem (yes, we can reduce it to 2-Dyck reachability).

\subsection{An Example of Reduction}

Suppose we have the following instance of 3-SAT problem. 
$$
\Phi = (\neg x_1 \vee x_2 \vee \neg x_3) \wedge (\neg x_2 \vee x_1 \vee x_3) \wedge (x_1 \vee \neg x_3 \vee x_2)
$$

Substitutions:
\begin{align*}
\gamma_1 = \{x_1 \leftarrow 0, x_2 \leftarrow 0, x_3 \leftarrow 0\} \\
\gamma_2 = \{x_1 \leftarrow 1, x_2 \leftarrow 0, x_3 \leftarrow 0\} \\
\gamma_3 = \{x_1 \leftarrow 0, x_2 \leftarrow 1, x_3 \leftarrow 0\} \\
\gamma_4 = \{x_1 \leftarrow 0, x_2 \leftarrow 0, x_3 \leftarrow 1\} \\
\gamma_5 = \{x_1 \leftarrow 1, x_2 \leftarrow 1, x_3 \leftarrow 0\} \\
\gamma_6 = \{x_1 \leftarrow 1, x_2 \leftarrow 0, x_3 \leftarrow 1\} \\
\gamma_7 = \{x_1 \leftarrow 0, x_2 \leftarrow 1, x_3 \leftarrow 1\} \\
\gamma_8 = \{x_1 \leftarrow 1, x_2 \leftarrow 1, x_3 \leftarrow 1\} 
\end{align*}

Graph for $\Phi$ is presented in figure~\ref{fig:sat_to_cfpq_graph_example}.

The grammar:
\begin{align*}
S & \to S_{\gamma_1} \mid S_{\gamma_2} \mid S_{\gamma_3} \mid S_{\gamma_4} \mid S_{\gamma_5} \mid S_{\gamma_6} \mid S_{\gamma_7} \mid S_{\gamma_8} \\
S_{\gamma_1} & \to  [\gamma_1:x_1 \leftarrow 0]^+ \ S_{\gamma_1} \ [\gamma_1:x_1 \leftarrow 0]^- \\
             & \mid [\gamma_1:x_2 \leftarrow 0]^+ \ S_{\gamma_1} \ [\gamma_1:x_2 \leftarrow 0]^- \\ 
             & \mid [\gamma_1:x_3 \leftarrow 0]^+ \ S_{\gamma_1} \ [\gamma_1:x_3 \leftarrow 0]^- \\ 
             & \mid \varepsilon \\
S_{\gamma_2} & \to  [\gamma_2:x_1 \leftarrow 1]^+ \ S_{\gamma_2} \ [\gamma_2:x_1 \leftarrow 1]^- \\
             & \mid [\gamma_2:x_2 \leftarrow 0]^+ \ S_{\gamma_2} \ [\gamma_2:x_2 \leftarrow 0]^- \\ 
             & \mid [\gamma_2:x_3 \leftarrow 0]^+ \ S_{\gamma_2} \ [\gamma_2:x_3 \leftarrow 0]^- \\ 
             & \mid \varepsilon \\
S_{\gamma_3} & \to  [\gamma_3:x_1 \leftarrow 0]^+ \ S_{\gamma_3} \ [\gamma_3:x_1 \leftarrow 0]^- \\
             & \mid [\gamma_3:x_2 \leftarrow 1]^+ \ S_{\gamma_3} \ [\gamma_3:x_2 \leftarrow 1]^- \\ 
             & \mid [\gamma_3:x_3 \leftarrow 0]^+ \ S_{\gamma_3} \ [\gamma_3:x_3 \leftarrow 0]^- \\ 
             & \mid \varepsilon \\ 
S_{\gamma_4} & \to  [\gamma_4:x_1 \leftarrow 0]^+ \ S_{\gamma_4} \ [\gamma_4:x_1 \leftarrow 0]^- \\
             & \mid [\gamma_4:x_2 \leftarrow 0]^+ \ S_{\gamma_4} \ [\gamma_4:x_2 \leftarrow 0]^- \\ 
             & \mid [\gamma_4:x_3 \leftarrow 1]^+ \ S_{\gamma_4} \ [\gamma_4:x_3 \leftarrow 1]^- \\ 
             & \mid \varepsilon \\
S_{\gamma_5} & \to  [\gamma_5:x_1 \leftarrow 1]^+ \ S_{\gamma_5} \ [\gamma_5:x_1 \leftarrow 1]^- \\
             & \mid [\gamma_5:x_2 \leftarrow 1]^+ \ S_{\gamma_5} \ [\gamma_5:x_2 \leftarrow 1]^- \\ 
             & \mid [\gamma_5:x_3 \leftarrow 0]^+ \ S_{\gamma_5} \ [\gamma_5:x_3 \leftarrow 0]^- \\ 
             & \mid \varepsilon \\
S_{\gamma_6} & \to  [\gamma_6:x_1 \leftarrow 1]^+ \ S_{\gamma_6} \ [\gamma_6:x_1 \leftarrow 1]^- \\
             & \mid [\gamma_6:x_2 \leftarrow 0]^+ \ S_{\gamma_6} \ [\gamma_6:x_2 \leftarrow 0]^- \\ 
             & \mid [\gamma_6:x_3 \leftarrow 1]^+ \ S_{\gamma_6} \ [\gamma_6:x_3 \leftarrow 1]^- \\ 
             & \mid \varepsilon \\
S_{\gamma_7} & \to  [\gamma_7:x_1 \leftarrow 0]^+ \ S_{\gamma_7} \ [\gamma_7:x_1 \leftarrow 0]^- \\
             & \mid [\gamma_7:x_2 \leftarrow 1]^+ \ S_{\gamma_7} \ [\gamma_7:x_2 \leftarrow 1]^- \\ 
             & \mid [\gamma_7:x_3 \leftarrow 1]^+ \ S_{\gamma_7} \ [\gamma_7:x_3 \leftarrow 1]^- \\ 
             & \mid \varepsilon \\ 
S_{\gamma_8} & \to  [\gamma_8:x_1 \leftarrow 1]^+ \ S_{\gamma_8} \ [\gamma_8:x_1 \leftarrow 1]^- \\
             & \mid [\gamma_8:x_2 \leftarrow 1]^+ \ S_{\gamma_8} \ [\gamma_8:x_2 \leftarrow 1]^- \\ 
             & \mid [\gamma_8:x_3 \leftarrow 1]^+ \ S_{\gamma_8} \ [\gamma_8:x_3 \leftarrow 1]^- \\ 
             & \mid \varepsilon 
\end{align*}

The intuition of such path finding is that substitution vertex ($V_{\gamma_i}$) should provide appropriate values for respective variable in appropriate order to satisfy the given formula.
It can be done by appropriate traversing of loops. 
After that, each edge from $c_i^j$ to  $c_l^k$ ``uses'' provided values to satisfie respective closure, and it can be done if and only if the respective vertex provides value required.
This fact is expressed by usung balanced-bracket grammar.
So, if there exists a path from $V_{\gamma_i}$ to $c_2^3$, such that the corresponded word is derivable from $S$, then $V_{\gamma_i}$ satisfy the given formula. 


\begin{figure*}
\begin{tikzpicture}[-,>=stealth',shorten >=1pt,auto,node distance=3.5cm,
  thick,main node/.style={circle,fill=black!20,draw}]

  
  \node[main node] (6) [] {$V_{\gamma_2}$};
  \node[main node] (7) [below of=6] {$V_{\gamma_3}$};
  \node[main node] (8) [below of=7] {$V_{\gamma_4}$};
  \node[main node] (9) [below of=8] {$V_{\gamma_5}$};
  \node[main node] (10) [below of=9] {$V_{\gamma_6}$};
  \node[draw=none,fill=none] (13) [right of=10] {};
  \node[main node] (11) [right of=13] {$V_{\gamma_7}$};
  


  \node[main node, minimum size = 40] (1) [right of=8] {$c_1^1$};
  \node[main node] (2) [right of=1] {$c_2^1$};
  \node[main node] (3) [right of=2] {$c_2^2$};
  \node[main node] (4) [right of=3] {$c_2^3$};

  \node[main node] (5) [above of=4] {$V_{\gamma_1}$};
  \node[main node] (12) [below of=4] {$V_{\gamma_8}$};

  \path[-> , every node/.style={font=\sffamily\small}]
    (5) edge [loop right, right] node  {$[\gamma_1: x_1 \leftarrow 0]^+$}   (5)    
    (5) edge [loop above, right] node  {$[\gamma_1: x_2 \leftarrow 0]^+$}  (5)    
    (5) edge [loop below, right] node  {$[\gamma_1: x_3 \leftarrow 0]^+$}  (5)

    (5) edge [bend right=20, sloped, above] node  {$[\gamma_1: x_1 \leftarrow 0]^+$}  (1)    
    (5) edge [bend right=40, sloped, above] node  {$[\gamma_1: x_2 \leftarrow 0]^+$}  (1)    
    (5) edge [bend right=60, sloped, above] node  {$[\gamma_1: x_3 \leftarrow 0]^+$}  (1)        


    (6) edge [loop left, left] node  {$[\gamma_1: x_1 \leftarrow 1]^+$}   (6)    
    (6) edge [loop above, left] node  {$[\gamma_1: x_2 \leftarrow 0]^+$}  (6)    
    (6) edge [loop below, left] node  {$[\gamma_1: x_3 \leftarrow 0]^+$}  (6)

    (6) edge [bend left=20, sloped, above] node  {$[\gamma_1: x_1 \leftarrow 1]^+$}  (1)    
    (6) edge [bend left=40, sloped, above] node  {$[\gamma_1: x_2 \leftarrow 0]^+$}  (1)    
    (6) edge [bend left=60, sloped, above] node  {$[\gamma_1: x_3 \leftarrow 0]^+$}  (1)        
         


    (7) edge [loop left, left] node  {$[\gamma_1: x_1 \leftarrow 0]^+$}   (7)    
    (7) edge [loop above, left] node  {$[\gamma_1: x_2 \leftarrow 1]^+$}  (7)    
    (7) edge [loop below, left] node  {$[\gamma_1: x_3 \leftarrow 0]^+$}  (7)

    (7) edge [bend left=30, sloped, above] node  {$[\gamma_1: x_1 \leftarrow 0]^+$}  (1)    
    (7) edge [bend left=10, sloped, above] node  {$[\gamma_1: x_2 \leftarrow 1]^+$}  (1)    
    (7) edge [bend right=10, sloped, above] node  {$[\gamma_1: x_3 \leftarrow 0]^+$}  (1)         



    (8) edge [loop left, left] node  {$[\gamma_1: x_1 \leftarrow 0]^+$}   (8)    
    (8) edge [loop above, left] node  {$[\gamma_1: x_2 \leftarrow 0]^+$}  (8)    
    (8) edge [loop below, left] node  {$[\gamma_1: x_3 \leftarrow 1]^+$}  (8)

    (8) edge [bend left=25,above] node  {$[\gamma_1: x_1 \leftarrow 0]^+$}  (1)    
    (8) edge [] node  {$[\gamma_1: x_2 \leftarrow 0]^+$}  (1)    
    (8) edge [bend right=20] node  {$[\gamma_1: x_3 \leftarrow 1]^+$}  (1)



    (9) edge [loop left, left] node  {$[\gamma_1: x_1 \leftarrow 1]^+$}   (9)    
    (9) edge [loop above, left] node  {$[\gamma_1: x_2 \leftarrow 1]^+$}  (9)    
    (9) edge [loop below, left] node  {$[\gamma_1: x_3 \leftarrow 0]^+$}  (9)

    (9) edge [bend right=30, sloped, above] node  {$[\gamma_1: x_1 \leftarrow 1]^+$}  (1)    
    (9) edge [bend right=10, sloped, above] node  {$[\gamma_1: x_2 \leftarrow 1]^+$}  (1)    
    (9) edge [bend left=10, sloped, above] node  {$[\gamma_1: x_3 \leftarrow 0]^+$}  (1)


    
    (10) edge [loop left, left] node  {$[\gamma_1: x_1 \leftarrow 1]^+$}   (10)    
    (10) edge [loop above, left] node  {$[\gamma_1: x_2 \leftarrow 0]^+$}  (10)    
    (10) edge [loop below, left] node  {$[\gamma_1: x_3 \leftarrow 1]^+$}  (10)

    (10) edge [bend right=5, sloped, above] node  {$[\gamma_1: x_1 \leftarrow 1]^+$}  (1)    
    (10) edge [bend right=15, sloped, above] node  {$[\gamma_1: x_2 \leftarrow 0]^+$}  (1)    
    (10) edge [bend right=30, sloped, above] node  {$[\gamma_1: x_3 \leftarrow 1]^+$}  (1)


    
    (11) edge [loop left, left] node  {$[\gamma_1: x_1 \leftarrow 0]^+$}   (11)    
    (11) edge [loop below, below] node  {$[\gamma_1: x_2 \leftarrow 1]^+$}  (11)    
    (11) edge [loop right, right] node  {$[\gamma_1: x_3 \leftarrow 1]^+$}  (11)

    (11) edge [bend left=35, sloped, below] node  {$[\gamma_1: x_1 \leftarrow 0]^+$}  (1)    
    (11) edge [bend left=20, sloped, below] node  {$[\gamma_1: x_2 \leftarrow 1]^+$}  (1)    
    (11) edge [bend left=5, sloped, below] node  {$[\gamma_1: x_3 \leftarrow 1]^+$}  (1)




    (12) edge [loop right, right] node  {$[\gamma_1: x_1 \leftarrow 1]^+$}   (12)    
    (12) edge [loop above, right] node  {$[\gamma_1: x_2 \leftarrow 1]^+$}  (12)    
    (12) edge [loop below, right] node  {$[\gamma_1: x_3 \leftarrow 1]^+$}  (12)

    (12) edge [bend left=20, sloped, above] node  {$[\gamma_1: x_1 \leftarrow 1]^+$}  (1)    
    (12) edge [bend left=40, sloped, above] node  {$[\gamma_1: x_2 \leftarrow 1]^+$}  (1)    
    (12) edge [bend left=60, sloped, above] node  {$[\gamma_1: x_3 \leftarrow 1]^+$}  (1)


    (1) edge [bend left=65, above] node  {$[\gamma_1: x_1 \leftarrow 0]^-$}  (2)
    (1) edge [bend left=30, above] node  {$[\gamma_1: x_2 \leftarrow 1]^-$}  (2)
    (1) edge [bend right=0, above] node  {$[\gamma_1: x_3 \leftarrow 0]^-$}  (2)
    (1) edge [bend right=30, above] node  {$[\gamma_2: x_1 \leftarrow 0]^-$} (2)
    (1) edge [bend right=45, above] node  {$\dots$}                          (2)
    (1) edge [bend right=85, above] node  {$[\gamma_7: x_3 \leftarrow 0]^-$} (2)
    

    (2) edge [bend left=65, above] node  {$[\gamma_1: x_1 \leftarrow 1]^-$}  (3)
    (2) edge [bend left=30, above] node  {$[\gamma_1: x_2 \leftarrow 0]^-$}  (3)
    (2) edge [bend right=0, above] node  {$[\gamma_1: x_3 \leftarrow 1]^-$}  (3)
    (2) edge [bend right=30, above] node  {$[\gamma_2: x_1 \leftarrow 1]^-$} (3)
    (2) edge [bend right=45, above] node  {$\dots$}                          (3)
    (2) edge [bend right=85, above] node  {$[\gamma_7: x_3 \leftarrow 1]^-$} (3)

    (3) edge [bend left=65, above] node  {$[\gamma_1: x_1 \leftarrow 1]^-$}  (4)
    (3) edge [bend left=30, above] node  {$[\gamma_1: x_2 \leftarrow 1]^-$}  (4)
    (3) edge [bend right=0, above] node  {$[\gamma_1: x_3 \leftarrow 0]^-$}  (4)
    (3) edge [bend right=30, above] node  {$[\gamma_2: x_1 \leftarrow 1]^-$} (4)
    (3) edge [bend right=45, above] node  {$\dots$}                          (4)
    (3) edge [bend right=85, above] node  {$[\gamma_7: x_3 \leftarrow 0]^-$} (4)
    
    ;

\end{tikzpicture}
\caption{Example of graph for $\Phi$}
\label{fig:sat_to_cfpq_graph_example}
\end{figure*}