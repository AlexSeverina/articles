\section{Details of dataset description}

Here we present examples of grammars and graphs representation and grammars for queries which we use for evaluation.

\begin{figure}[h]
    \centering
    \begin{subfigure}[b]{0.24\textwidth}
        \centering
        \[
         \begin{array}{rcl}
           s & \rightarrow & A \ s \ B \\
           s & \rightarrow & A \ B
         \end{array}
         \]
        \caption{Grammar $G_1$}
    \end{subfigure}%
    ~
    \begin{subfigure}[b]{0.24\textwidth}
        \centering
        \begin{verbatim}
     s a b
     s a s1
     s1 s b
     a A
     b B
        \end{verbatim}
        \caption{Representation of grammar $G_1$ in \texttt{yrd} file}
    \end{subfigure}
    \caption{Example of grammar representation in the \texttt{yrd} file}
    \label{fig:grammar_example}
\end{figure}


\begin{figure}[h]
    \centering
    \begin{subfigure}[b]{0.24\textwidth}
        \centering
        \begin{tikzpicture}[shorten >=1pt,node distance=2cm,on grid,auto]
   \node[state] (q_1)   {$1$};
   \node[state] (q_2) [above=of q_1] {$2$};
   \node[state] (q_3) [above right=of q_1, below right=of q_2] {$0$};
   \node[state] (q_4) [right=of q_3] {$3$};
    \path[->]
    (q_1) edge  node {A} (q_2)
    (q_2) edge  node {A} (q_3)
    (q_3) edge  node {A} (q_1)
    (q_3) edge[bend left, above]  node {B} (q_4)
    (q_4) edge[bend left, below]  node {B} (q_3);
\end{tikzpicture}

        \caption{Example of input graph $D_1$}
    \end{subfigure}%
    ~
    \begin{subfigure}[b]{0.24\textwidth}
        \centering
         \begin{verbatim}
     0 A 1
     1 A 2
     2 A 0
     0 B 3
     3 B 0
 \end{verbatim}

        \caption{Representation of the input graph $D_1$ in \texttt{txt} file}
    \end{subfigure}
    \caption{Example of graph representation in \texttt{txt} file}
    \label{fig:graph_example}
\end{figure}


\begin{figure}[h]
    \centering
    \begin{subfigure}[b]{0.24\textwidth}
        \centering
        \[
         \begin{array}{rcl}
           s & \rightarrow & s \ s \\
           s & \rightarrow & A
         \end{array}
         \]
        \caption{Simple ambiguous grammar $G_2$}
    \end{subfigure}%
    ~
    \begin{subfigure}[b]{0.24\textwidth}
        \centering
        \[
         \begin{array}{rcl}
           s & \rightarrow & s \ s \ s \\
           s & \rightarrow & A
         \end{array}
         \]
        \caption{Highly ambiguous grammar $G_3$}
    \end{subfigure}
    \caption{Queries for the \texttt{[Full]} dataset}
    \label{fig:grammar_full}
\end{figure}

\begin{figure}[h]
    \centering
    \begin{subfigure}[b]{0.24\textwidth}
        \centering
        \[
         \begin{array}{rcl}
           s & \rightarrow & SCOR \ s \ SCO \\
           s & \rightarrow & TR \ S \ T     \\
           s & \rightarrow & TR \ T     \\
           s & \rightarrow & SCOR \ SCO     \\
         \end{array}
         \]
        \caption{Same generation query $G_4$}
    \end{subfigure}%
    ~
    \begin{subfigure}[b]{0.24\textwidth}
        \centering
        \[
         \begin{array}{rcl}
           s & \rightarrow & SCOR \ s \ SCO \\
           s & \rightarrow & SCO
         \end{array}
         \]
        \caption{Same generation query $G_5$}
    \end{subfigure}
    \caption{Queries for the \texttt{[RDF]} dataset}
    \label{fig:grammar_rdf}
\end{figure}
