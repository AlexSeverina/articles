\section{Matrix-based Algorithm for CFPQ}

Matrix-based algorithm for CFPQ was proposed by Rustam Azimov~\cite{Azimov:2018:CPQ:3210259.3210264}.
This algorithm can be expressed as showed in in listing~\ref{lst:algo1} in terms of operations over matrices, and it is a sufficient advantage for implementation.
It was shown that GPGPU utilization for queries evaluation can significantly improve performance in comparison to other implementations~\cite{Azimov:2018:CPQ:3210259.3210264} even if float matrices used instead of boolean matrices.

Pseudocode of the algorithm is presented .


\begin{algorithm}
  \floatname{algorithm}{Listing}
\begin{algorithmic}[1]
\caption{Context-free path quering algorithm}
\label{lst:algo1}
\Function{contextFreePathQuerying}{D, G}

    \State{$n \gets$ the number of nodes in $D$}
    \State{$E \gets$ the directed edge-relation from $D$}
    \State{$P \gets$ the set of production rules in $G$}
    \State{$T \gets$ the matrix $n \times n$ in which each element is $\varnothing$}
    \ForAll{$(i,x,j) \in E$}
    \Comment{Matrix initialization}
        \State{$T_{i,j} \gets T_{i,j} \cup \{A~|~(A \rightarrow x) \in P \}$}
    \EndFor
    \While{matrix $T$ is changing}

        \State{$T \gets T \cup (T \times T)$}
        \Comment{Transitive closure calculation}
    \EndWhile
\State \Return $T$
\EndFunction
\end{algorithmic}
\end{algorithm}

Here $D = (V, E)$ be the input graph and $G = (N,\Sigma,P)$ be the input grammar.
Each cell of the matrix $T$ contains the set of nonterminals such that $N_k \in T[i,j] \iff \exists p = v_i \ldots v_j $---path in $D$, such that $N_k \xRightarrow[G]{*} \omega(p) $, where $\omega(p)$ is a word formed by labels along the path $p$.
Thus, this algorithm solves reachability problem, or, according to Hellings~\cite{hellingsRelational}, process CFPQs by using relational query semantics.

As you can see, performance-critical part of this algorithm is matrix multiplication.
Note, that the set of nonterminals is finite, and we can represent the matrix $T$ as a set of boolean matrices: one for each nonterminal.
In this case the matrix update operation be $T_{N_i} \leftarrow T_{N_i} + (T_{N_j} \times T_{N_k})$ for each production $N_i \rightarrow N_j \ N_k$ in $P$.
Thus we can reduse CFPQ to boolean matrices multiplication.
After such transfromation we can apply the next optimization: we can skip update if there are no changes in the matrices $T_{N_j}$ and $T_{N_k}$ at the previous iteration.

Thus, the most important part is efficient implementation of operations over boolean matrices, and in this work we compare effects of utilization of different approaches to matrices multiplication.
All our implemetations are based on the optimized version of the algorithm.
