\section{Introduction}

Language-constrained path querying~\cite{FLCpathProblem}, and particularly Context-Free Path Querying (CFPQ)~\cite{Yannakakis}, allows one to use formal grammars as constraints for paths: concatenation of the labels along the path is treated as a word, and a constraint on the path is a specification of the language which should contain specific words.
CFPQ is widely used for graph-structured data analysis in such domains as biological data analysis, RDF, network analysis.
Huge amount of the real-world data makes performance of CFPQ critical for practical tasks.
A number of algorithms for CFPQ based on such parsing techniques as (G)LL, (G)LR, and CYK is proposed recently~\cite{hellingsPathQuerying,Grigorev:2017:CPQ:3166094.3166104,Verbitskaia:2018:PCC:3241653.3241655,RDF,10.1007/978-3-319-91662-0_17,Medeiros:2018:EEC:3167132.3167265}.

One of the most promising algorithms is a matrix-based algorithm, proposed by Rustam Azimov~\cite{Azimov:2018:CPQ:3210259.3210264}.
This algorithm offloads the most critical computations onto boolean matrices multiplication.
As a result, it is easy to implement, and allows one to utilize modern massive-parallel hardware for CFPQ.
The implementation provided by the authors utilizes GPGPU by using cuSPARSE\footnote{cuSparse is a library for GPGPU utilization for sparse matrices multiplication. Official documentation: \url{https://docs.nvidia.com/cuda/cusparse/index.html}. Access date: 12.03.2019} library which is a floating point sparse matrices multiplication library.
Although it does not use advanced algorithms for boolean matrices, it outperforms existing algorithms.

It is necessary to investigate the effect of specific algorithms and implementation techniques on the performance of CFPQ.
One problem is that exists no publicly available standard dataset for CFPQ algorithms evaluation which includes both graph-structured data and queries.

In this work, we do an empirical performance comparison of several implementations of matrices multiplication based algorithm for CFPQ on both real data and synthetic data for the worst cases.
We make the following contributions in this paper.

\begin{enumerate}
\item We provide a number of implementations of the matrix multiplication based CFPQ algorithm, which utilizes different modern software and hardware.
The source code is available on GitHub: \href{link to the repository is hidden for anonymization purposes}{link to the repository is hidden for anonymization purposes} 
%\href{https://github.com/SokolovYaroslav/fast-boolean-semiring-matrix-multiplication-for-CFPQ}{https://github.com/SokolovYaroslav/fast-boolean-semiring-matrix-multiplication-for-CFPQ}
\item We present a dataset which contains both real data and syntatic data for worst cases.
This dataset contains data and queries in the simple textual format, so it can be easily used to evaluate other algorithms.
This dataset is available as a part of source code on GitHub, and we hope that this dataset can form a base of the unified benchmark for CFPQ algorithms.
\item We provide evaluation which shows that GPGPU utilization for CFPQ can significantly improve performance, and that there are still many open questions in this area.
\end{enumerate}
