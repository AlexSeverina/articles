\section{Introduction}

Language-constrained path querying~\cite{!!!}, and particulary Context-Free Path Querying (CFPQ)~\cite{!!!} widely used for graph-structured data analysis in such areas as biological data analysis, RDF, network analysis.
Huge amount of the real-world data makes performance of CFPQ evaluation critical for practical tasks, and number of algorithms for CFPQ evaluation proposed ricently~\cite{!!!}.

One of the most promising algorithm is a matrix-based algorithm, proposed by Rustam Azimov~\cite{!!!}.
This algorith offloads the most critical computattions onto boolean matrices multiplication.
As a result, it is pretty simple for implementattion and allows one to utilize modern massive-parallel hardware for CFPQs evaluation.
Implenetation provided by authors utilizes GPGPU by using CuSparse\footnote{!!!} library which is floating point sparse matrices multiplication library.
Even it does not use advanced algorithms for boolean matrices, it outperforms existing algorithms.
It is necessary to investigate an effect of specific algorithms and implementation techniques on performance of CFPQ.

One of problems is that there is no publically available standartd dataset for CFPQ algorithms evaluation which includes both graph-structured data and queryes.

In this work, we do empirical performance comparison of different implementations of matrices multiplication based algorithm for CFPQ on both real data and synthetic data for the worst cases.
We make the following contributions in this paper.

\begin{enumerate}
\item We provide a number of implementations of the matrix multiplication based CFPQ algoithm, which utilizes different modern software and hard. Source code is available on GitHub:\url{!!!}
\item We collect and publish a dataset which contains both real data and syntatic data for wirst cases.
This dataset contains data and queries in the simplee textual format, so it can be used for other algorithms evaluation easely.
We hope that this dataset can be a base for unified benchmark for CFPQ algorithms.
\item Evaluation. We show that !!!
\end{enumerate}
