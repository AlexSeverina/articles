\section{Dataset description}

We create and publish a dataset for CFPQ algorithms evaluation.
This dataset contains both the real data and syntetic data for different cpecific cases, such as theoretical worst case, or matrices representation specific worst cases.

Our goal is querying algorithms evaluation, not a graph storages or graph databases evaluation, so all data is presented in text-based format to simplify usage in different environments.
Grammars are in Chomscy Normal Form and are stored in the files with \verb|yrd| extension.
Each line is a rule in form of triple or pair. 
The example of grammar representation is presented in figure~\ref{fig:grammar_example}


\begin{figure}[h]
    \centering
    \begin{subfigure}[b]{0.24\textwidth}
        \centering
        \[
         \begin{array}{rcl}
           s & \rightarrow & A \ s \ B \\
           s & \rightarrow & A \ B 
         \end{array}        
         \]
        \caption{Grammar $G_1$}
    \end{subfigure}%
    ~ 
    \begin{subfigure}[b]{0.24\textwidth}
        \centering
        \begin{verbatim}
     s a b
     s a s1
     s1 s b
     a A
     b B
        \end{verbatim}
        \caption{Representation of grammar $G_1$ in \texttt{yrd} file}
    \end{subfigure}
    \caption{Example of grammar representation in the \texttt{yrd} file}
    \label{fig:grammar_example}
\end{figure}



Graphs are represented as a set of triples (edges) and are stored in the files with \verb|txt| extension. 
Example of graph is presented in figure~\ref{fig:graph_example}.

\begin{figure}[h]
    \centering
    \begin{subfigure}[b]{0.24\textwidth}
        \centering
        \begin{tikzpicture}[shorten >=1pt,node distance=2cm,on grid,auto]
   \node[state] (q_1)   {$1$};
   \node[state] (q_2) [above=of q_1] {$2$};
   \node[state] (q_3) [above right=of q_1, below right=of q_2] {$0$};
   \node[state] (q_4) [right=of q_3] {$3$};
    \path[->]
    (q_1) edge  node {A} (q_2)
    (q_2) edge  node {A} (q_3)
    (q_3) edge  node {A} (q_1)
    (q_3) edge[bend left, above]  node {B} (q_4)
    (q_4) edge[bend left, below]  node {B} (q_3);
\end{tikzpicture}

        \caption{Example of input graph $D_1$}
    \end{subfigure}%
    ~ 
    \begin{subfigure}[b]{0.24\textwidth}
        \centering
         \begin{verbatim}
     0 A 1
     1 A 2
     2 A 0
     0 B 3
     3 B 0
 \end{verbatim} 

        \caption{Representation of the input graph $D_1$ in \texttt{txt} file}
    \end{subfigure}
    \caption{Example of graph representation in \texttt{txt} file}
    \label{fig:graph_example}
\end{figure}

Each case is a pair of set of graphs and set of grammars: each query (grammar) should be applied to each graph.
Cases are placed in folders with case-specific name.
Grammars and graph are placed in subfolders with names \verb|Grammars| and \verb|Matrices| respectively.


It is known that varians of the \textit{same generation query}~\ref{FndDB} are classical example of queryes that are context-free but not regular, so we use this type of queryes in our evaluation.
The datatset includes data for next cases.
\begin{itemize}
\item[\textbf{[RDF]}] The set of real RDF files (ontologies) from~\cite{RDF} and two variants of the same generation query (figures~\ref{!!!}).

\item[\textbf{[Worst]}] Theoretical worst case for CFPQ time complexity which is proposed by Hellings~\cite{hellingsPathQuerying}: graph is a two cycles of coprime lengths with single common vertex.
First cycle labelled by open bracket and the second cysle is labelled by close bracked.
Query is a grammar for $A^nB^n$ languge (grammar $G_1$, figure~\ref{fig:grammar_example}).

\item[\textbf{[Full]}] The case when input graph is sparce, but result is a full graph. 
Such case may be a hard for sparse matrices representation.
As an input graph we use a cycle all edges of which is labelled by the same token.
As a queryes we use two grammars which describe arbitrary repetition of a token: unambiguous and highly ambiguous grammar (figure~\ref{!!!}). 

\item[\textbf{[Sparse]}] Sparse graphs from~\cite{fan2018scaling} which generated by the GTgraph graph generator, and emulates realistic sparse data. 
Names of these graphs have a form \texttt{Gn-p}, where \texttt{n} represents the total number of vertices, each pair of vertices is connected by probability \texttt{p}. 
Query is a same generation query.

\end{itemize}
