\section{От CFPQ к вычислению Datalog-запросов}\label{Subsection Datalog}

\subsection{Datalog}
Конечные Эрбрановы модели.
Наименьшая неподвижная точка.
$C \coloneq d$


\subsection{КС-запрос как запрос на Datalog}

Покажем, что для данного графа и КС-запроса можно построить эквивалентный запрос на Datalog.

Пусть дан граф
Граф преобразуется в набор фактов (базу данных).

Пусть есть граммтика $G$: $S \rightarrow a \ b \ | \ a \ S \ b$.
Она может быть преобразована в запрос следующего вида.
$s(X,Y) \coloneq a(X,Z), b(Z,Y). $
$s(X,Y) \coloneq a(X,Z), s(Z,W) b(W,Y). $
$? \coloneq s(X,Y)$

Наблюдения: появились пременные, есть порядок у конъюнктов, который задаёт порядок связывания.

\subsection{Обобщение GLL для вычисления Datalog-запросов}
Дескриптор --- состояние процесса: состояние автомата, результат проделанной работы, подстановка.
Задача --- найти подстановки.
На каждом шаге есть набор подстановок.

\subsection{Вопросы и задачи}
\begin{enumerate}
  \item Написать синтаксический анализатор раз.
  \item Написать синтаксический анализатор два.
  \item Побаловаться с неоднозначными грамматиками
  \item Побаловаться с конъюнктивными грамматиками.
  \item Графы?
\end{enumerate}
