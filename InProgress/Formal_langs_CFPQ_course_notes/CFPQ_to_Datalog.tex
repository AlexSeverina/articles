\section{От CFPQ к вычислению Datalog-запросов}\label{Subsection Datalog}
Рассмотрим грамматику  $S -> aSb | SS | \varepsilon$, заданную через набор предикатов:
\begin{itemize}
	\item a(i, w) --- Предикат, соответствующий терминалу. Обращается в True, если на i-том месте в строке w стоит символ a
	\item S(i, j, w) --- Предикат, соответствующий нетерминалу. Обращается в True, если выполняется одно из условий:
	\begin{enumerate}
		\item $i == j \quad(\varepsilon)$
		\item $\exists k: i \leq k \leq j \And S(i, k - 1, w) \And S(k, j, w)  \quad (SS)$
		\item  ${a(i, w) \And S(i+1, j-1, w) \And b(j-1, w)} \quad (aSb)$
	\end{enumerate}
\end{itemize}

Таким образом S(0, |w|, w) покажет, выводится ли строка w из данной грамматики,
а S(\_,\_,w) даст нам список всех цепочек внутри w, выводящихся из S.

\subsection{Datalog}\footnote{\url{https://en.wikipedia.org/wiki/Datalog}}
Datalog --- декларативный логический язык программирования. Используется для написания запросов к дедуктивным базам данных\footnote{\url{https://en.wikipedia.org/wiki/Deductive_database}}. \\

\begin{example}
	Пример программы на даталоге.
	\begin{enumerate}
		\item Набор фактов (часто факты находятся в базе данных):
		\begin{itemize}
			\item $a(0).$
			\item $b(1).$
			\item $a(2).$
			\item $b(3).$
			\item $s(I, I).$
		\end{itemize}
		\item Набор правил:
		\begin{itemize}
			\item $s(I, J) := s(I, K-1), s(K,J), (I \leq K \leq J).$
			\item $s(I,J):=a(I), s(I+1, J-1),b(J)$
		\end{itemize}
		\item Запросы:
		\begin{itemize}
			\item $?- s(I, J)$
		\end{itemize}
	\end{enumerate}
\end{example}

Таким образом мы описали на даталоге строку через набор фактов, грамматику, указанную выше, через набор фактов и правил и сделали запрос на все цепочки, выводящиеся из S. 

\textbf{NB!} Обратите внимание, что строки, начинающиеся с большой буквы, в даталоге считаются переменными. Также важно, что все переменные неявно квантифицированны.

\textbf{NB!} В даталоге отсутствуют функциональный символы (но есть в реализациях), а отрицание ограничено.
\subsection{Datalog для работы с графами}
На даталоге также можно задавать графы и писать к ним запросы.
\begin{example}
	Пример описания графа на даталоге.\\
	\begin{center}
		\begin{tikzpicture}[shorten >=1pt,on grid,auto]
		\node[state] (q_0)   {$0$};
		\node[state] (q_1) [above right=of q_0] {$1$};
		\node[state] (q_2) [right=of q_0] {$2$};
		\node[state] (q_3) [right=of q_2] {$3$};
		\path[->]
		(q_0) edge  node {$a$} (q_1)
		(q_1) edge  node {$a$} (q_2)
		(q_2) edge  node {$a$} (q_0)
		(q_2) edge[bend left, above]  node {$b$} (q_3)
		(q_3) edge[bend left, below]  node {$b$} (q_2);
		\end{tikzpicture}
	\end{center}
	$a(0, 1).\\
	a(1, 2).\\
	a(2, 0).\\
	b(2, 3).\\
	b(3, 2).\\
	S(I,I).\\$
\end{example}

Теперь зададим рассмотренную выше грамматику для работы с графом.\\
$s(I, I).\\
s(I,J) := s(I, K-1),s(K, J).\\
s(I, J):= a(I, L),S(L,M),b(M,J).\\$
Тогда запрос $?-s(I, J)$ выдаст нам все возможные пути в графе.\\

\subsection{Алгоритм Эрли}
Для парсинга контекстно-свободных грамматик может использоваться алгоритм Эрли.
Рассмотрим грамматику $G=(N,T,P,S)$, слово $a_1...a_n$,
и правило $A \rightarrow \alpha\beta$. Будем считать, что утверждение
$[A\rightarrow \alpha \bullet \beta](i,j), j \in [1..n]$ является истиной, если верно, что:
\begin{itemize}
	\item $\alpha \xrightarrow{\smash{*}} a_{i+1}...a_j$ (Последовательность выводится из $\alpha$)
	\item $S \xrightarrow{\smash{*}} a_1...a_jA_\gamma$ 
\end{itemize}


Рассмотрим правила вывода для подобных утверждений:\\

$\frac{S \rightarrow \alpha \in P}{[S \rightarrow \bullet \alpha](0,0)}$ Инициализация\\

$\frac{\left[A \rightarrow \alpha \bullet a_{j+1} \beta\right](i, j)}{\left[A \rightarrow \alpha a_{j+1} \bullet \beta\right](i, j+1)}$ Сканирование\\

$\frac{[A \rightarrow \alpha \cdot B \beta](i, j) \quad B \rightarrow \gamma \in P}{[B \rightarrow \bullet \gamma](j, j)}$ Предсказание\\

$\frac{[A \rightarrow \alpha \cdot B](i, j) \quad[B \rightarrow \gamma \bullet](j, k)}{[A \rightarrow \alpha B \bullet \beta](i, k)}$ Завершение\\

Идея алгоритма Эрли заключается в том, чтобы, начиная с инициализации, используя правила, вывести утверждение, содержащие данную строку слева от точки, и ничего справа, или попробовать все возможные выводы и признать, что строка не выводима.

Сложность алгоритма Эрли составляет $O(|P|^2n^3)$

\begin{example} Пример начала одной из веток дерева вывода для алгоритма Эрли для рассматриваемой грамматики\\
	
	$\underline{S \rightarrow SS} \quad Init$\\
	$\underline{[S \rightarrow \bullet SS](0,0), \space S \rightarrow aSb} \quad Predict$\\
	$\underline{[S \rightarrow \bullet aSb](0,0)} \quad Scan$\\
	$\underline{[S \rightarrow a \bullet Sb](0,1), S \rightarrow \varepsilon} \quad Predict$\\
	$\underline{[S \rightarrow \bullet](0, 1), [S \rightarrow aS \bullet b](0,1)} \quad Complete$\\
	$\underline{[S \rightarrow aS \bullet b](0,1)} \quad Scan$\\
	$\underline{[S \rightarrow \bullet SS](0,0),[S \rightarrow aSb \bullet](0,2)} \quad Complete$\\
	$[\underline{S \rightarrow S \bullet S](0,2)}$\\
	$\cdot\cdot\cdot$
	
\end{example}

Сложность можно понизить, изменив правила ``Предсказание'' и ``Завершение'' таким образом:
\begin{itemize}
	\item $\frac{[A \rightarrow \alpha \cdot B \beta](i, j) \quad B \rightarrow \gamma \in P}{[B \rightarrow \bullet \gamma](j, j)} \Rightarrow$ $
	\frac{[A \rightarrow \alpha \cdot B \beta](i, j)}{? B(j)}; \quad \frac{? B(j) \quad B \rightarrow \gamma \in P}{[B \rightarrow \cdot \gamma](j, j)}$
	\item $\frac{[A \rightarrow \alpha \cdot B](i, j) \quad[B \rightarrow \gamma \bullet](j, k)}{[A \rightarrow \alpha B \bullet \beta](i, k)} \Rightarrow$ $\frac{[B \rightarrow \gamma \bullet](j, k)}{B(j, k)}; \quad \frac{[A \rightarrow \alpha \cdot B \beta](i, j) \quad B(j, k)}{[A \rightarrow \alpha B \cdot \beta](i, k)}$
\end{itemize}
Так, разложив каждое правило на два, мы избавляемся от необходимости перевычислять дерево разбора каждого нетерминала после того, как однократно вычислим, что он выводим (мемоизируем его). Получаем сложность $O(|P|)$.




