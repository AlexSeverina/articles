\chapter{От CFPQ к вычислению Datalog-запросов}\label{Subsection Datalog}
Рассмотрим грамматику  $S \rightarrow aSb \mid SS \mid \varepsilon$, заданную через набор предикатов:
\begin{itemize}
	\item $a(i, w)$ --- Предикат, соответствующий терминалу. Обращается в True, если на $i$-том месте в строке $w$ стоит символ $a$
	\item $S(i, j, w)$ --- Предикат, соответствующий нетерминалу. Обращается в True, если выполняется одно из условий:
	\begin{enumerate}
		\item $i == j \quad(\varepsilon)$
		\item $\exists k: i \leq k \leq j \And S(i, k - 1, w) \And S(k, j, w)  \quad (SS)$
		\item  ${a(i, w) \And S(i+1, j-1, w) \And b(j-1, w)} \quad (aSb)$
	\end{enumerate}
\end{itemize}

Таким образом $S(0,|w|,w)$ покажет, выводится ли строка $w$ из данной грамматики,
а $S(\_,\_,w)$ даст нам список всех цепочек внутри $w$, выводящихся из $S$.

\subsection{Datalog}
Datalog~\cite{Datalog}\footnote{\url{https://www.computer.org/csdl/journal/tk/1989/01/k0146/13rRUx0xPIQ}} --- декларативный логический язык программирования. Используется для написания запросов к дедуктивным базам данных\footnote{\url{https://en.wikipedia.org/wiki/Deductive_database}}. 

\begin{example}
	Пример программы на даталоге.
	\begin{enumerate}
		\item Набор фактов (часто факты находятся в базе данных):
		\begin{itemize}
			\item $a(0).$
			\item $b(1).$
			\item $a(2).$
			\item $b(3).$
			\item $s(I, I).$
		\end{itemize}
		\item Набор правил:
		\begin{itemize}
			\item $s(I, J) \coloneq s(I, K-1), s(K,J), (I \leq K \leq J)$
			\item $s(I,J)\coloneq a(I), s(I+1, J-1),b(J)$
		\end{itemize}
		\item Запросы:
		\begin{itemize}
			\item $?- s(I, J)$
		\end{itemize}
	\end{enumerate}
\end{example}

Таким образом мы описали на даталоге строку через набор фактов, грамматику, указанную выше, через набор фактов и правил и сделали запрос на все цепочки, выводящиеся из $S$. 

\textbf{NB!} Обратите внимание~\cite{Datalog}, что строки, начинающиеся с большой буквы, в даталоге считаются переменными. Также важно, что все переменные неявно квантифицированны.


\subsection{Datalog для работы с графами}
На даталоге также можно задавать графы и писать к ним запросы.
\begin{example}
	Пример описания графа на даталоге.
	\begin{center}
		\begin{tikzpicture}[shorten >=1pt,on grid,auto]
		\node[state] (q_0)   {$0$};
		\node[state] (q_1) [above right=of q_0] {$1$};
		\node[state] (q_2) [right=of q_0] {$2$};
		\node[state] (q_3) [right=of q_2] {$3$};
		\path[->]
		(q_0) edge  node {$a$} (q_1)
		(q_1) edge  node {$a$} (q_2)
		(q_2) edge  node {$a$} (q_0)
		(q_2) edge[bend left, above]  node {$b$} (q_3)
		(q_3) edge[bend left, below]  node {$b$} (q_2);
		\end{tikzpicture}
	\end{center}
\begin{itemize}
	\item[] $a(0, 1).$
	\item[] $a(1, 2).$
	\item[] $a(2, 0).$
	\item[]	$b(2, 3).$
	\item[]	$b(3, 2).$
\end{itemize}
	
\end{example}

Теперь зададим рассмотренную выше грамматику для работы с графом.
\begin{itemize}
	\item[] $s(I, I).$
	\item[] $s(I,J) \coloneq s(I, K-1),s(K, J).$
	\item[] $s(I, J) \coloneq a(I, L),S(L,M),b(M,J).$
\end{itemize}

Тогда запрос $?-s(I, J)$ выдаст нам все такие пути в графе, что последовательность составляющих их вершин в порядке прохождения выводится их грамматики.

\subsection{Алгоритм Эрли}
Для распознавания контекстно-свободных грамматик может использоваться алгоритм Эрли~\cite{Earley} \footnote{\url{https://en.wikipedia.org/wiki/Earley}}.
Рассмотрим грамматику $G=(N,T,P,S)$, слово $a_1...a_n$,
и правило $A \rightarrow \alpha\beta$. Будем считать, что утверждение
$[A\rightarrow \alpha \bullet \beta](i,j), j \in [1..n]$ является истиной, если верно, что:
\begin{itemize}
	\item $\alpha \xrightarrow{\smash{*}} a_{i+1}...a_j$ (Последовательность выводится из $\alpha$)
	\item $S \xrightarrow{\smash{*}} a_1...a_jA_\gamma$ 
\end{itemize}


Рассмотрим правила вывода для подобных утверждений:

\begin{enumerate}
	\item $\frac{S \rightarrow \alpha \in P}{[S \rightarrow \bullet \alpha](0,0)}$ Инициализация (Init)
	
	\item $\frac{\left[A \rightarrow \alpha \bullet a_{j+1} \beta\right](i, j)}{\left[A \rightarrow \alpha a_{j+1} \bullet \beta\right](i, j+1)}$ Сканирование (Scan)
	
	\item $\frac{[A \rightarrow \alpha \cdot B \beta](i, j) \quad B \rightarrow \gamma \in P}{[B \rightarrow \bullet \gamma](j, j)}$ Предсказание (Predict)
	
	\item $\frac{[A \rightarrow \alpha \cdot B](i, j) \quad[B \rightarrow \gamma \bullet](j, k)}{[A \rightarrow \alpha B \bullet \beta](i, k)}$ Завершение (Complete)
	
\end{enumerate}

Идея алгоритма Эрли заключается в том, чтобы, начиная с инициализации, используя правила, вывести утверждение, содержащие данную строку слева от точки, и ничего справа, или попробовать все возможные выводы и признать, что строка не выводима.

Сложность алгоритма Эрли составляет $O(|P|^2n^3)$

\begin{example} Пример начала одной из веток дерева вывода для алгоритма Эрли для рассматриваемой грамматики
	
	$$\underline{S \rightarrow SS} \quad Init$$
	$$\underline{[S \rightarrow \bullet SS](0,0), \space S \rightarrow aSb} \quad Predict$$
	$$\underline{[S \rightarrow \bullet aSb](0,0)} \quad Scan$$
	$$\underline{[S \rightarrow a \bullet Sb](0,1), S \rightarrow \varepsilon} \quad Predict$$
	$$\underline{[S \rightarrow \bullet](0, 1), [S \rightarrow aS \bullet b](0,1)} \quad Complete$$
	$$\underline{[S \rightarrow aS \bullet b](0,1)} \quad Scan$$
	$$\underline{[S \rightarrow \bullet SS](0,0),[S \rightarrow aSb \bullet](0,2)} \quad Complete$$
	$$[\underline{S \rightarrow S \bullet S](0,2)}$$
	$$\cdot\cdot\cdot$$
	
\end{example}

Сложность можно понизить, изменив правила ``Предсказание'' и ``Завершение'' таким образом:
\begin{itemize}
	\item $\frac{[A \rightarrow \alpha \cdot B \beta](i, j) \quad B \rightarrow \gamma \in P}{[B \rightarrow \bullet \gamma](j, j)} \Rightarrow$ $
	\frac{[A \rightarrow \alpha \cdot B \beta](i, j)}{? B(j)}; \quad \frac{? B(j) \quad B \rightarrow \gamma \in P}{[B \rightarrow \cdot \gamma](j, j)}$
	\item $\frac{[A \rightarrow \alpha \cdot B](i, j) \quad[B \rightarrow \gamma \bullet](j, k)}{[A \rightarrow \alpha B \bullet \beta](i, k)} \Rightarrow$ $\frac{[B \rightarrow \gamma \bullet](j, k)}{B(j, k)}; \quad \frac{[A \rightarrow \alpha \cdot B \beta](i, j) \quad B(j, k)}{[A \rightarrow \alpha B \cdot \beta](i, k)}$
\end{itemize}
Так, разложив каждое правило на два, мы избавляемся от необходимости перевычислять дерево разбора каждого нетерминала после того, как однократно вычислим, что он выводим (мемоизируем его). Получаем сложность $O(|P|)$.

Описанный подход мемоизации~\cite{Magic} части используется для оптимизации программ на даталоге. Можно либо видоизменять правила, задающие грамматику в тексте программы, либо модифицировать компилятор, чтобы он пытался сделать это автоматически. 
\section{Вопросы и задачи}
\begin{enumerate}
  \item Написать синтаксический анализатор раз.
  \item Написать синтаксический анализатор два.
  \item Побаловаться с неоднозначными грамматиками
  \item Побаловаться с конъюнктивными грамматиками.
  \item Графы?
\end{enumerate}