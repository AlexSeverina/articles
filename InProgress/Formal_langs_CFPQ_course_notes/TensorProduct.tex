\section{Через тензорное произведение}

\subsection{Рекурсивные автоматы}

Определение, примеры.

\subsection{Тензорное произведение}

Матриц и графов

Сперва дадим классическое определение тензорного произведения двух неориентированных графов.
\begin{definition}

\end{definition}

Для того, чтобы построить тензорное произведение ориентированных графов, необходимо в предыдущем определении, в условии существования реба в результирующем графе, дополнительно потребовать, чтобы направления рёбер совпадали.
Таким образом, тензорное произведение ориентированных графов можно определить следующим образом.
\begin{definition}

\end{definition}

Осталось добавить метки к рёбрам.
Это приведёт к огичному усилению требованя к существованию ребра: метки рёбер в исходных графах должны совпадать.
Таким образом, мы получаем следующее определение тензорного произведения ориентированных графов с метками на рёбрах.
\begin{definition}

Пусть есть два ориентированных графа: $\mathcal{G}_1 = \langle V_1, E_1, L_1 \rangle$ и $\mathcal{G}_2 = \langle V_2, E_2, L_2 \rangle$.
Тензорнымм

\end{definition}


\subsection{Алгоритм}

По грамматике строим автомат.

В цикле: пересекли через тензорное произведениеб замкнули, чтобы нацти пути из начальной в конечную в граммтике, поставили туда нетерминалы.

Можно вычислять только разницу.
Для этого, правда, потребуется держать ещё одну матрицу.
И надо проверять, что вычислительно дешевле: поддерживать разницу и потом каждый раз поэлементно складывать две матрицы или каждый раз вычислять полностью произведение.

Всего несколько матриц.
Разреженные.
Необходимо отметить, что для реальных графов и запросов результат тензорного произведения будет очень разрежен.
На готовых либах должно быть быстро.
