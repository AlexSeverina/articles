\section{Алгоритм на матричных опреациях}


\subsection{Транзитивне замыкание графа}

Флойд-Уоршал, матрицы.

Рассуждения про субкубичность.
Про то, что булево полукольцо.

\subsection{Нормальная форма Хомского}

Данный алгоритм накладывает ограничение на форму грамматики: грамматика должна быть в ``ослабленной'' нормальной форме Хомского.

Определение НФХ.

Любую КС граммтику можно преобразовать в НФХ.

\subsection{Алгоритм CYK}

Построение и заполнение таблицы.

\subsection{Алгоритм для графов на основе CYK}

Обобщение. 
Смотрим на транзитивное замыкание.

\subsection{Алгоритм на основе матриц}

Ссылка на Валианта~\cite{!!!}.

Оригинальные матрицы (Рустам)~\cite{Azimov:2018:CPQ:3210259.3210264}

Кратчайшие.

Одно дерево?

\subsection{Конъюнктивные и булевы граммтики}

\subsubsection{Определения}

Охотин~\cite{!!!}.

\subsubsection{Для графов}

Классическая семантика --- работает для конъюнктивных для любых графов.
Для Булевых и конъюнктивных только для графов без циклов.

Альтернативная семантика (DataLog).
Трактуем конъюнкцию как в даталоге. Тогда всё хорошо.
Это похоже на даталог через линейную алгебру~\cite{!!!}

\subsection{Особенности реализации матричного алгоритма}

Кое-что про наши реализации~\cite{Mishin:2019:ECP:3327964.3328503}

Разреженные матрицы, плотные матрицы, GraphBLAS\footnote{!!!}, GPGPU, CUTLASS\footnote{Репозиторий библиотеки CUTLASS: \url{https://github.com/NVIDIA/cutlass}}.

Расположенеи в памяти: хорошо, когда всё влезло на одну карту.

Распределённые решения. 
Через GraphBLAS

\subsection{Обзор}

Китайцы~\cite{!!!}, Брэдфорд~\cite{!!!}, Ли~\cite{!!!}, Хеллингс~\cite{!!!}, OpenCypher~\cite{Kuijpers:2019:ESC:3335783.3335791}.


Рассуждения про ассимптотику.

Субкубический для частного случая (Брэдфорд)~\cite{8249039}.

\subsection{Вопросы и задачи}
\begin{enumerate}
  \item !!!
\end{enumerate}
