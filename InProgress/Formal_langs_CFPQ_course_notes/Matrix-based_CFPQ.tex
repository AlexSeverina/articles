\section{Алгоритм на матричных операциях}
\subsection{Описание}
\label{Matrix-CFPQ}
В главе~\ref{graph:CYK}~был изложен алгоритм для графов на основе CYK. Заметим, что обход матрицы напоминает матричное умножение:

\begin{gather*}
M_1 \times M_2 = M_3 \\
M_3[i,j] = \sum_{k=1}^{n} M[i,k] * M[k,j]
\end{gather*}
, где
\begin{gather*}
\sum_{k=1}^{n} = \bigcup_{k=1}^{n} \\
S_1 * S_2 = \{N_1^0 ... N_1^m\} * \{N_2^0 ... N_2^l\} = \{N_3~|~(N_3 \rightarrow N_1^i N_2^j) \in P\}
\end{gather*}

Для линейного входа существует алгоритм Валианта~\cite{Valiant:1975:GCR:1739932.1740048}, но он не обобщается на графы с сохранением асимптотики~\cite{Yannakakis}, поэтому рассмотрим идею, изложенную в статье Рустама Азимова~\cite{Azimov:2018:CPQ:3210259.3210264}.

Пусть $\mathcal{G} = (V, E)$ --- входной граф и $G = (N,\Sigma,P)$ --- входная грамматика.

\begin{algorithm}[H]
\begin{algorithmic}[1]
\caption{Context---free recognizer for graphs}
\label{alg:graphParse}
\Function{contextFreePathQuerying}{$\mathcal{G}$, G}
    
    \State{$n \gets$ количество узлов в $\mathcal{G}$}
    \State{$E \gets$ направленные ребра в $\mathcal{G}$}
    \State{$P \gets$ набор продукций из $G$}
    \State{$T \gets$ матрица $n \times n$, в которой каждый элемент $\varnothing$}
    \ForAll{$(i,x,j) \in E$}
    \Comment{Инициализация матрицы}
        \State{$T_{i,j} \gets T_{i,j} \cup \{A~|~(A \rightarrow x) \in P \}$}
    \EndFor    
    \While{матрица $T$ меняется}
       
        \State{$T \gets T \cup (T \times T)$}
        \Comment{Вычисление транзитивного замыкания} 
    \EndWhile
\State \Return $T$
\EndFunction
\end{algorithmic}
\end{algorithm}


\begin{example}[Пример работы]

Пусть есть граф $\mathcal{G}$:
\begin{center}
	\begin{tikzpicture}[shorten >=1pt,on grid,auto]
	\node[state] (q_0)   {$1$};
	\node[state] (q_1) [above right=of q_0] {$2$};
	\node[state] (q_2) [right=of q_0] {$0$};
	\node[state] (q_3) [right=of q_2] {$3$};
	\path[->]
	(q_0) edge  node {a} (q_1)
	(q_1) edge  node {a} (q_2)
	(q_2) edge  node {a} (q_0)
	(q_2) edge[bend left, above]  node {b} (q_3)
	(q_3) edge[bend left, below]  node {b} (q_2);
	\end{tikzpicture}
	
\end{center}

и грамматика $G$:
\begin{align*}
S   &\to A B \\ 
S  &\to A S_1 \\ 
S_1 &\to S B \\
A  &\to a \\ 
B   &\to b
\end{align*}

Тогда $T_0$, полученная напрямую из графа, выглядит так:


\[
T_0 = \begin{pmatrix}
    \varnothing & \{A\}       & \varnothing & \{B\}       \\
    \varnothing & \varnothing & \{A\}       & \varnothing \\
    \{A\}       & \varnothing & \varnothing & \varnothing \\
    \{B\}       & \varnothing & \varnothing & \varnothing \\
\end{pmatrix}
\]

Пусть $T_i$ --- матрица, полученная из $T$ после применения цикла на строках \textbf{8-9} алгоритма~\ref{alg:graphParse} $i$ раз. Далее показано получение матрицы $T_1$.


\[
T_0 \times T_0 = \begin{pmatrix}
    \varnothing & \varnothing & \varnothing & \varnothing \\
    \varnothing & \varnothing & \varnothing & \varnothing \\
    \varnothing & \varnothing & \varnothing & \{S\}       \\
    \varnothing & \varnothing & \varnothing & \varnothing \\
\end{pmatrix}
\]

\[
T_1 = T_0 \cup (T_0 \times T_0) = \begin{pmatrix}
    \varnothing & \{A\}       & \varnothing & \{B\}       \\
    \varnothing & \varnothing & \{A\}       & \varnothing \\
    \{A\}       & \varnothing & \varnothing & \{\pmb{S}\}       \\
    \{B\}       & \varnothing & \varnothing & \varnothing \\
\end{pmatrix}
\]

Когда алгоритм находит новые пути в графе $\mathcal{G}$, он добавляет соответствующие нетерминалы в матрицу $T$. Например, после первого цикла нетерминал $S$ добавляется к матрице $T$. Этот нетерминал добавляется в ячейку $[2,3]$. Это означает, что существует такой путь $\pi$ из вершины 2 в вершину 3, что $S \xrightarrow{*} \omega(\pi)$. В данном примере путь состоит из двух ребер с именами $a$ и $b$, так что $S \xrightarrow{*} ab$.

Вычисление транзитивного замыкания заканчивается через $k$ итераций, когда достигается фиксированная точка: $T_{k-1} = T_k$. Для данного примера $k = 13$, так как $T_{13} = T_{12}$. Процесс вычисления транзитивного замыкания показан ниже (на каждой итерации новые элементы выделены жирным).


\begin{alignat*}{7}
& &&T_2 &&= \begin{pmatrix}
\varnothing & \{A\}       & \varnothing & \{B\}       \\
\varnothing & \varnothing & \{A\}       & \varnothing \\
\{A, \pmb{S_1}\}  & \varnothing & \varnothing & \{S\}       \\
\{B\}       & \varnothing & \varnothing & \varnothing \\
\end{pmatrix} \ \ \ \ &&T_3 &&= \begin{pmatrix}
\varnothing & \{A\}       & \varnothing & \{B\}       \\
\{\pmb{S}\}       & \varnothing & \{A\}       & \varnothing \\
\{A, S_1\}  & \varnothing & \varnothing & \{S\}       \\
\{B\}       & \varnothing & \varnothing & \varnothing \\
\end{pmatrix} \\ & &&T_4 &&= \begin{pmatrix}
\varnothing & \{A\}       & \varnothing & \{B\}       \\
\{S\}       & \varnothing & \{A\}       & \{\pmb{S_1}\}     \\
\{A, S_1\}  & \varnothing & \varnothing & \{S\}       \\
\{B\}       & \varnothing & \varnothing & \varnothing \\
\end{pmatrix}  \ \ \ \ &&T_5 &&= \begin{pmatrix}
\varnothing & \{A\}       & \varnothing & \{B, \pmb{S}\}    \\
\{S\}       & \varnothing & \{A\}       & \{S_1\}     \\
\{A, S_1\}  & \varnothing & \varnothing & \{S\}       \\
\{B\}       & \varnothing & \varnothing & \varnothing \\
\end{pmatrix} \\ & &&T_6 &&= \begin{pmatrix}
\{\pmb{S_1}\}     & \{A\}       & \varnothing & \{B, S\}    \\
\{S\}       & \varnothing & \{A\}       & \{S_1\}     \\
\{A, S_1\}  & \varnothing & \varnothing & \{S\}       \\
\{B\}       & \varnothing & \varnothing & \varnothing \\
\end{pmatrix} \ \ \ \ &&T_7 &&= \begin{pmatrix}
\{S_1\}     & \{A\}       & \varnothing & \{B, S\}    \\
\{S\}       & \varnothing & \{A\}       & \{S_1\}     \\
\{A, S_1, \pmb{S}\}  & \varnothing & \varnothing & \{S\}    \\
\{B\}       & \varnothing & \varnothing & \varnothing \\
\end{pmatrix}  \\
& &&T_8 &&= \begin{pmatrix}
\{S_1\}     & \{A\}       & \varnothing & \{B, S\}    \\
\{S\}       & \varnothing & \{A\}       & \{S_1\}     \\
\{A, S_1, S\}  & \varnothing & \varnothing & \{S, \pmb{S_1}\} \\
\{B\}       & \varnothing & \varnothing & \varnothing \\
\end{pmatrix} \ \ \ \ &&T_9 &&= \begin{pmatrix}
\{S_1\}     & \{A\}       & \varnothing & \{B, S\}    \\
\{S\}       & \varnothing & \{A\}       & \{S_1, \pmb{S}\}     \\
\{A, S_1, S\}  & \varnothing & \varnothing & \{S, S_1\} \\
\{B\}       & \varnothing & \varnothing & \varnothing \\
\end{pmatrix} \\ & &&T_{10} &&= \begin{pmatrix}
\{S_1\}     & \{A\}       & \varnothing & \{B, S\}    \\
\{S, \pmb{S_1}\}       & \varnothing & \{A\}       & \{S_1, S\}     \\
\{A, S_1, S\}  & \varnothing & \varnothing & \{S, S_1\} \\
\{B\}       & \varnothing & \varnothing & \varnothing \\
\end{pmatrix}  \ \ \ \  &&T_{11} &&= \begin{pmatrix}
\{S_1, \pmb{S}\}     & \{A\}       & \varnothing & \{B, S\}    \\
\{S, S_1\}       & \varnothing & \{A\}       & \{S_1, S\}     \\
\{A, S_1, S\}  & \varnothing & \varnothing & \{S, S_1\} \\
\{B\}       & \varnothing & \varnothing & \varnothing \\
\end{pmatrix} \\ & &&T_{12} &&= \begin{pmatrix}
\{S_1, S\}     & \{A\}       & \varnothing & \{B, S, \pmb{S_1}\}    \\
\{S, S_1\}       & \varnothing & \{A\}       & \{S_1, S\}     \\
\{A, S_1, S\}  & \varnothing & \varnothing & \{S, S_1\} \\
\{B\}       & \varnothing & \varnothing & \varnothing \\
\end{pmatrix} \ \ \ \ &&T_{13} &&= \begin{pmatrix}
\{S_1, S\}     & \{A\}       & \varnothing & \{B, S, S_1\}    \\
\{S, S_1\}       & \varnothing & \{A\}       & \{S_1, S\}     \\
\{A, S_1, S\}  & \varnothing & \varnothing & \{S, S_1\} \\
\{B\}       & \varnothing & \varnothing & \varnothing \\
\end{pmatrix}
\end{alignat*}

Таким образом, результат алгоритма~\ref{alg:graphParse} для нашего примера --- это матрица $T_{13} = T_{12}$. 

\end{example}

\subsection{Конъюнктивные и булевы грамматики}

\subsubsection{Определения}

Впервые конъюнктивные и булевы грамматики были предложены Александром Охотиным~\cite{DBLP:journals/jalc/Okhotin01,Okhotin:2003:BG:1758089.1758123}. Дадим определение конъюнктивной грамматики.

\begin{definition}
	\textit{Конъюнктивной грамматикой} называется $G = (\Sigma,N,P,S)$, где:
	\begin{itemize}
		\item $\Sigma$ и $N$ --- дизъюнктивные конечные непустые множества терминалов и нетерминалов.
		\item $P$ --- конечное множество продукций, каждая вида
		\[
		A\rightarrow \alpha_1\&...\&\alpha_n
		\]
		,где $A \in N,n \geq 1$ и $\alpha_1,...,\alpha_n \in (\Sigma \cup N)^*$.
		\item $S \in N$  --- стартовый нетерминал.
	\end{itemize}
\end{definition}

Конъюнктивная грамматика генерирует строки, выводя их из начального символа, так же, как это происходит в контекстно-свободных грамматиках в параграфе~\ref{CFG}. Промежуточные строки, используемые в процессе вывода, являются формулами следующего вида:

\begin{definition}\label{Definition of conjunctive formula}
	Пусть $G = (\Sigma,N,P,S)$ --- конъюнктивная грамматика. Множество конъюнктивных формул $ \mathcal{F}$ определяется индуктивно:
	\begin{itemize}
		\item Пустая строка $\varepsilon$ --- конъюнктивная формула. 
		\item Любой символ из $(\Sigma \cup N)$ --- формула.
		\item Если $\mathcal{A}$ и $\mathcal{B}$ непустые формулы, тогда $\mathcal{AB}$ --- формула.
		\item Если $\mathcal{A}_1,\ldots,\mathcal{A}_n$ $(n \geqslant 1)$ --- формула, тогда $(\mathcal{A}_1\&\ldots\&\mathcal{A}_n)$ --- формула.
	\end{itemize}
\end{definition}

\begin{definition}
	Пусть $G = (\Sigma,N,P,S)$ --- конъюнктивная грамматика. Аналогично определению отношения непосредственной выводимости в контекстно-свободной грамматике~\ref{def derivability in CFG} определим $\xRightarrow[G]{}$ как отношение непосредственной выводимости на множестве конъюнктивных формул.
	\begin{itemize}
	    \item Любой нетерминал в любой формуле может быть перезаписан телом любого правила для этого терминала заключенным в скобки. То есть для любых $s^{'},s^{''} \in (\Sigma \cup N \cup \{(, \&, )\})^*$ и $A\in N$, таких что $s^{'}As^{''}$ --- формула, и для всех правил вида $A \rightarrow \alpha_1\&\ldots\&\alpha_n \in P$, имеем $s^{'}As^{''}\xRightarrow[G]{}s^{'}(\alpha_1\&\ldots\&\alpha_n)s^{''}$. 
	    \item Если формула содержит подформулу в виде конъюнкции одной или нескольких одинаковых терминальных строк, заключенных в скобки, тогда подформула может быть перезаписана терминальной строкой без скобок. То есть для любых $s^{'},s^{''} \in (\Sigma \cup N \cup \{(, \&, )\})^*$, $(n \geqslant 1)$ и $w \in \Sigma^*$, таких что $s^{'}(w\&\ldots\&w)s^{''}$ --- формула, имеем $s^{'}(w\&\ldots\&w)s^{''}\xRightarrow[G]{}s^{'}ws^{''}$.
	\end{itemize}
	Как и в случае контекстно-свободной грамматики обозначим $\xRightarrow[G]{}^*$ рефлексивное транзитивное замыкание отношения $\xRightarrow[G]{}$.
\end{definition}

\begin{definition}
	Пусть $G = (\Sigma,N,P,S)$ --- конъюнктивная грамматика. Язык, порождаемый формулой, это множество всех терминальных строк выводимых из этой формулы: $L_{G}(\mathcal{A}) = \{w\in\Sigma^* \mid \mathcal{A} \xRightarrow[G]{}^*w\}$. Очевидно, что язык порождаемый грамматикой, это язык порождаемый стартовым нетерминалом $S$ : $L(G) = L_{G}(S) = L(S)$.
\end{definition}

\begin{theorem}\label{Theorem language generated by a formula}
    Пусть $G = (\Sigma,N,P,S)$ --- конъюнктивная грамматика. Пусть $\mathcal{A}_1,\ldots,\mathcal{A}_n,\mathcal{B}$ --- формулы, $A \in N$, $a \in \Sigma$. Тогда,
    \begin{enumerate}
        \item $L(\varepsilon) = \{\varepsilon\}$.
        \item $L(a) = \{a\}$.
        \item $L(A) = \bigcup_{A \rightarrow \alpha_1\&\ldots\&\alpha_n \in P} L((\alpha_1\&\ldots\&\alpha_m))$.
        \item $L(\mathcal{AB}) = L(\mathcal{A})*L(\mathcal{B})$
        \item $L((\mathcal{A}_1\&\ldots\&\mathcal{A}_n)) = \bigcap_{i = 1}^{n}L(\mathcal{A}_i)$.
    \end{enumerate}
\end{theorem}

Теорема~\ref{Theorem language generated by a formula} уже подразумевает интерпретацию грамматики как системы уравнений. Используем математический подход, чтобы лучше охарактеризовать конъюнктивные языки с помощью систем уравнений.

\begin{definition}[Выражение]
    Пусть $\Sigma$ конечный непустой алфавит. Пусть $X = \{X_1,\ldots,X_N\}$ вектор переменных. Выражение над алфавитом $\Sigma$, зависящее от переменных $X$, определяется индуктивно:
    \begin{itemize}
	   \item $\varepsilon$ --- выражение.
	   \item Любой символ $a\in\Sigma$ --- выражение.
	   \item Любая переменная $X_i\in X$ --- выражение.
	   \item Если $\phi_1$ и $\phi_2$ выражения, то $\phi_1\phi_2, (\phi_1\mid\phi_2), (\phi_1\&\phi_2)$ также выражения.
	\end{itemize}
	Заметим, что любая формула, в терминах определения~\ref{Definition of conjunctive formula}, является выражением, где нетерминалы формулы это переменные выражения. С другой стороны, любое выражение, не содержащее дизъюнкции, формула.
\end{definition}

Предположим, что переменные $X_i$ приняли в качестве значений слова из языка над алфавитом $\Sigma$. Определим значение всего выражения.

\begin{definition}[Значение выражения]\label{Value of conjunctive expression}
    Пусть $L = (L_1,\ldots,L_n) (L_i \subseteq \Sigma^*)$ вектор из $n$ языков над $\Sigma$, где $n \geqslant 1$. Пусть $\phi$ выражение над $\Sigma$, зависящее от переменных $X_1,\ldots,X_n$. Значение выражения $\phi$ на векторе $L$ --- это язык над тем же алфавитом $\Sigma$. Обозначим его $\phi(L)$ и определим индуктивно на структуре выражения:
    \begin{itemize}
	   \item $\varepsilon(L) = \{\varepsilon\}$.
	   \item $a(L) = \{a\}$ для любого $a\in\Sigma$.
	   \item $X_i(L) = L_i$ для любого $X_i \in X$.
	   \item $\phi_1\phi_2 = \phi_1(L) * \phi_2(L), (\phi_1\mid\phi_2)(L) = \phi_1(L) \cup \phi_2(L), (\phi_1\&\phi_2)(L) = \phi_1(L) \cap \phi_2(L)$ для любых выражений $\phi_1$ и $\phi_2$.
	\end{itemize}
\end{definition}

Обобщим определение~\ref{Value of conjunctive expression} на случай вектора выражений.

\begin{definition}[Значение вектора выражений]
    Пусть $L = (L_1,\ldots,L_n) (L_i \subseteq \Sigma^*)$ вектор из $n$ языков над $\Sigma$, где $n \geqslant 1$. Пусть $\phi_1,\ldots,\phi_m$ выражения над $\Sigma$, зависящее от переменных $X_1,\ldots,X_n$. Значение вектора выражений $P = (\phi_1,\ldots,\phi_m)$ на векторе $L$ --- это вектор языков $P(L) = (\phi_1(L),\ldots,\phi_m(L))$ над тем же алфавитом $\Sigma$. 
\end{definition}

Зададим частичный порядок относительно включения $``\preccurlyeq"$ на множестве языков и расширим его на вектора языков длины $n$: $(L_1^{'},\ldots,L_n^{'})\preccurlyeq(L_1^{''},\ldots,L_n^{''})$ если и только если $L_i^{'} \subseteq L_i^{''}$ для любого $1\leqslant i \leqslant n$

\begin{definition}\label{Definition a conjuctive system of equations}
   $X = P(X)$ система уравнений над алфавитом $\Sigma$ и $X = \{X_1,\ldots,X_n\}$, где $P = (\phi_1,\ldots,\phi_n)$ вектор выражений над алфавитом $\Sigma$, зависящий от $X$.
   
   Вектор языков $L = (L_1,\ldots,L_n)$ является решением системы уравнений если $L = P(L)$.
   
   Наименьшее решение $L$ это вектор языков, такой что для любого другого сравнимого вектора языков $L^{'}$ выполняется $L \preccurlyeq L^{'}$.
\end{definition}

Заметим, что оператор $P$ на множестве $2^{\Sigma}\times\ldots\times2^{\Sigma}$, что решение системы~\ref{Definition a conjuctive system of equations} это неподвижная точка $P$ и что наименьшее решение системы это наименьшая неподвижная точка оператора $P$.

\begin{theorem}\label{Theorem of a least fixed point solution}
    Для любой системы из определения~\ref{Definition a conjuctive system of equations} с переменными $X_1,\ldots,X_n$, оператор $P = (\phi_1,\ldots,\phi_n)$ имеет наименьшую неподвижную точку $L = (L_1,\ldots,L_n) = \lim_{i\to\infty}P^{i}\underbrace{(\varnothing,\ldots,\varnothing)}_n$.
\end{theorem}

Приведем пример конъюнктивной грамматики.

\begin{example}[Пример конъюнктивной грамматики]
	Следующая конъюнктивная грамматика $G$ порождает язык $\{a^nb^nc^n\mid n \geq 0\}$:
	
    \begin{align*}
    1.\ S   &\to A B \& D C \\
    2.\ A  &\to a A \mid \varepsilon \\ 
	3.\ B &\to b B c \mid \varepsilon \\
	4.\ C   &\to c C \mid \varepsilon \\ 
	5.\ D   &\to aDb \mid \varepsilon
    \end{align*}
	
	Легко видеть, что $L(AB) = \{a^ib^jc^k\mid j = k\}$ и $L(DC) = \{a^ib^jc^k\mid i = j\}$. Тогда $L(S) = L(AB) \cap L(DC) = \{a^nb^nc^n\mid n \geq 0\}$. 
	
	В этой грамматике строка $abc$ может быть получена следующим образом. Для начала представим грамматику в виде системы уравнений:
	\begin{align*}
	S &= A B \cap D C \\ 
	A &= \{a\}A \cup \varepsilon \\ 
	B &= \{b\}B\{c\} \cup \varepsilon \\
	C &= \{c\}C \cup \varepsilon \\ 
	D &= \{a\}D\{b\} \cup \varepsilon
	\end{align*}
	Используя теорему~\ref{Theorem of a least fixed point solution}, будем итеративно вычислять $P^{i}\underbrace{(\varnothing,\ldots,\varnothing)}_5$. На каждом шаге будем подставлять все терминальные строки из языков, порожденных нетерминалами на предыдущем шаге, в соответствующие нетерминалы правой части каждого уравнения и записывать получившиеся терминальные строки в языки нетерминалов текущего шага. Продолжаем до тех пор пока язык, порождаемый нетерминалом $S$, не будет содержать терминальную строку $``abc''$.
	\begin{enumerate}
        \item На начальном этапе имеем $P^{0}(\varnothing,\ldots,\varnothing) = (S: \varnothing, A: \varnothing, B: \varnothing, C: \varnothing, D: \varnothing)$ 
        \item Подставляем в первое уравнение терминальные строки из шага 1 в соответствующие нетерминалы, т.е. 
        \begin{align*}
        	S:  \varnothing &= \varnothing\varnothing \cap \varnothing\varnothing \\ 
        	A: \{\varepsilon\} &= \{a\}\varnothing \cup \{\varepsilon\} \\ 
        	B: \{\varepsilon\} &= \{b\}\varnothing\{c\} \cup \{\varepsilon\} \\
        	C: \{\varepsilon\} &= \{c\}\varnothing \cup \{\varepsilon\} \\ 
        	D: \{\varepsilon\} &= \{a\}\varnothing\{b\} \cup \{\varepsilon\}
    	\end{align*}
    	В конце итерации получаем $P^{1}(\varnothing,\ldots,\varnothing) = (S: \varnothing, A: \{\varepsilon\}, B: \{\varepsilon\}, C: \{\varepsilon\}, D: \{\varepsilon\})$
    	\item Делаем еще одну итерацию,
    	\begin{align*}
        	S:  \{\varepsilon\} &= \{\varepsilon\}\{\varepsilon\} \cap \{\varepsilon\}\{\varepsilon\} \\ 
        	A: \{a, \varepsilon\} &= \{a\}\{\varepsilon\} \cup \{\varepsilon\} \\ 
        	B: \{bc, \varepsilon\} &= \{b\}\{\varepsilon\}\{c\} \cup \{\varepsilon\} \\
        	C: \{c, \varepsilon\} &= \{c\}\{\varepsilon\} \cup \{\varepsilon\} \\ 
        	D: \{ab, \varepsilon\} &= \{a\}\{\varepsilon\}\{b\} \cup \{\varepsilon\}
    	\end{align*}
    	В конце итерации получаем $P^{2}(\varnothing,\ldots,\varnothing) = (S: \{\varepsilon\}, A: \{a, \varepsilon\}, B: \{bc, \varepsilon\}, C: \{c, \varepsilon\}, D: \{ab, \varepsilon\})$
    	\item Еще одна итерация,
    	\begin{align*}
        	S:  \{\fbox{abc}, \varepsilon\} &= \{a, \varepsilon\}\{bc, \varepsilon\} \cap \{ab, \varepsilon\}\{c, \varepsilon\} \\ 
        	A: \{a, aa, \varepsilon\} &= \{a\}\{a, \varepsilon\} \cup \{\varepsilon\} \\ 
        	B: \{bc, bbcc, \varepsilon\} &= \{b\}\{bc, \varepsilon\}\{c\} \cup \{\varepsilon\} \\
        	C: \{c, cc, \varepsilon\} &= \{c\}\{c, \varepsilon\} \cup \{\varepsilon\} \\ 
        	D: \{ab, aabb, \varepsilon\} &= \{a\}\{ab, \varepsilon\}\{b\} \cup \{\varepsilon\}
    	\end{align*}
    	В конце итерации получили $P^{3}(\varnothing,\ldots,\varnothing) = (S: \{\fbox{abc}, \varepsilon\}, A: \{a, aa, \varepsilon\}, B: \{bc, bbcc, \varepsilon\}, C: \{c, cc, \varepsilon\}, D: \{ab, aabb, \varepsilon\})$. Заметим, что терминальная строка $``abc"$ появилась в языке, который порождает стартовый нетерминал $S$. Т.е. терминальная строка $``abc"$ выводима в грамматике $G$, что и требовалось показать.
    \end{enumerate}
    
    Заметим, что строку $``abc"$ также можно получить применением правил вывода. здесь цифра над стрелкой соответствует номеру примененного правила. 
	\begin{align*}
    	S &\xRightarrow{1}(AB\&DC) \\
    	&\xRightarrow{2}(aAB\&DC) \xRightarrow{2} (a\varepsilon B\&DC) \\
    	&\xRightarrow{3}(abBc\&DC) \xRightarrow{3}(ab\varepsilon c\&DC) \\
    	&\xRightarrow{5}(abc\&aDbC) \xRightarrow{5}(abc\&a\varepsilon bC) \\
    	&\xRightarrow{4}(abc\&abcC) \xRightarrow{4}(abc\&abc\varepsilon) \\
    	&\Rightarrow(abc\&abc) \Rightarrow abc
	\end{align*}
\end{example}

\begin{example}
    Конъюнктивная грамматика $G$ для языка $L = \{wcw \mid w \in \{a, b\}^*\}$:
    \begin{align*}
	S &\to C \& D \\ 
	C &\to aCa \mid aCb \mid bCa \mid bCb \mid c \\ 
	D &\to aA\&aD \mid bB\&bD \mid cE \\
	A &\to aAa \mid aAb \mid bAa \mid bAb \mid cEa \\
	B &\to aBa \mid aBb \mid bBa \mid bBb \mid cEb \\
	E &\to aE \mid bE \mid \varepsilon
	\end{align*}
\end{example}

Подробнее о конъюнктивных грамматиках можно прочитать в статьях~\cite{DBLP:journals/jalc/Okhotin01, Okhotin2002, DBLP:journals/tcs/Okhotin03a, f60a33d409364914be560cac0e54b12c}.

Дадим определение булевой грамматики.

\begin{definition}
	\textit{Булевой грамматикой} называется $G = (\Sigma,N,P,S)$, где:
	\begin{itemize}
		\item $\Sigma$ и $N$ --- дизъюнктивные конечные непустые множества терминалов и нетерминалов.
		\item $P$ --- конечное множество продукций, каждая вида
		\[
		A\rightarrow \alpha_1\&...\&\alpha_m\&\neg\beta_1\&...\&\neg\beta_n
		\]
		,где $A \in N, m, n >=0, m+n \geq 1$ и $\alpha_i,\beta_j \in (\Sigma \cup N)^*$.
		\item $S \in N$  --- стартовый нетерминал.
	\end{itemize}
\end{definition}

Приведем пример булевой грамматики.

\begin{example}
	Следующая булева грамматика порождает язык  $\{a^mb^nc^n\mid m,n \geq 0, m \neq n \}$:
	
	\begin{align*}
	S   &\to A B \& \neg D C \\ 
	A  &\to a A \mid \varepsilon \\ 
	B &\to b B c \mid \varepsilon \\
	C   &\to c C \mid \varepsilon \\ 
	D   &\to aDb \mid \varepsilon
	\end{align*}
	
	Очевидно, что $L(AB) = \{a^mb^nc^n\mid m,n \in \mathbb{N}\}$ и $L(DC) = \{a^nb^nc^m\mid m,n \in \mathbb{N}\}$. Тогда $L(AB)\cap\overline{L(DC)} = \{a^mb^nc^n\mid m,n \geq 0, m \neq n \}$.
\end{example}

Подробнее о булевых грамматиках можно прочитать в статьях~\cite{Okhotin:2003:BG:1758089.1758123,Okhotin:2014:PMM:2565359.2565379}.

\subsubsection{Матричный алгоритм для конъюнктивных грамматик}

Определим бинарную нормальную форму конъюнктивной грамматики.
\begin{definition}[Бинарная нормальная форма]
    Конъюнктивная грамматика $G = (\Sigma, N, P, S)$ находится в бинарной нормальной форме, если каждое правило из P имеет вид,
    \begin{itemize}
		\item $A \rightarrow B_1 C_1 \& \ldots\& B_m C_m$, где $m \geqslant 1; A,B_i,C_i \in N$.
		\item $A \rightarrow a$, где $A \in N, a \in \Sigma$.
		\item $S \rightarrow \varepsilon$, если только $S$ не содержится в правой части всех правил.
	\end{itemize}
\end{definition}

\begin{theorem}\label{Binary normal form conjunctive grammar theorem}
    Для каждой конъюнктивной грамматики $G$ можно построить конъюнктивную грамматику в бинарной нормальной форме $G^{'}$, такую что $L(G) = L(G^{'})$.
\end{theorem}
Доказательство теоремы~\ref{Binary normal form conjunctive grammar theorem} описано в статье~\cite{DBLP:journals/jalc/Okhotin01}.

Матричный алгоритм для конъюнктивных грамматик отличается от алгоритма~\ref{alg:graphParse} для контекстно-свободных грамматик только операцией умножения матриц, в остальном алгоритм остается без изменений. Определим операцию умножения матриц.
\begin{definition}
    Пусть $M_1$ и $M_2$ матрицы размера $n$. Определим операцию $\circ$. $M_1 \circ M_2 = M_3$, где $M_3 [i,j] = \{A \mid \exists (A \rightarrow B_1 C_1 \& \ldots \& B_m C_m) \in P, (B_k , C_k) \in d[i,j] \forall k = 1,\ldots,m\}$, где $d[i,j] = \bigcup_{k = 1}^{n} M_1 [i,k] \times M_2 [k,j]$.
\end{definition}

Важно заметить, что алгоритм для конъюнктивных грамматик, в отличие от алгоритма для контекстно-свободных грамматик, дает лишь верхнюю оценку. То есть все нетерминалы, которые должны быть в ячейках матрицы результата, содержатся там, но вместе с ними содержатся и лишние нетерминалы. Рассмотрим пример, иллюстрирующий появление лишних нетерминалов.

\begin{example}
    Грамматика $G$:
    \begin{align*}
	S &\to AB \& DC \\ 
	A &\to a \\
	B &\to BC \mid b \\
	C &\to c \\
	D &\to DC \mid b \\
	\end{align*}
	Очевидно, что грамматика $G$ задает язык из одного слова $L(G) = \{abc\} = \{abc^*\} \cap \{a^* bc\}$.
	
	Пусть есть граф $\mathcal{G}$:
    \begin{center}
    	\begin{tikzpicture}[shorten >=1pt,on grid,auto]
    	\node[state] (q_0)   {$0$};
    	\node[state] (q_1) [right=of q_0] {$1$};
    	\node[state] (q_2) [right=of q_1] {$2$};
    	\node[state] (q_6) [below=of q_2] {$6$};
    	\node[state] (q_3) [right=of q_2] {$3$};
    	\node[state] (q_5) [right=of q_6] {$5$};
    	\node[state] (q_4) [right=of q_3] {$4$};
    	\path[->]
    	(q_0) edge  node {a} (q_1)
	    (q_1) edge  node {b} (q_2)
	    (q_1) edge  node {a} (q_6)
	    (q_2) edge  node {c} (q_3)
	    (q_3) edge  node {c} (q_4)
	    (q_6) edge  node {b} (q_5)
	    (q_5) edge  node {c} (q_4);
    	\end{tikzpicture}
    \end{center}
    Применяя алгоритм, получим, что существует путь из вершины 0 в вершину 4 из нетерминала $S$. Очевидно, что такого пути нет. Это происходит из-за того, что существует путь ``abcc'', соответствующий $L(AB) = \{abc^*\}$ и путь ``aabc'', соответствующий $L(DC) = \{a^{*}bc\}$, но они различны и проверить их в общем случае нельзя. Поэтому для классической семантики результат работы алгоритма является оценкой сверху.
    
    Существует альтернативная семантика, когда мы трактуем конъюнкцию как в Datalog. Подробнее о Datalog в параграфе~\ref{Subsection Datalog}. Т.е если есть правило $S \to AB \& DC$, то должен быть путь принадлежащий языку $L(AB)$ и путь принадлежащий языку $L(DC)$. В альтернативной семантике алгоритм дает точный ответ.
\end{example}

Подробнее алгоритм описан в статье Рустама Азимова и Семена Григорьева~\cite{565CECD7E8F5C6063935B41DB41797AA37D53B04}.

В общем случае для булевых грамматик подобного алгоритма не существует.

\subsection{Особенности реализации матричного алгоритма}

Матричные алгоритмы удобны тем, что их удобно вычислять на GPU и для них можно создать модификацию, которая будет работать параллельно на нескольких вычислителях~\cite{Mishin:2019:ECP:3327964.3328503}.

Так как множество нетерминалов и правил конечно, то мы можем свести алгоритм к булевым матрицам: для каждого нетерминала заводим матрицу, такую что в ячейке стоит 1 тогда и только тогда, когда в исходной матрице в соответствующей ячейке был этот нетерминал.
Тогда перемножение пары таких матриц --- это применение правила.


\begin{example}
Представим в таком виде следующую матрицу:
\[
T_0 = \begin{pmatrix}
\varnothing & \{A\}       & \varnothing & \{B\}       \\
\varnothing & \varnothing & \{A\}       & \varnothing \\
\{A\}       & \varnothing & \varnothing & \varnothing \\
\{B\}       & \varnothing & \varnothing & \varnothing \\
\end{pmatrix}
\]

Тогда:
\begin{alignat*}{7}
& &&T_{0\_A} &&= \begin{pmatrix}
0 & 1       & 0 & 0       \\
0 & 0 & 1       & 0 \\
1  & 0 & 0 & 0       \\
0       & 0 & 0 & 0 \\
\end{pmatrix} \ \ \ \ &&T_{0\_B} &&= \begin{pmatrix}
0 & 0       & 0 & 1       \\
0       & 0 & 0       & 0 \\
0  & 0 & 0 & 0       \\
1       & 0 & 0 & 0 \\
\end{pmatrix}
\end{alignat*}
Тогда при наличии правила $S \to A B$ получим $T_{1\_S} =T_{0\_A} \times T_{0\_B}$~:
\[
T_{1\_S} = \begin{pmatrix}
0 & 0       & 0 & 0       \\
0       & 0 & 0       & 0 \\
0  & 0 & 0 & 1       \\
0       & 0 & 0 & 0 \\
\end{pmatrix}
\]
\end{example}

\medskip

С другой стороны, для небольших запросов практически может быть выгодно представлять множества нетерминалов в виде битового вектора.
Нумеруем все нетерминалы с нуля, в векторе стоит 1 на i позиции, если в множестве есть нетерминал с номером i.
Таким образом, в каждой ячейке хранится битовый вектор длины $|N|$.
Тогда операцию умножения надо определить следующим образом:
$$v_1 \times v_2 = \{v \mid \exists (v,v_3) \in P, \textit{append}(v_1, v_2) \& v_3 = v_3\},$$ где $\&$ --- побитовое \texttt{``и''}.
Чтобы это было возможно, правила надо кодировать соответственно: продукция это пара, где первый элемент --- битовый вектор длины $N$ с единственной единицей в позиции, соответствующей нетерминалу в правой части, а второй элемент --- вектор длины $2|N|$, с двумя единицами кодирующими первый и второй нетерминалы, соответственно.

\begin{example}
Пусть $N = \{S, A, B\}$ и есть продукция $S \to A B$. Тогда занумеруем нетерминалы $ S \to 0, A \to 1, B \to 2$ и продукция примет вид $[1, 0, 0] \to [0, 1, 0, 0, 0, 1]$. Матрица $T_0$ примет вид(в нашей нотации $.$ означает, что в ячейке стоит $[0,0,0]$):
\[
T_0 = \begin{pmatrix}
. & [0,1,0]       & . & [0,0,1]       \\
. & . & [0,1,0]       & . \\
[0,1,0]       & . & . & . \\
[0,0,1]      & . & . & . \\
\end{pmatrix}
\]

После выполнения умножения получим $T_1 = T_0 \times T_0$:
\[
T_1 = \begin{pmatrix}
. & [0,1,0]       & . & [0,0,1]       \\
. & . & [0,1,0]       & . \\
[0,1,0]       & . & . & [1,0,0] \\
[0,0,1]      & . & . & . \\
\end{pmatrix}
\]
\end{example}


На практике в роли векторов могут выступать беззнаковые целые. 
Например, 32 бита под ячейки в матрице и 64 бита под правила (или 8 и 16, если запросы совсем маленькие, или 16 и 32).
Тогда умножение выражается через битовые операции и сравнение.



Это может оказаться быстрее --- в данном случае скорость зависит от деталей реализации.

При таких представлениях данные часто оказываются разреженны --- возникает вопрос, как представлять матрицу. Среди способов --- разреженные матрицы, GraphBLAS\footnote{GraphBLAS --- открытый стандарт для графовых алгоритмов --- \url{https://github.com/gunrock/graphblast} }, GPGPU, CUTLASS\footnote{Репозиторий библиотеки CUTLASS: \url{https://github.com/NVIDIA/cutlass}}.
Quad Tree~\cite{quadtree}.

Для вычислений лучше всего, когда все нужные для вычисления матрицы помещаются на одну карту. Хуже --- если только одна пара матриц. Хуже всего, когда не помещается даже пара матриц. Поэтому хороши распределенные решения, например через GraphBLAS.

\subsection{Обзор}
\begin{itemize}
	\item Lee. О конвертации парсеров КС-грамматик в перемножение булевых матриц~\cite{Lee:2002:FCG:505241.505242}
	\item OpenCypher~\cite{Kuijpers:2019:ESC:3335783.3335791}
	\item J.Hellings. CFPQ~\cite{hellingsRelational,hellings2015querying,Hellings2015PathRF}
	\item Zhang. CFPQ on rdf graphs~\cite{10.1007/978-3-319-46523-4_38}
	\item Bradford~\cite{bradford2007quickest,ward2008distributed,bradford2016fast,Bradford:2008:LCG:1373936.1373946}
\end{itemize}

Асимптотически приведенные алгоритмы имеют большую сложность, например \\ $O(n^2 |N|^2|P|~ (MM(n))$, где $MM(n)$ --- сложность перемножения матриц $n \times n$, $|P|$ --- мощность множества продукций, $|N|$ --- мощность множества нетерминалов. Однако такая большая сложность компенсируется возможностью их распараллеливания, в результате чего они работают быстрее однопоточных алгоритов с лучшей сложностью.

Брэдфорт получил субкубическую сложность для частного случая --- языка Дика с одним типом скобок~\cite{8249039}.

\subsection{Вопросы и задачи}
\begin{enumerate}
	\item Находить кратчайшие пути в графах, используя идеи из секции~\ref{Matrix-CFPQ}.
	\item Превратить граф, использующийся для CFPQ, в дерево.
  	\item Реализовать предложенные идеи на различных архитектурах.
  	\item Замерить производительность и расходы памяти по сравнению с существующими реализациями.
\end{enumerate}
