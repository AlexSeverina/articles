\section{Алгоритм на матричных опреациях}

Введение, история вопроса.

Китайцы~\cite{10.1007/978-3-319-46523-4_38}, Брэдфорд~\cite{bradford2007quickest,ward2008distributed,bradford2016fast,Bradford:2008:LCG:1373936.1373946}, Хеллингс~\cite{hellingsRelational,hellings2015querying,Hellings2015PathRF}, OpenCypher~\cite{Kuijpers:2019:ESC:3335783.3335791}.

Рассуждения про ассимптотику.

Субкубический для частного случая (Брэдфорд)~\cite{8249039}.

\subsection{Матрицы для КС разбора}

Вэлиант~\cite{Valiant:1975:GCR:1739932.1740048}.

Ли~\cite{Lee:2002:FCG:505241.505242},


\subsection{Матрицы для КС запросов}
Вводное.
Про замыкание через произведение матриц.

Вспомниаем про то, что транзитивное замыкание можно искать через произведение матриц.


К сожалению, Вэлиант очевидным образом не обобщается на графы (с сохранением асимптотики).


Потому будем действовать наивно.
Оригинальные матрицы (Рустам)~\cite{Azimov:2018:CPQ:3210259.3210264}.

Let $D = (V, E)$ be the input graph and $G = (N,\Sigma,P)$ be the input grammar.

\begin{algorithm}[H]
\begin{algorithmic}[1]
\caption{Context-free recognizer for graphs}
\label{alg:graphParse}
\Function{contextFreePathQuerying}{D, G}
    
    \State{$n \gets$ the number of nodes in $D$}
    \State{$E \gets$ the directed edge-relation from $D$}
    \State{$P \gets$ the set of production rules in $G$}
    \State{$T \gets$ the matrix $n \times n$ in which each element is $\varnothing$}
    \ForAll{$(i,x,j) \in E$}
    \Comment{Matrix initialization}
        \State{$T_{i,j} \gets T_{i,j} \cup \{A~|~(A \rightarrow x) \in P \}$}
    \EndFor    
    \While{matrix $T$ is changing}
       
        \State{$T \gets T \cup (T \times T)$}
        \Comment{Transitive closure $T^{cf}$ calculation} 
    \EndWhile
\State \Return $T$
\EndFunction
\end{algorithmic}
\end{algorithm}


\begin{example}[Пример работы]

Описание. Добавить графы, как в тензорном произведении.


\[
T_0 = \begin{pmatrix}
    \varnothing & \{A\}       & \varnothing & \{B\}       \\
    \varnothing & \varnothing & \{A\}       & \varnothing \\
    \{A\}       & \varnothing & \varnothing & \varnothing \\
    \{B\}       & \varnothing & \varnothing & \varnothing \\
\end{pmatrix}
\]

Let $T_i$ be the matrix $T$, obtained after executing the loop in the lines \textbf{8-9} of the Algorithm~\ref{alg:graphParse} $i$ times. The calculation of the matrix $T_1$ is shown on Figure~\ref{ExampleQueryFirstIteration}.


\[
T_0 \times T_0 = \begin{pmatrix}
    \varnothing & \varnothing & \varnothing & \varnothing \\
    \varnothing & \varnothing & \varnothing & \varnothing \\
    \varnothing & \varnothing & \varnothing & \{S\}       \\
    \varnothing & \varnothing & \varnothing & \varnothing \\
\end{pmatrix}
\]

\[
T_1 = T_0 \cup (T_0 \times T_0) = \begin{pmatrix}
    \varnothing & \{A\}       & \varnothing & \{B\}       \\
    \varnothing & \varnothing & \{A\}       & \varnothing \\
    \{A\}       & \varnothing & \varnothing & \{\pmb{S}\}       \\
    \{B\}       & \varnothing & \varnothing & \varnothing \\
\end{pmatrix}
\]

When the algorithm at some iteration finds new paths in the graph $D$, then it adds corresponding nonterminals to the matrix $T$. For example, after the first loop iteration, non-terminal $S$ is added to the matrix $T$. This non-terminal is added to the element with a row index $i = 2$ and a column index $j = 3$. This means, that there is $i\pi j$ (a path $\pi$ from the node 2 to the node 3), such that $S \xrightarrow{*} l(\pi)$. For example, such a path consists of two edges with labels $a$ and $b$, and thus $S \xrightarrow{*} a \ b$.

The calculation of the transitive closure is completed after $k$ iterations, when a fixpoint is reached: $T_{k-1} = T_k$. For the example query, $k = 13$ since $T_{13} = T_{12}$. The remaining iterations of computing the transitive closure are presented on Figure~\ref{ExampleQueryFinalIterations} (new matrix elements on each iteration are shown in bold).


\begin{alignat*}{7}
& &&T_2 &&= \begin{pmatrix}
\varnothing & \{A\}       & \varnothing & \{B\}       \\
\varnothing & \varnothing & \{A\}       & \varnothing \\
\{A, \pmb{S_1}\}  & \varnothing & \varnothing & \{S\}       \\
\{B\}       & \varnothing & \varnothing & \varnothing \\
\end{pmatrix} \ \ \ \ &&T_3 &&= \begin{pmatrix}
\varnothing & \{A\}       & \varnothing & \{B\}       \\
\{\pmb{S}\}       & \varnothing & \{A\}       & \varnothing \\
\{A, S_1\}  & \varnothing & \varnothing & \{S\}       \\
\{B\}       & \varnothing & \varnothing & \varnothing \\
\end{pmatrix} \ \ \ \ &&T_4 &&= \begin{pmatrix}
\varnothing & \{A\}       & \varnothing & \{B\}       \\
\{S\}       & \varnothing & \{A\}       & \{\pmb{S_1}\}     \\
\{A, S_1\}  & \varnothing & \varnothing & \{S\}       \\
\{B\}       & \varnothing & \varnothing & \varnothing \\
\end{pmatrix}  \\
& &&T_5 &&= \begin{pmatrix}
\varnothing & \{A\}       & \varnothing & \{B, \pmb{S}\}    \\
\{S\}       & \varnothing & \{A\}       & \{S_1\}     \\
\{A, S_1\}  & \varnothing & \varnothing & \{S\}       \\
\{B\}       & \varnothing & \varnothing & \varnothing \\
\end{pmatrix} \ \ \ \ &&T_6 &&= \begin{pmatrix}
\{\pmb{S_1}\}     & \{A\}       & \varnothing & \{B, S\}    \\
\{S\}       & \varnothing & \{A\}       & \{S_1\}     \\
\{A, S_1\}  & \varnothing & \varnothing & \{S\}       \\
\{B\}       & \varnothing & \varnothing & \varnothing \\
\end{pmatrix} \ \ \ \ &&T_7 &&= \begin{pmatrix}
\{S_1\}     & \{A\}       & \varnothing & \{B, S\}    \\
\{S\}       & \varnothing & \{A\}       & \{S_1\}     \\
\{A, S_1, \pmb{S}\}  & \varnothing & \varnothing & \{S\}    \\
\{B\}       & \varnothing & \varnothing & \varnothing \\
\end{pmatrix}  \\
& &&T_8 &&= \begin{pmatrix}
\{S_1\}     & \{A\}       & \varnothing & \{B, S\}    \\
\{S\}       & \varnothing & \{A\}       & \{S_1\}     \\
\{A, S_1, S\}  & \varnothing & \varnothing & \{S, \pmb{S_1}\} \\
\{B\}       & \varnothing & \varnothing & \varnothing \\
\end{pmatrix} \ \ \ \ &&T_9 &&= \begin{pmatrix}
\{S_1\}     & \{A\}       & \varnothing & \{B, S\}    \\
\{S\}       & \varnothing & \{A\}       & \{S_1, \pmb{S}\}     \\
\{A, S_1, S\}  & \varnothing & \varnothing & \{S, S_1\} \\
\{B\}       & \varnothing & \varnothing & \varnothing \\
\end{pmatrix} \ \ \ \ &&T_{10} &&= \begin{pmatrix}
\{S_1\}     & \{A\}       & \varnothing & \{B, S\}    \\
\{S, \pmb{S_1}\}       & \varnothing & \{A\}       & \{S_1, S\}     \\
\{A, S_1, S\}  & \varnothing & \varnothing & \{S, S_1\} \\
\{B\}       & \varnothing & \varnothing & \varnothing \\
\end{pmatrix}  \\
& &&T_{11} &&= \begin{pmatrix}
\{S_1, \pmb{S}\}     & \{A\}       & \varnothing & \{B, S\}    \\
\{S, S_1\}       & \varnothing & \{A\}       & \{S_1, S\}     \\
\{A, S_1, S\}  & \varnothing & \varnothing & \{S, S_1\} \\
\{B\}       & \varnothing & \varnothing & \varnothing \\
\end{pmatrix} \ \ \ \ &&T_{12} &&= \begin{pmatrix}
\{S_1, S\}     & \{A\}       & \varnothing & \{B, S, \pmb{S_1}\}    \\
\{S, S_1\}       & \varnothing & \{A\}       & \{S_1, S\}     \\
\{A, S_1, S\}  & \varnothing & \varnothing & \{S, S_1\} \\
\{B\}       & \varnothing & \varnothing & \varnothing \\
\end{pmatrix} \ \ \ \ &&T_{13} &&= \begin{pmatrix}
\{S_1, S\}     & \{A\}       & \varnothing & \{B, S, S_1\}    \\
\{S, S_1\}       & \varnothing & \{A\}       & \{S_1, S\}     \\
\{A, S_1, S\}  & \varnothing & \varnothing & \{S, S_1\} \\
\{B\}       & \varnothing & \varnothing & \varnothing \\
\end{pmatrix}
\end{alignat*}

Thus, the result of the Algorithm~\ref{alg:graphParse} for the example query is the matrix $T_{13} = T_{12}$. 

\end{example}.

Интересные вопросы: кратчайшие пути и одно дерево (построить один путь).

\subsection{Конъюнктивные и булевы граммтики}

\subsubsection{Определения}

Охотин~\cite{Okhotin:2003:BG:1758089.1758123,f60a33d409364914be560cac0e54b12c,Okhotin:2014:PMM:2565359.2565379}.

\subsubsection{Для графов}

Классическая семантика --- работает для конъюнктивных для любых графов.
Для Булевых и конъюнктивных только для графов без циклов.

Альтернативная семантика (DataLog).
Трактуем конъюнкцию как в даталоге. Тогда всё хорошо.
Это похоже на даталог через линейную алгебру~\cite{!!!}

\subsection{Особенности реализации матричного алгоритма}

Кое-что про наши реализации~\cite{Mishin:2019:ECP:3327964.3328503}

Так как множество нетерминалов и правил конечно, то мы можем свести алгоритм к булевым матрицам: для каждого нетерминала заводим матрицу, такую что в ячейке стоит true тогда и только тогда, когда в исходной матрице в соответствующе ячейке был этот нетерминал.
Тогда перемноженеи пары таких матриц --- это применеие правила.

Описание модификации!

\begin{example}
Пример работы!
\end{example}

С другой стороны, для небольших запросов практически может быть выгодно представлять множества нетерминалов в виде битового вектора.
Нумеруем все нетерминалы с нуля, в векторе стит 1 на i позиции если в множетве есть нетерминал с номером i.
Таким образом, в каждой ячейке хранится битовый вектор длины $|N|$.
Тогда операцию умножения надо определить следующим образом:
$$v_1 \times v_2 = v \mid \exists (v,v_3) \in P, \textit{append}(v_1, v_2) \& v_3 = v_3,$$ где $\&$ --- побитовое \texttt{``и''}.
Для этого правила надо кодировать соответственно: продукция это пара, где первый элемент --- битовый вектор длины $N$ с единственной единицей в позиции, соответствующей нетерминалу в правой части, а второй элемент --- векто длины $2|N|$, с двумя единицами кодирующими первый и второй нетерминалы, соответственно.

\begin{example}
Пример работы!
\end{example}


На практике в роли векторов могут выступать беззнаковые целые. 
Например, 32 бита под ячейки в матрице и 64 бита под правила (или 8 и 16, если запросы совсем маленькие, или 16 и 32).
Тогда умножение выражается через битовые операции и сравнение.



Это может оказаться быстрее. Надо реализовыть и смотреть.

Так как данные чать разрежены, то много вопросов к представлению матриц.
Разреженные матрицы, плотные матрицы, GraphBLAS\footnote{!!!}, GPGPU, CUTLASS\footnote{Репозиторий библиотеки CUTLASS: \url{https://github.com/NVIDIA/cutlass}}.
Quad Tree~\cite{!!!}.

Расположенеи в памяти: хорошо, когда всё (всематрицы, нужные для вычислений) влезло на одну карту.
Хуже, когда только одна пара матриц.
Ещё хуже, когда даже одна матрица не помещается.

Потому нужно пробовать распределённые решения.
Например, через GraphBLAS.


\subsection{Вопросы и задачи}
\begin{enumerate}
  \item !!!
  \item !!!
  \item !!!
\end{enumerate}
