\section{CYK для вычисления КС запросов}

В данной главе мы рассмотрим алгоритм CYK, позволяющий установить принадлежность слова грамматике и предоставить его вывод, если таковой имеется.

Наш главный интерес заключается в возможности применения данного алгоритма для решения описанной в предыдущей главе задачи --- поиска путей с ограничениями в терминах формальных языков. Как уже было указано выше, будем рассматривать случай контекстно-свободных языков.

\subsection{Алгоритм CYK}

Алгоритм CYK (Cocke-Younger-Kasami) --- один из классических алгоритмов синтаксического анализа. Его асимптотическая сложность в худшем случае --- $O(n^3 * |G|)$ ($n$ --- размер входной строки, $G$ --- входная грамматика), что выгодно выделяет его среди других алгоритмов парсинга.~\cite{Hopcroft+Ullman/79/Introduction}

Для его применения необходимо, чтобы подаваемая на вход грамматика находилась в Нормальной Форме Хомского (НФХ)~\ref{section:CNF}.

В основе алгоритма лежит принцип динамического программирования. Используются два соображения (здесь $\omega$ --- слово, $A$, $B$, $C$ --- нетерминалы грамматики, $a$ --- терминал грамматики):

\begin{enumerate}
\item Для правила вида $A \to a$ очевидно, что из $A$ выводится $\omega$ (с применением этого правила) тогда и только тогда, когда $a = \omega$:

\[
  A \derives[] a \derives \omega \iff \omega = a\]

\item Для правила вида $A \to B C$ понятно, что из $A$ выводится $\omega$ (с применением этого правила) тогда и только тогда, когда существуют две цепочки $\omega_1$ и $\omega_2$ такие, что $\omega_1$ выводима из $B$, $\omega_2$ выводима из $C$ и при этом $\omega = \omega_1 \omega_2$:

\[A \derives[] B C \derives \omega \iff \exists \omega_1, \omega_2 : \omega = \omega_1 \omega_2, B \derives \omega_1, C \derives \omega_2\]

Или в терминах позиций в строке:

\[A \derives[] B C \derives \omega \iff \exists k \in [1 \dots |\omega|] : B \derives \omega[1 \dots k], C \derives \omega[k+1 \dots |\omega|]\]
\end{enumerate}

В процессе работы алгоритма заполняется булева трехмерная матрица размера $|N| \times n \times n$, где $n$~---  размер входной цепочки, $N$~--- множество нетерминалов в нормализованной грамматике.
$M[i, j, A] = true \iff A \derives \omega[i \dots j]$

Первым шагом инициализируем матрицу, заполнив значения $M[i, j, A] \text{, где }i = j$:

\begin{itemize}
  \item $M[i, i, A] = true \text{, если в грамматике есть правило } A \to \omega[i]$.
  \item $M[i, i, A] = false$, иначе.
\end{itemize}

Далее используем динамику: на шаге $m > 1$ предполагаем, что ячейки матрицы $M[i', j', A]$ заполнены для всех нетерминалов $A$ и пар $i', j': j' - i' < m$.
Тогда можно заполнить ячейки матрицы $M[i, j, A] \text{, где } j - i = m$

\[ M[i, j, A] = \bigvee_{A \to B C}^{}{\bigvee_{k=i}^{j-1}{M[i, k, B] \wedge M[k, j, C]}} \]

По итогу работы алгоритма значение в ячейке $M[0, |\omega|, S]$, где $S$ --- стартовый нетерминал грамматики, отвечает на вопрос о выводимости цепочки $\omega$ в грамматике.

\begin{example}
  Рассмотрим пример работы алгоритма CYK на грамматике правильных скобочных последовательностей в Нормальной Форме Хомского.

\begin{align*}
S &\to A S_2 \mid \varepsilon \\
S_1   &\to A S_2 \\
S_2  &\to b \mid B S_1 \mid S S_3 \\
S_3 &\to b \mid B S_1 \\
A   &\to a \\
B   &\to b
\end{align*}

Проверим выводимость цепочки $\omega = a a b b a b$.

Так как трехмерные матрицы рисовать на двумерной бумаге не очень удобно, мы будем иллюстрировать работу алгоритма двумерными матрицами размера $n \times n$, где в ячейках указано множество нетерминалов, выводящих соответствующую подстроку.

Шаг 1. Инициализируем матрицу элементами на главной диагонали:

\[
\begin{pmatrix}
\{A\}  		& \varnothing & \varnothing    & \varnothing 	  & \varnothing & \varnothing 	 \\
\varnothing & \{A\} 	  & \varnothing    & \varnothing 	  & \varnothing & \varnothing 	 \\
\varnothing & \varnothing & \{B, S_2, S_3\} & \varnothing 	  & \varnothing & \varnothing 	 \\
\varnothing & \varnothing & \varnothing    & \{B, S_2, S_3\}   & \varnothing & \varnothing 	 \\
\varnothing & \varnothing & \varnothing    & \varnothing 	  & \{A\} 	    & \varnothing 	 \\
\varnothing & \varnothing & \varnothing    & \varnothing 	  & \varnothing & \{B, S_2, S_3\} \\
\end{pmatrix}
\]

Шаг 2. Заполняем диагональ, находящуюся над главной:

\[
\begin{pmatrix}
\{A\}  		& \varnothing & \varnothing    & \varnothing 	  & \varnothing & \varnothing 	 \\
\varnothing & \{A\} 	  & \{S_1\}  		   & \varnothing 	  & \varnothing & \varnothing 	 \\
\varnothing & \varnothing & \{B, S_2, S_3\} & \varnothing 	  & \varnothing & \varnothing 	 \\
\varnothing & \varnothing & \varnothing    & \{B, S_2, S_3\}   & \varnothing & \varnothing 	 \\
\varnothing & \varnothing & \varnothing    & \varnothing 	  & \{A\} 	    & \{S_1\}	 		 \\
\varnothing & \varnothing & \varnothing    & \varnothing 	  & \varnothing & \{B, S_2, S_3\} \\
\end{pmatrix}
\]

Шаг 3. Заполняем следующую диагональ:

\[
\begin{pmatrix}
\{A\}  		& \varnothing & \varnothing    & \varnothing 	  & \varnothing & \varnothing 	 \\
\varnothing & \{A\} 	  & \{S_1\}  		   & \{S_2\}  		  & \varnothing & \varnothing 	 \\
\varnothing & \varnothing & \{B, S_2, S_3\} & \varnothing 	  & \varnothing & \varnothing 	 \\
\varnothing & \varnothing & \varnothing    & \{B, S_2, S_3\}   & \varnothing & \{S_2, S_3\}	 \\
\varnothing & \varnothing & \varnothing    & \varnothing 	  & \{A\} 	    & \{S_1\}	 		 \\
\varnothing & \varnothing & \varnothing    & \varnothing 	  & \varnothing & \{B, S_2, S_3\} \\
\end{pmatrix}
\]

Шаг 4. И следующую за ней:

\[
\begin{pmatrix}
\{A\}  		& \varnothing & \varnothing    & \{S_1, S\}	 	  & \varnothing & \varnothing 	 \\
\varnothing & \{A\} 	  & \{S_1\}  		   & \{S_2\}  		  & \varnothing & \varnothing 	 \\
\varnothing & \varnothing & \{B, S_2, S_3\} & \varnothing 	  & \varnothing & \varnothing 	 \\
\varnothing & \varnothing & \varnothing    & \{B, S_2, S_3\}   & \varnothing & \{S_2, S_3\}	 \\
\varnothing & \varnothing & \varnothing    & \varnothing 	  & \{A\} 	    & \{S_1\}	 		 \\
\varnothing & \varnothing & \varnothing    & \varnothing 	  & \varnothing & \{B, S_2, S_3\} \\
\end{pmatrix}
\]

Шаг 5 Заполняем предпоследнюю диагональ:

\[
\begin{pmatrix}
\{A\}  		& \varnothing & \varnothing    & \{S_1, S\}	 	  & \varnothing & \varnothing 	 \\
\varnothing & \{A\} 	  & \{S_1\}  		   & \{S_2\}  		  & \varnothing & \{S_2\}	 	 \\
\varnothing & \varnothing & \{B, S_2, S_3\} & \varnothing 	  & \varnothing & \varnothing 	 \\
\varnothing & \varnothing & \varnothing    & \{B, S_2, S_3\}   & \varnothing & \{S_2, S_3\}	 \\
\varnothing & \varnothing & \varnothing    & \varnothing 	  & \{A\} 	    & \{S_1\}	 		 \\
\varnothing & \varnothing & \varnothing    & \varnothing 	  & \varnothing & \{B, S_2, S_3\} \\
\end{pmatrix}
\]

\bigbreak
Шаг 6. Завершаем выполнение алгоритма:

\[
\begin{pmatrix}
\{A\}  		& \varnothing & \varnothing    & \{S_1, S\}	 	  & \varnothing & \{S_1, S\} 	 \\
\varnothing & \{A\} 	  & \{S_1\}  		   & \{S_2\}  		  & \varnothing & \{S_2\}	 	 \\
\varnothing & \varnothing & \{B, S_2, S_3\} & \varnothing 	  & \varnothing & \varnothing 	 \\
\varnothing & \varnothing & \varnothing    & \{B, S_2, S_3\}   & \varnothing & \{S_2, S_3\}	 \\
\varnothing & \varnothing & \varnothing    & \varnothing 	  & \{A\} 	    & \{S_1\}	 		 \\
\varnothing & \varnothing & \varnothing    & \varnothing 	  & \varnothing & \{B, S_2, S_3\} \\
\end{pmatrix}
\]


Стартовый нетерминал находится в верхней правой ячейке, а значит цепочка $a a b b a b$ выводима в нашей грамматике.
\end{example}

\begin{example}
Теперь выполним алгоритм на невыводимой цепочке $abaa$.

Шаг 1. Инициализируем таблицу:

\[
\begin{pmatrix}
\{A\}  		& \varnothing 	 & \varnothing & \varnothing 	\\
\varnothing & \{B, S_2, S_3\} & \varnothing & \varnothing   	\\
\varnothing & \varnothing 	 & \{A\} 	   & \varnothing    \\
\varnothing & \varnothing 	 & \varnothing & \{A\} 			\\
\end{pmatrix}
\]

Шаг 2. Заполняем следующую диагональ:

\[
\begin{pmatrix}
\{A\}  		& \{S_1, S\} 	 & \varnothing & \varnothing 	\\
\varnothing & \{B, S_2, S_3\} & \varnothing & \varnothing   	\\
\varnothing & \varnothing 	 & \{A\} 	   & \varnothing    \\
\varnothing & \varnothing 	 & \varnothing & \{A\} 			\\
\end{pmatrix}
\]

Больше ни одну ячейку в таблице заполнить нельзя, а значит эта строка не выводится в грамматике правильных скобочных последовательностей.

\end{example}

\subsection{Алгоритм для графов на основе CYK}
\label{graph:CYK}
Теперь изменим вид входного слова и немного модифицируем алгоритм. Прежде мы сопоставляли каждому символу слова номер его позиции в цепочке, поэтому при инициализации заполняли главную диагональ матрицы. Вместо этого обозначим числами позиции между буквами таким образом (в качестве примера рассмотрим слово $a a b b a b$ из предыдущего пункта):

\begin{center}
	\begin{tikzpicture}[shorten >=1pt,on grid,auto]
	\node[state] (q_0) at (0,0)  {$0$};
	\node[state] (q_1) at (2,0)  {$1$};
	\node[state] (q_2) at (4,0)  {$2$};
	\node[state] (q_3) at (6,0)  {$3$};
	\node[state] (q_4) at (8,0)  {$4$};
	\node[state] (q_5) at (10,0) {$5$};
	\node[state] (q_6) at (12,0) {$6$};
	\path[->]
	(q_0) edge  node {$a$} (q_1)
	(q_1) edge  node {$a$} (q_2)
	(q_2) edge  node {$b$} (q_3)
	(q_3) edge  node {$b$} (q_4)
	(q_4) edge  node {$a$} (q_5)
	(q_5) edge  node {$b$} (q_6);
	\end{tikzpicture}
\end{center}

Что нужно изменить в описании алгоритма, чтобы он продолжал работать при подобной нумерации? Каждая буква теперь идентифицируется не одним числом, а парой – номера слева и справа от нее. При этом чисел стало на одно больше, чем при прежнем способе нумерации.

Возьмем матрицу  $|N| \times (n + 1) \times (n + 1)$ и при инициализации будем заполнять не главную диагональ, а диагональ прямо над ней. Таким образом, мы начинаем наш алгоритм с определения значений $M[i, j, A] \text{, где } j = i + 1$. При этом наши дальнейшие действия в рамках алгоритма не изменятся.

На шаге инициализации матрица выглядит следующим образом:

\[
\begin{pmatrix}
\varnothing & \{A\}  	  & \varnothing & \varnothing    & \varnothing 	  & \varnothing & \varnothing 	 \\
\varnothing & \varnothing & \{A\} 	  & \varnothing    & \varnothing 	  & \varnothing & \varnothing 	 \\
\varnothing & \varnothing & \varnothing & \{B, S_2, S_3\} & \varnothing 	  & \varnothing & \varnothing 	 \\
\varnothing & \varnothing & \varnothing & \varnothing    & \{B, S_2, S_3\} & \varnothing & \varnothing 	 \\
\varnothing & \varnothing & \varnothing & \varnothing    & \varnothing 	  & \{A\} 	    & \varnothing 	 \\
\varnothing & \varnothing & \varnothing & \varnothing    & \varnothing 	  & \varnothing & \{B, S_2, S_3\} \\
\varnothing & \varnothing & \varnothing & \varnothing    & \varnothing 	  & \varnothing & \varnothing	 \\

\end{pmatrix}
\]

А в результате работы алгоритма имеем:

\[
\begin{pmatrix}
\varnothing & \{A\}  	  & \varnothing & \varnothing    & \{S_1, S\}	  & \varnothing & \{S_1, S\} 	 \\
\varnothing & \varnothing & \{A\} 	    & \{S_1\}  		 & \{S_2\}  		  & \varnothing & \{S_2\}	 	 \\
\varnothing & \varnothing & \varnothing & \{B, S_2, S_3\} & \varnothing 	  & \varnothing & \varnothing 	 \\
\varnothing & \varnothing & \varnothing & \varnothing    & \{B, S_2, S_3\} & \varnothing & \{S_2, S_3\}	 \\
\varnothing & \varnothing & \varnothing & \varnothing    & \varnothing 	  & \{A\} 	    & \{S_1\}	 		 \\
\varnothing & \varnothing & \varnothing & \varnothing    & \varnothing 	  & \varnothing & \{B, S_2, S_3\} \\
\varnothing & \varnothing & \varnothing & \varnothing    & \varnothing 	  & \varnothing & \varnothing	 \\
\end{pmatrix}
\]

Мы представили входную строку в виде линейного графа, а на шаге инициализации получили его матрицу смежности. Шаги алгоритма очень напоминают построение транзитивного замыкания графа. Различие заключается в том, что мы ``добавляем ребра`` только для тех пар нетерминалов, для которых существует правило в грамматике, их выводящее.

Алгоритм можно обобщить и на произвольные графы с метками, рассматриваемые в этом курсе. При этом накладывается ограничение на форму входной грамматики: она должна находиться в ослабленной Нормальной Форме Хомского~\ref{section:CNF}. То есть будем требовать выполнение только следующих 	правил:

\begin{itemize}
	\item $A \to B C \text{, где } A, B, C \in N $
	\item $A \to a \text{, где } A \in N, a \in \Sigma$
\end{itemize}

Шаг инициализации в алгоритме заключается в том, что мы с помощью продукций вида

\[A \to a \text{, где } A \in N, a \in \Sigma\]

заменяем терминалы на ребрах входного графа на нетерминалы, из которых они выводятся. Затем используем матрицу смежности получившегося графа (обозначим ее $M$) в качестве начального значения. Дальнейший ход алгоритма можно описать псевдокодом:

\begin{algorithm}[H]
	\begin{algorithmic}[1]
		\caption{Context-free recognizer for graphs}
		\label{alg:graphParse}
		\Function{contextFreePathQuerying}{G, $\mathcal{G}$}

		\State{$n \gets$ the number of nodes in $\mathcal{G}$}
		\State{$M \gets$ the modified adjacency matrix of $\mathcal{G}$}
		\State{$P \gets$ the set of production rules in $G$}
		\While{$M$ is changing}
		\For {$k \in 0..n$}
			\For {$i \in 0..n$}
				\For {$j \in 0..n$}
					\ForAll {$N_1 \in M[i, k]$, $N_2 \in M[k, j]$}
						\If {$N_3 \to N_1 N_2 \in P$ }
							\State{$M[i, j] \mathrel{+}= \{N_3\}$}
						\EndIf
					\EndFor
				\EndFor
			\EndFor
		\EndFor
		\EndWhile
		\State \Return $M$
		\EndFunction
	\end{algorithmic}
\end{algorithm}

Если в некоторой ячейке результируюшей матрицы с номером $(i, j)$ находятся стартовый нетерминал, то это означает, что существует путь из вершины $i$ в вершину $j$, удовлетворяющий данной грамматике.

\begin{example}
Рассмотрим работу алгоритма на графе

\begin{center}
	\begin{tikzpicture}[shorten >=1pt,on grid,auto]
	\node[state] (q_0)   {$0$};
	\node[state] (q_1) [above right=of q_0] {$1$};
	\node[state] (q_2) [right=of q_0] {$2$};
	\node[state] (q_3) [right=of q_2] {$3$};
	\path[->]
	(q_0) edge  node {$a$} (q_1)
	(q_1) edge  node {$a$} (q_2)
	(q_2) edge  node {$a$} (q_0)
	(q_2) edge[bend left, above]  node {$b$} (q_3)
	(q_3) edge[bend left, below]  node {$b$} (q_2);
	\end{tikzpicture}
\end{center}

и грамматике:

\begin{align*}
S 	& \to A B   \\
S 	& \to A S_1 \\
S_1 & \to S B   \\
A 	& \to a     \\
B 	& \to b	    \\
\end{align*}

Данный пример является классическим и еще не раз будет использоваться в рамках данного курса. \\

\textbf{Инициализация.}
Заменяем терминалы на ребрах графа на нетерминалы, из которых они выводятся, и строим матрицу смежности получившегося графа:

\begin{center}
	\begin{tikzpicture}[shorten >=1pt,on grid,auto]
	\node[state] (q_0)   {$0$};
	\node[state] (q_1) [above right=of q_0] {$1$};
	\node[state] (q_2) [right=of q_0] {$2$};
	\node[state] (q_3) [right=of q_2] {$3$};
	\path[->]
	(q_0) edge  node {$A$} (q_1)
	(q_1) edge  node {$A$} (q_2)
	(q_2) edge  node {$A$} (q_0)
	(q_2) edge[bend left, above]  node {$B$} (q_3)
	(q_3) edge[bend left, below]  node {$B$} (q_2);
	\end{tikzpicture}
\end{center}

\[
\begin{pmatrix}
\varnothing & \{A\}  	  & \varnothing & \varnothing \\
\varnothing & \varnothing & \{A\} 	    & \varnothing \\
\{A\} 		& \varnothing & \varnothing & \{B\} 	  \\
\varnothing & \varnothing & \{B\}	    & \varnothing \\
\end{pmatrix}
\]

\textbf{Итерация 1.}
Итерируемся по $k$, $i$ и $j$, пытаясь найти пары нетерминалов, для которых существуют правила вывода, их выводящие. Нам интересны следующие случаи:

\begin{itemize}
	\item $k = 2, i = 1, j = 3: A \in M[1, 2], B \in M[2, 3]$, так как в грамматике присутствует правило $S \to A B$, добавляем нетерминал $S$ в ячейку $M[1, 3]$.
	\item $k = 3, i = 1, j = 2: S \in M[1, 3], B \in M[3, 2]$, поскольку в грамматике есть правило $S_1 \to S B$, добавляем нетерминал $S_1$ в ячейку $M[1, 2]$.
\end{itemize}

В остальных случаях либо какая-то из клеток пуста, либо не существует продукции в грамматике, выводящей данные два нетерминала.

Матрица после данной итерации:

\[
\begin{pmatrix}
\varnothing & \{A\}  	  & \varnothing & \varnothing \\
\varnothing & \varnothing & \{A, S_1\} 	& \{S\} 	  \\
\{A\} 		& \varnothing & \varnothing & \{B\} 	  \\
\varnothing & \varnothing & \{B\}	    & \varnothing \\
\end{pmatrix}
\]

\textbf{Итерация 2.}
Снова итерируемся по $k$, $i$, $j$. Рассмотрим случаи:

\begin{itemize}
	\setlength\itemsep{1em}
	\item $k = 1, i = 0, j = 2: A \in M[0, 1], S_1 \in M[1, 2]$, так как в грамматике присутствует правило $S \to A S_1$, добавляем нетерминал $S$ в ячейку $M[0, 2]$.
	\item $k = 2, i = 0, j = 3: S \in M[0, 2], B \in M[2, 3]$, поскольку в грамматике есть правило $S_1 \to S B$, добавляем нетерминал $S_1$ в ячейку $M[0, 3]$.
\end{itemize}

Матрица на данном шаге:

\[
\begin{pmatrix}
\varnothing & \{A\}  	  & \{S\} 	    & \{S_1\}	  \\
\varnothing & \varnothing & \{A, S_1\} 	& \{S\} 	  \\
\{A\} 		& \varnothing & \varnothing & \{B\} 	  \\
\varnothing & \varnothing & \{B\}	    & \varnothing \\
\end{pmatrix}
\]

\textbf{Итерация 3.}
Рассматриваемые на данном шаге случаи:

\begin{itemize}
	\setlength\itemsep{1em}
	\item $k = 0, i = 2, j = 3: A \in M[2, 0], S_1 \in M[0, 3]$, так как в грамматике присутствует правило $S \to A S_1$, добавляем нетерминал $S$ в ячейку $M[2, 3]$.
	\item $k = 3, i = 2, j = 2: S \in M[2, 3], B \in M[3, 2]$, поскольку в грамматике есть правило $S_1 \to S B$, добавляем нетерминал $S_1$ в ячейку $M[2, 2]$.
\end{itemize}

Матрица после этой итерации:

\[
\begin{pmatrix}
\varnothing & \{A\}  	  & \{S\} 	   & \{S_1\} 	 \\
\varnothing & \varnothing & \{A, S_1\} & \{S\}  	 \\
\{A\} 		& \varnothing & \{S_1\}	   & \{B, S\} 	 \\
\varnothing & \varnothing & \{B\}	   & \varnothing \\
\end{pmatrix}
\]

\textbf{Итерация 4.}
Рассмариваемые случаи:

\begin{itemize}
	\setlength\itemsep{1em}
	\item $k = 2, i = 1, j = 2: A \in M[1, 2], S_1 \in M[2, 2]$, так как в грамматике присутствует правило $S \to A S_1$, добавляем нетерминал $S$ в ячейку $M[1, 2]$.
	\item $k = 2, i = 1, j = 3: S \in M[1, 2], B \in M[2, 3]$, поскольку в грамматике есть правило $S_1 \to S B$, добавляем нетерминал $S_1$ в ячейку $M[1, 3]$.
\end{itemize}

Матрица:

\[
\begin{pmatrix}
\varnothing & \{A\}  	  & \{S\} 	      & \{S_1\}     \\
\varnothing & \varnothing & \{A, S, S_1\} & \{S, S_1\}  \\
\{A\} 		& \varnothing & \{S_1\}	  	  & \{B, S\}    \\
\varnothing & \varnothing & \{B\}	      & \varnothing \\
\end{pmatrix}
\]

\textbf{Итерация 5.}
Рассмотрим на это шаге:

\begin{itemize}
	\setlength\itemsep{1em}
	\item $k = 1, i = 0, j = 3: A \in M[0, 1], S_1 \in M[1, 3]$, поскольку в грамматике есть правило $S \to A S_1$, добавляем нетерминал $S$ в ячейку $M[0, 3]$.
	\item $k = 3, i = 0, j = 2: S \in M[0, 3], B \in M[3, 2]$, поскольку в грамматике есть правило $S_1 \to S B$, добавляем нетерминал $S_1$ в ячейку $M[0, 2]$.
\end{itemize}

Матрица на этой итерации:
\[
\begin{pmatrix}
\varnothing & \{A\}  	  & \{S, S_1\} 	  & \{S, S_1\}  \\
\varnothing & \varnothing & \{A, S, S_1\} & \{S, S_1\}  \\
\{A\} 		& \varnothing & \{S_1\}	   	  & \{B, S\}    \\
\varnothing & \varnothing & \{B\}	   	  & \varnothing \\
\end{pmatrix}
\]

\textbf{Итерация 6.}
Интересующие нас на этом шаге случаи:

\begin{itemize}
	\setlength\itemsep{1em}
	\item $k = 0, i = 2, j = 2: A \in M[2, 0], S_1 \in M[0, 2]$, поскольку в грамматике есть правило $S \to A S_1$, добавляем нетерминал $S$ в ячейку $M[2, 2]$.
	\item $k = 2, i = 2, j = 3: S \in M[2, 2], B \in M[2, 3]$, поскольку в грамматике есть правило $S_1 \to S B$, добавляем нетерминал $S_1$ в ячейку $M[2, 3]$.
\end{itemize}

Матрица после данного шага:

\[
\begin{pmatrix}
\varnothing & \{A\}  	  & \{S, S_1\} 	  & \{S, S_1\}    \\
\varnothing & \varnothing & \{A, S, S_1\} & \{S, S_1\}    \\
\{A\} 		& \varnothing & \{S, S_1\}	  & \{B, S, S_1\} \\
\varnothing & \varnothing & \{B\}	   	  & \varnothing   \\
\end{pmatrix}
\]

На следующей итерации матрица не изменяется, поэтому заканчиваем работу алгоритма. В результате, если ячейка $M[i, j]$ содержит стартовый нетерминал $S$, то существует путь из $i$ в $j$, удовлетворяющий ограничениям, заданным грамматикой.
\end{example}

Можно заметить, что мы делаем много лишних итераций.
Можно переписать алгоритм так, чтобы он не просматривал заведомо пустые ячейки.
См. Хеллингс~\cite{!!!}

Псевдокод Хеллингса.

\begin{example}
  Пример запуска Хеллингса
\end{example}

Рассужения про сравнение и асимптотику.

\subsection{Вопросы и задачи}
\begin{enumerate}
	\item Проверить работу алгоритма CYK для цепочек на грамматике
	\begin{flushleft}
	$E \to E + E$ \\
	$E \to E * E$ \\
	$E \to (E)$   \\
	$E \to n$	  \\
	\end{flushleft}
	и словах (алфавит $\Sigma = \{n, +, *, (, )\}$)
	\begin{flushleft}
	$ (n + n) * n$    \\
	$ n + n * n$      \\
	$n + n + n + n$   \\
	$n + (n * n) + n$ \\
	\end{flushleft}

	\item Посчитать вычислительную сложность алгоритма CYK для матриц в зависимости от размера входного графа (размер грамматики считать фиксированным)

	\item Проверить работу алгоритма CYK для графов на графе

	\begin{center}
		\begin{tikzpicture}[shorten >=1pt,on grid,auto]
		\node[state] (q_0)  {$0$};
		\node[state] (q_1) [right=of q_0]  {$1$};
		\node[state] (q_2) [right=of q_1]  {$2$};
		\node[state] (q_3) [right=of q_2]  {$3$};
		\node[state] (q_4) [right=of q_3]  {$4$};
		\path[->]
		(q_0) edge  node {$a$} (q_1)
		(q_1) edge  node {$b$} (q_2)
		(q_2) edge  node {$a$} (q_3)
		(q_3) edge  node {$b$} (q_4)
		(q_1) edge[bend left, above]  node {$b$} (q_3)
		(q_4) edge[bend left, below]  node {$a$} (q_1);
		\end{tikzpicture}
	\end{center}

	И грамматике

	\begin{flushleft}
		$S \to S S$ \\
		$S \to A B$ \\
		$A \to a$   \\
		$B \to b$	  \\
	\end{flushleft}

\end{enumerate}
