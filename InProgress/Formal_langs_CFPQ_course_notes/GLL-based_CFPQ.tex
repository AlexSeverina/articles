\section{Алгоритм на основе нисходящего анализа}

GLL~\cite{Grigorev:2017:CPQ:3166094.3166104}

Другие реализации~\cite{MEDEIROS201975}

\subsection{Нисходящий синтаксический анализ}

Рекурсивный спуск, LL, таблицы, неоднозначности, левая рекурсия.

\subsubsection{Рекурсивный спуск}

\subsubsection{LL(k)-алгоритм синтаксического анализа}

LL(k)-алгоритм синтаксического анализа~--- нисходящий анализ без отката, но с предпросмотром. 
Решение о том, какую продукцию применять, принимается на основании k следующих за текущим символом. 
Временная сложность алгоритма $O(n)$, где $n$~--- длина слова. 

Алгоритм использует входной буфер, стек для хранения промежуточных данных и таблицу анализатора, которая управляет процессом разбора. 
В ячейке таблицы указано правило, которое нужно применять, если рассматривается нетерминал $A$, а следующий символ строки~--- $t$. 
Также в таблице выделена отдельная колонка для $\$$~--- маркера конца строки. 

\begin{center}
  \begin{tabular}{ c || c | c | c | c }
             & $\dots$ & t & $\dots$ & $\$$ \\ \hline  
    $\dots$  & $\dots$ & $\dots$ & $\dots$ & $\dots$ \\ \hline  
    $A$  & $\dots$ & $A \to \alpha$ & $\dots$ & $\dots$ \\ \hline  
    $\dots$  & $\dots$ & $\dots$ & $\dots$ & $\dots$ 
  \end{tabular}  
\end{center}

Для построения таблицы вычисляются множества $\first[k]$ и $\follow[k]$.

\begin{definition}
  Пусть $G = \langle N, \Sigma, P, S \rangle$~--- КС-грамматика. Множество $\first[k]$ определено для сентециальной формы $\alpha$ следующим образом:   
  \[ \first[k](\alpha) = \{ \omega \in \Sigma^* \mid \alpha \derives{} \omega \text{ и } |\omega| < k \text{ либо } \exists \beta: \alpha \derives{} \omega \beta \text{ и } |\omega| = k \} \text{, где } \alpha, \beta \in (N \cup \Sigma)^* \]
\end{definition}

\begin{definition}
  Пусть $G = \langle N, \Sigma, P, S \rangle$~--- КС-грамматика. Множество $\follow[k]$ определено для сентециальной формы $\beta$ следующим образом:
  \[\follow[k](\beta) = \{ \omega \in \Sigma^* \mid \exists \gamma, \alpha: S \derives{} \gamma \beta \alpha \text{ и } \omega \in \first[k](\alpha) \} \]
\end{definition}

В частном случае для $k = 1$: 

\[ \first(\alpha) = \{ a \in \Sigma \mid \exists \gamma \in (N \cup \Sigma)^*: \alpha \derives{} a \gamma \} \text{, где } \alpha \in (N \cup \Sigma)^* \]

\[ \follow(\beta) = \{ a \in \Sigma \mid \exists \gamma, \alpha \in (N \cup \Sigma)^* : S \derives{} \gamma \beta a \alpha \} \text{, где } \beta \in (N \cup \Sigma)^*  \]

Множество $\first$ можно вычислить, пользуясь следующими соотношениями:  

\begin{itemize}
  \item $\first(a \alpha) = \{a\}, a \in \Sigma, \alpha \in (N \cup \Sigma)^* $
  \item $\first(\varepsilon) = \{\varepsilon\}$
  \item $\first(\alpha \beta) = \first(\alpha) \cup (\first(\beta) \text{, если } \varepsilon \in \first(\alpha))$
  \item $\first(A) = \first(\alpha) \cup \first(\beta) \text{, если в грамматике есть правило } A \to \alpha \mid\beta$
\end{itemize}

Алгоритм для вычисления множества $\follow$: 

\begin{itemize}
  \item Положим $\follow(X) = \varnothing, \forall X \in N$
  \item $\follow(S) = \follow(S) \cup \{\$\} \text{, где } S \text{--- стартовый нетерминал}$
  \item Для всех правил вида $A \to \alpha X \beta: \follow(X) = \follow(X) \cup (\first(\beta) \setminus \{\varepsilon\} )$
  \item Для всех правил вида $A \to \alpha X \text{ и } A \to \alpha X \beta \text{, где } \varepsilon \in \first(\beta): \follow(X) = \follow(X) \cup \follow(A)$
  \item Последние два пункта применяются пока есть что добавлять в строящиеся множества. 
\end{itemize}

Пример множеств $\first$ для нетерминалов следующей грамматики: 

\begin{multicols}{2}
\begin{align*}
  S  &\to a S' \\ 
  S' &\to A b B S' \mid \varepsilon \\ 
  A  &\to a A' \mid \varepsilon \\ 
  A' &\to b \mid a \\ 
  B  &\to c \mid \varepsilon
\end{align*}

\columnbreak
    
\begin{align*}
  \first(S)  &= \{ a \} \\
  \first(A)  &= \{ a, \varepsilon \} \\ 
  \first(A') &= \{ a, b \} \\
  \first(B)  &= \{ c, \varepsilon \} \\
  \first(S') &= \{ a, b, \varepsilon \}  
\end{align*}
\end{multicols}

Пример множеств $\follow$ для нетерминалов следующей грамматики:

\begin{multicols}{2}
\begin{align*}
  S  &\to a S' \\ 
  S' &\to A b B S' \mid \varepsilon \\ 
  A  &\to a A' \mid \varepsilon \\ 
  A' &\to b \mid a \\ 
  B  &\to c \mid \varepsilon
\end{align*}

\columnbreak
    
\begin{align*}
  \follow(S)  &= \{ \$ \} & \\
  \follow(S') &= \{ \$ \} &(S \to a S')\\
  \follow(A)  &= \{ b \}  &(S' \to A b B S') \\ 
  \follow(A') &= \{ b \}  &(A \to a A')\\
  \follow(B)  &= \{ a, b, \$ \} &(S' \to A b B S', \varepsilon \in \first(S'))
\end{align*}  
\end{multicols}

Таблица заполняется следующим образом: продукции $A \to \alpha, \alpha \neq \varepsilon$ помещаются в ячейки $(A, a)$, где $a \in \first(A)$, продукции $A \to \varepsilon$~--- в ячейки $(A, a)$, где $a \in \follow(A)$

Пример таблицы для грамматики $S \to ( S ) \mid \varepsilon$

\begin{center}
\begin{tabular}{ r || c | c || c | c | c }
N & $\first$ & $\follow$ & ( & ) & $\$ $ \\ \hline  
$S$ & $\{ (, \varepsilon \}$ & $\{ ), \$ \}$ & $S \rightarrow (S)$ & $S \rightarrow \varepsilon$ & $S \rightarrow \varepsilon$ 
\end{tabular}  
\end{center}

\subsection{GLL для КС запросов}

\subsection{Вопросы и задачи}
\begin{enumerate}
  \item Задача 1
  \item Задача 2
\end{enumerate}
