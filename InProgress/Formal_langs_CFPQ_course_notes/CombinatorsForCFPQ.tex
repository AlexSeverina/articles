\chapter{Комбинаторы для КС запросов}

\section{Парсер комбинаторы}

Что это, с чем едят, плюсы, минусы. Про семантику, безопасность, левую рекурсию и т.д.
Набор примитивных парсеров и функций, которые умеют из существующих арсеров строить более сложные (собственно, комбинаторы парсеров).

Разобрать символ, разобрать последовательность, разобрать альтернативу. впринципе, этого достаточно, но это не очень удобно.

Проблемы с левой рекурсией.
Существуют решения. Одно из них --- Meerkat.
Подробно про него?

\section{Комбинаторы для КС запросов}

Вообще говоря, идея использовать комбинаторы для навигации по графам достаточно очевидно и не нова.
немного про Trails~\cite{Kroni:2013:PGA:2489837.2489844}.

Комбинаторы для запросов к графам на основе Meerkat~\cite{Verbitskaia:2018:PCC:3241653.3241655}

Обобщённые запросы, типобезопасность и всё такое.
Примеры запросов.

\section{Вопросы и задачи}
\begin{enumerate}
  \item Реализовать библиотеку парсер комбинаторов.
  \item Что-нибудь полезное с ними сделать.
\end{enumerate}
