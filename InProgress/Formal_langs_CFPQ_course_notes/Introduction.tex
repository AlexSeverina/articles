\section{Введение}

Поиск путей в графе.

Поиск путей с ограничениями.

Один из способов задавать ограничения на пути в графе основан на использовании языков.
Базоваое определение языка говорит нам, что язык --- это множество слов над некоторым алфавитом.
Если рассмотреть граф, рёбра которого помечены символами из алфавита, то путь в графе будет задавать слово: достаточно соединить последовательно символы, лежащие на рёбрах пути.
Множество же таких путей будет задавать множество слов или язык.
Таким образом, если мы хотим найти некоторое множество путей в графе, то в качестве ограничения можно описать язык, который должно задачать это множество.
Иными словами, завача поиска путей может быть сформулирована следующим образом: необходимо найти такие пути в графе, что слова, получаемые конкатенацей меток их рёбер, принадлежат заданному языку.

Рассмотрим различные варианты постановки задачи.

Различные алгоритмы решения.
