\chapter*{Введение}

Теория формальных языков находит применение не только для ставших уже классическими задач синтаксического анализа кода (языков программирования, искусственных языков) и естественных языков, но и в других областях, таких как статический анализ кода, графовые базы данных, биоинформатика, машинное обучение.

Напримемер, в машинном обучении использование формальных граммтик позволяет передать искусственной нейронной сети, предназначенной для генерации цепочек с определёнными свойствами (генеративной нейронной сети), знания о синтаксической структуре этих цепочек, что позволяет существенно упростить процесс обучения и повысить качество результата~\cite{!!!}. Вместе с этим, развиваются подходы, позволяющие нейронным сетям наоборот извлекать синтаксическую структуру (строить дерево вывода) для входных цепочек~\cite{!!!}.

В биоинформатике формальные граммтики нашли широкое применение для описания вторичноу структуры геномных и белковых последовательностей~\cite{!!!}.

%Интересная теория.
%В том числе Силван Салвати. Связь с графами. Связь с алгеброй.

%Языки бывают разными: язык деревьев, язык графов.

Таким образом, теория формальных языков выступает в качестве основы для многих прикладных областей. Нас же в данной работе будет интересовать её применение к графовым базам данных и мтатическому анализу кода. 

%В том числе про параллельность (шафл). Про типы в Java и CYK-подобный алгоритм.

Одна из классических задач, связанных с анализом графов --- это поиск путей в графе.
Возможны различные формулировки этой задачи.
В некоторых случайх необходимо выяснить, существует ли путь с определёнными свойствами между двумя выбранными вершинами.
В других же ситуациях необходимо найти все пути в графе, удовлетворяющие некоторым свойствам или ограничениям. 
Например, указать, что искомый путь должен быть простым, кратчайшим, гамильтоновым и так далее.

Один из способов задавать ограничения на пути в графе основан на использовании формальных языков.
Базоваое определение языка говорит нам, что язык --- это множество слов над некоторым алфавитом.
Если рассмотреть граф, рёбра которого помечены символами из алфавита, то путь в таком графе будет задавать слово: достаточно соединить последовательно символы, лежащие на рёбрах пути.
Множество же таких путей будет задавать множество слов или язык.
Таким образом, если мы хотим найти некоторое множество путей в графе, то в качестве ограничения можно описать язык, который должно задавать это множество.
Иными словами, задача поиска путей может быть сформулирована следующим образом: необходимо найти такие пути в графе, что слова, получаемые конкатенацей меток их рёбер, принадлежат заданному языку.
Такой класс задач будем называть задачами поиска путей с ограничениям в теринах формальных языков.

Подобный класс задач часто возникает в областях, связанных с анализом граф-структурированных данных и активно исследуется~\cite{doi:10.1137/S0097539798337716,axelsson2011formal,10.1007/978-3-642-22321-1_24,Ward:2010:CRL:1710158.1710234,barrett2007label,doi:10.1137/S0097539798337716}.
Исследуются как классы языков, применяемых для задания ограничений, так и различные постановки задачи.

Граф-структурированный данные встречаются не только в графовых базах данных, но и при статическом анализе кода: по программе можно построить различные графы отображающие её свойства. Скажем, граф вызовов, граф потока данны и так далее. 
Оказывается, что поиск путей в специального вида графах с сипользованием ограничений в терминах формальных языков позволяет исследовать некоторые свойства программы. Например проводить межпроцедурный анализ указателей или анализ алиасов~\cite{!!!}, строить срезы программ~\cite{!!!}, проводить анализ типов~\cite{!!!}.

Рассмотрим несколько простых примеров.

\begin{example}
Применения в графовых базах данных.
\end{example}

\begin{example}
Применения в статическом анализе кода.
\end{example}

Структура данной работы такова. 
Сперва, в главе~\ref{!!!} мы рассмотрим основные понятия из теории графов, необходимые в данной работе.
Затем, в главе~\ref{!!!} мы введём основные понятия из теории формальных языков.
Далее, в главе~\ref{!!!} рассмотрим различные варианты постановки задачи поиска путей с ограничениями в терминах формальнх языков, обсудим базовые свойства задач, её разрешимость в различных постановках и т.д.. И в тоге зафиксируем постановку, которую будем изучать далее. После этого, в главах~\ref{!!!}--\ref{!!!} мы будем подробно рассматривать различные алгоритмы решения этой задачи, попутно вводя специфичные для рассматриваемого алгоритма структуры данных. Большинство алгоритмов будут основаны на классических алгоритмах синтаксического анализа, таких как CYK или LR.
Все гравы, начиная с~\ref{!!!}, снабжены списком вопросов и задач для самостоятельного решения и закрепления материала.  