\documentclass[a4paper,12pt]{article}  % standard LaTeX, 12 point type
%\documentclass[12pt, a4paper]{book}

\usepackage{algpseudocode}
\usepackage{algorithm}
\usepackage{algorithmicx}

\usepackage{geometry}
\usepackage{amsfonts,latexsym}
\usepackage{amsthm}
\usepackage{amssymb}
\usepackage[utf8]{inputenc} % Кодировка
\usepackage[english,russian]{babel} % Многоязычность
\usepackage{mathtools}
\usepackage{hyperref}
\usepackage{tikz}
\usepackage{dsfont}
\usetikzlibrary{fit,calc,automata,positioning}

\theoremstyle{definition}
\newtheorem{definition}{Определение}[section]
\newtheorem{example}{Пример}[section]
\newtheorem{theorem}{Теорема}[section]
\newtheorem{proposition}[theorem]{Proposition}
\newtheorem{lemma}[theorem]{Лемма}
\newtheorem{corollary}[theorem]{Corollary}
\newtheorem{conjecture}[theorem]{Conjecture}


% unnumbered environments:

\theoremstyle{remark}
\newtheorem*{remark}{Remark}
\newtheorem*{notation}{Notation}
\newtheorem*{note}{Note}

\setlength{\parskip}{5pt plus 2pt minus 1pt}
%\setlength{\parindent}{0pt}


\algtext*{EndWhile}% Remove "end while" text
\algtext*{EndIf}% Remove "end if" text
\algtext*{EndFor}% Remove "end for" text
\algtext*{EndFunction}% Remove "end function" text


\usepackage{color}
\usepackage{listings}
\usepackage{caption}
\usepackage{graphicx}
\usepackage{ucs}

\graphicspath{{pics/}}

\geometry{left=2cm}
\geometry{right=1.5cm}
\geometry{top=2cm}
\geometry{bottom=2cm}




%\lstnewenvironment{algorithm}[1][]
%{
%    \lstset{
%        frame=tB,
%        numbers=left,
%        mathescape=true,
%        numberstyle=\small,
%        basicstyle=\small,
%        inputencoding=utf8,
%        extendedchars=\true,
%        keywordstyle=\color{black}\bfseries,
%        keywords={,function, procedure, return, datatype, function, in, if, else, for, foreach, while, denote, do, and, then, assert,}
%        numbers=left,
%        xleftmargin=.04\textwidth,
%        #1 % this is to add specific settings to an usage of this environment (for instnce, the caption and referable label)
%    }
%}
%{}

\newcommand{\tab}[1][0.3cm]{\ensuremath{\hspace*{#1}}}

\newcommand{\rvline}{\hspace*{-\arraycolsep}\vline\hspace*{-\arraycolsep}}

\newcommand{\derives}[1][*]{\xRightarrow[]{#1}}

\setcounter{MaxMatrixCols}{20}


\tikzset{
%->, % makes the edges directed
%>=stealth’, % makes the arrow heads bold
node distance=4cm, % specifies the minimum distance between two nodes. Change if necessary.
%every state/.style={thick, fill=gray!10}, % sets the properties for each ’state’ node
initial text=$ $, % sets the text that appears on the start arrow
}

\tikzstyle{symbol_node} = [shape=rectangle, rounded corners, draw, align=center]
\tikzstyle{prod_node} = [shape=rectangle, draw, align=center]

\tikzset{
    between/.style args={#1 and #2}{
         at = ($(#1)!0.5!(#2)$)
    }
}

%every node/.style = {shape=rectangle, rounded corners,
%      draw, align=center,
%      top color=white, bottom color=blue!20}

\title{Приложения теории формальных языков и синтаксического анализа}
\author{Семён Григорьев}
\date{\today}

\begin{document}
\maketitle
\newpage
\tableofcontents
\newpage

\section{Introduction}

Foundation in some areas: graphs, code analysis, etc.
Why is it important to proof B-H in Coq?
Bar-Hillel theorem is a main on �.
Short overview of current results.

\chapter{Общие сведения теории графов}

В данном разделе мы дадим определения базовым понятиям из теории графов, рассмотрим несколько классических задач из области анализа графов и алгоритмы их решения.
Всё это понадобится нам при последующей работе.

\section{Основные определения}

\begin{definition}
  \textit{Граф} $\mathcal{G} = \langle V, E, L \rangle$, где $V$ --- конечное множество вершин, $E$ --- конечное множество рёбер, т.ч. $E \subseteq V \times L \times V$, $L$ --- конечное множество меток рёбер.
\end{definition}

В дальнейшем нам будут нужны конечные ориентированные помеченные графы.
Мы будем использовать термин \textit{граф} подразумевая именно конечный ориентированный помеченный граф, если только не оговорено противное.

Также мы будем считать, что все вершины занумерованы подряд с нуля.
То есть можно считать, что $V$ --- это отрезок неотрицательный целых чисел.

\begin{example}[Пример графа и его графического представления]
  Пусть дан граф $$\mathcal{G}_1 = \langle \{0,1,2,3\}, \{(0,a,1), (1,a,2), (2,a,0), (2,b,3), (3,b,2)\}, \{a,b\} \rangle.$$
  Графическое представление графа $\mathcal{G}_1$:
  \begin{center}
  \begin{tikzpicture}[on grid, auto]
     \node[state] (q_0)   {$0$};
     \node[state] (q_1) [above right=1.4cm and 1.0cm of q_0] {$1$};
     \node[state] (q_2) [right=2.0cm of q_0] {$2$};
     \node[state] (q_3) [right=2.0cm of q_2] {$3$};
      \path[->]
      (q_0) edge  node {a} (q_1)
      (q_1) edge  node {a} (q_2)
      (q_2) edge  node {a} (q_0)
      (q_2) edge[bend left, above]  node {b} (q_3)
      (q_3) edge[bend left, below]  node {b} (q_2);
  \end{tikzpicture}
  \end{center}
\end{example}

\begin{definition}
  \textit{Ребро} ориентированного помеченного графа $\mathcal{G} = \langle V, E, L \rangle$ это упорядоченная тройка $e = (v_i,l,v_j) \in V \times L \times V$.
\end{definition}

\begin{example}
$(0,a,1)$  и $(3,b,2)$ --- это рёбра графа $\mathcal{G}_1$. При этом, $(3,b,2)$ $(2,b,3)$ --- это разные рёбра, что видно из рисунка.
\end{example}

\begin{definition}
  \textit{Путём} $\pi$ в графе $\mathcal{G}$ будем называть последовательность рёбер такую, что для любых двух последовательных рёбер $e_1=(u_1,l_1,v_1)$ и $e_2=(u_2,l_2,v_2)$ в этой последовательности, конечная вершина первого ребра является начальной вершиной второго, то есть $v_1 = u_2$. Будем обозначать путь из вершины $v_0$ в вершину $v_n$ как $$v_0 \pi v_n = e_0,e_1, \dots, e_{n-1} = (v_0, l_0, v_1),(v_1,l_1,v_2),\dots,(v_{n-1},l_n,v_n).$$

\begin{center}
  \begin{tikzpicture}[on grid, auto]
     \node[state] (v_1)   {$v_1$};
     \node[state] (v_n) [right=2.0cm of v_1] {$v_n$};
      \path[->]
      (v_1) edge [out=45] node {$\pi$} (v_n);
  \end{tikzpicture}
  \end{center}
\end{definition}

\begin{example}
$(0,a,1)(1,a,2) = 0\pi_1 2$  --- путь из вершины 0 в вершину 2 в графе $\mathcal{G}_1$.
При этом, $(0,a,1)(1,a,2)(2,b,3)(3,b,2) = 0\pi_2 2$ --- это тоже путь из вершины 0 в вершину 2 в графе $\mathcal{G}_1$, но он не равен $0\pi_1 2$.
\end{example}

Нам потребуется также отношение, отражающее факт существования пути между двумя вершинами.

\begin{definition}\label{def:reach}
  \textit{Отношение достижимости} в графе:
  $(v_i,v_j) \in P \iff \exists v_i \pi v_j$.
\end{definition}

Отметим, что рефлексивность этого отношения часто зависит от контекста.
В некоторых задачах по-умолчанию $(v_i,v_i) \notin P$, а чтобы это было верно, требуется явное наличие ребра-петли.

Один из способов задать граф --- это задать его \textit{матрицу смежности}.

\begin{definition}
  \textit{Матрица смежности} графа $\mathcal{G}=\langle V,E,L \rangle$ --- это квадратная матрица $M$ размера $n \times n$, где $|V| = n$ и ячейки которой содержат множества.
  При этом $l \in M[i,j] \iff \exists e = (i,l,j) \in E$.
\end{definition}

Заметим, что наше определение матрицы смежности отличается от классического, в котором матрица отражает лишь факт наличия хотя бы одного ребра и, соответственно, является булевой. То есть $M[i,j] = 1 \iff \exists e = (i,\_,j) \in E$.


Также можно встретить матрицы смежности в ячейках которых всё же хранится некоторая информация, однако, в единственном экземпляре. То есть запрещены параллельные рёбра.
Такой подход часто можно встретить в задачах о кратчайших путях: в этом случае в ячейке хранится расстояние между двумя вершинами.
При этом, так как в качестве весов часто рассматривают произвольные (в том числе отрицательные) числа, то в задачах о кратчайших путях отдельно вводят значение ``бесконечность'' для обозначения ситуации, когда между двумя вершинами нет пути или его длина ещё не известна.


\begin{example}
  Неориентированный граф:
  \begin{center}
  \begin{tikzpicture}[on grid,auto]
     \node[state] (q_0)   {$0$};
     \node[state] (q_1) [above right = 1.4cm and 1cm of q_0] {$1$};
     \node[state] (q_2) [right = 2cm of q_0] {$2$};
     \node[state] (q_3) [right = 2cm of q_2] {$3$};
      \path[-]
      (q_0) edge  node {} (q_1)
      (q_1) edge  node {} (q_2)
      (q_2) edge  node {} (q_0)
      (q_2) edge  node {} (q_3);
  \end{tikzpicture}
  \end{center}

  И его матрица смежности:
  $$
  \begin{pmatrix}
    1 & 1 & 1 & 0 \\
    1 & 1 & 1 & 0 \\
    1 & 1 & 1 & 1 \\
    0 & 0 & 1 & 1
  \end{pmatrix}
  $$
\end{example}

\begin{example}
  Ориентированный граф:
  \begin{center}
  \begin{tikzpicture}[shorten >=1pt,on grid,auto]
     \node[state] (q_0)   {$0$};
     \node[state] (q_1) [above right = 1.4cm and 1cm of q_0] {$1$};
     \node[state] (q_2) [right = 2cm of q_0] {$2$};
     \node[state] (q_3) [right = 2cm of q_2] {$3$};
      \path[->]
      (q_0) edge  node {} (q_1)
      (q_1) edge  node {} (q_2)
      (q_2) edge  node {} (q_0)
      (q_2) edge[bend left, above]  node {} (q_3)
      (q_3) edge[bend left, below]  node {} (q_2);
  \end{tikzpicture}
  \end{center}

  И его матрица смежности:
  $$
  \begin{pmatrix}
    1 & 1 & 0 & 0 \\
    0 & 1 & 1 & 0 \\
    1 & 0 & 1 & 1 \\
    0 & 0 & 1 & 1
  \end{pmatrix}
  $$
\end{example}

\begin{example}
  Помеченный граф:
  \begin{center}
  \begin{tikzpicture}[shorten >=1pt,on grid,auto]
     \node[state] (q_0)   {$0$};
     \node[state] (q_1) [above right = 1.4cm and 1cm of q_0] {$1$};
     \node[state] (q_2) [right = 2cm of q_0] {$2$};
     \node[state] (q_3) [right = 2cm of q_2] {$3$};
      \path[->]
      (q_0) edge  node {a} (q_1)
      (q_1) edge  node {a} (q_2)
      (q_2) edge  node {a} (q_0)
      (q_2) edge[bend left = 20]  node {a} (q_3)
      (q_2) edge[bend left = 60]  node {b} (q_3)
      (q_3) edge[bend left, below]  node {b} (q_2);
  \end{tikzpicture}
  \end{center}

  И его матрица смежности:
  $$
  \begin{pmatrix}
    \varnothing   & \{a\}       & \varnothing & \varnothing \\
    \varnothing   & \varnothing & \{a\}       & \varnothing \\
    \{a\}         & \varnothing & \varnothing & \{a,b\} \\
    \varnothing   & \varnothing & \{b\}       & \varnothing
  \end{pmatrix}
  $$
\end{example}

\begin{example}
  Взвешенный граф для задачи о кратчайших путях:
  \begin{center}
  \begin{tikzpicture}[shorten >=1pt,on grid,auto]
     \node[state] (q_0)   {$0$};
     \node[state] (q_1) [above right = 1.4cm and 1cm of q_0] {$1$};
     \node[state] (q_2) [right = 2cm of q_0] {$2$};
     \node[state] (q_3) [right = 2cm of q_2] {$3$};
      \path[->]
      (q_0) edge  node {-1.4} (q_1)
      (q_1) edge  node {2.2} (q_2)
      (q_2) edge  node {0.5} (q_0)
      (q_2) edge[bend left, above]  node {1.85} (q_3)
      (q_3) edge[bend left, below]  node {-0.76} (q_2);
  \end{tikzpicture}
  \end{center}

  И его матрица смежности (для задачи о кратчайших путях):
  $$
  \begin{pmatrix}
    0 & -1.4 & \infty & \infty \\
    \infty & 0 & 2.2 & \infty \\
    0.5 & \infty & 0 & 1.85 \\
    \infty & \infty & -0.76 & 0
  \end{pmatrix}
  $$
\end{example}

Мы ввели лишь общие понятия.
Специальные понятия, необходимые для изложения конкретного материала, будут даны в соответствующих главах.

\section{Задачи поиска путей}

Одна из классических задач анализа графов --- это задача поиска путей между вершинами с различными ограничениями.

При этом, возможны различные постановки задачи.
С одной стороны, по тому, что именно мы хотим получить в качестве результата:
\begin{itemize}
\item Наличие пути, удовлетворяющего ограничениям, в графе.
\item Наличие пути, удовлетворяющего ограничениям, между некоторыми вершинами: задача достижимости.
      Достижима ли вершина $v_1$ из вершины $v_2$ по пути, удовлетворяющему ограничениям.
      Такая постановка требует лишь проверить существование, но не обязательно предоставлять путь.
\item Поиск одного пути, удовлетворяющего ограничениям. Уже надо предъявлять путь.
\item Поиск всех путей. Надо предоставить все пути.
\end{itemize}

С другой стороны, задачи различаются ещё и по тому, как фиксируем вершины:
\begin{itemize}
\item из одной вершины до всех
\item между всеми парами вершин
\item межу фиксированной парой вершин
\item между двумя множествами вершин
\end{itemize}

Итого, можем сгенерировать прямое произведение различных постановок.

Ограничение, имеющее важное прикладное значение, --- минимальность длины.
Иными словами, важная прикладная задача --- \textit{поиск кратчайших путей в графе (англ. APSP --- all-pairs shortest paths)}

Часто добавляют ограничения, что путь должен быть простым и другие.

\section{APSP и транзитивное замыкание графа}

Заметим, что отношение достижимости (\ref{def:reach}) является транзитивным.
Действительно, если существует путь из $v_i$ в $v_j$ и путь из $v_j$ в $v_k$, то существует путь из $v_i$ в $v_k$.

\begin{definition}
  \textit{Транзитивным замыканием графа} называется транзитивное замыкание отношения достижимости по всему графу.
\end{definition}

Как несложно заметить, транзитивное замыкание графа --- это тоже граф.
Более того, если решить задачу поиска кратчайших путей между всеми парами вершин, то мы построим транзитивное замыкание.
Поэтому сразу рассмотрим алгоритм Флойда-Уоршелла~\cite{Floyd1962, Bernard1959, Warshall1962}, который является общим алгоритмом поиска кратчайших путей (умеет обрабатывать рёбра с отрицательными весами, чего не может, например, алгоритм Дейкстры). Его сложность: $O(n^3)$, где $n$ --- количество вершин в графе.
При этом, данный алгоритм легко упростить до алгоритма построения транзитивного замыкания.

\begin{algorithm}
  \floatname{algorithm}{Listing}
\begin{algorithmic}[1]
\caption{Алгоритм Флойда-Уоршелла}
\label{lst:algoFloydWarxhall}
\Function{FloydWarshall}{$\mathcal{G}$}
    \State{$M \gets$ матрица смежности $\mathcal{G}$}
    \State{$n \gets$ $|V(\mathcal{G})|$}
    \For{k = 0; k < n; k++}
      \For{i = 0; i < n; i++}
        \For{j = 0; j < n; j++}
          \State{$M[i,j] \gets$ min$(M[i,j], M[i,k] + M[k,j])$}
        \EndFor
      \EndFor
    \EndFor
\State \Return $M$
\EndFunction
\end{algorithmic}
\end{algorithm}


\begin{example}
  Пусть дан следующий граф:
  \begin{center}
  \begin{tikzpicture}[shorten >=1pt,on grid,auto]
     \node[state] (q_0)   {$0$};
     \node[state] (q_1) [above right = 1.4cm and 1cm of q_0] {$1$};
     \node[state] (q_2) [right = 2cm of q_0] {$2$};
     \node[state] (q_3) [right = 2cm of q_2] {$3$};
      \path[->]
      (q_0) edge  node {} (q_1)
      (q_1) edge  node {} (q_2)
      (q_2) edge  node {} (q_0)
      (q_2) edge[bend left, above]  node {} (q_3)
      (q_3) edge[bend left, below]  node {} (q_2);
  \end{tikzpicture}
  \end{center}

  Построим его транзитивное замыкание:
  \begin{center}
  \begin{tikzpicture}[shorten >=1pt,on grid,auto]
     \node[state] (q_0)   {$0$};
     \node[state] (q_1) [above right = 1.4cm and 1cm of q_0] {$1$};
     \node[state] (q_2) [right = 2cm of q_0] {$2$};
     \node[state] (q_3) [right = 2cm of q_2] {$3$};
      \path[->]
      (q_0) edge[loop below] node {} ()
      (q_1) edge[loop above] node {} ()
      (q_2) edge[loop below] node {} ()
      (q_3) edge[loop below] node {} ()

      (q_0) edge  node {} (q_1)
      (q_1) edge[bend right] node {} (q_0)
      (q_1) edge  node {} (q_2)
      (q_2) edge[bend right] node {} (q_1)
      (q_2) edge  node {} (q_0)
      (q_0) edge[bend right] node {} (q_2)
      (q_2) edge[bend left, above]  node {} (q_3)
      (q_3) edge[bend left, below]  node {} (q_2)
      (q_0) edge[bend right = 60]  node {} (q_3)
      (q_1) edge[bend left, above]  node {} (q_3);
  \end{tikzpicture}
  \end{center}
  \begin{remark}
    На самом деле, петли у вершины 3 может и не быть, т.к. это зависит от определения.
  \end{remark}
\end{example}

\begin{remark}
Приведем список интересных работ по APSP для ориентированных графов с вещественными весами (здесь $n$ --- количество вершин в графе):
\begin{itemize}
    \item M.L. Fredman (1976) --- $O(n^3(\log \log n / \log n)^\frac{1}{3})$ --- использование дерева решений~\cite{FredmanAPSP1976}
    \item W. Dobosiewicz (1990) --- $O(n^3 / \sqrt{\log n})$ --- использование операций на Word Random Access Machine~\cite{Dobosiewicz1990}
    \item T. Takaoka (1992) --- $O(n^3 \sqrt{\log \log n / \log n})$ --- использование таблицы поиска~\cite{Takaoka1992}
    \item Y. Han (2004) --- $O(n^3 (\log \log n / \log n)^\frac{5}{7})$~\cite{Han2004}
    \item T. Takoaka (2004) --- $O(n^3 (\log \log n)^2 / \log n)$~\cite{Takaoka2004}
    \item T. Takoaka (2005) --- $O(n^3 \log \log n / \log n)$~\cite{Takaoka2005}
    \item U. Zwick (2004) --- $O(n^3 \sqrt{\log \log n} / \log n)$~\cite{Zwick2004}
    \item T.M. Chan (2006) --- $O(n^3 / \log n)$ --- многомерный принцип ``разделяй и властвуй''~\cite{Chan2008}
    \item и др.
\end{itemize}
\end{remark}

\section{APSP и произведение матриц}
Алгоритм Флойда-Уоршелла можно переформулировать в терминах перемножения матриц. Для этого введём обозначение.


\begin{definition}
Пусть даны матрицы $A$ и $B$ размера $n \times n$. Определим операцию $\star$, которую называют \textit{Min-plus matrix multiplication}:

    $A \star B = C$ --- матрица размера $n \times n$, т.ч.
    $C[i,j] = \min_{k = 1 \dots n} \{ A[i,k] + B[k,j] \}$
\end{definition}

Также, обозначим за $D[i,j](k)$ минимальную длину пути из вершины $i$ в $j$, содержащий максимум $k$ рёбер.
Таким образом, $D(1) = M$, где $M$ --- это матрица смежности, а решением APSP является $D(n-1)$, т.к. чтобы соединить $n$ вершин требуется не больше $n-1$ рёбер.

\begin{center}
    $D(1) = M$ \\
    $D(2) = D(1) \star M = M^2$ \\
    $D(3) = D(2) \star M = M^3$ \\
    $\dots$ \\
    $D(n-1) = D(n-2) \star M = M^{n-1}$ \\
\end{center}

Итак, мы можем решить APSP за $O(n K(n))$, где $K(n)$ --- сложность алгоритма умножения матриц.
Сразу заметим, что, например, $D(3)$ считать не обязательно, т.к. можем посчитать $D(4)$ как $D(2) \star D(2)$.
Поэтому:

\begin{center}
    $D(1) = M$ \\
    $D(2) = M^2 = M \star M$ \\
    $D(4) = M^4 = M^2 \star M^2$ \\
    $D(8) = M^8 = M^4 \star M^4$ \\
    $\dots$ \\
    $D(2^{\log(n-1)}) = M^{2^{\log(n-1)}} = M^{2^{\log(n-1)} - 1} \star M^{2^{\log(n-1)} - 1}$ \\
    $D(n-1) = D(2^{\log(n-1)})$ \\
\end{center}

Теперь вместо $n$ итераций нам нужно $\log{n}$. В итоге, сложность --- $O(\log{n} K(n))$.
Данный алгоритм называется \textit{repeated squaring}\footnote{Пример решения APSP с помощью repeated squaring: \url{http://users.cecs.anu.edu.au/~Alistair.Rendell/Teaching/apac_comp3600/module4/all_pairs_shortest_paths.xhtml}}.

\section{Умножение матриц с субкубической сложностью}
В предыдущем подразделе мы свели проблему APSP к проблеме min-plus matrix multiplication, поэтому взглянем на эффективные алгоритмы матричного умножения.

Сложность наивного произведения двух матриц составляет $O(n^3)$, это приемлемо только для малых матриц, для больших же лучше использовать алгоритмы с субкубической сложностью.
Отметим, что мы называем сложность субкубической, если она равна $O(n^{3-\varepsilon})$, где $\varepsilon > 0$, иначе говоря, меньшей, чем $O(n^3)$.

Первый субкубический алгоритм опубликовал Ф. Штрассен в 1969 году, его сложность --- $O(n^{\log_2 7}) \approx O(n^{2.81})$~\cite{Strassen1969}. Основная идея --- рекурсивное разбиение на блоки $2 \times 2$ и вычисление их произведения с помощью только 7 умножений, а не 8.
Впоследствии алгоритмы усовершенствовались до ${\approx} O(n^{2.5})$~\cite{Pan1978,BiniCapoRoma1979,Schonhage1981,CoppWino1982}. В настоящее время наиболее асимптотически быстрым является алгоритм Копперсмита --- Винограда со сложностью $O(n^{2.376})$~\cite{CoppWino1990}.

Несмотря на то, что у приведенных алгоритмов неплохая алгоритмическая сложность, мы всё же не можем использовать их для вычисления min-plus matrix multiplication, так как в субкубических алгоритмах требуется, чтобы элементы образовывали кольцо. Покажем, что $\mathbb{R} \cup \{+\infty\}$ с операциями min и + являются полукольцом, а не кольцом:
\begin{enumerate}
    \item $min(a, b) = min(b, a)$
    \item $min(min(a, b)), c) = min(a, min(b, c)))$
    \item $min(a, +\infty) = min(+\infty, a) = a$, т.е. $+\infty$ --- нейтральный элемент относительно операции min

    \item $(a + b) + c = a + (b + c)$

    \item $min(a, b) + c = min(a + c, b + c)$
    \item $a + min(b, c) = min(a + b, a + c)$

    \item $a + \infty = \infty + a = \infty$
    \item Но для произвольного элемента $a$: $\nexists d$, т.ч. $min(a, d) = min(d, a) = +\infty$, т.е. для любого элемента нет обратного относительно операции min
\end{enumerate}

Таким образом, вопрос о субкубических алгоритмах решения APSP всё ещё открыт~\cite{Chan2010}.
Кроме того, более простая задача APSP с булевыми матрицами также не решена за субкубическую сложность. Приведем обоснование этого факта.

\begin{definition}
  Матрица называется \textit{булевой}, если она состоит из 0 и 1.
\end{definition}

Булевы матрицы с поэлементными операциями дизъюнкции и конъюнкции являются полукольцом. Покажем это: пусть $A$, $B$ и $C$ --- булевы матрицы, тогда:
\begin{enumerate}
    \item $A \vee B = B \vee A$
    \item $(A \vee B) \vee C = A \vee (B \vee C)$
    \item $A \vee N = N \vee A = A$, где $N$ --- матрица, полностью состоящая из 0

    \item $(A \wedge B) \wedge C = A \wedge (B \wedge C)$

    \item $(A \vee B) \wedge C = (A \wedge C) \vee (B \wedge C)$
    \item $A \wedge (B \vee C) = (A \wedge B) \vee (A \wedge C)$

    \item $A \wedge N = N \wedge A = N$
\end{enumerate}

Булевы матрицы тоже не являются кольцом, т.к. не имеют обратный элемент относительно операции дизъюнкции (т.е. для произвольной булевой матрицы $A$: $\nexists D$, т.ч. $D$ --- булева матрица и $A \vee D = D \vee A = N$). Следовательно, субкубические алгоритмы не подходят для перемножения булевых матриц, т.к. в них используются обратные элементы (например, для разности). В частности, они не применимы к классической матрице смежности, которая ведёт себя как булева матрица.

\section{Вопросы и задачи}
\begin{enumerate}
  \item Реализуйте абстракцию полукльца, позволяющую конструировать полукольца с произвольными операциями.
  \item Реализуйте алгоритм произведения матриц над произвольным полукольцом. Используйте результат решения предыдущей задачи.
  \item Используя результаты предыдущих задач, реализуйте алгоритм построения транзитивного замыкания через произведение матриц.
  \item Используя результаты предыдущих задач, реализуйте алгоритм решения задачи APSP для ориентированного через произведение матриц.
  \item Используя существующую библиотеку линейной алгебры для CPU, решите задачу построения транзитивного замыкания графа. 
  \item Используя существующую библиотеку линейной алгебры для CPU, решите задачу APSP для ориентированного графа.
  \item Используя существующую библиотеку линейной алгебры для GPGPU, решите задачу построения транзитивного замыкания графа. 
  \item Используя существующую библиотеку линейной алгебры для GPGPU, решите задачу APSP для ориентированного графа.
  \item Сравните произволительность решений предыдущих задач
\end{enumerate}

\section{Общие сведения теории формальных языков}

В данной главе мы рассмотрим основные понятия из теории формальных языков, которые пригодятся нам в дальнейшем изложении.

\begin{definition}
\textit{Алфавит} --- это конечное множество.
Элементы этого множества будем называть \textit{символами}.
\end{definition}

\begin{example}
  Примеры алфавитов

  \begin{itemize}
    \item Латинский алфавит $\Sigma = \{ a, b, c, \dots, z\}$
    \item Кириллический алфавит $\Sigma = \{ \text{а, б, в, \dots, я}\}$
    \item Алфавит чисел в шестнадцатеричной записи $\Sigma = \{0, 1, 2, 3, 4, 5, 6, 7 ,8,9, A, B, C, D, E, F \}$
  \end{itemize}
\end{example}

Традиционное обозначение для алфавита --- $\Sigma$.
Также мы будем использовать различные прописные буквы латинского алфавита. Для обозначения символов алфавита будем использовать строчные буквы латинского алфавита: $a, b, \dots, x, y, z$.

Будем считать, что над алфавитом $\Sigma$ всегда определена операция конкатенации $(\cdot): \Sigma^* \times \Sigma^* \to \Sigma^*$.
При записи выражений символ точки (обозначение операции конкатенации) часто будем опускать: $a \cdot b = ab$.

\begin{definition}
\textit{Слово} над алфавитом $\Sigma$ --- это конечная конкатенация символов алфавита $\Sigma$: $\omega = a_0 \cdot a_1 \cdot \ldots \cdot a_m$, где $\omega$ --- слово, а для любого $i$ $a_i \in \Sigma$.
\end{definition}

\begin{definition}
\textit{Слово} над алфавитом $\Sigma$ --- это результат конечной конкатенации символов алфавита $\Sigma$: $\omega = a_0 \cdot a_1 \cdot \ldots \cdot a_m$, где $\omega$ --- слово, а для любого $i$ $a_i \in \Sigma$.
Будем называть $m + 1$ длиной слова и обозначать как $|\omega|$.
\end{definition}

\begin{definition}
\textit{Язык} над алфавитом $\Sigma$ --- это множество слов над алфавитом $\Sigma$.
\end{definition}

\begin{example}
  Примеры языков

  \begin{itemize}
    \item Язык целых чисел в двоичной записи $\{0, 1, -1, 10, 11, -10, -11, \dots\}$
    \item Язык всех правильных скобочных последовательностей $\{(), (()), ()(), (())(), \dots\}$    
  \end{itemize}
\end{example}

Любой язык над алфавитом $\Sigma$ является подмножеством $\Sigma^*$ --- множества всех слов над алфавитом $\Sigma$

Заметим, что язык не обязан быть конечным множеством, в то время как алфавит всегда конечен и изучаем мы конечные слова.

\begin{definition}
\textit{Способы задания языков}
\begin{itemize}
\item Перечислить все элементы. Такой способ работает только для конечных языков. Перечислить бесконечное множество не получится.
\item Задать генератор --- процедуру, которая возвращает очередное слово языка.
\item Задаьть распознователь --- процедуру, которая по данному слову может определить, принадлежит оно заданному языку ли нет.
\end{itemize}


\subsection{Контекстно-свободные грамматики и языки}

Из всего многообразия нас будут интересовать прежде всего контекстно-свободные грамматики.

\begin{definition}
\textit{Контекстно-свободная грамматика} --- это четвёрка вида $\langle \Sigma, N, P, S \rangle$, где
\begin{itemize}
  \item $\Sigma$ --- это терминальный алфавит;
  \item $N$ --- это нетерминальный алфавит;
  \item $P$ --- это множество правил или продукций, таких что каждая продукция имеет вид $N_i \to \alpha$, где $N_i \in N$ и $\alpha \in \{\Sigma \cup N\}^* \cup {\varepsilon}$;
  \item $S$ --- стартовый нетерминал.
  Отметим, что $\Sigma \cap N = \varnothing$.
\end{itemize}
\end{definition}

\begin{example}
Грамматика, задающая язык целых чисел в двоичной записи без лидирующих нулей.

\[
\begin{array}{rcl}
S& \rightarrow & 0 \mid N \mid - N  \\
N& \rightarrow & 1 A \\
A& \rightarrow & 0 A \mid 1 A  \mid \varepsilon\\
\end{array}
\]
\end{example}

\begin{definition}
  \textit{Отношение непосредственной выводимости}. Мы говорим, что последовательность терминалов и нетерминалов $\gamma \alpha \delta$ \textit{непосредственно выводится из} $\gamma \beta \delta$ \textit{при помощи правила} $\alpha \rightarrow \beta$ ($\gamma \alpha \delta \Rightarrow \gamma \beta \delta$), если 
  \begin{itemize}
    \item $\alpha \rightarrow \beta \in P$
    \item $\gamma, \delta \in V^*$
  \end{itemize}
\end{definition}

\begin{definition}
\textit{Отношение выводимости} является рефлексивно-транзитивным замыканием отношения непосредственной выводимости
\begin{itemize}
  \item $\alpha \derives \beta$ означает $\exists \gamma_0, \dots \gamma_k: \ \alpha \derives[] \gamma_0 \derives[] \gamma_1 \derives[] \dots \derives[] \gamma_{k-1} \derives[] \gamma_{k} \derives[] \beta$
  \item Транзитивность: $\forall \alpha, \beta, \gamma \in V^*: \ \alpha \derives \beta, \beta \derives \gamma \Rightarrow \alpha \derives \gamma$
  \item Рефлексивность: $\forall \alpha \in V^*: \ \alpha \derives \alpha$
  \item $\alpha \derives \beta$ --- $\alpha$ выводится из $\beta$
  \item $\alpha \derives[k] \beta$ --- $\alpha$ выводится из $\beta$ за $k$ шагов
  \item $\alpha \derives[+] \beta$ --- при выводе использовалось хотя бы одно правило грамматики
\end{itemize}
\end{definition}


\begin{example}
Пример вывода цепочки $-1101$ в грамматике:
  
  \[
  \begin{array}{rcl}
  S& \rightarrow & 0 \mid N \mid - N  \\
  N& \rightarrow & 1 A \\
  A& \rightarrow & 0 A \mid 1 A  \mid \varepsilon\\
  \end{array}
  \]

  \[ S \Rightarrow - N \Rightarrow - 1 A \Rightarrow - 1 1 A \derives - 1 1 0 1 A \Rightarrow - 1 1 0 1 \]
\end{example}
  





\begin{definition}
  \textit{Рефлексивно-транзитивное замыкание отношения} --- это наименьшее рефлексивное и транзитивное отношение, содержащее исходное.
  \end{definition}

\begin{definition}
\textit{Вывод слова в грамматике}. Слово $\omega \in \Sigma^*$ \textit{выводимо в грамматике} $\langle \Sigma, N, P, S \rangle$, если существует некоторый вывод этого слова из начального нетерминала $S \derives \omega$.

\end{definition}

\begin{definition}
\textit{Левосторонний вывод}. На каждом шаге вывода заменяется самый левый нетерминал. 
\end{definition}

\begin{example}
Пример левостороннего вывода цепочки в грамматике

  \[
    \begin{array}{rcl}
    S& \rightarrow & A A \mid s  \\
    A& \rightarrow & A A \mid B b \mid a \\
    B& \rightarrow & c \mid d 
    \end{array}
  \]
  
  \[ \boldsymbol{S} \derives[] \boldsymbol{A} A \derives[] \boldsymbol{B} b A \derives[] c b \boldsymbol{A} \derives[] c b \boldsymbol{A} A \derives[] c b a \boldsymbol{A} \derives[] c b a a \]
\end{example}

Аналогично можно определить правосторонний вывод.

\begin{definition}
\textit{Язык, задаваемый грамматикой} --- множество строк, выводимых в грамматике $L(G) = \{ \omega \in \Sigma^* \mid S \derives \omega \}$.
\end{definition}

\begin{definition}
  Грамматики $G_1$ и $G_2$ называются \textit{эквивалентными}, если они задают один и тот же язык: $L(G_1) = L(G_2)$
\end{definition}


\begin{example}  Пример эквивалентных грамматик для языка целых чисел в двоичной системе счисления.
  
  \begin{tabular}{p{0.4\textwidth} | p{0.5\textwidth}}

    \[ 
      \begin{array}{rcl}
      \Sigma &=& \{ 0, 1, - \} \\
      N &=& \{ S, N, A \} \\~\\
      S& \rightarrow & 0 \mid N \mid - N  \\
      N& \rightarrow & 1 A \\
      A& \rightarrow & 0 A \mid 1 A  \mid \varepsilon\\
      \end{array}
    \]          
      
    &
    
    \[
      \begin{array}{rcl}
      \Sigma &=& \{ 0, 1, - \} \\
      N &=& \{ S, A \} \\~\\
      S& \rightarrow & 0 \mid 1 A  \mid - 1 A  \\
      A& \rightarrow &  0 A \mid 1 A  \mid \varepsilon\\
      \end{array}
    \]    
    \end{tabular} 
    
\end{example}


\begin{definition}
  \textit{Неоднозначная грамматика} --- грамматика, в которой существует 2 и более выводов для одного слова.
\end{definition}

\begin{example}
  Неоднозначная грамматика для правильных скобочных последовательностей

\[
    S \to (S) \mid S S \mid \varepsilon
\]
\end{example}

\begin{definition}
  \textit{Однозначная грамматика} --- грамматика, в которой существует не более одного вывода для каждого слова.
\end{definition}

\begin{example}
  Однозначная грамматика для правильных скобочных последовательностей

\[
    S \to (S)S \mid \varepsilon
\]
\end{example}

\begin{definition}
  \textit{Существенно неоднозначные языки} --- язык, для которого невозможно построить однозначную грамматику
\end{definition}

\begin{example}
  Пример существенно неоднозначного языка

\[\{a^n b^n c^m \mid n, m \in \mathds{Z}\} \cup \{a^n b^m c^m \mid n,m \in \mathds{Z}\}\]
\end{example}

\subsection{Дерево вывода}
В некоторых случаях не достаточно знать порядок применения правил.
Необходимо структурное представление вывода цепочки в грамматике.
Таким представлением является \textit{дерево вывода}.
\begin{definition}
Деревом вывода цепочки $\omega$ в грамматике $G=\langle \Sigma, N, S, P \rangle$ называется дерево, удовлетворяющее следующим свойствам.

\begin{enumerate}
  \item Помеченное: метка каждого внутреннего узла --- нетерминал, метка каждого листа --- терминал или $\varepsilon$.
  \item Корневое: корень помечен стартовым нетерминалом.
  \item Упорядоченное.
  \item В дереве есть узел с меткой $N_i$ и сыновьями $M_j \dots M_k$ тогда и только тогда, когда в грамматике есть правило вида $N_i \to M_j \dots M_k$.
  \item Крона образует цепочку.
\end{enumerate}
\end{definition}

\begin{example}
  Построим дерево вывода цепочки $ababab$ в грамматике 
  
  \[ G = \langle \{a,b\}, \{S\}, S, {S \to a \ S \ b \ S, S \to \varepsilon} \rangle \]

\begin{center}
  
    \begin{tikzpicture}[sibling distance=4em,
    every node/.style = {shape=rectangle, rounded corners,
      draw, align=center,
      top color=white, bottom color=blue!20}]]
    \node {S}
      child { node {a} }
      child { node {S}
        child { node {$\varepsilon$}}
      }
      child { node {b} }
      child { node {S}
        child {node {a}}
        child { node {S}
          child { node {$\varepsilon$}}
        }
        child { node {b} }
        child { node {S}
          child {node {a}}
          child {node {S}
            child {node {$\varepsilon$}}
          }
          child {node {b}}
          child {node {S}
            child {node {$\varepsilon$}}
          }
        }
      };
  \end{tikzpicture}
\end{center}

\end{example}

\begin{theorem}
  Пусть $G = \langle \Sigma, N, P, S \rangle$ --- КС-грамматика.     
  Вывод $S \derives \alpha$, где $\alpha \in (N \cup \Sigma)^*, \alpha \neq \varepsilon$ существует $\Leftrightarrow$ существует дерево вывода в грамматике $G$ с кроной $\alpha$.
\end{theorem}

\subsection{Пустота КС-языка}

\begin{theorem}
  Существует алгоритм, определяющий, является ли язык, порождаемый КС грамматикой, пустым
\end{theorem}

\begin{proof}
  Следующая лемма утверждает, что если в КС языке есть выводимое слово, то существует другое выводимое слово с деревом вывода не глубже количества нетерминалов грамматики. 
  Для доказательства теоремы достаточно привести алгоритм, последовательно строящий все деревья глубины не больше количества нетерминалов грамматики, и проверяющий, являются ли такие деревья деревьями вывода. 
  Если в результате работы алгоритма не удалось построить ни одного дерева, то грамматика порождает пустой язык.
\end{proof}

\begin{lemma}
  Если в данной грамматике выводится некоторая цепочка, то существует цепочка, дерево вывода которой не содержит ветвей длиннее m, где m --- количество нетерминалов грамматики
\end{lemma}

\begin{proof}
  Рассмотрим дерево вывода цепочки $\omega$. Если в нем есть 2 узла, соответствующих одному нетерминалу A, обозначим их $n_1$ и $n_2$.

  Предположим, $n_1$ расположен ближе к корню дерева, чем $n_2$.

  $S \derives \alpha A_{n_1} \beta \derives \alpha \omega_1 \beta; S \derives \alpha \gamma A_{n_2} \delta \beta \derives \alpha \gamma \omega_2 \delta \beta$, при этом $\omega_2$ является подцепочкой $\omega_1$.

  Заменим в изначальном дереве узел $n_1$ на $n_2$. Полученное дерево является деревом вывода $\alpha \omega_2 \delta$. 
  
  Повторяем процесс замены одинаковых нетерминалов до тех пор, пока в дереве не останутся только уникальные нетерминалы.
  
  В полученном дереве не может быть ветвей длины большей, чем m.

  По построению оно является деревом вывода.
\end{proof}


\subsection{Нормальная форма Хомского}
\label{section:CNF}

\begin{definition}
Контекстно-свободная грамматика $\langle \Sigma, N, P, S\rangle$ находится в \textit{Нормальной Форме Хомского}, если она содержит только правила следующего вида: 

\begin{itemize}
  \item $A \to B C \text{, где } A, B, C \in N^* $
  \item $A \to a \text{, где } A \in N, a \in \Sigma$
  \item $S \to \varepsilon$
\end{itemize}
\end{definition}

\begin{theorem}
Любую КС грамматику можно преобразовать в НФХ.
\end{theorem}

\begin{proof}
  Алгоритм преобразования в НФХ состоит из следующих шагов: 

  \begin{itemize}
    \item Замена неодиночных терминалов
    \item Удаление длинных правил
    \item Удаление $\varepsilon$-правил
    \item Удаление цепных правил
    \item Удаление бесполезных нетерминалов
  \end{itemize}

  То, что каждый из этих шагов преобразует грамматику к эквивалентной, при этом является алгоритмом, доказано в следующих леммах. 
\end{proof}

\begin{lemma}
  Для любой КС-грамматики можно построить эквивалентную, которая не содержит правила с неодиночными терминалами.
\end{lemma}

\begin{proof}
  Каждое правило $A \to B_0 B_1 \dots B_k, k \geq 1$ заменить на множество правил: 
  \begin{itemize}
    \item $A \to C_0 C_1 \dots C_k$
    \item $\{ C_i \to B_i \mid B_i \in \Sigma, C_i \text{ --- новый нетерминал} \}$
  \end{itemize} 
\end{proof}

\begin{lemma}
  Для любой КС-грамматики можно построить эквивалентную, которая не содержит правил длины больше 2. 
\end{lemma}

\begin{proof}
  Каждое правило $A \to B_0 B_1 \dots B_k, k \geq 2$ заменить на множество правил:
  \begin{itemize}
    \item $A \to B_0 C_0$
    \item $C_0 \to B_1 C_1$
    \item $\dots$
    \item $C_{k-3} \to B_{k-2} C_{k-2}$
    \item $C_{k-2} \to B_{k-1} B_k$
  \end{itemize}
\end{proof}


\begin{lemma}
  Для любой КС-грамматики можно построить эквивалентную, не содержащую $\varepsilon$-правил.
\end{lemma}

\begin{proof}
  Определим $\varepsilon$-правила: 
  \begin{itemize}
    \item $A \to \varepsilon$
    \item $A \to B_0 \dots B_k, \forall i: \ B_i$ --- $\varepsilon$-правило.
  \end{itemize}

  Каждое правило $A \to B_0 B_1 \dots B_k$ заменяем на множество правил, где каждое $\varepsilon$-правило удалено во всех возможных комбинациях.
\end{proof}

\begin{lemma}
  Можно удалить все цепные правила
\end{lemma}

\begin{lemma}
  Можно удалить все бесполезные нетерминалы
\end{lemma}

\subsection{Вопросы и задачи}
\begin{enumerate}
  \item Предъявить несколько выводов для одной цепочки.
  \item Построить выводы
  \item Построить деревья вывода !!! Перенести из раздела про SPPF
\end{enumerate}

\chapter{Задача о поиске путей с ограничениями в терминах формальных языков}



В данной главе сформулируем постановку задачи о поиске путей в графе с ограничениями. 
А также приведём несколько примеров областей, в которых применяются алгоритмы решения этой задачи.

\section{Постановка задачи }


Пусть нам дан конечный ориентированный помеченный граф $\mathcal{G}=\langle V,E,L \rangle$.
Функция $\omega(\pi) = \omega((v_0, l_0, v_1),(v_1,l_1,v_2),\dots,(v_{n-1},l_{n-1},v_n)) = l_0 \cdot l_1 \cdot \ldots \cdot l_{n-1} $ строит слово по пути посредством конкатенации меток рёбер вдоль этого пути.
Очевидно, для пустого пути данная функция будет возвращать пустое слово, а для пути длины $n  > 0$ --- непустое слово длины $n$.

Если теперь рассматривать задачу поиска путей, то окажется, что то множество путей, которое мы хотим найти, задаёт множество слов, то есть язык.
А значит, критерий поиска мы можем сформулировать следующим образом: нас интересуют такие пути, что слова из меток вдоль них принадлежат заданному языку.
\begin{definition} \label{def1}
    \textit{Задача поиска путей с ограничениями в терминах формальных языков} заключается в поиске множества путей $\Pi = \{\pi \mid \omega(\pi) \in \mathcal{L}\}$.
    
\end{definition}

В задаче поиска путей мы можем накладывать дополнительные ограничения на путь (например, чтобы он был простым, кратчайшим или Эйлеровым~\cite{kupferman2016eulerian}), но это уже другая история.

Другим вариантом постановки задачи является задача достижимости.

\begin{definition} \label{def2}
    \textit{Задача достижимости} заключается в поиске множества пар вершин, для которых найдется путь с началом и концом в этих вершинах, что слово, составленное из меток рёбер пути, будет принадлежать заданному языку.
    $\Pi' = \{(v_{i}, v_{j}) \mid \exists v_{i} \pi v_{j}, \omega(\pi) \in \mathcal{L}\}$.
    
\end{definition}

При этом, множество $\Pi$ может являться бесконечным, тогда как $\Pi'$ конечно, по причине конечности графа $\mathcal{G}$.

Язык $\mathcal{L}$ может принадлежать разным классам и быть задан разными способами. Например, он может быть регулярным, или контекстно-свободным, или многокомпонентным контекстно-свободным.

Если $\mathcal{L}$ --- регулярный, $\mathcal{G}$ можно рассматривать как недетерминированный конечный автомат (НКА), в котором все вершины и стартовые, и конечные.
Тогда задача поиска путей, в которой $\mathcal{L}$ --- регулярный, сводится к пересечению двух регулярных языков.

Более подробно мы рассмотрим случай, когда $\mathcal{L}$ --- контекстно-свободный язык.

Путь $G = \langle \Sigma, N, P \rangle$ --- контекстно-свободная граммтика.
Будем считать, что $L \subseteq \Sigma$.
Мы не фиксируем стартовый нетерминал в определении грамматики, поэтому, чтобы описать язык, задаваемый ей, нам необходимо отдельно зафиксировать стартовый нетерминал.
Таким образом, будем говорить, что $L(G,N_i) = \{ w | N_i \xRightarrow[G]{*} w  \}$ --- это язык задаваемый граммтикой $G$ со стартовым нетерминалом $N_i$.

\begin{example}
    Пример задачи поиска путей.
    
    Дана грамматика  $G:$
    \begin{align*}
    S   &\to a b \\ 
    S   &\to a S b
    \end{align*}
    
    Эта грамматика задаёт язык $\mathcal{L} = a^n b^n$.
    
    И дан граф $\mathcal{G}:$
    
    \begin{center}
        \begin{tikzpicture}[shorten >=1pt,on grid,auto]
        \node[state] (q_0)   {$0$};
        \node[state] (q_1) [above right=of q_0] {$1$};
        \node[state] (q_2) [right=of q_0] {$2$};
        \node[state] (q_3) [right=of q_2] {$3$};
        \path[->]
        (q_0) edge  node {a} (q_1)
        (q_1) edge  node {a} (q_2)
        (q_2) edge  node {a} (q_0)
        (q_2) edge[bend left, above]  node {b} (q_3)
        (q_3) edge[bend left, below]  node {b} (q_2);
        \end{tikzpicture}
        
    \end{center}
    
    Тогда примерами путей, принадлежащих множеству $\Pi = \{\pi \mid \omega(\pi) \in \mathcal{L}\}$, являются:
    
    \begin{center}
        \begin{tikzpicture}[shorten >=1pt,on grid,auto]
        \node[state] (q_1) {$1$};
        \node[state] (q_2) [right=of q_1] {$2$};
        \node[state] (q_3) [right=of q_2] {$3$};
        \path[->]
        (q_1) edge  node {a} (q_2)
        (q_2) edge  node {b} (q_3);
        \end{tikzpicture}
        
    \end{center}
    
    \begin{center}
        \begin{tikzpicture}[shorten >=1pt,on grid,auto]
        \node[state] (q_0)   {$0$};
        \node[state] (q_1) [right=of q_0] {$1$};
        \node[state] (q_2) [right=of q_1] {$2$};
        \node[state] (q_3) [right=of q_2] {$3$};
        \node[state] (q_4) [right=of q_3] {$2$};
        \path[->]
        (q_0) edge  node {a} (q_1)
        (q_1) edge  node {a} (q_2)
        (q_2) edge  node {b} (q_3)
        (q_3) edge  node {b} (q_4);
        \end{tikzpicture}
        
    \end{center}
    
\end{example}


\section{О разрешимости задачи}

Задачи из определения \ref{def1} и \ref{def2} сводятся к построению пересечения языка $\mathcal{L}$ и языка, задаваемого путями графа, $R$. 
А мы для обсуждения разрешимости задачи рассмотрим более слабую постановку задачи:

\begin{definition}
    Необходимо проверить, что существует хотя бы один такой путь $\pi$ для данного графа, для данного языка $\mathcal{L}$, что $\omega(\pi) \in \mathcal{L}$.
    
\end{definition}

Эта задача сводится к проверке пустоты пересечения языка $\mathcal{L}$ c $R$ --- регулярным языком, задаваемым графом. От класса языка $\mathcal{L}$ зависит её разрешимость:

\begin{itemize}
    \item Если $\mathcal{L}$ регулярный, то получаем задачу пересечения двух регулярных языков: 
    
    $\mathcal{L} \cap R = R'$.
    $R'$ --- также регулярный язык.
    Проверка регулярного языка на пустоту --- разрешимая проблема.
    
    \item Если $\mathcal{L}$ контекстно-свободный, то получаем задачу
    
    $\mathcal{L} \cap R = CF$ --- контекстно-свободный.
    Проверка контекстно-свободного языка на пустоту --- разрешимая проблема.
    
    \item Помимо иерархии Хомского существуют и другие классификации языков.
    Так например, класс конъюнктивных (Conj)
    языков~\cite{DBLP:journals/jalc/Okhotin01}
    является строгим расширением контекстно-свободных языков и все так же позволяет полиномиальный синтаксический анализ.
    
    Пусть $\mathcal{L}$ --- конъюнктивный. При пересечении конъюнктивного и регулярного языков получается конъюнктивный ($\mathcal{L} \cap R = Conj$), а проблема проверки Conj на пустоту не разрешима~\cite{DBLP:journals/tcs/Okhotin03a}.
    
    \item Ещё один класс языков из альтернативной иерархии, не сравнимой с Иерархией Хомского, --- MCFG (multiple context-free grammars)~\cite{SEKI1991191}.
    Как его частный случай --- TAG (tree adjoining grammar)~\cite{Joshi1997}.
    
    Если $\mathcal{L}$ принадлежит классу MCFG, то $\mathcal{L} \cap R$ также принадлежит MCFG. Проблема проверки пустоты MCFG разрешима~\cite{SEKI1991191}.
    
\end{itemize}

Существует ещё много других классификаций языков, но поиск универсальной иерархии до сих пор продолжается.

Далее, для изучения алгоритмов решения, нас будет интересовать задача $R \cap CF$.

\section{Области применения}
\begin{itemize}
    \item Статанализ. 
    Введено Томасом Репсом~\cite{Reps}.
    \item Социальные сети~\cite{Hellings2015PathRF}.
    \item RDF обработка~\cite{10.1007/978-3-319-46523-4_38}.
    \item Биоинформатика~\cite{cfpqBio}.
    \item Применяется для различных межпроцедурных задач~\cite{LabelFlowCFLReachability,specificationCFLReachability,Zheng}.
    \item Графовые БД
    Впервые предложил Михалис Яннакакис~\cite{Yannakakis}.
    
\end{itemize}

\section{Вопросы и задачи}
\begin{enumerate}
    \item Пусть есть граф. Задайте грамматику для поиска всех путей, таких, что....
    \item Существует ли в графе !!! путь из А в Б, такой что!!!
    \item Для графа !!! постройте все пути, удовлетворяющие !!!!
    
    \item Задача 1
    \item Задача 2
\end{enumerate}

\section{CYK для вычисления КС запросов}

В данной главе мы рассмотрим алгоритм CYK, позволяющий установить принадлежность слова грамматике и предоставить его вывод, если таковой имеется.

Наш главный интерес заключается в возможности применения данного алгоритма для решения описанной в предыдущей главе задачи --- поиска путей с ограничениями в терминах формальных языков. Как уже было указано выше, будем рассматривать случай контекстно-свободных языков.

\subsection{Алгоритм CYK}

Алгоритм CYK (Cocke-Younger-Kasami) --- один из классических алгоритмов синтаксического анализа. Его асимптотическая сложность в худшем случае --- $O(n^3 * |G|)$ ($n$ --- размер входной строки, $G$ --- входная грамматика), что выгодно выделяет его среди других алгоритмов парсинга.~\cite{Hopcroft+Ullman/79/Introduction}

Для его применения необходимо, чтобы подаваемая на вход грамматика находилась в Нормальной Форме Хомского (НФХ)~\ref{section:CNF}.  

В основе алгоритма лежит принцип динамического программирования. Используются два соображения (здесь $\omega$ --- слово, $A$, $B$, $C$ --- нетерминалы грамматики, $a$ --- терминал грамматики):  

\begin{enumerate}
\item Для правила вида $A \to a$ очевидно, что из $A$ выводится $\omega$ (с применением этого правила) тогда и только тогда, когда $a = \omega$:

\[
  A \derives[] a \derives \omega \iff \omega = a\]
  
\item Для правила вида $A \to B C$ понятно, что из $A$ выводится $\omega$ (с применением этого правила) тогда и только тогда, когда существуют две цепочки $\omega_1$ и $\omega_2$ такие, что $\omega_1$ выводима из $B$, $\omega_2$ выводима из $C$ и при этом $\omega = \omega_1 \omega_2$: 

\[A \derives[] B C \derives \omega \iff \exists \omega_1, \omega_2 : \omega = \omega_1 \omega_2, B \derives \omega_1, C \derives \omega_2\]

Или в терминах позиций в строке: 

\[A \derives[] B C \derives \omega \iff \exists k \in [1 \dots |\omega|] : B \derives \omega[1 \dots k], C \derives \omega[k+1 \dots |\omega|]\]
\end{enumerate}

В процессе работы алгоритма заполняется булева трехмерная матрица размера $|N| \times n \times n$, где $n$~---  размер входной цепочки, $N$~--- множество нетерминалов в нормализованной грамматике. 
$M[i, j, A] = true \iff A \derives \omega[i \dots j]$

Первым шагом инициализируем матрицу, заполнив значения $M[i, j, A] \text{, где }i = j$: 

\begin{itemize}
  \item $M[i, i, A] = true \text{, если в грамматике есть правило } A \to \omega[i]$.
  \item $M[i, i, A] = false$, иначе.
\end{itemize}

Далее используем динамику: на шаге $m > 1$ предполагаем, что ячейки матрицы $M[i', j', A]$ заполнены для всех нетерминалов $A$ и пар $i', j': j' - i' < m$.
Тогда можно заполнить ячейки матрицы $M[i, j, A] \text{, где } j - i = m$

\[ M[i, j, A] = \bigvee_{A \to B C}^{}{\bigvee_{k=i}^{j-1}{M[i, k, B] \wedge M[k, j, C]}} \]

По итогу работы алгоритма значение в ячейке $M[0, |\omega|, S]$, где $S$ --- стартовый нетерминал грамматики, отвечает на вопрос о выводимости цепочки $\omega$ в грамматике. 

\begin{example}
  Рассмотрим пример работы алгоритма CYK на грамматике правильных скобочных последовательностей в Нормальной Форме Хомского. 

\begin{align*}
S &\to A S_2 \mid \varepsilon \\ 
S_1   &\to A S_2 \\ 
S_2  &\to b \mid B S_1 \mid S S_3 \\ 
S_3 &\to b \mid B S_1 \\
A   &\to a \\ 
B   &\to b
\end{align*}
 
Проверим выводимость цепочки $\omega = a a b b a b$.

Так как трехмерные матрицы рисовать на двумерной бумаге не очень удобно, мы будем иллюстрировать работу алгоритма двумерными матрицами размера $n \times n$, где в ячейках указано множество нетерминалов, выводящих соответствующую подстроку. 

Шаг 1. Инициализируем матрицу элементами на главной диагонали: 

\[
\begin{pmatrix}
\{A\}  		& \varnothing & \varnothing    & \varnothing 	  & \varnothing & \varnothing 	 \\
\varnothing & \{A\} 	  & \varnothing    & \varnothing 	  & \varnothing & \varnothing 	 \\
\varnothing & \varnothing & \{B, S_2, S_3\} & \varnothing 	  & \varnothing & \varnothing 	 \\
\varnothing & \varnothing & \varnothing    & \{B, S_2, S_3\}   & \varnothing & \varnothing 	 \\
\varnothing & \varnothing & \varnothing    & \varnothing 	  & \{A\} 	    & \varnothing 	 \\
\varnothing & \varnothing & \varnothing    & \varnothing 	  & \varnothing & \{B, S_2, S_3\} \\
\end{pmatrix}
\]

Шаг 2. Заполняем диагональ, находящуюся над главной:

\[
\begin{pmatrix}
\{A\}  		& \varnothing & \varnothing    & \varnothing 	  & \varnothing & \varnothing 	 \\
\varnothing & \{A\} 	  & \{S_1\}  		   & \varnothing 	  & \varnothing & \varnothing 	 \\
\varnothing & \varnothing & \{B, S_2, S_3\} & \varnothing 	  & \varnothing & \varnothing 	 \\
\varnothing & \varnothing & \varnothing    & \{B, S_2, S_3\}   & \varnothing & \varnothing 	 \\
\varnothing & \varnothing & \varnothing    & \varnothing 	  & \{A\} 	    & \{S_1\}	 		 \\
\varnothing & \varnothing & \varnothing    & \varnothing 	  & \varnothing & \{B, S_2, S_3\} \\
\end{pmatrix}
\]

Шаг 3. Заполняем следующую диагональ:

\[
\begin{pmatrix}
\{A\}  		& \varnothing & \varnothing    & \varnothing 	  & \varnothing & \varnothing 	 \\
\varnothing & \{A\} 	  & \{S_1\}  		   & \{S_2\}  		  & \varnothing & \varnothing 	 \\
\varnothing & \varnothing & \{B, S_2, S_3\} & \varnothing 	  & \varnothing & \varnothing 	 \\
\varnothing & \varnothing & \varnothing    & \{B, S_2, S_3\}   & \varnothing & \{S_2, S_3\}	 \\
\varnothing & \varnothing & \varnothing    & \varnothing 	  & \{A\} 	    & \{S_1\}	 		 \\
\varnothing & \varnothing & \varnothing    & \varnothing 	  & \varnothing & \{B, S_2, S_3\} \\
\end{pmatrix}
\]

Шаг 4. И следующую за ней:

\[
\begin{pmatrix}
\{A\}  		& \varnothing & \varnothing    & \{S_1, S\}	 	  & \varnothing & \varnothing 	 \\
\varnothing & \{A\} 	  & \{S_1\}  		   & \{S_2\}  		  & \varnothing & \varnothing 	 \\
\varnothing & \varnothing & \{B, S_2, S_3\} & \varnothing 	  & \varnothing & \varnothing 	 \\
\varnothing & \varnothing & \varnothing    & \{B, S_2, S_3\}   & \varnothing & \{S_2, S_3\}	 \\
\varnothing & \varnothing & \varnothing    & \varnothing 	  & \{A\} 	    & \{S_1\}	 		 \\
\varnothing & \varnothing & \varnothing    & \varnothing 	  & \varnothing & \{B, S_2, S_3\} \\
\end{pmatrix}
\]

Шаг 5 Заполняем предпоследнюю диагональ:

\[
\begin{pmatrix}
\{A\}  		& \varnothing & \varnothing    & \{S_1, S\}	 	  & \varnothing & \varnothing 	 \\
\varnothing & \{A\} 	  & \{S_1\}  		   & \{S_2\}  		  & \varnothing & \{S_2\}	 	 \\
\varnothing & \varnothing & \{B, S_2, S_3\} & \varnothing 	  & \varnothing & \varnothing 	 \\
\varnothing & \varnothing & \varnothing    & \{B, S_2, S_3\}   & \varnothing & \{S_2, S_3\}	 \\
\varnothing & \varnothing & \varnothing    & \varnothing 	  & \{A\} 	    & \{S_1\}	 		 \\
\varnothing & \varnothing & \varnothing    & \varnothing 	  & \varnothing & \{B, S_2, S_3\} \\
\end{pmatrix}
\]

\bigbreak 
Шаг 6. Завершаем выполнение алгоритма:

\[
\begin{pmatrix}
\{A\}  		& \varnothing & \varnothing    & \{S_1, S\}	 	  & \varnothing & \{S_1, S\} 	 \\
\varnothing & \{A\} 	  & \{S_1\}  		   & \{S_2\}  		  & \varnothing & \{S_2\}	 	 \\
\varnothing & \varnothing & \{B, S_2, S_3\} & \varnothing 	  & \varnothing & \varnothing 	 \\
\varnothing & \varnothing & \varnothing    & \{B, S_2, S_3\}   & \varnothing & \{S_2, S_3\}	 \\
\varnothing & \varnothing & \varnothing    & \varnothing 	  & \{A\} 	    & \{S_1\}	 		 \\
\varnothing & \varnothing & \varnothing    & \varnothing 	  & \varnothing & \{B, S_2, S_3\} \\
\end{pmatrix}
\]


Стартовый нетерминал находится в верхней правой ячейке, а значит цепочка $a a b b a b$ выводима в нашей грамматике.
\end{example}

\begin{example}
Теперь выполним алгоритм на невыводимой цепочке $abaa$. 

Шаг 1. Инициализируем таблицу:

\[
\begin{pmatrix}
\{A\}  		& \varnothing 	 & \varnothing & \varnothing 	\\  
\varnothing & \{B, S_2, S_3\} & \varnothing & \varnothing   	\\
\varnothing & \varnothing 	 & \{A\} 	   & \varnothing    \\
\varnothing & \varnothing 	 & \varnothing & \{A\} 			\\
\end{pmatrix}
\]

Шаг 2. Заполняем следующую диагональ: 

\[
\begin{pmatrix}
\{A\}  		& \{S_1, S\} 	 & \varnothing & \varnothing 	\\  
\varnothing & \{B, S_2, S_3\} & \varnothing & \varnothing   	\\
\varnothing & \varnothing 	 & \{A\} 	   & \varnothing    \\
\varnothing & \varnothing 	 & \varnothing & \{A\} 			\\
\end{pmatrix}
\]

Больше ни одну ячейку в таблице заполнить нельзя, а значит эта строка не выводится в грамматике правильных скобочных последовательностей. 

\end{example}

\subsection{Алгоритм для графов на основе CYK}

Теперь изменим вид входного слова и немного модифицируем алгоритм. Прежде мы сопоставляли каждому символу слова номер его позиции в цепочке, поэтому при инициализации заполняли главную диагональ матрицы. Вместо этого обозначим числами позиции между буквами таким образом (в качестве примера рассмотрим слово $a a b b a b$ из предыдущего пункта):

\begin{center}
	\begin{tikzpicture}[shorten >=1pt,on grid,auto]
	\node[state] (q_0) at (0,0)  {$0$};
	\node[state] (q_1) at (2,0)  {$1$};
	\node[state] (q_2) at (4,0)  {$2$};
	\node[state] (q_3) at (6,0)  {$3$};
	\node[state] (q_4) at (8,0)  {$4$};
	\node[state] (q_5) at (10,0) {$5$};
	\node[state] (q_6) at (12,0) {$6$};
	\path[->]
	(q_0) edge  node {$a$} (q_1)
	(q_1) edge  node {$a$} (q_2)
	(q_2) edge  node {$b$} (q_3)
	(q_3) edge  node {$b$} (q_4)
	(q_4) edge  node {$a$} (q_5)
	(q_5) edge  node {$b$} (q_6);
	\end{tikzpicture}
\end{center}

Что нужно изменить в описании алгоритма, чтобы он продолжал работать при подобной нумерации? Каждая буква теперь идентифицируется не одним числом, а парой – номера слева и справа от нее. При этом чисел стало на одно больше, чем при прежнем способе нумерации. 

Возьмем матрицу  $|N| \times (n + 1) \times (n + 1)$ и при инициализации будем заполнять не главную диагональ, а диагональ прямо над ней. Таким образом, мы начинаем наш алгоритм с определения значений $M[i, j, A] \text{, где } j = i + 1$. При этом наши дальнейшие действия в рамках алгоритма не изменятся. 

На шаге инициализации матрица выглядит следующим образом:

\[
\begin{pmatrix}
\varnothing & \{A\}  	  & \varnothing & \varnothing    & \varnothing 	  & \varnothing & \varnothing 	 \\
\varnothing & \varnothing & \{A\} 	  & \varnothing    & \varnothing 	  & \varnothing & \varnothing 	 \\
\varnothing & \varnothing & \varnothing & \{B, S_2, S_3\} & \varnothing 	  & \varnothing & \varnothing 	 \\
\varnothing & \varnothing & \varnothing & \varnothing    & \{B, S_2, S_3\} & \varnothing & \varnothing 	 \\
\varnothing & \varnothing & \varnothing & \varnothing    & \varnothing 	  & \{A\} 	    & \varnothing 	 \\
\varnothing & \varnothing & \varnothing & \varnothing    & \varnothing 	  & \varnothing & \{B, S_2, S_3\} \\
\varnothing & \varnothing & \varnothing & \varnothing    & \varnothing 	  & \varnothing & \varnothing	 \\

\end{pmatrix}
\]

А в результате работы алгоритма имеем:

\[
\begin{pmatrix}
\varnothing & \{A\}  	  & \varnothing & \varnothing    & \{S_1, S\}	  & \varnothing & \{S_1, S\} 	 \\
\varnothing & \varnothing & \{A\} 	    & \{S_1\}  		 & \{S_2\}  		  & \varnothing & \{S_2\}	 	 \\
\varnothing & \varnothing & \varnothing & \{B, S_2, S_3\} & \varnothing 	  & \varnothing & \varnothing 	 \\
\varnothing & \varnothing & \varnothing & \varnothing    & \{B, S_2, S_3\} & \varnothing & \{S_2, S_3\}	 \\
\varnothing & \varnothing & \varnothing & \varnothing    & \varnothing 	  & \{A\} 	    & \{S_1\}	 		 \\
\varnothing & \varnothing & \varnothing & \varnothing    & \varnothing 	  & \varnothing & \{B, S_2, S_3\} \\
\varnothing & \varnothing & \varnothing & \varnothing    & \varnothing 	  & \varnothing & \varnothing	 \\
\end{pmatrix}
\]

Мы представили входную строку в виде линейного графа, а на шаге инициализации получили его матрицу смежности. Шаги алгоритма очень напоминают построение транзитивного замыкания графа. Различие заключается в том, что мы ``добавляем ребра`` только для тех пар нетерминалов, для которых существует правило в грамматике, их выводящее. 

Алгоритм можно обобщить и на произвольные графы с метками, рассматриваемые в этом курсе. При этом накладывается ограничение на форму входной грамматики: она должна находиться в ослабленной Нормальной Форме Хомского~\ref{section:CNF}. То есть будем требовать выполнение только следующих 	правил:

\begin{itemize}
	\item $A \to B C \text{, где } A, B, C \in N $
	\item $A \to a \text{, где } A \in N, a \in \Sigma$
\end{itemize}

Шаг инициализации в алгоритме заключается в том, что мы с помощью продукций вида 

\[A \to a \text{, где } A \in N, a \in \Sigma\]

заменяем терминалы на ребрах входного графа на нетерминалы, из которых они выводятся. Затем используем матрицу смежности получившегося графа (обозначим ее $M$) в качестве начального значения. Дальнейший ход алгоритма можно описать псевдокодом:

\begin{algorithm}[H]
	\begin{algorithmic}[1]
		\caption{Context-free recognizer for graphs}
		\label{alg:graphParse}
		\Function{contextFreePathQuerying}{G, $\mathcal{G}$}
		
		\State{$n \gets$ the number of nodes in $\mathcal{G}$}
		\State{$M \gets$ the modified adjacency matrix of $\mathcal{G}$}
		\State{$P \gets$ the set of production rules in $G$}
		\For {$k \in 0..n$}
			\For {$i \in 0..n$}
				\For {$j \in 0..n$}
					\ForAll {$N_1 \in M[i, k]$, $N_2 \in M[k, j]$}
						\If {$N_3 \to N_1 N_2 \in P$ }		
							\State{$M[i, j] \mathrel{+}= \{N_3\}$} 
						\EndIf
					\EndFor 
				\EndFor
			\EndFor 
		\EndFor 
		\State \Return $M$
		\EndFunction
	\end{algorithmic}
\end{algorithm}

Если в некоторой ячейке результируюшей матрицы с номером $(i, j)$ находятся стартовый нетерминал, то это означает, что существует путь из вершины $i$ в вершину $j$, удовлетворяющий данной грамматике.

\begin{example}
Рассмотрим работу алгоритма на графе 

\begin{center}
	\begin{tikzpicture}[shorten >=1pt,on grid,auto]
	\node[state] (q_0)   {$0$};
	\node[state] (q_1) [above right=of q_0] {$1$};
	\node[state] (q_2) [right=of q_0] {$2$};
	\node[state] (q_3) [right=of q_2] {$3$};
	\path[->]
	(q_0) edge  node {$a$} (q_1)
	(q_1) edge  node {$a$} (q_2)
	(q_2) edge  node {$a$} (q_0)
	(q_2) edge[bend left, above]  node {$b$} (q_3)
	(q_3) edge[bend left, below]  node {$b$} (q_2);
	\end{tikzpicture}
\end{center}

и грамматике:

\begin{align*}
S & \to A B \\ 
S & \to S S \\ 
A & \to a   \\
B & \to b	\\
\end{align*}


\textbf{Инициализация.} 
Заменяем терминалы на ребрах графа на нетерминалы, из которых они выводятся, и строим матрицу смежности получившегося графа:

\begin{center}
	\begin{tikzpicture}[shorten >=1pt,on grid,auto]
	\node[state] (q_0)   {$0$};
	\node[state] (q_1) [above right=of q_0] {$1$};
	\node[state] (q_2) [right=of q_0] {$2$};
	\node[state] (q_3) [right=of q_2] {$3$};
	\path[->]
	(q_0) edge  node {$A$} (q_1)
	(q_1) edge  node {$A$} (q_2)
	(q_2) edge  node {$A$} (q_0)
	(q_2) edge[bend left, above]  node {$B$} (q_3)
	(q_3) edge[bend left, below]  node {$B$} (q_2);
	\end{tikzpicture}
\end{center}

\[
\begin{pmatrix}
\varnothing & \{A\}  	  & \varnothing & \varnothing \\
\varnothing & \varnothing & \{A\} 	    & \varnothing \\
\{A\} 		& \{A\} 	  & \varnothing & \{B\} 	  \\
\varnothing & \varnothing & \{B\}	    & \varnothing \\
\end{pmatrix}
\]

\textbf{Итерация 1 ($k = 0$).}
Итерируясь по $i$ и $j$, мы рассматриваем первую строку и первый столбец нашей матрицы. Опуская случаи, когда $M[i, k] = \varnothing$ или $M[k, j] = \varnothing$, в силу их тривиальности, получаем следующую ситуацию:

\begin{itemize}
	\item $i = 2, j = 1:  M[2, 0] = \{A\}, M[0, 1] = \{A\}$, но при этом в грамматике отсутствует правило вида $C \to A A, C \in N$. 
\end{itemize}

Таким образом, на данном шаге граф и его матрица остаются без изменений. \\

\textbf{Итерация 2 ($k = 1$).}
Продолжаем рассматривать все $i$ и $j$:

\begin{itemize}
	\setlength\itemsep{1em}
	\item $i = 0, j = 2:  M[0, 1] = \{A\}, M[1, 2] = \{A\}$, но при этом нет соответствующего правила в грамматике.
	\item $i = 2, j = 2:  M[2, 1] = \{A\}, M[1, 2] = \{A\}$, аналогично предыдущему случаю. 
\end{itemize}

И снова граф и матрица не претерпевают изменений. \\

\textbf{Итерация 3 ($k = 2$).}
Рассматриваемые на данном шаге случаи:

\begin{itemize}
	\setlength\itemsep{1em}
	\item $i = 1, j = 0:  M[1, 2] = \{A\}, M[2, 0] = \{A\}$, нет соответствующей продукции.
	\item $i = 1, j = 1:  M[1, 2] = \{A\}, M[2, 1] = \{A\}$, аналогично.
	\item $i = 1, j = 3:  M[1, 2] = \{A\}, M[2, 3] = \{B\}$, в грамматике есть правило $S \to A B$, значит, добавляем в ячейку $M[1, 3]$ стартовый нетерминал S, а на графе появится соответствуюшее ребро.
	
	\item $i = 3, j = 0:  M[3, 2] = \{B\}, M[2, 0] = \{A\}$, правило вида $C \to B A, C \in N$ отсутствует в грамматике.
	\item $i = 3, j = 1:  M[3, 2] = \{B\}, M[2, 1] = \{A\}$, аналогично.
	\item $i = 3, j = 3:  M[3, 2] = \{B\}, M[2, 3] = \{B\}$, отсутствует продукция вида $C \to B B, C \in N$.
\end{itemize}

Граф и матрица после этой итерации:

\begin{center}
	\begin{tikzpicture}[shorten >=1pt,on grid,auto]
	\node[state] (q_0)   {$0$};
	\node[state] (q_1) [above right=of q_0] {$1$};
	\node[state] (q_2) [right=of q_0] {$2$};
	\node[state] (q_3) [right=of q_2] {$3$};
	\path[->]
	(q_0) edge  node {$A$} (q_1)
	(q_1) edge  node {$A$} (q_2)
	(q_2) edge  node {$A$} (q_0)
	(q_2) edge[bend left, above]  node {$B$} (q_3)
	(q_3) edge[bend left, below]  node {$B$} (q_2)
	(q_1) edge[bend left, above]  node {$S$} (q_3);
	\end{tikzpicture}
\end{center}

\[
\begin{pmatrix}
\varnothing & \{A\}  	  & \varnothing & \varnothing \\
\varnothing & \varnothing & \{A\} 	    & \{S\} 	  \\
\{A\} 		& \{A\} 	  & \varnothing & \{B\} 	  \\
\varnothing & \varnothing & \{B\}	    & \varnothing \\
\end{pmatrix}
\] 

\textbf{Итерация 4 ($k = 3$).}
Рассмариваемые случаи:

\begin{itemize}
	\setlength\itemsep{1em}
	\item $i = 1, j = 2:  M[1, 3] = \{S\}, M[3, 2] = \{B\}$, нет правила вида $C \to S B, C \in N$.
	\item $i = 2, j = 2:  M[2, 3] = \{B\}, M[3, 2] = \{B\}$, отсутствует продукция вида $C \to B B, C \in N$. 
\end{itemize}

На данном шаге матрица и граф не изменились. \\

\textbf{Итог.}
В результате матрица алгоритма выглядит следующим образом:

\[
\begin{pmatrix}
\varnothing & \{A\}  	  & \varnothing & \varnothing \\
\varnothing & \varnothing & \{A\} 	    & \{S\} 	  \\
\{A\} 		& \{A\} 	  & \varnothing & \{B\} 	  \\
\varnothing & \varnothing & \{B\}	    & \varnothing \\
\end{pmatrix}
\]

В ячейке $M[1, 3]$ находится стартовый нетерминал, а значит, существует путь из вершины 1 в вершину 3, удовлетворяющий ограничениям, заданным входной грамматикой.
\end{example}

\subsection{Вопросы и задачи}
\begin{enumerate} 
	\item Проверить работу алгоритма CYK для цепочек на грамматике
	\begin{flushleft}
	$E \to E + E$ \\ 
	$E \to E * E$ \\ 
	$E \to (E)$   \\
	$E \to n$	  \\
	\end{flushleft}	
	и словах (алфавит $\Sigma = \{n, +, *, (, )\}$)
	\begin{flushleft}
	$ (n + n) * n$    \\
	$ n + n * n$      \\
	$n + n + n + n$   \\
	$n + (n * n) + n$ \\
	\end{flushleft}
	
	\item Посчитать вычислительную сложность алгоритма CYK для матриц в зависимости от размера входного графа (размер грамматики считать фиксированным)
	
	\item Проверить работу алгоритма CYK для графов на графе 
	
	\begin{center}
		\begin{tikzpicture}[shorten >=1pt,on grid,auto]
		\node[state] (q_0)  {$0$};
		\node[state] (q_1) [right=of q_0]  {$1$};
		\node[state] (q_2) [right=of q_1]  {$2$};
		\node[state] (q_3) [right=of q_2]  {$3$};
		\node[state] (q_4) [right=of q_3]  {$4$};
		\path[->]
		(q_0) edge  node {$a$} (q_1)
		(q_1) edge  node {$b$} (q_2)
		(q_2) edge  node {$a$} (q_3)
		(q_3) edge  node {$b$} (q_4)
		(q_1) edge[bend left, above]  node {$b$} (q_3)
		(q_4) edge[bend left, below]  node {$a$} (q_1);
		\end{tikzpicture}
	\end{center}
	
	И грамматике
	
	\begin{flushleft}
		$S \to S S$ \\ 
		$S \to A B$ \\ 
		$A \to a$   \\
		$B \to b$	  \\
	\end{flushleft}	

\end{enumerate}

\section{Алгоритм на матричных опреациях}

Вводное.
Про замыкание через произведение матриц.

Вспомниаем про то, что транзитивное замыкание можно искать через произведение матриц.

Матрицы для линейного входа: ссылка на Валианта~\cite{!!!}.
Но этот алгоритм очевидным образом не обобщается на графы (с сохранением асимптотики).


Потому будем действовать наивно.
Оригинальные матрицы (Рустам)~\cite{Azimov:2018:CPQ:3210259.3210264}.

Let $D = (V, E)$ be the input graph and $G = (N,\Sigma,P)$ be the input grammar.

\begin{algorithm}[H]
\begin{algorithmic}[1]
\caption{Context-free recognizer for graphs}
\label{alg:graphParse}
\Function{contextFreePathQuerying}{D, G}
    
    \State{$n \gets$ the number of nodes in $D$}
    \State{$E \gets$ the directed edge-relation from $D$}
    \State{$P \gets$ the set of production rules in $G$}
    \State{$T \gets$ the matrix $n \times n$ in which each element is $\varnothing$}
    \ForAll{$(i,x,j) \in E$}
    \Comment{Matrix initialization}
        \State{$T_{i,j} \gets T_{i,j} \cup \{A~|~(A \rightarrow x) \in P \}$}
    \EndFor    
    \While{matrix $T$ is changing}
       
        \State{$T \gets T \cup (T \times T)$}
        \Comment{Transitive closure $T^{cf}$ calculation} 
    \EndWhile
\State \Return $T$
\EndFunction
\end{algorithmic}
\end{algorithm}


\begin{example}[Пример работы]

Описание


\[
T_0 = \begin{pmatrix}
    \varnothing & \{A\}       & \varnothing & \{B\}       \\
    \varnothing & \varnothing & \{A\}       & \varnothing \\
    \{A\}       & \varnothing & \varnothing & \varnothing \\
    \{B\}       & \varnothing & \varnothing & \varnothing \\
\end{pmatrix}
\]

Let $T_i$ be the matrix $T$, obtained after executing the loop in the lines \textbf{8-9} of the Algorithm~\ref{alg:graphParse} $i$ times. The calculation of the matrix $T_1$ is shown on Figure~\ref{ExampleQueryFirstIteration}.


\[
T_0 \times T_0 = \begin{pmatrix}
    \varnothing & \varnothing & \varnothing & \varnothing \\
    \varnothing & \varnothing & \varnothing & \varnothing \\
    \varnothing & \varnothing & \varnothing & \{S\}       \\
    \varnothing & \varnothing & \varnothing & \varnothing \\
\end{pmatrix}
\]

\[
T_1 = T_0 \cup (T_0 \times T_0) = \begin{pmatrix}
    \varnothing & \{A\}       & \varnothing & \{B\}       \\
    \varnothing & \varnothing & \{A\}       & \varnothing \\
    \{A\}       & \varnothing & \varnothing & \{\pmb{S}\}       \\
    \{B\}       & \varnothing & \varnothing & \varnothing \\
\end{pmatrix}
\]

When the algorithm at some iteration finds new paths in the graph $D$, then it adds corresponding nonterminals to the matrix $T$. For example, after the first loop iteration, non-terminal $S$ is added to the matrix $T$. This non-terminal is added to the element with a row index $i = 2$ and a column index $j = 3$. This means, that there is $i\pi j$ (a path $\pi$ from the node 2 to the node 3), such that $S \xrightarrow{*} l(\pi)$. For example, such a path consists of two edges with labels $a$ and $b$, and thus $S \xrightarrow{*} a \ b$.

The calculation of the transitive closure is completed after $k$ iterations, when a fixpoint is reached: $T_{k-1} = T_k$. For the example query, $k = 13$ since $T_{13} = T_{12}$. The remaining iterations of computing the transitive closure are presented on Figure~\ref{ExampleQueryFinalIterations} (new matrix elements on each iteration are shown in bold).


\begin{alignat*}{7}
& &&T_2 &&= \begin{pmatrix}
\varnothing & \{A\}       & \varnothing & \{B\}       \\
\varnothing & \varnothing & \{A\}       & \varnothing \\
\{A, \pmb{S_1}\}  & \varnothing & \varnothing & \{S\}       \\
\{B\}       & \varnothing & \varnothing & \varnothing \\
\end{pmatrix} \ \ \ \ &&T_3 &&= \begin{pmatrix}
\varnothing & \{A\}       & \varnothing & \{B\}       \\
\{\pmb{S}\}       & \varnothing & \{A\}       & \varnothing \\
\{A, S_1\}  & \varnothing & \varnothing & \{S\}       \\
\{B\}       & \varnothing & \varnothing & \varnothing \\
\end{pmatrix} \ \ \ \ &&T_4 &&= \begin{pmatrix}
\varnothing & \{A\}       & \varnothing & \{B\}       \\
\{S\}       & \varnothing & \{A\}       & \{\pmb{S_1}\}     \\
\{A, S_1\}  & \varnothing & \varnothing & \{S\}       \\
\{B\}       & \varnothing & \varnothing & \varnothing \\
\end{pmatrix}  \\
& &&T_5 &&= \begin{pmatrix}
\varnothing & \{A\}       & \varnothing & \{B, \pmb{S}\}    \\
\{S\}       & \varnothing & \{A\}       & \{S_1\}     \\
\{A, S_1\}  & \varnothing & \varnothing & \{S\}       \\
\{B\}       & \varnothing & \varnothing & \varnothing \\
\end{pmatrix} \ \ \ \ &&T_6 &&= \begin{pmatrix}
\{\pmb{S_1}\}     & \{A\}       & \varnothing & \{B, S\}    \\
\{S\}       & \varnothing & \{A\}       & \{S_1\}     \\
\{A, S_1\}  & \varnothing & \varnothing & \{S\}       \\
\{B\}       & \varnothing & \varnothing & \varnothing \\
\end{pmatrix} \ \ \ \ &&T_7 &&= \begin{pmatrix}
\{S_1\}     & \{A\}       & \varnothing & \{B, S\}    \\
\{S\}       & \varnothing & \{A\}       & \{S_1\}     \\
\{A, S_1, \pmb{S}\}  & \varnothing & \varnothing & \{S\}    \\
\{B\}       & \varnothing & \varnothing & \varnothing \\
\end{pmatrix}  \\
& &&T_8 &&= \begin{pmatrix}
\{S_1\}     & \{A\}       & \varnothing & \{B, S\}    \\
\{S\}       & \varnothing & \{A\}       & \{S_1\}     \\
\{A, S_1, S\}  & \varnothing & \varnothing & \{S, \pmb{S_1}\} \\
\{B\}       & \varnothing & \varnothing & \varnothing \\
\end{pmatrix} \ \ \ \ &&T_9 &&= \begin{pmatrix}
\{S_1\}     & \{A\}       & \varnothing & \{B, S\}    \\
\{S\}       & \varnothing & \{A\}       & \{S_1, \pmb{S}\}     \\
\{A, S_1, S\}  & \varnothing & \varnothing & \{S, S_1\} \\
\{B\}       & \varnothing & \varnothing & \varnothing \\
\end{pmatrix} \ \ \ \ &&T_{10} &&= \begin{pmatrix}
\{S_1\}     & \{A\}       & \varnothing & \{B, S\}    \\
\{S, \pmb{S_1}\}       & \varnothing & \{A\}       & \{S_1, S\}     \\
\{A, S_1, S\}  & \varnothing & \varnothing & \{S, S_1\} \\
\{B\}       & \varnothing & \varnothing & \varnothing \\
\end{pmatrix}  \\
& &&T_{11} &&= \begin{pmatrix}
\{S_1, \pmb{S}\}     & \{A\}       & \varnothing & \{B, S\}    \\
\{S, S_1\}       & \varnothing & \{A\}       & \{S_1, S\}     \\
\{A, S_1, S\}  & \varnothing & \varnothing & \{S, S_1\} \\
\{B\}       & \varnothing & \varnothing & \varnothing \\
\end{pmatrix} \ \ \ \ &&T_{12} &&= \begin{pmatrix}
\{S_1, S\}     & \{A\}       & \varnothing & \{B, S, \pmb{S_1}\}    \\
\{S, S_1\}       & \varnothing & \{A\}       & \{S_1, S\}     \\
\{A, S_1, S\}  & \varnothing & \varnothing & \{S, S_1\} \\
\{B\}       & \varnothing & \varnothing & \varnothing \\
\end{pmatrix} \ \ \ \ &&T_{13} &&= \begin{pmatrix}
\{S_1, S\}     & \{A\}       & \varnothing & \{B, S, S_1\}    \\
\{S, S_1\}       & \varnothing & \{A\}       & \{S_1, S\}     \\
\{A, S_1, S\}  & \varnothing & \varnothing & \{S, S_1\} \\
\{B\}       & \varnothing & \varnothing & \varnothing \\
\end{pmatrix}
\end{alignat*}

Thus, the result of the Algorithm~\ref{alg:graphParse} for the example query is the matrix $T_{13} = T_{12}$. 

\end{example}.

Интересные вопросы: кратчайшие пути и одно дерево (построить один путь).

\subsection{Конъюнктивные и булевы граммтики}

\subsubsection{Определения}

Охотин~\cite{!!!}.

\subsubsection{Для графов}

Классическая семантика --- работает для конъюнктивных для любых графов.
Для Булевых и конъюнктивных только для графов без циклов.

Альтернативная семантика (DataLog).
Трактуем конъюнкцию как в даталоге. Тогда всё хорошо.
Это похоже на даталог через линейную алгебру~\cite{!!!}

\subsection{Особенности реализации матричного алгоритма}

Кое-что про наши реализации~\cite{Mishin:2019:ECP:3327964.3328503}

Так как множество нетерминалов и правил конечно, то мы можем свести алгоритм к булевым матрицам: для каждого нетерминала заводим матрицу, такую что в ячейке стоит true тогда и только тогда, когда в исходной матрице в соответствующе ячейке был этот нетерминал.
Тогда перемноженеи пары таких матриц --- это применеие правила.

Описание модификации!

\begin{example}
Пример работы!
\end{example}

С другой стороны, для небольших запросов практически может быть выгодно представлять множества нетерминалов в виде битового вектора.
Нумеруем все нетерминалы с нуля, в векторе стит 1 на i позиции если в множетве есть нетерминал с номером i.
Таким образом, в каждой ячейке хранится битовый вектор длины $|N|$.
Тогда операцию умножения надо определить следующим образом:
$$v_1 \times v_2 = v \mid \exists (v,v_3) \in P, \textit{append}(v_1, v_2) \& v_3 = v_3,$$ где $\&$ --- побитовое \texttt{``и''}.
Для этого правила надо кодировать соответственно: продукция это пара, где первый элемент --- битовый вектор длины $N$ с единственной единицей в позиции, соответствующей нетерминалу в правой части, а второй элемент --- векто длины $2|N|$, с двумя единицами кодирующими первый и второй нетерминалы, соответственно.

\begin{example}
Пример работы!
\end{example}


На практике в роли векторов могут выступать беззнаковые целые. 
Например, 32 бита под ячейки в матрице и 64 бита под правила (или 8 и 16, если запросы совсем маленькие, или 16 и 32).
Тогда умножение выражается через битовые операции и сравнение.



Это может оказаться быстрее. Надо реализовыть и смотреть.

Так как данные чать разрежены, то много вопросов к представлению матриц.
Разреженные матрицы, плотные матрицы, GraphBLAS\footnote{!!!}, GPGPU, CUTLASS\footnote{Репозиторий библиотеки CUTLASS: \url{https://github.com/NVIDIA/cutlass}}.
Quad Tree~\cite{!!!}.

Расположенеи в памяти: хорошо, когда всё (всематрицы, нужные для вычислений) влезло на одну карту.
Хуже, когда только одна пара матриц.
Ещё хуже, когда даже одна матрица не помещается.

Потому нужно пробовать распределённые решения.
Например, через GraphBLAS.

\subsection{Обзор}

Китайцы~\cite{!!!}, Брэдфорд~\cite{!!!}, Ли~\cite{Lee:2002:FCG:505241.505242}, Хеллингс~\cite{!!!}, OpenCypher~\cite{Kuijpers:2019:ESC:3335783.3335791}.

Рассуждения про ассимптотику.

Субкубический для частного случая (Брэдфорд)~\cite{8249039}.

\subsection{Вопросы и задачи}
\begin{enumerate}
  \item !!!
\end{enumerate}

\section{Через тензорное произведение}

Предыдущий подход позволяет выразить задачу поиска путей с ограничениями в терминах ыормальных языков через набор матричных операций.
Это позволяет использовать высокопроизводительные библиотеки и массовопараллельные архитектуры и вообще круто.
Однако, такой подход требует, чтобы грамматика находилась в ослабленной нормальной форме Хомского, что приводит к её разростанию.
Можно ли как-то избежать этого?

В данном разделе мы попробуем предложить альтернативное сведение задачи поиска путей к матричным опреациям.
В результате мы сможем избежать преобразования граммтики в ОНФХ, однако, матрицы, с которыми нам предётся работать, будут существенно б\'{о}льшего размера.

В основе подхода лежит использование рекурсивных сетей или рекурсивных автоматов в качестве представления контекстно-свободных грамматик.

\subsection{Рекурсивные автоматы и сети}

Рекурсивный автомат или сеть --- это представление контекстно-свободных грамматик, обобщающее конечные автоматы.
В нашей работе мы будем придерживаться термина \textbf{рекурсивный автомат}. 
Классическое определение рекурсивного автомата выглядит следующим образом.

\begin{definition}
Рекурсивный автомат --- это
\end{definition}

\begin{figure}[h]
\begin{center}
\begin{tikzpicture}[shorten >=1pt,on grid,auto] 
   \node[state, initial] (q_0)   {$0 \{S\}$}; 
   \node[state] (q_1) [right=of q_0] {$1$}; 
   \node[state] (q_2) [right=of q_1] {$2$}; 
   \node[state, accepting] (q_3) [right=of q_2] {$3$};
    \path[->] 
    (q_0) edge  node {a} (q_1)          
    (q_1) edge  node {S} (q_2)
    (q_2) edge  node {b} (q_3)
    (q_1) edge[bend left, above]  node {b} (q_3);
\end{tikzpicture}
\end{center}

\caption{Рекурсивный автомат для грамматики!!!}
\label{input}
\end{figure}


\subsection{Тензорное произведение}

Тензорное произведение матриц или произведение Кронекера --- это бинарная опирация, обозначаемая $\otimes$ и определяемая следующим образом.

\begin{definition}
Пусть даны две матрицы: $A$ размера $m\times n$ и $B$ размера $p\times q$.
Произведение Кронекера или тензорное произведение матриц $A$ и $B$ --- это блочная матрица $C$ размера $mp \times nq$, вычисляемая следующим образом:
$$
C = A \otimes B = 
\begin{pmatrix}
A_{0,0}B   & \cdots & A_{0,n-1}B    \\
\vdots     & \ddots & \vdots       \\
A_{m-1,0}B & \cdots &  A_{m-1,n-1}B
\end{pmatrix}
$$
\end{definition}

\newcommand{\examplemtrx}
{
\begin{pmatrix}
5 & 6 & 7 & 8 \\
9 & 10 & 11 & 12 \\
13 & 14 & 15 & 16 
\end{pmatrix}
}

\begin{example}
\begin{align}
\begin{pmatrix}
1 & 2 \\
3 & 4
\end{pmatrix}
\otimes
\examplemtrx &=
\begin{pmatrix}
1\examplemtrx & 2\examplemtrx \\
3\examplemtrx & 4\examplemtrx
\end{pmatrix}
=\notag \\
&=
\left(\begin{array}{c c c c | c c c c}
5  & 6  & 7  & 8  & 10 & 12 & 14 & 16 \\
9  & 10 & 11 & 12 & 18 & 20 & 22 & 24 \\
13 & 14 & 15 & 16 & 26 & 28 & 30 & 32 \\
\hline
15 & 18 & 21 & 24 & 20 & 24 & 28 & 32 \\
27 & 30 & 33 & 36 & 36 & 40 & 44 & 48 \\
39 & 42 & 45 & 48 & 52 & 56 & 60 & 64 
\end{array}\right)
\end{align}
\end{example}

Заметим, что для определения тензорного произведения матриц достаточно определить опреацию умножения на элементах исходных матриц.
Также отметим, что произведение Кронекера не является коммутативной.
При этом всегда существуют такиедве матрицы перестоновок $P$ и $Q$, что $A \otimes B = P(B \otimes A)Q$.
Это свойство потребуется нам в дальнейшем.

Теперь перейдём к графам.
Сперва дадим классическое определение тензорного произведения двух неориентированных графов.
\begin{definition}
Пусть даны два графа: $\mathcal{G}_1 = \langle V_1, E_1\rangle$ и $\mathcal{G}_2 = \langle V_2, E_2\rangle$. 
Тензорным произведением этих графов будем называть граф $\mathcal{G}_3 = \langle V_3, E_3\rangle$, где $V_3 = V_1 \times V_2$, $E_3 = \{ ((v_1,v_2),(u_1,u_2)) \mid (v_1,u_1) \in E_1 \text{ и } (v_2,u_2) \in E_2 \}$.
\end{definition}

Иными словами, тензорным произведением двух графов является граф, такой что:
\begin{enumerate}
 \item множество вершин --- это прямое произведение множемтв вершин исходных графов;
 \item ребро между вершинами $v=(v_1,v_2)$ и $u=(u_1,u_2)$ существует тогда и только тогда, когда существуют рёбра между парами вершин $v_1$, $u_1$ и $v_2$, $u_2$ в соответсвующих графах. 
\end{enumerate}

Для того, чтобы построить тензорное произведение ориентированных графов, необходимо в предыдущем определении, в условии существования реба в результирующем графе, дополнительно потребовать, чтобы направления рёбер совпадали.
Данное требование получается естесвенным образом, если считать, что пары вершин, задающие ребро упорядочены, поэтому формальное определение отличаться не будет.

Осталось добавить метки к рёбрам.
Это приведёт к логичному усилению требованя к существованию ребра: метки рёбер в исходных графах должны совпадать.
Таким образом, мы получаем следующее определение тензорного произведения ориентированных графов с метками на рёбрах.
\begin{definition}

Пусть даны два ориентированных графа с метками на рёбрах: $\mathcal{G}_1 = \langle V_1, E_1, L_1 \rangle$ и $\mathcal{G}_2 = \langle V_2, E_2, L_2 \rangle$.
Тензорным произведением этих графов будем называть граф $\mathcal{G}_3 = \langle V_3, E_3, L_3\rangle$, где $V_3 = V_1 \times V_2$, $E_3 = \{ ((v_1,v_2),l,(u_1,u_2)) \mid (v_1,l,u_1) \in E_1 \text{ и } (v_2,l,u_2) \in E_2 \}$, $L_3=L_1 \cap L_2$.

\end{definition}

Нетрудно заметить, что матрица смежности графа $\mathcal{G}_3$ равна тенорному произведению матриц смежностей исходных графов $\mathcal{G}_1$ и $\mathcal{G}_2$.

Рассмотрим пример.
В качестве одного из графов возьмём рекурсивный автомат, построенный ранее!!!.
Его матрица смежности выглядит следующим образом.
$$ M_1 =
\begin{pmatrix} 
. & [a] & . & . \\
. & . & [S] & [b] \\
. & . & . & [b] \\
. & . & . & . 
\end{pmatrix}
$$


\begin{figure}[h]
\begin{center}
\begin{tikzpicture}[shorten >=1pt,on grid,auto] 
   \node[state] (q_0)   {$0$}; 
   \node[state] (q_1) [above right=of q_0] {$1$}; 
   \node[state] (q_2) [right=of q_0] {$2$}; 
   \node[state] (q_3) [right=of q_2] {$3$};
    \path[->] 
    (q_0) edge  node {a} (q_1)          
    (q_1) edge  node {a} (q_2)
    (q_2) edge  node {a} (q_0)
    (q_2) edge[bend left, above]  node {b} (q_3)
    (q_3) edge[bend left, below]  node {b} (q_2);
\end{tikzpicture}
\end{center}

\caption{The input graph}
\label{input}
\end{figure}


Второй граф представлен на изображении~\ref{input}.
Его матрица смежности имеет следующий вид.
$$ M_2 =
\begin{pmatrix} 
. & [a] & . & . \\
. & . & [a] & . \\
[a] & . & . & [b] \\
. & . & [b] & . 
\end{pmatrix}
$$

Теперь вычислим $M_1 \otimes M_2$.

\begin{align}
M_3 &= M_1 \otimes M_2 = 
\begin{pmatrix} 
. & [a] & . & . \\
. & . & [S] & [b] \\
. & . & . & [b] \\
. & . & . & . 
\end{pmatrix}
\otimes 
\begin{pmatrix} 
. & [a] & . & . \\
. & . & [a] & . \\
[a] & . & . & [b] \\
. & . & [b] & . 
\end{pmatrix}
=\notag\\
&=
\left(\begin{array}{c c c c | c c c c | c c c c | c c c c } 
. & . & . & .  &  .   & [a] & .   & .  &  . & . & . & .  &  . & . & . & .   \\
. & . & . & .  &  .   & .   & [a] & .  &  . & . & . & .  &  . & . & . & .   \\
. & . & . & .  &  [a] & .   & .   & .  &  . & . & . & .  &  . & . & . & .   \\
. & . & . & .  &  .   & .   & .   & .  &  . & . & . & .  &  . & . & . & .   \\
\hline
. & . & . & .  &  . & . & . & .    &  . & . & . & .  &  . & . & . & .   \\
. & . & . & .  &  . & . & . & .    &  . & . & . & .  &  . & . & . & .   \\
. & . & . & .  &  . & . & . & .    &  . & . & . & .  &  . & . & . & [b] \\
. & . & . & .  &  . & . & . & .    &  . & . & . & .  &  . & . & [b] & . \\
\hline
. & . & . & .  &  . & . & . & .    &  . & . & . & .  &  . & . & . & .   \\
. & . & . & .  &  . & . & . & .    &  . & . & . & .  &  . & . & . & .   \\
. & . & . & .  &  . & . & . & .    &  . & . & . & .  &  . & . & . & [b] \\
. & . & . & .  &  . & . & . & .    &  . & . & . & .  &  . & . & [b] & . \\
\hline
. & . & . & .  &  . & . & . & .    &  . & . & . & .  &  . & . & . & .   \\
. & . & . & .  &  . & . & . & .    &  . & . & . & .  &  . & . & . & .   \\
. & . & . & .  &  . & . & . & .    &  . & . & . & .  &  . & . & . & .   \\
. & . & . & .  &  . & . & . & .    &  . & . & . & .  &  . & . & . & . 
\end{array}\right)
\label{eq:graph_tm}
\end{align}

Побалуемся с некоммутативностью и перестановками.

\subsection{Алгоритм}

Идея алгоритма основана на обобщении пересечения двух конечных автоматов до пересечения рекурсивного автомата, построенного по грамматике, со входным графом.

Пересечение двух конечных автоматов --- тензорное произведение соответствующих графов.
Пересечение языкрв коммутативно, тензорное произведение нет, но это не страшно.

Несколько наблюдений
Путь из нетерминалов и терминалов.
При этом должны быть пути из терминалов. Иначе не задать язык.
Будем насыщать граф рёбрами, помеченными нетерминалами.

По грамматике строим автомат.
r~\ref{eq:graph_tm}

\begin{algorithm}
  \floatname{algorithm}{Listing}
\begin{algorithmic}[1]
\caption{Поиск путей через тензорное произведение}
\label{lst:algo1}
\Function{contextFreePathQueryingTP}{G, $\mathcal{G}$}
    \State{$N \gets$ рекурсивный автомат для $G$}
    \State{$M_1 \gets$ матрица смежности $N$}
    \State{$M_2 \gets$ матрица смежности $\mathcal{G}$}
    \State{$T \gets$ the matrix $n \times n$ in which each element is $\varnothing$}
    \ForAll{$(i,x,j) \in E$}
    \Comment{Matrix initialization}
        \State{$T_{i,j} \gets T_{i,j} \cup \{A~|~(A \rightarrow x) \in P \}$}
    \EndFor
    \While{matrix $T$ is changing}
        \State{$M_3 \gets M_1 \otimes M_2$}
        \Comment{Graphs intersection}
        \State{$M_3 \gets \textit{transitiveClosure}(M_3)$}
        \For{}
           \For{}
             \If{}
                 \State{$Nt \gets M_3[i,j] \cup \{\}$}
                 \State{$m \gets M_3[i,j] \cup \{\}$}
                 \State{$n \gets M_3[i,j] \cup \{\}$}
                 \State{$M_2[n,m] \gets M_2[n,m] \cup \{Nt\}$}
             \EndIf
           \EndFor
        \EndFor
    \EndWhile
\State \Return $T$
\EndFunction
\end{algorithmic}
\end{algorithm}


В цикле: пересекли через тензорное произведение, замкнули через обычное матричное произведение, чтобы найти пути из начальной в конечную в граммтике, поставили в соответствующие ячейки нетерминалы, добавили соответствующие рёбра в исходный граф.


Можно вычислять только разницу.
Для этого, правда, потребуется держать ещё одну матрицу.
И надо проверять, что вычислительно дешевле: поддерживать разницу и потом каждый раз поэлементно складывать две матрицы или каждый раз вычислять полностью произведение.

Всего несколько матриц.
Разреженные.
Необходимо отметить, что для реальных графов и запросов результат тензорного произведения будет очень разрежен.
На готовых либах должно быть быстро.

\subsection{Примеры}

Рассмотрим ряд примеров работы описанного алгоритма.
Разница выделена жирным.

\textbf{Пример 1.}


Худший случай.
Такой же как и для матричного.

\begin{align}
%\begin{split}
tc(M_3) =
\left(\begin{array}{c c c c | c c c c | c c c c | c c c c } 
. & . & . & .  &  . & [a] & . & .  &  . & . & . & .  &  . & . & . & .\\
. & . & . & .  &  . & . & [a] & .  &  . & . & . & .  &  . & . & . & \textbf{[ab]}   \\
. & . & . & .  &  [a] & . & . & .  &  . & . & . & .  &  . & . & . & .   \\
. & . & . & .  &  . & . & . & .    &  . & . & . & .  &  . & . & . & .   \\
\hline
. & . & . & .  &  . & . & . & .    &  . & . & . & .  &  . & . & . & .   \\
. & . & . & .  &  . & . & . & .    &  . & . & . & .  &  . & . & . & .   \\
. & . & . & .  &  . & . & . & .    &  . & . & . & .  &  . & . & . & [b] \\
. & . & . & .  &  . & . & . & .    &  . & . & . & .  &  . & . & [b] & . \\
\hline
. & . & . & .  &  . & . & . & .    &  . & . & . & .  &  . & . & . & .   \\
. & . & . & .  &  . & . & . & .    &  . & . & . & .  &  . & . & . & .   \\
. & . & . & .  &  . & . & . & .    &  . & . & . & .  &  . & . & . & [b] \\
. & . & . & .  &  . & . & . & .    &  . & . & . & .  &  . & . & [b] & . \\
\hline
. & . & . & .  &  . & . & . & .    &  . & . & . & .  &  . & . & . & .   \\
. & . & . & .  &  . & . & . & .    &  . & . & . & .  &  . & . & . & .   \\
. & . & . & .  &  . & . & . & .    &  . & . & . & .  &  . & . & . & .   \\
. & . & . & .  &  . & . & . & .    &  . & . & . & .  &  . & . & . & . 
\end{array}\right)
%\end{split}
\label{eq:graph_tm}
\end{align}

Мы видим, что в результате транзитивного замыкания появилось новое ребро с меткой $ab$ из вершины $(0,1)$ в вершину $(3,3)$.
Так как вершина 0 является стартовой в рекурсивном автоматие, а 3 является финальной, то слово вдоль пути из вершины 1 в вершину 3 во входном графе выводимо из нетерминала $S$.
Это означает, что в графе должно быть добавлено ребро из $0$ в $3$ с меткой $S$, после чего граф будет выглядеть следующим образом:

\begin{center}
\begin{tikzpicture}[shorten >=1pt,on grid,auto] 
   \node[state] (q_0)   {$0$}; 
   \node[state] (q_1) [above right=of q_0] {$1$}; 
   \node[state] (q_2) [right=of q_0] {$2$}; 
   \node[state] (q_3) [right=of q_2] {$3$};
    \path[->] 
    (q_0) edge  node {a} (q_1)          
    (q_1) edge  node {a} (q_2)
    (q_2) edge  node {a} (q_0)
    (q_1) edge[bend left, above]  node {\textbf{S}} (q_3)
    (q_2) edge[bend left, above]  node {b} (q_3)
    (q_3) edge[bend left, below]  node {b} (q_2);
\end{tikzpicture}
\end{center}

Матрица смежности обновлённого графа:

$$ M_2 =
\begin{pmatrix} 
. & [a] & . & . \\
. & . & [a] & \textbf{[S]} \\
[a] & . & . & [b] \\
. & . & [b] & . 
\end{pmatrix}
$$

Итерация закончена. 
Возвращаемся к началу цикла и вновь вычисляем тензорное произведение.

\textbf{Итерация 3.}

\begin{align}
M_3 &= M_1 \otimes M_2 = 
\begin{pmatrix} 
. & [a] & . & . \\
. & . & [S] & [b] \\
. & . & . & [b] \\
. & . & . & . 
\end{pmatrix}
\otimes 
\begin{pmatrix} 
. & [a] & . & . \\
. & . & [a] & [S] \\
[a] & . & . & [b] \\
. & . & [b] & . 
\end{pmatrix}
=\notag\\
&=
\left(\begin{array}{c c c c | c c c c | c c c c | c c c c } 
. & . & . & .  &  . & [a] & . & .  &  . & . & . & .    &  . & . & . & .   \\
. & . & . & .  &  . & . & [a] & .  &  . & . & . & .    &  . & . & . & .   \\
. & . & . & .  &  [a] & . & . & .  &  . & . & . & .    &  . & . & . & .   \\
. & . & . & .  &  . & . & . & .    &  . & . & . & .    &  . & . & . & .   \\
\hline
. & . & . & .  &  . & . & . & .    &  . & . & . & .    &  . & . & . & .   \\
. & . & . & .  &  . & . & . & .    &  . & . & . & \textbf{[S]}  &  . & . & . & .   \\
. & . & . & .  &  . & . & . & .    &  . & . & . & .    &  . & . & . & [b] \\
. & . & . & .  &  . & . & . & .    &  . & . & . & .    &  . & . & [b] & . \\
\hline
. & . & . & .  &  . & . & . & .    &  . & . & . & .    &  . & . & . & .   \\
. & . & . & .  &  . & . & . & .    &  . & . & . & .    &  . & . & . & .   \\
. & . & . & .  &  . & . & . & .    &  . & . & . & .    &  . & . & . & [b] \\
. & . & . & .  &  . & . & . & .    &  . & . & . & .    &  . & . & [b] & . \\
\hline
. & . & . & .  &  . & . & . & .    &  . & . & . & .    &  . & . & . & .   \\
. & . & . & .  &  . & . & . & .    &  . & . & . & .    &  . & . & . & .   \\
. & . & . & .  &  . & . & . & .    &  . & . & . & .    &  . & . & . & .   \\
. & . & . & .  &  . & . & . & .    &  . & . & . & .    &  . & . & . & . 
\end{array}\right)
\label{eq:graph_tm}
\end{align}

Транзитивное замыкание:

\begin{align}
tc(M_3) =
\left(\begin{array}{c c c c | c c c c | c c c c | c c c c } 
. & . & . & .  &  . & [a] & . & .  &  . & . & . & \textbf{[aS]}  &  . & . & \textbf{[aSb]} & .   \\
. & . & . & .  &  . & . & [a] & .  &  . & . & . & .              &  . & . & .              & [ab]   \\
. & . & . & .  &  [a] & . & . & .  &  . & . & . & .              &  . & . & .              & .   \\
. & . & . & .  &  . & . & . & .    &  . & . & . & .              &  . & . & .              & .   \\
\hline
. & . & . & .  &  . & . & . & .    &  . & . & . & .              &  . & . & . & .    \\
. & . & . & .  &  . & . & . & .    &  . & . & . & [S]            &  . & . & \textbf{[Sb]}    & .    \\
. & . & . & .  &  . & . & . & .    &  . & . & . & .              &  . & . & .    & [b]  \\
. & . & . & .  &  . & . & . & .    &  . & . & . & .              &  . & . & [b]  & .    \\
\hline                                                              
. & . & . & .  &  . & . & . & .    &  . & . & . & .              &  . & . & . & .   \\
. & . & . & .  &  . & . & . & .    &  . & . & . & .              &  . & . & . & .   \\
. & . & . & .  &  . & . & . & .    &  . & . & . & .              &  . & . & . & [b] \\
. & . & . & .  &  . & . & . & .    &  . & . & . & .              &  . & . & [b] & . \\
\hline                                                              
. & . & . & .  &  . & . & . & .    &  . & . & . & .              &  . & . & . & .   \\
. & . & . & .  &  . & . & . & .    &  . & . & . & .              &  . & . & . & .   \\
. & . & . & .  &  . & . & . & .    &  . & . & . & .              &  . & . & . & .   \\
. & . & . & .  &  . & . & . & .    &  . & . & . & .              &  . & . & . & . 
\end{array}\right)
\label{eq:graph_tm}
\end{align}

Обновлённый граф:
\begin{center}
\begin{tikzpicture}[shorten >=1pt,on grid,auto] 
   \node[state] (q_0)   {$0$}; 
   \node[state] (q_1) [above right=of q_0] {$1$}; 
   \node[state] (q_2) [right=of q_0] {$2$}; 
   \node[state] (q_3) [right=of q_2] {$3$};
    \path[->] 
    (q_0) edge  node {a} (q_1)          
    (q_1) edge  node {a} (q_2)
    (q_2) edge  node {a} (q_0)
    (q_1) edge[bend left, above]  node {S} (q_3)
    (q_0) edge[bend right, below]  node {\textbf{S}} (q_2)
    (q_2) edge[bend left, above]  node {b} (q_3)
    (q_3) edge[bend left, below]  node {b} (q_2);
\end{tikzpicture}
\end{center}

И матрица смежности:

$$ M_2 =
\begin{pmatrix} 
. & [a] & [S] & . \\
. & . & [a] & [S] \\
[a] & . & . & [b] \\
. & . & [b] & . 
\end{pmatrix}
$$


Следующая итерация основного цикла.

\begin{align}
%\begin{split}
M_3 &= M_1 \otimes M_2 = 
\begin{pmatrix} 
. & [a] & . & . \\
. & . & [S] & [b] \\
. & . & . & [b] \\
. & . & . & . 
\end{pmatrix}
\otimes 
\begin{pmatrix} 
. & [a] & [S] & . \\
. & . & [a] & [S] \\
[a] & . & . & [b] \\
. & . & [b] & . 
\end{pmatrix}
=\notag\\
&=
\left(\begin{array}{c c c c | c c c c | c c c c | c c c c } 
. & . & . & .  &  . & [a] & . & .  &  . & . & . & .    &  . & . & . & .   \\
. & . & . & .  &  . & . & [a] & .  &  . & . & . & .    &  . & . & . & .   \\
. & . & . & .  &  [a] & . & . & .  &  . & . & . & .    &  . & . & . & .   \\
. & . & . & .  &  . & . & . & .    &  . & . & . & .    &  . & . & . & .   \\
\hline
. & . & . & .  &  . & . & . & .    &  . & . & \textbf{[S]} & .    &  . & . & . & .   \\
. & . & . & .  &  . & . & . & .    &  . & . & .   & [S]  &  . & . & . & .   \\
. & . & . & .  &  . & . & . & .    &  . & . & .   & .    &  . & . & . & [b] \\
. & . & . & .  &  . & . & . & .    &  . & . & .   & .    &  . & . & [b] & . \\
\hline
. & . & . & .  &  . & . & . & .    &  . & . & . & .    &  . & . & . & .   \\
. & . & . & .  &  . & . & . & .    &  . & . & . & .    &  . & . & . & .   \\
. & . & . & .  &  . & . & . & .    &  . & . & . & .    &  . & . & . & [b] \\
. & . & . & .  &  . & . & . & .    &  . & . & . & .    &  . & . & [b] & . \\
\hline
. & . & . & .  &  . & . & . & .    &  . & . & . & .    &  . & . & . & .   \\
. & . & . & .  &  . & . & . & .    &  . & . & . & .    &  . & . & . & .   \\
. & . & . & .  &  . & . & . & .    &  . & . & . & .    &  . & . & . & .   \\
. & . & . & .  &  . & . & . & .    &  . & . & . & .    &  . & . & . & . 
\end{array}\right)
\label{eq:graph_tm}
\end{align}

Транзитивное замыкание:

\begin{align}
tc(M_3) =
\left(\begin{array}{c c c c | c c c c | c c c c | c c c c } 
. & . & . & .  &  . & [a] & . & .  &  . & . & . & [aS]           &  . & . & [aSb] & .     \\
. & . & . & .  &  . & . & [a] & .  &  . & . & . & .              &  . & . & .     & [ab]  \\
. & . & . & .  &  [a] & . & . & .  &  . & . & \textbf{[aS]} & .  &  . & . & .     & \textbf{[aSb]} \\
. & . & . & .  &  . & . & . & .    &  . & . & . & .              &  . & . & .     & .     \\
\hline
. & . & . & .  &  . & . & . & .    &  . & . & [S] & .            &  . & . & .    & \textbf{[Sb]}    \\
. & . & . & .  &  . & . & . & .    &  . & . & . & [S]            &  . & . & [Sb] & .    \\
. & . & . & .  &  . & . & . & .    &  . & . & . & .              &  . & . & .    & [b]  \\
. & . & . & .  &  . & . & . & .    &  . & . & . & .              &  . & . & [b]  & .    \\
\hline                                                              
. & . & . & .  &  . & . & . & .    &  . & . & . & .              &  . & . & . & .   \\
. & . & . & .  &  . & . & . & .    &  . & . & . & .              &  . & . & . & .   \\
. & . & . & .  &  . & . & . & .    &  . & . & . & .              &  . & . & . & [b] \\
. & . & . & .  &  . & . & . & .    &  . & . & . & .              &  . & . & [b] & . \\
\hline                                                              
. & . & . & .  &  . & . & . & .    &  . & . & . & .              &  . & . & . & .   \\
. & . & . & .  &  . & . & . & .    &  . & . & . & .              &  . & . & . & .   \\
. & . & . & .  &  . & . & . & .    &  . & . & . & .              &  . & . & . & .   \\
. & . & . & .  &  . & . & . & .    &  . & . & . & .              &  . & . & . & . 
\end{array}\right)
\label{eq:graph_tm}
\end{align}

Обновлённый граф:
\begin{center}
\begin{tikzpicture}[shorten >=1pt,on grid,auto] 
   \node[state] (q_0)   {$0$}; 
   \node[state] (q_1) [above right=of q_0] {$1$}; 
   \node[state] (q_2) [right=of q_0] {$2$}; 
   \node[state] (q_3) [right=of q_2] {$3$};
    \path[->] 
    (q_0) edge  node {a} (q_1)          
    (q_1) edge  node {a} (q_2)
    (q_2) edge  node {a} (q_0)
    (q_1) edge[bend left, above]  node {S} (q_3)
    (q_0) edge[bend right, below]  node {S} (q_2)
    (q_2) edge[bend left, above]  node {b,\textbf{S}} (q_3)
    (q_3) edge[bend left, below]  node {b} (q_2);
\end{tikzpicture}
\end{center}

И матрица смежности:

$$ M_2 =
\begin{pmatrix} 
. & [a] & [S] & . \\
. & . & [a] & [S] \\
[a] & . & . & [b, \textbf{S}] \\
. & . & [b] & . 
\end{pmatrix}
$$

Следующая итерация основного цикла.

\begin{align}
M_3 &= M_1 \otimes M_2 = 
\begin{pmatrix} 
. & [a] & . & . \\
. & . & [S] & [b] \\
. & . & . & [b] \\
. & . & . & . 
\end{pmatrix}
\otimes 
\begin{pmatrix} 
. & [a] & [S] & . \\
. & . & [a] & [S] \\
[a] & . & . & [b,S] \\
. & . & [b] & . 
\end{pmatrix}
=\notag\\
&=
\left(\begin{array}{c c c c | c c c c | c c c c | c c c c } 
. & . & . & .  &  . & [a] & . & .  &  . & . & . & .    &  . & . & . & .   \\
. & . & . & .  &  . & . & [a] & .  &  . & . & . & .    &  . & . & . & .   \\
. & . & . & .  &  [a] & . & . & .  &  . & . & . & .    &  . & . & . & .   \\
. & . & . & .  &  . & . & . & .    &  . & . & . & .    &  . & . & . & .   \\
\hline
. & . & . & .  &  . & . & . & .    &  . & . & [S] & .             &  . & . & . & .   \\
. & . & . & .  &  . & . & . & .    &  . & . & .   & [S]           &  . & . & . & .   \\
. & . & . & .  &  . & . & . & .    &  . & . & .   & \textbf{[S]}  &  . & . & . & [b] \\
. & . & . & .  &  . & . & . & .    &  . & . & .   & .             &  . & . & [b] & . \\
\hline
. & . & . & .  &  . & . & . & .    &  . & . & . & .    &  . & . & . & .   \\
. & . & . & .  &  . & . & . & .    &  . & . & . & .    &  . & . & . & .   \\
. & . & . & .  &  . & . & . & .    &  . & . & . & .    &  . & . & . & [b] \\
. & . & . & .  &  . & . & . & .    &  . & . & . & .    &  . & . & [b] & . \\
\hline
. & . & . & .  &  . & . & . & .    &  . & . & . & .    &  . & . & . & .   \\
. & . & . & .  &  . & . & . & .    &  . & . & . & .    &  . & . & . & .   \\
. & . & . & .  &  . & . & . & .    &  . & . & . & .    &  . & . & . & .   \\
. & . & . & .  &  . & . & . & .    &  . & . & . & .    &  . & . & . & . 
\end{array}\right)
\label{eq:graph_tm}
\end{align}

Транзитивное замыкание:

\begin{align}
tc(M_3) =
\left(\begin{array}{c c c c | c c c c | c c c c | c c c c } 
. & . & . & .  &  . & [a] & . & .  &  . & . & . & [aS]           &  . & . & [aSb]          & .     \\
. & . & . & .  &  . & . & [a] & .  &  . & . & . & \textbf{[aS]}  &  . & . & \textbf{[aSb]} & [ab]  \\
. & . & . & .  &  [a] & . & . & .  &  . & . & [aS] & .           &  . & . & .              & [aSb] \\
. & . & . & .  &  . & . & . & .    &  . & . & . & .              &  . & . & .              & .     \\
\hline
. & . & . & .  &  . & . & . & .    &  . & . & [S] & .            &  . & . & .             & [Sb]    \\
. & . & . & .  &  . & . & . & .    &  . & . & . & [S]            &  . & . & [Sb]          & .    \\
. & . & . & .  &  . & . & . & .    &  . & . & . & [S]            &  . & . & \textbf{[Sb]} & [b]  \\
. & . & . & .  &  . & . & . & .    &  . & . & . & .              &  . & . & [b]           & .    \\
\hline                                                              
. & . & . & .  &  . & . & . & .    &  . & . & . & .              &  . & . & . & .   \\
. & . & . & .  &  . & . & . & .    &  . & . & . & .              &  . & . & . & .   \\
. & . & . & .  &  . & . & . & .    &  . & . & . & .              &  . & . & . & [b] \\
. & . & . & .  &  . & . & . & .    &  . & . & . & .              &  . & . & [b] & . \\
\hline                                                              
. & . & . & .  &  . & . & . & .    &  . & . & . & .              &  . & . & . & .   \\
. & . & . & .  &  . & . & . & .    &  . & . & . & .              &  . & . & . & .   \\
. & . & . & .  &  . & . & . & .    &  . & . & . & .              &  . & . & . & .   \\
. & . & . & .  &  . & . & . & .    &  . & . & . & .              &  . & . & . & . 
\end{array}\right)
\label{eq:graph_tm}
\end{align}

Обновлённый граф:
\begin{center}
\begin{tikzpicture}[shorten >=1pt,on grid,auto] 
   \node[state] (q_0)   {$0$}; 
   \node[state] (q_1) [above right=of q_0] {$1$}; 
   \node[state] (q_2) [right=of q_0] {$2$}; 
   \node[state] (q_3) [right=of q_2] {$3$};
    \path[->] 
    (q_0) edge  node {a} (q_1)          
    (q_1) edge  node {a,\textbf{S}} (q_2)
    (q_2) edge  node {a} (q_0)
    (q_1) edge[bend left, above]  node {S} (q_3)
    (q_0) edge[bend right, below]  node {S} (q_2)
    (q_2) edge[bend left, above]  node {b,S} (q_3)
    (q_3) edge[bend left, below]  node {b} (q_2);
\end{tikzpicture}
\end{center}

И матрица смежности:

$$ M_2 =
\begin{pmatrix} 
. & [a] & [S] & . \\
. & . & [a, \textbf{S}] & [S] \\
[a] & . & . & [b,S] \\
. & . & [b] & . 
\end{pmatrix}
$$

Следующая итерация основного цикла.

\begin{align}
M_3 &= M_1 \otimes M_2 = 
\begin{pmatrix} 
. & [a] & . & . \\
. & . & [S] & [b] \\
. & . & . & [b] \\
. & . & . & . 
\end{pmatrix}
\otimes 
\begin{pmatrix} 
. & [a] & [S] & . \\
. & . & [a,S] & [S] \\
[a] & . & . & [b,S] \\
. & . & [b] & . 
\end{pmatrix}
=\notag\\
&=
\left(\begin{array}{c c c c | c c c c | c c c c | c c c c } 
. & . & . & .  &  . & [a] & . & .  &  . & . & . & .    &  . & . & . & .   \\
. & . & . & .  &  . & . & [a] & .  &  . & . & . & .    &  . & . & . & .   \\
. & . & . & .  &  [a] & . & . & .  &  . & . & . & .    &  . & . & . & .   \\
. & . & . & .  &  . & . & . & .    &  . & . & . & .    &  . & . & . & .   \\
\hline
. & . & . & .  &  . & . & . & .    &  . & . & [S]          & .    &  . & . & . & .   \\
. & . & . & .  &  . & . & . & .    &  . & . & \textbf{[S]} & [S]  &  . & . & . & .   \\
. & . & . & .  &  . & . & . & .    &  . & . & .            & [S]  &  . & . & . & [b] \\
. & . & . & .  &  . & . & . & .    &  . & . & .            & .    &  . & . & [b] & . \\
\hline
. & . & . & .  &  . & . & . & .    &  . & . & . & .    &  . & . & . & .   \\
. & . & . & .  &  . & . & . & .    &  . & . & . & .    &  . & . & . & .   \\
. & . & . & .  &  . & . & . & .    &  . & . & . & .    &  . & . & . & [b] \\
. & . & . & .  &  . & . & . & .    &  . & . & . & .    &  . & . & [b] & . \\
\hline
. & . & . & .  &  . & . & . & .    &  . & . & . & .    &  . & . & . & .   \\
. & . & . & .  &  . & . & . & .    &  . & . & . & .    &  . & . & . & .   \\
. & . & . & .  &  . & . & . & .    &  . & . & . & .    &  . & . & . & .   \\
. & . & . & .  &  . & . & . & .    &  . & . & . & .    &  . & . & . & . 
\end{array}\right)
\label{eq:graph_tm}
\end{align}

Транзитивное замыкание:

\begin{align}
tc(M_3) =
\left(\begin{array}{c c c c | c c c c | c c c c | c c c c } 
. & . & . & .  &  . & [a] & . & .  &  . & . & \textbf[aS] & [aS]  &  . & . & [aSb] & \textbf{[aSb]}  \\
. & . & . & .  &  . & . & [a] & .  &  . & . & .           & [aS]  &  . & . & [aSb] & [ab]          \\
. & . & . & .  &  [a] & . & . & .  &  . & . & [aS]        & .     &  . & . & .     & [aSb]         \\
. & . & . & .  &  . & . & . & .    &  . & . & .           & .     &  . & . & .     & .             \\
\hline
. & . & . & .  &  . & . & . & .    &  . & . & [S] & .             &  . & . & .    & [Sb]    \\
. & . & . & .  &  . & . & . & .    &  . & . & [S] & [S]           &  . & . & [Sb] & \textbf{[Sb]}    \\
. & . & . & .  &  . & . & . & .    &  . & . & .   & [S]           &  . & . & [Sb] & [b]  \\
. & . & . & .  &  . & . & . & .    &  . & . & .   & .             &  . & . & [b]  & .    \\
\hline                                                              
. & . & . & .  &  . & . & . & .    &  . & . & . & .               &  . & . & .    & .   \\
. & . & . & .  &  . & . & . & .    &  . & . & . & .               &  . & . & .    & .   \\
. & . & . & .  &  . & . & . & .    &  . & . & . & .               &  . & . & .    & [b] \\
. & . & . & .  &  . & . & . & .    &  . & . & . & .               &  . & . & [b]  & . \\
\hline                                                              
. & . & . & .  &  . & . & . & .    &  . & . & . & .               &  . & . & . & .   \\
. & . & . & .  &  . & . & . & .    &  . & . & . & .               &  . & . & . & .   \\
. & . & . & .  &  . & . & . & .    &  . & . & . & .               &  . & . & . & .   \\
. & . & . & .  &  . & . & . & .    &  . & . & . & .               &  . & . & . & . 
\end{array}\right)
\label{eq:graph_tm}
\end{align}

Обновлённый граф:
\begin{center}
\begin{tikzpicture}[shorten >=1pt,on grid,auto] 
   \node[state] (q_0)   {$0$}; 
   \node[state] (q_1) [above right=of q_0] {$1$}; 
   \node[state] (q_2) [right=of q_0] {$2$}; 
   \node[state] (q_3) [right=of q_2] {$3$};
    \path[->] 
    (q_0) edge  node {a} (q_1)          
    (q_1) edge  node {a,S} (q_2)
    (q_2) edge[bend right, above]  node {a} (q_0)
    (q_1) edge[bend left, above]  node {S} (q_3)
    (q_0) edge[bend right, above]  node {S} (q_2)
    (q_2) edge[bend left, above]  node {b,S} (q_3)
    (q_0) edge[bend right, below]  node {\textbf{S}} (q_3)
    (q_3) edge[bend left, above]  node {b} (q_2);
\end{tikzpicture}
\end{center}

И матрица смежности:

$$ M_2 =
\begin{pmatrix} 
. & [a] & [S] & \textbf{[S]} \\
. & . & [a, S] & [S] \\
[a] & . & . & [b,S] \\
. & . & [b] & . 
\end{pmatrix}
$$


И наконец последняя содержательная итерация основного цикла.

\begin{align}
M_3 &= M_1 \otimes M_2 = 
\begin{pmatrix} 
. & [a] & . & . \\
. & . & [S] & [b] \\
. & . & . & [b] \\
. & . & . & . 
\end{pmatrix}
\otimes 
\begin{pmatrix} 
. & [a] & [S] & [S] \\
. & . & [a,S] & [S] \\
[a] & . & . & [b,S] \\
. & . & [b] & . 
\end{pmatrix}
=\notag\\
&=
\left(\begin{array}{c c c c | c c c c | c c c c | c c c c } 
. & . & . & .  &  . & [a] & . & .  &  . & . & . & .    &  . & . & . & .   \\
. & . & . & .  &  . & . & [a] & .  &  . & . & . & .    &  . & . & . & .   \\
. & . & . & .  &  [a] & . & . & .  &  . & . & . & .    &  . & . & . & .   \\
. & . & . & .  &  . & . & . & .    &  . & . & . & .    &  . & . & . & .   \\
\hline
. & . & . & .  &  . & . & . & .    &  . & . & [S] & \textbf{[S]}    &  . & . & . & .   \\
. & . & . & .  &  . & . & . & .    &  . & . & [S] & [S]             &  . & . & . & .   \\
. & . & . & .  &  . & . & . & .    &  . & . & .   & [S]             &  . & . & . & [b] \\
. & . & . & .  &  . & . & . & .    &  . & . & .   & .               &  . & . & [b] & . \\
\hline
. & . & . & .  &  . & . & . & .    &  . & . & . & .    &  . & . & . & .   \\
. & . & . & .  &  . & . & . & .    &  . & . & . & .    &  . & . & . & .   \\
. & . & . & .  &  . & . & . & .    &  . & . & . & .    &  . & . & . & [b] \\
. & . & . & .  &  . & . & . & .    &  . & . & . & .    &  . & . & [b] & . \\
\hline
. & . & . & .  &  . & . & . & .    &  . & . & . & .    &  . & . & . & .   \\
. & . & . & .  &  . & . & . & .    &  . & . & . & .    &  . & . & . & .   \\
. & . & . & .  &  . & . & . & .    &  . & . & . & .    &  . & . & . & .   \\
. & . & . & .  &  . & . & . & .    &  . & . & . & .    &  . & . & . & . 
\end{array}\right)
\label{eq:graph_tm}
\end{align}

Транзитивное замыкание:

\begin{align}
tc(M_3) =
\left(\begin{array}{c c c c | c c c c | c c c c | c c c c } 
. & . & . & .  &  . & [a] & . & .  &  . & . & [aS] & [aS]           &  . & . & [aSb]          & [aSb]  \\
. & . & . & .  &  . & . & [a] & .  &  . & . & .    & [aS]           &  . & . & [aSb]          & [ab]          \\
. & . & . & .  &  [a] & . & . & .  &  . & . & [aS] & \textbf{[aS]}  &  . & . & \textbf{[aSb]} & [aSb]         \\
. & . & . & .  &  . & . & . & .    &  . & . & .    & .              &  . & . & .              & .             \\
\hline
. & . & . & .  &  . & . & . & .    &  . & . & [S] & \texttt{[S]}    &  . & . & \textbf{[Sb]}  & [Sb]    \\
. & . & . & .  &  . & . & . & .    &  . & . & [S] & [S]             &  . & . & [Sb] & [Sb]    \\
. & . & . & .  &  . & . & . & .    &  . & . & .   & [S]             &  . & . & [Sb] & [b]  \\
. & . & . & .  &  . & . & . & .    &  . & . & .   & .               &  . & . & [b]  & .    \\
\hline                                                              
. & . & . & .  &  . & . & . & .    &  . & . & . & .               &  . & . & .    & .   \\
. & . & . & .  &  . & . & . & .    &  . & . & . & .               &  . & . & .    & .   \\
. & . & . & .  &  . & . & . & .    &  . & . & . & .               &  . & . & .    & [b] \\
. & . & . & .  &  . & . & . & .    &  . & . & . & .               &  . & . & [b]  & . \\
\hline                                                              
. & . & . & .  &  . & . & . & .    &  . & . & . & .               &  . & . & . & .   \\
. & . & . & .  &  . & . & . & .    &  . & . & . & .               &  . & . & . & .   \\
. & . & . & .  &  . & . & . & .    &  . & . & . & .               &  . & . & . & .   \\
. & . & . & .  &  . & . & . & .    &  . & . & . & .               &  . & . & . & . 
\end{array}\right)
\label{eq:graph_tm}
\end{align}

Обновлённый граф:
\begin{center}
\begin{tikzpicture}[shorten >=1pt,on grid,auto] 
   \node[state] (q_0)   {$0$}; 
   \node[state] (q_1) [above right=of q_0] {$1$}; 
   \node[state] (q_2) [right=of q_0] {$2$}; 
   \node[state] (q_3) [right=of q_2] {$3$};
    \path[->] 
    (q_0) edge  node {a} (q_1)          
    (q_1) edge  node {a,S} (q_2)
    (q_2) edge[bend right, above]  node {a} (q_0)
    (q_2) edge[loop right]  node {\textbf{S}} (q_2)
    (q_1) edge[bend left, above]  node {S} (q_3)
    (q_0) edge[bend right, above]  node {S} (q_2)
    (q_2) edge[bend left, above]  node {b,S} (q_3)
    (q_0) edge[bend right, below]  node {S} (q_3)
    (q_3) edge[bend left, above]  node {b} (q_2);
\end{tikzpicture}
\end{center}

И матрица смежности:

$$ M_2 =
\begin{pmatrix} 
. & [a] & [S] & [S] \\
. & . & [a, S] & [S] \\
[a] & . & \textbf{[S]} & [b,S] \\
. & . & [b] & . 
\end{pmatrix}
$$


Следующая итерация не приведёт к изменению графа.
Читатель может убедиться в этом самостоятельно.
Соответственно алгоритм можно завершать.

\textbf{Пример 2.}

В данном примере мы увидем, как структура грамматики и, соответственно, рекурсивной сети, влияет на процесс вычислений.

Интуитивно понятно, что чем меньше состояний в рекурсивной сети, тем лучше.
То есть желательно получить как можно более компактное описание контекстно-свободного языка.

Для примера возьмём в качестве КС языка язык Дика на одном типе скобок и опишем его двумя различными грамматиками.
Первая граммтика классическая:
$$
G_1 = \langle \{a,\ b\}, \{ S \}, \{S \to a \ S \ b \ S \mid \varepsilon  \} \rangle
$$

Во второй грамматике мы будем использовать конструкции регулярных выражений в правой части правил.
То есть вторая грамматика находитмся в EBNF~\cite{!!!}.
$$
G_2 = \langle \{a, \ b\}, \{S\}, \{S \to (a \ S \ b)^{*}\} \rangle
$$

Построим рекурсивные автоматы $N_1$ и $N_2$ для этих граммтик.

Рекурсивный автомат $N_1$ для грамматики $G_1$:

\begin{center}
\begin{tikzpicture}[node distance=2cm,shorten >=1pt,on grid,auto] 
   \node[state, initial, accepting] (q_0)   {$0$}; 
   \node[state] (q_1) [right=of q_0] {$1$}; 
   \node[state] (q_2) [right=of q_1] {$2$}; 
   \node[state] (q_3) [right=of q_2] {$3$}; 
   \node[state, accepting] (q_4) [right=of q_3] {$4$}; 
    \path[->] 
    (q_0) edge  node {a} (q_1)          
    (q_1) edge  node {S} (q_2)
    (q_2) edge  node {b} (q_3)
    (q_3) edge  node {S} (q_4);
\end{tikzpicture}
\end{center}

Матрица смежности $N_1$:

$$
M_1^1 =
\begin{pmatrix}
. & [a] & .   & .   & .  \\
. & .   & [S] & .   & .  \\
. & .   & .   & [b] & .  \\
. & .   & .   & .   & [S] \\
. & .   & .   & .   & .
\end{pmatrix}
$$


Рекурсивный автомат $N_2$ для грамматики $G_2$:

\begin{center}
\begin{tikzpicture}[node distance=3cm,shorten >=1pt,on grid,auto] 
   \node[state, initial, accepting] (q_0)   {$0$}; 
   \node[state] (q_1) [above right=of q_0] {$1$}; 
   \node[state] (q_2) [right=of q_0] {$2$}; 
    \path[->] 
    (q_0) edge  node {a} (q_1)          
    (q_1) edge  node {S} (q_2)
    (q_2) edge  node {b} (q_0);
\end{tikzpicture}
\end{center}


Матрица смежности $N_2$:

$$
M_1^2 =
\begin{pmatrix}
.   & [a] & .    \\
.   & .   & [S]  \\
[b] & .   & . 
\end{pmatrix}
$$


Первое, очевидное, наблюдение --- количество состояний в $N_2$ меньше, чем в $N_1$.
Это значит, что матрицы будут меньше, считать быстрее.

Для того, чтобы проще было сделать второе, сперва выполним пошагово алгоритм для двух заданных рекурсивных сетей.


Вход возьмём линейный:
\begin{center}
\begin{tikzpicture}[node distance=2cm,shorten >=1pt,on grid,auto] 
   \node[state] (q_0)   {$0$}; 
   \node[state] (q_1) [right=of q_0] {$1$}; 
   \node[state] (q_2) [right=of q_1] {$2$}; 
   \node[state] (q_3) [right=of q_2] {$3$}; 
   \node[state] (q_4) [right=of q_3] {$4$}; 
   \node[state] (q_5) [right=of q_4] {$5$}; 
   \node[state] (q_6) [right=of q_5] {$6$}; 
    \path[->] 
    (q_0) edge  node {a} (q_1)          
    (q_1) edge  node {b} (q_2)
    (q_2) edge  node {a} (q_3)
    (q_3) edge  node {b} (q_4)          
    (q_4) edge  node {a} (q_5)
    (q_5) edge  node {b} (q_6);
\end{tikzpicture}
\end{center}


Сразу дополним матрицу смежности нетерминалами, выводящими пустую строку, и получим следующую матрицу:

$$
M_2 =
\begin{pmatrix}
[S] & [a] & .   & .   & .   & .   & .   \\
.   & [S] & [b] & .   & .   & .   & .   \\
.   & .   & [S] & [a] & .   & .   & .   \\
.   & .   & .   & [S] & [b] & .   & .   \\
.   & .   & .   & .   & [S] & [a] & .   \\
.   & .   & .   & .   & .   & [S] & [b] \\
.   & .   & .   & .   & .   & .   & [S] 
\end{pmatrix}
$$

Сперва запустим алгоритм на входе и $N_2$. 
Первый шаг первой итерации --- вычисление тензорного произведения $M_1^2 \otimes M_2$.

\begin{align}
M_3 &= M_1^2 \otimes M_2 = 
\begin{pmatrix}
.   & [a] & .    \\
.   & .   & [S]  \\
[b] & .   & . 
\end{pmatrix}
\otimes 
\begin{pmatrix}
[S] & [a] & .   & .   & .   & .   & .   \\
.   & [S] & [b] & .   & .   & .   & .   \\
.   & .   & [S] & [a] & .   & .   & .   \\
.   & .   & .   & [S] & [b] & .   & .   \\
.   & .   & .   & .   & [S] & [a] & .   \\
.   & .   & .   & .   & .   & [S] & [b] \\
.   & .   & .   & .   & .   & .   & [S] 
\end{pmatrix}
=\notag\\
&=
\left(\begin{array}{c c c c c c c | c c c c c c c | c c c c c c c } 
. & . & . & . & . & . & .  &  . & [a] & . & .   & . & .   & .  &  . & . & . & . & . & . & . \\
. & . & . & . & . & . & .  &  . & .   & . & .   & . & .   & .  &  . & . & . & . & . & . & . \\
. & . & . & . & . & . & .  &  . & .   & . & [a] & . & .   & .  &  . & . & . & . & . & . & . \\
. & . & . & . & . & . & .  &  . & .   & . & .   & . & .   & .  &  . & . & . & . & . & . & . \\
. & . & . & . & . & . & .  &  . & .   & . & .   & . & [a] & .  &  . & . & . & . & . & . & . \\
. & . & . & . & . & . & .  &  . & .   & . & .   & . & .   & .  &  . & . & . & . & . & . & . \\
. & . & . & . & . & . & .  &  . & .   & . & .   & . & .   & .  &  . & . & . & . & . & . & . \\
\hline
. & . & . & . & . & . & .  &  . & . & . & . & . & . & .  &  [S] & . & . & . & . & . & . \\
. & . & . & . & . & . & .  &  . & . & . & . & . & . & .  &  . & [S] & . & . & . & . & . \\
. & . & . & . & . & . & .  &  . & . & . & . & . & . & .  &  . & . & [S] & . & . & . & . \\
. & . & . & . & . & . & .  &  . & . & . & . & . & . & .  &  . & . & . & [S] & . & . & . \\
. & . & . & . & . & . & .  &  . & . & . & . & . & . & .  &  . & . & . & . & [S] & . & . \\
. & . & . & . & . & . & .  &  . & . & . & . & . & . & .  &  . & . & . & . & . & [S] & . \\
. & . & . & . & . & . & .  &  . & . & . & . & . & . & .  &  . & . & . & . & . & . & [S] \\
\hline
. & . & .   & . & .   & . & .    &  . & . & . & . & . & . & .  &  . & . & . & . & . & . & . \\
. & . & [b] & . & .   & . & .    &  . & . & . & . & . & . & .  &  . & . & . & . & . & . & . \\
. & . & .   & . & .   & . & .    &  . & . & . & . & . & . & .  &  . & . & . & . & . & . & . \\
. & . & .   & . & [b] & . & .    &  . & . & . & . & . & . & .  &  . & . & . & . & . & . & . \\
. & . & .   & . & .   & . & .    &  . & . & . & . & . & . & .  &  . & . & . & . & . & . & . \\
. & . & .   & . & .   & . & [b]  &  . & . & . & . & . & . & .  &  . & . & . & . & . & . & . \\
. & . & .   & . & .   & . & .    &  . & . & . & . & . & . & .  &  . & . & . & . & . & . & . 
\end{array}\right)
\end{align}

\newcommand{\tinybf}[1]{\textbf{\tiny{[#1]}}}
\newcommand{\tntm}[1]{\text{\tiny{#1}}}

Опустим промежуточные шаги вычисления транзитивного замыкания $M_3$ и сразу представим конечный результат:
\begingroup
\setlength\arraycolsep{2pt}
\begin{align}
&tc(M_3)=\notag\\
&
\left(\begin{array}{c c c c c c c | c c c c c c c | c c c c c c c } 
. & . & \tinybf{aSb} & . & \tinybf{aSbaSb} & . & \tinybf{aSbaSbaSb}           &         . & [a] & . & \tinybf{aSba} & . & \tinybf{aSbaSba} & .         &           .   & \tinybf{aS} & .   & \tinybf{aSbaS} & .   & \tinybf{aSbaSbaS} & . \\
. & . & .            & . & .               & . & .                            &         . & .   & . & .             & . & .                & .         &           .   & .           & .   & .              & .   & .                 & . \\
. & . & .            & . & \tinybf{aSb}    & . & \tinybf{aSbaSb}              &         . & .   & . & [a]           & . & \tinybf{aSba}    & .         &           .   & .           & .   & \tinybf{aS}    & .   & \tinybf{aSbaS}    & . \\
. & . & .            & . & .               & . & .                            &         . & .   & . & .             & . & .                & .         &           .   & .           & .   & .              & .   & .                 & . \\
. & . & .            & . & .               & . & \tinybf{aSb}                 &         . & .   & . & .             & . & [a]              & .         &           .   & .           & .   & .              & .   & \tinybf{aS}       & . \\
. & . & .            & . & .               & . & .                            &         . & .   & . & .             & . & .                & .         &           .   & .           & .   & .              & .   & .                 & . \\
. & . & .            & . & .               & . & .                            &         . & .   & . & .             & . & .                & .         &           .   & .           & .   & .              & .   & .                 & . \\
\hline                                                                                              
. & . & .            & . & .               & . & .                            &         . & .   & . & .             & . & .                & .         &           [S] & .           & .   & .              & .   & .                 & . \\
. & . & .            & . & .               & . & .                            &         . & .   & . & .             & . & .                & .         &           .   & [S]         & .   & .              & .   & .                 & . \\
. & . & .            & . & .               & . & .                            &         . & .   & . & .             & . & .                & .         &           .   & .           & [S] & .              & .   & .                 & . \\
. & . & .            & . & .               & . & .                            &         . & .   & . & .             & . & .                & .         &           .   & .           & .   & [S]            & .   & .                 & . \\
. & . & .            & . & .               & . & .                            &         . & .   & . & .             & . & .                & .         &           .   & .           & .   & .              & [S] & .                 & . \\
. & . & .            & . & .               & . & .                            &         . & .   & . & .             & . & .                & .         &           .   & .           & .   & .              & .   & [S]               & . \\
. & . & .            & . & .               & . & .                            &         . & .   & . & .             & . & .                & .         &           .   & .           & .   & .              & .   & .                 & [S] \\
\hline                                                                                              
. & . & .            & . & .               & . & .                            &         . & .   & . & .             & . & .                & .         &           .   & .           & .   & .              & .   & .                 & . \\
. & . & [b]          & . & .               & . & .                            &         . & .   & . & .             & . & .                & .         &           .   & .           & .   & .              & .   & .                 & . \\
. & . & .            & . & .               & . & .                            &         . & .   & . & .             & . & .                & .         &           .   & .           & .   & .              & .   & .                 & . \\
. & . & .            & . & [b]             & . & .                            &         . & .   & . & .             & . & .                & .         &           .   & .           & .   & .              & .   & .                 & . \\
. & . & .            & . & .               & . & .                            &         . & .   & . & .             & . & .                & .         &           .   & .           & .   & .              & .   & .                 & . \\
. & . & .            & . & .               & . & [b]                          &         . & .   & . & .             & . & .                & .         &           .   & .           & .   & .              & .   & .                 & . \\
. & . & .            & . & .               & . & .                            &         . & .   & . & .             & . & .                & .         &           .   & .           & .   & .              & .   & .                 & . 
\end{array}\right)
\end{align}
\endgroup


Результирующий граф с новыми рёбрами:
\begin{center}
\begin{tikzpicture}[node distance=2cm,shorten >=1pt,on grid,auto] 
   \node[state] (q_0)   {$0$}; 
   \node[state] (q_1) [right=of q_0] {$1$}; 
   \node[state] (q_2) [right=of q_1] {$2$}; 
   \node[state] (q_3) [right=of q_2] {$3$}; 
   \node[state] (q_4) [right=of q_3] {$4$}; 
   \node[state] (q_5) [right=of q_4] {$5$}; 
   \node[state] (q_6) [right=of q_5] {$6$}; 
    \path[->] 
    (q_0) edge  node {a} (q_1)
    (q_0) edge[bend left, above]  node {\textbf{S}} (q_2)
    (q_0) edge[bend right, above]  node {\textbf{S}} (q_4)
    (q_0) edge[bend right, above]  node {\textbf{S}} (q_6)
    (q_1) edge  node {b} (q_2)
    (q_2) edge  node {a} (q_3)
    (q_2) edge[bend left, above]  node {\textbf{S}} (q_4)
    (q_2) edge[bend right, above]  node {\textbf{S}} (q_6)
    (q_3) edge  node {b} (q_4)          
    (q_4) edge  node {a} (q_5)
    (q_4) edge[bend left, above]  node {\textbf{S}} (q_6)
    (q_5) edge  node {b} (q_6);
\end{tikzpicture}
\end{center}


Его матрица смежности:

$$
M_2 =
\begin{pmatrix}
[S] & [a] & \textbf{[S]} & .   & \textbf{[S]} & .   & \textbf{[S]} \\
.   & [S] & [b]          & .   & .            & .   & .            \\
.   & .   & [S]          & [a] & \textbf{[S]} & .   & \textbf{[S]} \\
.   & .   & .            & [S] & [b]          & .   & .            \\
.   & .   & .            & .   & [S]          & [a] & \textbf{[S]} \\
.   & .   & .            & .   & .            & [S] & [b]          \\
.   & .   & .            & .   & .            & .   & [S] 
\end{pmatrix}
$$

Таким образом видно, что для выбранных входных данных алгоритму достаточно двух итераций основного цикла (вторая итерация будет неполной, достаточно проверить, что результат тензорного произведения не изменился).
Читателю предлагается  выяснить, сколько умножений потребуется, стобы вычислить транзитивное замыкание на первой итерации.


Теперь запучтим алгоритм на второй грамматике.
\begingroup
\setlength\arraycolsep{2pt}
\begin{align}
&M_3 = M_1^1 \otimes M_2 = 
\begin{pmatrix}
. & [a] & .   & .   & .  \\
. & .   & [S] & .   & .  \\
. & .   & .   & [b] & .  \\
. & .   & .   & .   & [S] \\
. & .   & .   & .   & .
\end{pmatrix}
\otimes 
\begin{pmatrix}
[S] & [a] & .   & .   & .   & .   & .   \\
.   & [S] & [b] & .   & .   & .   & .   \\
.   & .   & [S] & [a] & .   & .   & .   \\
.   & .   & .   & [S] & [b] & .   & .   \\
.   & .   & .   & .   & [S] & [a] & .   \\
.   & .   & .   & .   & .   & [S] & [b] \\
.   & .   & .   & .   & .   & .   & [S] 
\end{pmatrix}
=\notag\\
&=
\left(\begin{array}{c c c c c c c | c c c c c c c | c c c c c c c | c c c c c c c | c c c c c c c} 
. & . & . & . & . & . & .  &  . & [a] & . & .   & . & .   & .  &  . & . & . & . & . & . & .  &  . & . & . & . & . & . & .  &  . & . & . & . & . & . & .   \\
. & . & . & . & . & . & .  &  . & .   & . & .   & . & .   & .  &  . & . & . & . & . & . & .  &  . & . & . & . & . & . & .  &  . & . & . & . & . & . & .   \\
. & . & . & . & . & . & .  &  . & .   & . & [a] & . & .   & .  &  . & . & . & . & . & . & .  &  . & . & . & . & . & . & .  &  . & . & . & . & . & . & .   \\
. & . & . & . & . & . & .  &  . & .   & . & .   & . & .   & .  &  . & . & . & . & . & . & .  &  . & . & . & . & . & . & .  &  . & . & . & . & . & . & .   \\
. & . & . & . & . & . & .  &  . & .   & . & .   & . & [a] & .  &  . & . & . & . & . & . & .  &  . & . & . & . & . & . & .  &  . & . & . & . & . & . & .   \\
. & . & . & . & . & . & .  &  . & .   & . & .   & . & .   & .  &  . & . & . & . & . & . & .  &  . & . & . & . & . & . & .  &  . & . & . & . & . & . & .   \\
. & . & . & . & . & . & .  &  . & .   & . & .   & . & .   & .  &  . & . & . & . & . & . & .  &  . & . & . & . & . & . & .  &  . & . & . & . & . & . & .   \\
\hline
. & . & . & . & . & . & .  &  . & . & . & . & . & . & .  &  [S] & . & . & . & . & . & .  &  . & . & . & . & . & . & .  &  . & . & . & . & . & . & .   \\
. & . & . & . & . & . & .  &  . & . & . & . & . & . & .  &  . & [S] & . & . & . & . & .  &  . & . & . & . & . & . & .  &  . & . & . & . & . & . & .   \\
. & . & . & . & . & . & .  &  . & . & . & . & . & . & .  &  . & . & [S] & . & . & . & .  &  . & . & . & . & . & . & .  &  . & . & . & . & . & . & .   \\
. & . & . & . & . & . & .  &  . & . & . & . & . & . & .  &  . & . & . & [S] & . & . & .  &  . & . & . & . & . & . & .  &  . & . & . & . & . & . & .   \\
. & . & . & . & . & . & .  &  . & . & . & . & . & . & .  &  . & . & . & . & [S] & . & .  &  . & . & . & . & . & . & .  &  . & . & . & . & . & . & .   \\
. & . & . & . & . & . & .  &  . & . & . & . & . & . & .  &  . & . & . & . & . & [S] & .  &  . & . & . & . & . & . & .  &  . & . & . & . & . & . & .   \\
. & . & . & . & . & . & .  &  . & . & . & . & . & . & .  &  . & . & . & . & . & . & [S]  &  . & . & . & . & . & . & .  &  . & . & . & . & . & . & .   \\
\hline
. & . & . & . & . & . & .  &  . & . & . & . & . & . & .  &  . & . & . & . & . & . & .  &  . & . & .   & . & .   & . & .    &  . & . & . & . & . & . & .   \\
. & . & . & . & . & . & .  &  . & . & . & . & . & . & .  &  . & . & . & . & . & . & .  &  . & . & [b] & . & .   & . & .    &  . & . & . & . & . & . & .   \\
. & . & . & . & . & . & .  &  . & . & . & . & . & . & .  &  . & . & . & . & . & . & .  &  . & . & .   & . & .   & . & .    &  . & . & . & . & . & . & .   \\
. & . & . & . & . & . & .  &  . & . & . & . & . & . & .  &  . & . & . & . & . & . & .  &  . & . & .   & . & [b] & . & .    &  . & . & . & . & . & . & .   \\
. & . & . & . & . & . & .  &  . & . & . & . & . & . & .  &  . & . & . & . & . & . & .  &  . & . & .   & . & .   & . & .    &  . & . & . & . & . & . & .   \\
. & . & . & . & . & . & .  &  . & . & . & . & . & . & .  &  . & . & . & . & . & . & .  &  . & . & .   & . & .   & . & [b]  &  . & . & . & . & . & . & .   \\
. & . & . & . & . & . & .  &  . & . & . & . & . & . & .  &  . & . & . & . & . & . & .  &  . & . & .   & . & .   & . & .    &  . & . & . & . & . & . & .   \\
\hline
. & . & . & . & . & . & .  &  . & . & . & . & . & . & .  &  . & . & . & . & . & . & .  &  . & . & . & . & . & . & .  &  [S] & . & . & . & . & . & .   \\
. & . & . & . & . & . & .  &  . & . & . & . & . & . & .  &  . & . & . & . & . & . & .  &  . & . & . & . & . & . & .  &  . & [S] & . & . & . & . & .   \\
. & . & . & . & . & . & .  &  . & . & . & . & . & . & .  &  . & . & . & . & . & . & .  &  . & . & . & . & . & . & .  &  . & . & [S] & . & . & . & .   \\
. & . & . & . & . & . & .  &  . & . & . & . & . & . & .  &  . & . & . & . & . & . & .  &  . & . & . & . & . & . & .  &  . & . & . & [S] & . & . & .   \\
. & . & . & . & . & . & .  &  . & . & . & . & . & . & .  &  . & . & . & . & . & . & .  &  . & . & . & . & . & . & .  &  . & . & . & . & [S] & . & .   \\
. & . & . & . & . & . & .  &  . & . & . & . & . & . & .  &  . & . & . & . & . & . & .  &  . & . & . & . & . & . & .  &  . & . & . & . & . & [S] & .   \\
. & . & . & . & . & . & .  &  . & . & . & . & . & . & .  &  . & . & . & . & . & . & .  &  . & . & . & . & . & . & .  &  . & . & . & . & . & . & [S]   \\
\hline
. & . & . & . & . & . & .  &  . & . & . & . & . & . & .  &  . & . & . & . & . & . & .  &  . & . & . & . & . & . & .  &  . & . & . & . & . & . & .   \\
. & . & . & . & . & . & .  &  . & . & . & . & . & . & .  &  . & . & . & . & . & . & .  &  . & . & . & . & . & . & .  &  . & . & . & . & . & . & .   \\
. & . & . & . & . & . & .  &  . & . & . & . & . & . & .  &  . & . & . & . & . & . & .  &  . & . & . & . & . & . & .  &  . & . & . & . & . & . & .   \\
. & . & . & . & . & . & .  &  . & . & . & . & . & . & .  &  . & . & . & . & . & . & .  &  . & . & . & . & . & . & .  &  . & . & . & . & . & . & .   \\
. & . & . & . & . & . & .  &  . & . & . & . & . & . & .  &  . & . & . & . & . & . & .  &  . & . & . & . & . & . & .  &  . & . & . & . & . & . & .   \\
. & . & . & . & . & . & .  &  . & . & . & . & . & . & .  &  . & . & . & . & . & . & .  &  . & . & . & . & . & . & .  &  . & . & . & . & . & . & .   \\
. & . & . & . & . & . & .  &  . & . & . & . & . & . & .  &  . & . & . & . & . & . & .  &  . & . & . & . & . & . & .  &  . & . & . & . & . & . & .   
\end{array}\right)
\end{align}
\endgroup

Уже сейчас можно заметить, что размер матриц, с которыми нам придётся работать, существенно увеличелся, по сравнению с предыдущим вариантом.
Это, конечно, закономерно, ведт в рекурсивной сети для предыдущего варианта меньше состояний, а значит и матрица смежности имеет меньший размер.

Транзитивное замыкание.

\begingroup
\setlength\arraycolsep{1pt}
\begin{align}
&tc(M_3)=\notag\\
&=
\left(\begin{array}{c c c c c c c | c c c c c c c | c c c c c c c | c c c c c c c | c c c c c c c} 
. & . & . & . & . & . & .   &   . & [a] & . & .   & . & .   & .   &   . & \tinybf{aS} & . & .           & . & .           & .  &  . & . & \tinybf{aSb} & . & .            & . & .             &  . & . & \tinybf{aSbS} & . & .             & . & .   \\
. & . & . & . & . & . & .   &   . & .   & . & .   & . & .   & .   &   . & .           & . & .           & . & .           & .  &  . & . & .            & . & .            & . & .             &  . & . & .             & . & .             & . & .   \\
. & . & . & . & . & . & .   &   . & .   & . & [a] & . & .   & .   &   . & .           & . & \tinybf{aS} & . & .           & .  &  . & . & .            & . & \tinybf{aSb} & . & .             &  . & . & .             & . & \tinybf{aSbS} & . & .   \\
. & . & . & . & . & . & .   &   . & .   & . & .   & . & .   & .   &   . & .           & . & .           & . & .           & .  &  . & . & .            & . & .            & . & .             &  . & . & .             & . & .             & . & .   \\
. & . & . & . & . & . & .   &   . & .   & . & .   & . & [a] & .   &   . & .           & . & .           & . & \tinybf{aS} & .  &  . & . & .            & . & .            & . & \tinybf{aSb}  &  . & . & .             & . & .             & . & \tinybf{aSbS}   \\
. & . & . & . & . & . & .   &   . & .   & . & .   & . & .   & .   &   . & .           & . & .           & . & .           & .  &  . & . & .            & . & .            & . & .             &  . & . & .             & . & .             & . & .   \\
. & . & . & . & . & . & .   &   . & .   & . & .   & . & .   & .   &   . & .           & . & .           & . & .           & .  &  . & . & .            & . & .            & . & .             &  . & . & .             & . & .             & . & .   \\
\hline                                                                                
. & . & . & . & . & . & .   &   . & . & . & . & . & . & .   &   [S] & .   & .   & .   & .   & .   & .    &  . & . & . & . & . & . & .  &  . & . & . & . & . & . & .   \\
. & . & . & . & . & . & .   &   . & . & . & . & . & . & .   &   .   & [S] & .   & .   & .   & .   & .    &  . & . & . & . & . & . & .  &  . & . & . & . & . & . & .   \\
. & . & . & . & . & . & .   &   . & . & . & . & . & . & .   &   .   & .   & [S] & .   & .   & .   & .    &  . & . & . & . & . & . & .  &  . & . & . & . & . & . & .   \\
. & . & . & . & . & . & .   &   . & . & . & . & . & . & .   &   .   & .   & .   & [S] & .   & .   & .    &  . & . & . & . & . & . & .  &  . & . & . & . & . & . & .   \\
. & . & . & . & . & . & .   &   . & . & . & . & . & . & .   &   .   & .   & .   & .   & [S] & .   & .    &  . & . & . & . & . & . & .  &  . & . & . & . & . & . & .   \\
. & . & . & . & . & . & .   &   . & . & . & . & . & . & .   &   .   & .   & .   & .   & .   & [S] & .    &  . & . & . & . & . & . & .  &  . & . & . & . & . & . & .   \\
. & . & . & . & . & . & .   &   . & . & . & . & . & . & .   &   .   & .   & .   & .   & .   & .   & [S]  &  . & . & . & . & . & . & .  &  . & . & . & . & . & . & .   \\
\hline                                                                                
. & . & . & . & . & . & .  &  . & . & . & . & . & . & .  &  . & . & . & . & . & . & .  &  . & . & .   & . & .   & . & .    &  . & . & . & . & . & . & .   \\
. & . & . & . & . & . & .  &  . & . & . & . & . & . & .  &  . & . & . & . & . & . & .  &  . & . & [b] & . & .   & . & .    &  . & . & . & . & . & . & .   \\
. & . & . & . & . & . & .  &  . & . & . & . & . & . & .  &  . & . & . & . & . & . & .  &  . & . & .   & . & .   & . & .    &  . & . & . & . & . & . & .   \\
. & . & . & . & . & . & .  &  . & . & . & . & . & . & .  &  . & . & . & . & . & . & .  &  . & . & .   & . & [b] & . & .    &  . & . & . & . & . & . & .   \\
. & . & . & . & . & . & .  &  . & . & . & . & . & . & .  &  . & . & . & . & . & . & .  &  . & . & .   & . & .   & . & .    &  . & . & . & . & . & . & .   \\
. & . & . & . & . & . & .  &  . & . & . & . & . & . & .  &  . & . & . & . & . & . & .  &  . & . & .   & . & .   & . & [b]  &  . & . & . & . & . & . & .   \\
. & . & . & . & . & . & .  &  . & . & . & . & . & . & .  &  . & . & . & . & . & . & .  &  . & . & .   & . & .   & . & .    &  . & . & . & . & . & . & .   \\
\hline
. & . & . & . & . & . & .  &  . & . & . & . & . & . & .  &  . & . & . & . & . & . & .  &  . & . & . & . & . & . & .  &  [S] & . & . & . & . & . & .   \\
. & . & . & . & . & . & .  &  . & . & . & . & . & . & .  &  . & . & . & . & . & . & .  &  . & . & . & . & . & . & .  &  . & [S] & . & . & . & . & .   \\
. & . & . & . & . & . & .  &  . & . & . & . & . & . & .  &  . & . & . & . & . & . & .  &  . & . & . & . & . & . & .  &  . & . & [S] & . & . & . & .   \\
. & . & . & . & . & . & .  &  . & . & . & . & . & . & .  &  . & . & . & . & . & . & .  &  . & . & . & . & . & . & .  &  . & . & . & [S] & . & . & .   \\
. & . & . & . & . & . & .  &  . & . & . & . & . & . & .  &  . & . & . & . & . & . & .  &  . & . & . & . & . & . & .  &  . & . & . & . & [S] & . & .   \\
. & . & . & . & . & . & .  &  . & . & . & . & . & . & .  &  . & . & . & . & . & . & .  &  . & . & . & . & . & . & .  &  . & . & . & . & . & [S] & .   \\
. & . & . & . & . & . & .  &  . & . & . & . & . & . & .  &  . & . & . & . & . & . & .  &  . & . & . & . & . & . & .  &  . & . & . & . & . & . & [S]   \\
\hline
. & . & . & . & . & . & .  &  . & . & . & . & . & . & .  &  . & . & . & . & . & . & .  &  . & . & . & . & . & . & .  &  . & . & . & . & . & . & .   \\
. & . & . & . & . & . & .  &  . & . & . & . & . & . & .  &  . & . & . & . & . & . & .  &  . & . & . & . & . & . & .  &  . & . & . & . & . & . & .   \\
. & . & . & . & . & . & .  &  . & . & . & . & . & . & .  &  . & . & . & . & . & . & .  &  . & . & . & . & . & . & .  &  . & . & . & . & . & . & .   \\
. & . & . & . & . & . & .  &  . & . & . & . & . & . & .  &  . & . & . & . & . & . & .  &  . & . & . & . & . & . & .  &  . & . & . & . & . & . & .   \\
. & . & . & . & . & . & .  &  . & . & . & . & . & . & .  &  . & . & . & . & . & . & .  &  . & . & . & . & . & . & .  &  . & . & . & . & . & . & .   \\
. & . & . & . & . & . & .  &  . & . & . & . & . & . & .  &  . & . & . & . & . & . & .  &  . & . & . & . & . & . & .  &  . & . & . & . & . & . & .   \\
. & . & . & . & . & . & .  &  . & . & . & . & . & . & .  &  . & . & . & . & . & . & .  &  . & . & . & . & . & . & .  &  . & . & . & . & . & . & .   
\end{array}\right)
\end{align}
\endgroup

Обновлённый граф
\begin{center}
\begin{tikzpicture}[node distance=2cm,shorten >=1pt,on grid,auto] 
   \node[state] (q_0)   {$0$}; 
   \node[state] (q_1) [right=of q_0] {$1$}; 
   \node[state] (q_2) [right=of q_1] {$2$}; 
   \node[state] (q_3) [right=of q_2] {$3$}; 
   \node[state] (q_4) [right=of q_3] {$4$}; 
   \node[state] (q_5) [right=of q_4] {$5$}; 
   \node[state] (q_6) [right=of q_5] {$6$}; 
    \path[->] 
    (q_0) edge  node {a} (q_1)
    (q_0) edge[bend left, above]  node {\textbf{S}} (q_2)
    (q_1) edge  node {b} (q_2)
    (q_2) edge  node {a} (q_3)
    (q_2) edge[bend left, above]  node {\textbf{S}} (q_4)
    (q_3) edge  node {b} (q_4)          
    (q_4) edge  node {a} (q_5)
    (q_4) edge[bend left, above]  node {\textbf{S}} (q_6)
    (q_5) edge  node {b} (q_6);
\end{tikzpicture}
\end{center}


Его матрица смежности:

$$
M_2 =
\begin{pmatrix}
[S] & [a] & \textbf{[S]} & .   & .            & .   & .            \\
.   & [S] & [b]          & .   & .            & .   & .            \\
.   & .   & [S]          & [a] & \textbf{[S]} & .   & .            \\
.   & .   & .            & [S] & [b]          & .   & .            \\
.   & .   & .            & .   & [S]          & [a] & \textbf{[S]} \\
.   & .   & .            & .   & .            & [S] & [b]          \\
.   & .   & .            & .   & .            & .   & [S] 
\end{pmatrix}
$$

Потребуется ещё одна итерация.

\begingroup
\setlength\arraycolsep{2pt}
\begin{align}
&M_3 = M_1^1 \otimes M_2 = 
\begin{pmatrix}
. & [a] & .   & .   & .  \\
. & .   & [S] & .   & .  \\
. & .   & .   & [b] & .  \\
. & .   & .   & .   & [S] \\
. & .   & .   & .   & .
\end{pmatrix}
\otimes 
\begin{pmatrix}
[S] & [a] & [S] & .   & .   & .   & .   \\
.   & [S] & [b] & .   & .   & .   & .   \\
.   & .   & [S] & [a] & [S] & .   & .   \\
.   & .   & .   & [S] & [b] & .   & .   \\
.   & .   & .   & .   & [S] & [a] & [S] \\
.   & .   & .   & .   & .   & [S] & [b] \\
.   & .   & .   & .   & .   & .   & [S] 
\end{pmatrix}
=\notag\\
&=
\left(\begin{array}{c c c c c c c | c c c c c c c | c c c c c c c | c c c c c c c | c c c c c c c} 
. & . & . & . & . & . & .  &  . & [a] & . & .   & . & .   & .  &  . & . & . & . & . & . & .  &  . & . & . & . & . & . & .  &  . & . & . & . & . & . & .   \\
. & . & . & . & . & . & .  &  . & .   & . & .   & . & .   & .  &  . & . & . & . & . & . & .  &  . & . & . & . & . & . & .  &  . & . & . & . & . & . & .   \\
. & . & . & . & . & . & .  &  . & .   & . & [a] & . & .   & .  &  . & . & . & . & . & . & .  &  . & . & . & . & . & . & .  &  . & . & . & . & . & . & .   \\
. & . & . & . & . & . & .  &  . & .   & . & .   & . & .   & .  &  . & . & . & . & . & . & .  &  . & . & . & . & . & . & .  &  . & . & . & . & . & . & .   \\
. & . & . & . & . & . & .  &  . & .   & . & .   & . & [a] & .  &  . & . & . & . & . & . & .  &  . & . & . & . & . & . & .  &  . & . & . & . & . & . & .   \\
. & . & . & . & . & . & .  &  . & .   & . & .   & . & .   & .  &  . & . & . & . & . & . & .  &  . & . & . & . & . & . & .  &  . & . & . & . & . & . & .   \\
. & . & . & . & . & . & .  &  . & .   & . & .   & . & .   & .  &  . & . & . & . & . & . & .  &  . & . & . & . & . & . & .  &  . & . & . & . & . & . & .   \\
\hline
. & . & . & . & . & . & .  &  . & . & . & . & . & . & .  &  [S] & .   & \textbf{[S]} & .   & .            & .   & .             &  . & . & . & . & . & . & .  &  . & . & . & . & . & . & .   \\
. & . & . & . & . & . & .  &  . & . & . & . & . & . & .  &  .   & [S] & .            & .   & .            & .   & .             &  . & . & . & . & . & . & .  &  . & . & . & . & . & . & .   \\
. & . & . & . & . & . & .  &  . & . & . & . & . & . & .  &  .   & .   & [S]          & .   & \textbf{[S]} & .   & .             &  . & . & . & . & . & . & .  &  . & . & . & . & . & . & .   \\
. & . & . & . & . & . & .  &  . & . & . & . & . & . & .  &  .   & .   & .            & [S] & .            & .   & .             &  . & . & . & . & . & . & .  &  . & . & . & . & . & . & .   \\
. & . & . & . & . & . & .  &  . & . & . & . & . & . & .  &  .   & .   & .            & .   & [S]          & .   & \textbf{[S]}  &  . & . & . & . & . & . & .  &  . & . & . & . & . & . & .   \\
. & . & . & . & . & . & .  &  . & . & . & . & . & . & .  &  .   & .   & .            & .   & .            & [S] & .             &  . & . & . & . & . & . & .  &  . & . & . & . & . & . & .   \\
. & . & . & . & . & . & .  &  . & . & . & . & . & . & .  &  .   & .   & .            & .   & .            & .   & [S]           &  . & . & . & . & . & . & .  &  . & . & . & . & . & . & .   \\
\hline
. & . & . & . & . & . & .  &  . & . & . & . & . & . & .  &  . & . & . & . & . & . & .  &  . & . & .   & . & .   & . & .    &  . & . & . & . & . & . & .   \\
. & . & . & . & . & . & .  &  . & . & . & . & . & . & .  &  . & . & . & . & . & . & .  &  . & . & [b] & . & .   & . & .    &  . & . & . & . & . & . & .   \\
. & . & . & . & . & . & .  &  . & . & . & . & . & . & .  &  . & . & . & . & . & . & .  &  . & . & .   & . & .   & . & .    &  . & . & . & . & . & . & .   \\
. & . & . & . & . & . & .  &  . & . & . & . & . & . & .  &  . & . & . & . & . & . & .  &  . & . & .   & . & [b] & . & .    &  . & . & . & . & . & . & .   \\
. & . & . & . & . & . & .  &  . & . & . & . & . & . & .  &  . & . & . & . & . & . & .  &  . & . & .   & . & .   & . & .    &  . & . & . & . & . & . & .   \\
. & . & . & . & . & . & .  &  . & . & . & . & . & . & .  &  . & . & . & . & . & . & .  &  . & . & .   & . & .   & . & [b]  &  . & . & . & . & . & . & .   \\
. & . & . & . & . & . & .  &  . & . & . & . & . & . & .  &  . & . & . & . & . & . & .  &  . & . & .   & . & .   & . & .    &  . & . & . & . & . & . & .   \\
\hline
. & . & . & . & . & . & .  &  . & . & . & . & . & . & .  &  . & . & . & . & . & . & .  &  . & . & . & . & . & . & .  &  [S] & .   & \textbf{[S]} & .   & .            & .   & .   \\
. & . & . & . & . & . & .  &  . & . & . & . & . & . & .  &  . & . & . & . & . & . & .  &  . & . & . & . & . & . & .  &  .   & [S] & .            & .   & .            & .   & .   \\
. & . & . & . & . & . & .  &  . & . & . & . & . & . & .  &  . & . & . & . & . & . & .  &  . & . & . & . & . & . & .  &  .   & .   & [S]          & .   & \textbf{[S]} & .   & .   \\
. & . & . & . & . & . & .  &  . & . & . & . & . & . & .  &  . & . & . & . & . & . & .  &  . & . & . & . & . & . & .  &  .   & .   & .            & [S] & .            & .   & .   \\
. & . & . & . & . & . & .  &  . & . & . & . & . & . & .  &  . & . & . & . & . & . & .  &  . & . & . & . & . & . & .  &  .   & .   & .            & .   & [S]          & .   & \textbf{[S]}   \\
. & . & . & . & . & . & .  &  . & . & . & . & . & . & .  &  . & . & . & . & . & . & .  &  . & . & . & . & . & . & .  &  .   & .   & .            & .   & .            & [S] & .   \\
. & . & . & . & . & . & .  &  . & . & . & . & . & . & .  &  . & . & . & . & . & . & .  &  . & . & . & . & . & . & .  &  .   & .   & .            & .   & .            & .   & [S]   \\
\hline
. & . & . & . & . & . & .  &  . & . & . & . & . & . & .  &  . & . & . & . & . & . & .  &  . & . & . & . & . & . & .  &  . & . & . & . & . & . & .   \\
. & . & . & . & . & . & .  &  . & . & . & . & . & . & .  &  . & . & . & . & . & . & .  &  . & . & . & . & . & . & .  &  . & . & . & . & . & . & .   \\
. & . & . & . & . & . & .  &  . & . & . & . & . & . & .  &  . & . & . & . & . & . & .  &  . & . & . & . & . & . & .  &  . & . & . & . & . & . & .   \\
. & . & . & . & . & . & .  &  . & . & . & . & . & . & .  &  . & . & . & . & . & . & .  &  . & . & . & . & . & . & .  &  . & . & . & . & . & . & .   \\
. & . & . & . & . & . & .  &  . & . & . & . & . & . & .  &  . & . & . & . & . & . & .  &  . & . & . & . & . & . & .  &  . & . & . & . & . & . & .   \\
. & . & . & . & . & . & .  &  . & . & . & . & . & . & .  &  . & . & . & . & . & . & .  &  . & . & . & . & . & . & .  &  . & . & . & . & . & . & .   \\
. & . & . & . & . & . & .  &  . & . & . & . & . & . & .  &  . & . & . & . & . & . & .  &  . & . & . & . & . & . & .  &  . & . & . & . & . & . & .   
\end{array}\right)
\end{align}
\endgroup

Транзитивное замыкание.

\begingroup
\setlength\arraycolsep{1pt}
\begin{align}
&tc(M_3)=\notag\\
&=
\left(\begin{array}{c c c c c c c | c c c c c c c | c c c c c c c | c c c c c c c | c c c c c c c} 
. & . & . & . & . & . & .   &   . & [a] & . & .   & . & .   & .   &   . & \tntm{[aS]} & . & \tinybf{aS} & . & .           & .  &  . & . & \tntm{[aSb]} & . & \tinybf{aSb} & . & .             &  . & . & \tntm{[aSbS]} & . & \tinybf{aSbS} & . & .               \\
. & . & . & . & . & . & .   &   . & .   & . & .   & . & .   & .   &   . & .           & . & .           & . & .           & .  &  . & . & .            & . & .            & . & .             &  . & . & .             & . & .             & . & .               \\
. & . & . & . & . & . & .   &   . & .   & . & [a] & . & .   & .   &   . & .           & . & \tntm{[aS]} & . & \tinybf{aS} & .  &  . & . & .            & . & \tntm{[aSb]} & . & \tinybf{aSb}  &  . & . & .             & . & \tntm{[aSbS]} & . & \tinybf{aSbS}   \\
. & . & . & . & . & . & .   &   . & .   & . & .   & . & .   & .   &   . & .           & . & .           & . & .           & .  &  . & . & .            & . & .            & . & .             &  . & . & .             & . & .             & . & .               \\
. & . & . & . & . & . & .   &   . & .   & . & .   & . & [a] & .   &   . & .           & . & .           & . & \tntm{[aS]} & .  &  . & . & .            & . & .            & . & \tntm{[aSb]}  &  . & . & .             & . & .             & . & \tntm{[aSbS]}   \\
. & . & . & . & . & . & .   &   . & .   & . & .   & . & .   & .   &   . & .           & . & .           & . & .           & .  &  . & . & .            & . & .            & . & .             &  . & . & .             & . & .             & . & .               \\
. & . & . & . & . & . & .   &   . & .   & . & .   & . & .   & .   &   . & .           & . & .           & . & .           & .  &  . & . & .            & . & .            & . & .             &  . & . & .             & . & .             & . & .               \\
\hline                                                                                
. & . & . & . & . & . & .   &   . & . & . & . & . & . & .   &   [S] & .   & [S] & .   & .   & .   & .    &  . & . & . & . & . & . & .  &  . & . & . & . & . & . & .   \\
. & . & . & . & . & . & .   &   . & . & . & . & . & . & .   &   .   & [S] & .   & .   & .   & .   & .    &  . & . & . & . & . & . & .  &  . & . & . & . & . & . & .   \\
. & . & . & . & . & . & .   &   . & . & . & . & . & . & .   &   .   & .   & [S] & .   & [S] & .   & .    &  . & . & . & . & . & . & .  &  . & . & . & . & . & . & .   \\
. & . & . & . & . & . & .   &   . & . & . & . & . & . & .   &   .   & .   & .   & [S] & .   & .   & .    &  . & . & . & . & . & . & .  &  . & . & . & . & . & . & .   \\
. & . & . & . & . & . & .   &   . & . & . & . & . & . & .   &   .   & .   & .   & .   & [S] & .   & [S]  &  . & . & . & . & . & . & .  &  . & . & . & . & . & . & .   \\
. & . & . & . & . & . & .   &   . & . & . & . & . & . & .   &   .   & .   & .   & .   & .   & [S] & .    &  . & . & . & . & . & . & .  &  . & . & . & . & . & . & .   \\
. & . & . & . & . & . & .   &   . & . & . & . & . & . & .   &   .   & .   & .   & .   & .   & .   & [S]  &  . & . & . & . & . & . & .  &  . & . & . & . & . & . & .   \\
\hline                                                                                
. & . & . & . & . & . & .  &  . & . & . & . & . & . & .  &  . & . & . & . & . & . & .  &  . & . & .   & . & .   & . & .    &  . & . & . & . & . & . & .   \\
. & . & . & . & . & . & .  &  . & . & . & . & . & . & .  &  . & . & . & . & . & . & .  &  . & . & [b] & . & .   & . & .    &  . & . & . & . & . & . & .   \\
. & . & . & . & . & . & .  &  . & . & . & . & . & . & .  &  . & . & . & . & . & . & .  &  . & . & .   & . & .   & . & .    &  . & . & . & . & . & . & .   \\
. & . & . & . & . & . & .  &  . & . & . & . & . & . & .  &  . & . & . & . & . & . & .  &  . & . & .   & . & [b] & . & .    &  . & . & . & . & . & . & .   \\
. & . & . & . & . & . & .  &  . & . & . & . & . & . & .  &  . & . & . & . & . & . & .  &  . & . & .   & . & .   & . & .    &  . & . & . & . & . & . & .   \\
. & . & . & . & . & . & .  &  . & . & . & . & . & . & .  &  . & . & . & . & . & . & .  &  . & . & .   & . & .   & . & [b]  &  . & . & . & . & . & . & .   \\
. & . & . & . & . & . & .  &  . & . & . & . & . & . & .  &  . & . & . & . & . & . & .  &  . & . & .   & . & .   & . & .    &  . & . & . & . & . & . & .   \\
\hline
. & . & . & . & . & . & .  &  . & . & . & . & . & . & .  &  . & . & . & . & . & . & .  &  . & . & . & . & . & . & .  &  [S] & .   & [S] & .   & .   & .   & .   \\
. & . & . & . & . & . & .  &  . & . & . & . & . & . & .  &  . & . & . & . & . & . & .  &  . & . & . & . & . & . & .  &  .   & [S] & .   & .   & .   & .   & .   \\
. & . & . & . & . & . & .  &  . & . & . & . & . & . & .  &  . & . & . & . & . & . & .  &  . & . & . & . & . & . & .  &  .   & .   & [S] & .   & [S] & .   & .   \\
. & . & . & . & . & . & .  &  . & . & . & . & . & . & .  &  . & . & . & . & . & . & .  &  . & . & . & . & . & . & .  &  .   & .   & .   & [S] & .   & .   & .   \\
. & . & . & . & . & . & .  &  . & . & . & . & . & . & .  &  . & . & . & . & . & . & .  &  . & . & . & . & . & . & .  &  .   & .   & .   & .   & [S] & .   & [S]   \\
. & . & . & . & . & . & .  &  . & . & . & . & . & . & .  &  . & . & . & . & . & . & .  &  . & . & . & . & . & . & .  &  .   & .   & .   & .   & .   & [S] & .   \\
. & . & . & . & . & . & .  &  . & . & . & . & . & . & .  &  . & . & . & . & . & . & .  &  . & . & . & . & . & . & .  &  .   & .   & .   & .   & .   & .   & [S]   \\
\hline
. & . & . & . & . & . & .  &  . & . & . & . & . & . & .  &  . & . & . & . & . & . & .  &  . & . & . & . & . & . & .  &  . & . & . & . & . & . & .   \\
. & . & . & . & . & . & .  &  . & . & . & . & . & . & .  &  . & . & . & . & . & . & .  &  . & . & . & . & . & . & .  &  . & . & . & . & . & . & .   \\
. & . & . & . & . & . & .  &  . & . & . & . & . & . & .  &  . & . & . & . & . & . & .  &  . & . & . & . & . & . & .  &  . & . & . & . & . & . & .   \\
. & . & . & . & . & . & .  &  . & . & . & . & . & . & .  &  . & . & . & . & . & . & .  &  . & . & . & . & . & . & .  &  . & . & . & . & . & . & .   \\
. & . & . & . & . & . & .  &  . & . & . & . & . & . & .  &  . & . & . & . & . & . & .  &  . & . & . & . & . & . & .  &  . & . & . & . & . & . & .   \\
. & . & . & . & . & . & .  &  . & . & . & . & . & . & .  &  . & . & . & . & . & . & .  &  . & . & . & . & . & . & .  &  . & . & . & . & . & . & .   \\
. & . & . & . & . & . & .  &  . & . & . & . & . & . & .  &  . & . & . & . & . & . & .  &  . & . & . & . & . & . & .  &  . & . & . & . & . & . & .   
\end{array}\right)
\end{align}
\endgroup

Обновлённый граф

\begin{center}
\begin{tikzpicture}[node distance=2cm,shorten >=1pt,on grid,auto] 
   \node[state] (q_0)   {$0$}; 
   \node[state] (q_1) [right=of q_0] {$1$}; 
   \node[state] (q_2) [right=of q_1] {$2$}; 
   \node[state] (q_3) [right=of q_2] {$3$}; 
   \node[state] (q_4) [right=of q_3] {$4$}; 
   \node[state] (q_5) [right=of q_4] {$5$}; 
   \node[state] (q_6) [right=of q_5] {$6$}; 
    \path[->] 
    (q_0) edge  node {a} (q_1)
    (q_0) edge[bend left, above]  node {S} (q_2)
    (q_0) edge[bend right, above]  node {\textbf{S}} (q_4)
    (q_1) edge  node {b} (q_2)
    (q_2) edge  node {a} (q_3)
    (q_2) edge[bend left, above]  node {S} (q_4)
    (q_2) edge[bend right, above]  node {\textbf{S}} (q_6)
    (q_3) edge  node {b} (q_4)          
    (q_4) edge  node {a} (q_5)
    (q_4) edge[bend left, above]  node {S} (q_6)
    (q_5) edge  node {b} (q_6);
\end{tikzpicture}
\end{center}

Его матрица смежности:

$$
M_2 =
\begin{pmatrix}
[S] & [a] & [S]          & .   & \textbf{[S]} & .   &              \\
.   & [S] & [b]          & .   & .            & .   & .            \\
.   & .   & [S]          & [a] & [S]          & .   & \textbf{[S]} \\
.   & .   & .            & [S] & [b]          & .   & .            \\
.   & .   & .            & .   & [S]          & [a] & [S]          \\
.   & .   & .            & .   & .            & [S] & [b]          \\
.   & .   & .            & .   & .            & .   & [S] 
\end{pmatrix}
$$

И, наконец, последняя содержательная итерация.

\begingroup
\setlength\arraycolsep{2pt}
\begin{align}
&M_3 = M_1^1 \otimes M_2 = 
\begin{pmatrix}
. & [a] & .   & .   & .  \\
. & .   & [S] & .   & .  \\
. & .   & .   & [b] & .  \\
. & .   & .   & .   & [S] \\
. & .   & .   & .   & .
\end{pmatrix}
\otimes 
\begin{pmatrix}
[S] & [a] & [S] & .   & [S] & .   & .   \\
.   & [S] & [b] & .   & .   & .   & .   \\
.   & .   & [S] & [a] & [S] & .   & [S] \\
.   & .   & .   & [S] & [b] & .   & .   \\
.   & .   & .   & .   & [S] & [a] & [S] \\
.   & .   & .   & .   & .   & [S] & [b] \\
.   & .   & .   & .   & .   & .   & [S] 
\end{pmatrix}
=\notag\\
&=
\left(\begin{array}{c c c c c c c | c c c c c c c | c c c c c c c | c c c c c c c | c c c c c c c} 
. & . & . & . & . & . & .  &  . & [a] & . & .   & . & .   & .  &  . & . & . & . & . & . & .  &  . & . & . & . & . & . & .  &  . & . & . & . & . & . & .   \\
. & . & . & . & . & . & .  &  . & .   & . & .   & . & .   & .  &  . & . & . & . & . & . & .  &  . & . & . & . & . & . & .  &  . & . & . & . & . & . & .   \\
. & . & . & . & . & . & .  &  . & .   & . & [a] & . & .   & .  &  . & . & . & . & . & . & .  &  . & . & . & . & . & . & .  &  . & . & . & . & . & . & .   \\
. & . & . & . & . & . & .  &  . & .   & . & .   & . & .   & .  &  . & . & . & . & . & . & .  &  . & . & . & . & . & . & .  &  . & . & . & . & . & . & .   \\
. & . & . & . & . & . & .  &  . & .   & . & .   & . & [a] & .  &  . & . & . & . & . & . & .  &  . & . & . & . & . & . & .  &  . & . & . & . & . & . & .   \\
. & . & . & . & . & . & .  &  . & .   & . & .   & . & .   & .  &  . & . & . & . & . & . & .  &  . & . & . & . & . & . & .  &  . & . & . & . & . & . & .   \\
. & . & . & . & . & . & .  &  . & .   & . & .   & . & .   & .  &  . & . & . & . & . & . & .  &  . & . & . & . & . & . & .  &  . & . & . & . & . & . & .   \\
\hline
. & . & . & . & . & . & .  &  . & . & . & . & . & . & .  &  [S] & .   & [S] & .   & \textbf{[S]} & .   & .             &  . & . & . & . & . & . & .  &  . & . & . & . & . & . & .   \\
. & . & . & . & . & . & .  &  . & . & . & . & . & . & .  &  .   & [S] & .   & .   & .            & .   & .             &  . & . & . & . & . & . & .  &  . & . & . & . & . & . & .   \\
. & . & . & . & . & . & .  &  . & . & . & . & . & . & .  &  .   & .   & [S] & .   & [S]          & .   & \textbf{[S]}  &  . & . & . & . & . & . & .  &  . & . & . & . & . & . & .   \\
. & . & . & . & . & . & .  &  . & . & . & . & . & . & .  &  .   & .   & .   & [S] & .            & .   & .             &  . & . & . & . & . & . & .  &  . & . & . & . & . & . & .   \\
. & . & . & . & . & . & .  &  . & . & . & . & . & . & .  &  .   & .   & .   & .   & [S]          & .   & [S]           &  . & . & . & . & . & . & .  &  . & . & . & . & . & . & .   \\
. & . & . & . & . & . & .  &  . & . & . & . & . & . & .  &  .   & .   & .   & .   & .            & [S] & .             &  . & . & . & . & . & . & .  &  . & . & . & . & . & . & .   \\
. & . & . & . & . & . & .  &  . & . & . & . & . & . & .  &  .   & .   & .   & .   & .            & .   & [S]           &  . & . & . & . & . & . & .  &  . & . & . & . & . & . & .   \\
\hline
. & . & . & . & . & . & .  &  . & . & . & . & . & . & .  &  . & . & . & . & . & . & .  &  . & . & .   & . & .   & . & .    &  . & . & . & . & . & . & .   \\
. & . & . & . & . & . & .  &  . & . & . & . & . & . & .  &  . & . & . & . & . & . & .  &  . & . & [b] & . & .   & . & .    &  . & . & . & . & . & . & .   \\
. & . & . & . & . & . & .  &  . & . & . & . & . & . & .  &  . & . & . & . & . & . & .  &  . & . & .   & . & .   & . & .    &  . & . & . & . & . & . & .   \\
. & . & . & . & . & . & .  &  . & . & . & . & . & . & .  &  . & . & . & . & . & . & .  &  . & . & .   & . & [b] & . & .    &  . & . & . & . & . & . & .   \\
. & . & . & . & . & . & .  &  . & . & . & . & . & . & .  &  . & . & . & . & . & . & .  &  . & . & .   & . & .   & . & .    &  . & . & . & . & . & . & .   \\
. & . & . & . & . & . & .  &  . & . & . & . & . & . & .  &  . & . & . & . & . & . & .  &  . & . & .   & . & .   & . & [b]  &  . & . & . & . & . & . & .   \\
. & . & . & . & . & . & .  &  . & . & . & . & . & . & .  &  . & . & . & . & . & . & .  &  . & . & .   & . & .   & . & .    &  . & . & . & . & . & . & .   \\
\hline
. & . & . & . & . & . & .  &  . & . & . & . & . & . & .  &  . & . & . & . & . & . & .  &  . & . & . & . & . & . & .  &  [S] & .   & [S] & .   & \textbf{[S]} & .   & .             \\
. & . & . & . & . & . & .  &  . & . & . & . & . & . & .  &  . & . & . & . & . & . & .  &  . & . & . & . & . & . & .  &  .   & [S] & .   & .   & .            & .   & .             \\
. & . & . & . & . & . & .  &  . & . & . & . & . & . & .  &  . & . & . & . & . & . & .  &  . & . & . & . & . & . & .  &  .   & .   & [S] & .   & [S]          & .   & \textbf{[S]}  \\
. & . & . & . & . & . & .  &  . & . & . & . & . & . & .  &  . & . & . & . & . & . & .  &  . & . & . & . & . & . & .  &  .   & .   & .   & [S] & .            & .   & .             \\
. & . & . & . & . & . & .  &  . & . & . & . & . & . & .  &  . & . & . & . & . & . & .  &  . & . & . & . & . & . & .  &  .   & .   & .   & .   & [S]          & .   & [S]           \\
. & . & . & . & . & . & .  &  . & . & . & . & . & . & .  &  . & . & . & . & . & . & .  &  . & . & . & . & . & . & .  &  .   & .   & .   & .   & .            & [S] & .             \\
. & . & . & . & . & . & .  &  . & . & . & . & . & . & .  &  . & . & . & . & . & . & .  &  . & . & . & . & . & . & .  &  .   & .   & .   & .   & .            & .   & [S]           \\
\hline
. & . & . & . & . & . & .  &  . & . & . & . & . & . & .  &  . & . & . & . & . & . & .  &  . & . & . & . & . & . & .  &  . & . & . & . & . & . & .   \\
. & . & . & . & . & . & .  &  . & . & . & . & . & . & .  &  . & . & . & . & . & . & .  &  . & . & . & . & . & . & .  &  . & . & . & . & . & . & .   \\
. & . & . & . & . & . & .  &  . & . & . & . & . & . & .  &  . & . & . & . & . & . & .  &  . & . & . & . & . & . & .  &  . & . & . & . & . & . & .   \\
. & . & . & . & . & . & .  &  . & . & . & . & . & . & .  &  . & . & . & . & . & . & .  &  . & . & . & . & . & . & .  &  . & . & . & . & . & . & .   \\
. & . & . & . & . & . & .  &  . & . & . & . & . & . & .  &  . & . & . & . & . & . & .  &  . & . & . & . & . & . & .  &  . & . & . & . & . & . & .   \\
. & . & . & . & . & . & .  &  . & . & . & . & . & . & .  &  . & . & . & . & . & . & .  &  . & . & . & . & . & . & .  &  . & . & . & . & . & . & .   \\
. & . & . & . & . & . & .  &  . & . & . & . & . & . & .  &  . & . & . & . & . & . & .  &  . & . & . & . & . & . & .  &  . & . & . & . & . & . & .   
\end{array}\right)
\end{align}
\endgroup

Транзитивное замыкание.

\begingroup
\setlength\arraycolsep{1pt}
\begin{align}
&tc(M_3)=\notag\\
&=
\left(\begin{array}{c c c c c c c | c c c c c c c | c c c c c c c | c c c c c c c | c c c c c c c} 
. & . & . & . & . & . & .   &   . & [a] & . & .   & . & .   & .   &   . & \tntm{[aS]} & . & \tntm{[aS]} & . & \tinybf{aS} & .  &  . & . & \tntm{[aSb]} & . & \tntm{[aSb]} & . & \tinybf{aSb}  &  . & . & \tntm{[aSbS]} & . & \tntm{[aSbS]} & . & \tinybf{aSbS}   \\
. & . & . & . & . & . & .   &   . & .   & . & .   & . & .   & .   &   . & .           & . & .           & . & .           & .  &  . & . & .            & . & .            & . & .             &  . & . & .             & . & .             & . & .               \\
. & . & . & . & . & . & .   &   . & .   & . & [a] & . & .   & .   &   . & .           & . & \tntm{[aS]} & . & \tntm{[aS]} & .  &  . & . & .            & . & \tntm{[aSb]} & . & \tntm{[aSb]}  &  . & . & .             & . & \tntm{[aSbS]} & . & \tntm{[aSbS]}   \\
. & . & . & . & . & . & .   &   . & .   & . & .   & . & .   & .   &   . & .           & . & .           & . & .           & .  &  . & . & .            & . & .            & . & .             &  . & . & .             & . & .             & . & .               \\
. & . & . & . & . & . & .   &   . & .   & . & .   & . & [a] & .   &   . & .           & . & .           & . & \tntm{[aS]} & .  &  . & . & .            & . & .            & . & \tntm{[aSb]}  &  . & . & .             & . & .             & . & \tntm{[aSbS]}   \\
. & . & . & . & . & . & .   &   . & .   & . & .   & . & .   & .   &   . & .           & . & .           & . & .           & .  &  . & . & .            & . & .            & . & .             &  . & . & .             & . & .             & . & .               \\
. & . & . & . & . & . & .   &   . & .   & . & .   & . & .   & .   &   . & .           & . & .           & . & .           & .  &  . & . & .            & . & .            & . & .             &  . & . & .             & . & .             & . & .               \\
\hline                                                                                
. & . & . & . & . & . & .   &   . & . & . & . & . & . & .   &   [S] & .   & [S] & .   & [S] & .   & .    &  . & . & . & . & . & . & .  &  . & . & . & . & . & . & .   \\
. & . & . & . & . & . & .   &   . & . & . & . & . & . & .   &   .   & [S] & .   & .   & .   & .   & .    &  . & . & . & . & . & . & .  &  . & . & . & . & . & . & .   \\
. & . & . & . & . & . & .   &   . & . & . & . & . & . & .   &   .   & .   & [S] & .   & [S] & .   & [S]  &  . & . & . & . & . & . & .  &  . & . & . & . & . & . & .   \\
. & . & . & . & . & . & .   &   . & . & . & . & . & . & .   &   .   & .   & .   & [S] & .   & .   & .    &  . & . & . & . & . & . & .  &  . & . & . & . & . & . & .   \\
. & . & . & . & . & . & .   &   . & . & . & . & . & . & .   &   .   & .   & .   & .   & [S] & .   & [S]  &  . & . & . & . & . & . & .  &  . & . & . & . & . & . & .   \\
. & . & . & . & . & . & .   &   . & . & . & . & . & . & .   &   .   & .   & .   & .   & .   & [S] & .    &  . & . & . & . & . & . & .  &  . & . & . & . & . & . & .   \\
. & . & . & . & . & . & .   &   . & . & . & . & . & . & .   &   .   & .   & .   & .   & .   & .   & [S]  &  . & . & . & . & . & . & .  &  . & . & . & . & . & . & .   \\
\hline                                                                                
. & . & . & . & . & . & .  &  . & . & . & . & . & . & .  &  . & . & . & . & . & . & .  &  . & . & .   & . & .   & . & .    &  . & . & . & . & . & . & .   \\
. & . & . & . & . & . & .  &  . & . & . & . & . & . & .  &  . & . & . & . & . & . & .  &  . & . & [b] & . & .   & . & .    &  . & . & . & . & . & . & .   \\
. & . & . & . & . & . & .  &  . & . & . & . & . & . & .  &  . & . & . & . & . & . & .  &  . & . & .   & . & .   & . & .    &  . & . & . & . & . & . & .   \\
. & . & . & . & . & . & .  &  . & . & . & . & . & . & .  &  . & . & . & . & . & . & .  &  . & . & .   & . & [b] & . & .    &  . & . & . & . & . & . & .   \\
. & . & . & . & . & . & .  &  . & . & . & . & . & . & .  &  . & . & . & . & . & . & .  &  . & . & .   & . & .   & . & .    &  . & . & . & . & . & . & .   \\
. & . & . & . & . & . & .  &  . & . & . & . & . & . & .  &  . & . & . & . & . & . & .  &  . & . & .   & . & .   & . & [b]  &  . & . & . & . & . & . & .   \\
. & . & . & . & . & . & .  &  . & . & . & . & . & . & .  &  . & . & . & . & . & . & .  &  . & . & .   & . & .   & . & .    &  . & . & . & . & . & . & .   \\
\hline
. & . & . & . & . & . & .  &  . & . & . & . & . & . & .  &  . & . & . & . & . & . & .  &  . & . & . & . & . & . & .  &  [S] & .   & [S] & .   & [S] & .   & .   \\
. & . & . & . & . & . & .  &  . & . & . & . & . & . & .  &  . & . & . & . & . & . & .  &  . & . & . & . & . & . & .  &  .   & [S] & .   & .   & .   & .   & .   \\
. & . & . & . & . & . & .  &  . & . & . & . & . & . & .  &  . & . & . & . & . & . & .  &  . & . & . & . & . & . & .  &  .   & .   & [S] & .   & [S] & .   & [S] \\
. & . & . & . & . & . & .  &  . & . & . & . & . & . & .  &  . & . & . & . & . & . & .  &  . & . & . & . & . & . & .  &  .   & .   & .   & [S] & .   & .   & .   \\
. & . & . & . & . & . & .  &  . & . & . & . & . & . & .  &  . & . & . & . & . & . & .  &  . & . & . & . & . & . & .  &  .   & .   & .   & .   & [S] & .   & [S] \\
. & . & . & . & . & . & .  &  . & . & . & . & . & . & .  &  . & . & . & . & . & . & .  &  . & . & . & . & . & . & .  &  .   & .   & .   & .   & .   & [S] & .   \\
. & . & . & . & . & . & .  &  . & . & . & . & . & . & .  &  . & . & . & . & . & . & .  &  . & . & . & . & . & . & .  &  .   & .   & .   & .   & .   & .   & [S] \\
\hline
. & . & . & . & . & . & .  &  . & . & . & . & . & . & .  &  . & . & . & . & . & . & .  &  . & . & . & . & . & . & .  &  . & . & . & . & . & . & .   \\
. & . & . & . & . & . & .  &  . & . & . & . & . & . & .  &  . & . & . & . & . & . & .  &  . & . & . & . & . & . & .  &  . & . & . & . & . & . & .   \\
. & . & . & . & . & . & .  &  . & . & . & . & . & . & .  &  . & . & . & . & . & . & .  &  . & . & . & . & . & . & .  &  . & . & . & . & . & . & .   \\
. & . & . & . & . & . & .  &  . & . & . & . & . & . & .  &  . & . & . & . & . & . & .  &  . & . & . & . & . & . & .  &  . & . & . & . & . & . & .   \\
. & . & . & . & . & . & .  &  . & . & . & . & . & . & .  &  . & . & . & . & . & . & .  &  . & . & . & . & . & . & .  &  . & . & . & . & . & . & .   \\
. & . & . & . & . & . & .  &  . & . & . & . & . & . & .  &  . & . & . & . & . & . & .  &  . & . & . & . & . & . & .  &  . & . & . & . & . & . & .   \\
. & . & . & . & . & . & .  &  . & . & . & . & . & . & .  &  . & . & . & . & . & . & .  &  . & . & . & . & . & . & .  &  . & . & . & . & . & . & .   
\end{array}\right)
\end{align}
\endgroup


В конечном итоге мы получаем такой же результат как и при первом запуске.
Однако нам потребовалось выполнить существенно больше итераций внешнего цикла, а именно четыре (три содержательных и одна проверочная).
При этом, в ходе работы нам пришлось оперировать сущесвенно б\'{о}льшими матрицами: $35 \times 35$ против $21 \times 21$


\subsection{Вопросы и задачи}
\begin{enumerate}
\item Оценить пространсвенную сложность алгоритма.
\item Оценить временную сложность алгоритма.
\item Найти библиотеку для тензорного произведения. 
Реализовать алгоритм. 
Можно предпологать, что запросы содержат ограниченное число терминалов и нетерминалов. 
Провести замеры. 
Сравнить с матричным.
\end{enumerate}
\section{Сжатое представление леса разбора}

Матричный алгоритм даёт нам ответ на вопрос о достижимости, но не предоставляет самих путей.
Что делать, если мы хотим построить все пути, удовлетворяющие ограичениям?

Проблема в том, что искомое множество путей может быть бесконечным.
Можем ли мы предложить конечную структуру, однозначно описывающую такое множество?
Вспомним, что пересечение контекстно-свободного языка с регулярным --- это контекстно-свободный язык.
Мы знаем, что конекстно-свободный язык можно описать коньекстно-своюодной граммтикой, которая конечна.
Это и есть решение нашего вопроса. 
Осталось толко научиться строить такую грамматику.

Прежде, чем двинуться дальше, рекомендуется вспомнить всё, что касается деревьев вывода~\ref{sect:DerivTree}.

\subsection{Лес разбора как представление контекстно-свободной грамматики}

Для начала нам потребуется внести некоторые изменения в конструкцию дерева вывода.

Во-первых, заметим, что в дереве вывода каждый узел соответсвует выводу какой-то подстроки с известными позициями начала и конца.
Давайте будем сохранять эту информацию в узлах дерева. 
Таким образом, метка любого узла это тройка вида $(i,q,j)$, где $i$ --- координата начала подстроки, соответствующей этому узлу, $j$ --- координата конца, $q \in \Sigma \cup N$ --- метка как в исходном определении.

Во-вторых, заметим, что внутренний узел со своими сыновьями связаны с продукцией в граммтике: узел появляется благодаря применению конкретной продукции в процессе вывода.
Давайте занумеруем все продукции в граммтике и добавим в дерево вывода ещё один тип узлов (дополнительные узлы), в которых будем хранить номер применённой продукции.
Получим следующую конструкцию: непосредственный предок дополнительного узла --- это левая часть продукции, а непосредственные сыновья дополнительного узла --- это правая часть продукции.  

\begin{example}
  Построим модифицированное дерево вывода цепочки $_0a_1b_2a_3b_4a_5b_6$ в грамматике

  \begin{align*}
  G = \langle &&& \{a,b\}, \{S\},  S, \\
  & \{ && \\
       && (0) S & \to a \ S \ b \ S, \\
       && (1) S & \to \varepsilon \\
  & \} && \\ 
  &&& \rangle 
  \end{align*}
  


\begin{center}
\begin{tikzpicture}[shorten >=1pt,on grid,auto,node distance=1.8cm] 
   \node[symbol_node] (s_0_6)   {$(0,S,6)$}; 
   \node[prod_node] (p_0_1) [below=of s_0_6] {$0$}; 
   \node[prod_node,draw=none] (dummy1) [below =of p_0_1] {}; 
   \node[symbol_node] (s_1_1) [below left=of p_0_1] {$(1,S,1)$}; 
   \node[symbol_node] (s_2_6) [below right=of p_0_1]  {$(2,S,6)$}; 

   \node[prod_node] (p_0_2) [below right=of s_2_6] {$0$}; 

   \node[symbol_node] (s_3_3) [below left=of p_0_2] {$(3,S,3)$}; 
   \node[symbol_node] (s_4_6) [below right=of p_0_2] {$(4,S,6)$};

   \node[prod_node] (p_0_3) [below right =of s_4_6] {0}; 

   \node[symbol_node] (s_5_5) [below left =of p_0_3] {$(5,S,5)$}; 
   \node[symbol_node] (s_6_6) [below right=of p_0_3] {$(6,S,6)$}; 

%   \node[state,draw=none] (dummy1) [below =of s_0_6] {}; 
%   \node[state,draw=none] (dummy2) [below =of dummy1] {}; 
   
%   \node[state,draw=none] (dummy3) [below left=of p_0_3] {}; 
%   \node[state,draw=none] (dummy4) [below right=of p_0_4] {}; 


%   \node[symbol_node] (s_0_2) [left=of dummy3] {$(0,S,2)$};    
%   
%   \node[symbol_node] (s_2_4) [between=s_0_2 and s_4_6] {$(2,S,4)$};  
   

   \node[prod_node] (p_1_1) [below =of s_1_1] {$1$}; 
   \node[prod_node] (p_1_2) [below =of s_3_3] {$1$}; 
   \node[prod_node] (p_1_3) [below left =of s_5_5] {$1$}; 
   \node[prod_node] (p_1_4) [below right=of s_6_6] {$1$}; 
   
   

%   \node[prod_node] (p_2_1) [below =of s_1_1] {$2$}; 
%   \node[prod_node] (p_2_2) [below =of s_3_3] {$2$}; 
%   \node[prod_node] (p_2_3) [below =of s_5_5] {$2$}; 

   \node[symbol_node] (eps_6_6) [below right=of p_1_4] {$(6,\varepsilon,6)$}; 
   \node[symbol_node] (b_5_6)   [left=of eps_6_6] {$(5,b,6)$}; 
   \node[symbol_node] (eps_5_5) [left =of b_5_6] {$(5,\varepsilon,5)$}; 
   \node[symbol_node] (a_4_5)   [left=of eps_5_5] {$(4,a,5)$}; 
   \node[symbol_node] (b_3_4)   [left=of a_4_5] {$(3,b,4)$}; 
   \node[symbol_node] (eps_3_3) [left =of b_3_4] {$(3,\varepsilon,3)$}; 
   \node[symbol_node] (a_2_3)   [left=of eps_3_3] {$(2,a,3)$}; 
   \node[symbol_node] (b_1_2)   [left=of a_2_3] {$(1,b,2)$}; 
   \node[symbol_node] (eps_1_1) [left =of b_1_2] {$(1,\varepsilon,1)$}; 
   \node[symbol_node] (a_0_1) [left=of eps_1_1] {$(0,a,1)$};    


    \path[->] 
    (s_0_6) edge (p_0_1)          

    (p_0_1) edge [bend right] (a_0_1)
    (p_0_1) edge (s_1_1)
    (p_0_1) edge (b_1_2)
    (p_0_1) edge (s_2_6)

    (s_2_6) edge (p_0_2)          

    (p_0_2) edge (a_2_3)
    (p_0_2) edge (s_3_3)
    (p_0_2) edge (b_3_4)
    (p_0_2) edge (s_4_6)

    (s_4_6) edge (p_0_3)          

    (p_0_3) edge (a_4_5)
    (p_0_3) edge (s_5_5)
    (p_0_3) edge (b_5_6)
    (p_0_3) edge (s_6_6)

    (s_1_1) edge (p_1_1)
    (p_1_1) edge (eps_1_1)

    (s_3_3) edge (p_1_2)
    (p_1_2) edge (eps_3_3)

    (s_5_5) edge (p_1_3)
    (p_1_3) edge (eps_5_5)

    (s_6_6) edge (p_1_4)
    (p_1_4) edge (eps_6_6)


    ;
\end{tikzpicture}
\end{center}


\end{example}



Сохраняемая нами дополнительная информация позволит переиспользовать узлы в том случае, если деревьев вывода оказалось несколько (в случае неоднозначной грамматики).
При этом мы можем не бояться, что переиспользование узлов может привести к появлению ранее несуществовавших деревьев вывода, так как дополнительная информация позволяет делать только ``безопасные'' склейки и затем восстанавливать только корректные деревья.


\begin{example}
  Сжатие леса вывода.
  Построим несколько деревьев вывода цепочки $_0a_1b_2a_3b_4a_5b_6$ в грамматике

  \begin{align*}
   G_1 = \langle &&& \{a,b\}, \{S\},  S, \\
  & \{ && \\
       && (0) S & \to S S, \\
       && (1) S & \to a \ S \ b, \\
       && (2) S & \to \varepsilon \\
  & \} && \\ 
  &&& \rangle 
  \end{align*}

Пердположим, что мы строим левосторонний вывод.
Тогда после первого применеия продукции 0 у нас есть два варианта переписывания первого нетерминала: либо с применением продукции 0, либо с применением продукции 1.
В первом случае мы примени переписываение по подукции 0.
Дальнейшие шаги деретерминированы и в результате мы получим следующее дерево разбора:

\begin{center}
\begin{tikzpicture}[shorten >=1pt,on grid,auto,node distance=1.8cm] 
   \node[symbol_node] (s_0_6)   {$(0,S,6)$}; 
   \node[prod_node] (p_0_1) [below left=of s_0_6] {$0$}; 
   \node[prod_node,draw=none] (p_0_2) [below right=of s_0_6] {}; 
   \node[symbol_node] (s_0_4) [below left=of p_0_1]  {$(0,S,4)$}; 
   \node[symbol_node,draw=none] (s_2_6) [below right=of p_0_2]  {}; 
   \node[prod_node] (p_0_3) [below =of s_0_4] {$0$}; 
   \node[prod_node,draw=none] (p_0_4) [below =of s_2_6] {}; 

   \node[state,draw=none] (dummy1) [below =of s_0_6] {}; 
   \node[state,draw=none] (dummy2) [below =of dummy1] {}; 
   
   \node[state,draw=none] (dummy3) [below left=of p_0_3] {}; 
   \node[state,draw=none] (dummy4) [below right=of p_0_4] {}; 


   \node[symbol_node] (s_0_2) [left=of dummy3] {$(0,S,2)$};    
   \node[symbol_node] (s_4_6) [right=of dummy4] {$(4,S,6)$};
   \node[symbol_node] (s_2_4) [between=s_0_2 and s_4_6] {$(2,S,4)$};  
   
   \node[prod_node] (p_1_1) [below =of s_0_2] {$1$}; 
   \node[prod_node] (p_1_2) [below =of s_2_4] {$1$}; 
   \node[prod_node] (p_1_3) [below =of s_4_6] {$1$}; 

   \node[symbol_node] (s_1_1) [below =of p_1_1] {$(1,S,1)$}; 
   \node[symbol_node] (s_3_3) [below =of p_1_2] {$(3,S,3)$}; 
   \node[symbol_node] (s_5_5) [below =of p_1_3] {$(5,S,5)$}; 

   \node[prod_node] (p_2_1) [below =of s_1_1] {$2$}; 
   \node[prod_node] (p_2_2) [below =of s_3_3] {$2$}; 
   \node[prod_node] (p_2_3) [below =of s_5_5] {$2$}; 

   \node[symbol_node] (eps_1_1) [below =of p_2_1] {$(1,\varepsilon,1)$}; 
   \node[symbol_node] (eps_3_3) [below =of p_2_2] {$(3,\varepsilon,3)$}; 
   \node[symbol_node] (eps_5_5) [below =of p_2_3] {$(5,\varepsilon,5)$}; 

   \node[symbol_node] (a_0_1) [left=of eps_1_1] {$(0,a,1)$}; 
   \node[symbol_node] (a_2_3) [left=of eps_3_3] {$(2,a,3)$}; 
   \node[symbol_node] (a_4_5) [left=of eps_5_5] {$(4,a,5)$}; 

   \node[symbol_node] (b_1_2) [right=of eps_1_1] {$(1,b,2)$}; 
   \node[symbol_node] (b_3_4) [right=of eps_3_3] {$(3,b,4)$}; 
   \node[symbol_node] (b_5_6) [right=of eps_5_5] {$(5,b,6)$}; 


    \path[->] 
    (s_0_6) edge (p_0_1)          
%    (s_0_6) edge (p_0_2)
    (p_0_1) edge (s_0_4)
    (p_0_1) edge (s_4_6)
%    (p_0_2) edge (s_0_2)
%    (p_0_2) edge (s_2_6)
    (s_0_4) edge (p_0_3)          
%    (s_2_6) edge (p_0_4)
    (p_0_3) edge (s_0_2)
    (p_0_3) edge (s_2_4)
%    (p_0_4) edge (s_2_4)
%    (p_0_4) edge (s_4_6)

    (s_0_2) edge (p_1_1)
    (s_2_4) edge (p_1_2)
    (s_4_6) edge (p_1_3)

    (p_1_1) edge [bend right] (a_0_1)
    (p_1_1) edge (s_1_1)
    (p_1_1) edge [bend left] (b_1_2)

    (p_1_2) edge [bend right] (a_2_3)
    (p_1_2) edge (s_3_3)
    (p_1_2) edge [bend left] (b_3_4)

    (p_1_3) edge [bend right] (a_4_5)
    (p_1_3) edge (s_5_5)
    (p_1_3) edge [bend left] (b_5_6)

    (s_1_1) edge (p_2_1)
    (p_2_1) edge (eps_1_1)

    (s_3_3) edge (p_2_2)
    (p_2_2) edge (eps_3_3)

    (s_5_5) edge (p_2_3)
    (p_2_3) edge (eps_5_5)

    ;
\end{tikzpicture}
\end{center}

Второй вариант --- применить продукцию 1.
Остальные шаги также детерминированы.
Тогда мы получим следующее дерево вывода:

\begin{center}
\begin{tikzpicture}[shorten >=1pt,on grid,auto,node distance=1.8cm] 
   \node[symbol_node] (s_0_6)   {$(0,S,6)$}; 
   \node[prod_node,draw=none] (p_0_1) [below left=of s_0_6] {}; 
   \node[prod_node] (p_0_2) [below right=of s_0_6] {$0$}; 
   \node[symbol_node,draw=none] (s_0_4) [below left=of p_0_1]  {}; 
   \node[symbol_node] (s_2_6) [below right=of p_0_2]  {$(2,S,6)$}; 
   \node[prod_node,draw=none] (p_0_3) [below =of s_0_4] {}; 
   \node[prod_node] (p_0_4) [below =of s_2_6] {$0$}; 

   \node[state,draw=none] (dummy1) [below =of s_0_6] {}; 
   \node[state,draw=none] (dummy2) [below =of dummy1] {}; 
   
   \node[state,draw=none] (dummy3) [below left=of p_0_3] {}; 
   \node[state,draw=none] (dummy4) [below right=of p_0_4] {}; 


   \node[symbol_node] (s_0_2) [left=of dummy3] {$(0,S,2)$};    
   \node[symbol_node] (s_4_6) [right=of dummy4] {$(4,S,6)$};
   \node[symbol_node] (s_2_4) [between=s_0_2 and s_4_6] {$(2,S,4)$};  
   
   \node[prod_node] (p_1_1) [below =of s_0_2] {$1$}; 
   \node[prod_node] (p_1_2) [below =of s_2_4] {$1$}; 
   \node[prod_node] (p_1_3) [below =of s_4_6] {$1$}; 

   \node[symbol_node] (s_1_1) [below =of p_1_1] {$(1,S,1)$}; 
   \node[symbol_node] (s_3_3) [below =of p_1_2] {$(3,S,3)$}; 
   \node[symbol_node] (s_5_5) [below =of p_1_3] {$(5,S,5)$}; 

   \node[prod_node] (p_2_1) [below =of s_1_1] {$2$}; 
   \node[prod_node] (p_2_2) [below =of s_3_3] {$2$}; 
   \node[prod_node] (p_2_3) [below =of s_5_5] {$2$}; 

   \node[symbol_node] (eps_1_1) [below =of p_2_1] {$(1,\varepsilon,1)$}; 
   \node[symbol_node] (eps_3_3) [below =of p_2_2] {$(3,\varepsilon,3)$}; 
   \node[symbol_node] (eps_5_5) [below =of p_2_3] {$(5,\varepsilon,5)$}; 

   \node[symbol_node] (a_0_1) [left=of eps_1_1] {$(0,a,1)$}; 
   \node[symbol_node] (a_2_3) [left=of eps_3_3] {$(2,a,3)$}; 
   \node[symbol_node] (a_4_5) [left=of eps_5_5] {$(4,a,5)$}; 

   \node[symbol_node] (b_1_2) [right=of eps_1_1] {$(1,b,2)$}; 
   \node[symbol_node] (b_3_4) [right=of eps_3_3] {$(3,b,4)$}; 
   \node[symbol_node] (b_5_6) [right=of eps_5_5] {$(5,b,6)$}; 


    \path[->] 
    %(s_0_6) edge (p_0_1)          
    (s_0_6) edge (p_0_2)
    %(p_0_1) edge (s_0_4)
    %(p_0_1) edge (s_4_6)
    (p_0_2) edge (s_0_2)
    (p_0_2) edge (s_2_6)
    %(s_0_4) edge (p_0_3)          
    (s_2_6) edge (p_0_4)
    %(p_0_3) edge (s_0_2)
    %(p_0_3) edge (s_2_4)
    (p_0_4) edge (s_2_4)
    (p_0_4) edge (s_4_6)

    (s_0_2) edge (p_1_1)
    (s_2_4) edge (p_1_2)
    (s_4_6) edge (p_1_3)

    (p_1_1) edge [bend right] (a_0_1)
    (p_1_1) edge (s_1_1)
    (p_1_1) edge [bend left] (b_1_2)

    (p_1_2) edge [bend right] (a_2_3)
    (p_1_2) edge (s_3_3)
    (p_1_2) edge [bend left] (b_3_4)

    (p_1_3) edge [bend right] (a_4_5)
    (p_1_3) edge (s_5_5)
    (p_1_3) edge [bend left] (b_5_6)

    (s_1_1) edge (p_2_1)
    (p_2_1) edge (eps_1_1)

    (s_3_3) edge (p_2_2)
    (p_2_2) edge (eps_3_3)

    (s_5_5) edge (p_2_3)
    (p_2_3) edge (eps_5_5)

    ;
\end{tikzpicture}
\end{center}

В двух построенных деревьях большое количество одинаковых узлов.
Построим структуру, которая содержит оба дерева и при этом никакие нетерминальные и терминальные узлы не встречаются дважды.
В результате мы молучим следующий граф:

\begin{center}
\begin{tikzpicture}[shorten >=1pt,on grid,auto,node distance=1.8cm] 
   \node[symbol_node] (s_0_6)   {$(0,S,6)$}; 
   \node[prod_node] (p_0_1) [below left=of s_0_6] {$0$}; 
   \node[prod_node] (p_0_2) [below right=of s_0_6] {$0$}; 
   \node[symbol_node] (s_0_4) [below left=of p_0_1]  {$(0,S,4)$}; 
   \node[symbol_node] (s_2_6) [below right=of p_0_2]  {$(2,S,6)$}; 
   \node[prod_node] (p_0_3) [below =of s_0_4] {$0$}; 
   \node[prod_node] (p_0_4) [below =of s_2_6] {$0$}; 

   \node[state,draw=none] (dummy1) [below =of s_0_6] {}; 
   \node[state,draw=none] (dummy2) [below =of dummy1] {}; 
   
   \node[state,draw=none] (dummy3) [below left=of p_0_3] {}; 
   \node[state,draw=none] (dummy4) [below right=of p_0_4] {}; 


   \node[symbol_node] (s_0_2) [left=of dummy3] {$(0,S,2)$};    
   \node[symbol_node] (s_4_6) [right=of dummy4] {$(4,S,6)$};
   \node[symbol_node] (s_2_4) [between=s_0_2 and s_4_6] {$(2,S,4)$};  
   
   \node[prod_node] (p_1_1) [below =of s_0_2] {$1$}; 
   \node[prod_node] (p_1_2) [below =of s_2_4] {$1$}; 
   \node[prod_node] (p_1_3) [below =of s_4_6] {$1$}; 

   \node[symbol_node] (s_1_1) [below =of p_1_1] {$(1,S,1)$}; 
   \node[symbol_node] (s_3_3) [below =of p_1_2] {$(3,S,3)$}; 
   \node[symbol_node] (s_5_5) [below =of p_1_3] {$(5,S,5)$}; 

   \node[prod_node] (p_2_1) [below =of s_1_1] {$2$}; 
   \node[prod_node] (p_2_2) [below =of s_3_3] {$2$}; 
   \node[prod_node] (p_2_3) [below =of s_5_5] {$2$}; 

   \node[symbol_node] (eps_1_1) [below =of p_2_1] {$(1,\varepsilon,1)$}; 
   \node[symbol_node] (eps_3_3) [below =of p_2_2] {$(3,\varepsilon,3)$}; 
   \node[symbol_node] (eps_5_5) [below =of p_2_3] {$(5,\varepsilon,5)$}; 

   \node[symbol_node] (a_0_1) [left=of eps_1_1] {$(0,a,1)$}; 
   \node[symbol_node] (a_2_3) [left=of eps_3_3] {$(2,a,3)$}; 
   \node[symbol_node] (a_4_5) [left=of eps_5_5] {$(4,a,5)$}; 

   \node[symbol_node] (b_1_2) [right=of eps_1_1] {$(1,b,2)$}; 
   \node[symbol_node] (b_3_4) [right=of eps_3_3] {$(3,b,4)$}; 
   \node[symbol_node] (b_5_6) [right=of eps_5_5] {$(5,b,6)$}; 


    \path[->] 
    (s_0_6) edge (p_0_1)          
    (s_0_6) edge (p_0_2)
    (p_0_1) edge (s_0_4)
    (p_0_1) edge (s_4_6)
    (p_0_2) edge (s_0_2)
    (p_0_2) edge (s_2_6)
    (s_0_4) edge (p_0_3)          
    (s_2_6) edge (p_0_4)
    (p_0_3) edge (s_0_2)
    (p_0_3) edge (s_2_4)
    (p_0_4) edge (s_2_4)
    (p_0_4) edge (s_4_6)

    (s_0_2) edge (p_1_1)
    (s_2_4) edge (p_1_2)
    (s_4_6) edge (p_1_3)

    (p_1_1) edge [bend right] (a_0_1)
    (p_1_1) edge (s_1_1)
    (p_1_1) edge [bend left] (b_1_2)

    (p_1_2) edge [bend right] (a_2_3)
    (p_1_2) edge (s_3_3)
    (p_1_2) edge [bend left] (b_3_4)

    (p_1_3) edge [bend right] (a_4_5)
    (p_1_3) edge (s_5_5)
    (p_1_3) edge [bend left] (b_5_6)

    (s_1_1) edge (p_2_1)
    (p_2_1) edge (eps_1_1)

    (s_3_3) edge (p_2_2)
    (p_2_2) edge (eps_3_3)

    (s_5_5) edge (p_2_3)
    (p_2_3) edge (eps_5_5)

    ;
\end{tikzpicture}
\end{center}


\end{example}


Мы получили очень простой вариянт сжатого представления леса разбора (Shared Packed Parse Forest, SPPF). 
Впервые подобная идея была предложена Джоаном Рекерсом в его кандидатской диссертации~\cite{SPPF}.
В дальнейшем она нашла широкое применеие в обобщённом (generalized) синтаксическом анализе и получила серьёзное развитие.
В частности, наш вариант, хоть и позволяет избежать экспоненциального разростания леса разбора, всё же не является оптимальным.
Оптимальное асимптотическое поведение достигается при использовании бинаризованного SPPF~\cite{Billot:1989:SSF:981623.981641} --- в этом случае объём леса составляет $O(n^3)$, где $n$ --- это длина входной строки.

SPPF применяется в таких алгоритмах синтаксического анализа, как RNGLR~\cite{Scott:2006:RNG:1146809.1146810}, бинаризованная верся SPPF в BRNGLR~\cite{Scott:2007:BCT:1289813.1289815} и GLL~\cite{Scott:2010:GP:1860132.1860320,10.1007/978-3-662-46663-6_5}\footnote{Ещё немного полезной информации про SPPF: \url{http://www.bramvandersanden.com/post/2014/06/shared-packed-parse-forest/}.}.

В действительности SPPF может содержать в себе циклы. Для линейного входа можно получить граф, когда есть возможность выводить по грамматике бесконечные эпсилон-цепочки. Циклы будут вырожденными, но они будут. SPPF также можно построить и для графов.

В графе может существовать множество способов получить путь из одной вершины в другую. И точно так же при построении деревьев вывода путей может появиться несколько одинаковых нетерминалов, получаемых в разных деревьях по-разному. При объединении в SPPF может оказаться, что какой-то путь из вершины $a$ в вершину $b$ является подпутем другого пути из вершины $a$ в вершину $b$, просто более длинного. То есть появятся циклические зависимости.


%В графе может существовать множество способов получить путь из одной вершины в другую. Поэтому может оказаться, что один путь из вершины 0 в вершину 2 является подпутем другого пути из вершины 0 в вершину 2, более длинного. То есть появятся циклические зависимости. 


\begin{example}
    Дан граф $\mathcal{G}:$
    
    \begin{center}
		\begin{tikzpicture}[shorten >=1pt,on grid,auto]
		\node[state] (q_0)   {$0$};
		\node[state] (q_1) [above right=of q_0] {$1$};
		\node[state] (q_2) [right=of q_0] {$2$};
		\node[state] (q_3) [right=of q_2] {$3$};
		\path[->]
		(q_0) edge  node {a} (q_1)
		(q_1) edge  node {a} (q_2)
		(q_2) edge  node {a} (q_0)
		(q_2) edge[bend left, above]  node {b} (q_3)
		(q_3) edge[bend left, below]  node {b} (q_2);
		\end{tikzpicture}
		
	\end{center}
	
	Дана грамматика  $G=\langle\{a,b\},\{S\},\{S\rightarrow \ a \ S \ b\ | \ a \ b \}, S \rangle$
	
	
	Попробуем найти все пути из вершины 2 в вершину 2, выводимые из нетерминала $S$. Методом пристального взгляда найдем один из них. Пусть это будет $_2a_0a_1a_2a_0a_1a_2b_3b_2b_3b_2b_3b_2$. Построим дерево его вывода.
	
    \begin{center}
	\begin{tikzpicture}[shorten >=1pt,on grid,auto,node distance=1.8cm]
	    \node[symbol_node] (s_2_2)   {$(2,S,2)$};
	    
	    \node[prod_node] (p_0_1) [below =of s_0_6] {$0$}; 
	    \node[symbol_node] (s_0_3) [below =of p_0_1]   {$(0,S,3)$}; 
	    \node[symbol_node] (a_2_0_1) [left =of s_0_3]   {$(2,a,0)$};
	    \node[symbol_node] (b_3_2_1) [right=of s_0_3]   {$(3,b,2)$};
	    
	    \node[prod_node] (p_0_2) [below =of s_0_3] {$0$}; 
	    \node[symbol_node] (s_1_2) [below =of p_0_2]   {$(1,S,2)$}; 
	    \node[symbol_node] (a_0_1_1) [left =of s_1_2]   {$(0,a,1)$};
	    \node[symbol_node] (b_2_3_1) [right=of s_1_2]   {$(2,b,3)$};
	    
	    \node[prod_node] (p_0_3) [below =of s_1_2] {$0$}; 
	    \node[symbol_node] (s_2_3) [below =of p_0_3]   {$(2,S,3)$}; 
	    \node[symbol_node] (a_1_2_1) [left =of s_2_3]   {$(1,a,2)$};
	    \node[symbol_node] (b_3_2_2) [right=of s_2_3]   {$(3,b,2)$};
	    
	    \node[prod_node] (p_0_4) [below =of s_2_3] {$0$}; 
	    \node[symbol_node] (s_0_2) [below =of p_0_4]   {$(0,S,2)$}; 
	    \node[symbol_node] (a_2_0_2) [left =of s_0_2]   {$(2,a,0)$};
	    \node[symbol_node] (b_2_3_2) [right=of s_0_2]   {$(2,b,3)$};
	    
	    \node[prod_node] (p_0_5) [below =of s_0_2] {$0$}; 
	    \node[symbol_node] (s_1_3) [below =of p_0_5]   {$(1,S,3)$}; 
	    \node[symbol_node] (a_0_1_2) [left =of s_1_3]   {$(0,a,1)$};
	    \node[symbol_node] (b_3_2_3) [right=of s_1_3]   {$(3,b,2)$};
	    
	    \node[prod_node] (p_1_1) [below =of s_1_3] {$1$};
	    \node[symbol_node] (a_1_2_2) [below left =of p_1_1]   {$(1,a,2)$};
	    \node[symbol_node] (b_2_3_3) [below right=of p_1_1]   {$(2,b,3)$};
	    
	    \path[->] 
	    (s_2_2) edge (p_0_1)
	    
	    (p_0_1) edge (s_0_3)
	    (p_0_1) edge (a_2_0_1)
	    (p_0_1) edge (b_3_2_1)
	    
	    (s_0_3) edge (p_0_2)
	    
	    (p_0_2) edge (s_1_2)
	    (p_0_2) edge (a_0_1_1)
	    (p_0_2) edge (b_2_3_1)
	    
	    (s_1_2) edge (p_0_3)
	    
	    (p_0_3) edge (s_2_3)
	    (p_0_3) edge (a_1_2_1)
	    (p_0_3) edge (b_3_2_2)
	    
	    (s_2_3) edge (p_0_4)
	    
	    (p_0_4) edge (s_0_2)
	    (p_0_4) edge (a_2_0_2)
	    (p_0_4) edge (b_2_3_2)
	    
	    (s_0_2) edge (p_0_5)
	    
	    (p_0_5) edge (s_1_3)
	    (p_0_5) edge (a_0_1_2)
	    (p_0_5) edge (b_3_2_3)
	    
	    (s_1_3) edge (p_1_1)
	    
	    (p_1_1) edge (a_1_2_2)
	    (p_1_1) edge (b_2_3_3)
	    
	    ;
	\end{tikzpicture}
	\end{center}
	
	
	Так как это только один путь, а таких путей гораздо больше, то к нетерминалу $_1S_3$  применим также применим продукцию 0. Тогда мы получим нетерминал $_2S_2$, который и создаст цикл.
	
	\begin{center}
	\begin{tikzpicture}[shorten >=1pt,on grid,auto,node distance=1.8cm]
	    \node[symbol_node] (s_2_2)   {$(2,S,2)$};
	    
	    \node[prod_node] (p_0_1) [below =of s_0_6] {$0$}; 
	    \node[symbol_node] (s_0_3) [below =of p_0_1]   {$(0,S,3)$}; 
	    \node[symbol_node] (a_2_0_1) [left =of s_0_3]   {$(2,a,0)$};
	    \node[symbol_node] (b_3_2_1) [right=of s_0_3]   {$(3,b,2)$};
	    
	    \node[prod_node] (p_0_2) [below =of s_0_3] {$0$}; 
	    \node[symbol_node] (s_1_2) [below =of p_0_2]   {$(1,S,2)$}; 
	    \node[symbol_node] (a_0_1_1) [left =of s_1_2]   {$(0,a,1)$};
	    \node[symbol_node] (b_2_3_1) [right=of s_1_2]   {$(2,b,3)$};
	    
	    \node[prod_node] (p_0_3) [below =of s_1_2] {$0$}; 
	    \node[symbol_node] (s_2_3) [below =of p_0_3]   {$(2,S,3)$}; 
	    \node[symbol_node] (a_1_2_1) [left =of s_2_3]   {$(1,a,2)$};
	    \node[symbol_node] (b_3_2_2) [right=of s_2_3]   {$(3,b,2)$};
	    
	    \node[prod_node] (p_0_4) [below =of s_2_3] {$0$}; 
	    \node[symbol_node] (s_0_2) [below =of p_0_4]   {$(0,S,2)$}; 
	    \node[symbol_node] (a_2_0_2) [left =of s_0_2]   {$(2,a,0)$};
	    \node[symbol_node] (b_2_3_2) [right=of s_0_2]   {$(2,b,3)$};
	    
	    \node[prod_node] (p_0_5) [below =of s_0_2] {$0$}; 
	    \node[symbol_node] (s_1_3) [below =of p_0_5]   {$(1,S,3)$}; 
	    \node[symbol_node] (a_0_1_2) [left =of s_1_3]   {$(0,a,1)$};
	    \node[symbol_node] (b_3_2_3) [right=of s_1_3]   {$(3,b,2)$};
	    
	    \node[state,draw=none] (dummy1) [below left=of s_1_3] {};
	    \node[state,draw=none] (dummy2) [below right=of s_1_3] {};
	    \node[prod_node] (p_1_1) [left =of dummy1] {$1$};
	    \node[symbol_node] (a_1_2_2) [below left =of p_1_1]   {$(1,a,2)$};
	    \node[symbol_node] (b_2_3_3) [below right=of p_1_1]   {$(2,b,3)$};
	    \node[prod_node] (p_0_6) [right =of dummy2] {$0$};
	    \node[symbol_node] (a_1_2_3) [below left =of p_0_6]   {$(1,a,2)$};
	    \node[symbol_node] (b_2_3_4) [below right=of p_0_6]   {$(2,b,3)$};
	    
	    
	    \path[->] 
	    (s_2_2) edge (p_0_1)
	    
	    (p_0_1) edge (s_0_3)
	    (p_0_1) edge (a_2_0_1)
	    (p_0_1) edge (b_3_2_1)
	    
	    (s_0_3) edge (p_0_2)
	    
	    (p_0_2) edge (s_1_2)
	    (p_0_2) edge (a_0_1_1)
	    (p_0_2) edge (b_2_3_1)
	    
	    (s_1_2) edge (p_0_3)
	    
	    (p_0_3) edge (s_2_3)
	    (p_0_3) edge (a_1_2_1)
	    (p_0_3) edge (b_3_2_2)
	    
	    (s_2_3) edge (p_0_4)
	    
	    (p_0_4) edge (s_0_2)
	    (p_0_4) edge (a_2_0_2)
	    (p_0_4) edge (b_2_3_2)
	    
	    (s_0_2) edge (p_0_5)
	    
	    (p_0_5) edge (s_1_3)
	    (p_0_5) edge (a_0_1_2)
	    (p_0_5) edge (b_3_2_3)
	    
	    (s_1_3) edge (p_1_1)
	    
	    (p_1_1) edge (a_1_2_2)
	    (p_1_1) edge (b_2_3_3)
	    
	    (s_1_3) edge (p_0_6)
	    
	    (p_0_6) edge (a_1_2_3)
	    (p_0_6) edge (b_2_3_4)
	    (p_0_6) edge [bend right] (s_2_2)
	    
	    ;
	\end{tikzpicture}
	\end{center}
	
	Таким образом мы построили SPPF. Обойдя эту структуру необходимое количество раз, мы можем получить любой путь, удовлетворяющий условию. Более того, в полученном графе можно получать любые другие пути по соответствующим нетерминалам, содержащимся в узлах леса.
	
	По полученному графу построим грамматику:
	\begin{align*}
	_2S_2   &\to\ _2a_0\ _0S_3\ _3b_2 \\ 
	_0S_3   &\to\ _0a_1\ _1S_2\ _2b_3 \\
	_1S_2   &\to\ _1a_2\ _2S_3\ _3b_2 \\ 
	_2S_3   &\to\ _2a_0\ _0S_2\ _2b_3 \\ 
	_0S_2   &\to\ _0a_1\ _1S_3\ _3b_2 \\ 
	_1S_3   &\to\ _1a_2\ _2S_2\ _2b_3 \\ 
	_1S_3   &\to\ _1a_2\ _2b_3 \\ 
	\end{align*}
	Видим, что для одного единственного нетерминала существует 2 правила, одно из которых рекурсивное.
    
\end{example}


\subsection{Вопросы и задачи}
\begin{enumerate}
  \item Постройте дерево вывода цепочки $w=aababb$ в грамматике $G=\langle\{a,b\},\{S\},\{S\rightarrow \varepsilon \ | \ a \ S \ b \ S \}, S \rangle$.
  \item Постройте все левосторонние выводы цепочки $w=ababab$ в грамматике $G=\langle\{a,b\},\{S\},\{S\rightarrow \varepsilon \ | \ a \ S \ b \ | S \ S\}, S \rangle$.
  \item Постройте все правосторонние выводы цепочки $w=ababab$ в грамматике $G=\langle\{a,b\},\{S\},\{S\rightarrow \varepsilon \ | \ a \ S \ b \ | S \ S\}, S \rangle$.
  \item \label{t1}Постройте все деревья вывода цепочки $w=ababab$ в грамматике $G=\langle\{a,b\},\{S\},\{S\rightarrow \varepsilon \ | \ a \ S \ b \ | S \ S\}, S \rangle$, соответствующие левосторонним выводам.
  \item \label{t2}Постройте все деревья вывода цепочки $w=ababab$ в грамматике $G=\langle\{a,b\},\{S\},\{S\rightarrow \varepsilon \ | \ a \ S \ b \ | S \ S\}, S \rangle$, соответствующие правосторонним выводам.
  \item Как связаны между собой леса, полученные в предыдущих двух задачах (\ref{t1} и \ref{t2})? Какие выводы можно сделать из такой связи?
  \item Постройте сжатое представление леса разбора, полученного в задаче~\ref{t1}.
  \item Постройте сжатое представление леса разбора, полученного в задаче~\ref{t2}.
  \item \label{t3}Предъявите контекстно-свободную граммтику существенно неоднозначного языка. 
        Возьмите цепочку длины болше пяти, при надлежащую этому языку, и постройте все деревья вывода этой цепочки в предъявленной граммтике. 
  \item Постройте сжатое представление леса, полученного в задаче~\ref{t3}.
\end{enumerate}

\chapter{Алгоритм на основе восходящего анализа}

Традиционно, алгоритмы, применяемые для анализа языков программирования как раз умеют строить дерево разбора --- то, что нам надо.
Только нам бы лес.
Вот и посмотрим, как это можно сделать.

Сперва поговорим про классический синтаксический анализ, потом про его адаптацию к анализу графов.

\section{Восходящий синтаксический анализ}

LR(k) --- алгоритм восходящего синтаксического анализа. 
Идея заключается в следующем: входная последовательность символов считывается слева направо с попутным добавлением в стек и выполнением сворачивания --- замены последовательности терминалов и нетерминалов, лежащих наверху стека, на нетерминал, если существует соответствующее правило в исходной грамматике.

\begin{definition}
Слот (item) --- правило грамматики, в правой части которого имеется точка, отделяющая уже разобранную часть правила (слева от точки) от того, что еще предстоит распознать (справа от точки).
\end{definition}

\begin{definition}
LR-автомат --- автомат с магазинной памятью, состояния которого задаются ``слотами'', также он имеет следующий набор инструкций: \\
1. shift p --- прочитать следующий символ входной последовательности, положив его в стек, и перейти в состояние p \\
2. reduce k --- применить k-ое правило грамматики, правая часть которого уже лежит на стеке: снимаем правую часть и кладём левую часть
\end{definition}

\begin{definition}
Предпросмотр --- метод, применяемый в синтаксическом анализе. Заключается в следующем: устанавливается максимальное количество входящих символов, которое может быть использовано анализатором для решения того, какое правило использовать (в случае восходящего анализа: к какому правилу нужно свернуться).
\end{definition}

\begin{definition}
Управляющая таблица --- таблица, которая для всех состояний LR-автомата содержит: инструкции для выполнения, если на вершине стека --- терминал (при этом в случае LR(k) в каждой ячейке может находиться не более одной инструкции), номер состояния, в которое нужно перейти, если на вершине стека --- нетерминал.
\end{definition}

Когда в текущем состоянии '.' стоит в конце, мы можем выполнить reduce-инструкцию и перейти в новое состояние. Однако при этом могут возникать следующие конфликты:
\begin{itemize}
\item shift-reduce --- ситуация, когда не понятно, читать ли следующий символ или выполнить reduce. Например, если правая часть одного из правил является префиксом правой части другого правила: $N \rightarrow w, M \rightarrow ww'$.
\item reduce-reduce --- ситуация, когда не понятно, к какому правилу нужно применить reduce. Например, если есть два правила с одинаковыми правыми частями: $N \rightarrow w, M \rightarrow w$.
\end{itemize}

Возьмем следующую грамматику:
\begin{align*}
0) S & \rightarrow a S b S \\
1) S & \rightarrow \varepsilon
\end{align*}

Расширим вышеупомянутую грамматику, добавив новый стартовый нетерминал S', и далее будем работать с этой расширенной грамматикой:
\begin{align*}
0) & S \rightarrow a S b S \\
1) & S \rightarrow \varepsilon \\
2) & S' \rightarrow S \$
\end{align*}

\begin{definition}
Замыкание --- обобщение понятия ``item'', заключающееся в добавлении для каждого item'а вида $N \rightarrow \alpha.M\beta$ item'ов вида $M \rightarrow .\gamma$
\end{definition}

\begin{definition}
Ядро --- исходный item, до применения к нему замыкания.
\end{definition}

\begin{example}
Пример ядра и замыкания. 

Возьмем правило 2 нашей грамматики, предположим, что мы только начинаем разбирать данное правило.

Ядром в таком случае является item исходного правила: $S' \rightarrow .S \$$

При замыкании добавятся ещё два item'a с правилами по выводу нетерминала 'S', поэтому получаем три item'a: $S' \rightarrow .S\$$, $S \rightarrow .aSbS$ и $S \rightarrow .\varepsilon$
\end{example}

\begin{example}
Пример построения LR-автомата для нашей грамматики с применением замыкания.
\begin{enumerate}
\item Добавляем стартовое состояние: item правила 0 и его замыкание (вместо item'a $S \rightarrow .\varepsilon$ будем писать $S \rightarrow .$). \\ \\
\tikzset{every picture/.style={line width=0.75pt}}
\begin{tikzpicture}[x=0.75pt,y=0.75pt,yscale=-1,xscale=1]
%Rounded Rect
\draw   (139.33,32.07) .. controls (139.33,25.22) and (144.89,19.67) .. (151.74,19.67) -- (212.44,19.67) .. controls (219.29,19.67) and (224.85,25.22) .. (224.85,32.07) -- (224.85,69.3) .. controls (224.85,76.15) and (219.29,81.71) .. (212.44,81.71) -- (151.74,81.71) .. controls (144.89,81.71) and (139.33,76.15) .. (139.33,69.3) -- cycle ;
\draw (174.09,32.69) node  [align=left] {$S' \rightarrow .S\$$};
\draw (182.09,50.69) node  [align=left] {$S \rightarrow .aSbS$};
\draw (164,69) node  [align=left] {$S \rightarrow .$};
\draw (232.67,49.33) node  [align=left] {\textbf{{\small \textcolor{red}{0}}}};
\end{tikzpicture}

\item По 'S' добавляем переход из стартового состояния в новое состояние 1. \\ \\
\tikzset{every picture/.style={line width=0.75pt}}
\begin{tikzpicture}[x=0.75pt,y=0.75pt,yscale=-1,xscale=1]
%Rounded Rect
\draw   (139.33,32.07) .. controls (139.33,25.22) and (144.89,19.67) .. (151.74,19.67) -- (212.44,19.67) .. controls (219.29,19.67) and (224.85,25.22) .. (224.85,32.07) -- (224.85,69.3) .. controls (224.85,76.15) and (219.29,81.71) .. (212.44,81.71) -- (151.74,81.71) .. controls (144.89,81.71) and (139.33,76.15) .. (139.33,69.3) -- cycle ;
%Rounded Rect
\draw   (145.81,123.98) .. controls (145.81,121.26) and (148.02,119.05) .. (150.74,119.05) -- (211.87,119.05) .. controls (214.6,119.05) and (216.81,121.26) .. (216.81,123.98) -- (216.81,138.78) .. controls (216.81,141.51) and (214.6,143.72) .. (211.87,143.72) -- (150.74,143.72) .. controls (148.02,143.72) and (145.81,141.51) .. (145.81,138.78) -- cycle ;
%Straight Lines 
\draw    (180.81,81.92) -- (180.53,116.51) ;
\draw [shift={(180.51,118.51)}, rotate = 270.47] [color={rgb, 255:red, 0; green, 0; blue, 0 }  ][line width=0.75]    (10.93,-3.29) .. controls (6.95,-1.4) and (3.31,-0.3) .. (0,0) .. controls (3.31,0.3) and (6.95,1.4) .. (10.93,3.29)   ;
\draw (174.09,32.69) node  [align=left] {$S' \rightarrow .S\$$};
\draw (182.09,50.69) node  [align=left] {$S \rightarrow .aSbS$};
\draw (164,69) node  [align=left] {$S \rightarrow .$};
\draw (180.33,131.67) node  [align=left] {$S' \rightarrow S.\$$};
\draw (188,99) node  [align=left] {S};
\draw (232.67,49.33) node  [align=left] {\textbf{{\small \textcolor{red}{0}}}};
\draw (224.67,131.33) node  [align=left] {\textbf{{\small \textcolor{red}{1}}}};
\end{tikzpicture}

\item По '\$' добавляем переход из состояния 1 в новое состояние 2. \\ \\
\tikzset{every picture/.style={line width=0.75pt}}
\begin{tikzpicture}[x=0.75pt,y=0.75pt,yscale=-1,xscale=1]
%Rounded Rect
\draw   (139.33,32.07) .. controls (139.33,25.22) and (144.89,19.67) .. (151.74,19.67) -- (212.44,19.67) .. controls (219.29,19.67) and (224.85,25.22) .. (224.85,32.07) -- (224.85,69.3) .. controls (224.85,76.15) and (219.29,81.71) .. (212.44,81.71) -- (151.74,81.71) .. controls (144.89,81.71) and (139.33,76.15) .. (139.33,69.3) -- cycle ;
%Rounded Rect
\draw   (145.81,123.98) .. controls (145.81,121.26) and (148.02,119.05) .. (150.74,119.05) -- (211.87,119.05) .. controls (214.6,119.05) and (216.81,121.26) .. (216.81,123.98) -- (216.81,138.78) .. controls (216.81,141.51) and (214.6,143.72) .. (211.87,143.72) -- (150.74,143.72) .. controls (148.02,143.72) and (145.81,141.51) .. (145.81,138.78) -- cycle ;
%Rounded Rect
\draw   (145.81,185.98) .. controls (145.81,183.26) and (148.02,181.05) .. (150.74,181.05) -- (211.87,181.05) .. controls (214.6,181.05) and (216.81,183.26) .. (216.81,185.98) -- (216.81,200.78) .. controls (216.81,203.51) and (214.6,205.72) .. (211.87,205.72) -- (150.74,205.72) .. controls (148.02,205.72) and (145.81,203.51) .. (145.81,200.78) -- cycle ;
%Straight Lines 
\draw    (180.81,81.92) -- (180.53,116.51) ;
\draw [shift={(180.51,118.51)}, rotate = 270.47] [color={rgb, 255:red, 0; green, 0; blue, 0 }  ][line width=0.75]    (10.93,-3.29) .. controls (6.95,-1.4) and (3.31,-0.3) .. (0,0) .. controls (3.31,0.3) and (6.95,1.4) .. (10.93,3.29)   ;
%Straight Lines 
\draw    (180.81,143.92) -- (180.53,178.51) ;
\draw [shift={(180.51,180.51)}, rotate = 270.47] [color={rgb, 255:red, 0; green, 0; blue, 0 }  ][line width=0.75]    (10.93,-3.29) .. controls (6.95,-1.4) and (3.31,-0.3) .. (0,0) .. controls (3.31,0.3) and (6.95,1.4) .. (10.93,3.29)   ;
\draw (174.09,32.69) node  [align=left] {$S' \rightarrow .S\$$};
\draw (182.09,50.69) node  [align=left] {$S \rightarrow .aSbS$};
\draw (164,69) node  [align=left] {$S \rightarrow .$};
\draw (180.33,131.67) node  [align=left] {$S' \rightarrow S.\$$};
\draw (180.33,193.67) node  [align=left] {$S' \rightarrow S\$.$};
\draw (188,99) node  [align=left] {S};
\draw (188,161) node  [align=left] {\$};
\draw (232.67,49.33) node  [align=left] {\textbf{{\small \textcolor{red}{0}}}};
\draw (224.67,131.33) node  [align=left] {\textbf{{\small \textcolor{red}{1}}}};
\draw (224.67,193.33) node  [align=left] {\textbf{{\small \textcolor{red}{2}}}};
\end{tikzpicture}

\item По 'a' добавляем переход из стартового состояния в новое состояние 3 и делаем его замыкание. Также добавляем переход по 'a' из этого состояния в себя же. \\ \\
\tikzset{every picture/.style={line width=0.75pt}}
\begin{tikzpicture}[x=0.75pt,y=0.75pt,yscale=-1,xscale=1]
%Rounded Rect
\draw   (139.33,32.07) .. controls (139.33,25.22) and (144.89,19.67) .. (151.74,19.67) -- (212.44,19.67) .. controls (219.29,19.67) and (224.85,25.22) .. (224.85,32.07) -- (224.85,69.3) .. controls (224.85,76.15) and (219.29,81.71) .. (212.44,81.71) -- (151.74,81.71) .. controls (144.89,81.71) and (139.33,76.15) .. (139.33,69.3) -- cycle ;
%Rounded Rect
\draw   (145.81,123.98) .. controls (145.81,121.26) and (148.02,119.05) .. (150.74,119.05) -- (211.87,119.05) .. controls (214.6,119.05) and (216.81,121.26) .. (216.81,123.98) -- (216.81,138.78) .. controls (216.81,141.51) and (214.6,143.72) .. (211.87,143.72) -- (150.74,143.72) .. controls (148.02,143.72) and (145.81,141.51) .. (145.81,138.78) -- cycle ;
%Rounded Rect
\draw   (145.81,185.98) .. controls (145.81,183.26) and (148.02,181.05) .. (150.74,181.05) -- (211.87,181.05) .. controls (214.6,181.05) and (216.81,183.26) .. (216.81,185.98) -- (216.81,200.78) .. controls (216.81,203.51) and (214.6,205.72) .. (211.87,205.72) -- (150.74,205.72) .. controls (148.02,205.72) and (145.81,203.51) .. (145.81,200.78) -- cycle ;
%Rounded Rect 
\draw   (16.67,31.41) .. controls (16.67,24.56) and (22.22,19) .. (29.07,19) -- (89.77,19) .. controls (96.62,19) and (102.18,24.56) .. (102.18,31.41) -- (102.18,68.63) .. controls (102.18,75.48) and (96.62,81.04) .. (89.77,81.04) -- (29.07,81.04) .. controls (22.22,81.04) and (16.67,75.48) .. (16.67,68.63) -- cycle ;
%Straight Lines 
\draw    (180.81,81.92) -- (180.53,116.51) ;
\draw [shift={(180.51,118.51)}, rotate = 270.47] [color={rgb, 255:red, 0; green, 0; blue, 0 }  ][line width=0.75]    (10.93,-3.29) .. controls (6.95,-1.4) and (3.31,-0.3) .. (0,0) .. controls (3.31,0.3) and (6.95,1.4) .. (10.93,3.29)   ;
%Straight Lines 
\draw    (180.81,143.92) -- (180.53,178.51) ;
\draw [shift={(180.51,180.51)}, rotate = 270.47] [color={rgb, 255:red, 0; green, 0; blue, 0 }  ][line width=0.75]    (10.93,-3.29) .. controls (6.95,-1.4) and (3.31,-0.3) .. (0,0) .. controls (3.31,0.3) and (6.95,1.4) .. (10.93,3.29)   ;
%Straight Lines
\draw    (139.5,60.47) -- (122.5,60.47) -- (104.51,60.12) ;
\draw [shift={(102.51,60.09)}, rotate = 361.09000000000003] [color={rgb, 255:red, 0; green, 0; blue, 0 }  ][line width=0.75]    (10.93,-3.29) .. controls (6.95,-1.4) and (3.31,-0.3) .. (0,0) .. controls (3.31,0.3) and (6.95,1.4) .. (10.93,3.29)   ;
%Curve Lines
\draw    (99.51,23.34) .. controls (140.25,22.37) and (122.65,34.57) .. (104.22,40.43) ;
\draw [shift={(102.51,40.96)}, rotate = 343.53999999999996] [color={rgb, 255:red, 0; green, 0; blue, 0 }  ][line width=0.75]    (10.93,-3.29) .. controls (6.95,-1.4) and (3.31,-0.3) .. (0,0) .. controls (3.31,0.3) and (6.95,1.4) .. (10.93,3.29)   ;
\draw (174.09,32.69) node  [align=left] {$S' \rightarrow .S\$$};
\draw (182.09,50.69) node  [align=left] {$S \rightarrow .aSbS$};
\draw (164,69) node  [align=left] {$S \rightarrow .$};
\draw (180.33,131.67) node  [align=left] {$S' \rightarrow S.\$$};
\draw (180.33,193.67) node  [align=left] {$S' \rightarrow S\$.$};
\draw (59.33,32.33) node  [align=left] {$S \rightarrow a.SbS$};
\draw (59.42,50.02) node  [align=left] {$S \rightarrow .aSbS$};
\draw (41,68) node  [align=left] {$S \rightarrow .$};
\draw (188,99) node  [align=left] {S};
\draw (188,161) node  [align=left] {\$};
\draw (123,53) node  [align=left] {a};
\draw (130,26) node  [align=left] {a};
\draw (232.67,49.33) node  [align=left] {\textbf{{\small \textcolor{red}{0}}}};
\draw (224.67,131.33) node  [align=left] {\textbf{{\small \textcolor{red}{1}}}};
\draw (224.67,193.33) node  [align=left] {\textbf{{\small \textcolor{red}{2}}}};
\draw (109.67,50.33) node  [align=left] {\textbf{{\small \textcolor{red}{3}}}};
\end{tikzpicture}

\item По 'S' добавляем переход из состояния 3 в новое состояние 4. \\ \\
\tikzset{every picture/.style={line width=0.75pt}}
\begin{tikzpicture}[x=0.75pt,y=0.75pt,yscale=-1,xscale=1]
%Rounded Rect
\draw   (139.33,32.07) .. controls (139.33,25.22) and (144.89,19.67) .. (151.74,19.67) -- (212.44,19.67) .. controls (219.29,19.67) and (224.85,25.22) .. (224.85,32.07) -- (224.85,69.3) .. controls (224.85,76.15) and (219.29,81.71) .. (212.44,81.71) -- (151.74,81.71) .. controls (144.89,81.71) and (139.33,76.15) .. (139.33,69.3) -- cycle ;
%Rounded Rect
\draw   (145.81,123.98) .. controls (145.81,121.26) and (148.02,119.05) .. (150.74,119.05) -- (211.87,119.05) .. controls (214.6,119.05) and (216.81,121.26) .. (216.81,123.98) -- (216.81,138.78) .. controls (216.81,141.51) and (214.6,143.72) .. (211.87,143.72) -- (150.74,143.72) .. controls (148.02,143.72) and (145.81,141.51) .. (145.81,138.78) -- cycle ;
%Rounded Rect
\draw   (145.81,185.98) .. controls (145.81,183.26) and (148.02,181.05) .. (150.74,181.05) -- (211.87,181.05) .. controls (214.6,181.05) and (216.81,183.26) .. (216.81,185.98) -- (216.81,200.78) .. controls (216.81,203.51) and (214.6,205.72) .. (211.87,205.72) -- (150.74,205.72) .. controls (148.02,205.72) and (145.81,203.51) .. (145.81,200.78) -- cycle ;
%Rounded Rect 
\draw   (16.67,31.41) .. controls (16.67,24.56) and (22.22,19) .. (29.07,19) -- (89.77,19) .. controls (96.62,19) and (102.18,24.56) .. (102.18,31.41) -- (102.18,68.63) .. controls (102.18,75.48) and (96.62,81.04) .. (89.77,81.04) -- (29.07,81.04) .. controls (22.22,81.04) and (16.67,75.48) .. (16.67,68.63) -- cycle ;
%Rounded Rect
\draw   (16.67,102.98) .. controls (16.67,100.26) and (18.88,98.05) .. (21.6,98.05) -- (97.25,98.05) .. controls (99.97,98.05) and (102.18,100.26) .. (102.18,102.98) -- (102.18,117.78) .. controls (102.18,120.51) and (99.97,122.72) .. (97.25,122.72) -- (21.6,122.72) .. controls (18.88,122.72) and (16.67,120.51) .. (16.67,117.78) -- cycle ;
%Straight Lines 
\draw    (180.81,81.92) -- (180.53,116.51) ;
\draw [shift={(180.51,118.51)}, rotate = 270.47] [color={rgb, 255:red, 0; green, 0; blue, 0 }  ][line width=0.75]    (10.93,-3.29) .. controls (6.95,-1.4) and (3.31,-0.3) .. (0,0) .. controls (3.31,0.3) and (6.95,1.4) .. (10.93,3.29)   ;
%Straight Lines 
\draw    (180.81,143.92) -- (180.53,178.51) ;
\draw [shift={(180.51,180.51)}, rotate = 270.47] [color={rgb, 255:red, 0; green, 0; blue, 0 }  ][line width=0.75]    (10.93,-3.29) .. controls (6.95,-1.4) and (3.31,-0.3) .. (0,0) .. controls (3.31,0.3) and (6.95,1.4) .. (10.93,3.29)   ;
%Straight Lines
\draw    (139.5,60.47) -- (122.5,60.47) -- (104.51,60.12) ;
\draw [shift={(102.51,60.09)}, rotate = 361.09000000000003] [color={rgb, 255:red, 0; green, 0; blue, 0 }  ][line width=0.75]    (10.93,-3.29) .. controls (6.95,-1.4) and (3.31,-0.3) .. (0,0) .. controls (3.31,0.3) and (6.95,1.4) .. (10.93,3.29)   ;
%Curve Lines
\draw    (99.51,23.34) .. controls (140.25,22.37) and (122.65,34.57) .. (104.22,40.43) ;
\draw [shift={(102.51,40.96)}, rotate = 343.53999999999996] [color={rgb, 255:red, 0; green, 0; blue, 0 }  ][line width=0.75]    (10.93,-3.29) .. controls (6.95,-1.4) and (3.31,-0.3) .. (0,0) .. controls (3.31,0.3) and (6.95,1.4) .. (10.93,3.29)   ;
%Straight Lines
\draw    (59.51,81.05) -- (59.5,90) -- (59.5,96) ;
\draw [shift={(59.5,98)}, rotate = 270.03] [color={rgb, 255:red, 0; green, 0; blue, 0 }  ][line width=0.75]    (10.93,-3.29) .. controls (6.95,-1.4) and (3.31,-0.3) .. (0,0) .. controls (3.31,0.3) and (6.95,1.4) .. (10.93,3.29)   ;
\draw (174.09,32.69) node  [align=left] {$S' \rightarrow .S\$$};
\draw (182.09,50.69) node  [align=left] {$S \rightarrow .aSbS$};
\draw (164,69) node  [align=left] {$S \rightarrow .$};
\draw (180.33,131.67) node  [align=left] {$S' \rightarrow S.\$$};
\draw (180.33,193.67) node  [align=left] {$S' \rightarrow S\$.$};
\draw (59.33,32.33) node  [align=left] {$S \rightarrow a.SbS$};
\draw (59.42,50.02) node  [align=left] {$S \rightarrow .aSbS$};
\draw (41,68) node  [align=left] {$S \rightarrow .$};
\draw (59.33,110.67) node  [align=left] {$S \rightarrow aS.bS$};
\draw (188,99) node  [align=left] {S};
\draw (188,161) node  [align=left] {\$};
\draw (123,53) node  [align=left] {a};
\draw (130,26) node  [align=left] {a};
\draw (69,90) node  [align=left] {S};
\draw (232.67,49.33) node  [align=left] {\textbf{{\small \textcolor{red}{0}}}};
\draw (224.67,131.33) node  [align=left] {\textbf{{\small \textcolor{red}{1}}}};
\draw (224.67,193.33) node  [align=left] {\textbf{{\small \textcolor{red}{2}}}};
\draw (109.67,50.33) node  [align=left] {\textbf{{\small \textcolor{red}{3}}}};
\draw (109.67,110.33) node  [align=left] {\textbf{{\small \textcolor{red}{4}}}};
\end{tikzpicture}

\item По 'b' добавляем переход из состояния 4 в новое состояние 5 и делаем его замыкание. Также добавляем переход по 'a' из этого состояния в состояние 3. \\ \\
\tikzset{every picture/.style={line width=0.75pt}}
\begin{tikzpicture}[x=0.75pt,y=0.75pt,yscale=-1,xscale=1]
%Rounded Rect
\draw   (139.33,32.07) .. controls (139.33,25.22) and (144.89,19.67) .. (151.74,19.67) -- (212.44,19.67) .. controls (219.29,19.67) and (224.85,25.22) .. (224.85,32.07) -- (224.85,69.3) .. controls (224.85,76.15) and (219.29,81.71) .. (212.44,81.71) -- (151.74,81.71) .. controls (144.89,81.71) and (139.33,76.15) .. (139.33,69.3) -- cycle ;
%Rounded Rect
\draw   (145.81,123.98) .. controls (145.81,121.26) and (148.02,119.05) .. (150.74,119.05) -- (211.87,119.05) .. controls (214.6,119.05) and (216.81,121.26) .. (216.81,123.98) -- (216.81,138.78) .. controls (216.81,141.51) and (214.6,143.72) .. (211.87,143.72) -- (150.74,143.72) .. controls (148.02,143.72) and (145.81,141.51) .. (145.81,138.78) -- cycle ;
%Rounded Rect
\draw   (145.81,185.98) .. controls (145.81,183.26) and (148.02,181.05) .. (150.74,181.05) -- (211.87,181.05) .. controls (214.6,181.05) and (216.81,183.26) .. (216.81,185.98) -- (216.81,200.78) .. controls (216.81,203.51) and (214.6,205.72) .. (211.87,205.72) -- (150.74,205.72) .. controls (148.02,205.72) and (145.81,203.51) .. (145.81,200.78) -- cycle ;
%Rounded Rect 
\draw   (16.67,31.41) .. controls (16.67,24.56) and (22.22,19) .. (29.07,19) -- (89.77,19) .. controls (96.62,19) and (102.18,24.56) .. (102.18,31.41) -- (102.18,68.63) .. controls (102.18,75.48) and (96.62,81.04) .. (89.77,81.04) -- (29.07,81.04) .. controls (22.22,81.04) and (16.67,75.48) .. (16.67,68.63) -- cycle ;
%Rounded Rect
\draw   (16.67,102.98) .. controls (16.67,100.26) and (18.88,98.05) .. (21.6,98.05) -- (97.25,98.05) .. controls (99.97,98.05) and (102.18,100.26) .. (102.18,102.98) -- (102.18,117.78) .. controls (102.18,120.51) and (99.97,122.72) .. (97.25,122.72) -- (21.6,122.72) .. controls (18.88,122.72) and (16.67,120.51) .. (16.67,117.78) -- cycle ;
%Rounded Rect
\draw   (16.67,155.74) .. controls (16.67,148.89) and (22.22,143.33) .. (29.07,143.33) -- (89.77,143.33) .. controls (96.62,143.33) and (102.18,148.89) .. (102.18,155.74) -- (102.18,192.97) .. controls (102.18,199.82) and (96.62,205.37) .. (89.77,205.37) -- (29.07,205.37) .. controls (22.22,205.37) and (16.67,199.82) .. (16.67,192.97) -- cycle ;
%Straight Lines 
\draw    (180.81,81.92) -- (180.53,116.51) ;
\draw [shift={(180.51,118.51)}, rotate = 270.47] [color={rgb, 255:red, 0; green, 0; blue, 0 }  ][line width=0.75]    (10.93,-3.29) .. controls (6.95,-1.4) and (3.31,-0.3) .. (0,0) .. controls (3.31,0.3) and (6.95,1.4) .. (10.93,3.29)   ;
%Straight Lines 
\draw    (180.81,143.92) -- (180.53,178.51) ;
\draw [shift={(180.51,180.51)}, rotate = 270.47] [color={rgb, 255:red, 0; green, 0; blue, 0 }  ][line width=0.75]    (10.93,-3.29) .. controls (6.95,-1.4) and (3.31,-0.3) .. (0,0) .. controls (3.31,0.3) and (6.95,1.4) .. (10.93,3.29)   ;
%Straight Lines
\draw    (139.5,60.47) -- (122.5,60.47) -- (104.51,60.12) ;
\draw [shift={(102.51,60.09)}, rotate = 361.09000000000003] [color={rgb, 255:red, 0; green, 0; blue, 0 }  ][line width=0.75]    (10.93,-3.29) .. controls (6.95,-1.4) and (3.31,-0.3) .. (0,0) .. controls (3.31,0.3) and (6.95,1.4) .. (10.93,3.29)   ;
%Curve Lines
\draw    (99.51,23.34) .. controls (140.25,22.37) and (122.65,34.57) .. (104.22,40.43) ;
\draw [shift={(102.51,40.96)}, rotate = 343.53999999999996] [color={rgb, 255:red, 0; green, 0; blue, 0 }  ][line width=0.75]    (10.93,-3.29) .. controls (6.95,-1.4) and (3.31,-0.3) .. (0,0) .. controls (3.31,0.3) and (6.95,1.4) .. (10.93,3.29)   ;
%Straight Lines
\draw    (59.51,81.05) -- (59.5,90) -- (59.5,96) ;
\draw [shift={(59.5,98)}, rotate = 270.03] [color={rgb, 255:red, 0; green, 0; blue, 0 }  ][line width=0.75]    (10.93,-3.29) .. controls (6.95,-1.4) and (3.31,-0.3) .. (0,0) .. controls (3.31,0.3) and (6.95,1.4) .. (10.93,3.29)   ;
%Straight Lines  
\draw    (60.51,123.05) -- (60.5,132) -- (60.5,141.08) ;
\draw [shift={(60.5,143.08)}, rotate = 270] [color={rgb, 255:red, 0; green, 0; blue, 0 }  ][line width=0.75]    (10.93,-3.29) .. controls (6.95,-1.4) and (3.31,-0.3) .. (0,0) .. controls (3.31,0.3) and (6.95,1.4) .. (10.93,3.29)   ;
%Curve Lines
\draw    (100.5,149.38) .. controls (127.95,117.68) and (124.65,99.76) .. (100.98,78.35) ;
\draw [shift={(99.51,77.03)}, rotate = 401.35] [color={rgb, 255:red, 0; green, 0; blue, 0 }  ][line width=0.75]    (10.93,-3.29) .. controls (6.95,-1.4) and (3.31,-0.3) .. (0,0) .. controls (3.31,0.3) and (6.95,1.4) .. (10.93,3.29)   ;
\draw (174.09,32.69) node  [align=left] {$S' \rightarrow .S\$$};
\draw (182.09,50.69) node  [align=left] {$S \rightarrow .aSbS$};
\draw (164,69) node  [align=left] {$S \rightarrow .$};
\draw (180.33,131.67) node  [align=left] {$S' \rightarrow S.\$$};
\draw (180.33,193.67) node  [align=left] {$S' \rightarrow S\$.$};
\draw (59.33,32.33) node  [align=left] {$S \rightarrow a.SbS$};
\draw (59.42,50.02) node  [align=left] {$S \rightarrow .aSbS$};
\draw (41,68) node  [align=left] {$S \rightarrow .$};
\draw (59.33,110.67) node  [align=left] {$S \rightarrow aS.bS$};
\draw (59.33,156.67) node  [align=left] {$S \rightarrow aSb.S$};
\draw (59.42,174.35) node  [align=left] {$S \rightarrow .aSbS$};
\draw (41,192) node  [align=left] {$S \rightarrow .$};
\draw (188,99) node  [align=left] {S};
\draw (188,161) node  [align=left] {\$};
\draw (123,53) node  [align=left] {a};
\draw (130,26) node  [align=left] {a};
\draw (69,90) node  [align=left] {S};
\draw (70,133) node  [align=left] {b};
\draw (126,109) node  [align=left] {a};
\draw (232.67,49.33) node  [align=left] {\textbf{{\small \textcolor{red}{0}}}};
\draw (224.67,131.33) node  [align=left] {\textbf{{\small \textcolor{red}{1}}}};
\draw (224.67,193.33) node  [align=left] {\textbf{{\small \textcolor{red}{2}}}};
\draw (109.67,50.33) node  [align=left] {\textbf{{\small \textcolor{red}{3}}}};
\draw (109.67,110.33) node  [align=left] {\textbf{{\small \textcolor{red}{4}}}};
\draw (110.67,175.33) node  [align=left] {\textbf{{\small \textcolor{red}{5}}}};
\end{tikzpicture}

\item По 'S' добавляем переход из состояния 5 в новое состояние 6. Завершаем построение LR-автомата. \\ \\
\tikzset{every picture/.style={line width=0.75pt}}
\begin{tikzpicture}[x=0.75pt,y=0.75pt,yscale=-1,xscale=1]
%Rounded Rect
\draw   (139.33,32.07) .. controls (139.33,25.22) and (144.89,19.67) .. (151.74,19.67) -- (212.44,19.67) .. controls (219.29,19.67) and (224.85,25.22) .. (224.85,32.07) -- (224.85,69.3) .. controls (224.85,76.15) and (219.29,81.71) .. (212.44,81.71) -- (151.74,81.71) .. controls (144.89,81.71) and (139.33,76.15) .. (139.33,69.3) -- cycle ;
%Rounded Rect 
\draw   (16.67,31.41) .. controls (16.67,24.56) and (22.22,19) .. (29.07,19) -- (89.77,19) .. controls (96.62,19) and (102.18,24.56) .. (102.18,31.41) -- (102.18,68.63) .. controls (102.18,75.48) and (96.62,81.04) .. (89.77,81.04) -- (29.07,81.04) .. controls (22.22,81.04) and (16.67,75.48) .. (16.67,68.63) -- cycle ;
%Rounded Rect
\draw   (16.67,155.74) .. controls (16.67,148.89) and (22.22,143.33) .. (29.07,143.33) -- (89.77,143.33) .. controls (96.62,143.33) and (102.18,148.89) .. (102.18,155.74) -- (102.18,192.97) .. controls (102.18,199.82) and (96.62,205.37) .. (89.77,205.37) -- (29.07,205.37) .. controls (22.22,205.37) and (16.67,199.82) .. (16.67,192.97) -- cycle ;
%Rounded Rect
\draw   (16.67,102.98) .. controls (16.67,100.26) and (18.88,98.05) .. (21.6,98.05) -- (97.25,98.05) .. controls (99.97,98.05) and (102.18,100.26) .. (102.18,102.98) -- (102.18,117.78) .. controls (102.18,120.51) and (99.97,122.72) .. (97.25,122.72) -- (21.6,122.72) .. controls (18.88,122.72) and (16.67,120.51) .. (16.67,117.78) -- cycle ;
%Rounded Rect
\draw   (16.67,230.65) .. controls (16.67,227.93) and (18.88,225.72) .. (21.6,225.72) -- (97.25,225.72) .. controls (99.97,225.72) and (102.18,227.93) .. (102.18,230.65) -- (102.18,245.45) .. controls (102.18,248.18) and (99.97,250.38) .. (97.25,250.38) -- (21.6,250.38) .. controls (18.88,250.38) and (16.67,248.18) .. (16.67,245.45) -- cycle ;
%Rounded Rect
\draw   (145.81,123.98) .. controls (145.81,121.26) and (148.02,119.05) .. (150.74,119.05) -- (211.87,119.05) .. controls (214.6,119.05) and (216.81,121.26) .. (216.81,123.98) -- (216.81,138.78) .. controls (216.81,141.51) and (214.6,143.72) .. (211.87,143.72) -- (150.74,143.72) .. controls (148.02,143.72) and (145.81,141.51) .. (145.81,138.78) -- cycle ;
%Rounded Rect
\draw   (145.81,185.98) .. controls (145.81,183.26) and (148.02,181.05) .. (150.74,181.05) -- (211.87,181.05) .. controls (214.6,181.05) and (216.81,183.26) .. (216.81,185.98) -- (216.81,200.78) .. controls (216.81,203.51) and (214.6,205.72) .. (211.87,205.72) -- (150.74,205.72) .. controls (148.02,205.72) and (145.81,203.51) .. (145.81,200.78) -- cycle ;
%Straight Lines 
\draw    (180.81,81.92) -- (180.53,116.51) ;
\draw [shift={(180.51,118.51)}, rotate = 270.47] [color={rgb, 255:red, 0; green, 0; blue, 0 }  ][line width=0.75]    (10.93,-3.29) .. controls (6.95,-1.4) and (3.31,-0.3) .. (0,0) .. controls (3.31,0.3) and (6.95,1.4) .. (10.93,3.29)   ;
%Straight Lines 
\draw    (180.81,143.92) -- (180.53,178.51) ;
\draw [shift={(180.51,180.51)}, rotate = 270.47] [color={rgb, 255:red, 0; green, 0; blue, 0 }  ][line width=0.75]    (10.93,-3.29) .. controls (6.95,-1.4) and (3.31,-0.3) .. (0,0) .. controls (3.31,0.3) and (6.95,1.4) .. (10.93,3.29)   ;
%Straight Lines
\draw    (59.51,81.05) -- (59.5,90) -- (59.5,96) ;
\draw [shift={(59.5,98)}, rotate = 270.03] [color={rgb, 255:red, 0; green, 0; blue, 0 }  ][line width=0.75]    (10.93,-3.29) .. controls (6.95,-1.4) and (3.31,-0.3) .. (0,0) .. controls (3.31,0.3) and (6.95,1.4) .. (10.93,3.29)   ;
%Straight Lines  
\draw    (60.51,123.05) -- (60.5,132) -- (60.5,141.08) ;
\draw [shift={(60.5,143.08)}, rotate = 270] [color={rgb, 255:red, 0; green, 0; blue, 0 }  ][line width=0.75]    (10.93,-3.29) .. controls (6.95,-1.4) and (3.31,-0.3) .. (0,0) .. controls (3.31,0.3) and (6.95,1.4) .. (10.93,3.29)   ;
%Straight Lines
\draw    (60.51,205.05) -- (60.5,214) -- (60.5,223.62) ;
\draw [shift={(60.5,225.62)}, rotate = 270.02] [color={rgb, 255:red, 0; green, 0; blue, 0 }  ][line width=0.75]    (10.93,-3.29) .. controls (6.95,-1.4) and (3.31,-0.3) .. (0,0) .. controls (3.31,0.3) and (6.95,1.4) .. (10.93,3.29)   ;
%Straight Lines
\draw    (139.5,60.47) -- (122.5,60.47) -- (104.51,60.12) ;
\draw [shift={(102.51,60.09)}, rotate = 361.09000000000003] [color={rgb, 255:red, 0; green, 0; blue, 0 }  ][line width=0.75]    (10.93,-3.29) .. controls (6.95,-1.4) and (3.31,-0.3) .. (0,0) .. controls (3.31,0.3) and (6.95,1.4) .. (10.93,3.29)   ;
%Curve Lines
\draw    (99.51,23.34) .. controls (140.25,22.37) and (122.65,34.57) .. (104.22,40.43) ;
\draw [shift={(102.51,40.96)}, rotate = 343.53999999999996] [color={rgb, 255:red, 0; green, 0; blue, 0 }  ][line width=0.75]    (10.93,-3.29) .. controls (6.95,-1.4) and (3.31,-0.3) .. (0,0) .. controls (3.31,0.3) and (6.95,1.4) .. (10.93,3.29)   ;
%Curve Lines
\draw    (100.5,149.38) .. controls (127.95,117.68) and (124.65,99.76) .. (100.98,78.35) ;
\draw [shift={(99.51,77.03)}, rotate = 401.35] [color={rgb, 255:red, 0; green, 0; blue, 0 }  ][line width=0.75]    (10.93,-3.29) .. controls (6.95,-1.4) and (3.31,-0.3) .. (0,0) .. controls (3.31,0.3) and (6.95,1.4) .. (10.93,3.29)   ;
\draw (174.09,32.69) node  [align=left] {$S' \rightarrow .S\$$};
\draw (59.33,32.33) node  [align=left] {$S \rightarrow a.SbS$};
\draw (59.33,156.67) node  [align=left] {$S \rightarrow aSb.S$};
\draw (59.33,110.67) node  [align=left] {$S \rightarrow aS.bS$};
\draw (59.33,238.33) node  [align=left] {$S \rightarrow aSbS.$};
\draw (180.33,131.67) node  [align=left] {$S' \rightarrow S.\$$};
\draw (180.33,193.67) node  [align=left] {$S' \rightarrow S\$.$};
\draw (59.42,50.02) node  [align=left] {$S \rightarrow .aSbS$};
\draw (41,68) node  [align=left] {$S \rightarrow .$};
\draw (182.09,50.69) node  [align=left] {$S \rightarrow .aSbS$};
\draw (164,69) node  [align=left] {$S \rightarrow .$};
\draw (59.42,174.35) node  [align=left] {$S \rightarrow .aSbS$};
\draw (41,192) node  [align=left] {$S \rightarrow .$};
\draw (126,109) node  [align=left] {a};
\draw (70,133) node  [align=left] {b};
\draw (123,53) node  [align=left] {a};
\draw (130,26) node  [align=left] {a};
\draw (69,90) node  [align=left] {S};
\draw (70,216) node  [align=left] {S};
\draw (188,99) node  [align=left] {S};
\draw (188,161) node  [align=left] {\$};
\draw (232.67,49.33) node  [align=left] {\textbf{{\small \textcolor{red}{0}}}};
\draw (109.67,50.33) node  [align=left] {\textbf{{\small \textcolor{red}{3}}}};
\draw (110.67,175.33) node  [align=left] {\textbf{{\small \textcolor{red}{5}}}};
\draw (224.67,131.33) node  [align=left] {\textbf{{\small \textcolor{red}{1}}}};
\draw (224.67,193.33) node  [align=left] {\textbf{{\small \textcolor{red}{2}}}};
\draw (109.67,110.33) node  [align=left] {\textbf{{\small \textcolor{red}{4}}}};
\draw (109.67,238.33) node  [align=left] {\textbf{{\small \textcolor{red}{6}}}};
\end{tikzpicture}
\end{enumerate}
\end{example}

\begin{example}
Пример управляющей таблицы для работы с построенным ранее LR-автоматом.

\begin{tabular}{|c|c|c|c||c|} 
    \hline & a & b & \$ & S \\ [0.5ex]
    \hline 0 & shift 3 & reduce 1 & reduce 1 & 1 \\
    \hline 1 & & & ACCEPT & \\
    \hline 2 & & & & \\
    \hline  3 & shift 3 & reduce 1 & reduce 1 & 4 \\
    \hline 4 & & shift 5 & & \\
    \hline 5 & shift 3 & reduce 1 & reduce 1 & 6 \\
    \hline 6 & & reduce 0 & reduce 0 & \\ [1ex] 
    \hline
\end{tabular}
\end{example}

\textbf{Ход работы LR-парсера.}
Пусть у нас есть входная строка, LR-автомат со стеком и управляющая таблица. \\
В начальный момент на стеке лежит стартовое состояние LR-автомата, позиция во входной строке соответствует её началу.
На каждом шаге анализируется текущий символ входа и текущее состояние, в котором находится автомат, и совершается одно из действий: 
\begin{itemize}
\item Если текущая позиция --- конец строки и в стеке --- стартовый нетерминал исходной грамматики, то успешно завершаем разбор.
\item Если в управляющей таблице нет инструкции для текущего состояния автомата и текущего символа на входе, то завершаем разбор с ошибкой.
\item Иначе выполняем инструкцию: \\
1) в случае shift --- кладем на стек текущий символ входа, сдвигая при этом текущую позицию, и номер нового состояния с переходом в него. \\
2) в случае reduce --- снимаем со стека 2k элементов: k состояний и k терминалов/нетерминалов (где k --- длина правой части правила, участвующего в свёртке), кладём на стек нетерминал левой части правила и, оказавшись в некотором состоянии, в котором мы были ранее (самое близкое к вершине из хранимых на стеке состояний), выполняем переход в новое состояние с добавлением его номера в стек, если в управляющей таблице пересечение текущего состояния и добавленного ранее нетерминала --- не пусто.
\end{itemize}

\begin{example}
Пример LR-разбора входного слова abab\$ из языка нашей грамматики с использованием построенных ранее LR-автомата и управляющей таблицы.
\begin{enumerate}
\item Начало разбора. На стеке --- стартовое состояние 0. \\ \\
Вход: \,
\begin{tabular}[c]{ |c|c|c|c|c| } 
    \hline \textcolor{red}{a} & b & a & b & \$ \\ \hline
\end{tabular}
\qquad Стек: \,
\begin{tabular}[c]{ |c|c } 
    \hline 0 & \\ \hline
\end{tabular}  
\\
\item Выполняем shift 3: сдвигаем указатель на входе, кладем на стек 'a', новое состояние 3 и переходим в него. \\ \\
Вход: \,
\begin{tabular}[c]{ |c|c|c|c|c| } 
    \hline a & \textcolor{red}{b} & a & b & \$ \\ \hline
\end{tabular}
\qquad Стек: \,
\begin{tabular}[c]{ |c|c|c|c } 
    \hline 0 & a & 3 & \\ \hline
\end{tabular}
\\ 
\item Выполняем reduce 1 (кладем на стек 'S'), кладем новое состояние 4 и переходим в него. \\ \\
Вход: \,
\begin{tabular}[c]{ |c|c|c|c|c| } 
    \hline a & \textcolor{red}{b} & a & b & \$ \\ \hline
\end{tabular}
\qquad Стек: \,
\begin{tabular}[c]{ |c|c|c|c|c|c } 
    \hline 0 & a & 3 & S & 4 & \\ \hline
\end{tabular}
\\ 
\item Выполняем shift 5: сдвигаем указатель на входе, кладем на стек 'b', новое состояние 5 и переходим в него. \\ \\
Вход: \,
\begin{tabular}[c]{ |c|c|c|c|c| } 
    \hline a & b & \textcolor{red}{a} & b & \$ \\ \hline
\end{tabular}
\qquad Стек: \,
\begin{tabular}[c]{ |c|c|c|c|c|c|c|c } 
    \hline 0 & a & 3 & S & 4 & b & 5 & \\ \hline
\end{tabular}
\\
\item Выполняем shift 3. \\ \\
Вход: \,
\begin{tabular}[c]{ |c|c|c|c|c| } 
    \hline a & b & a & \textcolor{red}{b} & \$ \\ \hline
\end{tabular}
\qquad Стек: \,
\begin{tabular}[c]{ |c|c|c|c|c|c|c|c|c|c } 
    \hline 0 & a & 3 & S & 4 & b & 5 & a & 3 & \\ \hline
\end{tabular}
\\
\item Выполняем reduce 1, кладем новое состояние 4 и переходим в него. \\ \\
Вход: \,
\begin{tabular}[c]{ |c|c|c|c|c| } 
    \hline a & b & a & \textcolor{red}{b} & \$ \\ \hline
\end{tabular}
\qquad Стек: \,
\begin{tabular}[c]{ |c|c|c|c|c|c|c|c|c|c|c|c } 
    \hline 0 & a & 3 & S & 4 & b & 5 & a & 3 & S & 4 & \\ \hline
\end{tabular}
\\
\item Выполняем shift 5. \\ \\
Вход: \,
\begin{tabular}[c]{ |c|c|c|c|c| } 
    \hline a & b & a & b & \textcolor{red}{\$} \\ \hline
\end{tabular}
\qquad Стек: \,
\begin{tabular}[c]{ |c|c|c|c|c|c|c|c|c|c|c|c|c|c } 
    \hline 0 & a & 3 & S & 4 & b & 5 & a & 3 & S & 4 & b & 5 & \\ \hline
\end{tabular}
\\
\item Выполняем reduce 1, кладем новое состояние 6 и переходим в него. \\ \\
Вход: \,
\begin{tabular}[c]{ |c|c|c|c|c| } 
    \hline a & b & a & b & \textcolor{red}{\$} \\ \hline
\end{tabular}
\qquad Стек: \,
\begin{tabular}[c]{ |c|c|c|c|c|c|c|c|c|c|c|c|c|c|c|c } 
    \hline 0 & a & 3 & S & 4 & b & 5 & a & 3 & S & 4 & b & 5 & S & 6 & \\ \hline
\end{tabular}
\\
\item Выполняем reduce 0 (снимаем со стека 8 элементов и кладем 'S'), оказываемся в состоянии 5 и делаем переход в новое состояние 6 с добавлением его на стек. \\ \\
Вход: \,
\begin{tabular}[c]{ |c|c|c|c|c| } 
    \hline a & b & a & b & \textcolor{red}{\$} \\ \hline
\end{tabular}
\qquad Стек: \,
\begin{tabular}[c]{ |c|c|c|c|c|c|c|c|c|c } 
    \hline 0 & a & 3 & S & 4 & b & 5 & S & 6 & \\ \hline
\end{tabular}
\\
\item Снова выполняем reduce 0, оказываемся в состоянии 0 и делаем переход в новое состояние 1 с добавлением его на стек. Заканчиваем разбор. \\ \\
Вход: \,
\begin{tabular}[c]{ |c|c|c|c|c| } 
    \hline a & b & a & b & \textcolor{red}{\$} \\ \hline
\end{tabular}
\qquad Стек: \,
\begin{tabular}[c]{ |c|c|c|c } 
    \hline 0 & S & 1 & \\ \hline
\end{tabular}
\end{enumerate}
\end{example}

На практике конфликты стараются решать ещё и на этапе генерации.
Да, реальные тулы могут сгенерировать парсер по неоднозначной грамматике: из переноса или свёртки выбирать перенос, из нескольких свёрток --- первую в каком-то порядке (обычно в порядке появления соответствующих продукций в грамматике).

Существует также модификация LR-разбора, которая называется SLR (Simple LR). 
Основная идея заключается в том, чтобы хранить в управляющей таблице reduce-инструкции только для тех терминалов, которые встречаются в правилах грамматики сразу после нетерминала, к которому выполняется свертка. 
Также существует LALR модификация (Look-Ahead LR).
В ней применяется склеивание нескольких состояний автомата, входные дуги которых имеют общие символы переходы, в одно состояние.
Данные модификации позволяют избежать большее число конфликтов, однако иногда могут иметь таблицы большего размера, чем в классическом LR.
Стоит отметить, что LALR(1)-парсер --- менее мощный, чем LR(1)-парсер, но более мощный, чем SLR(1)-парсер.

\section{GLR и его применение для КС запросов}

Алгоритм LR довольно эффективен, однако позволяет работать не со всеми КС-грамматиками, а только с их подмножеством LR(k). Если грамматика находится за рамками допускаемого класса, некоторые ячейки управляющей таблицы могут содержать несколько значений. В этом случае грамматика отвергалась анализатором.

Чтобы допустить множестенные значения в ячейках управляющей таблицы, потребуется некоторый вид недетерминизма, который даст возможность анализатору обрабатывать несколько возможных вариантов синтаксического разбора параллельно. Именно это и предлагает анализатор Generalized LR (GLR)~\cite{tomita-1987-efficient}. Далее мы рассмотрим общий принцип работы, проиллюстрируем его с помощью примера, а также рассмотрим модификации GLR.

\subsection{Классический GLR алгоритм}

Впервые GLR парсер был представлен Масару Томитой в 1987~\cite{tomita-1987-efficient}. В целом, алгоритм работы идентичен LR с несколькими принципиальными исключениями:
\begin{itemize}
    \item Управляющая таблица модифицирована таким образом, чтобы допускать множественные значения в ячейках.
    \item Для каждой операции reduce, которую мы можем применить на каком-то этапе разбора, создается копия всего стека, после чего, к ней применяется эта операция.
    \item Если к стеку нельзя применить ни одну операцию shift на следующем входном символе, то этот стек отбрасывается.
\end{itemize}

Однако, полное копирование стеков приводит к тому, что дублируется слишком много информации. Поэтому можно сделать следующее:
\begin{itemize}
    \item Объединять одинаковые состояния.
    \item Объединять одинаковые правые префиксы стеков (то есть верхние их части, к которым быстрее всего доберутся операции reduce).
\end{itemize}

Для этого стоит использовать более сложную структуру стека: \textit{граф-структурированный стек} или (\textit{GSS}, Graph Structured Stack). Это направленный граф, в котором вершины соответствуют элементам стека, а ребра их соединяют по правилам управляющей таблицы. У каждой вершины может быть несколько входящих и исходящих узлов: таким образом реализуется то самое объединение, упомянутое в предыдущем абзаце.

\begin{example}
    \label{glr:example}
    Рассмотрим пример GLR разбора с использованием GSS.
    
    Возьмем грамматику $G$ следующего вида:
    \begin{align*}
        &0.\quad S' \to S\$ \\
        &1.\quad S \to abC \\
        &2.\quad S \to aBC \\
        &3.\quad B \to b \\
        &4.\quad C \to c 
    \end{align*}
    
    Входное слово $ w $:
    \begin{align*}
        w = abc\$
    \end{align*}
    
    Построим для данной грамматики LR автомат:
    
    \begin{tikzpicture}[x=0.75pt,y=0.75pt,yscale=-1,xscale=1]
    %uncomment if require: \path (0,306); %set diagram left start at 0, and has height of 306
    
    
    % Text Node
    \draw    (21.5,33) .. controls (21.5,30.24) and (23.74,28) .. (26.5,28) -- (94.5,28) .. controls (97.26,28) and (99.5,30.24) .. (99.5,33) -- (99.5,89) .. controls (99.5,91.76) and (97.26,94) .. (94.5,94) -- (26.5,94) .. controls (23.74,94) and (21.5,91.76) .. (21.5,89) -- cycle  ;
    \draw (60.5,61) node [color={rgb, 255:red, 62; green, 45; blue, 45 }  ,opacity=1 ]  {$ \begin{array}{l}
        S'\rightarrow .S\$\\
        S\rightarrow .abC\\
        S\rightarrow .aBC
        \end{array}$};
    % Text Node
    \draw    (25,141) .. controls (25,138.24) and (27.24,136) .. (30,136) -- (91,136) .. controls (93.76,136) and (96,138.24) .. (96,141) -- (96,156) .. controls (96,158.76) and (93.76,161) .. (91,161) -- (30,161) .. controls (27.24,161) and (25,158.76) .. (25,156) -- cycle  ;
    \draw (60.5,148.5) node   {$S'\rightarrow S.\$$};
    % Text Node
    \draw (446,39) node [scale=0.9,color={rgb, 255:red, 255; green, 0; blue, 0 }  ,opacity=1 ]  {$5$};
    % Text Node
    \draw (93,18) node [scale=0.9,color={rgb, 255:red, 255; green, 0; blue, 0 }  ,opacity=1 ]  {$0$};
    % Text Node
    \draw (213,18) node [scale=0.9,color={rgb, 255:red, 255; green, 0; blue, 0 }  ,opacity=1 ]  {$2$};
    % Text Node
    \draw (89,126) node [scale=0.9,color={rgb, 255:red, 255; green, 0; blue, 0 }  ,opacity=1 ]  {$1$};
    % Text Node
    \draw (330,18) node [scale=0.9,color={rgb, 255:red, 255; green, 0; blue, 0 }  ,opacity=1 ]  {$3$};
    % Text Node
    \draw (212,117) node [scale=0.9,color={rgb, 255:red, 255; green, 0; blue, 0 }  ,opacity=1 ]  {$4$};
    % Text Node
    \draw    (141.5,33) .. controls (141.5,30.24) and (143.74,28) .. (146.5,28) -- (214.5,28) .. controls (217.26,28) and (219.5,30.24) .. (219.5,33) -- (219.5,89) .. controls (219.5,91.76) and (217.26,94) .. (214.5,94) -- (146.5,94) .. controls (143.74,94) and (141.5,91.76) .. (141.5,89) -- cycle  ;
    \draw (180.5,61) node [color={rgb, 255:red, 62; green, 45; blue, 45 }  ,opacity=1 ]  {$ \begin{array}{l}
        S\rightarrow a.BC\\
        S\rightarrow a.bC\\
        B\rightarrow .b
        \end{array}$};
    % Text Node
    \draw (121,51) node [scale=0.9,color={rgb, 255:red, 0; green, 0; blue, 0 }  ,opacity=1 ]  {$a$};
    % Text Node
    \draw    (261.5,33) .. controls (261.5,30.24) and (263.74,28) .. (266.5,28) -- (332.5,28) .. controls (335.26,28) and (337.5,30.24) .. (337.5,33) -- (337.5,89) .. controls (337.5,91.76) and (335.26,94) .. (332.5,94) -- (266.5,94) .. controls (263.74,94) and (261.5,91.76) .. (261.5,89) -- cycle  ;
    \draw (299.5,61) node [color={rgb, 255:red, 62; green, 45; blue, 45 }  ,opacity=1 ]  {$ \begin{array}{l}
        S\rightarrow ab.C\\
        B\rightarrow b.\\
        C\rightarrow .c
        \end{array}$};
    % Text Node
    \draw    (25,209) .. controls (25,206.24) and (27.24,204) .. (30,204) -- (91,204) .. controls (93.76,204) and (96,206.24) .. (96,209) -- (96,224) .. controls (96,226.76) and (93.76,229) .. (91,229) -- (30,229) .. controls (27.24,229) and (25,226.76) .. (25,224) -- cycle  ;
    \draw (60.5,216.5) node   {$S'\rightarrow S\$.$};
    % Text Node
    \draw (69,111) node [scale=0.9,color={rgb, 255:red, 0; green, 0; blue, 0 }  ,opacity=1 ]  {$S$};
    % Text Node
    \draw (69,178) node [scale=0.9,color={rgb, 255:red, 0; green, 0; blue, 0 }  ,opacity=1 ]  {$\$$};
    % Text Node
    \draw    (378,53.5) .. controls (378,50.74) and (380.24,48.5) .. (383,48.5) -- (449,48.5) .. controls (451.76,48.5) and (454,50.74) .. (454,53.5) -- (454,68.5) .. controls (454,71.26) and (451.76,73.5) .. (449,73.5) -- (383,73.5) .. controls (380.24,73.5) and (378,71.26) .. (378,68.5) -- cycle  ;
    \draw (416,61) node [color={rgb, 255:red, 62; green, 45; blue, 45 }  ,opacity=1 ]  {$S\rightarrow abC.$};
    % Text Node
    \draw (240,51) node [scale=0.9,color={rgb, 255:red, 0; green, 0; blue, 0 }  ,opacity=1 ]  {$b$};
    % Text Node
    \draw (356,51) node [scale=0.9,color={rgb, 255:red, 0; green, 0; blue, 0 }  ,opacity=1 ]  {$C$};
    % Text Node
    \draw    (142,132) .. controls (142,129.24) and (144.24,127) .. (147,127) -- (215,127) .. controls (217.76,127) and (220,129.24) .. (220,132) -- (220,167) .. controls (220,169.76) and (217.76,172) .. (215,172) -- (147,172) .. controls (144.24,172) and (142,169.76) .. (142,167) -- cycle  ;
    \draw (181,149.5) node [color={rgb, 255:red, 62; green, 45; blue, 45 }  ,opacity=1 ]  {$ \begin{array}{l}
        S\rightarrow aB.C\\
        C\rightarrow .c
        \end{array}$};
    % Text Node
    \draw    (142,210) .. controls (142,207.24) and (144.24,205) .. (147,205) -- (215,205) .. controls (217.76,205) and (220,207.24) .. (220,210) -- (220,225) .. controls (220,227.76) and (217.76,230) .. (215,230) -- (147,230) .. controls (144.24,230) and (142,227.76) .. (142,225) -- cycle  ;
    \draw (181,217.5) node [color={rgb, 255:red, 62; green, 45; blue, 45 }  ,opacity=1 ]  {$S\rightarrow aBC.$};
    % Text Node
    \draw    (271,142) .. controls (271,139.24) and (273.24,137) .. (276,137) -- (324,137) .. controls (326.76,137) and (329,139.24) .. (329,142) -- (329,157) .. controls (329,159.76) and (326.76,162) .. (324,162) -- (276,162) .. controls (273.24,162) and (271,159.76) .. (271,157) -- cycle  ;
    \draw (300,149.5) node [color={rgb, 255:red, 62; green, 45; blue, 45 }  ,opacity=1 ]  {$C\rightarrow c.$};
    % Text Node
    \draw (247,139) node [scale=0.9,color={rgb, 255:red, 0; green, 0; blue, 0 }  ,opacity=1 ]  {$c$};
    % Text Node
    \draw (308,111) node [scale=0.9,color={rgb, 255:red, 0; green, 0; blue, 0 }  ,opacity=1 ]  {$c$};
    % Text Node
    \draw (190,184) node [scale=0.9,color={rgb, 255:red, 0; green, 0; blue, 0 }  ,opacity=1 ]  {$C$};
    % Text Node
    \draw (322,127) node [scale=0.9,color={rgb, 255:red, 255; green, 0; blue, 0 }  ,opacity=1 ]  {$6$};
    % Text Node
    \draw (213,195) node [scale=0.9,color={rgb, 255:red, 255; green, 0; blue, 0 }  ,opacity=1 ]  {$7$};
    % Text Node
    \draw (84.5,194) node [scale=0.9,color={rgb, 255:red, 255; green, 0; blue, 0 }  ,opacity=1 ]  {$acc$};
    % Connection
    \draw    (99.5,61) -- (139.5,61) ;
    \draw [shift={(141.5,61)}, rotate = 180] [color={rgb, 255:red, 0; green, 0; blue, 0 }  ][line width=0.75]    (10.93,-3.29) .. controls (6.95,-1.4) and (3.31,-0.3) .. (0,0) .. controls (3.31,0.3) and (6.95,1.4) .. (10.93,3.29)   ;
    
    % Connection
    \draw    (60.5,94) -- (60.5,134) ;
    \draw [shift={(60.5,136)}, rotate = 270] [color={rgb, 255:red, 0; green, 0; blue, 0 }  ][line width=0.75]    (10.93,-3.29) .. controls (6.95,-1.4) and (3.31,-0.3) .. (0,0) .. controls (3.31,0.3) and (6.95,1.4) .. (10.93,3.29)   ;
    
    % Connection
    \draw    (60.5,161) -- (60.5,202) ;
    \draw [shift={(60.5,204)}, rotate = 270] [color={rgb, 255:red, 0; green, 0; blue, 0 }  ][line width=0.75]    (10.93,-3.29) .. controls (6.95,-1.4) and (3.31,-0.3) .. (0,0) .. controls (3.31,0.3) and (6.95,1.4) .. (10.93,3.29)   ;
    
    % Connection
    \draw    (219.5,61) -- (259.5,61) ;
    \draw [shift={(261.5,61)}, rotate = 180] [color={rgb, 255:red, 0; green, 0; blue, 0 }  ][line width=0.75]    (10.93,-3.29) .. controls (6.95,-1.4) and (3.31,-0.3) .. (0,0) .. controls (3.31,0.3) and (6.95,1.4) .. (10.93,3.29)   ;
    
    % Connection
    \draw    (337.5,61) -- (376,61) ;
    \draw [shift={(378,61)}, rotate = 180] [color={rgb, 255:red, 0; green, 0; blue, 0 }  ][line width=0.75]    (10.93,-3.29) .. controls (6.95,-1.4) and (3.31,-0.3) .. (0,0) .. controls (3.31,0.3) and (6.95,1.4) .. (10.93,3.29)   ;
    
    % Connection
    \draw    (180.69,94) -- (180.86,125) ;
    \draw [shift={(180.87,127)}, rotate = 269.68] [color={rgb, 255:red, 0; green, 0; blue, 0 }  ][line width=0.75]    (10.93,-3.29) .. controls (6.95,-1.4) and (3.31,-0.3) .. (0,0) .. controls (3.31,0.3) and (6.95,1.4) .. (10.93,3.29)   ;
    
    % Connection
    \draw    (181,172) -- (181,203) ;
    \draw [shift={(181,205)}, rotate = 270] [color={rgb, 255:red, 0; green, 0; blue, 0 }  ][line width=0.75]    (10.93,-3.29) .. controls (6.95,-1.4) and (3.31,-0.3) .. (0,0) .. controls (3.31,0.3) and (6.95,1.4) .. (10.93,3.29)   ;
    
    % Connection
    \draw    (299.69,94) -- (299.92,135) ;
    \draw [shift={(299.93,137)}, rotate = 269.68] [color={rgb, 255:red, 0; green, 0; blue, 0 }  ][line width=0.75]    (10.93,-3.29) .. controls (6.95,-1.4) and (3.31,-0.3) .. (0,0) .. controls (3.31,0.3) and (6.95,1.4) .. (10.93,3.29)   ;
    
    % Connection
    \draw    (220,149.5) -- (269,149.5) ;
    \draw [shift={(271,149.5)}, rotate = 180] [color={rgb, 255:red, 0; green, 0; blue, 0 }  ][line width=0.75]    (10.93,-3.29) .. controls (6.95,-1.4) and (3.31,-0.3) .. (0,0) .. controls (3.31,0.3) and (6.95,1.4) .. (10.93,3.29)   ;
    
    
    \end{tikzpicture}
    
    И управляющую таблицу:
    
    \begin{tabular}{|c|c|c|c|c||c|c|c|} 
        \hline   & a & b & c & \$ & B & C & S \\ [0.5ex]
        \hline 0 & shift 2 & & & & & goto 1 & \\
        \hline 1 &   &   &   &  accept  &   &   & \\
        \hline 2 &   & shift 3  &   &    & goto 4  &   & \\
        \hline 3 &   &   & shift 6 OR reduce 3  &    &   & goto 5  & \\
        \hline 4 &   &   & shift 6  &    &   & goto 7  & \\
        \hline 5 &   &   &   & reduce 1   &   &   & \\
        \hline 6 &   &   &   & reduce 4   &   &   & \\
        \hline 7 &   &   &   & reduce 2   &   &   & \\ [1ex] 
        \hline
    \end{tabular}
    
    Разберем слово $w$ с помощью алгоритма GLR. Использована следующая аннотация: вершины-состояния обозначены кругами, вершины-символы --- прямоугольниками.
    \begin{enumerate}
        \item Инициализируем GSS стартовым состоянием $v_0$: \\ \\
        Вход: \,
        \begin{tabular}[c]{ |c|c|c|c| } 
            \hline a & b & c & \$ \\ \hline
        \end{tabular}
        \qquad GSS: \,
        \begin{tikzpicture}[x=0.5pt,y=0.5pt,yscale=-1,xscale=1]
        %uncomment if require: \path (0,306); %set diagram left start at 0, and has height of 306
        
        
        % Text Node
        \draw  [line width=0.75]   (92, 109) circle [x radius= 13.6, y radius= 13.6]   ;
        \draw (92,109) node [color={rgb, 255:red, 62; green, 45; blue, 45 }  ,opacity=1 ]  {$0$};
        % Text Node
        \draw (110,89) node [color={rgb, 255:red, 62; green, 45; blue, 45 }  ,opacity=1 ]  {$v_{0}$};
        
        
        \end{tikzpicture}
        \\
        
        \item Видим входной символ '$a$', ищем соответствующую ему операцию в управляющей таблице --- $shift\ 2$, строим новый узел $v_1$: \\ \\
        Вход: \,
        \begin{tabular}[c]{ |c|c|c|c| } 
            \hline \textcolor{red}{a} & b & c & \$ \\ \hline
        \end{tabular}
        \qquad GSS: \,
        \begin{tikzpicture}[x=0.5pt,y=0.5pt,yscale=-1,xscale=1]
        %uncomment if require: \path (0,306); %set diagram left start at 0, and has height of 306
        
        
        % Text Node
        \draw  [line width=0.75]   (92, 109) circle [x radius= 13.6, y radius= 13.6]   ;
        \draw (92,109) node [color={rgb, 255:red, 62; green, 45; blue, 45 }  ,opacity=1 ]  {$0$};
        % Text Node
        \draw  [line width=0.75]   (138,98) -- (156,98) -- (156,122) -- (138,122) -- cycle  ;
        \draw (147,110) node [scale=1,color={rgb, 255:red, 62; green, 45; blue, 45 }  ,opacity=1 ]  {а};
        % Text Node
        \draw  [line width=0.75]   (203, 110) circle [x radius= 13.6, y radius= 13.6]   ;
        \draw (203,110) node [color={rgb, 255:red, 62; green, 45; blue, 45 }  ,opacity=1 ]  {$2$};
        % Text Node
        \draw (110,89) node [color={rgb, 255:red, 62; green, 45; blue, 45 }  ,opacity=1 ]  {$v_{0}$};
        % Text Node
        \draw (220,89) node [color={rgb, 255:red, 62; green, 45; blue, 45 }  ,opacity=1 ]  {$v_{1}$};
        % Connection
        \draw    (138,109.84) -- (107.6,109.28) ;
        \draw [shift={(105.6,109.25)}, rotate = 361.03999999999996] [color={rgb, 255:red, 0; green, 0; blue, 0 }  ][line width=0.75]    (10.93,-3.29) .. controls (6.95,-1.4) and (3.31,-0.3) .. (0,0) .. controls (3.31,0.3) and (6.95,1.4) .. (10.93,3.29)   ;
        
        % Connection
        \draw    (189.4,110) -- (158,110) ;
        \draw [shift={(156,110)}, rotate = 360] [color={rgb, 255:red, 0; green, 0; blue, 0 }  ][line width=0.75]    (10.93,-3.29) .. controls (6.95,-1.4) and (3.31,-0.3) .. (0,0) .. controls (3.31,0.3) and (6.95,1.4) .. (10.93,3.29)   ;
        
        
        \end{tikzpicture}
        \\
        
        \item Повторяем для символа '$b$', операции $shift\ 3$ и узла $v_2$: \\ \\
        Вход: \,
        \begin{tabular}[c]{ |c|c|c|c| } 
            \hline a & \textcolor{red}{b} & c & \$ \\ \hline
        \end{tabular}
        \qquad GSS: \,
        \begin{tikzpicture}[x=0.5pt,y=0.5pt,yscale=-1,xscale=1]
        %uncomment if require: \path (0,306); %set diagram left start at 0, and has height of 306
        
        
        % Text Node
        \draw  [line width=0.75]   (92, 109) circle [x radius= 13.6, y radius= 13.6]   ;
        \draw (92,109) node [color={rgb, 255:red, 62; green, 45; blue, 45 }  ,opacity=1 ]  {$0$};
        % Text Node
        \draw  [line width=0.75]   (138,98) -- (156,98) -- (156,122) -- (138,122) -- cycle  ;
        \draw (147,110) node [scale=1,color={rgb, 255:red, 62; green, 45; blue, 45 }  ,opacity=1 ]  {a};
        % Text Node
        \draw  [line width=0.75]   (203, 110) circle [x radius= 13.6, y radius= 13.6]   ;
        \draw (203,110) node [color={rgb, 255:red, 62; green, 45; blue, 45 }  ,opacity=1 ]  {$2$};
        % Text Node
        \draw (110,89) node [color={rgb, 255:red, 62; green, 45; blue, 45 }  ,opacity=1 ]  {$v_{0}$};
        % Text Node
        \draw (220,89) node [color={rgb, 255:red, 62; green, 45; blue, 45 }  ,opacity=1 ]  {$v_{1}$};
        % Text Node
        \draw  [line width=0.75]   (258,98) -- (276,98) -- (276,122) -- (258,122) -- cycle  ;
        \draw (267,110) node [scale=1,color={rgb, 255:red, 62; green, 45; blue, 45 }  ,opacity=1 ]  {b};
        % Text Node
        \draw  [line width=0.75]   (323, 110) circle [x radius= 13.6, y radius= 13.6]   ;
        \draw (323,110) node [color={rgb, 255:red, 62; green, 45; blue, 45 }  ,opacity=1 ]  {$3$};
        % Text Node
        \draw (340,90) node [color={rgb, 255:red, 62; green, 45; blue, 45 }  ,opacity=1 ]  {$v_{2}$};
        % Connection
        \draw    (138,109.84) -- (107.6,109.28) ;
        \draw [shift={(105.6,109.25)}, rotate = 361.03999999999996] [color={rgb, 255:red, 0; green, 0; blue, 0 }  ][line width=0.75]    (10.93,-3.29) .. controls (6.95,-1.4) and (3.31,-0.3) .. (0,0) .. controls (3.31,0.3) and (6.95,1.4) .. (10.93,3.29)   ;
        
        % Connection
        \draw    (189.4,110) -- (158,110) ;
        \draw [shift={(156,110)}, rotate = 360] [color={rgb, 255:red, 0; green, 0; blue, 0 }  ][line width=0.75]    (10.93,-3.29) .. controls (6.95,-1.4) and (3.31,-0.3) .. (0,0) .. controls (3.31,0.3) and (6.95,1.4) .. (10.93,3.29)   ;
        
        % Connection
        \draw    (309.4,110) -- (278,110) ;
        \draw [shift={(276,110)}, rotate = 360] [color={rgb, 255:red, 0; green, 0; blue, 0 }  ][line width=0.75]    (10.93,-3.29) .. controls (6.95,-1.4) and (3.31,-0.3) .. (0,0) .. controls (3.31,0.3) and (6.95,1.4) .. (10.93,3.29)   ;
        
        % Connection
        \draw    (258,110) -- (218.6,110) ;
        \draw [shift={(216.6,110)}, rotate = 360] [color={rgb, 255:red, 0; green, 0; blue, 0 }  ][line width=0.75]    (10.93,-3.29) .. controls (6.95,-1.4) and (3.31,-0.3) .. (0,0) .. controls (3.31,0.3) and (6.95,1.4) .. (10.93,3.29)   ;
        
        
        \end{tikzpicture}
        \\
        
        \item При обработке узла $v_3$ у нас возникает конфликт shift-reduce: $shift\ 6\ OR\ reduce\ 3$. Мы смотрим на вершины, смежные $v_2$, на управляющую таблицу и на правило вывода под номером 3 для поиска альтернативного построения стека. Находим $goto\ 4$ и строим вершину $v_3$ с соответствующим переходом по нетерминалу $B$ из $v_1$ (т.к. количество символов в правой части правила вывода 3 равняется 1, значит мы в дереве опустимся на глубину 1 по вершинам-состояниям):\\ \\
        Вход: \,
        \begin{tabular}[c]{ |c|c|c|c| } 
            \hline a & b & c & \$ \\ \hline
        \end{tabular}
        \qquad GSS: \,
        \begin{tikzpicture}[x=0.5pt,y=0.5pt,yscale=-1,xscale=1]
        %uncomment if require: \path (0,422); %set diagram left start at 0, and has height of 422
        
        
        % Text Node
        \draw  [line width=0.75]   (92, 110) circle [x radius= 13.6, y radius= 13.6]   ;
        \draw (92,110) node [color={rgb, 255:red, 62; green, 45; blue, 45 }  ,opacity=1 ]  {$0$};
        % Text Node
        \draw  [line width=0.75]   (138,98) -- (156,98) -- (156,122) -- (138,122) -- cycle  ;
        \draw (147,110) node [scale=1,color={rgb, 255:red, 62; green, 45; blue, 45 }  ,opacity=1 ]  {а};
        % Text Node
        \draw  [line width=0.75]   (203, 110) circle [x radius= 13.6, y radius= 13.6]   ;
        \draw (203,110) node [color={rgb, 255:red, 62; green, 45; blue, 45 }  ,opacity=1 ]  {$2$};
        % Text Node
        \draw (110,89) node [color={rgb, 255:red, 62; green, 45; blue, 45 }  ,opacity=1 ]  {$v_{0}$};
        % Text Node
        \draw (220,89) node [color={rgb, 255:red, 62; green, 45; blue, 45 }  ,opacity=1 ]  {$v_{1}$};
        % Text Node
        \draw  [line width=0.75]   (258,98) -- (276,98) -- (276,122) -- (258,122) -- cycle  ;
        \draw (267,110) node [scale=1,color={rgb, 255:red, 62; green, 45; blue, 45 }  ,opacity=1 ]  {b};
        % Text Node
        \draw  [line width=0.75]   (323, 110) circle [x radius= 13.6, y radius= 13.6]   ;
        \draw (323,110) node [color={rgb, 255:red, 62; green, 45; blue, 45 }  ,opacity=1 ]  {$3$};
        % Text Node
        \draw (340,90) node [color={rgb, 255:red, 62; green, 45; blue, 45 }  ,opacity=1 ]  {$v_{2}$};
        % Text Node
        \draw  [line width=0.75]   (258,158) -- (276,158) -- (276,182) -- (258,182) -- cycle  ;
        \draw (267,170) node [scale=1,color={rgb, 255:red, 62; green, 45; blue, 45 }  ,opacity=1 ]  {B};
        % Text Node
        \draw  [line width=0.75]   (323, 170) circle [x radius= 13.6, y radius= 13.6]   ;
        \draw (323,170) node [color={rgb, 255:red, 62; green, 45; blue, 45 }  ,opacity=1 ]  {$4$};
        % Text Node
        \draw (340,149) node [color={rgb, 255:red, 62; green, 45; blue, 45 }  ,opacity=1 ]  {$v_{3}$};
        % Connection
        \draw    (138,110) -- (107.6,110) ;
        \draw [shift={(105.6,110)}, rotate = 360] [color={rgb, 255:red, 0; green, 0; blue, 0 }  ][line width=0.75]    (10.93,-3.29) .. controls (6.95,-1.4) and (3.31,-0.3) .. (0,0) .. controls (3.31,0.3) and (6.95,1.4) .. (10.93,3.29)   ;
        
        % Connection
        \draw    (189.4,110) -- (158,110) ;
        \draw [shift={(156,110)}, rotate = 360] [color={rgb, 255:red, 0; green, 0; blue, 0 }  ][line width=0.75]    (10.93,-3.29) .. controls (6.95,-1.4) and (3.31,-0.3) .. (0,0) .. controls (3.31,0.3) and (6.95,1.4) .. (10.93,3.29)   ;
        
        % Connection
        \draw    (309.4,110) -- (278,110) ;
        \draw [shift={(276,110)}, rotate = 360] [color={rgb, 255:red, 0; green, 0; blue, 0 }  ][line width=0.75]    (10.93,-3.29) .. controls (6.95,-1.4) and (3.31,-0.3) .. (0,0) .. controls (3.31,0.3) and (6.95,1.4) .. (10.93,3.29)   ;
        
        % Connection
        \draw    (258,110) -- (218.6,110) ;
        \draw [shift={(216.6,110)}, rotate = 360] [color={rgb, 255:red, 0; green, 0; blue, 0 }  ][line width=0.75]    (10.93,-3.29) .. controls (6.95,-1.4) and (3.31,-0.3) .. (0,0) .. controls (3.31,0.3) and (6.95,1.4) .. (10.93,3.29)   ;
        
        % Connection
        \draw    (309.4,170) -- (278,170) ;
        \draw [shift={(276,170)}, rotate = 360] [color={rgb, 255:red, 0; green, 0; blue, 0 }  ][line width=0.75]    (10.93,-3.29) .. controls (6.95,-1.4) and (3.31,-0.3) .. (0,0) .. controls (3.31,0.3) and (6.95,1.4) .. (10.93,3.29)   ;
        
        % Connection
        \draw    (258,168.07) .. controls (230.15,163.58) and (213.3,149.18) .. (207.42,124.89) ;
        \draw [shift={(206.99,123.01)}, rotate = 437.91] [color={rgb, 255:red, 0; green, 0; blue, 0 }  ][line width=0.75]    (10.93,-3.29) .. controls (6.95,-1.4) and (3.31,-0.3) .. (0,0) .. controls (3.31,0.3) and (6.95,1.4) .. (10.93,3.29)   ;
        
        
        \end{tikzpicture}
        \\
        
        \item Читаем символ '$c$' и ищем в управляющей таблице переходы из состояний 3 и 4 (так как узлы $v_2$ и $v_3$ находятся на одном уровне, то есть были построены после чтения одного символа из входного слова). Таким переходом оказывается $shift\ 6$ в обоих случаях, поэтому соединяем узел $v_4$ с обоими рассмотренными узлами:\\ \\
        Вход: \,
        \begin{tabular}[c]{ |c|c|c|c| } 
            \hline a & b & \textcolor{red}{c} & \$ \\ \hline
        \end{tabular}
        \qquad GSS: \,
        \begin{tikzpicture}[x=0.5pt,y=0.5pt,yscale=-1,xscale=1]
        %uncomment if require: \path (0,422); %set diagram left start at 0, and has height of 422
        
        
        % Text Node
        \draw  [line width=0.75]   (92, 110) circle [x radius= 13.6, y radius= 13.6]   ;
        \draw (92,110) node [color={rgb, 255:red, 62; green, 45; blue, 45 }  ,opacity=1 ]  {$0$};
        % Text Node
        \draw  [line width=0.75]   (138,98) -- (156,98) -- (156,122) -- (138,122) -- cycle  ;
        \draw (147,110) node [scale=1,color={rgb, 255:red, 62; green, 45; blue, 45 }  ,opacity=1 ]  {а};
        % Text Node
        \draw  [line width=0.75]   (203, 110) circle [x radius= 13.6, y radius= 13.6]   ;
        \draw (203,110) node [color={rgb, 255:red, 62; green, 45; blue, 45 }  ,opacity=1 ]  {$2$};
        % Text Node
        \draw (110,89) node [color={rgb, 255:red, 62; green, 45; blue, 45 }  ,opacity=1 ]  {$v_{0}$};
        % Text Node
        \draw (220,89) node [color={rgb, 255:red, 62; green, 45; blue, 45 }  ,opacity=1 ]  {$v_{1}$};
        % Text Node
        \draw  [line width=0.75]   (258,98) -- (276,98) -- (276,122) -- (258,122) -- cycle  ;
        \draw (267,110) node [scale=1,color={rgb, 255:red, 62; green, 45; blue, 45 }  ,opacity=1 ]  {b};
        % Text Node
        \draw  [line width=0.75]   (323, 110) circle [x radius= 13.6, y radius= 13.6]   ;
        \draw (323,110) node [color={rgb, 255:red, 62; green, 45; blue, 45 }  ,opacity=1 ]  {$3$};
        % Text Node
        \draw (340,90) node [color={rgb, 255:red, 62; green, 45; blue, 45 }  ,opacity=1 ]  {$v_{2}$};
        % Text Node
        \draw  [line width=0.75]   (258,158) -- (276,158) -- (276,182) -- (258,182) -- cycle  ;
        \draw (267,170) node [scale=1,color={rgb, 255:red, 62; green, 45; blue, 45 }  ,opacity=1 ]  {B};
        % Text Node
        \draw  [line width=0.75]   (323, 170) circle [x radius= 13.6, y radius= 13.6]   ;
        \draw (323,170) node [color={rgb, 255:red, 62; green, 45; blue, 45 }  ,opacity=1 ]  {$4$};
        % Text Node
        \draw (340,149) node [color={rgb, 255:red, 62; green, 45; blue, 45 }  ,opacity=1 ]  {$v_{3}$};
        % Text Node
        \draw  [line width=0.75]   (374,98) -- (392,98) -- (392,122) -- (374,122) -- cycle  ;
        \draw (383,110) node [scale=1,color={rgb, 255:red, 62; green, 45; blue, 45 }  ,opacity=1 ]  {c};
        % Text Node
        \draw  [line width=0.75]   (374,158) -- (392,158) -- (392,182) -- (374,182) -- cycle  ;
        \draw (383,170) node [scale=1,color={rgb, 255:red, 62; green, 45; blue, 45 }  ,opacity=1 ]  {c};
        % Text Node
        \draw  [line width=0.75]   (437, 110) circle [x radius= 13.6, y radius= 13.6]   ;
        \draw (437,110) node [color={rgb, 255:red, 62; green, 45; blue, 45 }  ,opacity=1 ]  {$6$};
        % Text Node
        \draw (450,89) node [color={rgb, 255:red, 62; green, 45; blue, 45 }  ,opacity=1 ]  {$v_{4}$};
        % Connection
        \draw    (138,110) -- (107.6,110) ;
        \draw [shift={(105.6,110)}, rotate = 360] [color={rgb, 255:red, 0; green, 0; blue, 0 }  ][line width=0.75]    (10.93,-3.29) .. controls (6.95,-1.4) and (3.31,-0.3) .. (0,0) .. controls (3.31,0.3) and (6.95,1.4) .. (10.93,3.29)   ;
        
        % Connection
        \draw    (189.4,110) -- (158,110) ;
        \draw [shift={(156,110)}, rotate = 360] [color={rgb, 255:red, 0; green, 0; blue, 0 }  ][line width=0.75]    (10.93,-3.29) .. controls (6.95,-1.4) and (3.31,-0.3) .. (0,0) .. controls (3.31,0.3) and (6.95,1.4) .. (10.93,3.29)   ;
        
        % Connection
        \draw    (309.4,110) -- (278,110) ;
        \draw [shift={(276,110)}, rotate = 360] [color={rgb, 255:red, 0; green, 0; blue, 0 }  ][line width=0.75]    (10.93,-3.29) .. controls (6.95,-1.4) and (3.31,-0.3) .. (0,0) .. controls (3.31,0.3) and (6.95,1.4) .. (10.93,3.29)   ;
        
        % Connection
        \draw    (258,110) -- (218.6,110) ;
        \draw [shift={(216.6,110)}, rotate = 360] [color={rgb, 255:red, 0; green, 0; blue, 0 }  ][line width=0.75]    (10.93,-3.29) .. controls (6.95,-1.4) and (3.31,-0.3) .. (0,0) .. controls (3.31,0.3) and (6.95,1.4) .. (10.93,3.29)   ;
        
        % Connection
        \draw    (309.4,170) -- (278,170) ;
        \draw [shift={(276,170)}, rotate = 360] [color={rgb, 255:red, 0; green, 0; blue, 0 }  ][line width=0.75]    (10.93,-3.29) .. controls (6.95,-1.4) and (3.31,-0.3) .. (0,0) .. controls (3.31,0.3) and (6.95,1.4) .. (10.93,3.29)   ;
        
        % Connection
        \draw    (258,168.07) .. controls (230.15,163.58) and (213.3,149.18) .. (207.42,124.89) ;
        \draw [shift={(206.99,123.01)}, rotate = 437.91] [color={rgb, 255:red, 0; green, 0; blue, 0 }  ][line width=0.75]    (10.93,-3.29) .. controls (6.95,-1.4) and (3.31,-0.3) .. (0,0) .. controls (3.31,0.3) and (6.95,1.4) .. (10.93,3.29)   ;
        
        % Connection
        \draw    (374,110) -- (338.6,110) ;
        \draw [shift={(336.6,110)}, rotate = 360] [color={rgb, 255:red, 0; green, 0; blue, 0 }  ][line width=0.75]    (10.93,-3.29) .. controls (6.95,-1.4) and (3.31,-0.3) .. (0,0) .. controls (3.31,0.3) and (6.95,1.4) .. (10.93,3.29)   ;
        
        % Connection
        \draw    (423.4,110) -- (394,110) ;
        \draw [shift={(392,110)}, rotate = 360] [color={rgb, 255:red, 0; green, 0; blue, 0 }  ][line width=0.75]    (10.93,-3.29) .. controls (6.95,-1.4) and (3.31,-0.3) .. (0,0) .. controls (3.31,0.3) and (6.95,1.4) .. (10.93,3.29)   ;
        
        % Connection
        \draw    (435.82,123.55) .. controls (435.9,150.31) and (416.89,165.24) .. (393.78,168.34) ;
        \draw [shift={(392,168.55)}, rotate = 353.9] [color={rgb, 255:red, 0; green, 0; blue, 0 }  ][line width=0.75]    (10.93,-3.29) .. controls (6.95,-1.4) and (3.31,-0.3) .. (0,0) .. controls (3.31,0.3) and (6.95,1.4) .. (10.93,3.29)   ;
        
        % Connection
        \draw    (374,170) -- (338.6,170) ;
        \draw [shift={(336.6,170)}, rotate = 360] [color={rgb, 255:red, 0; green, 0; blue, 0 }  ][line width=0.75]    (10.93,-3.29) .. controls (6.95,-1.4) and (3.31,-0.3) .. (0,0) .. controls (3.31,0.3) and (6.95,1.4) .. (10.93,3.29)   ;
        
        
        \end{tikzpicture}
        \\
        
        \item При обработке узла $v_4$ находим соответствующею 6-ому состоянию редукцию по правилу 4. Его правая часть содержит один символ '$c$', 2 вершины-символа с которым достижимы из $v_4$. Находим вершины-состояния, которые смежны с этими вершинами-символами и обрабатываем переходы по левой части правила 4. Такими переходами по нетерминалу $C$ оказываются $goto\ 5$ и $goto\ 7$. Строим соответствующие им вершины $v_5$ и $v_6$:\\ \\
        Вход: \,
        \begin{tabular}[c]{ |c|c|c|c| } 
            \hline a & b & c & \$ \\ \hline
        \end{tabular}
        \qquad GSS: \,
        \begin{tikzpicture}[x=0.5pt,y=0.5pt,yscale=-1,xscale=1]
        %uncomment if require: \path (0,422); %set diagram left start at 0, and has height of 422
        
        
        % Text Node
        \draw  [line width=0.75]   (92, 110) circle [x radius= 13.6, y radius= 13.6]   ;
        \draw (92,110) node [color={rgb, 255:red, 62; green, 45; blue, 45 }  ,opacity=1 ]  {$0$};
        % Text Node
        \draw  [line width=0.75]   (138,98) -- (156,98) -- (156,122) -- (138,122) -- cycle  ;
        \draw (147,110) node [scale=1,color={rgb, 255:red, 62; green, 45; blue, 45 }  ,opacity=1 ]  {а};
        % Text Node
        \draw  [line width=0.75]   (203, 110) circle [x radius= 13.6, y radius= 13.6]   ;
        \draw (203,110) node [color={rgb, 255:red, 62; green, 45; blue, 45 }  ,opacity=1 ]  {$2$};
        % Text Node
        \draw (110,89) node [color={rgb, 255:red, 62; green, 45; blue, 45 }  ,opacity=1 ]  {$v_{0}$};
        % Text Node
        \draw (220,89) node [color={rgb, 255:red, 62; green, 45; blue, 45 }  ,opacity=1 ]  {$v_{1}$};
        % Text Node
        \draw  [line width=0.75]   (258,98) -- (276,98) -- (276,122) -- (258,122) -- cycle  ;
        \draw (267,110) node [scale=1,color={rgb, 255:red, 62; green, 45; blue, 45 }  ,opacity=1 ]  {b};
        % Text Node
        \draw  [line width=0.75]   (323, 110) circle [x radius= 13.6, y radius= 13.6]   ;
        \draw (323,110) node [color={rgb, 255:red, 62; green, 45; blue, 45 }  ,opacity=1 ]  {$3$};
        % Text Node
        \draw (340,90) node [color={rgb, 255:red, 62; green, 45; blue, 45 }  ,opacity=1 ]  {$v_{2}$};
        % Text Node
        \draw  [line width=0.75]   (258,158) -- (276,158) -- (276,182) -- (258,182) -- cycle  ;
        \draw (267,170) node [scale=1,color={rgb, 255:red, 62; green, 45; blue, 45 }  ,opacity=1 ]  {B};
        % Text Node
        \draw  [line width=0.75]   (323, 170) circle [x radius= 13.6, y radius= 13.6]   ;
        \draw (323,170) node [color={rgb, 255:red, 62; green, 45; blue, 45 }  ,opacity=1 ]  {$4$};
        % Text Node
        \draw (340,149) node [color={rgb, 255:red, 62; green, 45; blue, 45 }  ,opacity=1 ]  {$v_{3}$};
        % Text Node
        \draw  [line width=0.75]   (374,98) -- (392,98) -- (392,122) -- (374,122) -- cycle  ;
        \draw (383,110) node [scale=1,color={rgb, 255:red, 62; green, 45; blue, 45 }  ,opacity=1 ]  {c};
        % Text Node
        \draw  [line width=0.75]   (374,158) -- (392,158) -- (392,182) -- (374,182) -- cycle  ;
        \draw (383,170) node [scale=1,color={rgb, 255:red, 62; green, 45; blue, 45 }  ,opacity=1 ]  {c};
        % Text Node
        \draw  [line width=0.75]   (437, 110) circle [x radius= 13.6, y radius= 13.6]   ;
        \draw (437,110) node [color={rgb, 255:red, 62; green, 45; blue, 45 }  ,opacity=1 ]  {$6$};
        % Text Node
        \draw (450,89) node [color={rgb, 255:red, 62; green, 45; blue, 45 }  ,opacity=1 ]  {$v_{4}$};
        % Text Node
        \draw  [line width=0.75]   (372,38) -- (392,38) -- (392,62) -- (372,62) -- cycle  ;
        \draw (382,50) node [scale=1,color={rgb, 255:red, 62; green, 45; blue, 45 }  ,opacity=1 ]  {C};
        % Text Node
        \draw  [line width=0.75]   (372,218) -- (392,218) -- (392,242) -- (372,242) -- cycle  ;
        \draw (382,230) node [scale=1,color={rgb, 255:red, 62; green, 45; blue, 45 }  ,opacity=1 ]  {C};
        % Text Node
        \draw  [line width=0.75]   (437, 50) circle [x radius= 13.6, y radius= 13.6]   ;
        \draw (437,50) node [color={rgb, 255:red, 62; green, 45; blue, 45 }  ,opacity=1 ]  {$5$};
        % Text Node
        \draw  [line width=0.75]   (437, 231) circle [x radius= 13.6, y radius= 13.6]   ;
        \draw (437,231) node [color={rgb, 255:red, 62; green, 45; blue, 45 }  ,opacity=1 ]  {$7$};
        % Text Node
        \draw (452,28) node [color={rgb, 255:red, 62; green, 45; blue, 45 }  ,opacity=1 ]  {$v_{5}$};
        % Text Node
        \draw (452,208) node [color={rgb, 255:red, 62; green, 45; blue, 45 }  ,opacity=1 ]  {$v_{6}$};
        % Connection
        \draw    (138,110) -- (107.6,110) ;
        \draw [shift={(105.6,110)}, rotate = 360] [color={rgb, 255:red, 0; green, 0; blue, 0 }  ][line width=0.75]    (10.93,-3.29) .. controls (6.95,-1.4) and (3.31,-0.3) .. (0,0) .. controls (3.31,0.3) and (6.95,1.4) .. (10.93,3.29)   ;
        
        % Connection
        \draw    (189.4,110) -- (158,110) ;
        \draw [shift={(156,110)}, rotate = 360] [color={rgb, 255:red, 0; green, 0; blue, 0 }  ][line width=0.75]    (10.93,-3.29) .. controls (6.95,-1.4) and (3.31,-0.3) .. (0,0) .. controls (3.31,0.3) and (6.95,1.4) .. (10.93,3.29)   ;
        
        % Connection
        \draw    (309.4,110) -- (278,110) ;
        \draw [shift={(276,110)}, rotate = 360] [color={rgb, 255:red, 0; green, 0; blue, 0 }  ][line width=0.75]    (10.93,-3.29) .. controls (6.95,-1.4) and (3.31,-0.3) .. (0,0) .. controls (3.31,0.3) and (6.95,1.4) .. (10.93,3.29)   ;
        
        % Connection
        \draw    (258,110) -- (218.6,110) ;
        \draw [shift={(216.6,110)}, rotate = 360] [color={rgb, 255:red, 0; green, 0; blue, 0 }  ][line width=0.75]    (10.93,-3.29) .. controls (6.95,-1.4) and (3.31,-0.3) .. (0,0) .. controls (3.31,0.3) and (6.95,1.4) .. (10.93,3.29)   ;
        
        % Connection
        \draw    (309.4,170) -- (278,170) ;
        \draw [shift={(276,170)}, rotate = 360] [color={rgb, 255:red, 0; green, 0; blue, 0 }  ][line width=0.75]    (10.93,-3.29) .. controls (6.95,-1.4) and (3.31,-0.3) .. (0,0) .. controls (3.31,0.3) and (6.95,1.4) .. (10.93,3.29)   ;
        
        % Connection
        \draw    (258,168.07) .. controls (230.15,163.58) and (213.3,149.18) .. (207.42,124.89) ;
        \draw [shift={(206.99,123.01)}, rotate = 437.91] [color={rgb, 255:red, 0; green, 0; blue, 0 }  ][line width=0.75]    (10.93,-3.29) .. controls (6.95,-1.4) and (3.31,-0.3) .. (0,0) .. controls (3.31,0.3) and (6.95,1.4) .. (10.93,3.29)   ;
        
        % Connection
        \draw    (374,110) -- (338.6,110) ;
        \draw [shift={(336.6,110)}, rotate = 360] [color={rgb, 255:red, 0; green, 0; blue, 0 }  ][line width=0.75]    (10.93,-3.29) .. controls (6.95,-1.4) and (3.31,-0.3) .. (0,0) .. controls (3.31,0.3) and (6.95,1.4) .. (10.93,3.29)   ;
        
        % Connection
        \draw    (423.4,110) -- (394,110) ;
        \draw [shift={(392,110)}, rotate = 360] [color={rgb, 255:red, 0; green, 0; blue, 0 }  ][line width=0.75]    (10.93,-3.29) .. controls (6.95,-1.4) and (3.31,-0.3) .. (0,0) .. controls (3.31,0.3) and (6.95,1.4) .. (10.93,3.29)   ;
        
        % Connection
        \draw    (435.82,123.55) .. controls (435.9,150.31) and (416.89,165.24) .. (393.78,168.34) ;
        \draw [shift={(392,168.55)}, rotate = 353.9] [color={rgb, 255:red, 0; green, 0; blue, 0 }  ][line width=0.75]    (10.93,-3.29) .. controls (6.95,-1.4) and (3.31,-0.3) .. (0,0) .. controls (3.31,0.3) and (6.95,1.4) .. (10.93,3.29)   ;
        
        % Connection
        \draw    (374,170) -- (338.6,170) ;
        \draw [shift={(336.6,170)}, rotate = 360] [color={rgb, 255:red, 0; green, 0; blue, 0 }  ][line width=0.75]    (10.93,-3.29) .. controls (6.95,-1.4) and (3.31,-0.3) .. (0,0) .. controls (3.31,0.3) and (6.95,1.4) .. (10.93,3.29)   ;
        
        % Connection
        \draw    (372,50.52) .. controls (341.88,50.18) and (326.06,64.88) .. (324.53,94.64) ;
        \draw [shift={(324.46,96.48)}, rotate = 271.78] [color={rgb, 255:red, 0; green, 0; blue, 0 }  ][line width=0.75]    (10.93,-3.29) .. controls (6.95,-1.4) and (3.31,-0.3) .. (0,0) .. controls (3.31,0.3) and (6.95,1.4) .. (10.93,3.29)   ;
        
        % Connection
        \draw    (372,229.65) .. controls (341.88,230.53) and (326.05,215.79) .. (324.5,185.41) ;
        \draw [shift={(324.42,183.53)}, rotate = 448.21] [color={rgb, 255:red, 0; green, 0; blue, 0 }  ][line width=0.75]    (10.93,-3.29) .. controls (6.95,-1.4) and (3.31,-0.3) .. (0,0) .. controls (3.31,0.3) and (6.95,1.4) .. (10.93,3.29)   ;
        
        % Connection
        \draw    (423.4,50) -- (394,50) ;
        \draw [shift={(392,50)}, rotate = 360] [color={rgb, 255:red, 0; green, 0; blue, 0 }  ][line width=0.75]    (10.93,-3.29) .. controls (6.95,-1.4) and (3.31,-0.3) .. (0,0) .. controls (3.31,0.3) and (6.95,1.4) .. (10.93,3.29)   ;
        
        % Connection
        \draw    (423.4,230.75) -- (394,230.22) ;
        \draw [shift={(392,230.18)}, rotate = 361.03999999999996] [color={rgb, 255:red, 0; green, 0; blue, 0 }  ][line width=0.75]    (10.93,-3.29) .. controls (6.95,-1.4) and (3.31,-0.3) .. (0,0) .. controls (3.31,0.3) and (6.95,1.4) .. (10.93,3.29)   ;
        
        
        \end{tikzpicture}
        \\
        
        \item При обработке узлов $v_5$ и $v_6$ находим редукции с символом '$S$' в левой части и тремя символами в правой. Возвращаемся на 3 вершины-состояния назад и строим вершину $v_7$ с переходом по $S$: \\ \\
        Вход: \,
        \begin{tabular}[c]{ |c|c|c|c| } 
            \hline a & b & c & \$ \\ \hline
        \end{tabular}
        \qquad GSS: \,
        \begin{tikzpicture}[x=0.5pt,y=0.5pt,yscale=-1,xscale=1]
        %uncomment if require: \path (0,422); %set diagram left start at 0, and has height of 422
        
        
        % Text Node
        \draw  [line width=0.75]   (92, 110) circle [x radius= 13.6, y radius= 13.6]   ;
        \draw (92,110) node [color={rgb, 255:red, 62; green, 45; blue, 45 }  ,opacity=1 ]  {$0$};
        % Text Node
        \draw  [line width=0.75]   (138,98) -- (156,98) -- (156,122) -- (138,122) -- cycle  ;
        \draw (147,110) node [scale=1,color={rgb, 255:red, 62; green, 45; blue, 45 }  ,opacity=1 ]  {а};
        % Text Node
        \draw  [line width=0.75]   (203, 110) circle [x radius= 13.6, y radius= 13.6]   ;
        \draw (203,110) node [color={rgb, 255:red, 62; green, 45; blue, 45 }  ,opacity=1 ]  {$2$};
        % Text Node
        \draw (110,89) node [color={rgb, 255:red, 62; green, 45; blue, 45 }  ,opacity=1 ]  {$v_{0}$};
        % Text Node
        \draw (220,89) node [color={rgb, 255:red, 62; green, 45; blue, 45 }  ,opacity=1 ]  {$v_{1}$};
        % Text Node
        \draw  [line width=0.75]   (258,98) -- (276,98) -- (276,122) -- (258,122) -- cycle  ;
        \draw (267,110) node [scale=1,color={rgb, 255:red, 62; green, 45; blue, 45 }  ,opacity=1 ]  {b};
        % Text Node
        \draw  [line width=0.75]   (323, 110) circle [x radius= 13.6, y radius= 13.6]   ;
        \draw (323,110) node [color={rgb, 255:red, 62; green, 45; blue, 45 }  ,opacity=1 ]  {$3$};
        % Text Node
        \draw (340,90) node [color={rgb, 255:red, 62; green, 45; blue, 45 }  ,opacity=1 ]  {$v_{2}$};
        % Text Node
        \draw  [line width=0.75]   (258,158) -- (276,158) -- (276,182) -- (258,182) -- cycle  ;
        \draw (267,170) node [scale=1,color={rgb, 255:red, 62; green, 45; blue, 45 }  ,opacity=1 ]  {B};
        % Text Node
        \draw  [line width=0.75]   (323, 170) circle [x radius= 13.6, y radius= 13.6]   ;
        \draw (323,170) node [color={rgb, 255:red, 62; green, 45; blue, 45 }  ,opacity=1 ]  {$4$};
        % Text Node
        \draw (340,149) node [color={rgb, 255:red, 62; green, 45; blue, 45 }  ,opacity=1 ]  {$v_{3}$};
        % Text Node
        \draw  [line width=0.75]   (374,98) -- (392,98) -- (392,122) -- (374,122) -- cycle  ;
        \draw (383,110) node [scale=1,color={rgb, 255:red, 62; green, 45; blue, 45 }  ,opacity=1 ]  {c};
        % Text Node
        \draw  [line width=0.75]   (374,158) -- (392,158) -- (392,182) -- (374,182) -- cycle  ;
        \draw (383,170) node [scale=1,color={rgb, 255:red, 62; green, 45; blue, 45 }  ,opacity=1 ]  {c};
        % Text Node
        \draw  [line width=0.75]   (437, 110) circle [x radius= 13.6, y radius= 13.6]   ;
        \draw (437,110) node [color={rgb, 255:red, 62; green, 45; blue, 45 }  ,opacity=1 ]  {$6$};
        % Text Node
        \draw (450,89) node [color={rgb, 255:red, 62; green, 45; blue, 45 }  ,opacity=1 ]  {$v_{4}$};
        % Text Node
        \draw  [line width=0.75]   (372,38) -- (392,38) -- (392,62) -- (372,62) -- cycle  ;
        \draw (382,50) node [scale=1,color={rgb, 255:red, 62; green, 45; blue, 45 }  ,opacity=1 ]  {C};
        % Text Node
        \draw  [line width=0.75]   (372,218) -- (392,218) -- (392,242) -- (372,242) -- cycle  ;
        \draw (382,230) node [scale=1,color={rgb, 255:red, 62; green, 45; blue, 45 }  ,opacity=1 ]  {C};
        % Text Node
        \draw  [line width=0.75]   (437, 50) circle [x radius= 13.6, y radius= 13.6]   ;
        \draw (437,50) node [color={rgb, 255:red, 62; green, 45; blue, 45 }  ,opacity=1 ]  {$5$};
        % Text Node
        \draw  [line width=0.75]   (437, 231) circle [x radius= 13.6, y radius= 13.6]   ;
        \draw (437,231) node [color={rgb, 255:red, 62; green, 45; blue, 45 }  ,opacity=1 ]  {$7$};
        % Text Node
        \draw (452,28) node [color={rgb, 255:red, 62; green, 45; blue, 45 }  ,opacity=1 ]  {$v_{5}$};
        % Text Node
        \draw (452,208) node [color={rgb, 255:red, 62; green, 45; blue, 45 }  ,opacity=1 ]  {$v_{6}$};
        % Text Node
        \draw  [line width=0.75]   (373,278) -- (391,278) -- (391,302) -- (373,302) -- cycle  ;
        \draw (382,290) node [scale=1,color={rgb, 255:red, 62; green, 45; blue, 45 }  ,opacity=1 ]  {S};
        % Text Node
        \draw  [line width=0.75]   (437, 290) circle [x radius= 13.6, y radius= 13.6]   ;
        \draw (437,290) node [color={rgb, 255:red, 62; green, 45; blue, 45 }  ,opacity=1 ]  {$1$};
        % Text Node
        \draw (452,268) node [color={rgb, 255:red, 62; green, 45; blue, 45 }  ,opacity=1 ]  {$v_{7}$};
        % Connection
        \draw    (138,110) -- (107.6,110) ;
        \draw [shift={(105.6,110)}, rotate = 360] [color={rgb, 255:red, 0; green, 0; blue, 0 }  ][line width=0.75]    (10.93,-3.29) .. controls (6.95,-1.4) and (3.31,-0.3) .. (0,0) .. controls (3.31,0.3) and (6.95,1.4) .. (10.93,3.29)   ;
        
        % Connection
        \draw    (189.4,110) -- (158,110) ;
        \draw [shift={(156,110)}, rotate = 360] [color={rgb, 255:red, 0; green, 0; blue, 0 }  ][line width=0.75]    (10.93,-3.29) .. controls (6.95,-1.4) and (3.31,-0.3) .. (0,0) .. controls (3.31,0.3) and (6.95,1.4) .. (10.93,3.29)   ;
        
        % Connection
        \draw    (309.4,110) -- (278,110) ;
        \draw [shift={(276,110)}, rotate = 360] [color={rgb, 255:red, 0; green, 0; blue, 0 }  ][line width=0.75]    (10.93,-3.29) .. controls (6.95,-1.4) and (3.31,-0.3) .. (0,0) .. controls (3.31,0.3) and (6.95,1.4) .. (10.93,3.29)   ;
        
        % Connection
        \draw    (258,110) -- (218.6,110) ;
        \draw [shift={(216.6,110)}, rotate = 360] [color={rgb, 255:red, 0; green, 0; blue, 0 }  ][line width=0.75]    (10.93,-3.29) .. controls (6.95,-1.4) and (3.31,-0.3) .. (0,0) .. controls (3.31,0.3) and (6.95,1.4) .. (10.93,3.29)   ;
        
        % Connection
        \draw    (309.4,170) -- (278,170) ;
        \draw [shift={(276,170)}, rotate = 360] [color={rgb, 255:red, 0; green, 0; blue, 0 }  ][line width=0.75]    (10.93,-3.29) .. controls (6.95,-1.4) and (3.31,-0.3) .. (0,0) .. controls (3.31,0.3) and (6.95,1.4) .. (10.93,3.29)   ;
        
        % Connection
        \draw    (258,168.07) .. controls (230.15,163.58) and (213.3,149.18) .. (207.42,124.89) ;
        \draw [shift={(206.99,123.01)}, rotate = 437.91] [color={rgb, 255:red, 0; green, 0; blue, 0 }  ][line width=0.75]    (10.93,-3.29) .. controls (6.95,-1.4) and (3.31,-0.3) .. (0,0) .. controls (3.31,0.3) and (6.95,1.4) .. (10.93,3.29)   ;
        
        % Connection
        \draw    (374,110) -- (338.6,110) ;
        \draw [shift={(336.6,110)}, rotate = 360] [color={rgb, 255:red, 0; green, 0; blue, 0 }  ][line width=0.75]    (10.93,-3.29) .. controls (6.95,-1.4) and (3.31,-0.3) .. (0,0) .. controls (3.31,0.3) and (6.95,1.4) .. (10.93,3.29)   ;
        
        % Connection
        \draw    (423.4,110) -- (394,110) ;
        \draw [shift={(392,110)}, rotate = 360] [color={rgb, 255:red, 0; green, 0; blue, 0 }  ][line width=0.75]    (10.93,-3.29) .. controls (6.95,-1.4) and (3.31,-0.3) .. (0,0) .. controls (3.31,0.3) and (6.95,1.4) .. (10.93,3.29)   ;
        
        % Connection
        \draw    (435.82,123.55) .. controls (435.9,150.31) and (416.89,165.24) .. (393.78,168.34) ;
        \draw [shift={(392,168.55)}, rotate = 353.9] [color={rgb, 255:red, 0; green, 0; blue, 0 }  ][line width=0.75]    (10.93,-3.29) .. controls (6.95,-1.4) and (3.31,-0.3) .. (0,0) .. controls (3.31,0.3) and (6.95,1.4) .. (10.93,3.29)   ;
        
        % Connection
        \draw    (374,170) -- (338.6,170) ;
        \draw [shift={(336.6,170)}, rotate = 360] [color={rgb, 255:red, 0; green, 0; blue, 0 }  ][line width=0.75]    (10.93,-3.29) .. controls (6.95,-1.4) and (3.31,-0.3) .. (0,0) .. controls (3.31,0.3) and (6.95,1.4) .. (10.93,3.29)   ;
        
        % Connection
        \draw    (372,50.52) .. controls (341.88,50.18) and (326.06,64.88) .. (324.53,94.64) ;
        \draw [shift={(324.46,96.48)}, rotate = 271.78] [color={rgb, 255:red, 0; green, 0; blue, 0 }  ][line width=0.75]    (10.93,-3.29) .. controls (6.95,-1.4) and (3.31,-0.3) .. (0,0) .. controls (3.31,0.3) and (6.95,1.4) .. (10.93,3.29)   ;
        
        % Connection
        \draw    (372,229.65) .. controls (341.88,230.53) and (326.05,215.79) .. (324.5,185.41) ;
        \draw [shift={(324.42,183.53)}, rotate = 448.21] [color={rgb, 255:red, 0; green, 0; blue, 0 }  ][line width=0.75]    (10.93,-3.29) .. controls (6.95,-1.4) and (3.31,-0.3) .. (0,0) .. controls (3.31,0.3) and (6.95,1.4) .. (10.93,3.29)   ;
        
        % Connection
        \draw    (423.4,50) -- (394,50) ;
        \draw [shift={(392,50)}, rotate = 360] [color={rgb, 255:red, 0; green, 0; blue, 0 }  ][line width=0.75]    (10.93,-3.29) .. controls (6.95,-1.4) and (3.31,-0.3) .. (0,0) .. controls (3.31,0.3) and (6.95,1.4) .. (10.93,3.29)   ;
        
        % Connection
        \draw    (423.4,230.75) -- (394,230.22) ;
        \draw [shift={(392,230.18)}, rotate = 361.03999999999996] [color={rgb, 255:red, 0; green, 0; blue, 0 }  ][line width=0.75]    (10.93,-3.29) .. controls (6.95,-1.4) and (3.31,-0.3) .. (0,0) .. controls (3.31,0.3) and (6.95,1.4) .. (10.93,3.29)   ;
        
        % Connection
        \draw    (423.4,290) -- (393,290) ;
        \draw [shift={(391,290)}, rotate = 360] [color={rgb, 255:red, 0; green, 0; blue, 0 }  ][line width=0.75]    (10.93,-3.29) .. controls (6.95,-1.4) and (3.31,-0.3) .. (0,0) .. controls (3.31,0.3) and (6.95,1.4) .. (10.93,3.29)   ;
        
        % Connection
        \draw    (373,289.72) .. controls (234.96,286.59) and (143.34,231.38) .. (98.15,124.08) ;
        \draw [shift={(97.47,122.46)}, rotate = 427.47] [color={rgb, 255:red, 0; green, 0; blue, 0 }  ][line width=0.75]    (10.93,-3.29) .. controls (6.95,-1.4) and (3.31,-0.3) .. (0,0) .. controls (3.31,0.3) and (6.95,1.4) .. (10.93,3.29)   ;
        
        
        \end{tikzpicture}
        \\
        \item Наконец, обрабатывая вершину $v_7$, читаем символ '$\$$' и строим узел $v_8$, который соответствует допускающим состоянием: \\ \\
        Вход: \,
        \begin{tabular}[c]{ |c|c|c|c| } 
            \hline a & b & c & \textcolor{red}{\$} \\ \hline
        \end{tabular}
        \qquad GSS: \,
        \begin{tikzpicture}[x=0.5pt,y=0.5pt,yscale=-1,xscale=1]
        %uncomment if require: \path (0,422); %set diagram left start at 0, and has height of 422
        
        
        % Text Node
        \draw  [line width=0.75]   (92, 110) circle [x radius= 13.6, y radius= 13.6]   ;
        \draw (92,110) node [color={rgb, 255:red, 62; green, 45; blue, 45 }  ,opacity=1 ]  {$0$};
        % Text Node
        \draw  [line width=0.75]   (138,98) -- (156,98) -- (156,122) -- (138,122) -- cycle  ;
        \draw (147,110) node [scale=1,color={rgb, 255:red, 62; green, 45; blue, 45 }  ,opacity=1 ]  {а};
        % Text Node
        \draw  [line width=0.75]   (203, 110) circle [x radius= 13.6, y radius= 13.6]   ;
        \draw (203,110) node [color={rgb, 255:red, 62; green, 45; blue, 45 }  ,opacity=1 ]  {$2$};
        % Text Node
        \draw (110,89) node [color={rgb, 255:red, 62; green, 45; blue, 45 }  ,opacity=1 ]  {$v_{0}$};
        % Text Node
        \draw (220,89) node [color={rgb, 255:red, 62; green, 45; blue, 45 }  ,opacity=1 ]  {$v_{1}$};
        % Text Node
        \draw  [line width=0.75]   (258,98) -- (276,98) -- (276,122) -- (258,122) -- cycle  ;
        \draw (267,110) node [scale=1,color={rgb, 255:red, 62; green, 45; blue, 45 }  ,opacity=1 ]  {b};
        % Text Node
        \draw  [line width=0.75]   (323, 110) circle [x radius= 13.6, y radius= 13.6]   ;
        \draw (323,110) node [color={rgb, 255:red, 62; green, 45; blue, 45 }  ,opacity=1 ]  {$3$};
        % Text Node
        \draw (340,90) node [color={rgb, 255:red, 62; green, 45; blue, 45 }  ,opacity=1 ]  {$v_{2}$};
        % Text Node
        \draw  [line width=0.75]   (258,158) -- (276,158) -- (276,182) -- (258,182) -- cycle  ;
        \draw (267,170) node [scale=1,color={rgb, 255:red, 62; green, 45; blue, 45 }  ,opacity=1 ]  {B};
        % Text Node
        \draw  [line width=0.75]   (323, 170) circle [x radius= 13.6, y radius= 13.6]   ;
        \draw (323,170) node [color={rgb, 255:red, 62; green, 45; blue, 45 }  ,opacity=1 ]  {$4$};
        % Text Node
        \draw (340,149) node [color={rgb, 255:red, 62; green, 45; blue, 45 }  ,opacity=1 ]  {$v_{3}$};
        % Text Node
        \draw  [line width=0.75]   (374,98) -- (392,98) -- (392,122) -- (374,122) -- cycle  ;
        \draw (383,110) node [scale=1,color={rgb, 255:red, 62; green, 45; blue, 45 }  ,opacity=1 ]  {c};
        % Text Node
        \draw  [line width=0.75]   (374,158) -- (392,158) -- (392,182) -- (374,182) -- cycle  ;
        \draw (383,170) node [scale=1,color={rgb, 255:red, 62; green, 45; blue, 45 }  ,opacity=1 ]  {c};
        % Text Node
        \draw  [line width=0.75]   (437, 110) circle [x radius= 13.6, y radius= 13.6]   ;
        \draw (437,110) node [color={rgb, 255:red, 62; green, 45; blue, 45 }  ,opacity=1 ]  {$6$};
        % Text Node
        \draw (450,89) node [color={rgb, 255:red, 62; green, 45; blue, 45 }  ,opacity=1 ]  {$v_{4}$};
        % Text Node
        \draw  [line width=0.75]   (372,38) -- (392,38) -- (392,62) -- (372,62) -- cycle  ;
        \draw (382,50) node [scale=1,color={rgb, 255:red, 62; green, 45; blue, 45 }  ,opacity=1 ]  {C};
        % Text Node
        \draw  [line width=0.75]   (372,218) -- (392,218) -- (392,242) -- (372,242) -- cycle  ;
        \draw (382,230) node [scale=1,color={rgb, 255:red, 62; green, 45; blue, 45 }  ,opacity=1 ]  {C};
        % Text Node
        \draw  [line width=0.75]   (437, 50) circle [x radius= 13.6, y radius= 13.6]   ;
        \draw (437,50) node [color={rgb, 255:red, 62; green, 45; blue, 45 }  ,opacity=1 ]  {$5$};
        % Text Node
        \draw  [line width=0.75]   (437, 231) circle [x radius= 13.6, y radius= 13.6]   ;
        \draw (437,231) node [color={rgb, 255:red, 62; green, 45; blue, 45 }  ,opacity=1 ]  {$7$};
        % Text Node
        \draw (452,28) node [color={rgb, 255:red, 62; green, 45; blue, 45 }  ,opacity=1 ]  {$v_{5}$};
        % Text Node
        \draw (452,208) node [color={rgb, 255:red, 62; green, 45; blue, 45 }  ,opacity=1 ]  {$v_{6}$};
        % Text Node
        \draw  [line width=0.75]   (373,278) -- (391,278) -- (391,302) -- (373,302) -- cycle  ;
        \draw (382,290) node [scale=1,color={rgb, 255:red, 62; green, 45; blue, 45 }  ,opacity=1 ]  {S};
        % Text Node
        \draw  [line width=0.75]   (437, 290) circle [x radius= 13.6, y radius= 13.6]   ;
        \draw (437,290) node [color={rgb, 255:red, 62; green, 45; blue, 45 }  ,opacity=1 ]  {$1$};
        % Text Node
        \draw (452,268) node [color={rgb, 255:red, 62; green, 45; blue, 45 }  ,opacity=1 ]  {$v_{7}$};
        % Text Node
        \draw  [line width=0.75]   (488,278) -- (506,278) -- (506,302) -- (488,302) -- cycle  ;
        \draw (497,290) node [scale=1,color={rgb, 255:red, 62; green, 45; blue, 45 }  ,opacity=1 ]  {\$};
        % Text Node
        \draw  [line width=0.75]   (557, 290) circle [x radius= 17.8, y radius= 17.8]   ;
        \draw (557,290) node [color={rgb, 255:red, 62; green, 45; blue, 45 }  ,opacity=1 ]  {acc};
        % Text Node
        \draw (575,262) node [color={rgb, 255:red, 62; green, 45; blue, 45 }  ,opacity=1 ]  {$v_{8}$};
        % Connection
        \draw    (138,110) -- (107.6,110) ;
        \draw [shift={(105.6,110)}, rotate = 360] [color={rgb, 255:red, 0; green, 0; blue, 0 }  ][line width=0.75]    (10.93,-3.29) .. controls (6.95,-1.4) and (3.31,-0.3) .. (0,0) .. controls (3.31,0.3) and (6.95,1.4) .. (10.93,3.29)   ;
        
        % Connection
        \draw    (189.4,110) -- (158,110) ;
        \draw [shift={(156,110)}, rotate = 360] [color={rgb, 255:red, 0; green, 0; blue, 0 }  ][line width=0.75]    (10.93,-3.29) .. controls (6.95,-1.4) and (3.31,-0.3) .. (0,0) .. controls (3.31,0.3) and (6.95,1.4) .. (10.93,3.29)   ;
        
        % Connection
        \draw    (309.4,110) -- (278,110) ;
        \draw [shift={(276,110)}, rotate = 360] [color={rgb, 255:red, 0; green, 0; blue, 0 }  ][line width=0.75]    (10.93,-3.29) .. controls (6.95,-1.4) and (3.31,-0.3) .. (0,0) .. controls (3.31,0.3) and (6.95,1.4) .. (10.93,3.29)   ;
        
        % Connection
        \draw    (258,110) -- (218.6,110) ;
        \draw [shift={(216.6,110)}, rotate = 360] [color={rgb, 255:red, 0; green, 0; blue, 0 }  ][line width=0.75]    (10.93,-3.29) .. controls (6.95,-1.4) and (3.31,-0.3) .. (0,0) .. controls (3.31,0.3) and (6.95,1.4) .. (10.93,3.29)   ;
        
        % Connection
        \draw    (309.4,170) -- (278,170) ;
        \draw [shift={(276,170)}, rotate = 360] [color={rgb, 255:red, 0; green, 0; blue, 0 }  ][line width=0.75]    (10.93,-3.29) .. controls (6.95,-1.4) and (3.31,-0.3) .. (0,0) .. controls (3.31,0.3) and (6.95,1.4) .. (10.93,3.29)   ;
        
        % Connection
        \draw    (258,168.07) .. controls (230.15,163.58) and (213.3,149.18) .. (207.42,124.89) ;
        \draw [shift={(206.99,123.01)}, rotate = 437.91] [color={rgb, 255:red, 0; green, 0; blue, 0 }  ][line width=0.75]    (10.93,-3.29) .. controls (6.95,-1.4) and (3.31,-0.3) .. (0,0) .. controls (3.31,0.3) and (6.95,1.4) .. (10.93,3.29)   ;
        
        % Connection
        \draw    (374,110) -- (338.6,110) ;
        \draw [shift={(336.6,110)}, rotate = 360] [color={rgb, 255:red, 0; green, 0; blue, 0 }  ][line width=0.75]    (10.93,-3.29) .. controls (6.95,-1.4) and (3.31,-0.3) .. (0,0) .. controls (3.31,0.3) and (6.95,1.4) .. (10.93,3.29)   ;
        
        % Connection
        \draw    (423.4,110) -- (394,110) ;
        \draw [shift={(392,110)}, rotate = 360] [color={rgb, 255:red, 0; green, 0; blue, 0 }  ][line width=0.75]    (10.93,-3.29) .. controls (6.95,-1.4) and (3.31,-0.3) .. (0,0) .. controls (3.31,0.3) and (6.95,1.4) .. (10.93,3.29)   ;
        
        % Connection
        \draw    (435.82,123.55) .. controls (435.9,150.31) and (416.89,165.24) .. (393.78,168.34) ;
        \draw [shift={(392,168.55)}, rotate = 353.9] [color={rgb, 255:red, 0; green, 0; blue, 0 }  ][line width=0.75]    (10.93,-3.29) .. controls (6.95,-1.4) and (3.31,-0.3) .. (0,0) .. controls (3.31,0.3) and (6.95,1.4) .. (10.93,3.29)   ;
        
        % Connection
        \draw    (374,170) -- (338.6,170) ;
        \draw [shift={(336.6,170)}, rotate = 360] [color={rgb, 255:red, 0; green, 0; blue, 0 }  ][line width=0.75]    (10.93,-3.29) .. controls (6.95,-1.4) and (3.31,-0.3) .. (0,0) .. controls (3.31,0.3) and (6.95,1.4) .. (10.93,3.29)   ;
        
        % Connection
        \draw    (372,50.52) .. controls (341.88,50.18) and (326.06,64.88) .. (324.53,94.64) ;
        \draw [shift={(324.46,96.48)}, rotate = 271.78] [color={rgb, 255:red, 0; green, 0; blue, 0 }  ][line width=0.75]    (10.93,-3.29) .. controls (6.95,-1.4) and (3.31,-0.3) .. (0,0) .. controls (3.31,0.3) and (6.95,1.4) .. (10.93,3.29)   ;
        
        % Connection
        \draw    (372,229.65) .. controls (341.88,230.53) and (326.05,215.79) .. (324.5,185.41) ;
        \draw [shift={(324.42,183.53)}, rotate = 448.21] [color={rgb, 255:red, 0; green, 0; blue, 0 }  ][line width=0.75]    (10.93,-3.29) .. controls (6.95,-1.4) and (3.31,-0.3) .. (0,0) .. controls (3.31,0.3) and (6.95,1.4) .. (10.93,3.29)   ;
        
        % Connection
        \draw    (423.4,50) -- (394,50) ;
        \draw [shift={(392,50)}, rotate = 360] [color={rgb, 255:red, 0; green, 0; blue, 0 }  ][line width=0.75]    (10.93,-3.29) .. controls (6.95,-1.4) and (3.31,-0.3) .. (0,0) .. controls (3.31,0.3) and (6.95,1.4) .. (10.93,3.29)   ;
        
        % Connection
        \draw    (423.4,230.75) -- (394,230.22) ;
        \draw [shift={(392,230.18)}, rotate = 361.03999999999996] [color={rgb, 255:red, 0; green, 0; blue, 0 }  ][line width=0.75]    (10.93,-3.29) .. controls (6.95,-1.4) and (3.31,-0.3) .. (0,0) .. controls (3.31,0.3) and (6.95,1.4) .. (10.93,3.29)   ;
        
        % Connection
        \draw    (423.4,290) -- (393,290) ;
        \draw [shift={(391,290)}, rotate = 360] [color={rgb, 255:red, 0; green, 0; blue, 0 }  ][line width=0.75]    (10.93,-3.29) .. controls (6.95,-1.4) and (3.31,-0.3) .. (0,0) .. controls (3.31,0.3) and (6.95,1.4) .. (10.93,3.29)   ;
        
        % Connection
        \draw    (373,289.72) .. controls (234.96,286.59) and (143.34,231.38) .. (98.15,124.08) ;
        \draw [shift={(97.47,122.46)}, rotate = 427.47] [color={rgb, 255:red, 0; green, 0; blue, 0 }  ][line width=0.75]    (10.93,-3.29) .. controls (6.95,-1.4) and (3.31,-0.3) .. (0,0) .. controls (3.31,0.3) and (6.95,1.4) .. (10.93,3.29)   ;
        
        % Connection
        \draw    (488,290) -- (452.6,290) ;
        \draw [shift={(450.6,290)}, rotate = 360] [color={rgb, 255:red, 0; green, 0; blue, 0 }  ][line width=0.75]    (10.93,-3.29) .. controls (6.95,-1.4) and (3.31,-0.3) .. (0,0) .. controls (3.31,0.3) and (6.95,1.4) .. (10.93,3.29)   ;
        
        % Connection
        \draw    (539.2,290) -- (508,290) ;
        \draw [shift={(506,290)}, rotate = 360] [color={rgb, 255:red, 0; green, 0; blue, 0 }  ][line width=0.75]    (10.93,-3.29) .. controls (6.95,-1.4) and (3.31,-0.3) .. (0,0) .. controls (3.31,0.3) and (6.95,1.4) .. (10.93,3.29)   ;
        
        
        \end{tikzpicture}
        \\
    \end{enumerate}
    
    
    
\end{example}

\subsection{Модификации GLR}
Алгоритм, представленный Томитой имел большой недостаток: он корректно работал не со всеми КС грамматиками, хоть и расширял класс допустимых LR анализаторами. Объем потребляемой памяти классическим GLR можно оценить как $ O(n^3)$ с учетом оптимизаций, о которых говорилось ранее.

Спустя пару лет после публикации Томита-парсера, Элизабет Скотт и Эндриан Джонстоун представили $RNGLR$ (Right Nulled GLR)~\cite{Scott:2006:RNG:1146809.1146810} --- модифицированная версия GLR, которая решала проблему скрытых рекурсий. Это позволило расширить класс допускаемых грамматик до КС. Однако объем потребляемой памяти можно оценить сверху уже полиномом $O(n^{k+1})$, где k --- длина самого длинного правила грамматики, что несколько ухудшило оценку классического GLR.

С этой проблемой справился BRNGLR (Binary RNGLR)~\cite{Scott:2007:BCT:1289813.1289815}. За счет бинаризации удалось получить кубическую оценку сложности и при этом также, как и RNGLR, допускать все КС грамматики.

Кроме того, GLR довольно естесственно обобщается до последовательности входных $строк$ вместо набора $символов$. Это происходит следующим образом: элементами во входной структуре теперь будем считать не позиции символа в слове, а вершины графа (то есть "позиция" и множество смежных вершин). Это приводит к тому, что при применении операции shift, следующих символов может быть несколько и каждый из них должен быть рассмотрен отдельно, сдвигаясь по соответствующему ребру и проходя входной граф в ширину. Подробное описание алгоритма и псевдокод представлены в работе~\cite{10.1007/978-3-319-41579-6_22}.

\section{Вопросы и задачи}
\begin{enumerate}
    \item Постройте LR автомат и управляющую таблицу для грамматики $G_1$: $S \to a S b$; $S \to \epsilon$.
    \item Постройте LR автомат и управляющую таблицу для грамматики $G_2$: $S \to S S S$; $S \to S S$; $S \to a$.
    \item Проведите GLR разбор для грамматики $G_2$ и входного слова $w = aaa\$$.
    \item Реализуйте LR анализатор на любом языке программирования. Программа должна принимать на вход файл с однозначной грамматикой и входное слово, строить LR автомат и управляющую таблицу (во внутреннем представлении), и сообщать, выводимо ли входное слово в данной грамматике.
    \item[6*.] Реализуйте GLR анализатор на любом языке программирования. Программа должна принимать на вход файл с однозначной грамматикой и входное слово, работать согласно алгоритму GLR и сохранять GSS, а также сообщать, выводимо ли входное слово в данной грамматике.
\end{enumerate}

\section{Алгоритм на основе нисходящего анализа}

GLL~\cite{Grigorev:2017:CPQ:3166094.3166104}

Другие реализации~\cite{MEDEIROS201975}

\subsection{Нисходящий синтаксический анализ}

Рекурсивный спуск, LL, таблицы, неоднозначности, левая рекурсия.

\subsubsection{Рекурсивный спуск}

\subsubsection{LL(k)-алгоритм синтаксического анализа}

LL(k) --- алгоритм синтаксического анализа --- нисходящий анализ без отката, но с предпросмотром. 
Решение о том, какую продукцию применять, принимается на основании k следующих за текущим символом. 
Временная сложность алгоритма $O(n)$, где $n$~--- длина слова. 

Алгоритм использует входной буфер, стек для хранения промежуточных данных и таблицу анализатора, которая управляет процессом разбора. 
В ячейке таблицы указано правило, которое нужно применять, если рассматривается нетерминал $A$, а следующие k символов строки~--- $t_{1} \dots t_{k}$. 
Также в таблице выделена отдельная колонка для $\$$~--- маркера конца строки. 

\begin{center}
  \begin{tabular}{ c || c | c | c | c }
             & $\dots$ & $t_{1} \dots t_{k}$ & $\dots$ & $\$$ \\ \hline  
    $\dots$  & $\dots$ & $\dots$ & $\dots$ & $\dots$ \\ \hline  
    $A$  & $\dots$ & $A \to \alpha$ & $\dots$ & $\dots$ \\ \hline  
    $\dots$  & $\dots$ & $\dots$ & $\dots$ & $\dots$ 
  \end{tabular}  
\end{center}

Для построения таблицы вычисляются множества $\first[k]$ и $\follow[k]$. Идейно их можно понимать, как первые или последующие $k$ символов в результирующем выводе, при использовании нетерминала $A$. Данную мысль хорошо иллюстрирует рисунок:

\begin{center}
    \begin{tikzpicture}
        \draw[gray, thick] (0,0) -- (2,4);
        \draw[gray, thick] (2,4) -- (4,0);
        \draw[gray, thick] (1,0) -- (2,2);
        \draw[gray, thick] (2,2) -- (3,0);
        \draw[red, thick] (1,0) -- (1.5,0);
        \draw[red, thick] (3,0) -- (3.5,0);
        \filldraw[black] (2,4) circle (1pt) node[anchor=west] {S};
        \filldraw[black] (2,2) circle (1pt) node[anchor=west] {A};
        \filldraw[black] (1.8,0) circle (0pt) node[anchor=north] {\small $\first[k]$};
        \filldraw[black] (4,0) circle (0pt) node[anchor=north] {\small $\follow[k]$};
    \end{tikzpicture}
\end{center}

Определим их формально:

\begin{definition}
  Пусть $G = \langle N, \Sigma, P, S \rangle$~--- КС-грамматика. Множество $\first[k]$ определено для сентециальной формы $\alpha$ следующим образом:   
  \[ \first[k](\alpha) = \{ \omega \in \Sigma^* \mid \alpha \derives{} \omega \text{ и } |\omega| < k \text{ либо } \exists \beta: \alpha \derives{} \omega \beta \text{ и } |\omega| = k \} \text{, где } \alpha, \beta \in (N \cup \Sigma)^* \]
\end{definition}

\begin{definition}
  Пусть $G = \langle N, \Sigma, P, S \rangle$~--- КС-грамматика. Множество $\follow[k]$ определено для сентециальной формы $\beta$ следующим образом:
  \[\follow[k](\beta) = \{ \omega \in \Sigma^* \mid \exists \gamma, \alpha: S \derives{} \gamma \beta \alpha \text{ и } \omega \in \first[k](\alpha) \} \]
\end{definition}

В частном случае для $k = 1$: 

\[ \first(\alpha) = \{ a \in \Sigma \mid \exists \gamma \in (N \cup \Sigma)^*: \alpha \derives{} a \gamma \} \text{, где } \alpha \in (N \cup \Sigma)^* \]

\[ \follow(\beta) = \{ a \in \Sigma \mid \exists \gamma, \alpha \in (N \cup \Sigma)^* : S \derives{} \gamma \beta a \alpha \} \text{, где } \beta \in (N \cup \Sigma)^*  \]

Множество $\first$ можно вычислить, пользуясь следующими соотношениями:  

\begin{itemize}
  \item $\first(a \alpha) = \{a\}, a \in \Sigma, \alpha \in (N \cup \Sigma)^* $
  \item $\first(\varepsilon) = \{\varepsilon\}$
  \item $\first(\alpha \beta) = \first(\alpha) \cup (\first(\beta) \text{, если } \varepsilon \in \first(\alpha))$
  \item $\first(A) = \first(\alpha) \cup \first(\beta) \text{, если в грамматике есть правило } A \to \alpha \mid\beta$
\end{itemize}

Алгоритм для вычисления множества $\follow$: 

\begin{itemize}
  \item Положим $\follow(X) = \varnothing, \forall X \in N$
  \item $\follow(S) = \follow(S) \cup \{\$\} \text{, где } S \text{--- стартовый нетерминал}$
  \item Для всех правил вида $A \to \alpha X \beta: \follow(X) = \follow(X) \cup (\first(\beta) \setminus \{\varepsilon\} )$
  \item Для всех правил вида $A \to \alpha X \text{ и } A \to \alpha X \beta \text{, где } \varepsilon \in \first(\beta): \follow(X) = \follow(X) \cup \follow(A)$
  \item Последние два пункта применяются пока есть что добавлять в строящиеся множества. 
\end{itemize}

Пример множеств $\first$ для нетерминалов следующей грамматики: 

\begin{multicols}{2}
\begin{align*}
  S  &\to a S' \\ 
  S' &\to A b B S' \mid \varepsilon \\ 
  A  &\to a A' \mid \varepsilon \\ 
  A' &\to b \mid a \\ 
  B  &\to c \mid \varepsilon
\end{align*}

\columnbreak
    
\begin{align*}
  \first(S)  &= \{ a \} \\
  \first(A)  &= \{ a, \varepsilon \} \\ 
  \first(A') &= \{ a, b \} \\
  \first(B)  &= \{ c, \varepsilon \} \\
  \first(S') &= \{ a, b, \varepsilon \}  
\end{align*}
\end{multicols}

Пример множеств $\follow$ для нетерминалов следующей грамматики:

\begin{multicols}{2}
\begin{align*}
  S  &\to a S' \\ 
  S' &\to A b B S' \mid \varepsilon \\ 
  A  &\to a A' \mid \varepsilon \\ 
  A' &\to b \mid a \\ 
  B  &\to c \mid \varepsilon
\end{align*}

\columnbreak
    
\begin{align*}
  \follow(S)  &= \{ \$ \} & \\
  \follow(S') &= \{ \$ \} &(S \to a S')\\
  \follow(A)  &= \{ b \}  &(S' \to A b B S') \\ 
  \follow(A') &= \{ b \}  &(A \to a A')\\
  \follow(B)  &= \{ a, b, \$ \} &(S' \to A b B S', \varepsilon \in \first(S'))
\end{align*}  
\end{multicols}

Таблица заполняется следующим образом: продукции $A \to \alpha, \alpha \neq \varepsilon$ помещаются в ячейки $(A, a)$, где $a \in \first(A)$, продукции $A \to \varepsilon$~--- в ячейки $(A, a)$, где $a \in \follow(A)$

\begin{example}

Пример таблицы для грамматики $S \to ( S ) \mid \varepsilon$

\begin{center}
\begin{tabular}{ r || c | c || c | c | c }
N & $\first$ & $\follow$ & ( & ) & $\$ $ \\ \hline  
$S$ & $\{ (, \varepsilon \}$ & $\{ ), \$ \}$ & $S \rightarrow (S)$ & $S \rightarrow \varepsilon$ & $S \rightarrow \varepsilon$ 
\end{tabular}  
\end{center}

\end{example}

Однако, не для всех грамматик по множествам $\first[k]$ и $\follow[k]$ возможно выбрать применяемую продукцию, а значит, нельзя однозначно построить таблицу, необходимую для работы алгоритма, поэтому данный алгоритм применим только для грамматик особого класса --- LL(k).

\begin{definition}
  LL(k) грамматика --- грамматика, для которой на основании множеств $\first[k]$ и $\follow[k]$ можно однозначно определить, какую продукцию применять.
\end{definition}

Важно заметить, что при больших $k$ строимая нами таблица сильно разрастается, поэтому на практике данный алгоритм применим для небольших значений $k$.

\paragraph{Ход работы:}

Интерпретатор автомата принимает входную строку и построенную управляющую таблицу и работает следующим образом. 
В каждый момент времени конфигурация автомата это позиция во входной строке и стек. 
В начальный момент времени стэк пуст, а позиция во входной строке соответствует её началу.
На певом шаге в стек добавляются последовательно сперва симаол концы строки, затем стартовый нетерминал.
На каждом шаге анализируется существующая конфигурация и совершается одно из действий.
\begin{itemize}
\item Если текущая позиция --- конец строки и вершина стека --- символ конца строки, то успешно завершаем разбор.
\item Если текушая вершина стека --- терминал, то проверяем, что позиция в строке соответствует этому терминалу. Если да, то снимаем элемент со стека, сдвигаем позицию на единицу и продолжаем разбор. Иначе завершаем разбор с ошибкой.
\item Если текущая врешина стека --- нетерминал $N_i$ и текущий входной символ $t_j$, то ищем в управляющей таблице ячейку с координатами $(N_i, t_j)$ и записываем на стек содержимое этой ячейки.
\end{itemize}

\begin{example}Пример работы LL анализатора.
Рассмотрим грамматику $S \to aSbS \mid \varepsilon$ и выводимое слово $\omega = abab$.

Построим таблицу:

\begin{center}
    \begin{tabular}{ r || c | c | c }
    N & a & b & $\$ $ \\ \hline  
    $S$ & $S \rightarrow aSbS$ & $S \rightarrow \varepsilon$ & $S \rightarrow \varepsilon$ 
    \end{tabular}  
\end{center}

Расмотрим пошагово работу алгоритма:

\begin{enumerate} 
  \item Начало работы:

    Стек: \,
    \begin{tabular}[c]{ |c| } 
        \\ \hline
        $S$ \\ \hline
        \$ \\ \hline
    \end{tabular}  
    \qquad  \qquad \qquad  \qquad входное слово: \,
    \begin{tabular}[c]{ |c|c|c|c|c| } 
        \hline
        \textcolor{red}{a} & b & a & b & \$ \\ \hline
    \end{tabular}

На стеке лежат стартовый и финальный символы, указатель указывает на первый символ слова.
    
  \item Ищем ячейку с координатами (S, a), применяем продукцию из ячейки.

    Стек: \,
    \begin{tabular}[c]{ |c| } 
        \\ \hline
        $a$ \\ \hline
        $S$ \\ \hline
        $b$ \\ \hline
        $S$ \\ \hline
        \$ \\ \hline
    \end{tabular}  
    \qquad  \qquad \qquad  \qquad входное слово: \,
    \begin{tabular}[c]{ |c|c|c|c|c| } 
        \hline
        \textcolor{red}{a} & b & a & b & \$ \\ \hline
    \end{tabular}

\item Снимаем терминал $a$ со стека и двигаем указатель.
    
    Стек: \,
    \begin{tabular}[c]{ |c| } 
        \\ \hline
        $S$ \\ \hline
        $b$ \\ \hline
        $S$ \\ \hline
        \$ \\ \hline
    \end{tabular}  
    \qquad  \qquad \qquad  \qquad входное слово: \,
    \begin{tabular}[c]{ |c|c|c|c|c| } 
        \hline
        a & \textcolor{red}{b} & a & b & \$ \\ \hline
    \end{tabular}

\item Ищем ячейку с координатами (S, b), применяем продукцию из ячейки.

    Стек: \,
    \begin{tabular}[c]{ |c| } 
        \\ \hline
        $b$ \\ \hline
        $S$ \\ \hline
        \$ \\ \hline
    \end{tabular}  
    \qquad  \qquad \qquad  \qquad входное слово: \,
    \begin{tabular}[c]{ |c|c|c|c|c| } 
        \hline
        a & \textcolor{red}{b} & a & b & \$ \\ \hline
    \end{tabular}

\item Снимаем терминал $b$ со стека и двигаем указатель.

    Стек: \,
    \begin{tabular}[c]{ |c| } 
        \\ \hline
        $S$ \\ \hline
        \$ \\ \hline
    \end{tabular}  
    \qquad  \qquad \qquad  \qquad входное слово: \,
    \begin{tabular}[c]{ |c|c|c|c|c| } 
        \hline
        a & b & \textcolor{red}{a} & b & \$ \\ \hline
    \end{tabular}
  
  \item Ищем ячейку с координатами (S, a), применяем продукцию из ячейки.

    Стек: \,
    \begin{tabular}[c]{ |c| } 
        \\ \hline
        $a$ \\ \hline
        $S$ \\ \hline
        $b$ \\ \hline
        $S$ \\ \hline
        \$ \\ \hline
    \end{tabular}  
    \qquad  \qquad \qquad  \qquad входное слово: \,
    \begin{tabular}[c]{ |c|c|c|c|c| } 
        \hline
        a & b & \textcolor{red}{a} & b & \$ \\ \hline
    \end{tabular}

\item Снимаем терминал $a$ со стека и двигаем указатель.
    
    Стек: \,
    \begin{tabular}[c]{ |c| } 
        \\ \hline
        $S$ \\ \hline
        $b$ \\ \hline
        $S$ \\ \hline
        \$ \\ \hline
    \end{tabular}  
    \qquad  \qquad \qquad  \qquad входное слово: \,
    \begin{tabular}[c]{ |c|c|c|c|c| } 
        \hline
        a & b & a & \textcolor{red}{b} & \$ \\ \hline
    \end{tabular}

\item Ищем ячейку с координатами (S, b), применяем продукцию из ячейки.

    Стек: \,
    \begin{tabular}[c]{ |c| } 
        \\ \hline
        $b$ \\ \hline
        $S$ \\ \hline
        \$ \\ \hline
    \end{tabular}  
    \qquad  \qquad \qquad  \qquad входное слово: \,
    \begin{tabular}[c]{ |c|c|c|c|c| } 
        \hline
        a & b & a & \textcolor{red}{b} & \$ \\ \hline
    \end{tabular}

\item Снимаем терминал $b$ со стека и двигаем указатель.

    Стек: \,
    \begin{tabular}[c]{ |c| } 
        \\ \hline
        $S$ \\ \hline
        \$ \\ \hline
    \end{tabular}  
    \qquad  \qquad \qquad  \qquad входное слово: \,
    \begin{tabular}[c]{ |c|c|c|c|c| } 
        \hline
        a & b & a & b & \textcolor{red}{\$} \\ \hline
    \end{tabular}

\item Ищем ячейку с координатами (S, \$), применяем продукцию из ячейки.

    Стек: \,
    \begin{tabular}[c]{ |c| } 
        \\ \hline
        \$ \\ \hline
    \end{tabular}  
    \qquad  \qquad \qquad  \qquad входное слово: \,
    \begin{tabular}[c]{ |c|c|c|c|c| } 
        \hline
        a & b & a & b & \textcolor{red}{\$} \\ \hline
    \end{tabular}    
 
\item Оказались в конце строки и на вершине стека символ конца --- завершаем разбор.

\end{enumerate}

\end{example}

Еще одна существенная проблема данного алгоритма --- грамматики, содержащие леворекурсивные правила, т.е правила вида $Q \rightarrow Q\omega$. Действительно, встретив на вершине стека нетерминал $Q$  и применив данное правило, на вершине стека снова окажется $Q$ и мы будем вынуждены вновь применять это правило, таким образом алгоритм зациклится. Та же проблема встречаестя и в грамматиках со скрытой левой рекурсией.

Можно построить анализатор, работающий с произвольными КС-граммтиками.
Generalized LL (GLL)~\cite{Scott:2010:GP:1860132.1860320,10.1007/978-3-662-46663-6_5}


\subsection{GLL для КС запросов}
e~\cite{Grigorev:2017:CPQ:3166094.3166104}
Заметим, что позиции всё так же вершины графа.
Дальше всё само собой.

Но нам надо строить SPPF, чтобы получать пути.


\subsection{Вопросы и задачи}
\begin{enumerate}
  \item Задача 1
  \item Задача 2
\end{enumerate}


\section{Комбинаторы для КС апросов}

\subsection{Парсер комбинаторы}

Что это, с чем едят, плюсы, минусы. Про семантику, безопасность, левую рекурсию и т.д.
Набор примитивных парсеров и функций, которые умеют из существующих арсеров строить более сложные (собственно, комбинаторы парсеров).

Разобрать символ, разобрать последовательность, разобрать альтернативу. впринципе, этого достаточно, но это не очень удобно.

Проблемы с левой рекурсией.
Существуют решения. Одно из них --- Meerkat.
Подробно про него?

\subsection{Комбинаторы для КС запросов}

Вообще говоря, идея использовать комбинаторы для навигации по графам достаточно очевидно и не нова.
немного про Trails.

Комбинаторы для запросов к графам на основе Meerkat~\cite{Verbitskaia:2018:PCC:3241653.3241655}

Обобщённые запросы, типобезопасность и всё такое.
Примеры запросов.

\subsection{Вопросы и задачи}
\begin{enumerate}
  \item Реализовать библиотеку парсер комбинаторов.
  \item Что-нибудь полезное с ними сделать.
\end{enumerate}

\section{Производные для КС запросов}

\subsection{Производные}

Общая теория.

Определения.

\subsection{Парсинг на производных}

Статьи~\cite{DBLP:journals/corr/abs-1010-5023,Adams:2016:CPP:2908080.2908128,Might:2011:PDF:2034574.2034801,andersenparsing}
Реализации.
На Scala~\footnote{\url{https://github.com/djspiewak/parseback}}, на Racket~\footnote{\url{https://bitbucket.org/ucombinator/derp-3/src/86bca8a720231e010a3ad6aefd1aa1c0f35cbf6b/src/derp.rkt?at=master&fileviewer=file-view-default}}.

\subsection{Адоптация для КС запросов}

Для регулярных запросов над графами~\cite{Nole:2016:RPQ:2949689.2949711}.
Хорошо работают в распределённых системах, в которых реализовван параллелизм уровня вершин. 
Например Google Pregel.



\subsection{Вопросы и задачи}
\begin{enumerate}
  \item Предъявить несколько выводов для одной цепочки.
  \item Построить выводы
  \item Построить деревья вывода !!! Перенести из раздела про SPPF
\end{enumerate}


\section{От CFPQ к вычислению Datalog-запросов}\label{Subsection Datalog}
Рассмотрим грамматику  $S \rightarrow aSb \mid SS \mid \varepsilon$, заданную через набор предикатов:
\begin{itemize}
	\item $a(i, w)$ --- Предикат, соответствующий терминалу. Обращается в True, если на i-том месте в строке w стоит символ a
	\item $S(i, j, w)$ --- Предикат, соответствующий нетерминалу. Обращается в True, если выполняется одно из условий:
	\begin{enumerate}
		\item $i == j \quad(\varepsilon)$
		\item $\exists k: i \leq k \leq j \And S(i, k - 1, w) \And S(k, j, w)  \quad (SS)$
		\item  ${a(i, w) \And S(i+1, j-1, w) \And b(j-1, w)} \quad (aSb)$
	\end{enumerate}
\end{itemize}

Таким образом $S(0,|w|,w)$ покажет, выводится ли строка $w$ из данной грамматики,
а $S(\_,\_,w)$ даст нам список всех цепочек внутри $w$, выводящихся из $S$.

\subsection{Datalog}
Datalog~\cite{Datalog}\footnote{\url{https://www.computer.org/csdl/journal/tk/1989/01/k0146/13rRUx0xPIQ}} --- декларативный логический язык программирования. Используется для написания запросов к дедуктивным базам данных\footnote{\url{https://en.wikipedia.org/wiki/Deductive_database}}. 

\begin{example}
	Пример программы на даталоге.
	\begin{enumerate}
		\item Набор фактов (часто факты находятся в базе данных):
		\begin{itemize}
			\item $a(0).$
			\item $b(1).$
			\item $a(2).$
			\item $b(3).$
			\item $s(I, I).$
		\end{itemize}
		\item Набор правил:
		\begin{itemize}
			\item $s(I, J) \coloneq s(I, K-1), s(K,J), (I \leq K \leq J)$
			\item $s(I,J)\coloneq a(I), s(I+1, J-1),b(J)$
		\end{itemize}
		\item Запросы:
		\begin{itemize}
			\item $?- s(I, J)$
		\end{itemize}
	\end{enumerate}
\end{example}

Таким образом мы описали на даталоге строку через набор фактов, грамматику, указанную выше, через набор фактов и правил и сделали запрос на все цепочки, выводящиеся из $S$. 

\textbf{NB!} Обратите внимание~\cite{Datalog}, что строки, начинающиеся с большой буквы, в даталоге считаются переменными. Также важно, что все переменные неявно квантифицированны.


\subsection{Datalog для работы с графами}
На даталоге также можно задавать графы и писать к ним запросы.
\begin{example}
	Пример описания графа на даталоге.
	\begin{center}
		\begin{tikzpicture}[shorten >=1pt,on grid,auto]
		\node[state] (q_0)   {$0$};
		\node[state] (q_1) [above right=of q_0] {$1$};
		\node[state] (q_2) [right=of q_0] {$2$};
		\node[state] (q_3) [right=of q_2] {$3$};
		\path[->]
		(q_0) edge  node {$a$} (q_1)
		(q_1) edge  node {$a$} (q_2)
		(q_2) edge  node {$a$} (q_0)
		(q_2) edge[bend left, above]  node {$b$} (q_3)
		(q_3) edge[bend left, below]  node {$b$} (q_2);
		\end{tikzpicture}
	\end{center}
\begin{itemize}
	\item[] $a(0, 1).$
	\item[] $a(1, 2).$
	\item[] $a(2, 0).$
	\item[]	$b(2, 3).$
	\item[]	$b(3, 2).$
\end{itemize}
	
\end{example}

Теперь зададим рассмотренную выше грамматику для работы с графом.
\begin{itemize}
	\item[] $s(I, I).$
	\item[] $s(I,J) \coloneq s(I, K-1),s(K, J).$
	\item[] $s(I, J) \coloneq a(I, L),S(L,M),b(M,J).$
\end{itemize}

Тогда запрос $?-s(I, J)$ выдаст нам все такие пути в графе, что последовательность составляющих их вершин в порядке прохождения выводится их грамматики.

\subsection{Алгоритм Эрли}
Для распознавания контекстно-свободных грамматик может использоваться алгоритм Эрли~\cite{Earley} \footnote{\url{https://en.wikipedia.org/wiki/Earley}}.
Рассмотрим грамматику $G=(N,T,P,S)$, слово $a_1...a_n$,
и правило $A \rightarrow \alpha\beta$. Будем считать, что утверждение
$[A\rightarrow \alpha \bullet \beta](i,j), j \in [1..n]$ является истиной, если верно, что:
\begin{itemize}
	\item $\alpha \xrightarrow{\smash{*}} a_{i+1}...a_j$ (Последовательность выводится из $\alpha$)
	\item $S \xrightarrow{\smash{*}} a_1...a_jA_\gamma$ 
\end{itemize}


Рассмотрим правила вывода для подобных утверждений:

\begin{enumerate}
	\item $\frac{S \rightarrow \alpha \in P}{[S \rightarrow \bullet \alpha](0,0)}$ Инициализация (Init)
	
	\item $\frac{\left[A \rightarrow \alpha \bullet a_{j+1} \beta\right](i, j)}{\left[A \rightarrow \alpha a_{j+1} \bullet \beta\right](i, j+1)}$ Сканирование (Scan)
	
	\item $\frac{[A \rightarrow \alpha \cdot B \beta](i, j) \quad B \rightarrow \gamma \in P}{[B \rightarrow \bullet \gamma](j, j)}$ Предсказание (Predict)
	
	\item $\frac{[A \rightarrow \alpha \cdot B](i, j) \quad[B \rightarrow \gamma \bullet](j, k)}{[A \rightarrow \alpha B \bullet \beta](i, k)}$ Завершение (Complete)
	
\end{enumerate}

Идея алгоритма Эрли заключается в том, чтобы, начиная с инициализации, используя правила, вывести утверждение, содержащие данную строку слева от точки, и ничего справа, или попробовать все возможные выводы и признать, что строка не выводима.

Сложность алгоритма Эрли составляет $O(|P|^2n^3)$

\begin{example} Пример начала одной из веток дерева вывода для алгоритма Эрли для рассматриваемой грамматики
	
	$$\underline{S \rightarrow SS} \quad Init$$
	$$\underline{[S \rightarrow \bullet SS](0,0), \space S \rightarrow aSb} \quad Predict$$
	$$\underline{[S \rightarrow \bullet aSb](0,0)} \quad Scan$$
	$$\underline{[S \rightarrow a \bullet Sb](0,1), S \rightarrow \varepsilon} \quad Predict$$
	$$\underline{[S \rightarrow \bullet](0, 1), [S \rightarrow aS \bullet b](0,1)} \quad Complete$$
	$$\underline{[S \rightarrow aS \bullet b](0,1)} \quad Scan$$
	$$\underline{[S \rightarrow \bullet SS](0,0),[S \rightarrow aSb \bullet](0,2)} \quad Complete$$
	$$[\underline{S \rightarrow S \bullet S](0,2)}$$
	$$\cdot\cdot\cdot$$
	
\end{example}

Сложность можно понизить, изменив правила ``Предсказание'' и ``Завершение'' таким образом:
\begin{itemize}
	\item $\frac{[A \rightarrow \alpha \cdot B \beta](i, j) \quad B \rightarrow \gamma \in P}{[B \rightarrow \bullet \gamma](j, j)} \Rightarrow$ $
	\frac{[A \rightarrow \alpha \cdot B \beta](i, j)}{? B(j)}; \quad \frac{? B(j) \quad B \rightarrow \gamma \in P}{[B \rightarrow \cdot \gamma](j, j)}$
	\item $\frac{[A \rightarrow \alpha \cdot B](i, j) \quad[B \rightarrow \gamma \bullet](j, k)}{[A \rightarrow \alpha B \bullet \beta](i, k)} \Rightarrow$ $\frac{[B \rightarrow \gamma \bullet](j, k)}{B(j, k)}; \quad \frac{[A \rightarrow \alpha \cdot B \beta](i, j) \quad B(j, k)}{[A \rightarrow \alpha B \cdot \beta](i, k)}$
\end{itemize}
Так, разложив каждое правило на два, мы избавляемся от необходимости перевычислять дерево разбора каждого нетерминала после того, как однократно вычислим, что он выводим (мемоизируем его). Получаем сложность $O(|P|)$.

Описанный подход мемоизации~\cite{Magic} части используется для оптимизации программ на даталоге. Можно либо видоизменять правила, задающие грамматику в тексте программы, либо модифицировать компилятор, чтобы он пытался сделать это автоматически. 
%\section{Conclusion}

We propose and implement in C\# programming language the generic framework for interprocedural static code analysis implementation.
This framework allows one to implement arbitrary interprocedural analysis in terms of CFL-reachability.
By using the proposed framework, we implement a plugin upon ReSharper infrastructure which provides simple taint analysis and demonstrate that our solution can handle important real-world cases.
Also we show that the proposed framework can be used for real-world solutions analysis.

One of the directions for future work is a creation of analysis and its evaluation on real-world projects.
By this way, we want to get information which helps to improve the usability of our framework: tune performance, improve API, etc.
Also we should improve documentation and create more examples of usage.

Another direction is a practical evaluation of automatic fix location prediction by using minimum cuts method~\cite{10.1007/978-3-319-63390-9_27}.

Also we want to compare the proposed approach with other generic CFL-reachability based approaches for interprocedural code analysis cretion. For example, fith generation-based approach~\cite{LPAR-21:Cauliflower_Solver_Generator_for}, which idea is similar to parser generators.


\bibliographystyle{abbrv}
\bibliography{Formal_lang_CFPQ_course_notes}


\end{document}
