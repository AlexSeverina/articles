\section{Постановка задачи о поиске путей с ограничениями в терминах формальных языков}

Пусть $\mathcal{G} = \langle V,E,L \rangle$ --- конечный ориентрованный граф с метками на рёбрах, где $V$ --- конечное множество вершин, $E$ --- конечное множество рёбер, $L$ --- конечное множество или алфавит меток.
Рёбра будем представлять в виде упорядоченных троек из $V \times L \times V$.
Путём $\pi$ в графе $\mathcal{G}$ будем называть последовательность рёбер такую, что для любых двух последовательных рёбер $e_1=(u_1,l_1,v_1)$ и $e_2=(u_2,l_2,v_2)$ в этой последовательности, конечная вершина первого ребра является начальной вершиной второго, то есть $v_1 = u_2$.
Будем обобзначать путь из вершины $v_0$ в вершину $v_n$ как $v_0 \pi v_n = e_0,e_1, \dots, e_{n-1} = (v_0, l_0, v_1),(v_1,l_1,v_2),\dots,(v_{n-1},l_n,v_n)$.

Функция $\omega(\pi) = \omega((v_0, l_0, v_1),(v_1,l_1,v_2),\dots,(v_{n-1},l_n,v_n)) = l_0 \cdot l_1 \cdot \ldots \cdot l_n $ строит слово по пути посредством конкатенации меток рёбер вдоль этого пути.
Очевидно, для пустого пути данная функция будет возвращать пустое слово, а для пути длины $n  > 0$ --- непустое слово длины $n$.

Путь $G = \langle \Sigma, N, P \rangle$ --- контекстно-свободная граммтика.
Будем считать, что $L \subseteq \Sigma$.
Мы не фиксируем стартовый нетерминал в определении граммтики, поэтому, чтобы описать язык, задаваемый ей, нам необходимо отдельно зафиксировать стартовый нетерминал.
Таким образом, будем говорить, что $L(G,N_i) = \{ w | N_i \xRightarrow[G]{*} w  \}$ --- это язык задаваемый граммтикой $G$ со стартовым нетерминалом $N_i$.

Задача достижимости:

Задача поиска путей:


\subsection{Вопросы и задачи}
\begin{enumerate}
  \item Задача 1
  \item Задача 2
\end{enumerate}
