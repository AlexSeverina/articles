\section{Общие сведения теории графов}

основные определения: рёбра, вершины, графы ориентированные, неориентированные, помеченные.

Специальные понятия, необходимые для иложения конкретного материала, будут даны в соответсвующих главах.

Матрицы смежности

\subsection{Транзитивне замыкание графа}

Флойд-Уоршал, матрицы.

Рассуждения про субкубичность.
Про то, что булево полукольцо.


\subsection{Задачи поиска путей}

Достижимость, все пути, один путь.

Про кратчайшие, простые и прочее.

Про классические задачи и классическое сведение к матричным операциям.

Тра-та-та.

\subsection{Вопросы и задачи}
\begin{enumerate}
  \item Реализуйте алгоритм построения транзитивного замыкания через матрицы.
  Реализовать матрицы самим.
  Взять готовую библиотеку матричных операций: CPU, GPGPU.
  \item Реализуйте поиск кратчайших путей через матричные операции.
  Реализовать перемножение матриц самостоятельно.
  Взять готовую библиотеку матричных операций: CPU, GPGPU.
\end{enumerate}
