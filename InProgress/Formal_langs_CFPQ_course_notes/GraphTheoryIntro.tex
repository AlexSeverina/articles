\section{Общие сведения теории графов}

В данном разделе мы дадим определения базовым понятиям из теории графов, рассмотрим несколько классических задач из области анализа графов и алгоритмы их решения.
Всё это понадобится нам при последующей работе.

В дальнейшем нам будут нужны конечные ориентированные помеченные графы.
Мы будем использовать термин \textit{граф} подразумевая именно конечный ориентированный помеченный граф, если только не оговорено противное.

\subsection{Основные определения}

\begin{definition}
  \textit{Граф} $\mathcal{G} = \langle V, E, L \rangle$, где $V$ --- конечное множество вершин, $E$ --- конечное множество рёбер, т.ч. $E \subseteq V \times L \times V$, $L$ --- конечное множество меток рёбер. 
\end{definition}

Мы будем считать, что все вершины занумерованы подряд с нуля.
То есть можно считать, что $V$ --- это отрезок неотрицательный целых чисел.

\begin{example}[Пример графа и его графического представления]
  Пусть граф \\ $\mathcal{G}_1 = \langle \{0,1,2,3\}, \{(0,a,1), (1,a,2), (2,a,0), (2,b,3), (3,b,2)\}, \{a,b\} \rangle$.
  Графическое представление графа $\mathcal{G}_1$:
  \begin{center}
  \begin{tikzpicture}[on grid, auto]
     \node[state] (q_0)   {$0$};
     \node[state] (q_1) [above right=1.4cm and 1.0cm of q_0] {$1$};
     \node[state] (q_2) [right=2.0cm of q_0] {$2$};
     \node[state] (q_3) [right=2.0cm of q_2] {$3$};
      \path[->] 
      (q_0) edge  node {a} (q_1)
      (q_1) edge  node {a} (q_2)
      (q_2) edge  node {a} (q_0)
      (q_2) edge[bend left, above]  node {b} (q_3)
      (q_3) edge[bend left, below]  node {b} (q_2);
  \end{tikzpicture}
  \end{center}
\end{example}

\begin{definition}
  \textit{Ребро} ориентированного помеченного графа $\mathcal{G} = \langle V, E, L \rangle$ это упорядоченная тройка из $V \times L \times V$.
\end{definition}

\begin{example}
$(0,a,1)$  и $(3,b,2)$ --- это рёбра графа $\mathcal{G}_1$. При этом, $(3,b,2)$ $(2,b,3)$ --- это разные рёбра, что видно из рисунка.
\end{example}

\begin{definition}
  \textit{Путём} $\pi$ в графе $\mathcal{G}$ будем называть последовательность рёбер такую, что для любых двух последовательных рёбер $e_1=(u_1,l_1,v_1)$ и $e_2=(u_2,l_2,v_2)$ в этой последовательности, конечная вершина первого ребра является начальной вершиной второго, то есть $v_1 = u_2$. Будем обозначать путь из вершины $v_0$ в вершину $v_n$ как $v_0 \pi v_n = e_0,e_1, \dots, e_{n-1} = (v_0, l_0, v_1),(v_1,l_1,v_2),\dots,(v_{n-1},l_n,v_n)$.
  
\begin{center}
  \begin{tikzpicture}[on grid, auto]
     \node[state] (v_1)   {$v_1$};
     \node[state] (v_n) [right=2.0cm of v_1] {$v_n$};
      \path[->] 
      (v_1) edge [out=45] node {$\pi$} (v_n);
  \end{tikzpicture}
  \end{center}
\end{definition}

\begin{example}
$(0,a,1)(1,a,2) = 1\pi_1 2$  --- путь из вершины 0 в вершину 2 в графе $\mathcal{G}_1$.
При этом, $(0,a,1)(1,a,2)(2,b,3)(3,b,2) = 1\pi_2 2$ --- это тоже путь из вершины 0 в вершину 2 в графе $\mathcal{G}_1$, но он не равен $0\pi_1 2$.
\end{example}

Нам потребуется также отношение, отражающее факт существования пути между двумя вершинами.

\begin{definition}\label{def:reach}
  \textit{Отношение достижимости} в графе:
  $(v_i,v_j) \in P \iff \exists v_i \pi v_j$.
\end{definition}

Отметим, что рефлексивность этого отношения часто зависит от контекста.
В некоторых задачах по-умолчанию $(v_i,v_i) \notin P$, а чтобы это было верно, требуется явное наличие ребра-петли.

Один из способов задать граф --- это задать его \textit{матрицу смежности}.

\begin{definition}
  \textit{Матрица смежности} графа $\mathcal{G}=\langle V,E,L \rangle$ --- это квадратная матрица $M$ размера $n \times n$, где $|V| = n$ и ячейки которой содержат множества.
  При этом $l \in M[i,j] \iff \exists e = (i,l,j) \in E$.
\end{definition}

Заметим, что наше определение матрицы смежности отличается от классического, в котором матрица отражает лишь факт наличия хотя бы одного ребра и, соответственно, является булевой. То есть $M[i,j] = 1 \iff \exists e = (i,\_,j) \in E$.


Также можно встретить матрицы смежности в ячейках которых всё же хранится некоторая информация, однако, в единственном экземпляре. То есть запрещены параллельные рёбра.
Такой подход часто можно встретить в задачах о кратчайших путях: в этом случае в ячейке хранится расстояние между двумя вершинами.
При этом, так как в качестве весов часто рассматривают произвольные (в том числе отрицательные) числа, то в задачах о кратчайших путях отдельно вводят значение ``бесконечность'' для обозначения ситуации, когда между двумя вершинами нет пути или его длина ещё не известна.


\begin{example}
  Неориентированный граф:
  \begin{center}
  \begin{tikzpicture}[on grid,auto]
     \node[state] (q_0)   {$0$};
     \node[state] (q_1) [above right = 1.4cm and 1cm of q_0] {$1$};
     \node[state] (q_2) [right = 2cm of q_0] {$2$};
     \node[state] (q_3) [right = 2cm of q_2] {$3$};
      \path[-]
      (q_0) edge  node {} (q_1)
      (q_1) edge  node {} (q_2)
      (q_2) edge  node {} (q_0)
      (q_2) edge  node {} (q_3);
  \end{tikzpicture}
  \end{center}

  И его матрица смежности:
  $$
  \begin{pmatrix}
    1 & 1 & 1 & 0 \\
    1 & 1 & 1 & 0 \\
    1 & 1 & 1 & 1 \\
    0 & 0 & 1 & 1
  \end{pmatrix}
  $$
\end{example}

\begin{example}
  Ориентированный граф:
  \begin{center}
  \begin{tikzpicture}[shorten >=1pt,on grid,auto]
     \node[state] (q_0)   {$0$};
     \node[state] (q_1) [above right = 1.4cm and 1cm of q_0] {$1$};
     \node[state] (q_2) [right = 2cm of q_0] {$2$};
     \node[state] (q_3) [right = 2cm of q_2] {$3$};
      \path[->]
      (q_0) edge  node {} (q_1)
      (q_1) edge  node {} (q_2)
      (q_2) edge  node {} (q_0)
      (q_2) edge[bend left, above]  node {} (q_3)
      (q_3) edge[bend left, below]  node {} (q_2);
  \end{tikzpicture}
  \end{center}

  И его матрица смежности:
  $$
  \begin{pmatrix}
    1 & 1 & 0 & 0 \\
    0 & 1 & 1 & 0 \\
    1 & 0 & 1 & 1 \\
    0 & 0 & 1 & 1
  \end{pmatrix}
  $$
\end{example}

\begin{example}
  Помеченный граф:
  \begin{center}
  \begin{tikzpicture}[shorten >=1pt,on grid,auto]
     \node[state] (q_0)   {$0$};
     \node[state] (q_1) [above right = 1.4cm and 1cm of q_0] {$1$};
     \node[state] (q_2) [right = 2cm of q_0] {$2$};
     \node[state] (q_3) [right = 2cm of q_2] {$3$};
      \path[->]
      (q_0) edge  node {a} (q_1)
      (q_1) edge  node {a} (q_2)
      (q_2) edge  node {a} (q_0)
      (q_2) edge[bend left = 20]  node {a} (q_3)
      (q_2) edge[bend left = 60]  node {b} (q_3)
      (q_3) edge[bend left, below]  node {b} (q_2);
  \end{tikzpicture}
  \end{center}

  И его матрица смежности:
  $$
  \begin{pmatrix}
    \varnothing   & \{a\}       & \varnothing & \varnothing \\
    \varnothing   & \varnothing & \{a\}       & \varnothing \\
    \{a\}         & \varnothing & \varnothing & \{a,b\} \\
    \varnothing   & \varnothing & \{b\}       & \varnothing
  \end{pmatrix}
  $$
\end{example}

\begin{example}
  Взвешенный граф для задачи о кратчайших путях:
  \begin{center}
  \begin{tikzpicture}[shorten >=1pt,on grid,auto]
     \node[state] (q_0)   {$0$};
     \node[state] (q_1) [above right = 1.4cm and 1cm of q_0] {$1$};
     \node[state] (q_2) [right = 2cm of q_0] {$2$};
     \node[state] (q_3) [right = 2cm of q_2] {$3$};
      \path[->]
      (q_0) edge  node {-1.4} (q_1)
      (q_1) edge  node {2.2} (q_2)
      (q_2) edge  node {0.5} (q_0)
      (q_2) edge[bend left, above]  node {1.85} (q_3)
      (q_3) edge[bend left, below]  node {-0.76} (q_2);
  \end{tikzpicture}
  \end{center}

  И его матрица смежности (для задачи о кратчайших путях):
  $$
  \begin{pmatrix}
    0 & -1.4 & \infty & \infty \\
    \infty & 0 & 2.2 & \infty \\
    0.5 & \infty & 0 & 1.85 \\
    \infty & \infty & -0.76 & 0
  \end{pmatrix}
  $$
\end{example}

Мы ввели лишь общие понятия.
Специальные понятия, необходимые для изложения конкретного материала, будут даны в соответствующих главах.

\subsection{Задачи поиска путей}

Одна из классических задач анализа графов --- это задача поиска путей между вершинами с различными ограничениями.

При этом, возможны различные постановки задачи.
С одной стороны, по тому, что именно мы хотим получить в качестве результата:
\begin{itemize}
\item Наличие пути, удовлетворяющего ограничениям, в графе.
\item Наличие пути, удовлетворяющего ограничениям, между некоторыми вершинами: задача достижимости. 
      Достижима ли вершина $v_1$ из вершины $v_2$ по пути, удовлетворяющему ограничениям.
      Такая постановка требует лишь проверить существование, но не обязательно предоставлять путь.
\item Поиск одного пути, удовлетворяющего ограничениям. Уже надо предъявлять путь.
\item Поиск всех путей. Надо предоставить все пути.
\end{itemize}

С другой стороны, задачи различаются ещё и по тому, как фиксируем вершины:
\begin{itemize}
\item из одной вершины до всех
\item между всеми парами вершин
\item межу фиксированной парой вершин
\item между двумя множествами вершин
\end{itemize}

Итого, можем сгенерировать прямое произведение различных постановок.

Ограничение, имеющее важное прикладное значение, --- минимальность длины. 
Иными словами, важная прикладная задача --- \textit{поиск кратчайших путей в графе (англ. APSP --- all-pairs shortest paths)}

Часто добавляют ограничения, что путь должен быть простым и другие.

\subsection{APSP и транзитивное замыкание графа}

Заметим, что отношение достижимости (\ref{def:reach}) является транзитивным.
Действительно, если существует путь из $v_i$ в $v_j$ и путь из $v_j$ в $v_k$, то существует путь из $v_i$ в $v_k$.

\begin{definition}
  \textit{Транзитивным замыканием графа} называется транзитивное замыкание отношения достижимости по всему графу.
\end{definition}

Как несложно заметить, транзитивное замыкание графа --- это тоже граф.
Более того, если решить задачу поиска кратчайших путей между всеми парами вершин, то мы построим транзитивное замыкание.
Поэтому сразу рассмотрим алгоритм Флойда-Уоршелла~\cite{Floyd1962, Bernard1959, Warshall1962}, который является общим алгоритмом поиска кратчайших путей (умеет обрабатывать рёбра с отрицательными весами, чего не может, например, алгоритм Дейкстры). Его сложность: $O(n^3)$, где $n$ --- количество вершин в графе.
При этом, данный алгоритм легко упростить до алгоритма построения транзитивного замыкания.

\begin{algorithm}
  \floatname{algorithm}{Listing}
\begin{algorithmic}[1]
\caption{Алгоритм Флойда-Уоршелла}
\label{lst:algoFloydWarxhall}
\Function{FloydWarshall}{$\mathcal{G}$}
    \State{$M \gets$ матрица смежности $\mathcal{G}$}
    \State{$n \gets$ $|V(\mathcal{G})|$}
    \For{k = 0; k < n; k++}
      \For{i = 0; i < n; i++}
        \For{j = 0; j < n; j++} 
          \State{$M[i,j] \gets$ min$(M[i,j], M[i,k] + M[k,j])$}
        \EndFor
      \EndFor
    \EndFor
\State \Return $M$
\EndFunction
\end{algorithmic}
\end{algorithm}


\begin{example}
  Пусть дан следующий граф:
  \begin{center}
  \begin{tikzpicture}[shorten >=1pt,on grid,auto]
     \node[state] (q_0)   {$0$};
     \node[state] (q_1) [above right = 1.4cm and 1cm of q_0] {$1$};
     \node[state] (q_2) [right = 2cm of q_0] {$2$};
     \node[state] (q_3) [right = 2cm of q_2] {$3$};
      \path[->]
      (q_0) edge  node {} (q_1)
      (q_1) edge  node {} (q_2)
      (q_2) edge  node {} (q_0)
      (q_2) edge[bend left, above]  node {} (q_3)
      (q_3) edge[bend left, below]  node {} (q_2);
  \end{tikzpicture}
  \end{center}
  
  Построим его транзитивное замыкание:
  \begin{center}
  \begin{tikzpicture}[shorten >=1pt,on grid,auto]
     \node[state] (q_0)   {$0$};
     \node[state] (q_1) [above right = 1.4cm and 1cm of q_0] {$1$};
     \node[state] (q_2) [right = 2cm of q_0] {$2$};
     \node[state] (q_3) [right = 2cm of q_2] {$3$};
      \path[->]
      (q_0) edge[loop below] node {} ()
      (q_1) edge[loop above] node {} ()
      (q_2) edge[loop below] node {} ()
      (q_3) edge[loop below] node {} ()
      
      (q_0) edge  node {} (q_1)
      (q_1) edge[bend right] node {} (q_0)
      (q_1) edge  node {} (q_2)
      (q_2) edge[bend right] node {} (q_1)
      (q_2) edge  node {} (q_0)
      (q_0) edge[bend right] node {} (q_2)
      (q_2) edge[bend left, above]  node {} (q_3)
      (q_3) edge[bend left, below]  node {} (q_2)
      (q_0) edge[bend right = 60]  node {} (q_3)
      (q_1) edge[bend left, above]  node {} (q_3);
  \end{tikzpicture}
  \end{center}
  \begin{remark}
    На самом деле, петли у вершины 3 может и не быть, т.к. это зависит от определения.
  \end{remark}
\end{example}

\begin{remark}
Приведем список интересных работ по APSP для ориентированных графов с вещественными весами (здесь $n$ --- количество вершин в графе):
\begin{itemize}
    \item M.L. Fredman (1976) --- $O(n^3(\log \log n / \log n)^\frac{1}{3})$ --- использование дерева решений~\cite{FredmanAPSP1976}
    \item W. Dobosiewicz (1990) --- $O(n^3 / \sqrt{\log n})$ --- использование операций на Word Random Access Machine~\cite{Dobosiewicz1990}
    \item T. Takaoka (1992) --- $O(n^3 \sqrt{\log \log n / \log n})$ --- использование таблицы поиска~\cite{Takaoka1992}
    \item Y. Han (2004) --- $O(n^3 (\log \log n / \log n)^\frac{5}{7})$~\cite{Han2004}
    \item T. Takoaka (2004) --- $O(n^3 (\log \log n)^2 / \log n)$~\cite{Takaoka2004}
    \item T. Takoaka (2005) --- $O(n^3 \log \log n / \log n)$~\cite{Takaoka2005}
    \item U. Zwick (2004) --- $O(n^3 \sqrt{\log \log n} / \log n)$~\cite{Zwick2004}
    \item T.M. Chan (2006) --- $O(n^3 / \log n)$ --- многомерный принцип ``разделяй и властвуй``~\cite{Chan2008}
    \item и др.
\end{itemize}
\end{remark}

\subsection{APSP и произведение матриц}
Алгоритм Флойда-Уоршелла можно переформулировать в терминах перемножения матриц. Для этого введём обозначение.


\begin{definition}
Пусть даны матрицы $A$ и $B$ размера $n \times n$. Определим операцию $\star$, которую называют \textit{Min-plus matrix multiplication}:

    $A \star B = C$ --- матрица размера $n \times n$, т.ч. 
    $C[i,j] = \min_{k = 1 \dots n} \{ A[i,k] + B[k,j] \}$
\end{definition}

Также, обозначим за $D[i,j](k)$ минимальную длину пути из вершины $i$ в $j$, содержащий максимум $k$ рёбер.
Таким образом, $D(1) = M$, где $M$ --- это матрица смежности, а решением APSP является $D(n-1)$, т.к. чтобы соединить $n$ вершин требуется не больше $n-1$ рёбер.

\begin{center}
    $D(1) = M$ \\
    $D(2) = D(1) \star M = M^2$ \\
    $D(3) = D(2) \star M = M^3$ \\
    $\dots$ \\
    $D(n-1) = D(n-2) \star M = M^{n-1}$ \\
\end{center}

Итак, мы можем решить APSP за $O(n K(n))$, где $K(n)$ --- сложность алгоритма умножения матриц.
Сразу заметим, что, например, $D(3)$ считать не обязательно, т.к. можем посчитать $D(4)$ как $D(2) \star D(2)$.
Поэтому:

\begin{center}
    $D(1) = M$ \\
    $D(2) = M^2 = M \star M$ \\
    $D(4) = M^4 = M^2 \star M^2$ \\
    $D(8) = M^8 = M^4 \star M^4$ \\
    $\dots$ \\
    $D(2^{\log(n-1)}) = M^{2^{\log(n-1)}} = M^{2^{\log(n-1)} - 1} \star M^{2^{\log(n-1)} - 1}$ \\
    $D(n-1) = D(2^{\log(n-1)})$ \\
\end{center}

Теперь вместо $n$ итераций нам нужно $\log{n}$. В итоге, сложность --- $O(\log{n} K(n))$.
Данный алгоритм называется \textit{repeated squaring}\footnote{Пример решения APSP с помощью repeated squaring: \url{http://users.cecs.anu.edu.au/~Alistair.Rendell/Teaching/apac_comp3600/module4/all_pairs_shortest_paths.xhtml}}.

\subsection{Умножение матриц с субкубической сложностью}
В предыдущем подразделе мы свели проблему APSP к проблеме min-plus matrix multiplication, поэтому взглянем на эффективные алгоритмы матричного умножения.

Сложность наивного произведения двух матриц составляет $O(n^3)$, это приемлемо только для малых матриц, для больших же лучше использовать алгоритмы с субкубической сложностью. 
Отметим, что мы называем сложность субкубической, если она равна $O(n^{3-\varepsilon})$, где $\varepsilon > 0$, иначе говоря, меньшей, чем $O(n^3)$.

Первый субкубический алгоритм опубликовал Ф. Штрассен в 1969 году, его сложность --- $O(n^{\log_2 7}) \approx O(n^{2.81})$~\cite{Strassen1969}. Основная идея --- рекурсивное разбиение на блоки $2 \times 2$ и вычисление их произведения с помощью только 7 умножений, а не 8.
Впоследствии алгоритмы усовершенствовались до ${\approx} O(n^{2.5})$~\cite{Pan1978,BiniCapoRoma1979,Schonhage1981,CoppWino1982}. В настоящее время наиболее асимптотически быстрым является алгоритм Копперсмита --- Винограда со сложностью $O(n^{2.376})$~\cite{CoppWino1990}.

Несмотря на то, что у приведенных алгоритмов неплохая алгоритмическая сложность, мы всё же не можем использовать их для вычисления min-plus matrix multiplication, так как в субкубических алгоритмах требуется, чтобы элементы образовывали кольцо. Покажем, что $\mathbb{R} \cup \{+\infty\}$ с операциями min и + являются полукольцом, а не кольцом:
\begin{enumerate}
    \item $min(a, b) = min(b, a)$
    \item $min(min(a, b)), c) = min(a, min(b, c)))$
    \item $min(a, +\infty) = min(+\infty, a) = a$, т.е. $+\infty$ --- нейтральный элемент относительно операции min
    
    \item $(a + b) + c = a + (b + c)$
    
    \item $min(a, b) + c = min(a + c, b + c)$
    \item $a + (B \vee c) = (a + b) \vee (a + c)$
    
    \item $a + \infty = \infty + a = \infty$
    \item Но для произвольного элемента a: $\nexists d$, т.ч. $min(a, d) = min(d, a) = +\infty$, т.е. для любого элемента нет обратного относительно операции min
\end{enumerate}

Таким образом, вопрос о субкубических алгоритмах решения APSP всё ещё открыт~\cite{Chan2010}.
Кроме того, более простая задача APSP с булевыми матрицами также не решена за субкубическую сложность. Приведем обоснование этого факта.

\begin{definition}
  Матрица называется \textit{булевой}, если она состоит из 0 и 1.
\end{definition}

Булевы матрицы с поэлементными операциями дизъюнкции и конъюнкции являются полукольцом. Покажем это: пусть $A$, $B$ и $C$ --- булевы матрицы, тогда:
\begin{enumerate}
    \item $A \vee B = B \vee A$
    \item $(A \vee B) \vee C = A \vee (B \vee C)$
    \item $A \vee N = N \vee A = A$, где $N$ --- матрица, полностью состоящая из 0
    
    \item $(A \wedge B) \wedge C = A \wedge (B \wedge C)$
    
    \item $(A \vee B) \wedge C = (A \wedge C) \vee (B \wedge C)$
    \item $A \wedge (B \vee C) = (A \wedge B) \vee (A \wedge C)$
    
    \item $A \wedge N = N \wedge A = N$
\end{enumerate}

Булевы матрицы тоже не являются кольцом, т.к. не имеют обратный элемент относительно операции дизъюнкции (т.е. для произвольной булевой матрицы $A$: $\nexists D$, т.ч. $D$ --- булева матрица и $A \vee D = D \vee A = N$). Следовательно, субкубические алгоритмы не подходят для перемножения булевых матриц, т.к. в них используются обратные элементы (например, для разности). В частности, они не применимы к классической матрице смежности, которая ведёт себя как булева матрица.

\subsection{Вопросы и задачи}
\begin{enumerate}
  \item Реализуйте алгоритм построения транзитивного замыкания через матрицы.
  \item Реализовать матрицы самим.
  \item Взять готовую библиотеку матричных операций: CPU, GPGPU.
  \item Реализуйте поиск кратчайших путей через матричные операции.
  \item Взять готовую библиотеку матричных операций: CPU, GPGPU.
\end{enumerate}