\section{Introduction}


Context-Free Path Querying (CFPQ) is a sublcass of Language-constrained path problem, where language is set to be Context-Free.

Importance of CFPQ. Application areas. RDF, Graph database querying, Graph Segmentation in Data provenance, Biological data analysis, static code analysis.

\subsection{An Example}

Example of graph and query. Should be used in explataion below.

\subsection{Existing CFPQ Algorithms}

Number of problem-specific solutions in static code analysis.

Hellings, Ciro et al, Kujpers, Sevon, Verbitskaya, Azimov, Ragozina 

\subsection{Existing Theoretical Results}

Existing theoretical results

Linear input. Valiant~\cite{Valiant:1975:GCR:1739932.1740048}, Lee~\cite{Lee:2002:FCG:505241.505242}.

Yannacacis~\cite{Yannakakis}? Reps?

Bradford~\cite{8249039}

RSM~\cite{10.1145/1328438.1328460}.

C alias analysis~\cite{10.1145/2714064.2660213}

Chatterjee~\cite{10.1145/3158118} 

For trees

Truly-subcubic for Language Editing Distance~\cite{doi:10.1137/17M112720X}.


Truly-subcubic algorithm is stil an open problem.

\subsection{Our Contribution}


This paper is organized as follows. !!!!!

