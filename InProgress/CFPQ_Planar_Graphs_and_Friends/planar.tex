\section{CFPQ on planar digraphs}

We want to consider algorithms for semi dynamic transitive closure for the case of planar graph. Our algorithm needs to support only edge insertions.

There are several algorithms that look into the problem of dynamic reachability in planar digraphs and have sub linear both update and query time~\cite{subramanian1993fully}, \cite{karczmarz2018data}, \cite{kao2008encyclopedia}. 

Some algorithms \cite{diks2007dynamic}, \cite{karczmarz2018data} restrict edge insertions so that they respect the specific embedding of the planar graph. 

\todo[inline]{maybe we can choose one embedding at random and hope that nothing will violate it. So it will be monte-carlo (why it will have good probability? where from can we get all the possible non-isomorphic embeddings?)}  

The algorithms that allow edge insertions not with respect to the embedding of the graph, but to the planarity, are described in \cite{subramanian1993fully}, \cite{galil1999fully}. The main idea of both articles is the same: planar graph in separated into \textit{clusters} - edge induced sub-graphs - for which reachability is shown by their sparse substitution $S$. \textit{Sparse substitution} is a graph that contains the reachability information between some set of vertices that lie in the outer face of the graph.

We show the results of \cite{subramanian1993fully} more closely as they differ from the results of \cite{galil1999fully} only by logarithmic factor, have the same idea and are easier for understanding. 

Updates and queries are the following: adding or deleting the edge between vertices $u, v$, checking reachability, checking if new edge $(u, v)$ violates planarity. After splitting graph into clusters and looking into its sparse substitute update or query can be done by placing into the graph of sparse substitutes clusters of the corresponding vertices $u, v \in V$. 

We only need planarity for each cluster and will build an embedding for each of them at the beginning and after every $\mathcal{O}(n^{\frac{1}{3}})$ edge-insertions. 

The results are the following: the amount of time for preprocessing is $\mathcal{O}(n \log n)$ - in this time we find separation of the given graph $G$ into $\mathcal{O}(n^{\frac{1}{3}} \log n)$ clusters and build sparse substitution sub-graph $S$ for this clusters. After that we can perform add operation in $\mathcal{O}(n^{\frac{2}{3}} \log n)$ amortized time, delete operation in $\mathcal{O}(n^{\frac{2}{3}} \log n)$ worst-case time, check if graph is planar in $\mathcal{O}(n^{\frac{1}{2}})$ amortized time. 

\begin{theorem}
We can maintain the structure described in \cite{subramanian1993fully} not only for planar graphs, but for planar graphs with $\mathcal{O}(n^{\frac{1}{3}})$ edges that violate planarity. 
\end{theorem}

\begin{proof}
In \cite{subramanian1993fully} number of clusters for each of which we need planarity is $\mathcal{O}(n^{\frac{1}{3}})$. After every $\mathcal{O}(n^{\frac{1}{3}})$ edge insertions we rebuild sparse substitute graph $S$. It takes $\mathcal{O}(n^{\frac{2}{3}}\log n)$ time, that is distributed over all $\mathcal{O}(n^{\frac{1}{3}})$ insertions. We can additionally maintain the set of non-planar edges, each edge of this set will present its own cluster and will not take part in rebuilding of sparse substitution. Edge is \textit{non-planar} if its insertion to current planar graph $G$ makes in non-planar.
\end{proof}

Let us look closely on how can we use the power of sparse substitution on our incremental transitive closure problem.

\begin{theorem}
We can add edges in graph and print out new pairs of reachable vertices for every insertion in $\mathcal{O}(n^{\frac{5}{3}})$ total time for every sequence of insertions if our model allow parallel computations.
\end{theorem}

\begin{proof}
From \cite{subramanian1993fully} we can add an edge in $\mathcal{O}(n^{\frac{2}{3}})$ amortized time. After that we can run DFS from the ends $u, v$ of the edge: DFS from $v$ and DFS on reversed edge from $u$. We want to get all pairs of vertices that are now connected through the new edge. DFS runs in time proportional to the size of the graph of sparse substitution $S$ - $\mathcal{O}(n^{\frac{2}{3}} \log n)$. After that for every cluster in original graph $G$ we create dummy vertex $s$ and edges from $s$ to all boundary vertices in cluster and run DFS from this dummy vertex $s$. If any vertex $w$ is reachable from $v$ (without loss of generation), then it is a boundary vertex of lie in some cluster and is reachable from some boundary vertex of this cluster. In both cases one of DFS's will print it out.

If we can run these DFS's in parallel (there are $\mathcal{O}(n^{\frac{1}{3}})$ clusters and the same number of DFS's), then each of them will take $\mathcal{O}(n^{\frac{2}{3}})$ amount of time (all cluster have $\mathcal{O}(n^{\frac{2}{3}})$ edges by definition in \cite{subramanian1993fully}). In this case total amount of time will be $\mathcal{O}(n^{\frac{2}{3}})$.
\end{proof}

If our model can not run algorithms in parallel, then the amount of time taken by iterating these DFS's for every cluster is linear - $\mathcal{O}(n)$. This means that usual DFS on the original graph $G$ has the same complexity and we will spend in total $\mathcal{O}(n^2)$ time and that planarity does not get us any advantage in solving the problem of dynamic reachability (in amortized time per edge and query). 

\todo[inline]{maybe we can spare some space ($\mathcal{O}(n^2)$?) so we can store list of reachable vertices from any boundary vertex and somehow update them during the edge additions?}
