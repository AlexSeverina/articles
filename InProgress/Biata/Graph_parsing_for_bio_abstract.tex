\documentclass[12pt]{article}  % standard LaTeX, 12 point type
\usepackage{amsfonts,latexsym}
\usepackage{amsthm}
\usepackage{amssymb}
\usepackage[utf8x]{inputenc} % Кодировка
\usepackage[english]{babel} % Многоязычность

\newtheorem{theorem}{Theorem}[section]
\newtheorem{proposition}[theorem]{Proposition}
\newtheorem{lemma}[theorem]{Lemma}
\newtheorem{corollary}[theorem]{Corollary}
\newtheorem{conjecture}[theorem]{Conjecture}

\theoremstyle{definition}
\newtheorem{definition}{Определение}[section]
\newtheorem{example}{Example}[section]

% unnumbered environments:

\theoremstyle{remark}
\newtheorem*{remark}{Remark}
\newtheorem*{notation}{Notation}
\newtheorem*{note}{Note}

\setlength{\parskip}{5pt plus 2pt minus 1pt}
%\setlength{\parindent}{0pt}

\usepackage{color}
\usepackage{listings}
\usepackage{caption}
\usepackage{graphicx}
\usepackage{ucs}

\newcommand{\tab}[1][0.3cm]{\ensuremath{\hspace*{#1}}}
% A generalized view on parsing and translation
% http://dl.acm.org/citation.cfm?id=2206331
\title{Graph parsing application for bio problems}
% Context-free path querying ...
\author{Semyon Grigorev, Artem Gorokhov
\\
       {Saint Petersburg State University}\\
       {7/9 Universitetskaya nab.}\\
       {St. Petersburg, 199034 Russia}\\
       semen.grigorev@jetbrains.com, gorohov.art@gmail.com
       }
\date{}

\begin{document}

\maketitle

Biomedical databases contain huge amounts of rich data which can be represented as a labeled graph.
In order to investigate such data, it may be useful to extract connections with specific constraints.
One of natural way to provide constraints is specify language of paths' labels, which can be done by using different classes of grammars.
For example, one can use context-free grammars as one of powerful context-free path querying.
%of classical examples is a context-free path querying for graph data bases where input is a graph and path constraints are specified by a grammar.

One of examples is a graph where vertices correspond to entities and concepts such as gene, phenotype, and edges represent known relationships such as ``codes for'', ``interacts with'', etc.
Paths with special constraints may provide information about links between vertices were unknown before, forming the basis for new hypotheses.

Another example of graph structured data is metagenomic assemblies.
Secondary structure can be described in terms of context-free grammar (Eddy et al), and grammar can be used for finding and classification.
But only for linear data. 
dispired the fact of tools existing, Graph structured data processing is still a challenge
Context-free pattern search in metagenomic assemblies. 

%Solution of problems described defore can be based on is a context-free path querying for graph data bases where input is a graph and path constraints are specified by a context-free grammar.

We have some experience in graph parsing~\cite{GraphGLL, RelaxedRNGLR}.
GLL-based context-free path querying algorithm~\cite{GraphGLL} implemented by the authors is faster than solution which was presented at ISWC-2016~\cite{CFRDFParsing}. 
We have some ideas of graph parsing applications in bio data analysys.
Existing solution have problems (earley --- cycles), Metagenomic analysys -- GPGPU and manycore 
Create applications for biology based on our experience.

\begin{thebibliography}{9}

%UniProt Consortium et al. UniProt: a hub for protein information //Nucleic acids research. – 2014. – С. gku989.

\bibitem{Earley}
  Sevon, Petteri, and Lauri Eronen.
  ``Subgraph queries by context-free grammars.''
  \emph{Journal of Integrative Bioinformatics (JIB)}
  5.2 (2008): 157-172.

\bibitem{GraphGLL}
  Grigorev, Semyon, and Anastasiya Ragozina. 
  ``Context-Free Path Querying with Structural Representation of Result.''
   \emph{arXiv preprint arXiv:1612.08872}
    (2016).

\bibitem{GLL}
  Scott, Elizabeth, and Adrian Johnstone.   
  ``GLL parsing.'',
  \emph{Electronic Notes in Theoretical Computer Science},
  253.7 (2010): 177--189.

\bibitem{RelaxedRNGLR}
  Verbitskaia, Ekaterina, Semyon Grigorev, and Dmitry Avdyukhin.
  ``Relaxed Parsing of Regular Approximations of String-Embedded Languages.''
  \emph{International Andrei Ershov Memorial Conference on Perspectives of System Informatics.}
  Springer International Publishing, 2015.

\bibitem{CFRDFParsing}
  Zhang, Xiaowang, et al.
  ``Context-free path queries on RDF graphs.'' 
  \emph{International Semantic Web Conference.}
   Springer International Publishing, 2016.
   632--648.

\end{thebibliography}


\end{document}