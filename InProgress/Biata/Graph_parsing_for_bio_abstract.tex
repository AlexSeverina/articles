\documentclass[12pt]{article}  % standard LaTeX, 12 point type
\usepackage{amsfonts,latexsym}
\usepackage{amsthm}
\usepackage{amssymb}
\usepackage[utf8x]{inputenc} % Кодировка
\usepackage[english]{babel} % Многоязычность

\newtheorem{theorem}{Theorem}[section]
\newtheorem{proposition}[theorem]{Proposition}
\newtheorem{lemma}[theorem]{Lemma}
\newtheorem{corollary}[theorem]{Corollary}
\newtheorem{conjecture}[theorem]{Conjecture}

\theoremstyle{definition}
\newtheorem{definition}{Определение}[section]
\newtheorem{example}{Example}[section]

% unnumbered environments:

\theoremstyle{remark}
\newtheorem*{remark}{Remark}
\newtheorem*{notation}{Notation}
\newtheorem*{note}{Note}

\setlength{\parskip}{5pt plus 2pt minus 1pt}
%\setlength{\parindent}{0pt}

\usepackage{color}
\usepackage{listings}
\usepackage{caption}
\usepackage{graphicx}
\usepackage{ucs}

\newcommand{\tab}[1][0.3cm]{\ensuremath{\hspace*{#1}}}
% A generalized view on parsing and translation
% http://dl.acm.org/citation.cfm?id=2206331
\title{Graph parsing application for bio problems}
% Context-free path querying ...
\author{Semyon Grigorev, Artem Gorokhov
\\
       {Saint Petersburg State University}\\
       {7/9 Universitetskaya nab.}\\
       {St. Petersburg, 199034 Russia}\\
       semen.grigorev@jetbrains.com, gorohov.art@gmail.com
       }
\date{}

\begin{document}

\maketitle

Biomedical databases contain huge amounts of rich data which can be represented as a labeled graph.
In order to investigate such data, it may be useful to extract connections with specific constraints.
One of natural way to provide constraints is specify language of paths' labels, which can be done by using different classes of grammars.
For example, one can use context-free grammars with productions $\{S \rightarrow a S b; S \rightarrow \varepsilon \}$ to find paths which labels should looks like $ab; aabb; aaabbb; ...$, or, generally, should contains in language $L = \{a^n b^n, n \geq 0\}$.
This approach is named \emph{context-free path querying} and can be useful in some bioinformatic applications.

One of examples is an analysis of graphs where vertices correspond to entities and concepts such as gene, phenotype, and edges represent known relationships such as ``codes for'', ``interacts with'', etc (UniProt~\cite{UniProt} dataset, for example).
Paths with special constraints may provide information about links between vertices were unknown before, forming the basis for new hypotheses.

Another example of graph structured data is metagenomic assemblies, and one of problem is long subsequences detection and reconstruction.
Some sequences have specific secondary structure, which can be described in terms of context-free grammar, and this grammar can be used for finding and classification.
There is a big number of research in this area and tools based on this approach, but most of them aimed on linear data processing. 
Despite the fact of existence tools for metagenomic assemblies analysis, context-free search in graph structured assembly is still a challenge.
%Context-free pattern search in metagenomic assemblies can be used for sequences detection. 

%Solution of problems described defore can be based on is a context-free path querying for graph data bases where input is a graph and path constraints are specified by a context-free grammar.

Examples above and other applications can be implemented on one techniques: graph parsing --- classical parsing techniques for graphs.
We have some experience in graph parsing~\cite{GraphGLL, RelaxedRNGLR}.
Not only theory but some implemented algorithms.
Existing solution have problems (earley --- cycles).
We want to create applications for biology based on our experience. 
we are working on 16s detection and connection.
We want to find other applications for this techniques.
Metagenomic analysis -- GPGPU and manycore. 


\begin{thebibliography}{9}

\bibitem{UniProt}
  UniProt Consortium et al.
  ``UniProt: a hub for protein information.''
  \emph{Nucleic acids research.}
  (2014).

\bibitem{Earley}
  Sevon, Petteri, and Lauri Eronen.
  ``Subgraph queries by context-free grammars.''
  \emph{Journal of Integrative Bioinformatics (JIB)}
  5.2 (2008): 157-172.

\bibitem{GraphGLL}
  Grigorev, Semyon, and Anastasiya Ragozina. 
  ``Context-Free Path Querying with Structural Representation of Result.''
   \emph{arXiv preprint arXiv:1612.08872}
    (2016).

%\bibitem{GLL}
%  Scott, Elizabeth, and Adrian Johnstone.   
%  ``GLL parsing.'',
%  \emph{Electronic Notes in Theoretical Computer Science},
%  253.7 (2010): 177--189.

\bibitem{RelaxedRNGLR}
  Verbitskaia, Ekaterina, Semyon Grigorev, and Dmitry Avdyukhin.
  ``Relaxed Parsing of Regular Approximations of String-Embedded Languages.''
  \emph{International Andrei Ershov Memorial Conference on Perspectives of System Informatics.}
  Springer International Publishing, 2015.

%\bibitem{CFRDFParsing}
%  Zhang, Xiaowang, et al.
%  ``Context-free path queries on RDF graphs.'' 
%  \emph{International Semantic Web Conference.}
%   Springer International Publishing, 2016.
%   632--648.

\end{thebibliography}


\end{document}