\section{Introduction}

Concurrent systems are widely spread and its verification is a nontrivial and important problem.
There are a lot of papers that describe concurrent programs behavior via Push Down Systems or Context-Free languages~\cite{gange2015tool, bouajjani2003generic, chaki2006verifying, ModelPDA1}, and our interest is around a \textit{shuffle} of Context-Free Languages (CFL)~\cite{CFLShuffle}.
This languages describe the interleaving of CFLs (or PDA) and look perfect to describe the interleaved behavior of concurrent programs.

First of all we introduce the notion of \textit{shuffle} operation ($\odot$), that can be defined for sequences as follows:
\begin{itemize}
    \item $\varepsilon \odot u = u \odot \varepsilon = {u}$, for every sequence $ u \in \Sigma^*$;
    \item $\alpha_1 u_1 \odot \alpha_2 u_2 = \{\alpha_1 w | w \in (u_1 \odot \alpha_2 u_2) \} \cup \newline
    \{\alpha_2 w | w \in (\alpha_1 u1 \odot u_2 ) \},  \forall \alpha_1, \alpha_2 \in \Sigma$ and $\forall u_1, u_2 \in \Sigma^*$.
\end{itemize}
For example, $"ab" \odot "123" = \{a123b,a1b23, 12ab3, 123ab, etc.\}$.

Shuffle can be extended to languages as $$L_1 \odot L_2 = \bigcup\limits_{u_1\in L_1, u_2\in L_2} u_1 \odot u_2.$$

We can describe required aspects of behavior of functions (or methods, or subsystems) $f_1, f_2 ... f_n$ from our system $\mathcal{S}$ that run concurrently as shuffle of context-free languages $L_{f_1}, L_{f_2} ... L_{f_n}$ generated for each of them.
As a result, language  $\mathcal{L} = L_{f_1} \odot L_{f_2} \odot... \odot L_{f_n}$ over alphabet $\Sigma$ describes all possible executions of our system.
If we want to check a correctness of $\mathcal{S}$, then we should check whether $\mathcal{L}$ contains any ``bad execution''.
Let suppose that the set of bad executions can be described by some regular language $R_1$ over the same alphabet $\Sigma$.
Now we should inspect an intersection $\mathcal{L} \cap R_1$ --- its emptiness means that $\mathcal{S}$ can not demonstrate bad behavior.

The idea described above is used in the paper~\cite{stenman2011approximating}. 
As far as shuffled context-free languages are not closed under intersection with the regular one~\cite{CFLShuffle} and the problem of defining either string is in the shuffle of CFL is NP-Complete,
 authors use a context-free approximation of shuffle of CFL and intersect it with error traces, but since the approximation was used this approach didn't found some of known bugs. 

While NP-completeness may looks like death warrant, there are SAT-solvers which deal with NP problems very successfully.
In this paper we show how to reduce emptiness checking of shuffled CFL and finite regular language intersection to SAT.
Our reduction is very native and use some classical parsing techniques.
Generalization for arbitrary regular language is a topic for future research.
