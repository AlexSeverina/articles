\section{Conclusion}
\label{sec:conclusion}

We present mechanized in Coq proof of the Bar-Hillel theorem, the fundamental theorem on the closure of context-free languages under intersection with the regular set.
By this, we increase mechanized part of formal language theory and provide a base for reasoning about many applicative algorithms which are based on languages intersection.
We generalize the results of Gert Smolka and Jana Hofmann: the definition of the terminal and nonterminal alphabets in context-free grammar were made generic, and all related definitions and theorems were adjusted to work with the updated definition.
It makes previously existing results more flexible and eases reusing.
All results are published at GitHub and are equipped with automatically generated documentation.

The first open question is the integration of our results with other results on formal languages theory mechanization in Coq. 
There are two independent sets of results in this area: works of Ruy de Queiroz and works of Gert Smolka.
We use part of Smolka's results in our work, but even here we do not use existing results on regular languages.
We believe that theory mechanization should be unified and results should be generalized.
We think that these and other related questions should be discussed in the community.

One direction for future research is mechanization of practical algorithms which are just implementation of the Bar-Hillel theorem.
For example, context-free path querying algorithm, based on CYK~\cite{hellingsPathQuerying,zhang2016context} or even on GLL~\cite{scott2010gll} parsing algorithm~\cite{grigorev2016context}.
Final target here is the certified algorithm for context-free constrained path querying for graph databases.

Another direction is mechanization of other problems on language intersection which can be useful for applications.
For example, the intersection of two context-free grammars one of which describes finite language~\cite{nederhof2002parsing, nederhof2004language}.
It may be useful for compressed data processing~\cite{Lohrey2012AlgorithmicsOS} or speech recognition~\cite{Nederhof:2002:PNC:1073083.1073104,NEDERHOF2004172}.
And we believe all these works should share the common base of mechanized theoretical results.



