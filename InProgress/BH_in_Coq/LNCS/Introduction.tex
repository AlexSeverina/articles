\section{Introduction}

Formal language theory has a deep connection with different areas such as static code analysis~\cite{Reps:1995:PID:199448.199462,vardoulakis2010cfa2,Yan:2011:DCA:2001420.2001440,rehof2001type,lu2013incremental,pratikakis2006existential,zhang2017context}, graph database querying~\cite{hellingsRelational,hellingsPathQuerying,zhang2016context,koschmieder2012regular}, formal verification~\cite{10.1007/11730637_17,10.1007/3-540-63141-0_10}, and others.
One of the most frequent uses is to formulate a problem in terms of languages intersection.
In verification, one language can serve as a model of a program and another language describe undesirable behaviors.
When the intersection of these two languages is not empty, one can conclude that the program is incorrect.
Usually, the only concern is the decidability of the languages intersection emptiness problem.
But in some cases, a constructive representation of the intersection may prove useful.
This is the case, for example, when the intersection of the languages models graph querying: a language produced by intersection is a query result and to be able to process it, one needs the appropriate representation of the intersection result.

Let us consider several applications starting with the user input validation.
The problem is to check if the input provided by the user is correct with respect to some validation template such as a regular expression for e-mail validation.
User input can be represented as a one word language.
The intersection of such a language with the language specifying the validation template is either empty or contains the only string: the user input.
If the intersection is empty, then the input should be rejected.

Checking that a program is syntactically correct is another example.
The AST for the program (or lack thereof) is just a constructive representation of the intersection of the one-word language (the program) and the programming language itself.

Graph database regular querying serves as an example of the intersection of two regular languages~\cite{ABITEBOUL1999428,koschmieder2012regular,alkhateeb:tel-00293206}.
Next and one of the most comprehensive cases with decidable emptiness problem is an intersection of a regular language with a context-free language.
This case is relevant for program analysis~\cite{Reps:1995:PID:199448.199462,vardoulakis2010cfa2,Yan:2011:DCA:2001420.2001440}, graph analysis~\cite{hellingsPathQuerying,zhang2016context,grigorev2016context}, context-free compressed data processing~\cite{MANETH201819}, and other areas.
The constructive intersection representation in these applications is helpful for further analysis.

The intersection of some classes of languages is not generally decidable.
For example, the intersection of the linear conjunctive and the regular languages, used in the static code analysis~\cite{zhang2017context}, is undecidable while multiple context-free languages (MCFL) is closed under intersection with regular languages and emptiness problem for MCFLs is decidable~\cite{SEKI1991191}.
Is it possible to express any useful properties in terms of regular and multiple context-free languages intersection?
This question is beyond the scope of this paper but provides a good reason for future research in this area.
Moreover, the history of pumping lemma for MCFG shows the necessity to mechanize formal language theory.
In this paper, we focus on the intersection of regular and context-free languages.

Some applications mentioned above require certifications.
For verification this requirement is evident.
For databases it is necessary to reason about security aspects and, thus, we should create certified solutions for query executing.
Certified parsing may be critical for secure data loading (for example in Web), as well as certified regular expressions for input validation.
As a result, there is a significant number of papers focusing on regular expressions mechanization and certification~\cite{10.1007/978-3-319-03545-1_7}, and a number on certified parsers~\cite{10.1007/978-3-642-00590-9_12,firsov2014certified,Gross2015ParsingPA}.
On the other hand, mechanization (formalization) is important by itself as theoretical results mechanization and verification, and there is a lot of work done on formal languages theory mechanization~\cite{firsov2015certified,ramos2016formalization,1885-16399}.
Also, it is desirable to have a base to reason about parsing algorithms and other problems of languages intersection.

Context-free languages are closed under intersection with regular languages.
It is stated as the Bar-Hillel theorem~\cite{bar1961formal} which provides a constructive proof and construction for the resulting language description.
We believe that the mechanization of the Bar-Hillel theorem is a good starting point for certified application development and since it is one of the fundamental theorems, it is an important part of formal language theory mechanization.
And this work aims to provide such mechanization in Coq.

Our current work is the first step: we provide mechanization of theoretical results on context-free and regular languages intersection.
We choose the result of Gert Smolka and Jana Hofmann on context-free languages mechanization~\cite{smolkaHofmann2016} as a base for our work.
The main contribution of this paper may be summarized as follows.
\begin{itemize}
\item We provide the constructive proof of the Bar-Hillel theorem in Coq.
\item We generalize results of Gert Smolka: alphabets (nonterminals and terminals) in context-free grammar definition are changed from Nat to generic, and all the code affected by this change is also modified to work with the updated definition.
\item All code is published on GitHub: \url{https://github.com/YaccConstructor/YC_in_Coq}.
\end{itemize}

%This work is organized as follows.
%In section~\ref{sec:b-h-th} we formulate Bar-Hillel theorem and provide the sketch of its proof.
%The next part is a brief discussion of the Chomsky normal form in section~\ref{sec:cnf}.
%After that, we describe our solution in section~\ref{sec:main}.
%This description is split into steps with respect to provided sketch and contains basic definitions, Smolka's results generalization, handling of trivial cases, and steps summarization as final proof.
%Finally, we discuss related works in section~\ref{sec:rel-work} and conclude with the discussion of the presented work and possible directions for future research in section~\ref{sec:conclusion}.

