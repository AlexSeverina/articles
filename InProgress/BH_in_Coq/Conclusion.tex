\section{Conclusion}

We present mechanized in Coq proof of the Bar-Hillel theorem --- the fundamental theorem on closure of context-free languages under intersection with regular set.
By this we increase mechanized part of formal language theory and provide a base for reasoning about many applicative algorithms which are based on languages intersection.
Also we generalize results of Gert Smolka and Jana Hofmann: generalized terminal alphabet. 
It makes previously existing results more flexible and ease for reusing.
All results are published at GitHub and equipped with automatically generated documentation.

The first open question, and seams that very important questuin, is integration of our results with other results on foral languages theory mechanization in Coq. 
There are two independent sets of results in this area: works of Ruy de Queiroz and works of Gert Smolka.
We use part of Smolka's results in our work, but even here we do not use existing results on regular languages.
We think that theoy mechanization should be unified and results should be generalized.
We think that these and other related questions shoild be discussed in community.

One of direction of future research is mechanization of practical algorithms which are just implementation of Bar-Hillel theorem.
For example, context-free path querying algorithm, based on CYK~\cite{hellingsPathQuerying,zhang2016context} or even on GLL~\cite{scott2010gll} parsing algorithm~\cite{grigorev2016context}.
Final target here is certified algorithm for context-free constrained path querying for graph databases.

Yet another direction is mechanization of other problems on language intersection which can be useful for applications.
For example, intersection of two context-free grammars one of which describes finite language~\cite{nederhof2002parsing, nederhof2004language}.
It may be useful for compressed data processing~\cite{!!!} or speech recognition~\cite{!!!}.
And, of course, all these works should be done on the common base of mechanized theoretical results.



