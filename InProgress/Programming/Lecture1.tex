\section{Лекция 1: Введение}

Программирование --- не только написание кода.
Документация, сборка, тестирование, версионирование, обработка отзывов пользователей.

Инфраструктура проекта, рабочее окружение, система контроля версий, непрерывная сборка.

Соответствующие решения на примере инфраструктуры вокруг GitHub. GithubActions, внешние сервисы для CI (\url{https://travis-ci.org/}, \url{https://www.appveyor.com/}, \url{https://circleci.com/}). 

Практика развёртывания соответствующей инфраструктуры. 
\begin{enumerate}
  \item Для начала, завести аккаунт на GitHub (\url{https://github.com/}). 
Важно, чтобы имя аккаунта (логин) было \texttt{NameSurname} или \texttt{Name\_Surname}.
   
  \item Создаём репозиторий для проекта (для всех домашних работ). Название должно отражать сожержимое репозитория. Не забываем добавить описание. Лицензию, readme и gitignore лучше не добавлять.
  \item Устанавливаем git (\url{https://git-scm.com/}) и графическую оболочку для работы с ним (если кому нужно).
  \item Теперь пора приступать к созданию проекта.
Так как дальше мы будем пользоваться F\#, то в качестве шаблона предлагается использовать \url{https://github.com/TheAngryByrd/MiniScaffold}. 
\begin{enumerate}
  \item Установить .NET Core: \url{https://dotnet.microsoft.com/download}
  \item Прочитать инструкции (\href{https://github.com/TheAngryByrd/MiniScaffold#install-the-dotnet-template-from-nuget}{https://github.com/TheAngryByrd/MiniScaffold\#install-the-dotnet-template-from-nuget}) и выполнить соостветсвующие шаги. Нам нужно создать консольное приложение. Это может занять некоторое время.
\end{enumerate}
\item Устанавливаем связь только что созданного локального репозитория с удалённым репозиторием: \href{https://docs.github.com/en/github/importing-your-projects-to-github/adding-an-existing-project-to-github-using-the-command-line}{https://docs.github.com/en/github/importing-your-projects-to-github/adding-an-existing-project-to-github-using-the-command-line}

\end{enumerate}

С этого момента домашние работы только через GitHub с налаженной сборкой.