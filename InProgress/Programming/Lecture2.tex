\section{Лекция 2}

\subsection{Первое знакомство с F\#}

    Общие сведения о платформе .NET, общие представления об архитектуре платформы.

    F\# --- это один из языков платформы. 
    Некоторые ресурсы дл того, чтобы начать изучать F\#.
    \begin{itemize}
    	\item Главный сайт сообщества: \url{https://fsharp.org/}. Там много чего интересного, от библиотек и инструментов до ссылок на другие ресурсы. 
    	\item Неплохая книга по основам языка: \url{https://en.wikibooks.org/wiki/F_Sharp_Programming}
    	\item Подборка обучающих материалов от сообщества: \url{https://fsharp.org/learn/}
    	\item Официальная документация от Microsoft: \url{https://docs.microsoft.com/ru-ru/dotnet/fsharp/}
    \end{itemize}
    Основные особенности F\#, примеры кода, базовые языковые конструкции и типы.

    Примитивные типы: логисекие, числовые, символы, строки.
    Массивы, основные функции работы с ними: инициализация, взятие элемента, запись элемента.

    Базовые констукции управления: ветвления (if), циклы (различные варианты for, while). 
    Двумерный синтаксис.

    Структура программы. Точка входа, модули, функции. Модуль и пространство имён.

    Основы работы с консолью, библиотека \href{https://fsprojects.github.io/Argu/}{Argu}. 

\subsection{Тестирование программ}
    Тестирование программ: ручное, автоматизированное, автоматическое. 

    Доказательство корректности vs тестирование. Тестирование не есть доказательство корректности.
    А можно ли всё таки доказать корректность? Да (иногда).
    \begin{itemize}
    	\item \href{https://en.wikipedia.org/wiki/Formal_verification#Formal_verification_for_software}{Формальная верификация} используя внешние инструменты.
        \item Использование специальных языков программирования, таких как \href{https://coq.inria.fr/}{Coq} (на самом деле это целая система, так называемый proof assistant), \href{https://wiki.portal.chalmers.se/agda/pmwiki.php}{Agda}, \href{https://www.idris-lang.org/}{Idris}, \href{https://www.fstar-lang.org/}{F*},   
    \end{itemize}

    Типы тестов и особенности их применения: модульные, интеграционные, unit и т.д. Автоматизация создания тестов. \href{https://docs.microsoft.com/en-us/visualstudio/test/intellitest-manual/?view=vs-2019}{Intellitest} --- пример инстумента для автоматической генерации тестов. Примеры инструментов для тестирования: \href{https://github.com/haf/expecto}{Expecto}, \href{http://fsprojects.github.io/FsUnit/index.html}{FsUnit}, \href{https://nunit.org/}{NUnit}, \href{https://fscheck.github.io/FsCheck/}{FsCheck}, \href{http://lefthandedgoat.github.io/canopy/}{Canopy}. 

    С этого момента все домашние работы должны быть снабжены автоматически запускаемыми при сборке тестами.
