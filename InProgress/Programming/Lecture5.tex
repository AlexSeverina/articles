\section{Лекция 5}
 
Основы анализа алгоритмов. Модель вычислителя. Понятие элементарной операции. Асимптотика, ``О''-символика.

Постановка эксперимента и оформление результатов. Эксперимент по сравнению и анализу производительности. Точность проведения замеров. ``Масштабы времени'', цель эксперимента и точность измерений, инструменты измерений. Базовая статистическая обработка данных: медиана и среднее, выбросы, распределение. Проверка гипотез.  

Технические средства.
О системе вёрски \LaTeX. \href{https://www.tug.org/texlive/}{TexLive}, \href{https://www.overleaf.com/}{Overleaf}.
Способы визуализации результатов. Python + \href{https://matplotlib.org/3.3.1/index.html}{Matplotlib}, другие библиотеки. \href{http://www.gnuplot.info/}{GnuPlot}.  

Отдельно про скрипты сборки. Теперь мы руками попробуем сделать свой маленький скрипт. Shell, cmd (PowerShell).