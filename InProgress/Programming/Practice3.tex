\section{Домашняя работа 3}

Для всех задач обеспечить чтение $n$ из консоли и печать результата в консоль.

\begin{enumerate}
    \item \textbf{(1 балл)} Реализовать вычисление $n$-ого числа Фибоначчи рекурсивным методом. 
    \item \textbf{(1 балл)} Реализовать вычисление $n$-ого числа Фибоначчи итеративным методом. 
    \item \textbf{(1 балл)} Реализовать вычисление $n$-ого числа Фибоначчи используя хвостовую рекурсию (не используя \texttt{mutable} и других изменяемых структур). Подсказка: нужно использовать рекурсию с аккумулятором.
    \item \textbf{(2 балла)} Реализовать вычисление $n$-ого числа Фибоначчи через перемножение матриц ``наивным'' методом. Функции построения единичной матрицы, умножения и возведения в степень должны быть реализованы в общем виде.
    \item \textbf{(2 балла)} Реализовать вычисление $n$-ого числа Фибоначчи через перемножение матриц за логарифм.
    \item \textbf{(1 балл)} Реализовать вычисление всех чисел Фибоначчи до $n$-ого включительно.
\end{enumerate}