\section{Практика 3}

Домашняя работа 3
    1. (1 балл) Реализовать вычисление n-ого числа Фибоначчи рекурсивным методом. Обеспечить чтение n из консоли и печать результата в консоль.
    2. (1 балл) Реализовать вычисление n-ого числа Фибоначчи итеративным методом. Обеспечить чтение n из консоли и печать результата в консоль.
    3. (1 балл) Реализовать вычисление n-ого числа Фибоначчи итеративным методом, но не используя ref-переменных (и mutable) и других изменяемых структур. Подсказка: нужно использовать рекурсию с аккумулятором. Обеспечить чтение n из консоли и печать результата в консоль.
    4. (2 балла) Реализовать вычисление n-ого числа Фибоначчи через перемножение матриц «наивным» методом. Обеспечить чтение n из консоли и печать результата в консоль.
    5. (2 балла) Реализовать вычисление n-ого числа Фибоначчи через перемножение матриц за логарифм. Обеспечить чтение n из консоли и печать результата в консоль.
    6. (1 балл) Реализовать вычисление всех чисел Фибоначчи до n-ого включительно. Обеспечить чтение n из консоли и печать результата в консоль.
Проверяемые компетенции: ОПК-2, ОПК-5, ПКП-3
Критерии оценивания: решения задач 1, 2, 3, 6 оцениваются по шкале от 0 (нет решения или решение имеет существенные недостатки) до 1 (решение работоспособно, аккуратно реализовано). Решения задач 4, 5 оцениваются по аналогичным критериям по шкале от 0 до 2.
Домашняя работа 4
    1. (1 балл) Реализовать сортировку пузырьком массива. Реализовать чтение массива из файла и печать результата в файл. Функции чтения и записи необходимо переиспользовать.
    2. (1 балл) Реализовать сортировку пузырьком списка. Реализовать чтение списка из файла и печать результата в файл. Функции чтения и записи необходимо переиспользовать.
    3. (1 балл) Реализовать быструю сортировку для списка. Реализовать чтение списка из файла и печать результата в файл. Функции чтения и записи необходимо переиспользовать.
    4. (1 балл) Реализовать быструю сортировку для массива. Реализовать чтение массива из файла и печать результата в файл. Функции чтения и записи необходимо переиспользовать.
    5. (1 балл) Реализовать запаковку двух 32-битных чисел в одно 64-битное и распаковку обратно. Реализовать чтение входных данных с консоли и печать результата в консоль.
    6. (1 балл) Реализовать запаковку четырёх 16-битных чисел в одно 64-битное и распаковку обратно. Реализовать чтение входных данных с консоли и печать результата в консоль.
Проверяемые компетенции: ОПК-2, ПКП-3, ПКП-4
Критерии оценивания: решения всех задач оцениваются по шкале от 0 (нет решения или решение имеет существенные недостатки) до 1 (решение работоспособно, аккуратно реализовано). 
Домашняя работа 5
    1. (5 баллов) Провести сравнительное исследование реализованных в предыдущей домашней работе сортировок и стандартных реализаций сортировок соответствующих коллекций. Оформить отчёт: постановка эксперимента, результаты экспериментов, анализ результатов. Отчёт оформляется в TeX, исходники выкладываются так же как и обычный код, снабжаются скриптом сборки.
Проверяемые компетенции: ОПК-2, ОПК-4, ОПК-5, ПКП-3, ПКП-4, УКБ-3
Критерии оценивания: решения задачи оцениваются по шкале от 0 (нет решения или решение имеет существенные недостатки) до 5 (решение работоспособно, аккуратно реализовано). 
Домашняя работа 6
    1. (1 балл) Предположим, что мы храним булевы матрицы в виде списка координат ячеек, значение которых «true». Необходимо реализовать соответствующие типы: единицы измерения для строк и столбцов, пара «строка-столбец», список пар «строка-столбец». 
    2. (2 балла) Реализовать функцию, перемножающую две матрицы, заданных в формате, описанном в предыдущей задаче. Не забыть проверку корректности входных данных. Реализовать подгрузку матриц из файла и запись результата в файл. Файлы с данными и результатом указываются через консоль. Формат хранения матрицы из m строк и n столбцов: в файле m строк, каждая строка состоит из n символов 0 или 1. 
Проверяемые компетенции: ОПК-2, ОПК-4, ПКП-3, ПКП-4, УКБ-3
Критерии оценивания: решения задачи 1 оцениваются по шкале от 0 (нет решения или решение имеет существенные недостатки) до 1 (решение работоспособно, аккуратно реализовано), решения задачи 2 оцениваются по тем же критериям, но по шкале от 0 до 2.
Домашняя работа 7 
    1. (1 балл) Реализовать самостоятельно полиморфный список (далее будем называть этот тип MyList). Реализовать для него функции сортировки, вычисления длины, конкатенации.
    2. (1 балл) На основе MyList реализовать тип MyString, представляющий строку как список символов. Реализовать преобразование стандартной строки в MyString и конкатенацию строк для  MyString.
    3. (1 балл) Реализовать тип дерева с произвольным количеством потомков в каждом узле (использовать MyList) MyTree. Каждый узел должен хранить данные произвольного типа. Реализовать функцию обхода в глубину для такого дерева.
    4. (1 балл) Пусть есть MyTree, хранящий в узлах целые числа. Реализовать функции, которые находят максимальный хранимый элемент, среднее значение всех хранимых элементов.
Проверяемые компетенции: ОПК-2, ОПК-5, ПКП-3
Критерии оценивания: решения всех задач оцениваются по шкале от 0 (нет решения или решение имеет существенные недостатки) до 1 (решение работоспособно, аккуратно реализовано). 
Домашняя работа 8 
    1. (4 балла) Используя тип MyList из предыдущей домашней работы, реализовать целочисленную длинную арифметику: операции сложения, умножения, вычитания, целочисленного деления. 
Проверяемые компетенции: ОПК-2, ПКП-3, ПКП-4
Критерии оценивания: решения задачи оцениваются по шкале от 0 (нет решения или решение имеет существенные недостатки) до 4 (решение работоспособно, аккуратно реализовано). 
Домашняя работа 9
    1. (1 балл) Реализовать представление разреженных матриц в виде дерева квадрантов. Реализовать функцию поэлементного сложения двух матриц в таком формате.
    2. (1 балл) Реализовать функцию умножения двух матриц в формате дерева квадрантов.
    3. (1 балл) Реализовать функцию тензорного умножения двух матриц в формате дерева квадрантов.
    4. (3 балла) Реализовать построение транзитивного замыкания ориентированного графа через произведение матриц. Использовать представление матриц из первой задачи. Визуализировать результат с помощью GraphViz: исходный граф и выделенные рёбра, появившиеся в результате транзитивного замыкания. Для задания графа использовать формат из задачи 2 6-й домашней работы.
    5. (3 балла) Реализовать вычисление кратчайших путей между всеми парами вершин в ориентированном графе. Использовать представление матриц из первой задачи. Визуализировать результат с помощью GraphViz: исходный граф и выделенные рёбра со значением кратчайшего пути между соответствующей парой вершин. Для задания графа использовать формат, аналогичный формату из задачи 2 6-й домашней работы.
Проверяемые компетенции: ОПК-2, ОПК-5, ПКП-3, ПКП-4, УКБ-3
Критерии оценивания: решения задач 1-3 оцениваются по шкале от 0 (нет решения или решение имеет существенные недостатки) до 1 (решение работоспособно, аккуратно реализовано), решения задач 4-5 оцениваются по тем же критериям, но по шкале от 0 до 3.
Домашняя работа 10
    1. (3 балла) Реализовать с помощью одного из инструментов ANTLR, FParsec, fslex+fsyacc синтаксический анализ языка регулярных выражений и операций над ними. Регулярные выражения используют алфавит латинских символов и операции конкатенации, |, *, группирующие скобки. Можно добавить + и ?. Операции над выражениями – пересечение и объединение. Обеспечить чтение входа из файла и визуализацию дерева разбора в GraphViz.
    2. (3 балла) Реализовать построение автомата по регулярному выражению. В качестве основы использовать представление матриц из предыдущей работы. Разбор входного регулярного выражения осуществлять средствами из предыдущей задачи.
    3. (5 баллов) Реализовать вычислитель для длинной арифметики. Реализовать синтаксический анализатор арифметических выражений: бинарные операции +, -, *, /; унарный минус; возведение в степень; группирующие скобки; возможность именовать подвыражения и использовать имена в других выражениях. Числа только целые и могут быть «очень большими». Обеспечить чтение входа из файла и визуализацию дерева разбора в GraphViz. Реализовать вычисление значения выражения на основе реализованной ранее длинной арифметики.
Проверяемые компетенции: ОПК-2, ОПК-5, ПКП-3, ПКП-4, УКБ-3
Критерии оценивания: решения задач 1-2 оцениваются по шкале от 0 (нет решения или решение имеет существенные недостатки) до 3 (решение работоспособно, аккуратно реализовано), решения задачи 3 оцениваются по тем же критериям, но по шкале от 0 до 5.
Домашняя работа 11 
    1. (1 балл) Расширить язык из предыдущей работы командой, которая будет принимать выражение над регулярками (автомат) и строку и проверять, принимается ли строка автоматом.
    2. (2 балл) Расширить язык возможностью именовать подвыражения и использовать имена в других выражениях. 
    3. (5 баллов) Реализовать интерпретатор получившегося языка. Для построения автоматов по регулярным выражениям использовать результаты предыдущей работы. Для операций над автоматами использовать тензорное произведение из работы 9. Вероятно, нужно будет определить абстракцию полукольца. Снабдить интерпретатор возможностями обрабатывать файл и работать в интерактивном режиме.   
Проверяемые компетенции: ОПК-2, ОПК-5, ПКП-3, ПКП-4
Критерии оценивания: решения задачи 1 оцениваются по шкале от 0 (нет решения или решение имеет существенные недостатки) до 1 (решение работоспособно, аккуратно реализовано), решения задачи 2 оцениваются по тем же критериям, но по шкале от 0 до 2, решения задачи 3 оцениваются по тем же критериям, но по шкале от 0 до 5.
Семестр 2.

Домашняя работа 1 
    1. (1 балл) Оформление калькулятора или регулярных выражений как отдельного проекта. Создать репозиторий, снабдить всеми необходимыми элементами экосистемы: сборка, тесты, лицензия, readme.
    2. (2 балла) Создать документацию. Описать цели и задачи проекта, конкретный синтаксис языка, привести примеры.
    3. (2 балла) Создать диаграмму (наиболее подходящего типа), описывающую структуру проекта, выбранного выше. Добавить её в документацию.
Проверяемые компетенции: ОПК-4, ОПК-5, ПКП-3, ПКП-4, УКБ-3
Критерии оценивания: решения задачи 1 оцениваются по шкале от 0 (нет решения или решение имеет существенные недостатки) до 1 (решение работоспособно, аккуратно реализовано), решения задач 2 и 3 оцениваются по тем же критериям, но по шкале от 0 до 2.
Домашняя работа 2. Мини-IDE для языка из предыдущей работы.
    1. (2 балла) Расширить синтаксис соответствующего языка логическими выражениями (с переменными) и условными операторами.
    2. (5 баллов) Разработать среду разработки для полученного языка. Среда должна предоставлять следующие возможности:
        1. Редактировать код.
        2. Работать с файлами: создать новый, открыть существующий, сохранить изменения. 
        3. Выводить сообщения о (синтаксических) ошибках.
        4. Запустить программу на исполнение.
        5. Видеть результат исполнения в «консоли».
    3. (3 балла) Расширить IDE возможностью подсветки синтаксиса.
    4. (5 баллов) Расширить IDE возможностью устанавливать точки останова. В момент остановки должна быть возможность просмотреть значения всех «живых» переменных.
Проверяемые компетенции: ОПК-2, ОПК-4, ОПК-5, ПКП-3, ПКП-4, УКБ-3
Критерии оценивания: решения задачи 1 оцениваются по шкале от 0 (нет решения или решение имеет существенные недостатки) до 2 (решение работоспособно, аккуратно реализовано), решения задач 2 и 4 оцениваются по тем же критериям, но по шкале от 0 до 5, решения задачи 3 оцениваются по тем же критериям, но по шкале от 0 до 3.
Домашняя работа 3 
    1. (1 балл) Реализовать список с конкатенацией за константу. Реализовать функцию конкатенации для соответствующего типа.
    2. (2 балла) Реализовать функции вычисления длины, взятия головы и хвоста, добавления элемента в голову для списка из предыдущего пункта.
    3. (2 балла) Реализовать сортировку для списка из первого пункта.
    4. (3 балла) Реализовать map, foldr, foldl, rev для списка из первого пункта.
    5. (3 балла) Реализовать zipper для стандартного списка.
    6. (4 балла) Реализовать zipper для списка из первого пункта.
Проверяемые компетенции: ОПК-2, ПКП-3, ПКП-4
Критерии оценивания: решения задачи 1 оцениваются по шкале от 0 (нет решения или решение имеет существенные недостатки) до 1 (решение работоспособно, аккуратно реализовано), решения задач 2 и 3 оцениваются по тем же критериям, но по шкале от 0 до 2, решения задач 4 и 5 оцениваются по тем же критериям, но по шкале от 0 до 3, решение задачи 6 оценивается по тем же критериям, но по шкале от 0 до 4.
  
Домашняя работа 4 
    1. Реализовать консольный генератор матриц. На вход принимается размер матрицы, тип данных, количество матриц, метрика разреженности, возможно другие необходимые параметры. В результате генерируется набор файлов с матрицами в формате из первого семестра. 
    2. Реализовать параллельное умножение для плотных матриц. Исследовать варианты с распараллеливанием различных циклов. Для исследования использовать созданный ранее генератор. Оформить соответствующий отчёт.
    3. Реализовать параллельное умножение матриц, представленных в виде дерева квадрантов.
    4. Сравнить производительность решений из первых двух пунктов на разных типах матриц. Для исследования использовать созданный ранее генератор. Оформить соответствующий отчёт. 
Проверяемые компетенции: ОПК-2, ОПК-4, ОПК-5, ПКП-3, ПКП-4, УКБ-3
Критерии оценивания: решения всех задач оцениваются по шкале от 0 (нет решения или решение имеет существенные недостатки) до 4 (решение работоспособно, аккуратно реализовано).

Домашняя работа 5
    1. На основе Mailboxprocessor или Hopac реализовать решение, в котором есть следующие конкурентно выполняющиеся типы задачи: 
        1. Подгрузка пар матриц из файлов (для генерации использовать генератор из предыдущей работы)
        2. Различные алгоритмы перемножения матриц (для разреженных, для плотных параллельно и последовательно)
        3. Балансировщик, знающий, кому какие матрицы отправлять для обработки.
Предусмотреть два режима работы: 
            1. Обработать все матрицы, доступные на входе
            2. Обработать заданное количество пар матриц
	
    2. Проанализировать масшатабируемость полученной системы. Какое количество конкурентных задач оптимально для определённой конфигурации системы? Оформить соответствующий отчёт.
Проверяемые компетенции: ОПК-2, ОПК-4, ОПК-5, ПКП-3, ПКП-4, УКБ-3
Критерии оценивания: решения всех задач оцениваются по шкале от 0 (нет решения или решение имеет существенные недостатки) до 4 (решение работоспособно, аккуратно реализовано).

Семестр 3.

Домашняя работа 1
    1. (5 баллов) Сформулировать гипотезы относительно зависимости времени работы алгоритмов  сортировок относительно размера входных данных. Проверить их. Обосновать полученные результаты. Оформить соответствующий отчёт. Должны быть проанализированы следующие алгоритмы.
        1. Стандартная сортировка для List
        2. Различные варианты реализации быстрой сортировки для MyList.
        3. Сортировка пузырьком для List.
        4. Сортировка пузырьком для Array.
        5. Можно включить сортировки для списка с конкатенацией за константу, а также другие реализации быстрой сортировки (на массиве, например).
    2. (7 баллов) Сформулировать гипотезы относительно зависимости времени работы матричных алгоритмов относительно размера входных матриц. Проверить их. Обосновать полученные результаты. Оформить соответствующий отчёт. Должны быть проанализированы следующие алгоритмы.
        1. Последовательный для дерева квадрантов
        2. Параллельный для дерева квадрантов
        3. Последовательный для плотных матриц
        4. Параллельный для плотных матриц
Проверяемые компетенции: ОПК-2, ОПК-4, ОПК-5, ПКП-3, ПКП-4, УКБ-3
Критерии оценивания: решения задачи 1 оцениваются по шкале от 0 (нет решения или решение имеет существенные недостатки) до 5 (решение работоспособно, аккуратно реализовано), решения задачи 2 оцениваются по тем же критериям, но по шкале от 0 до 7.

Домашняя работа 2
    1. (6 баллов) Подготовить презентацию «Анализ времени работы алгоритмов сортировок» или «Анализ времени работы алгоритмов умножения матриц». Презентация готовится в TeX, по установленному шаблону.
    2. (8 баллов) Подготовить отчёт на тему «Анализ времени работы алгоритмов сортировок» или «Анализ времени работы алгоритмов умножения матриц». Отчёт готовится в TeX, по установленному шаблону.
Проверяемые компетенции: ОПК-2, ОПК-4, ОПК-5, ПКП-3, ПКП-4, УКБ-3
Критерии оценивания: решения задачи 1 оцениваются по шкале от 0 (нет презентации или презентация имеет существенные недостатки) до 6 (презентация полностью раскрывает тему и аккуратно оформлена), решения задачи 2 оцениваются по тем же критериям, но по шкале от 0 до 8.

Домашняя работа 3
    1. (3 балла) Реализовать «логгер» с использованием workflow builder. Необходимо предоставить возможность логгировать входные аргументы некоторых функций. Предусмотреть возможность указывать, куда печатать вывод: в консоль, файл, куда-то ещё.
    2. (5 баллов) Реализовать list builder для типа MyList или списка с конкатенацией за константу. Ориентироваться на билдер seq для лучшего понимания ожидаемой функциональности.
Проверяемые компетенции: ОПК-2, ОПК-5, ПКП-3
Критерии оценивания: решения задачи 1 оцениваются по шкале от 0 (нет решения или решение имеет существенные недостатки) до 3 (решение работоспособно, аккуратно реализовано), решения задачи 2 оцениваются по тем же критериям, но по шкале от 0 до 5.

Домашняя работа 4
    1. (10 баллов) Встроить на основе F# quotations разработанный ранее язык арифметики или регулярных выражений. Обеспечить вычисление задаваемых выражений и возможность работы с полученными значениями на стороне F#. Необходимо максимальное переиспользование готовых компонент.
    2. (4 баллов) Внедрить RxExtensions для работы с событиями в разработанной ранее IDE. 
Проверяемые компетенции: ОПК-2, ОПК-5, ПКП-3, ПКП-4
Критерии оценивания: решения задачи 1 оцениваются по шкале от 0 (нет решения или решение имеет существенные недостатки) до 10 (решение работоспособно, аккуратно реализовано), решения задачи 2 оцениваются по тем же критериям, но по шкале от 0 до 4.

Домашняя работа 5
    1. (4 баллов) Реализовать функцию перемножения двух плотных матриц с использованием Brahma.FSharp. 
    2. (9 баллов) Расширить решение из предыдущего семестра возможностью конкурентно умножать матрицы на GPGPU (использовать предыдущее решение).
    3. (9 баллов) Проанализировать полученное в предыдущем пункте решение и оформить соответствующий отчёт.
        1. В каких случаях лучше использовать CPU, а в каких GPGPU?
        2. Какая конфигурация конкурентно выполняющихся задач оптимальна? Имеет ли смысл поддерживать больше одного агента, работающего с GPGPU?
        3. Имеет ли смысл использовать Brahma.FSharp для выполнения кода на CPU?
Проверяемые компетенции: ОПК-2, ОПК-4, ОПК-5, ПКП-3, ПКП-4, УКБ-3
Критерии оценивания: решения задачи 1 оцениваются по шкале от 0 (нет решения или решение имеет существенные недостатки) до 4 (решение работоспособно, аккуратно реализовано), решения задач 2 и 3 оцениваются по тем же критериям, но по шкале от 0 до 9.

