\section{Домашняя работа 4}

Во всех задачах на сортировку необходимо реализовать чтение массива из файла и печать результата в файл. Функции чтения и записи необходимо переиспользовать.

В задачах на битовые операции продолжаем рабоать с консолью: чтение данных с консоли и печать результата туда же.

Для данной домашней работы необходимо создать отдельный проект.

При создании тестов необходимо, в задачах на сортировку, убедиться, что, во-первых, сортировки ведут себя одинаоково на одинаковых данных, во-вторых, что они ведут себя так же, как системные сортировки для соответствующих коллекций. Для задачи о запаковке и распаковке надо проверить, что реализованные функции являются взаимно обратными. FsCheck (testProperty) в помощь.

\begin{enumerate}
    \item \textbf{(1 балл)} Реализовать сортировку пузырьком массива. 
    \item \textbf{(1 балл)} Реализовать сортировку пузырьком списка.
    \item \textbf{(1 балл)} Реализовать быструю сортировку для списка.
    \item \textbf{(1 балл)} Реализовать быструю сортировку для массива.
    \item \textbf{(1 балл)} Реализовать запаковку двух 32-битных чисел в одно 64-битное и распаковку обратно. 
    \item \textbf{(1 балл)} Реализовать запаковку четырёх 16-битных чисел в одно 64-битное и распаковку обратно. 
\end{enumerate}