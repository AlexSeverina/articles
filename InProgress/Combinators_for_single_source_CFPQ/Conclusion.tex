\section{Conclusion and Future Work}

We show that single-source context-free path querying can be !!!
We demonstrate a combinator-based approach implemented in Meerkat.Graph Scala library, but this approach can be implemented in almost any high-level programing language.
While combinators  is a very powerful way to specify context-free queries, it may seem hard to understand for many users.
There are other algorithms for context-free path queries which should be applicable for single-source path querying and we hope that they can be integrated with the existing graph database in a more convenient way.
But it is necessary more research in this direction.

We should investigate wore datasets to detect other shapes of query results.
For example, we should investigate the behavior of single-source querying in the case when a number of resulting paths is small, but paths are relatively long.
And the first question is which data analysis tasks lead to this scenario.

One of important direction of the future reserach is to optimize performance of proposed solution.
One of possible solution is deep integration with Neo4j infrastructure to utilize cache system.

Another direction is combinators library improvement.
First of all, it is necessary to make cimbinators syntax more user-friendly.
Also, it is necessary to create set of query templates (see same-generation template).
