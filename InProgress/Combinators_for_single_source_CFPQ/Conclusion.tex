\section{Conclusion and Future Work}

We show that single-source context-free path queries can feasibly be evaluated for real-world graphs.
We demonstrate that the combinators-based approach to CFPQ is flexible and powerful.

We demonstrate a combinator-based approach implemented in Meerkat.Graph Scala library, but this approach can be implemented in almost any high-level programming language.
While combinators are a powerful way to specify context-free queries, the technique seem hard for many users.
Other algorithms for CFPQs should be applicable for single-source path querying (GLL-based~\cite{Grigorev:2017:CPQ:3166094.3166104, MEDEIROS201975} or GLR-based~\cite{10.1007/978-3-319-41579-6_22, 10.1007/978-3-319-91662-0_17}) and we believe that they can be integrated with the existing graph databases in a more convenient way.
% But it is necessary more research in this direction.

We should investigate more datasets to detect other shapes of query results.
For example, when a number of resulting paths in the single-source querying is small, but the paths are relatively long.
The first question here is which data analysis tasks lead to such scenario.
Also, we should provide detailed theoretical analysis of single-source CFPQ.

One important direction for future research is to optimize the performance of the proposed solution.
We believe that it is possible to reduce the use of graph size-dependent structures and thus make query execution time depend only on the size of the result.
One possible solution is a deep integration with the Neo4j infrastructure to utilize its cache system.

Another direction is combinators library improvement.
It is necessary to make combinators syntax more user-friendly and to create a set of query templates (such as same-generation template).