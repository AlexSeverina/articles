\section{Introduction}

Context-Free path querying (CFPQ) is an actively developed area in graph datatbase analysis.

CFPQ is widely used for static code analysis.

Languages for language-constrained queryes specification.
CfSparql and proposal for Cypher.

Integration with general purpose programming language. Typing~\cite{10.1145/2076623.2076653}.

Combinators~\cite{10.1145/3241653.3241655}.

Single source scenario.
Instead of traditional all pairs.
Some of algorithms inheritantly calculate only all pairs reachability.

In this paper we make the following contributions.
\begin{itemize}
  \item Introduce example and show how to use combinators for context-free path querying.
  We demonstarte main features of combinator-based approach such as type-safety, flexibility (compositionality and generics), IDE support and user-defined actions.
  \item We evaluate single source context-free path querying on some real-world RDFs.
  We find that the case when number of paths in answer in big, but length of these paths is relatevely small is the main case in classical RDF context-free queryes.
  And in this case single-source CFPQ can be evaluated in reasonable time and space.
\end{itemize}
