\section{Introduction}

Context-Free path querying (CFPQ) is an actively developed area in graph datatbase analysis.
CFPQ is widely used for static code analysis~\cite{!!!}, RDF querying~\cite{!!!}, biological data analysis~\cite{!!!}.

While lots of research aimed to CFPQ evalustion algorithm develomplent~\cite{!!!}, languages which supports context-free constraints specification are not investigated anough.
In our knolage, only extension for Sparql CfSparql~\cite{!!!} supports context-free constarints.
There is also proposal for Cypher\footnote{!!!} which is not implemneted yet.
So, ways to envolve context-free constraints for graph querying should be investigated.

Note, that graph analysis often is only a part of more complex solution.
So, graph query languages should be integrated with general-purpose programming languages.
Typing~\cite{10.1145/2076623.2076653}.

Combinators can solve these problems{\huge{ EV!!!}}~\cite{10.1145/3241653.3241655}.

Single source scenario instead of traditional all pairs.
Also useful.
For manual data analysis.
Some of algorithms inheritantly calculate only all pairs reachability.

In this paper we make the following contributions.
\begin{itemize}
  \item Introduce example and show how to use combinators for context-free path querying.
  We demonstarte main features of combinator-based approach such as type-safety, flexibility (compositionality and generics), IDE support and user-defined actions.
  \item We evaluate single source context-free path querying on some real-world RDFs.
  We find that the case when number of paths in answer in big, but length of these paths is relatevely small is the main case in classical RDF context-free queryes.
  And we show thst in this case single-source CFPQ can be evaluated in reasonable time and space.
  Also our evaluation demonstrates that detailed analysis of theoretical time and space complexity of CFPQ algorithms id required.
\end{itemize}
