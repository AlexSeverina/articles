\section*{Введение}


% Абзац о том, что много документации ,она важна для разработки и сопровождения продуктов 
На сегодняшний день существует множество программных продуктов, и их количество с каждым годом лишь увеличивается.
% Вообще
Размер проектов исчисляется в строках кода и затраченных человеко-часах на соответствующую разработку.
Программные продукты часто могут содержать миллионы строк кода, на написание которых затрачены миллионы человеко-часов\footnote{Код ядра линукса содержит более 27 млн строк кода, а на разработку ушло несколько сотен тысяч человеко-лет, \emph{https://www.linux.com/news/linux-in-2020-27-8-million-lines-of-code-in-the-kernel-1-3-million-in-systemd/, дата обращения 26.05.2020}}.
Соответственно, разработка и сопровождение таких сложных систем немыслимы без документации, количество которой лишь увеличивается.
% аналогичным образом растет с каждым новым продуктом.

% Детализируем, ссылками и мыслью, что качество документации важно и актуально
Важность сопроводительной документации не подвергается сомнению~\cite{kipyegen2013importance,chomal2014significance}.
Более конкретно, её качество напрямую влияет на жизненный цикл разработки системы, её конечную стоимость, время разработки, сопровождение и эксплуатацию и пр~\cite{plosch2014value}. 
К критериям качества документации относятся точность, структурированность, последовательность изложения, понятность.
Иными словами, задачи поддержания качества документации (всех ее критериев) на высоком уровне, а так же её написания, сопровождения и улучшения являются актуальными при создании, сопровождении и эксплуатации программных систем.
% Таким образом, различные виды программной документации играют важную роль в жизненном цикле разработки систем.

% Немного о клонах и их влиянии, -> их надо искать
Как в программном коде, так и в документации могут появляться \emph{текстовые повторы (текстовые клоны)}.
Влияние  \emph{текстовых повторов} на документацию различно.
% До сих пор нет четкого ответа на вопрос о влиянии \emph{текстовых повторов} на документацию.
К положительным факторам наличия \emph{повторов} в документации можно отнести
унификацию представления информации и создание общего контекста, что позволяет улучшить читаемость документации, её структурированность и передачу знаний (\emph{knowledge transfer}).
Несмотря на это, наличие клонов в документации влечет, как и в программном коде,  распространение ошибок и опечаток.
Также при наличии группы повторов, т.е множества похожих  фрагментов текста, при внесении изменения в один из таких текстовых фрагментов необходимо произвести соответствующие изменения и в других "раскопированных" элементах во избежание нарушения консистентности информации, что усложняет процесс сопровождения документации.
Более того, существуют исследования, в которых показано, что клоны в документации могут привести к клонам в программном коде~\cite{juergens2010can}.
% Таким образом, нахождение дубликатов в документации
Соответственно, нахождение повторов в документации является актуальной проблемой, и текущие исследования лишь подтверждают утверждение~\cite{horie2010tool, poruban2014reusable, poruban2016preliminary, juergens2010can, oumaziz2017documentation}.

% Способы как ищут -> сводим к строковым алгоритмам
Важным этапом при решении задачи поиска повторов является выбор того, как будет измеряться похожесть двух текстовых фрагментов и что при выбранном подходе считать клонами.

% Существующие подходы можно разделить на несколько видов: лексикографическая похожесть, семантическая похожесть, синтаксически-лексическая похожесть и комбинация вышеперечисленных.
% К семантической похожести относятся подходы, при которых похожесть (или же расстояние между фрагментами) основывается на  их семантической близости, т.е на том, насколько одинаковый смысл несут эти фрагменты.
% Синтаксически-лексические подходы основываются на структурной и синтаксической похожести.
% При лексикографической похожести (с точностью до написания) учитывается лишь то, насколько посимвольно похожи два выбранных фрагмента.

Применение строковых алгоритмов является наиболее естественным способом решения задачи поиска повторов как в произвольном тексте, так и в программной документации.
Алгоритмы решения задач поиска наибольшей общей подпоследовательности~(\emph{LCS --- longest common subsequence}) и выравнивания двух последовательностей~(\emph{SA  --- sequence alignment}) --- широко известные алгоритмы, которые имеют разные приложения, в том числе и к задаче поиска повторов.

\emph{LCS} и \emph{SA} измеряют насколько текстовые фрагменты похожи глобально (в общем) друг на друга. 
Интуитивно понятно, что этого бывает недостаточно, потому что часто два фрагмента бывают схожи лишь небольшой общей частью.
Решить эту проблему помогает обобщение на так называемый полулокальный случай, а именно, \emph{полулокальные задачи наибольшей общей подпоследовательности и выравнивания последовательностей} (\emph{semi-local lcs, semi-local sa})~\cite{tiskin2006all}.

Автором данного обобщения является А.\,В. Тискин.
Он внес огромный вклад в развитие теории вокруг данных задач~\cite{tiskin2015fast,tiskin2019bounded,krusche2009parallel,tiskin2006longest,tiskin2008semi,tiskin2011towards}.
В частности, им изобретены эффективные алгоритмы.
Также им пишется труд~\cite{tiskin2006all}, в котором собрана вся актуальная информация, связанная c этой теорией.



% За прошедшее десятилетие Тискиным \red{А.\,В.} была развита теория вокруг данных задач, и изобретены эффективные алгоритмы решающие поставленные задачи.

Несмотря на то, что алгоритмы имеют хорошие теоретические свойства, до конца неясно, насколько они применимы на практике к задаче поиска повторов, но
относительно недавние успехи в применении этих алгоритмов в области биоинформатики~\cite{baxter2012conserved,davies2015analysis, picot2010evolutionary} дают все основания полагать, что их можно успешно адаптировать к задаче поиска повторов в документации.
Важно отметить, что большая часть алгоритмов еще ни разу не была реализована на практике.

В данной работе представлено решение задачи поиска повторов в документации с помощью применения и адаптации алгоритмов решения задач \emph{semi-local lcs} и \emph{semi-local sa}. Для оценки применимости решения была осуществлена реализация приложения и библиотеки алгоритмов. Апробация результатов произведена на API-документации.

%  Подведение к целям и задачам

