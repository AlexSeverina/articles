\section{Обзор}
В этой главе представлен обзор предметной области, в частности работ, связанных с повторами в документации ПО, описана модель повторов, которая будет использована в дальнейшем. 
Исходя из текущих исследований, сформулированы задачи поиска повторов в документации ПО.
Также произведено исследование полулокальных задач поиска наибольшей общей подпоследовательности и выравнивания строк, описаны решающие их алгоритмы и идеи со свойствами, лежащими за ними.

\subsection{Повторы в документации ПО}\label{duplicateReport}

За последние десять лет появилось множество работ, посвященных документации ПО, в частности проблемам, связанным с наличием и выявлением в ней повторов.
% связанным с повторами. 
Одни работы посвящены эмпирическим исследованиям о количестве повторов  в различных видах документации ПО~\cite{poruban2016preliminary,juergens2010can,oumaziz2017documentation}, другие --- фокусируются на реализации механизма переиспользования повторяющихся фрагментов информации в документации~\cite{koznov2015clone,horie2010tool,poruban2014reusable}, третьи --- на алгоритмах и подходах поиска повторов~\cite{luciv2018detecting,luciv2019interactive,blasi2018replicomment,rago2016identifying, soto2015similarity}.

Эльмар Юргенсен и др.~\cite{juergens2010can} провели широкомасштабное исследование спецификаций требований различных проектов на предмет наличия количество повторов, содержащихся в них. 
Их исследование показало, что количество повторов может быть значительно (вплоть до 70\%). 
Стоит отметить, что они адаптировали \emph{CONQAT}\footnote{https://www.cqse.eu/, дата обращения 26.05.2020} --- инструмент предназначенный для непрерывной оценки качества документации, для поиска клонов.
В их неформальной модели, повтор -- это часть (подстрока) спецификации, которая повторяется более двух раз, а группой считаются повторы, которые разделяют общую часть. 
Соответственно, их адаптированное решение находило лишь точные повторы.

Милан Носаль и др. ~\cite{poruban2016preliminary} провели эмпирическое исследование пяти крупных \emph{JavaDoc} проектов, целью которого являлось проверка гипотезы о том, что комментарии к коду содержат большое количество повторов.
Саму модель повторов они определили неформально, а инструмент для нахождения повторов\footnote{CPD (copy-paste detector) инструмент на основе строкового алгоритма Рабин-Карпа, который позволяет находить только точные повторы.} адаптировали, как и авторы~\cite{juergens2010can}.
Результаты их работы подтвердили гипотезу исследования. 


Работа Мохамеда А. Умазиза и др.~\cite{oumaziz2017documentation} также относится к эмпирическому исследованию \emph{JavaDoc} проектов. 
Их работа мотивирована критической важностью документации программного кода в процессе разработки ПО.
В работе повторы рассматриваются как цельные \emph{JavaDoc} комментарии, которые встречаются в документации более одного раза, т.е рассматривается неформальная модель точных повторов.
Для нахождения повторов был адаптирован инструмент \emph{GumTree tool}\footnote{https://github.com/GumTreeDiff/gumtree, дата обращения 26.05.2020},  предназначенный для анализа программного кода на основе построения синтаксических деревьев.
Результаты исследования показали, что  разработчики часто  переиспользуют части документации.
Также их исследование показало, что текущие инструменты для работы с \emph{JavaDoc} документацией не позволяют в полной мере избавиться от повторов.


%запашок
В работе \cite{blasi2018replicomment} авторами создан прототип инструмента \emph{RepliComment} для нахождения точных повторов в \emph{JavaDoc} документации, точнее комментариях к методам (использована неформальная модель повторов).
Также инструмент позволяет классифицировать найденные решения, что позволило исследователям найти ряд ошибок и неточностей и, как следствие, улучшить качество документации (они отправили разработчикам список найденных ошибок и неточностей).
В итоге,  результаты анализа нескольких проектов c помощью \emph{RepliComment} показали, что как код может иметь "плохой запах" (code smelss), так и комментарии к нему. 
% и его достаточно много т.е в комментариях может содержаться много повторов.

Авторы \cite{rago2016identifying} так же, как и \cite{juergens2010can}, делают акцент работы на повторах в спецификации требований, в частности на повторах функциональных требований в \emph{use-case} диаграммах.
Для анализа этого вида документации они реализовали комплексный инструмент  \emph{RegAligner}, который использует комбинированный подход для детектирования повторов.
Подход основан на конвейерной архитектуре. 
На первом этапе применяются техники обработки естественного языка (nlp), затем используется метод машинного обучения для трансляции результатов в промежуточный язык. Таким образом, получается множество цепочек, которые попарно выравниваются.
\emph{ReqAligner} в силу своего подхода позволяет находить не только точные повторы, но и семантически схожие.
Также необходимо отметить, что авторы не определили формальной модели повторов.
Авторы провели апробацию инструмента на нескольких спецификациях, получив в результате 86\% полноту и 63\% точность.
Их результаты апробации также подтвердили факт наличия повторов в \emph{use-case} диаграммах.


Д.\,В. Луцив и коллеги внесли большой вклад своими исследованиями в области поиска повторов в документации ПО~\cite{luciv2018detecting,luciv2018duplicate,luciv2019interactive,koznov2015clone,koznov2017duplicate,luciv2016fuzzy}. 
Они разработали комплексный инструмент \
\emph{Duplicate Finder}, позволяющий  находить повторы в документации, визуализировать их, а так же вносить изменения в исходную документацию.
Разработали формальную модель повторов в документации ПО и на ее основе создали алгоритмы для поиска повторов:  алгоритм для поиска повторов по заданному шаблону и  алгоритм для поиска групп повторов.
Формальная модель позволила им доказать ряд свойств в отношении этих алгоритмов.

Алгоритм поиска групп повторов~\cite{luciv2016fuzzy} основан на идее искусственного конструирования повторов.
Сперва находятся все точные повторы, а затем на их основе строятся группы нечетких повторов.
Такая интерпретация происхождения неточных повторов имеет право на жизнь, но она искусственна, что справедливо приводит к высокому уровню ошибок при нахождении повторов (\emph{false-positive}) в результатах апробации в данной статьей.
% Про цельность?

% Несмотря на то, что представленный алгоритм обладает приемлемыми характеристиками, он игнорирует семантику выделяемых повторов, что оказывается
% важным для практических применений

Алгоритм поиска повторов по образцу~\cite{luciv2019interactive} основан на идее так называемого скользящего окна.
Сперва находятся все похожие фрагменты, а затем производится их фильтрация и сжатие.
Выведенная авторами асимптотика алгоритма оценивается как $O(|p|^4)$, где $p$ --- шаблон, или же как $O(|p|^2 \times m)$, где $m$ --- размер текста.
Авторами доказана полнота алгоритма.
Несмотря на это, наличие квадрата в асимптотической оценке делает его не применимым на потенциально больших данных.
Также присутствует ограничение на используемую схему подсчета похожести --- алгоритм обладает полнотой лишь при условии использования редакционного расстояния.

Таким образом, во-первых, для различных видов документации, в частности JavaDoc  документации, задача поиска повторов является  как никогда актуальной.
Во-вторых, несмотря на большое количество работ, посвященных  данной тематике, существующие подходы и алгоритмы могут быть  улучшены как в теоретических оценках, так и на практике.
В-третьих, в работах преобладает неформальная модель повторов.
В-четвертых, исследователи помимо нахождения пар повторов, также ищут и группы повторов и осуществляют поиск повторов по образцу.


\subsection{Модель повторов в документации ПО}\label{Model}
% добавить про модели
Как было описано выше, существует множество неформальных моделей, по-разному определяющих понятие \emph{повтора} в документации ПО.
% что такое \emph{повторы}.
Можно сказать более точно, в каждой статье дается свое (авторское) видение.
Формальная модель присутствует только в работах Д.\,В. Луцива.
% моделей меньше, если не сказать немного.

В данной работе будет использована наиболее общая модель, согласно которой \emph{повтор} --- это отношение между двумя непересекающимися фрагментами $a$ и $b$, которое задается с помощью функции похожести $g$.
Отметим, что в данной модели не накладывается никаких ограничений на функцию похожести $g$.
Например, в качестве $g$ можно выбрать функцию, которая высчитывает количество одинаковых символов в двух фрагментах и сравнивает его с каким-то пороговым значением $h$. Заметим, что в этом случае функция симметрична по отношению к своим аргументам, т.е $g(a,b) = g(b,a)$.
% Например, из того, что  фрагмент $a$ является повтором фрагмента $b$, может не следовать то, что $b$ также является повтором $a$ (функция может быть не симметрична по отношению к своим аргументам, т.е $g(a,b) \neq  g(b,a)$). 

% a->b b->c   d->c  -> a,b,c,d
% GOVNO PRAV

Набор \emph{повторов} может образовывать \emph{группу повторов}, если он удовлетворяет определенным свойствам. Иными словами, \emph{группа повторов}  --- это множество повторов, которые в совокупности удовлетворяют заданному предикату.
Предикатом может служить следующее утверждение: \emph{В графе, построенном по группе повторов, существует хотя бы одна вершина, из которой достижимы все остальные. Вершинами графа являются повторы, ребра --- значение функции $g$  для выбранных вершин}.

% TODO/Заметим, что модель (или не может?) является частным случаем (удалить?) .

\subsection{Задачи поиска повторов в документации ПО}

Можно выделить несколько задач относящихся к поиску повторов в документации ПО:
\begin{itemize}
    \item Поиск всех \emph{повторов}
    \item Поиск \emph{групп повторов}
    \item Поиск \emph{повторов} по образцу
\end{itemize}

\textbf{Поиск всех повторов}.
Задача поиска всех пар \emph{повторов} формулируется следующим образом: \emph{Дан набор текстов $t$. В $t$ необходимо найти все пары повторов,  согласно выбранной функции похожести $g$.
}

\textbf{Поиск групп повторов}.
Задача поиска групп повторов формулируется следующим образом:
\emph{Дан набор  текстов $t$. В $t$ необходимо найти все непересекающиеся группы повторов согласно выбранной функции похожести $g$ и предикату $\gamma$}.
Эта задача является вариацией предыдущей задачи.

\textbf{Поиск по образцу}.
Задача поиска по образцу формулируется следующим образом:
\emph{Дан шаблон $p$ и набор текстов $t$. Необходимо найти все непересекающиеся повторы шаблона $p$ в  $t$ согласно выбранной функции похожести $g$}. Стоит отметить, что могут существовать разные способы разбиения на непересекающиеся повторы.

% Над набором javaDoc комментариев нужно построить группы повторов, иными словами найти такое разбиение
% Задачу поиска групп повторов в Javadoc документации
% The semi-local LCS problem represents a generalization of a well-knownLCS problem. It is defined as follows




% В данном разделе представлены базовые математические модели и
% определения, использующиеся при решении поставленной задачи, даны
% определения метрик качества стабилизации видео, которые задействованы в алгоритмах стабилизации и калибровки, а также описан метод
% сбора тестовых данных и их форма
% \subsection{Документация ПО}







% \subsubsection{Виды документации ПО}
% % табличка должна быть


\subsection{Полулокальные задачи поиска}

В этой секции будут даны общие сведения о полулокальных задачах наибольшей общей подпоследовательности и выравнивания строк.
Детальные описания задач, теорем и их доказательств могут быть найдены в  \cite{alex2007semilocal}.

% рассказать про lcs и sa
\subsubsection{Задачи поиска наибольшей общей подпоследовательности (LCS) и выравнивания строк (SA)}

Для начала, необходимо дать определение, что такое \emph{LCS} и \emph{SA}.

\paragraph*{Задача о наибольшей общей подпоследовательности (LCS)}\mbox{}

Даны две строки $a$ и $b$ длин $m$ и $n$ соответственно.
Необходимо найти наибольшую общую подпоследовательность\footnote{Тоже, что и в математике.
Подпоследовательность --- это последовательность, которая может быть получена из другой последовательности путем удаления части ($\geq 0$) её элементов без изменения порядка следования элементов.} символов для строк $a$ и $b$, а именно значение длины этой подпоследовательности т.е $lcs$-значение.
Пример:
\begin{center}
    $lcs(abbba,aa)=aa$ (ее длина 2)
\end{center}
 
\paragraph*{Задача о выравнивании двух последовательностей (SA)}\mbox{}

Даны две строки $a$ и $b$ длин $m$ и $n$ соответственно.
Необходимо найти наибольшее  значение функции выравнивания двух строк ($sa$) с учетом выбранной схемы оценки $w = (w_{+}, w_{0} , w_{-})=$ (пара символов совпала, пара символов не совпала, символ выровнен по пропуску).
Схема оценки отвечает за учет стоимости выравнивания пары символов из соответствующих строк, а сама функция выравнивания определяется следующим образом:
\begin{equation}\label{formula:sa}
\begin{aligned}
    sa(a,b,w) = w_{+}k^{+} + w_{0}k^{0} + w_{-} (m + n - 2k^{+} - 2k^{0}) =\\
    k^{+} (w_{+} - 2w_{-} ) + k^{0}  (w_{0} - 2w_{-}) + w_{-}(m + n),
\end{aligned}
\end{equation}


где $k^{+}$ и $k^{0}$ --- количество совпадающих и несовпадающих пар символов в выравнивании.


Иными словами, когда две строки выровнены, пара соответствующих символов $\gamma$ из $a$ и $\beta$  из $b$ будет называться выровненной.
Символы в паре могут совпасть ($w_{+}$), а могут быть различны ($w_{0}$).
Также при выравнивании символ одной строки может  быть выровнен не по символу из другой строки, а по так называемому пропуску ($gap$, $w_{-}$), т.е при выравнивании происходит разрыв и ставится пропуск '-'.
Пример выравнивания для строк $a=ABCA$, $b=ACBA$ и схемы оценки $w = (1, -0.3, -0.5)$:
\begin{center}
    \begin{tabular}{ccccc}
    A & - & B &  C  &  A \\
    A & C & B &  - &  A
    \end{tabular}
\end{center}
Здесь результат выравнивания будет равен $sa(a,b) = 3*1.0-2*0.5 = 2.0$.

Заметим, что задача поиска наибольшей подпоследовательности \\(LCS) является частным случаем задачи выравнивания строк (SA) при схеме подсчёта $(1,0,0)$.

Обе описанные задачи решаются динамическим программированием
и имеют сложность $O(mn)$ (см. алгоритм~\ref{alg:lcssa}).

% Заметим, что обе задачи сводятся к поиску на прямоугольной решетке (графе) пути между  наибольшего значения  ${\color{red} Вставить картинку   }$.

Как было отмечено во введении, LCS  и SA позволяют найти насколько в глобальном смысле похожи заданные строки.
Во многих задачах интересны так называемые локальные и полулокальные случаи.
Локальный случай относится к нахождению таких пар участков из $a$ и $b$, которые максимально похожи друг на друга.
Описание полулокальных задач дано ниже.
% Полулокальный же вариант --- это, то , что лежит между глобальным и полностью локальным случаем.
\begin{algorithm}
\caption{Префикс SA (LCS)}\label{alg:lcssa}
Вход: строки $a$ и $b$ длин $m$ и $n$ соответственно\\
Выход: наибольшее значение функции \emph{sa} (\emph{lcs}) для всех пар префиксов, в частности для $a[0:n]$ и $b[0:n]$\\
Комментарии: матрица $h$ хранит соответствующие значения т.е элемент матрицы $h[i,j]$ хранит наибольшее значение \emph{sa} (\emph{lcs}) для префиксов строк $a[0:i]$ и  $b[0:j]$\\
Псевдокод: 
\begin{algorithmic}[1]
\State $h[0, j] := 0$ для всех $j \in 0..n$ 
\State $h[i, 0] := 0$ для всех $i \in 0..m$
\For{i in 0...m}
\For{j in 0 ..n}
\State $h[i,j] := max \begin{cases}
			h[i-1,j-1] +  
			\begin{cases}
			w_{+}, & \text{если $a[i]= b[j]$} \\
			w_{0}, &  \text{если $a[i]\neq b[j]$}
			\end{cases}, & \text{} \\
            h[i-1,j] + w_{-}, & \text{} \\
            h[i,j-1] + w_{-}, & \text{}
		 \end{cases}$
\EndFor
\EndFor
\end{algorithmic}
\end{algorithm}


\subsubsection{Semi-local LCS и SA}

% Semi-local LCS def
\paragraph*{Задача поиска полулокальной наибольшей общей подпоследовательности (semi-local lcs)}\mbox{}

Даны две строки $a$ и $b$ длин $m$ и $n$ соответственно. 
Необходимо найти значение $lcs$ для следующих подзадач:
\begin{itemize}
    \item \textbf{string-subtring}: для каждой из подстрок строки $b$ найти наибольшую общую подпоследовательность со строкой $a$, а именно значение $lcs$;
    \item \textbf{subtring-string} для каждой из подстрок строки $a$ найти наибольшую общую подпоследовательность со строкой $b$, а именно значение $lcs$;
    \item \textbf{prefix-suffix} для каждой пары, состоящей из префикса строки $a$ и суффикса строки $b$, найти наибольшую общую подпоследовательность;
    \item \textbf{suffix-prefix} для каждой пары, состоящей из префикса строки $b$ и суффикса строки $a$, найти наибольшую общую подпоследовательность.
\end{itemize}

Решение данной задачи представляется в виде квадратной матрицы $H_{a,b}$, где каждый квадрант отвечает за одну из описанных подзадач: 
\begin{equation}
 H_{a,b} = \begin{bmatrix}
suffix-prefix & substring-string \\
string-substring & prefix-suffix 
\end{bmatrix}    
\end{equation}

Более того, каждая ячейка содержит ответ для заданной пары подстрок.
Таким образом, сразу очевиден наивный алгоритм решения  задачи --- последовательное  заполнении ячеек матрицы, что в худшем случае дает $O((n+m)^2) \times O((n+m)^2) = O((n+ m)^4)  $ сложность.

Оказывается, разработанная теория вокруг  данной задачи позволяет ее решать намного быстрее.
Точнее, существует несколько алгоритмов, имеющих асимптотику $O(n \times m)$ и $O(n \times m \times \log n)$ соответственно.
Данные алгоритмы опираются на свойствах матрицы $H_{a,b}$, которыми она обладает.


% Все они основаны на свойствах матрицы $H_{a,b}$ и теоремах связанных  с ней.

Во-первых, по построению, матрица относится к классу так называемых  \emph{юнит aнти-матриц Монжа} (\emph{unit anti-Monge}).
Матрица $H$ называется (анти-)матрицей Монжа, если выполняется следующее:
\begin{equation}
    H[i, j] + H[i^{'},j^{'}] \leq(\geq) H[i', j] + H[i,j'] ,\quad \forall \: i \leq i' , j \leq j'
\end{equation}
(Анти-)матрица Монжа $H$ называется \emph{юнит (анти)-матрицей Монжа}, если матрица (-)$H^{\msquare}$, получающаяся в результате взятия перекрестной разности анти-диагонали и диагонали для всех смежных квадратов 2 на 2,   является перестановочной. Пример \emph{юнит матрицы Монжа} (первая матрица):
\begin{equation}
\begin{bmatrix}
0 & 2 & 3 \\
0 & 1 & 2 \\
0 & 1 & 1
\end{bmatrix} ^ { \msquare} =
\begin{bmatrix}
(2 + 0) - (1 + 0)  & (3 + 1) - (2 + 2)  \\
(1 + 0) - (1 + 0) &  (2 + 1) - (1 + 1) 
\end{bmatrix} = 
\begin{bmatrix}
1 & 0  \\
0 & 1 
\end{bmatrix} 
\end{equation}


Во-вторых, данный класс матриц образует моноид с операцией $\otimes$ над этими матрицами , которая известна как \emph{distance matrix multipliсa-\\tion}\footnote{Операция над матрицами в тропической алгебре. Это обычное матричное умножение, в котором умножение (*) заменено на сложение (+), а сложение (+) - на взятие минимума (min).}.

В-третьих, доказано~\cite{tiskin2006all}, что можно произвести декомпозицию данных матриц и хранить их неявно в виде перестановочных матриц  $P_{a,b}$ (так называемое ядро матрицы $H_{a,b}$), что позволяет существенно сократить объем используемой памяти.
Эти ядра также образуют моноид, называемый \emph{моноидом липких кос}, для которых также определена своя операция умножения $\odot$.
Удивительным образом, эти два моноида изоморфны друг другу~\cite{tiskin2006all}, что позволяет использовать $\odot$ вместо $\otimes$. 
Иными словами, для того, чтобы посчитать произведение двух монжевых матриц, можно перейти к косам, произвести над ними операцию $\odot$, тем самым получив результат исходной задачи (справедливо и обратное для липких кос и операции $\otimes$).
    
В-четвертых, доказано~\cite{tiskin2006all}, что  ядро $P_{a,b}$ может быть выражено через конкатенацию двух меньших решений:
\begin{equation}\label{formulaKernelCompistion}
P_{a,b}  = P_{a^{'},b} \odot P_{a^{''},b}, a  := a^{'}a^{''}
\end{equation}
Это позволяет использовать известный принцип \emph{разделяй и властвуй}.

Стоит отметить, что А.\,B. Тискиным изобретен быстрый алгоритм для перемножения двух липких кос --- \emph{алгоритм муравья}~\cite{tiskin2015fast}, имеющий асимптотику $O(n\times \log n)$, где $n$ --- количество ненулевых элементов в перестановочной матрице.
Однако неясно, насколько практическая реализация будет эффективна.

Из всего вышесказанного сразу вытекают следующие два рекурсивных алгоритма, основанные на умножении липких кос (алгоритм~\ref{alg:ant})  и монжевого матричного умножения через $\otimes$ (алгоритм~\ref{alg:monge}).
% algo implitic
\begin{algorithm}[h]
\caption{Рекурсивный алгоритм для решения semi-local lcs через липкое умножение кос}\label{alg:ant}
Вход: строки $a$ и $b$ длин $m$ и $n$ соответственно\\
Выход: ядро  матрицы $H_{a,b}$\\
Комментарии: Сложность алгоритма $O(n \times m)$\\
Псевдокод:

% , не имеет практической реализации
\begin{algorithmic}[1]
\If {$n=1$ и $m=1$}    \Comment{База рекурсии, длина строк 1}
    \If{$a=b$}
        \State Вернуть ядро  $2 \times 2$, отвечающее диагональной
        \par\hskip\algorithmicindent
        единичной матрице
    \Else
        \State Вернуть ядро размера 2 на 2, отвечающее антидиагональной 
        \par\hskip\algorithmicindent единичной матрице
    \EndIf
\Else
    \State Выбрать строку с наибольшей длиной
    \State Разбить её на конкатенацию двух подстрок
    \State Рекурсивно запустить алгоритм над этими двумя половинами
    \State Произвести конкатенацию решений через  $\odot$     
    \State Вернуть полученное решение
\EndIf
\end{algorithmic}
% Стоит отметить, что неизвестно, насколько данный алгоритм эффективен на практике, поскольку он не не имеет практической реализации.
\end{algorithm}
% algos explciit
\begin{algorithm}[h]
\caption{Рекурсивный алгоритм для решения semi-local lcs через явное умножение матриц}\label{alg:monge}
Вход: строки $a$ и $b$ длин $m$ и $n$ соответственно\\
Выход:  матрица $H_{a,b}$\\
Комментарии: Сложность алгоритма $O(n \times m \times \log n)$\\
Псевдокод:

\begin{algorithmic}[1]
\If {$n=1$ и $m=1$}    \Comment{База рекурсии, длина строк 1}
    \If{$a=b$}
        \State Вернуть матрицу, которой отвечает ядро размера 2 на 2,
        \par\hskip\algorithmicindent единичная диагональная матрица
    \Else
        \State Вернуть матрицу, которой отвечает ядро размера 2 на 2,
        \par\hskip\algorithmicindent единичная анти-диагональная матрица
    \EndIf
\Else
    \State Выбрать строку с наибольшей длиной
    \State Разбить её на конкатенацию двух подстрок
    \State Рекурсивно запустить алгоритм над этими двумя половинами
    \State Произвести конкатенацию найденный решений через  $\otimes$ 
    % (умножение через алгоритм swamk )
    \State Вернуть полученное решение
\EndIf
\end{algorithmic}
\end{algorithm}

% В силу того, что матрица $H$--- уни анти монжева, и выполняется (5), и благодаря изоморфизму можно восопльзоваться быстрым умножением монжевых матриц котороые основано на smawk алгоритме.
% Это позволяет добиться сложности $O(n \times m)$.

Также существует итеративный алгоритм, имеющий асимптотическую сложность $O(n \times m)$, основанный на распутывании кос\footnote{ С каждой перестановочной матрицей размера $n \times n$ ассоциирована \emph{сокращенная липкая коса} (\emph{reduced sticky braid}). Одной \emph{сокращенной липкой косе} соответствует бесконечное множество липких кос. Таким образом, произвольную липкую косу можно  \emph{распутать} до  (свести к) \emph{сокращенной липкой косе}, которой уже будет отвечать ядро $P_{a,b}$
}.

% Факт того, что ядро $P_{a,b}$ относится к \emph{моноиду липких кос}, позволяет решить задачу \emph{semi-local lcs} неявно через операцию распутывания кос, которая 
% через так называемое сведение косы к 



% Также, из-за того, что 
% существует итеративный алгоритм, основанный на том факте, что \emph{липкую косу}, можно встроить в  

% Также стоит упомянуть, известный еще с годов итеративный алгоритм~\cite{alavaes}, который также позволяет решить задачу за $O(n \times m)$  

Еще раз отметим, что для простоты изложения некоторые детали при описании опущены для упрощения. Детальное описание можно найти в ~\cite{alex2007semilocal}, как было отмечено выше.

% Таким образом, часть из упомянутых  алгоритмов позволяют решать за ту же асимптотику что и обычный \emph{lcs}.

%  рассказать про semi local

% Аналогичным образом присутствует обобщение на полулокальную версию выравнивания строк  (semi-local sa).

\newpage

% Semi-local SA def
\paragraph*{Задача поиска полулокального выравнивания двух строк (semi-local sa)}\mbox{}

Даны две строки $a$ и $b$ длин $m$ и $n$ соответственно и схема оценки $w = (w_{+}, w_{0} , w_{-})$. 
Необходимо найти значение выравнивания ($sa$) для следующих подзадач:
\begin{itemize}
    \item \textbf{string-subtring}: для каждой из подстрок строки $b$ найти максимальное выравнивание со строкой $a$; 
    \item \textbf{subtring-string}: для каждой из подстрок строки $a$ найти максимальное выравнивание со строкой $b$; 
    \item \textbf{prefix-suffix}: для каждой пары, состоящей из префикса строки $a$ и суффикса строки $b$, найти максимальное выравнивание; 
    \item \textbf{suffix-prefix}: для каждой пары, состоящей из префикса строки $b$ и суффикса строки $a$, найти максимальное выравнивание. 
\end{itemize}
% описать аолгори
Подход для решения этой задачи заключается в следующем.
Заметим, что  схему оценки в  формуле (\ref{formula:sa}) можно свести (нормализовать) к $регулярной схеме оценки$ $w^{'} = (1,\frac{\mu}{v} ,0)$:
\begin{equation}\label{weightNormalization}
    \begin{aligned}
    w = (w_{+}, w_{0} , w_{-}) \xrightarrow{} (w_{+} +2x , w_{0} + 2x , w_{-} + x) =\\ ( \frac{w_{+} +2x}{w_{+} +2x} , \frac {w_{0} + 2x}{w_{+} +2x} , \frac{w_{-} + x}{w_{+} +2x})_{x=-w_{-}} = (1,\frac{\mu}{v} ,0) 
    \end{aligned}
\end{equation}
Тогда для того, чтобы получить значение $sa$, при  известном $sa_{normalized}$ нужно применить обратную регуляризацию:
\begin{equation}
    \begin{aligned}
    sa(a,b,w) = sa_{normalized}  (w_{+} - 2w_{-}) +  w_{-} (m + n)
    \end{aligned}
\end{equation}

Сведение схемы оценки к \emph{регулярной} позволяет уменьшить количество параметров.
Это, в свою очередь, позволяет решить задачу $semi-local$ $sa$ следующим образом\footnote{Детали доказательств и сведений  \cite{tiskin2006all}.}. 
При использовании техники $blown-up$\footnote{Исходные задача увеличивается в $v^2$ раз и сводится к задаче \emph{semi-local lcs}.} асимптотика решения через  итеративный увеличится в $v^2$ раз до $O(m \times n \times v^2)$.
Структура рекурсивного алгоритма через умножение кос позволяет добиться линейной зависимости от параметра $v$. 

При большом значении параметра становится выгоднее использовать алгоритм, который не зависит от схемы оценки и не "раздувает" задачу. 
В данном случае идет речь об описанном ранее алгоритме решения через матричное умножение $\otimes$ (применяется без существенных изменений шагов алгоритма). 
В этом случае асимптотика решения не зависит от $v$ и будет $O(  n \times m \times \log n)$.



\vspace{10 mm}В конце обзора необходимо отметить, что матрица $H_{a,b}$ содержит большое количество информации, что позволяет решать такие задачи, как $ThresholdAMath$, $WindowAMatch$, $FragmentSubstring$, $Window-\\Subsring$, $Smith-Waterman$  $alignment$ и др.
Часть из них будет описаны в соответствующих разделах, относящихся к реализациям алгоритмов для поиска повторов.

Таким образом, реализация большей части алгоритмов, связанных с данной теорией представляет интерес как с точки зрения инженерной задачи, так и со стороны научного исследования.
\begin{itemize}
    \item Алгоритмы обладают хорошими  теоретическими свойствами, но большая часть из них не имеет практической реализации --- неясно насколько они применимы на практике.
    \item Могут ли алгоритмы, решающие задачи \emph{semi-local}, быть адаптированы и успешно применены к области поиска повторов в документации ПО?
\end{itemize}

%  про существующие вариации задач boudnend length и пр

