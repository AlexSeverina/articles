\section*{Problem statement}\label{ps}
The aim of the work is the practical evaluation of whether any performance enhancement could be brought %by partial evaluation technique
by partially evaluating memory accesses through the utilization of AnyDSL framework, compared to CUDA implementations, considering GPU microarchitecture details that affect the result. In order to achieve the aim, the following objectives have been set.
\begin{itemize}
    \item Implement experimental scenarios in both AnyDSL and CUDA.
    \item Collect relevant datasets for the evaluation to be more practical.
    \item Perform the evaluation and analyze the results.
\end{itemize}

More specifically, the work performs the evaluation on string matching and convolutional filtering scenarios, providing some relevant CUDA assembly examples to ground the effects being observed.

% Firstly, the hypothesis of whether GPU-based string pattern matching program performance speed up could be achieved via partially evaluating memory accesses, should have been verified. A modern GPU has different types of memory, varying by access latency. In order to achieve maximal performance every type of memory should be utilized carefully, satisfying alignment and access patterns requirements. Moreover, several cache levels are extensively used to mitigate the latency, and, to some extent, caches could keep up the performance of an application, even if a proper access pattern is hard to achieve. There is a partial evaluator, being developed as part of AnyDSL framework~\cite{LeiBa}, with the support for generation of specialized Nvidia CUDA C code. It has been utilized to verify the hypothesis.   

% Next, the bottlenecks that arise during string pattern matching program specialization should be identified. Particular program transformations could potentially hurt further parallelization or simply not achieve expected effects. Thus, string pattern matching algorithms specialization should be examined using available partial evaluators.

% Then the partial evaluator should be implemented considering the identified bottlenecks either from scratch or as an extension for available ones.

% Finally, the obtained partial evaluator efficiency should be evaluated through performance comparison between specialized programs and manually fine-tuned ones.   