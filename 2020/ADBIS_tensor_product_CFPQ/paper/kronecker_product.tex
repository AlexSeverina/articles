\section{Kronecker Product}
\label{kronecker:section}

In this section, we introduce the Kronecker product definition and its relation to the RSM and a directed graph intersection.

A directed labeled graph could be interpreted as an FSM, where transitions correspond to the labeled edges between vertices of the graph. 
As initial and final states could be chosen all vertices of the graph since this formal decision does not affect the algorithm idea.
As shown in the section~\ref{section:rsm}, an RSM is composed of a set of component state machines, which behave as a normal FSM. 
Notice, that these component machines are structurally independent, and actual communications are done only on \textit{recursive calls} in time of the machine execution. 
It makes an intuition that one could apply automata theory to intersect an RSM with a directed graph.

The classical intersection algorithm of two deterministic FSMs is presented in~\cite{automata:theory:10.5555/1177300}. 
The idea behind this algorithm is to create new FSM, which imitates the parallel work of both machines through an increase in the number of states and transitions. 
For any given FSMs $C_1 = (\Sigma \cup M,Q_1,q_1^0,F_1,\delta_1)$ and $C_2 = (\Sigma \cup M,Q_2,q_2^0,F_2,\delta_2)$ the intersection machine $C$ defines as follows: 

$C = (\Sigma \cup M, Q, q^0, F, \delta)$, where:

\begin{itemize}
    \item States $Q = \{(q_1,q_2)~:~q_1 \in Q_1, q_2 \in Q_2 \}$
    \item Final states $F = \{(q_1,q_2)~:~q_1 \in F_1, q_2 \in F_2 \}$
    \item Initial state $q = (q_1^0,q_2^0)$
    \item Transition function $\delta ((q_1,q_2),a) \to (\delta_1(q_1,a),\delta_2(q_2,a))$, where $a \in \Sigma \cup M$, only if $(\delta_1(q_1,a)$ and $\delta_2(q_2,a)$ are present.
\end{itemize}

A structurally similar procedure for constructing an intersection machine can be implemented using the Kronecker product for the corresponding transition matrices for FSMs if we provide a satisfying operation of elementwise multiplication, which allows further use of high-performance math libraries for intersection computation.

Since an RSM transitions matrix is composed as a blocked matrix of component state machines adjacency matrices, and a directed labeled graph could be presented as an adjacency matrix, it is convenient to implement the intersection of such objects via Kronecker product of the corresponding matrices. As an elementwise operation for one can employ the following function $\bullet: \Sigma \cup M \times \Sigma \cup M \to Boolean$, defined as follows: 

\begin{itemize}
    \item $A_1 \bullet A_2 = true$, if $A_1 \cap A_2 \neq \O$
    \item $A_1 \bullet A_2 = false$, otherwise
\end{itemize}

An example of applying the Kronecker product is illustrated in the section~\ref{example:section}. 

The operation defined in this way allows us to construct an adjacency Boolean matrix for some directed graph, where a path between two vertices exists only if corresponding paths exist in some component state machine of the RSM and in source graph at the same time. 
This fact with transitive closure could be employed in order to extract all the path and components labels from $M$ for context-free reachability problem-solving.