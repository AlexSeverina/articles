\documentclass[12pt]{article}
\usepackage[left=2cm,right=2cm,top=2cm,bottom=2cm,bindingoffset=0cm]{geometry}
\usepackage{fontspec}
\usepackage{polyglossia}
\usepackage{proof}
\usepackage{amsfonts}
\usepackage{amsmath}
\usepackage{ stmaryrd }

\setdefaultlanguage{russian}
\setmainfont[Mapping=tex-text]{CMU Serif}

\newcommand{\sem}[1]{\llbracket #1 \rrbracket}

\begin{document}

Семантика --- отображение из программ в семантический домен.
Программы обычно представляются как абстрактные синтаксические деревья.
Семантический домен --- то, что описывает смысл программы в контексте решаемой задачи.

Семантика бывает разной.
Классические примеры: динамическая семантика (вычисление программ) и статическая семантика (вывод типов).

Доменом динамической семантики арифметических выражений является множество функций из функции, сопоставляющей переменным целочисленные значения, в целые числа.



Семантика часто обозначается скобками $\llbracket$ и $\rrbracket$.

Семаника арифметических выражений:

$$\sem{\cdot} :: (X \rightarrow \mathbb{N}) \rightarrow \mathbb{N}$$


\begin{align*}
  \sem{n}(\sigma) &= n, n \in \mathbb{N} \\
  \sem{v}(\sigma) &= \sigma(v) \\
  \sem{e_1 \oplus e_2}(\sigma) &= \sem{e_1}(\sigma) \otimes \sem{e_2}(\sigma) \\
  & \\
  \sem{!e}(\sigma) &= 0, \sem{e}(\sigma) \neq 0 \\
  \sem{!e}(\sigma) &= 1, \sem{e}(\sigma) \equiv 0 \\ & \\
  \sem{-e}(\sigma) &= \sem{0-e}(\sigma)
\end{align*}

\bigskip

В языке L есть переменные, операции чтения из входного потока и записи в него.
Поэтому один из возможных семантических доменов --- множество функций из тройки (значения переменных, входной поток, выходной поток) в целые числа.

Будем описывать операционную семантику большого шага в терминах правил вывода.
Стрелка $\xrightarrow{p}$ связывает состояние входного $i$ и выходного $o$ потоков и значений переменных $\sigma$ до выполнения инструкции $p$ и после нее.

Абстрактный синтаксис языка L:

\begin{align*}
  data \ LAst  &= If \{ cond :: Expr, thn :: LAst, els :: LAst \} \\
               &\mid \ While \{ cond :: Expr, body :: LAst \} \\
               &\mid \ Assign \{ var :: Var, expr :: Expr \} \\
               &\mid \ Read \{ var :: Var \} \\
               &\mid \ Write \{ expr :: Expr \} \\
               &\mid \ Seq \{ statements :: [LAst] \}
\end{align*}

\bigskip

При обработке инструкции чтения снимаем значение со входного потока и сопоставляем его переменной $x$.

\bigskip

\infer[]{\langle \sigma, (h:t), o \rangle \xrightarrow{read (x)} \langle \sigma[x \mapsto h], t, o \rangle }{}

\bigskip

При обработке инструкции записи вычисляем значение выражения и кладем его во входной поток.

\bigskip


\infer[]{\langle \sigma, i, o \rangle \xrightarrow{write (e)} \langle \sigma, i, (\sem{e}(\sigma)) : o \rangle }{}

\bigskip

При обработке инструкции присвоения вычисляем значение выражения и сопоставляем его переменной.

\bigskip

\infer[, \sem{e}(\sigma) \equiv n]{\langle \sigma, i, o \rangle \xrightarrow{assign \ x \ (e)} \langle \sigma[x \mapsto n], i, o \rangle }{}

\bigskip

При обработке условного перехода вычисляем значение условия, в зависимости от его истинности выполняем соответствующую ветку.

\bigskip

%0 == false
\infer[, \sem{cond}(\sigma) \neq 0]{c \xrightarrow{if \ (cond) \ (thn) \ (els)} c'}{c \xrightarrow{thn} c'}


\bigskip

%0 == false
\infer[, \sem{cond}(\sigma) \equiv 0]{c \xrightarrow{if \ (cond) \ (thn) \ (els)} c'}{c \xrightarrow{els} c'}

\bigskip

Если значение условия в цикле ложно, то конфигурация не меняется

\bigskip

\infer[, \sem{cond}(\sigma) \equiv 0]{c \xrightarrow{while \ (cond) \ body} c}{}

\bigskip

Если значение условия в цикле истинно, то выполняем тело один раз, затем запускаем цикл заново.

\bigskip

\infer[, \sem{cond}(\sigma) \neq 0]{c \xrightarrow{while \ (cond) \ body} c''}{c \xrightarrow{body} c', \ \  c' \xrightarrow{while \ (cond) \ body} c''}

\bigskip

При обработке пустой последовательности инструкций конфигурация не меняется.

\bigskip

\infer[]{c \xrightarrow{seq \ []} c}{}


\bigskip

При обработке непустой последовательности инструкций сначала вычисляем результат выполнения первой инструкции, затем на нем --- всех остальных.

\bigskip

\infer[]{c \xrightarrow{seq \ (h:t)} c''}{c \xrightarrow{h} c', \ \ c' \xrightarrow{seq \ t} c''}

\end{document}

