\documentclass{beamer}
\usepackage{beamerthemesplit}
\usepackage{wrapfig}
\usetheme{SPbGU}
\usepackage{pdfpages}
\usepackage{amsmath}
\usepackage{cmap} 
\usepackage[T2A]{fontenc} 
\usepackage[utf8]{inputenc}
\usepackage[english,russian]{babel}
\usepackage{indentfirst}
\usepackage{amsmath}
\usepackage{tikz}
\usepackage{multirow}
\usepackage[noend]{algpseudocode}
\usepackage{algorithm}
\usepackage{algorithmicx}
\usetikzlibrary{shapes,arrows}
\usepackage{fancyvrb}
\newtheorem{rutheorem}{Теорема}
\newtheorem{ruproof}{Доказательство}
\newtheorem{rudefinition}{Определение}
\newtheorem{rulemma}{Лемма}
\setbeamertemplate{itemize item}[circle]
\beamertemplatenavigationsymbolsempty

\title[]{Теория автоматов и формальных языков}
\subtitle[]{Организационные моменты}
\institute[]{
Санкт-Петербургский государственный электротехнический университет <<ЛЭТИ>>\\
}

\author[]{Екатерина Вербицкая}

\date{5 сентября 2019}

\definecolor{orange}{RGB}{179,36,31}

\begin{document}
{
\begin{frame}
  \titlepage
\end{frame}

}

\begin{frame}[fragile]
  \transwipe[direction=90]
  \frametitle{Контакты}
  \begin{itemize}
    \item Екатерина [Андреевна] Вербицкая
    \item JetBrains Programming Languages and Tools Lab 
    \item email: \href{mailto:kajigor@gmail.com}{kajigor@gmail.com}
    \item email: \href{mailto:ekaterina.verbitskaya@jetbrains.com}{ekaterina.verbitskaya@jetbrains.com} 
    \item Телефон: +7 (921) 765 00 42
    \begin{itemize}
      \item telegram, signal, imessage \dots
    \end{itemize}
    \item Материалы: \url{https://bit.ly/2ltulvt}
  \end{itemize}
\end{frame}

\begin{frame}[fragile]
  \transwipe[direction=90]
  \frametitle{Занятия}
  \begin{itemize}
    \item Занятия по четвергам: лекции и практики
    \item По пятницам, как правило, занятий не будет: только контрольные
    \item Домашние задания выполняются дома, сдаются онлайн
    \begin{itemize}
      \item Сроки сдачи необходимо строго соблюдать
      \item Типичный срок сдачи: до 23:59 среды перед занятием
    \end{itemize}
    \item 2 контрольных
    \item Очный экзамен
  \end{itemize}
\end{frame}

\begin{frame}[fragile]
  \transwipe[direction=90]
  \frametitle{Условия успешной сдачи}
  \begin{itemize}
    \item Домашние задания не обязательные
    \item Контрольные оцениваются баллами
    \item Автоматы за экзамен выставляются в зависимости от суммы баллов по контрольным
  \end{itemize}
  \begin{itemize}
    \item Оценку можно будет улучшить на очном экзамене
  \end{itemize}
\end{frame}

\begin{frame}[fragile]
  \transwipe[direction=90]
  \frametitle{Требования к оформлению домашних заданий}
  \begin{itemize}
    \item Задачи, не подразумевающие кодирование
    \begin{itemize}
      \item pdf-файл, сверстанный по 
шаблону (\url{https://goo.gl/PR2QwI})
      \item Прислать на почту с темой письма ``[ElTech\_FL] HWij''
      \begin{itemize}
        \item ij --- порядковый номер домашнего задания: HW01 --- первое задание; HW11 --- одиннадцатое. 
      \end{itemize}
    \end{itemize}
    \item Инструкции по сдаче кода воспоследуют
    \begin{itemize}
      \item Будьте готовы, что код должен собираться и запускаться под Ubuntu
    \end{itemize}
    \item Списывание \textbf{не допускается}
    \begin{itemize}
      \item Списанные задания не зачитываются
      \item Неоднократно пойманные за списыванием не могут рассчитывать на оценку выше 3 и будут обязаны сдавать экзамен очно
    \end{itemize}
  \end{itemize}
\end{frame}

\begin{frame}[fragile]
  \transwipe[direction=90]
  \frametitle{Литература и материалы}
  \begin{itemize}
    \item Хопкрофт Дж. Э., Мотвани Р., Ульман Дж. Д. Введение в теорию автоматов, языков и вычислений, 2-е изд./Пер. с англ. – М.: Вильямс, 2002.
    \item Ахо А., Ульман Дж. Теория синтаксического анализа, перевода и компиляции. Т.1/ Пер. с англ. – М.: Мир, 1978
    \item Вики-конспект ИТМО по аналогичному курсу: \url{http://goo.gl/we8kvB}
    \item Dick Grune and Ceriel J.H. Jacobs, Parsing Techniques - A Practical Guide, Second Edition (2008): \url{https://bit.ly/2lxP4y2} 
  \end{itemize}
\end{frame}


\end{document}
