\section{Preliminaries}
\label{section_preliminaries}

Let $\Sigma$ be a finite set of edge labels. Define an \emph{edge-labeled directed graph} as a tuple \mbox{$D = (V, E)$} with a set of nodes $V$ and a directed edge relation \mbox{$E \subseteq V \times \Sigma \times V$}.

A path $\pi$ is a list of labeled edges $[e_1,\ldots,e_n]$ where $e_i \in E$. The concatenation of a path $\pi_1$ with a path $\pi_2$ we denote by $\pi_1 + \pi_2$.

For a path $\pi$ in a graph $D$, we denote the unique word, obtained by concatenating the labels of the edges along the path $\pi$ as \mbox{$l(\pi)$}. Also, we write \mbox{$n \pi m$} to indicate, that the path $\pi$ starts at the node \mbox{$n \in V$} and ends at the node \mbox{$m \in V$}.

A \emph{context-free grammar} is a triple \mbox{$G = (N, \Sigma, P)$}, where $N$ is a finite set of non-terminals, $\Sigma$ is a finite set of terminals, and $P$ is a finite set of productions of the following forms:

\begin{itemize}
    \item $A \rightarrow B C$, for $A,B,C \in N$,
    \item $A \rightarrow x$, for $A \in N$ and $x \in \Sigma \cup \{\varepsilon\}$.   
\end{itemize}

We use the conventional notation \mbox{$A \xRightarrow[G]{*} w$} to denote, that a string \mbox{$w \in \Sigma^*$} can be derived from a non-terminal $A$ by some sequence of production rule applications from $P$ in grammar $G$. The \emph{language} of a grammar $G$ with respect to a start non-terminal \mbox{$S \in N$} is defined by $L(G_S) = \{w \in \Sigma^*~|~S \xRightarrow[G]{*} w\}.$

For a given graph \mbox{$D = (V, E)$} and a context-free grammar $G = (N, \Sigma, P)$, we define \emph{context-free relations} \mbox{$R_A \subseteq V \times V$} for every \mbox{$A \in N$}, such that $R_A = \{(n,m)~|~\exists n \pi m~(l(\pi) \in L(G_A))\}.$

For the context-free path query evaluation w.r.t. the single-path query semantics, we must provide such a path for each node pair from $R_A$. In order to do this, we introduce the 
$$\textit{PathIndex} = (\textit{left},\textit{right},\textit{middle},\textit{height},\textit{length})$$ 
--- the elements of matrices which describe the found paths as concatenations of two smaller paths and help to restore each path and derivation tree for it at the end of evaluation. Here \textit{left} and \textit{right} stand for the indexes of starting and ending node in the founded path, \textit{middle} --- the index of intermediate node used in the concatenation of two smaller paths, \textit{height} --- the height of the derivation tree of the string corresponding to the founded path, and \textit{length} is a length of founded path. When we do not find the path for some node pair $i,j$, we use the $\textit{PathIndex} = \bot = (0,0,0,0,0)$.

Also, we will use the notation \textit{proper matrix} which means that for every element of the matrix with indexes $i,j$ it either $\textit{PathIndex} = (i,j,\_,\_,\_)$ or $\bot$.

For proper matrices we use a binary operation $\otimes$ defined for PathIndexes \mbox{$PI_1, PI_2$} which are not equal to $\bot$ and with $PI_1.\textit{right} = PI_2.\textit{left}$ as 

\begin{align*}
PI_1 \otimes PI_2 = (&PI_1.left, PI_2.right, PI_1.right, \\
                     &max(PI_1.height, PI_2.height)+1,\\
                     &PI_1.length + PI_2.length).
\end{align*}

If at least one operand is equal to $\bot$ then $PI_1 \otimes PI_2 = \bot$.

For proper matrices we also use a binary operation $\oplus$ defined for PathIndexes \mbox{$PI_1, PI_2$} which are not equal to $\bot$ with $PI_1.\textit{left} = PI_2.\textit{left}$ and $PI_1.\textit{right} = PI_2.\textit{right}$ as $PI_1$ if $PI_1.\textit{height} \leq PI_2.\textit{height}$ and $PI_2$ otherwise. If only one operand is equal to $\bot$ then $PI_1 \oplus PI_2$ equal to another operand. If both operands are equal to $\bot$ then $PI_1 \oplus PI_2 = \bot$.

Using $\otimes$ as multiplication of PathIndexes, and $\oplus$ as an addition, we can define a \emph{matrix multiplication}, \mbox{$a \odot b = c$}, where $a$ and $b$ are matrices of a suitable size, that have PathIndexes as elements, as $c_{i,j} = \bigoplus^{n}_{k=1}{a_{i,k} \otimes b_{k,j}}.$

Also, we use the element-wise $+$ operation on matrices $a$ and $b$ with the same size: \mbox{$a + b = c$}, where $c_{i,j} = a_{i,j} \oplus b_{i,j}.$
