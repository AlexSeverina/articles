\section{Matrix-based CFPQ for Single-Path Semantics}
In this section, we propose the matrix-based algorithm for CFPQ w.r.t. the single-path query semantics (see listing~\ref{lst:algo2}). This algorithm constructs the set of matrices $T$ with PathIndexes as elements.
{\footnotesize
	\begin{algorithm}
		\floatname{algorithm}{Listing}
		\begin{algorithmic}[1]
			\caption{CFPQ algorithm w.r.t. single-path query semantics}
			\label{lst:algo2}
			\Function{evalCFPQ}{$D=(V,E), G=(N,\Sigma,P)$}
			\State{$n \gets$ |V|}
			\State{$T \gets \{T^{A_i} \mid A_i \in N, T^{A_i}$ is a matrix $n \times n$, $T^{A_i}_{k,l} \gets \bot$ \} }
			\ForAll{$(i,x,j) \in E$, $A_k \mid A_k \to x \in P$}
			%\Comment{Matrices initialization}
			%\For{$A_k \mid A_k \to x \in P$}
			{$T^{A_k}_{i,j} \gets (i,j,i,1,1)$}
			%\EndFor
			\EndFor
			\For{$A_k \mid A_k \to \varepsilon \in P$}
			{$T^{A_k}_{i,i} \gets (i,i,i,1,0)$}
			\EndFor
			
			\While{any matrix in $T$ is changing}
			%\Comment{Transitive closure calculation}
			\For{$A_i \to A_j A_k \in P$}
			{ $T^{A_i} \gets T^{A_i} + (T^{A_j} \odot T^{A_k})$ } 
			\EndFor
			\EndWhile
			\State \Return $T$
			\EndFunction
		\end{algorithmic}
	\end{algorithm}
}

After constructing a set of matrices $T$, we can extract a path $i \pi j$ for every node pair $i, j$ and non-terminal $A$ such that $A \xRightarrow[G]{*} l(\pi)$ if such path exists. We also propose the algorithm (see listing~\ref{lst:algo3}) for extracting one of those paths which forms a string with minimal height of derivation tree. Our algorithm returns the empty path $[]$ only if $i = j$ and $A \to \varepsilon \in P$. Note that if the PathIndex for given $i,j,A$ is equal to $\bot$ then our algorithm returns a special path $\pi_{\emptyset}$ to denote that such a path does not exist.

{\footnotesize
	\begin{algorithm}
		\floatname{algorithm}{Listing}
		\begin{algorithmic}[1]
			\caption{Path extraction algorithm}
			\label{lst:algo3}
			\Function{extractPath}{$i, j, A, T=\{T^{A_i}\}, G=(N,\Sigma,P)$}
			\State{$index \gets T^{A}_{i,j}$ }
			
			\If{$index = \bot$}
			\State \Return $\pi_{\emptyset}$
			\Comment{Such a path does not exist}
			\EndIf
			
			\If{$index.height = 1$}
			\If{$index.length = 0$}
			\State \Return $[]$
			\Comment{Return an empty path}
			\EndIf
			\ForAll{$ x \mid (i,x,j) \in E$}
			\If{$A \to x \in P$}
			\State \Return $[(i,x,j)]$
			\Comment{Return a path of length one}
			\EndIf
			\EndFor
			\EndIf
			
			\ForAll{$A \to B C \in P$}
			\State{$index_B \gets T^{B}_{i,index.middle}$ }
			\State{$index_C \gets T^{C}_{index.middle,j}$ }			
			\If{$(index_B \neq \bot) \wedge (index_C \neq \bot)$}
			\State{$maxH \gets max(index_B.height, index_C.height)$ }
			\If{$index.height = maxH + 1$}
			
						
			\State{$\pi_1 \gets$ \Call{extractPath}{$i, index.middle, B, T, G$}}
			\State{$\pi_2 \gets$ \Call{extractPath}{$index.middle, j, C, T, G$}}
			\State \Return $\pi_1 + \pi_2$
			\Comment{Return the concatenation of paths}
			\EndIf
			\EndIf
			\EndFor
			\EndFunction
		\end{algorithmic}
	\end{algorithm}
}

\subsection{Correctness}

The following correctness theorem holds (proof can be found in appendix~\ref{section:lemma_proof}).

\begin{mytheorem}\label{thm:correct}
	Let $D = (V,E)$ be a graph and let $G =(N,\Sigma,P)$ be a grammar. Then for any $i, j$ and for any non-terminal $A \in N$, $index = T^A_{i,j}$ and $index = (i,j,k,h,l) \neq \bot$ iff $(i,j) \in R_A$ and $i \pi j$, such that there is a derivation tree of the minimal height $h$ for the string $l(\pi)$ of length $l$ and a context-free grammar $G_A = (N,\Sigma,P,A)$.
\end{mytheorem}

Now, using the theorem~\ref{thm:correct} and induction on the length of the path, it can be easily shown that the following theorem holds.

\begin{mytheorem}\label{thm:correct_extraction}
	Let $D = (V,E)$ be a graph, let $G =(N,\Sigma,P)$ be a grammar and $T$ be a set of matrices returned by the algorithm in listing~\ref{lst:algo2}. Then for any $i, j$ and for any non-terminal $A \in N$ such that $index = T^A_{i,j}$ and $index = (i,j,k,h,l) \neq \bot$, the algorithm in listing~\ref{lst:algo3} for these parameters will return a path $i \pi j$ such that $(i,j) \in R_A$ and there is a derivation tree of the minimal height $h$ for the string $l(\pi)$ of length $l$ and a context-free grammar $G_A = (N,\Sigma,P,A)$.
\end{mytheorem}

We can, therefore, determine whether $(i,j) \in R_A$ by asking whether $T^A_{i,j} = \bot$. Also, we can extract such a path which forms a string with a derivation tree of minimal height by using our algorithm in listing~\ref{lst:algo3}. Thus, we show how the context-free path query evaluation w.r.t. the single-path semantics can be solved in terms of matrix operations.

\subsection{Complexity}

Denote the number of elementary operations executed by the algorithm of multiplying two $n \times n$ matrices with PathIndexes as $MM(n)$. Also, denote the number of elementary operations, executed by the matrix element-wise + operation of two $n \times n$ matrices with PathIndexes as $MA(n)$. Since the line \textbf{7} of the algorithm in listing~\ref{lst:algo2} is executed no more than $|V|^2|N|$ times (for the same reasons as in the original paper~\cite{Azimov:2018:CPQ:3210259.3210264} of the matrix-based CFPQ algorithm), the following theorem holds.

\begin{myproposition}\label{thm:time}
	Let $D = (V,E)$ be a graph and let $G =(N,\Sigma,P)$ be a grammar. The algorithm in listing~\ref{lst:algo2} calculates the set of matrices $T$ in $O(|V|^2|N|^3(MM(|V|) + MA(|V|)))$.
\end{myproposition}

Also, denote the time complexity of the access to the PathIndex in the $n \times n$ matrix as $Access(n)$. Then the following theorem on the time complexity of the path extraction algorithm holds.

\begin{myproposition}\label{thm:time_extraction}
	Let $D = (V,E)$ be a graph, let $G =(N,\Sigma,P)$ be a grammar and $T$ be a set of matrices returned by the algorithm in listing~\ref{lst:algo2}. Then for any $i, j$ and for any non-terminal $A \in N$ such that $index = T^A_{i,j}$ and $index = (i,j,k,h,l) \neq \bot$, the algorithm in listing~\ref{lst:algo3} for these parameters calculates a path $i \pi j$ in $O(l \times N \times Access(|V|))$.
\end{myproposition}

\subsection{An Example}
In this section, we provide a step-by-step demonstration of the proposed algorithms. For this, we consider the example with the worst-case time complexity.

We run the query on a graph $D$, presented in Figure~\ref{Example_Graph}. We provide a step-by-step demonstration of the work of algorithm in listing~\ref{lst:algo2} with the given graph $D$ and grammar $G'$, presented in Figure~\ref{ProductionRulesExampleQueryCNF}. After the matrix initialization in lines \textbf{3-5} of this algorithm, we have a matrix $T^{(1)}$, presented on Figure~\ref{ExampleQueryInitMatrix}.

{\footnotesize
\begin{figure}[h]
	\[
	T^{(1),A} = \begin{pmatrix}
	\bot & (0,1,0,1,1)       & \bot & \bot       \\
	\bot & \bot & (1,2,1,1,1)       & \bot \\
	(2,0,2,1,1)       & \bot & \bot & \bot \\
	\bot       & \bot & \bot & \bot \\
	\end{pmatrix}
	\]
	\[
	T^{(1),B} = \begin{pmatrix}
	\bot & \bot       & \bot & (0,3,0,1,1)       \\
	\bot & \bot & \bot       & \bot \\
	\bot       & \bot & \bot & \bot \\
	(3,0,3,1,1)      & \bot & \bot & \bot \\
	\end{pmatrix}
	\]
	\caption{The initial matrix for the example query. The PathIndexes $T^{(1),S_1}_{i,j}$ and $T^{(1),S}_{i,j}$ are equal to $\bot$ for every $i,j$.}
	\label{ExampleQueryInitMatrix}
\end{figure}
}

After the initialization, the only matrices which will be updated are $T^{S_1}$ and $T^{S}$. These matrices obtained after the first loop iteration is shown in Figure~\ref{ExampleQueryFirstIteration}.

{\footnotesize
\begin{figure}[h]
	\[
	T^{(2),S} = \begin{pmatrix}
	\bot & \bot       & \bot & \bot       \\
	\bot & \bot & \bot       & \bot \\
	\bot       & \bot & \bot & (2,3,0,2,2) \\
	\bot       & \bot & \bot & \bot \\
	\end{pmatrix}
	\]
	\caption{The first iteration of computing the transitive closure for the example query. The PathIndexes $T^{(1),S_1}_{i,j}$ are equal to $\bot$ for every $i,j$.}
	\label{ExampleQueryFirstIteration}
\end{figure}
}

When the algorithm at some iteration finds new paths for some non-terminal in the graph $D$, then it adds corresponding PathIndexes to the matrix for this non-terminal. For example, after the first loop iteration, PathIndex $(2,3,0,2,2)$ is added to the matrix $T^{S}$. This PathIndex is added to the element with a row index $i = 2$ and a column index $j = 3$. This means, that there is $i\pi j$ (a path $\pi$ from the node 2 to the node 3), such that $S \xRightarrow[G']{*} l(\pi)$, this path obtained by concatenation of smaller paths via node 0, the length of the path is equal to 2, and the derivation tree for $l(\pi)$ has a height 2.

The calculation of the transitive closure is completed after $k$ iterations, when a fixpoint is reached: $T^{(k)} = T^{(k+1)}$. For the example query, $k = 13$ since $T_{13} = T_{14}$. The resulted matrix for non-terminal $S$ is presented on Figure~\ref{ExampleQueryFinalMatrices}.

{\footnotesize
\begin{figure}[h]
	\[
	T^{(14),S} = \begin{pmatrix}
	(0,0,1,12,12) & \bot       & \bot & (0,3,1,6,6)       \\
	(1,0,2,4,4) & \bot & \bot       & (1,3,2,10,10) \\
	(2,0,0,8,8)       & \bot & \bot & (2,3,0,2,2) \\
	\bot       & \bot & \bot & \bot \\
	\end{pmatrix}
	\]
	\caption{The final matrix for non-terminal $S$ after computing the transitive closure.}
	\label{ExampleQueryFinalMatrices}
\end{figure}
}

Now, after constructing the transitive closure, we can construct the context-free relations. For example, the relation $R_S=\{(0,0),(0,3),(1,0),(1,3),(2,0),(2,3)\}$.

In the relation $R_S$, we have all node pairs corresponding to paths, whose labeling is in the language $L(G'_S) = \{a^n b^n \mid n \geq 1\}$. Using the algorithm in listing~\ref{lst:algo3} we can restore paths for each node pair from context-free relations. For example, given $i=j=0$, non-terminal $S$, set of resulted matrices $T$, and context-free grammar $G'$, the algorithm in listing~\ref{lst:algo3} returns a path $0\pi 0$ whose labeling forms a string $l(\pi) = a^6 b^6$. The length of path $\pi$ is equal to 12 and the height of the derivation tree for $l(\pi)$ is equal to 12, which is consistent with the corresponding PathIndex $T^{(14),S}_{0,0}$.