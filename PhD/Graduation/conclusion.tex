% У заключения нет номера главы
\clearpage

\section*{Заключение}
В статье описан алгоритм синтаксического анализа регулярного множества. Данный алгоритм может быть применён, например, для синтаксического анализа динамически формируемых выражений, в случае, 
когда множество возможных значений выражения приближается регулярным множеством. Подобные задачи могут возникать при поддержке встроенных текстовых языков в интегрированных средах разработки, 
при анализе и модификации программного обеспечения в процессе реинжиниринга. 

Представленный алгоритм был реализован на языке F\#~\cite{FSharp} в качестве модуля проекта YaccConstructor с переиспользованием ранее реализованного генератора парсеров на основе RNGLR-алгоритма. 
На основе нового модуля реализован плагин~\cite{SECR14} к ReSharper, который предоставляет поддержку встроенного T-SQL в C\#. Выполняется подсветка синтаксиса, подсветка парных элементов,  статический поиск 
ошибок и их подсветка в редакторе. Исходный код опубликован в открытом доступе и доступен на сайте \url{https://github.com/YaccConstructor/YaccConstructor}.

В дальнейшем необходимо выполнить теоретическую оценку сложности алгоритма и требуемого объёма памяти. С практической точки зрения требуется оптимизация алгоритма и улучшение качества диагностики ошибок. 
Отдельных исследований требует  работа с семантикой и возможность трансляции и трансформаций динамически формируемых выражений.                                                                                                                                                                                                                                                                                              
