\chapter*{Введение}                         % Заголовок
\addcontentsline{toc}{chapter}{Введение}    % Добавляем его в оглавление
\textbf{Актуальность работы}

Статический анализ исходного кода~\cite{StaticCodeAnalysis3,StaticCodeAnalysis2,StaticCodeAnalysis1} --- получение знаний о программе без её исполнения --- является неотъемлемой частью многих процессов, связанных с разработкой программного обеспечения. Он может использоваться, например, в средах разработки для упрощения работы с кодом --- подсветка синтаксиса, навигация по коду, контекстные подсказки; для обнаружения проблем на ранних стадиях (до запуска программы) --- статический поиск ошибок.  Кроме того, статический анализ используется при решении задач трансформации исходного кода и реинжиниринге~\cite{reengANT}. Однако многие языки программирования позволяют использовать конструкции, которые существенно затрудняют статический анализ кода.

Например, взаимодействие приложений с базами данных, часто реализуется с помощью встроенных языков: приложение, созданное на одном языке, генерирует программу на другом языке и передаёт её на выполнение в соответствующее окружение. Примерами могут служить не только динамические SQL-запросы к базам данных в Java-коде, но и формирование HTML-страниц в PHP-приложениях, динамический SQL (Dynamic SQL~\cite{DSQLISO}), фреймворк JSP~\cite{JSP}, PHP mySQL interface~\cite{PHPmySQL}. Генерируемый код собирается из строк таким образом, чтобы в момент выполнения результирующая строка представляла собой корректное выражение на соответствующем языке. Примеры использования встроенных языков представлены в листингах~\ref{lst:dsql1},~\ref{lst:JsJava} и~\ref{lst:PhPSqlHtml}. более того, одна программа может генерировать код на нескольких языках (листинг~\ref{lst:PhPSqlHtml}).

Такой подход весьма гибок, так как позволяет использовать для формирования выражений различные строковые операции (replace, substring и т.д.) и получать части кода из различных источников (например, учитывать текстовый ввод пользователя, что часто используется для задания фильтров при конструировании SQL-запросов). Кроме того, использование динамически формируемых строковых выражений избавлено от  дополнительных накладных расходов, присущих таким технологиям, как ORM\footnote{ORM или Object-Relational Mapping --- технология программирования, которая связывает базы данных с концепциями объектно-ориентированных языков программирования~\cite{ORM}.}, что позволяет достичь высокой производительности. Благодаря этим факторам, использование динамически генерируемых программ было широко распространено и большое количество систем используют такой подход. Вместе с этим, несмотря на появление новых технологий, динамическая генерация SQL-запросов активно используется и в настоящее время~\cite{DSQLInActiveUse}.

\fvset{frame=lines,framesep=5pt,fontsize=\small}\

\begin{listing}
    \begin{pyglist}[language=sql,numbers=left,numbersep=5pt]

CREATE PROCEDURE [dbo].[MyProc]  @TABLERes   VarChar(30)
AS
    EXECUTE ('INSERT INTO ' + @TABLERes + ' (sText1)' +
             ' SELECT ''Additional condition: '' + sName' +
             ' from #tt where sAction = ''1000000''')
GO
    \end{pyglist}
\caption{Код с использованием динамического SQL}
\label{lst:dsql1}
\end{listing} 
 
\fvset{frame=lines,framesep=5pt}
\begin{listing}
    \begin{pyglist}[language=java,numbers=left,numbersep=5pt]
import javax.script.*;  
public class InvokeScriptFunction {  
    public static void main(String[] args) throws Exception {  
        ScriptEngineManager manager = new ScriptEngineManager();  
        ScriptEngine engine = manager.getEngineByName("JavaScript");  
        // JavaScript code in a String  
        String script = 
            "function hello(name) { print('Hello, ' + name); }";  
        // evaluate script  
        engine.eval(script);  
        // javax.script.Invocable is an optional interface.  
        // Check whether your script engine implements or not!  
        // Note that the JavaScript engine implements
        // Invocable interface.  
        Invocable inv = (Invocable) engine;  
        // invoke the global function named "hello"  
        inv.invokeFunction("hello", "Scripting!!" );  
    }  
}
    \end{pyglist}
\caption{Вызов JavaScript из Java}
\label{lst:JsJava}
\end{listing}


\fvset{frame=lines,framesep=5pt}
\begin{listing}
    \begin{pyglist}[language=php,numbers=left,numbersep=5pt]

<?php
    // Embedded SQL
    $query = 'SELECT * FROM ' . $my_table; 
    $result = mysql_query($query);
    
    // HTML markup generation
    echo "<table>\n";
    while ($line = mysql_fetch_array($result, MYSQL_ASSOC)) {
        echo "\t<tr>\n";    
        foreach ($line as $col_value) {
            echo "\t\t<td>$col_value</td>\n";
        }
        echo "\t</tr>\n";
    }
    echo "</table>\n";
?>
    \end{pyglist}
\caption{Использование нескольких встроенных в PHP языков (MySQL, HTML)}
\label{lst:PhPSqlHtml}
\end{listing}



Динамически формируемые выражения часто конструируются посредством конкатенации в циклах, ветках условных операторов или рекурсивных процедурах, что приводит к получению множества различных значений для каждого выражения в момент выполнения. При этом фрагменты кода на встроенных  языках воспринимаются компилятором исходного языка как простые строки, не подлежащие дополнительному анализу. Невозможность статической проверки корректности формируемого выражения приводит к высокой вероятности возникновения ошибок во время выполнения программы. В худшем случае такая ошибка не приведёт к прекращению работы приложения, что указало бы на проблемы, однако целостность данных при этом может оказаться нарушена. Более того, использование динамически формируемых выражений затрудняет как разработку информационных систем, так и реинжиниринг уже созданных. Для реинжиниринга важно иметь возможность изучать систему и модифицировать её, сохраняя функциональность. Однако, например, при наличии в коде приложения встроенного SQL нельзя, не проанализировав все динамически формируемые выражения, точно ответить на вопрос о том, с какими элементами базы данных не взаимодействует система, и удалить их. При переносе такой системы на другую СУБД необходимо гарантировать, что для всех динамически формируемых выражений значение в момент выполнения будет корректным кодом на языке новой СУБД~\cite{JSquash}. С другой стороны, распространённой практикой при написании кода является использование интегрированных сред разработки (Integrated Development Environment, IDE), выполняющих подсветку синтаксиса и автодополнение, сигнализирующих о синтаксических ошибках, предоставляющих возможность проводить рефакторинг кода. Такая функциональность значительно упрощает процесс разработки и отладки приложений и полезна не только для основного языка, но и для встроенных языков. Для решения таких задач необходимы инструменты, проводящие статический анализ множества выражений, которые могут быть получены на этапе исполнения из строковых выражений исходного языка.


\textbf{Степень разработанности темы}

Анализу динамически формируемых строковых выражений посвящены работы таких зарубежных учёных как Кюнг-Гу Дох (Kyung-Goo Doh)~\cite{LrAbstract1,LrAbstract2,LRAbstractParsingSema}, Ясухико Минамиде (Minamide Yasuhiko)~\cite{PHPSA}, Андерс Мёллер (Anders M/oller)~\cite{JSA} и отечественных учёных А.А.~Бреслава~\cite{Alvor1,Alvor2} и других. Вопросы проверки корректности динамически формируемых выражений и поиска фрагментов кода, уязвимых для SQL-инъекций~\cite{SQLInjection,Dasgupta:2009:SAF:1546683.1547548} достаточно подробно изучены, однако актуальным представляется так же проведение более сложных видов статического анализа, требующих построения структурного представления динамически формируемого кода. То есть требуется механизм синтаксического анализа динамически формируемых выражений, позволяющий строить лес вывода для всех возможных значений соответствующего выражения. 

Методы обобщённого синтаксического анализа, лежащие в основе данной работы, изложены в трудах таких учёных как Масару Томита (Masaru Tomita)~\cite{Tomita}, Элизабет Скотт (Elizabeth Scott) и Адриан Джонстон (Adrian Johnstone)~\cite{RNGLR,RIGLR} из университета Royal Holloway (Великобритания), Ян Рекерс (Jan Rekers, University of Amsterdam)~\cite{SPPF}, Элко Виссер (Eelco Visser)~\cite{RNGLRSyntaxerror2,RNGLRSyntaxerror3} и других.

Так же важным является предоставление компонентов, упрощающих создание новых инструментов для решения конкретных задач. Данных подход хорошо исследован в области разработки компиляторов, где широкое распространение получили генераторы анализаторов и пакеты стандартных библиотек (работы А.~Ахо~\cite{Dragon}, А.~Брукера~\cite{CompilerCompiler}, С.~Джонсона~\cite{yaccBook} и др.). 

В работах отечественных учёных М.Д.~Шапот и Э.В.~Попова~\cite{DynamicDSQLTranslation}, а так же зарубежных учёных Антони Клеви (Anthony Cleve), Жан-Люк Эно (Jean-Luc Hainaut)~\cite{DSQLReverseEngineering}, Йост Виссер (Joost Visser)~\cite{DSQLQualityMesure} и других рассматриваются различные аспекты реинжиниринга систем, использующих встроенные SQL-запросы, однако не формулируется общего метода реинжиниринга таких систем. Разработка такого метода является актуальной задачей.

Таким образом, актуальной является задача дальнейшего исследования синтаксического анализа динамически формируемых строковых выражений, а так же возможностей построения структурного представления динамически формируемого кода. Кроме этого важным является решение вопросов практического применения средств анализа динамически формируемого кода: упрощение разработки инструментов анализа и создание методов их применения в реинжиниринге программного обеспечения.

\textbf{Объект исследования}

 Объектом исследования являются методы, алгоритмы и программные средства обработки динамически формируемых программ, а также задачи реинжиниринга информационных систем.

\textbf{Цель и задачи диссертационной работы}

\textbf{Целью} данной работы является создание комплексного подхода к статическому синтаксическому анализу динамически формируемых программ.

Для достижения поставленной цели сформулированы следующие \textbf{задачи}.
\begin{enumerate}
    \item Разработать алгоритм синтаксического анализа динамически формируемых программ, позволяющий обрабатывать регулярную аппроксимацию множества значений выражения в точке выполнения, и гарантирующий конечность представления леса вывода.
    \item Разработать архитектуру инструментария для разработки программных средств статического анализа динамически формируемых строковых выражений.
    \item Разработать метод анализа и обработки встроенного программного кода в проектах по реинжинирингу информационных систем.
\end{enumerate}

\textbf{Методология и методы исследования}

Методология исследования основана на подходе к спецификации и анализу формальных языков, который начал активно развиваться в 50-х годах 20-го века при изучении естественных языков (работы Н. Хомского~\cite{chomskyMethod ,chomskySyntactic}). В последствии этот подход получил широкое распространение в областях, связанных с обработкой языков программирования.
При этом основными элементами данного подхода являются алфавит и грамматика языка, разбиение автоматической обработки языка на выполнение таких шагов, как лексический, синтаксический и семантический анализ. Решаемые в связи с этим задачи связаны с поиском эффективных алгоритмов, выполняющих эти шаги. 

В работе применяется алгоритм обобщённого восходящего синтаксического анализа RNGLR~\cite{RNGLR}, созданный Элизабет Скотт (Elizabeth Scott) и Адриан Джонстон (Adrian Johnstone) из университета Royal Holloway (Великобритания). Для компактного хранения леса вывода использовалась структура данных Shared Packed Parse Forest (SPPF), которую предложил Ян Рекерс (Jan Rekers, University of Amsterdam).

Доказательство завершаемости и корректности предложенного алгоритма проводилось с применением теории формальных языков, теории графов и теории сложности алгоритмов. Приближение множества значений динамически формируемого выражения строилось в виде регулярного множества, описываемого с помощью конечного автомата.


\textbf{Положения, выносимые на защиту}
\begin{enumerate}
    \item Разработан алгоритм синтаксического анализа динамически формируемых выражений, позволяющий обрабатывать произвольную регулярную аппроксимацию множества значений выражения в точке выполнения, реализующий 
    эффективное управление стеком и гарантирующий конечность представления леса вывода. Доказана завершаемость и корректность предложенного алгоритма при анализе регулярной аппроксимации, представимой в виде произвольного конечного автомата без $\varepsilon$-переходов. 
    \item Создана архитектура инструментария для разработки программных средств статического анализа динамически формируемых строковых выражений.
    \item Разработан метод анализа и обработки встроенного программного кода в проектах по реинжинирингу информационных систем. 
\end{enumerate}


\textbf{Научная новизна работы}

Научная новизна полученных в ходе исследования результатов заключается в следующем.

\begin{enumerate}

\item Алгоритм, предложенный в диссертации, отличается от аналогов (работы Андрея Бреслава~\cite{Alvor1, Alvor2}, Кюнг-Гу Дох~\cite{LrAbstract1, LrAbstract2}, Ясухико Минамиде~\cite{PHPSA}) возможностью построения компактной структуры данных, содержащей деревья вывода для всех корректных значений выражения, при проведении синтаксического анализа динамически формируемых выражений. Это позволяет как проверять корректность анализируемых выражений, так и проводить более сложные виды анализа, используя деревья вывода, хранящиеся в построенной структуре данных. Многие из существующих алгоритмов предназначены только для проверки корректности выражений (JSA~\cite{JSA}~\cite{Alvor1, Alvor2}, PHPSA~\cite{PHPSA}), основанной на решении задачи о включении одного языка в другой. Выполнение более сложных видов анализа, трансформаций или построения леса разбора не предполагается. В работах А. Бреслава и Кюнг-Гу Дох (Kyung-Goo Doh)~\cite{LrAbstract1, LrAbstract2} рассматривается применение механизмов синтаксического анализа для работы с динамически формируемыми выражениями, однако не решается вопрос 
эффективного представления результатов разбора. 

\item Новизна представленной архитектуры заключается в том, что она позволяет создать платформу для разработки инструментов, решающих широкий круг задач анализа динамически формируемого кода. Существующие архитектуры готовых инструментов для анализа динамически формируемых строковых выражений (JSA, PHPSA, Alvor, Varis) предназначены для решения конкретных задач для определённых языков. Решение новых задач или поддержка других языков с помощью этих инструментов затруднены ввиду ограничений, накладываемых архитектурой и возможностями используемого алгоритма анализа. 

\item Метод анализа и обработки встроенного программного кода в проектах по реинжинирингу информационных систем предложен впервые. Как отмечают К.В.~Ахтырченко и Т.П.~Сорокваша~\cite{SoftwareReengMethods}, присутствуют проблемы в методологической основе реинжиниринга: существующие работы в области реинжиниринга программного обеспечения либо содержат высокоуровневые решения, не касающиеся деталей, важных при решении прикладных задач (например, работы К. Вагнера~\cite{SoftwareReeng3}, Х. Миллера~\cite{SoftwareReeng2}), либо являются набором подходов к решению конкретных задач (например, работы~\cite{SoftwareReeng1, reengANT}). При этом, встроенный программный код часто не учитывается. С другой стороны, работы, например, М.Д.~Шапот и Э.В.~Попова~\cite{DynamicDSQLTranslation}, С.Л.~Трошина~\cite{reengANT}, А.~Клеви~\cite{DSQLReverseEngineering},  посвящены решению конкретных задач обработки встроенного программного кода в контексте реинжиниринга информационных систем, но не предлагают формального метода их решения.

\end{enumerate}


\textbf{Теоретическая и практическая значимость работы}

Теоретическая значимость диссертационного исследования заключается в разработке формального алгоритма синтаксического анализа динамически формируемого кода, решающего задачу построения конечного представления леса вывода, не решаемую ранее, а также в формальном доказательстве завершаемости и корректности разработанного алгоритма. 

На основе полученных в работе научных результатов был разработан инструментарий (Software Development Kit, SDK), предназначенный для создания средств статического анализа динамически формируемых выражений. Данный инструментарий позволяет автоматизировать создание лексических и синтаксических анализаторов при разработке программных средств для решения задач реинжиниринга --- изучения и инвентаризации систем, автоматизации трансформации выражений на встроенных языках. Также данный инструментарий может использоваться при реализации поддержки встроенных языков в интегрированных средах разработки.

С использованием разработанного инструментария было реализовано расширение к инструменту ReSharper (ООО ``ИнтеллиДжей Лабс'', Россия), предоставляющее поддержку встроенного T-SQL в проектах на языке программирования C\# в среде разработки Microsoft Visual Studio. Так же было выполнено внедрение результатов работы в промышленный проект по переносу хранимого SQL-кода с MS-SQL Server 2005 на Oraclе 11gR2 (ЗАО ``Ланит-Терком'', Россия). 

\textbf{Степень достоверности и апробация результатов}

Достоверность и обоснованность результатов исследования опирается на использование формальных методов исследуемой области, проведенные доказательства, рассуждения и эксперименты.

Основные результаты работы были доложены на ряде научных конференций: SECR-2012, SECR-2013, SECR-2014, TMPA-2014, Parsing@SLE-2013, Рабочий семинар ``Наукоемкое программное обеспечение'' при конференции PSI-2014. Доклад на SECR-2014 награждён премией Бертрана Мейера за лучшую исследовательскую работу в области программной инженерии. Разработка инструментальных средств на основе предложенного алгоритма была поддержана Фондом содействия развитию малых форм предприятий в технической сфере (программа УМНИК~\cite{UMNIC}, проект \textnumero~162ГУ1/2013 и проект \textnumero~5609ГУ1/2014).

\textbf{Публикации по теме диссертации}

Результаты диссертации изложены в 7 научных работах из которых 3~\cite{YCArticle,SELforIDEru,AbstractGLL} опубликованы в журналах из перечня рецензируемых научных изданий, 
1~\cite{GLRAbsPars} индексируются Scopus. В~\cite{Syrcose} С. Григорьеву принадлежит реализация ядра платформы YaccConstructor. В~\cite{SELforIDEru, AbstractGLL} и~\cite{SELforIDE} С. Григорьеву принадлежит постановка задачи, формулирование требований к разрабатываемым инструментальным средствам, работа над текстом. 
В~\cite{GLRAbsPars} автору принадлежит идея, описание и реализация анализа встроенных языков на основе RNGLR алгоритма.  В~\cite{YCArticle} С. Григорьеву принадлежит реализация инструментальных средств, проведение экспериментов, работа над текстом. В работе~\cite{RelaxedARNGLR} автору принадлежит алгоритм синтаксического анализа динамически формируемого кода.


\textbf{Структура работы}

Диссертация состоит из введения, шести глав, заключения и построена следующим образом. В первой главе проводится обзор области исследования. Рассматриваются подходы к анализу динамически формируемых строковых выражений и соответствующих инструментов. Кроме того, описывается алгоритм обобщённого восходящего синтаксического анализа RNGLR, положенный в основу алгоритма, предложенного в данной работе. Также описываются 
проекты YaccConstructor и ReSharper SDK, использующиеся в качестве основы разработанного инструментального пакета. Во второй главе формализуется основная задача исследования и излагается алгоритм, её решающий, --- алгоритм синтаксического анализа регулярного множества на основе RNGLR, строящий конечную структуру данных, содержащую деревья вывода для всех цепочек анализируемого множества. Приводятся доказательства завершаемости и корректности представленного алгоритма, поясняются шаги работы алгоритма на примерах. В третьей главе описывается инструментальный пакет YC.SEL.SDK, разработанный в ходе данной работы на основе алгоритма, описанного во второй главе. YC.SEL.SDK предназначен для разработки инструментов анализа динамически формируемых выражений. Описывается архитектура компонентов и особенности их реализации. Также описывается YC.SEL.SDK.ReSharper --- ``обёртка'' для YC.SEL.SDK, позволяющая создавать расширения к ReSharper для поддержки встроенных языков. В четвёртой главе описывается метод реинжиниринга встроенного программного кода.  В пятой главе приводятся результаты экспериментального исследования YC.SEL.SDK. Шестая глава содержит результаты сравнения и соотнесения реализованного алгоритма с основными существующими аналогами.


\clearpage
