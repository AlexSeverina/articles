\chapter*{Заключение}                       % Заголовок
\addcontentsline{toc}{chapter}{Заключение}  % Добавляем его в оглавление

\section*{Итоги диссертационной работы}

В качестве итогов работы приведём основные полученные результаты.В ходе выполнения исследования получены следующие основные результаты.

\begin{enumerate}
    \item Разработан алгоритм синтаксического анализа динамически формируемых выражений, позволяющий обрабатывать произвольную регулярную аппроксимацию множества значений выражения в точке выполнения, реализующий 
    эффективное управление стеком и гарантирующий конечность представления леса вывода. Доказана завершаемость и корректность предложенного алгоритма при анализе регулярной аппроксимации, представимой в виде произвольного конечного автомата без $\varepsilon$-переходов.
    \item Создана архитектура инструментария для разработки программных средств статического анализа динамически формируемых строковых выражений.
    \item Разработан метод реинжиниринга встроенного программного кода в проектах по реинжинирингу информационных систем. Данный метод применён в проекте компании ЗАО ``Ланит-Терком'' по переносу информационной системы с MS-SQL Server на Oracle Server, для чего реализованы соответствующие программные компоненты.
\end{enumerate}

Кроме того, реализован инструментальный пакет для разработки средств статического анализа динамически формируемых выражений. На его основе реализован плагин для ReSharper. Код опубликован на сервисе GitHub под лицензией Apache License Version 2.0~\cite{YCUrl}.

\section*{Рекомендации по применению результатов работы}

При применении результатов работы в индустрии и научных исследованиях необходимо учитывать следующие аспекты.

\begin{itemize}
    \item Множество, являющееся аппроксимацией значений динамически формируемого выражения, принимаемое на вход алгоритмом синтаксического анализа, должно быть регулярным.
    \item Эталонный язык должен быть описан детерминированной контекстно-свободной грамматикой.
    \item Так как платформа создавалась с ориентацией на создание инструментов для реинжинирига, то некоторые компоненты направлены на увеличение точности анализа в ущерб производительности. 
    При этом важно, что возможности платформы позволяют комбинировать различные реализации компонент.
    С другой стороны, текущая реализация содержит возможности для различного рода оптимизаций: некоторые алгоритмы могут быть ускорены с помощью распараллеливания. Выбор оптимальных структур данных, например для конечных автоматов, активно использующихся в рамках платформы, является темой отдельного исследования~\cite{DataStructureForFA}.
    \item Теоретическое исследование вопросов, связанных с обработкой SPPF в общем виде, является открытой исследовательской задачей. С другой стороны, было доказано, что из построенного SPPF могут быть извлечены деревья вывода для любых цепочек из аппроксимации, а с деревьями можно работать с помощью стандартных методов. 
\end{itemize}

\section*{Перспективы дальнейшей разработки тематики}

Проблемой, подлежащей дальнейшему исследованию, является возможность выполнения семантических действий непосредственно над SPPF. Это необходимо для рефакторинга, улучшения качества трансляции, автоматизации перехода на более надёжные средства метапрограммирования~\cite{JSStagedMetaProgramming,JSStagedMetaProgrammingFull}.

Важной задачей является теоретическая оценка сложности предложенного алгоритма синтаксического анализа. В известных работах не приводится строгих оценок подобных алгоритмов, поэтому данная задача является самостоятельным исследованием.

С целью обобщения предложенного подхода к синтаксическому анализу, а также для получения лучшей производительности и возможностей для более качественной диагностики ошибок, планируется переход на алгоритм обобщённого LL-анализа (GLL)~\cite{GLL,AbstractGLL}. Планируется исследовать возможность улучшения предложенного алгоритма при переходе на другие алгоритмы обобщённого LR-анализа~\cite{GeneralisedlrBIG}, например, такие как BRNGLR~\cite{BRNGLR} и RIGLR~\cite{RIGLR}.

Кроме того, важной задачей является реализация диагностики ошибок, решение которой для обобщённого восходящего анализа активно исследуется~\cite{RNGLRSyntaxerror1,RNGLRSyntaxerror2, RNGLRSyntaxerror3, RNGLRSyntaxerror4}. Адаптация предложенных решений для применения в представленном алгоритме требует отдельной работы.

Также в дальнейшем планируется развитие платформы и плагина. На уровне платформы необходимо реализовать механизмы, требующиеся для трансформаций кода на встроенных языках. Механизмы трансформации встроенных языков требуются для проведения миграции с одной СУБД на другую~\cite{Syrcose} или для миграции на новые технологии, например, LINQ. Эта задача связана с двумя проблемами: возможностью проведения нетривиальных трансформаций и доказательство корректности трансформаций. Планируется реализация проверки корректности типов. Для SQL это должна быть как проверка типов внутри запроса, так и проверка того, что тип возвращаемого запросом результата соответствует типу хост-переменной, выделенной для сохранения результата в основном коде.

\clearpage
