\section{Introduction}

Complex information systems are often implemented using more than one programming language. 
Sometimes this variety takes form of one \emph{host} and one or few \emph{string-embedded}
languages. Textual representation of clauses in a string-embedded language is built at 
run time by a host program and then analyzed, compiled or interpreted by a dedicated 
runtime component (database, web browser etc.) Most general-purpose programming languages 
may play role of the host; one of the most evident examples of string-embedded language is 
dynamic SQL which was specified in ISO SQL standard in 1992~\cite{ISO} and since then is 
supported by the majority of DBMS. 

%allow to use string-embedded languages: dynamically build 
%expressions in other languages as string values and evaluate them with intended processor 
%(database, web browser, or interpreter). 
%In the current work we use dynamic SQL as an 
%example of string-embedded languages. The mechanism of dynamic SQL queries contraction 
%and execution was specified in ISO SQL standard in 1992~\cite{ISO} and most DBMS
%support this. 

%Dynamically constructed expressions is a string expressions in any programming language 
%and for compiler they are just string constants. 

String-embedded languages may help to compensate the lack of expressivity of general-purpose
language in a domain-specific settings or to integrate heterogeneous components of large system;
however this approach comes with some price. In particular even the syntax analysis of 
string-embedded part of a system is undecidable in general case since its source code
is represented implicitly using string-manipulation primitives, procedures and libraries, and
generated ``on the fly''. In a na\"ive implementation syntax analysis of embedded clauses is
completely outsourced to the runtime environment which postpones many errors from being 
discovered prior to execution and thus compromises the ideas of code safety and static control.

\emph{Abstract parsing} is the approach which was developed to overcome the 
aforementioned defficiency. In abstract parsing the source code of a host application is statically 
analyzed to provide some constructive representation of the set of string-embedded language clauses which
can possible be generated at run time~\cite{AbstrParsing,StringExpr}. This representation is then analyzed 
by a certain parsing algorithm which is usually derived from some existing one for plain strings~\cite{Grune}.
Abstract parsing technique is utilized in a number of tools~\cite{JSA,PHPSA,ALVOR1,ALVOR2} for
program analysis and understanding.

While abstract parsing can help in application analysis it cannot handle the case of application
\emph{transformation}. As a practical use case for string-embedded language transformation we can mention
reengireening. During reengineering it is sometimes necessary to migrate from one database management 
system to another~\cite{NetDbTransform}; this migration may require a transformation of string-embedded 
clauses. 

One of the options is dynamic translation at run time~\cite{OpenSystemsDBMS}. However this solution
not always desirable. First, it may degrade the performance of the system due to introduction of extra 
processing stage. Next, with dynamic translation the ultimate goals of the reengineering are not achieved 
since some part of the original system escaped transformation. 

Another approach includes translation of stored SQL which is supported by a number of existing production tools 
for database application development~\cite{PLSQL,SwissSQL,SQLWays}. However, these tools do not support 
dynamic SQL translation and thus provide only partial solution. 

The contribution of this paper is an approach for \emph{abstract translation}. 
%We present a generalizationof abstract parsing algorithm from~\cite{AbstrParsing} to a translation case. 
Similar to abstract parsing first we perform static analysis to build an approximation for the set of all 
generated clauses. Then our algorithm performs transformation which, if succeeded, delivers new correct 
assignments for all relevant string values in the host program. We discuss some heuristic which helps to 
reduce the complexity of the algorithm in many practical cases and present the results of its 
application for the migration of a real-world industrial project from MS-SQL Server 2005 to Oracle 
11gR2 platform.


%As a result of abstract translation we should calculate new values for all string variables 
%to ensure   correctness of dynamically constructed expressions in the target system. As we 
%have already mentioned, the problem of static string-embedded language translation is not 
%solved yet.

%and application of the abstract parsing 
%algorithm ~\cite{AbstrParsing} to solve the problem of string-embedded language translation. 
%We describe necessary modifications of basic abstract parsing algorithm to make it applicable 
%to string-embedded language translation in the context of real-world information systems 
%migration. 

%There is a set of tools for string-embedded 
%languages processing, such as PHP String Analyzer ~\cite{PHPSA} or Alvor, but these tools 
%also cannot be used for dynamically constructed expression translation.


%Dynamic queries translation is necessary for information systems reengineering and 
%database applications migration from one database management system to another~\cite{NetDbTransform}. 
%Translation of one dialect of SQL to another is often required during this process. If 
%source code contains dynamic queries then we should guarantee that all string values 
%which are used for dynamic queries building remain correct after translation. By 
%correctness of resulting values we mean that queries formed from them are correct 
%in the context of target DBMS.


%If we want to change execution method or 
%runtime of this expression (e.g. change target database from MySQL to Oracle) then we 
%possibly should transform it. For classical case we have a translator from one language 
%to another, and the process called "translation". Now we want to describe similar process 
%for string-embedded languages and we call it "abstract translation".

 
%Various objects modifications may be performed during information system reengineering 
%for example renaming or dead code elimination . In this case we should guarantee not 
%only syntax correctness of translation results but also semantic correctness in the 
%context of performed environment modifications. 

%For practical usage it is very important that information systems can contain a big 
%number of dynamic queries and each one of them can contains hundreds of 
%branches~\cite{TiunovaUIInt}. The necessity of analyzing all possible values for 
%each query can cause  performance problems and exponential growth of required 
%resources.

%The second possible approach is static translation. It means that it can be done statically 
%as part of information system migration process by the similar way as classical translation. 
%So,  we should generalize translation techniques for application to string embedded languages.  
%As a result of abstract translation we should calculate new values for all string variables 
%to ensure   correctness of dynamically constructed expressions in the target system. As we 
%have already mentioned, the problem of static string-embedded language translation is not 
%solved yet.

%The paper is organized as follows: The next Section introduces and describe abstract parsing. 
%Section~\ref{sec:AbstractTranslation} introduces abstract translation and describes basic 
%ideas of string-embedded language translation algorithm. Section~\ref{sec:Optimizations}
%describes optimizations of abstract translation algorithm discussed in the previous Section. 
%Section~\ref{sec:Evaluation} presents results of proposed algorithm application to real-world 
%migration of the information system which contains more than 2 million lines of code and 
%more than 3000 dynamic SQL queries from MS-SQL Server to Oracle and Section~\ref{sec:Conclusion}
%discuss some problems of abstract translation and possible ways to solve them.

