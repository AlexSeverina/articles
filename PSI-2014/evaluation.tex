\section{Evaluation}
\label{sec:Evaluation}

The algorithm of abstract parsing with modifications described was applied in real-world project of 
information system migration from MS-SQL Server 2005 to Oracle 11gR2. The source system consists of 
850 stored procedures and contains more than 3000 dynamic queries. Total size of migrated system is 
2,7 million lines of code. The number of operations to construct values of more than half of queries 
is between 7 and 212 and average is 40.

We have used the next PC configuration for tests: 
\begin{itemize}
    \item 16 GB of RAM
    \item Intel Core i7 2.6 GHz
\end{itemize}

Our implementation of abstract translator is based on FsYacc tool. FsYacc is a generator of LALR(1)
parsers for F\# programming language. Generator was fully reused and custom interpreter with stacks 
copying mechanism was implemented. 

First implementation was directly based on abstract parsing algorithm. This version was not adapted 
to process complex queries and was turn system to active swapping. So, analysis could not be finished
in acceptable time. Timeout (64 seconds) was added to limit one query processing time. Experiments 
show that increasing of timeout do not increase the number of processed queries. The category for 
queries terminated by tymeout was named "Dynamic SQL queries with exponential growth of parsing 
forest".

In the table below we present statistics for dynamic SQL query processing by two algorithms: original
algorithm with timeout and algorithm with states merging.

\begin{center}
\begin{table}
\caption{Comparison of the original algorithm with timeout and the algorithm with states merging.}
\begin{tabular}[c c c]{| p{5.5cm} | p{3cm} | p{3cm} |}
\hline
Category description & Original algorithm with timeout & The algorithm with states merging
\\
\hline
The total number of dynamic SQL queries & 3122 & 3122
\\
\hline
The number of successfully processed dynamic SQL queries & 2181 & 2253
\\
\hline
 & &
\\
\hline
\bfseries{The number of partially processed dynamic SQL queries} & 408 & 522
\\
\hline

 %\ \ \ \ 
 Lexer errors & 283 & 289
\\
\hline

 %\ \ \ \ 
 Parser errors & 354 & 468
\\
\hline
 & &
\\
\hline

\bfseries{The number of not processed dynamic SQL queries} & 533 & 347
\\
\hline
 %\ \ \ \
  Lexer errors & 140 & 134
\\
\hline

 %\ \ \ \
 Parser errors & 280 & 305
\\
\hline

 %\ \ \ \ 
 Dynamic SQL queries with exponential growth of parsing forest. & 253 & 41

\\
\hline
 & &
\\
\hline


Percentage of successfully processed dynamic SQL queries & 69.86\% & 72.17\%
\\
\hline

Percentage of partially processed dynamic SQL queries & 13.07\% & 16.72\%
\\
\hline

Percentage of dynamic SQL queries with non-empty forest & 82.93\% & 88.89\%
\\
\hline
 
\end{tabular}
\end{table}
\end{center}


Partially processed queries is a set of queries with non-empty parsing forest but with parsing or lexing errors. 
This category is the most difficult for processing because error may be a false alarm. Such situation may 
occur if query value which produce errors cannot be generated in run time.

Table shows that the number of queries with exponential growth of parsing forest size decreased
from 253 to 42. So, application of proposed optimizations allows to improve big queries processing 
more than 6 times.
