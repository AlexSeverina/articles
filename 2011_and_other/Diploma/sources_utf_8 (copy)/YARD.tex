\section{Приложения}
\subsection{Внутреннее представления грамматики в инструменте YARD}
Ниже приведено описание структуры данных, соответствующее внутренниму представлению грамматики в инструменте YARD, на языке F\# .

\begin{verbatim}
module Definition = 
  struct
    type ('patt,'expr) t = 
    { 
     head    :'expr option; 
     grammar : Grammar.t<'patt,'expr>;
     foot    :'expr option 
    }
end
\end{verbatim}

Модуль \verb Definition \ соответствует фвйлу с описанием грамматики. Элементы \verb head \ и \verb foot \ соответствуют длополнительным описаниям в начале и в конце файла соответсвенно. Например, подключение модулей, какие-либо дополнительные действия, которые должны быть включены в целевой код "`как есть"'.

\begin{verbatim}
module Grammar = 
  struct
    type t <'patt,'expr> = (Rule.t<'patt,'expr>) list 
  end 
\end{verbatim}

Модуль \verb Grammar \ -- представление грамматики. Грамматика это список правил.

\begin{verbatim}
module Rule = 
  struct
    type t <'patt,'expr> = 
    { 
     name    : string;
     args    : 'patt list;
     body    : (Production.t <'patt,'expr>);
     _public : bool; 
     metaArgs: 'patt list
    }
  end
\end{verbatim}

Модуль \verb Rule \ -- представление правила грамматики. Правило имеет имя. Оно может применяться к аргуметтам или мета аргументам (\verb adgs \ и \verb metaArgs \ соответственно). Более подробно об этом можно прочеть в работе \cite{Diploma}. Элемент \verb _pablic \ указывает, является ли правило стартовым. В общем случае таких правил в грамматике может быть несколько. Элемент \verb body \ -- правая часть правила( продукция ).

\begin{verbatim}
module Production = 
  struct
    type elem <'patt,'expr> = 
    {
     omit:bool; 
     rule:(t<'patt,'expr>);
     binding:'patt option; 
     checker:'expr option 
    }   
    and
    t <'patt,'expr> =    
    |PAlt     of (t <'patt,'expr>) * (t<'patt,'expr>)
    |PSeq     of (elem <'patt,'expr>) list * 'expr option 
    |PToken   of Source.t 
    |PRef     of Source.t * 'expr option // Vanilla rule reference with an optional args list.
    |PMany    of (t <'patt,'expr>) //expr*
    |PMetaRef of Source.t * 'expr option * 'expr list // Metarule reference like in "a: mr<x> y z"
    |PLiteral of Source.t 
    |PRepet   of (t <'patt,'expr>) * int option * int option  //extended regexp repetition, "man egrep" for details
    |PPerm    of (t <'patt,'expr>) list //permutation (A || B || C)
    |PSome    of (t <'patt,'expr>) //expr+
    |POpt     of (t <'patt,'expr>) //expr?

  end
\end{verbatim}

Модуль \verb Production -- продукция. %% похоже за более подробным описанием - к Якову Александровичу...

\begin{verbatim}
module Source = 
  struct
   type t = string * (int * int) 
   
   let toString ((r,_):t):string = r

  end
\end{verbatim}