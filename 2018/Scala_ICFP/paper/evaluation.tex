\section{Evaluation}
\label{sec:evaluation}

In this section, we present an evaluation of our graph querying library.
We measure its performance on a classical set of ontology graphs~\cite{CFGonRDF}: both when the graph is loaded into the RAM and for the integration with the Neo4j database and compare it with the solution based on the GLL parsing algorithm~\cite{GrigorevR16} and the Trails~\cite{ScalaGraphParsing} library for graph traversals. We also show how may-alias static code analysis can be done by means of the library.

All tests have been performed on a computer running Fedora 27 with quad-core Intel Core i7 2.5 GHz CPU and 8 GB of RAM.

\begin{table*}[t]
\small
\centering
\caption{Comparison of Meerkat, Trails and GLL performance on ontologies}
\label{table:rdfs}
\begin{tabular}{|l|ll|lllll|lllll|}
\hline
\multirow{2}{*}{Ontology} &
\multirow{2}{*}{\#nodes} &
\multirow{2}{*}{\#edges} &

\multicolumn{5}{c|}{Query 1}  &
\multicolumn{5}{c|}{Query 2}\\
\cline{4-13}
&
&
&
{\begin{tabular}[c]{@{}c@{}}\#results\end{tabular}} &
{\begin{tabular}[c]{@{}c@{}}In memory \\ graph\\ (ms)\end{tabular}} &
{\begin{tabular}[c]{@{}c@{}}DB \\ query \\ (ms)\end{tabular}} &
{\begin{tabular}[c]{@{}c@{}}Trails\\ (ms)\end{tabular}} &
{\begin{tabular}[c]{@{}c@{}}GLL \\ (ms)\end{tabular}} &
{\begin{tabular}[c]{@{}c@{}}\#results\end{tabular}} &
{\begin{tabular}[c]{@{}c@{}}In memory \\ graph\\ (ms)\end{tabular}} &
{\begin{tabular}[c]{@{}c@{}}DB \\ query \\ (ms)\end{tabular}} &
{\begin{tabular}[c]{@{}c@{}}Trails \\ (ms)\end{tabular}} &
{\begin{tabular}[c]{@{}c@{}}GLL \\ (ms)\end{tabular}} \\



\hline\hline
atom-primitive              & 291 & 685  & 15454 & 64  & 67 & 2849 & 232 & 122  &  29 &  79 & 453 & 19 \\
\shortstack[l]{biomedical- \\
 measure-primitive}         & 341 & 711  & 15156 & 112 & 108 & 3715 & 482 & 2871 &  18 & 18 & 60  & 26 \\
foaf                        & 256 & 815  & 4118  & 11  & 11  & 432  & 29  & 10   &  1  & 1  & 1   & 1 \\
funding                     & 778 & 1480 & 17634 & 69  & 68  & 367  & 179 & 1158 &  9  & 9   & 76  & 13 \\
generations                 & 129 & 351  & 2164  & 4   & 4   & 9    & 12  & 0    &  1  & 0  & 0   & 0 \\
people\_pets                & 337 & 834  & 9472  & 37  & 37  & 75   & 80  & 37   &  1  & 1  & 2   & 1 \\
pizza                       & 671 & 2604 & 56195 & 333 & 325 & 7764 & 793 & 1262 &  17 & 18 & 905 & 50 \\
skos                        & 144 & 323  & 810   & 1   & 2   & 6    & 6   & 1    &  1  & 0  & 0   & 0 \\
travel                      & 131 & 397  & 2499  & 11  & 13  & 34   & 21  & 63   &  1  & 1  & 1   & 2 \\
univ-bench                  & 179 & 413  & 2540  & 10  & 10  & 31   & 24  & 81   &  1  & 1  & 2   & 1 \\
wine                        & 733 & 2450 & 66572 & 405 & 401 & 3156 & 606 & 133  &  2  & 3  & 4   & 5 \\
\hline
\end{tabular}
\end{table*}


\subsection{Ontology Querying}
\label{sec:ontology}

Querying for ontologies is a well-known graph querying problem. We evaluate our library on some popular ontologies in the form of RDF files from the paper~\cite{CFGonRDF}.
First, we convert RDF files to a labelled directed graph in the following manner: we create two edges $(subject,\ predicate,\ object)$ and $(object,\ predicate^{-1},\ subject)$ for every RDF triple $(subject,\ predicate,\ object)$. Then the graph is either loaded into the Neo4j database or is loaded directly into the memory.
Then we run two queries from the paper~\cite{GrigorevR16} for these graphs. The grammars for the queries are presented in Fig.~\ref{fig:queryGen}.

The performance results are shown in Table~\ref{table:rdfs} where $\#results$ is the number of pairs of the nodes between which exists at least one S-path.

The Meerkat-based and the GLL-based~\cite{GrigorevR16} solutions show the same results (column $\#results$) and the Queries 1 and 2 run at least $1.5$ times faster on the Meerkat-based solution than on the GLL-based. Querying the Neo4j database is as performant as the querying the graphs located in the RAM.


\subsection{Static Code Analysis}

Alias analysis is a fundamental static analysis problem~\cite{Marlowe}: it checks may-alias relations between code expressions and can be formulated as a context-free language (CFL) reachability problem~\cite{Reps}, which is closely related to the context-free path querying problem.
In this analysis, a program is represented as a Program Expression Graph (PEG)~\cite{Zheng}.
Vertices in a PEG correspond to program expressions and edges are relations between them.
There are two types of edges possible while analyzing source code written in the C programming language: \textbf{D}-edge and \textbf{A}-edge.

\begin{itemize}
    \item Pointer dereference edge (\textbf{D}-edge). For each pointer dereference $*e$, there is a directed \textbf{D}-edge from $e$ to $*e$.
    \item Pointer assignment edge (\textbf{A}-edge). For each assignment $*e_1=e_2$, there is a directed \textbf{A}-edge from $e_2$ to $*e_1$
\end{itemize}

For the sake of simplicity, we add edges labelled by $\overline{D}$ and $\overline{A}$ which correspond to the reversed \textbf{D}-edge and \textbf{A}-edge, respectively.

The grammar for the may-alias problem from~\cite{Zheng} and its implementation in Meerkat are shown in Fig.~\ref{fig:aliasMeerkat}.
There are two nonterminals \textbf{M} and \textbf{V}, which we can query for.
\textbf{M}~production means that two l-value expressions are memory aliases, i.e. may refer to the same memory location.
\textbf{V}~production means that two expressions are value aliases, i.e. may evaluate to the same pointer value.

\begin{figure}[h]
\begin{align*}
& M \rightarrow\ \overline{D}\ V\ D\\
& V \rightarrow (M ?\ \overline{A})^{*} \ M?\ (A\ M?)^{*}
\end{align*}

\begin{lstlisting}
val M  = syn("nd" ~ V ~ "d")
val V  = syn((M.? ~ "na").* ~ M.? ~ ("a" ~ M.?).*)
\end{lstlisting}
\caption{Context-Free grammar and its Meerkat representation for the may-alias problem}
\label{fig:aliasMeerkat}
\end{figure}

We run the \textbf{M} and \textbf{V} queries on some open-source C projects.
We used Crystal\footnote{Crystal: a program analysis system for C~\url{https://www.cs.cornell.edu/projects/crystal/}. Access date: 30.07.2018.} to construct Program Expression Graph and ran the queries over it.
The results are shown in Table~\ref{table:staticAnalysis}.
We conclude that our solution is expressive and performant enough to be used for static analysis problems, which can be expressed as CFPQs.


\begin{table*}[t]
\small
\centering
\caption{Running may-alias queries on Meerkat on some C open-source projects}
\label{table:staticAnalysis}
\begin{tabular}{|l|lll|lll|ll|}
\hline
{Program} &
{\#edges} &
{\#nodes} &
{Code Size (KLOC)} &

{\begin{tabular}[c]{@{}c@{}}In memory \\ graph (ms)\end{tabular}} &
%\multicolumn{1}{c|}{In memory graph (ms)} &
\multicolumn{1}{c|}{Neo4j graph (ms)} &
\multicolumn{1}{c|}{Trails graph (ms)} &
\multicolumn{1}{c|}{M aliases} &
\multicolumn{1}{c|}{V aliases}  \\
\hline
\hline
wc-5.0      & 332  & 770  & 0.5 & 0   & 2   & 3   & 174  & 107 \\
pr-5.0      & 815  & 2062 & 1.7 & 11  & 12  & 14  & 1131 & 63  \\
ls-5.0      & 1687 & 4734 & 2.8 & 43  & 51  & 170 & 5682 & 253 \\
bzip2-1.0.6 & 632  & 1508 & 5.8 & 8   & 13  & 21  & 813  & 71  \\
gzip-1.8    & 2687 & 7510 & 31  & 111 & 120 & 537 & 4567 & 227 \\
\hline
\end{tabular}
\end{table*}

\begin{table*}[t]
\small
\centering
\caption{Running regular queries using Meerkat and Cypher}
\label{table:movies}
\begin{tabular}{|l|c|c|}
\hline
{Query} &
{Neo4j + Meerkat (ms)} &
{Neo4j + Cypher (ms)} \\
\hline
\hline
Actors who played in some film & 89.78 & 11.31 \\
Most prolific actors & 2333.37 & 189.32 \\
Mutual friend recommendations & 106.86 & 17.26 \\
Directed more than 2 films, acted in more than 10 & 348.11 & 44.82 \\
\hline
\end{tabular}
\end{table*}

\subsection{Classical Movies Queries}

To examine the applicability of our library for the real data processing, we evaluate queries presented in section~\ref{sec:examples} on the classical movie database which contains more than 60000 nodes and more than 100000 edges.

All queries are performed using the Neo4j database connected to Meerkat.
During the evaluation of these queries, caching in Neo4j was disabled to simplify time measurement.
The results are presented in Table~\ref{table:movies}.

Our library implements a general parsing algorithm which supports arbitrary context-free queries.
Unfortunately, when the queries are regular, there is a significant overhead for processing them.
The queries mentioned in this section are all regular, thus it is not a surprise that our implementation is dramatically slower than the native Neo4j solution. Besides the overhead itself, our library does not take any advantages from the internal Neo4j optimizations.

Despite the seeming failure of this test, we still can conclude that the combinators can process quite big datasets in a reasonable time: 6 times bigger in terms of the number of vertices than the other tests.
It also shows that further optimization of the implementation is justified.

\subsection{Comparison with Trails}

Trails~\cite{ScalaGraphParsing} is a Scala graph combinator library.
It provides traversers for describing paths in graphs which resemble parser combinators and calculates possibly infinite stream of all possible paths described by the composition of the basic traversers.
Both Trails and Meerkat-based solution support parsing of the graphs located in RAM, so we compare the performance of the Trails and the Meerkat-based solution on the ontology queries described above: the results are presented in Table~\ref{table:rdfs}.
Trails and Meerkat-based solution compute the same results, but Trails is up to 10 times slower than the Meerkat-based solution.

To summarize, we demonstrated that parser combinators suit for implementing real queries.
Our implementation is as performant as the other existing combinators library and is comparable to the GLL-based solution.
