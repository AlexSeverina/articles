\section{Context-Free Path Querying Problem}
\label{sec:CFPQ}

In this section, we formally describe the context-free path querying problem and the context-free reachability problem.

First, we introduce the necessary definitions.
\begin{itemize}
  \item A context-free grammar is a quadruple $G=(N, \Sigma, P, S)$, where $N$ is a set of nonterminal symbols, $\Sigma$ is a set of terminal symbols, $S \in N$ is a start nonterminal, and $P$ is a set of productions.
  \item $\mathcal{L}(G)$ denotes a language specified by the grammar $G$, and is a set of terminal strings derived from the start nonterminal of $G$: $\mathcal{L}(G) = \{\omega \mid S \Rightarrow_{G}^{*} \omega\}$.
  \item A directed graph is a triple $M = (V,E,L)$, where $V$ is a set of vertices, $L \subseteq \Sigma$ is a set of labels, and a set of edges $E\subseteq V\times L\times V$.
  Note that there are not parallel edges with equal labels: if $(v, l_1, u) \in E$ and $(v,l_2,u) \in E$, then $l_1 \neq l_2$.
  \item $tag: E \rightarrow L$ is a function which returns a tag of an edge. $$tag((\_,l,\_)) = l$$
  \item $\oplus: L^+ \times L^+ \rightarrow L^+$ denotes a tag concatenation operation.
  \item $\Omega$ is a helper function which constructs a string produced by the given path. For every $p \text{ path in } M$
  $$ \Omega(p = e_{0},e_{1},\dots,e_{n-1}) = tag (e_{0}) \oplus \dots \oplus tag (e_{n-1}).$$
\end{itemize}

We define the context-free language constrained path querying as, given a query in the form of a grammar $G$, to construct the set of the paths from the input graph $M$ which are also strings derivable in the grammar $G$: $$\{p \mid p \text{ is a path in } M, \Omega(p) \in \mathcal{L}(G)\}.$$
The CFL reachability problem is pretty similar, but here one is only concerned with the existence of the paths derived by the grammar. It is formulated as follows: $$\{ (v_0,v_n) \mid p \text{ is a path in } M, p = v_0 \rightarrow \cdots \rightarrow v_n, \Omega(p) \in \mathcal{L}(G)\}.$$

Note that the query result can be an infinite set, hence it cannot be represented explicitly.
We show how to construct a compact data structure which stores all the elements of the query result in a finite space; every path can be extracted from this representation.