\appendix
%Appendix A
\section{Equivalence of transitive closure definitions}\label{def_eq}

To show the equivalence of $a^{cf}$ and $a^+$ definitions of transitive closure, we introduce the partial order $\succeq$ on matrices with a fixed size that have subsets of $N$ as elements. For square matrices $a, b$ of the same size we denote $a \succeq b$ iff $a_{i,j} \supseteq b_{i,j}$, for every $i, j$. For these two definitions of transitive closure, the following lemmas and theorem hold.

\begin{lemma}\label{lemma:cf_geq_valiant}
	Let $G =(N,\Sigma,P,S)$ be a context-free grammar in Chomsky Normal Form, let $a$ be a square matrix. Then $a^{(k)} \succeq a^{(k)}_+$ for any $k \geq 1$.
\end{lemma}
\begin{proof}(Proof by Induction)
	
	\textbf{Base case}: The statement of the lemma holds for $k = 1$, since $$a^{(1)} = a^{(1)}_+ = a.$$
	
	\textbf{Inductive step}: Assume that the statement of the lemma holds for any $k \leq (p - 1)$ and show that it also holds for $k = p$ where $p \geq 2$. For any $i \geq 2$ $$a^{(i)} = a^{(i-1)} \cup (a^{(i-1)} \times a^{(i-1)}) \Rightarrow a^{(i)} \succeq a^{(i-1)}.$$ Hence, by the inductive hypothesis, for any $i \leq (p-1)$ $$a^{(p-1)} \succeq a^{(i)} \succeq a^{(i)}_+.$$ Let $1 \leq j \leq (p - 1)$. The following holds $$(a^{(p-1)} \times a^{(p-1)}) \succeq (a^{(j)}_+ \times a^{(p-j)}_+),$$ since $a^{(p-1)} \succeq a^{(j)}_+$ and $a^{(p-1)} \succeq a^{(p-j)}_+$. By the definition, $$a^{(p)}_+ = \bigcup^{p-1}_{j=1}{a^{(j)}_+ \times a^{(p-j)}_+}$$ and from this it follows that $$(a^{(p-1)} \times a^{(p-1)}) \succeq a^{(p)}_+.$$ By the definition, $$a^{(p)} = a^{(p-1)} \cup (a^{(p-1)} \times a^{(p-1)}) \Rightarrow a^{(p)} \succeq (a^{(p-1)} \times a^{(p-1)}) \succeq a^{(p)}_+$$ and this completes the proof of the lemma.
\end{proof}

\begin{lemma}\label{lemma:valiant_geq_cf}
	Let $G =(N,\Sigma,P, S)$ be a context-free grammar in Chomsky Normal Form, let $a$ be a square matrix. Then for any $k \geq 1$ there is $j \geq 1$, such that $(\bigcup^{j}_{i=1}{a^{(i)}_+}) \succeq a^{(k)}$.
\end{lemma}
\begin{proof}(Proof by Induction)
	
	\textbf{Base case}: For $k = 1$ there is $j = 1$, such that $$a^{(1)}_+ = a^{(1)} = a.$$ Thus, the statement of the lemma holds for $k = 1$.
	
	\textbf{Inductive step}: Assume that the statement of the lemma holds for any $k \leq (p - 1)$ and show that it also holds for $k = p$ where $p \geq 2$. By the inductive hypothesis, there is $j \geq 1$, such that $$(\bigcup^{j}_{i=1}{a^{(i)}_+}) \succeq a^{(p-1)}.$$ By the definition, $$a^{(2j)}_+ = \bigcup^{2j-1}_{i=1}{a^{(i)}_+ \times a^{(2j-i)}_+}$$ and from this it follows that $$(\bigcup^{2j}_{i=1}{a^{(i)}_+}) \succeq (\bigcup^{j}_{i=1}{a^{(i)}_+}) \times (\bigcup^{j}_{i=1}{a^{(i)}_+}) \succeq (a^{(p-1)} \times a^{(p-1)}).$$ The following holds $$(\bigcup^{2j}_{i=1}{a^{(i)}_+}) \succeq a^{(p)} = a^{(p-1)} \cup (a^{(p-1)} \times a^{(p-1)}),$$ since $$(\bigcup^{2j}_{i=1}{a^{(i)}_+}) \succeq (\bigcup^{j}_{i=1}{a^{(i)}_+}) \succeq a^{(p-1)}$$ and $$(\bigcup^{2j}_{i=1}{a^{(i)}_+}) \succeq (a^{(p-1)} \times a^{(p-1)}).$$ Therefore there is $2j$, such that $$(\bigcup^{2j}_{i=1}{a^{(i)}_+}) \succeq a^{(p)}$$ and this completes the proof of the lemma.	
\end{proof}

\begin{mytheorem}\label{thm:closures}
	Let $G =(N,\Sigma,P,S)$ be a context-free grammar in Chomsky Normal Form, let $a$ be a square matrix. Then $a^+ = a^{cf}$.
\end{mytheorem}
\begin{proof}
	
	By the lemma~\ref{lemma:cf_geq_valiant}, for any $k \geq 1$, $a^{(k)} \succeq a^{(k)}_+$. Therefore $$a^{cf} = a^{(1)} \cup a^{(2)} \cup \cdots \succeq a^{(1)}_+ \cup a^{(2)}_+ \cup \cdots = a^+.$$ By the lemma~\ref{lemma:valiant_geq_cf}, for any $k \geq 1$ there is $j \geq 1$, such that $$(\bigcup^{j}_{i=1}{a^{(i)}_+}) \succeq a^{(k)}.$$ Hence $$a^+ = (\bigcup^{\infty}_{i=1}{a^{(i)}_+}) \succeq a^{(k)},$$ for any $k \geq 1$. Therefore $$a^+ \succeq a^{(1)} \cup a^{(2)} \cup \cdots = a^{cf}.$$ Since $a^{cf} \succeq a^+$ and $a^+ \succeq a^{cf}$, $$a^+ = a^{cf}$$ and this completes the proof of the theorem.
\end{proof}