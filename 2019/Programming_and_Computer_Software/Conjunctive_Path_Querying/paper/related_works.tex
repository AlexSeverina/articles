\section{Related works} \label{section_related}
The regular language constrained path querying is widely used for graph analysis~\cite{abiteboul1997regular,fan2011adding,nole2016regular,reutter2017regular}.

There are a number of solutions~\cite{azimov2018context,hellingsRelational,GraphQueryWithEarley,RDF} for context-free path query evaluation w.r.t. the relational query semantics, which employ such parsing algorithms as CYK~\cite{kasami,younger} or Earley~\cite{Grune}. Other examples of path query semantics are \textit{single-path} and \textit{all-path query semantics}. The all-path query semantics requires presenting all possible paths from node $m$ to node $n$ whose labeling is derived from a nonterminal $A$ for all triples $(A, m, n)$ evaluated using the relational query semantics. While the single-path query semantics requires presenting only one such a path for all the node-pairs $(m, n)$. Hellings~\cite{hellingsPathQuerying} presented algorithms for the context-free path query evaluation using the single-path and the all-path query semantics. If a context-free path query w.r.t. the all-path query semantics is evaluated on cyclic graphs, then the query result can be an infinite set of paths. For this reason, in~\cite{hellingsPathQuerying}, annotated grammars are proposed as a possible solution.

In~\cite{GLL}, the algorithm for context-free path query evaluation w.r.t. the all-path query semantics is proposed. This algorithm is based on the generalized top-down parsing algorithm~---~GLL~\cite{scott2010gll}. This solution uses derivation trees for the result representation which is more native for grammar-based analysis. The algorithms in~\cite{GLL,hellingsPathQuerying} for the context-free path query evaluation w.r.t. the all-path query semantics can also be used for query evaluation using the relational and the single-path semantics.

Hellings~\cite{hellingsRelational} presented an algorithm for the context-free path query evaluation using the relational query semantics. According to Hellings, for a given graph $D = (V, E)$ and a grammar $G = (N, \Sigma, P)$ the context-free path query evaluation w.r.t. the relational query semantics reduces to a calculation of the context-free relations $R_A$. Thus, in this work, we focus on the calculation of conjunctive relations which are similar to the context-free relations.

There is an algorithm~\cite{zhang2017context} for path querying with linear conjunctive grammars and relational query semantics. These grammars have no more than one nonterminal in each conjunct of the rule. The possibility of creating an algorithm for path query evaluation w.r.t. arbitrary conjunctive grammars is an open problem since the linear conjunctive grammars are known to be strictly less powerful than the arbitrary conjunctive grammars~\cite{okhotinConjAndBool}.