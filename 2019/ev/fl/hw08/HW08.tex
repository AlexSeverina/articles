\documentclass[12pt]{article}
\usepackage[left=2cm,right=2cm,top=2cm,bottom=2cm,bindingoffset=0cm]{geometry}
\usepackage[utf8x]{inputenc}
\usepackage[english,russian]{babel}
\usepackage{cmap}
\usepackage{amssymb}
\usepackage{amsmath}
\usepackage{url}
\usepackage{hyperref}
\usepackage{pifont}
\usepackage{tikz}
\usepackage{verbatim}

\usetikzlibrary{shapes,arrows}
\usetikzlibrary{positioning,automata}
\tikzset{
  every state/.style={minimum size=0.2cm},
  initial text={}
}

\newenvironment{myauto}[1][3]
{
  \begin{center}
    \begin{tikzpicture}[> = stealth,node distance=#1cm, on grid, very thick]
}
{
    \end{tikzpicture}
  \end{center}
}


\begin{document}
\begin{center} {\LARGE Формальные языки} \end{center}

\begin{center} \Large домашнее задание до 23:59 18.12 \end{center}
\bigskip

\begin{enumerate}
  \item Реализовать функцию \verb!star!, которая создает парсер, анализирующий тот же язык, что 0-или-больше повторений входного парсера.
  \item Реализовать функцию \verb!not!, которая создает парсер, завершающийся неудачей, если входной парсер принимающий префикс удачно, и наоборот. 
  \item Реализовать PEG, принимающий арифметические выражения над целыми числами с операциями сложения, умножения, вычитания и деления с естественными приоритетами и ассоциативностью. 
\end{enumerate}
\end{document}
