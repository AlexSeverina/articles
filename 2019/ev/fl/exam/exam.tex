\documentclass[12pt]{article}
\usepackage[left=1cm,right=2cm,top=1.5cm,bottom=1.5cm,bindingoffset=0cm]{geometry}
\usepackage{hyperref}
\usepackage{fontspec}
\usepackage{polyglossia}
\setdefaultlanguage{russian}
\setmainfont[Mapping=tex-text]{CMU Serif}
\pagestyle{empty}

\newcommand{\litem}[1]{\item #1 }% \begin{center} \hline \end{center}}

\begin{document}

\begin{center} 
\LARGE Формальные языки

\Large вопросы к экзамену
\end{center}

\begin{enumerate}
  \litem {Контекст, в котором возникают формальные языки. Метаязык описания языков. Алфавит, цепочка (строка), операции над цепочками (конкатенация, обращение, степень, длина), их свойства.} 
  \litem {Форма Бэкуса-Наура, расширенная форма Бэкуса-Наура, синтаксические диаграммы Вирта как примеры метаязыков. Примеры описания языков с использованием метаязыков. }
  \litem {Формальный язык, формальная грамматика (по Хомскому), примеры. Непосрественная выводимость, выводимость, порождаемый грамматикой язык. Эквивалентность грамматик. }
  \litem {Контекстно-свободные грамматики и деревья вывода, примеры. Теорема о соотношении вывода и дерева вывода, доказательство.}
  \litem {Конечные автоматы, полные конечные автоматы, путь в конечном автомате, такт работы КА, распознавание слова КА, язык, распознаваемый КА, примеры. }
  \litem {Эквивалентность КА, проверка на эквивалентность. Минимальность КА, алгоритм минимизации путем выделения классов эквивалентности, корректность алгоритма, сложность, пример использования. }
  \litem {Недетерминированные КА, их соотношение с детерминированными. Распознавание слова НКА, алгоритм, проверяющий допустимость слова НКА. Детерминизация: алгоритм Томпсона. Эквивалентность ДКА и НКА.}
  \litem {Произведение автоматов, пример. Нахождение пересечения, объединения, разности регулярных языков при помощи произведения автоматов. Замкнутость автоматных языков относительно теоретико-множественных операций. }
  \litem {Регулярные множества (языки), примеры регулярных языков, академические регулярные выражения, примеры. Замкнутость регулярных языков относительно различных операций. Свойства регулярных выражений. }
  \litem {Теорема Клини. НКА с $\varepsilon$-переходами, эквивалентность НКА без $\varepsilon$-переходов, $\varepsilon$-замыкание. Доказательство теоремы Клини (в обе стороны). Примеры построения НКА по регулярному выражению и регулярного выражения по НКА.}
  \litem {Праволинейные/леволинейные грамматики, регулярные грамматики, эквивалентность регулярных грамматик и НКА. Лемма о накачке для регулярных языков, доказательство, применение. }
  \litem {Контекстно-свободные грамматики. Вывод в КС-грамматике, пример. Теорема о существовании левостороннего вывода. Однозначность и неоднозначность грамматик. Неразрешимость проверки однозначности грамматики. }
  \litem {Контекстно-свободные языки, существенная неоднозначность. Проверка пустоты порождаемого языка, доказательство. Удаление непродуктивных нетерминалов грамматики, приведение грамматики, удаление цепных правил. }
  \litem {Нормальная форма Хомского, алгоритм приведения к НФХ, пример, важность порядка операций при приведении к НФХ, разрастание грамматики при нормализации. CYK-алгоритм, пример, сложность работы. }
  \litem {Восходящий и нисходящий синтаксический анализ. Функции FIRST, FOLLOW. LL-грамматики, Фундаментальное свойство LL-грамматик. Пример LL(k) грамматики, простая LL(1) грамматика. LL(k)-грамматика: необходимое и достаточное условие. LL(1)-грамматика: необходимое и достаточное условие. LL-грамматики и левая рекурсия.}
  \litem {Типы нисходящих синтаксических анализаторов. нисходящий синтаксический анализ с откатом, пример, сложность. Нисходящий синтаксический анализ без отката. Рекурсивный спуск, пример. }
  \litem {LL(k) анализаторы. Избавление от левой рекурсии (явной, неявной, взаимной) в грамматиках. Левая факторизация грамматики, пример. Вычисление множеств FIRST и FOLLOW с примерами. LL(1) анализ, построение таблиц анализатора, сложность, пример, ограничения LL-анализаторов. }
  \litem {Восходящий синтаксический анализ. Алгоритмы LR(0), SLR(1), CLR(1), построение таблиц, принцип работы, различия, сложность работы, ограничения алгоритмов, примеры.}
  \litem {Dangling else problem, Parsing expression grammar, различия между PEG и КС грамматиками. Преимущества и недостатки PEG, примеры. }
  \litem {Магазинный автомат: неформальное понимание. Детерминированные и недетерминированные магазинные автоматы. Отношение переходов, семантика магазинного автомата, 2 варианта принятия слова: по достижении конечного состояния, по опустошению стека. Пример. Построение МА по КС грамматике. Лемма о накачке для КС языка, пример использования. }
  \litem {Синтаксически управляемая трансляция, схемы синтаксически управляемой трансляции, выводимость в схеме, пример. Обобщенные схемы синтаксически управляемой трансляции, вывод, пример. }
  \litem {Транслирующие грамматики, пример. Постфиксная транслирующая грамматика. Атрибутная транслирующая грамматика. Понятие атрибута, типы атрибутов, S-атрибутные и L-атрибутные грамматики, примеры. }
  \litem {Магазинные преобразователи, отношение переходов, семантика. Детерминированные МП, пример МП, взаимоотношения между СУ-схемами и МП.}
  \litem {Контекстно-зависимые и неукорачивающие грамматики, их эквивалентность. КЗ языки, линейно-ограниченные автоматы, их эквивалентность, примеры. Рекурсивность КЗ грамматик. }
  \litem {Иерархия Хомского: 4 типа грамматик, 4 типа языков и соответствующих распознавателей, их соотношение.}
\end{enumerate}

\newpage

\begin{center}
{\LARGE Расстрельный список определений}

{\emph{ Знание всех определений --- необходимое условие положительной оценки за экзамен }}
\end{center}


\bigskip
\begin{enumerate}
  \litem {Множество, подмножество, множество всех подмножеств, операции над множествами.}
  \litem {Алфавит, цепочка (строка), формальный язык} 
  \litem {Формальная грамматика, порождаемый грамматикой язык.}
  \litem {Контекстно-свободные грамматики и деревья вывода.}
  \litem {Конечные автоматы; язык, распознаваемый КА. }
  \litem {Регулярные множества (языки). }
  \litem {Теорема Клини.}
  \litem {Регулярные грамматики.}
  \litem {Контекстно-свободные грамматики, вывод в КС-грамматике.}
  \litem {Контекстно-свободные языки.}
  \litem {Нормальная форма Хомского.}
  \litem {Магазинный автомат.}
  \litem {Контекстно-зависимые и неукорачивающие грамматики, КЗ языки.}
  \litem {Иерархия Хомского.}
\end{enumerate}


\end{document}
