\documentclass{beamer}
\usepackage{beamerthemesplit}
\usepackage{wrapfig}
\usetheme{SPbGU}
\usepackage{pdfpages}
\usepackage{amsmath}
\usepackage{mathtools}
\usepackage{cmap} 
\usepackage[T2A]{fontenc} 
\usepackage[utf8]{inputenc}
\usepackage[english,russian]{babel}
\usepackage{indentfirst}
\usepackage{amsmath}
\usepackage{tikz}
\usepackage{multirow}
\usepackage[noend]{algpseudocode}
\usepackage{algorithm}
\usepackage{algorithmicx}
\usepackage{ stmaryrd }
\usepackage{qtree}
\usetikzlibrary{shapes,arrows}
\usepackage{fancyvrb}

\usetikzlibrary{positioning}
\usetikzlibrary{decorations.pathmorphing}

\usetikzlibrary{shapes,arrows}
\usetikzlibrary{positioning,automata}
\tikzset{
  snake it/.style={decorate, decoration=snake},
  every state/.style={minimum size=0.2cm},
  initial text={}
}

\newtheorem{rutheorem}{Теорема}
\newtheorem{ruproof}{Доказательство}
\newtheorem{rudefinition}{Определение}
\newtheorem{rulemma}{Лемма}
\beamertemplatenavigationsymbolsempty

\setbeamertemplate{itemize item}[circle]
\setbeamertemplate{enumerate item}[circle]

\newcommand{\derives}[1][*]{\xRightarrow[]{#1}}
\newcommand{\deriveg}[1]{\xRightarrow[#1]{*}}
\newcommand{\derivegone}[1]{\xRightarrow[#1]{}}

\title[]{Теория автоматов и формальных языков}
\subtitle[]{Контекстно-свободные языки}
\institute[]{
Санкт-Петербургский государственный электротехнический университет <<ЛЭТИ>>\\
}

\author[]{Екатерина Вербицкая}

\date{23 октября 2019}

\definecolor{orange}{RGB}{179,36,31}

\begin{document}
{
  \begin{frame}
    \titlepage
  \end{frame}
}


\begin{frame}[fragile]
  \transwipe[direction=90]
  \frametitle{В предыдущей серии}
  \begin{itemize}
    \item Регулярные выражения, регулярные грамматики и конечные автоматы задают класс регулярных языков
    \item Класс регулярных языков замкнут относительно теоретико-множественных операций, конкатенации, итерации, гомоморфизма цепочек
    \item Определение принадлежности слова языку осуществляется за $O(n)$ операций
    \item Однако класс регулярных языков достаточно узок, ни один используемый в промышленности язык программирования не является регулярным
    \begin{itemize}
      \item Лемма о накачке для доказательства нерегулярности языка
      \item Язык правильных скобочных последовательностей, язык палиндромов не являются регулярными
    \end{itemize}
  \end{itemize}
\end{frame}

\begin{frame}[fragile]
  \transwipe[direction=90]
  \frametitle{Контекстно-свободная грамматика}
  
  \begin{center}
    Четверка $\langle V_T, V_N, P, S \rangle$
  \end{center}

   \begin{itemize}
     \item $V_T$~--- алфавит терминальных символов (терминалов) 
     \item $V_N$~--- алфавит нетерминальных символов (нетерминалов)
     \begin{itemize} 
        \item $V_T \cap V_N = \varnothing$ 
        \item $V ::= V_T \cup V_N$
     \end{itemize}
     \item P~--- конечное множество правил вида $A \to \alpha$
     \begin{itemize}
       \item $A \in V_N $
       \item $\alpha \in V^*$
     \end{itemize}  
     \item S~--- начальный нетерминал грамматики, $S  \in V_N$
  \end{itemize}

\vfill

\begin{center}
  Пример: арифметические выражения
\end{center}
\begin{align*}
  E &\to E + E \mid E * E \mid N \\ 
  N &\to 0 \mid 1 \mid \dots \mid 9 
\end{align*}
\end{frame}

\begin{frame}[fragile]
  \transwipe[direction=90]
  \frametitle{Вывод в грамматике}
  \begin{itemize}
    \item \textbf{Отношение выводимости}: $\forall \alpha, \gamma, \delta \in V^*, A \in V_N: A \to \alpha \in P: \gamma A \delta \derives[] \gamma \alpha \delta$
    \item \textbf{Вывод}~--- транизитивное, рефлексивное замыкание отношения выводимости ($\derives, \derives[+], \derives[k]$)
    \item \textbf{Левосторонний (правосторонний) вывод}~--- на каждом шаге заменяем самый левый (правый) нетерминал
    \begin{itemize}
      \item Если не специфицируется, подразумевается левосторонний вывод
    \end{itemize}
    \item По сути, правила грамматики рассматриваются как правила переписывания
  \end{itemize}
\end{frame}

\begin{frame}[fragile]
  \transwipe[direction=90]
  \frametitle{Пример вывода}
  Построим левосторонний вывод цепочки $2+3*4$ в грамматике \[\langle \{ 0, 1, \dots, 9, +, *\}, \{E, N\}, P, E \rangle\]
  
\begin{align*}
  E &\to E + N \mid E * N \mid N \\ 
  N &\to 0 \mid 1  \mid \dots \mid 9
\end{align*}

\[ \underline{E} \derives[]  \underline{E} * N \derives[] \underline{E} + N * N \derives[] \underline{N} + N * N \derives[] 2 + N * N \derives[2] 2 + 3 * 4\]
\end{frame}

\begin{frame}[fragile]
  \transwipe[direction=90]
  \frametitle{Существование левостороннего вывода}
  \begin{rutheorem}
    Если для цепочки $\omega$ существует некоторый вывод $S \derives \omega$, 
    
    то существует и левосторонний вывод для этой цепочки $S \xRightarrow[l]{*} \omega$
  \end{rutheorem}
  \begin{proof}
  Докажем более общее утверждение: 
  
  если существует $A \derives \omega$, то существует $A \xRightarrow[l]{*} \omega$, где $A \in V_N$. 
  
  Доказываем по индукции по длине вывода $k$
  
    $k = 1: A \derives[] \omega$~--- тривиально. 
    
  $k \mapsto k + 1: \sphericalangle A \derives[] \alpha \derives \omega$. 
  
  Обозначим $ \alpha = B_1 B_2 \dots B_m \derives \omega_1 \omega_2 \dots \omega_m = \omega; \forall i: B_i \xRightarrow[]{l_i} \omega_i, l_i \leq n$
  
  По индукционному предположению $\forall i: B_i \xRightarrow[l]{*} \omega_i$
  
  $\Mapsto: A \to B_1 B_2 \dots B_m \xRightarrow[l]{*} \omega_1  B_2 \dots B_m \xRightarrow[l]{*} \omega$~--- левосторонний вывод
  \end{proof}
\end{frame}

\begin{frame}[fragile]
  \transwipe[direction=90]
  \frametitle{Единственность вывода}
  
  \begin{center}
    Не всегда (левосторонний) вывод единственен: 
    
    2 вывода строки $2+3*4$
  \end{center}
  
\vspace{-0.8cm}

\begin{align*}
  E &\to E + E \mid E * E \mid N \\
  N &\to 0 \mid 1  \mid \dots \mid 9
\end{align*}

\begin{tabular}{p{5.5cm} p{6cm}}
  
\Tree [.E [.E [.N 2 ] ] + [.E [.E [.N 3 ] ] * [.E [.N 4 ] ] ] ] 
& 
\Tree [.E [.E [.E [.N 2  ] ]  + [.E [.N 3 ] ] ] * [.E [.N 4 ] ] ]  
\end{tabular}
\end{frame}

\begin{frame}[fragile]
  \transwipe[direction=90]
  \frametitle{Однозначность грамматики}
  \begin{itemize}
    \item Грамматика называется \textbf{однозначной}, если для \emph{любого} слова языка существует \emph{единственный} (левосторонний) вывод
    \item Грамматика называется \textbf{неоднозначной}, если \emph{существует} слово языка, такое что для него \emph{существует} \emph{несколько} (левосторонних) выводов
  \end{itemize}

\pause

  \begin{itemize}
    \item По однозначной грамматике можно тривиальным образом построить неоднозначную: продублировать правило
    \begin{columns}
      \begin{column}{0.5\textwidth}
        \begin{align*}
          S &\to A \\
          A &\to a
        \end{align*}
      \end{column}
      \begin{column}{0.5\textwidth}
        \begin{align*}
          S &\to A \mid B \\
          A &\to a \\
          B &\to a 
        \end{align*}
      \end{column}
    \end{columns}

     \item Не существует общего алгоритма преобразования неоднозначной грамматики в однозначную
   \end{itemize}
\end{frame}

\begin{frame}[fragile]
  \transwipe[direction=90]
  \frametitle{Примеры однозначной и неоднозначной грамматики}
  
\begin{center}
  Неоднозначная грамматика
\end{center}

\vspace{-0.8cm}

  \begin{align*}
    E &\to E + E \mid E * E \mid N \\
    N &\to 0 \mid 1  \mid \dots \mid 9
  \end{align*}

\vfill

\begin{center}
  Однозначная грамматика
\end{center}

\vspace{-0.8cm}

  \begin{align*}
    E &\to E + N \mid E * N \mid N \\
    N &\to 0 \mid 1  \mid \dots \mid 9
  \end{align*}  
\end{frame}

\begin{frame}[fragile]
  \transwipe[direction=90]
  \frametitle{Проверка однозначности грамматики неразрешима}

\begin{rutheorem}
  Не существует алгоритма, определяющего по произвольной грамматике, что она является неоднозначной
\end{rutheorem}

\begin{proof}
  Сведение решения проблемы соответствий Поста к нашей задаче
\end{proof}


\vfill

\begin{center}
  Проблема соответствий Поста
\end{center}

 Даны списки $A = (a_1, \dots, a_n)$ и $B = (b_1 ,\dots ,b_n)$, где $\forall i: a_i, b_i \in \Sigma ^*$. 
 
 Cуществует ли непустая последовательность $(i_1 , \dots, i_k)$, удовлетворяющая условию $a_{i_1} \dots a_{i_k} = b_{i_1} \dots b_{i_k}$, где $\forall j: 1 \leq i_j \leq n$?

\end{frame}

\begin{frame}[fragile]
  \transwipe[direction=90]
  \frametitle{Проверка однозначности грамматики неразрешима}

\begin{rutheorem}
  Не существует алгоритма, определяющего по произвольной грамматике, что она является неоднозначной
\end{rutheorem}

\begin{proof}
  $\sphericalangle \text{ алфавит } \Sigma = \{a_1, \dots, a_n, b_1, \dots, b_n\} \sqcup \{z_1, \dots, z_n\}, \text{ где } a_i, b_i \text{ из ПСП }$

  $\sphericalangle$ грамматику: 

\vspace{-1.2cm}

  \begin{align*}
    S &\to A \mid B \\ 
    A &\to a_i A z_i \mid \varepsilon \\
    B &\to b_i B z_i \mid \varepsilon
  \end{align*}

  \vspace{-0.35cm}

  Если грамматика неоднозначна, то существует выводимое слово вида: 

  \vspace{-0.5cm}
  \[a_{i_1} a_{i_2} \dots a_{i_k} z_{i_k} z_{i_{k-1}} \dots z_{i_1} = \omega = b_{i_1} b_{i_2} \dots b_{i_k} z_{i_k} z_{i_{k-1}} \dots z_{i_1}\]

  Если бы умели решать ПСП, то мы могли бы проверить грамматику на однозначность, но ПСП неразрешима, а значит и проверка на однозначность неразрешима 
\end{proof}

\end{frame}


\begin{frame}[fragile]
  \transwipe[direction=90]
  \frametitle{Контекстно-свободный язык}
  \begin{itemize}
    \item Язык называется \textbf{контекстно-свободным}, если для него \emph{существует} контекстно-свободная грамматика
    \item Язык, задаваемый КС грамматикой $\langle V_T, V_N, P, S\rangle$: $\{ \omega \in V_T^* \mid S \derives \omega \}$
    \item КС язык называется \textbf{существенно неоднозначным}, если для него не существует однозначной грамматики
    \begin{itemize}
      \item $\{0^a 1^b 2^c \mid a = b \text{ либо } b = c \}$ \hfill доказательство в книге Ахо Ульмана
    \end{itemize}
  \end{itemize}
\end{frame}


\begin{frame}[fragile]
  \transwipe[direction=90]
  \frametitle{Пустота КС языка}
   \begin{rutheorem}
   Существует алгоритм, определяющий, является ли язык, порождаемый КС грамматикой, пустым
   \end{rutheorem}
   \begin{proof}
     Для доказательства потребуется следующая лемма
   \end{proof}
\end{frame}
  


  
\begin{frame}[fragile]
  \transwipe[direction=90]
  \frametitle{Лемма}
   \begin{rutheorem}
   Если в данной грамматике выводится некоторая цепочка, то существует цепочка, дерево вывода которой не содержит ветвей длиннее $m$, где $m$~--- количество нетерминалов грамматики
   \end{rutheorem}

   \begin{proof}
   Рассмотрим дерево вывода цепочки $\omega$. Если в нем есть 2 узла, соответствующих одному нетерминалу $A$, обозначим их $n_1$ и $n_2$. Предположим, $n_1$ расположен ближе к корню дерева, чем $n_2$; $S \derives \alpha A_{n_1} \beta \derives \alpha \omega_1 \beta; S \derives \gamma A_{n_2} \delta \derives \gamma \omega_2 \delta$; $\omega_2$ является подцепочкой $\omega_1$. 
   
   Заменим в изначальном дереве узел $n_1$ на $n_2$. Полученное дерево является деревом вывода $\alpha \omega_2 \delta$. Повторяем процесс замены одинаковых нетерминалов до тех пор, пока в дереве не останутся только уникальные нетерминалы. 
   
   В полученном дереве не может быть ветвей длины большей, чем $m$. По постороению оно является деревом вывода. 
   \end{proof}
\end{frame}

\begin{frame}[fragile]
  \transwipe[direction=90]
  \frametitle{Лемма}
   \begin{rutheorem}
   Если в данной грамматике выводится некоторая цепочка, то существует цепочка, дерево вывода которой не содержит ветвей длиннее $m$, где $m$~--- количество нетерминалов грамматики
   \end{rutheorem}
\begin{columns}[T]
  \begin{column}{0.45\textwidth}
    \begin{center}
      \begin{tikzpicture}
      \node[draw=none, fill=none] at (0, 3.5) {S};
      \node[draw=none, fill=none] at (-0.1, 2.3) {A};
      \node[draw=none, fill=none] at (0.1, 1) {A};
  
      \draw (-2,0) -- (0,3.3);
      \draw (0,3.3) -- (2,0);
      \draw (-1.5,0) -- (-0.1,2.1);
      \draw (-0.1,2.1) -- (1.5,0);
      \draw (-1,0) -- (0.1,0.8);
      \draw (0.1,0.8) -- (1,0);
  
      \draw[draw=black] (-2,-0.5) rectangle ++(4,0.5);
      \fill[lightgray] (-1.5,-0.5) rectangle ++(3,0.5);
      \draw[draw=black] (-1.5,-0.5) rectangle ++(3,0.5);
      \draw[draw=black] (-2,-1) rectangle ++(4,0.5);
      \draw[draw=black] (-1,-1) rectangle ++(2,0.5);
      \draw[dashed] (-1,-0.5) rectangle ++(2,0.5);
  
      \draw [thick,densely dotted, rounded corners=1mm](0,3.3) --  (0.1, 2.9) -- (-0.15, 2.7) -- (-0.1, 2.4);
  
      \draw [thick,densely dotted, rounded corners=1mm](-0.1, 2.1) --  (-0.1, 1.7) -- (0.15, 1.4) -- (0.1, 1.1);
  
  
      \node[draw=none, fill=none] at (-1.75,-0.25) {$\alpha$};
      \node[draw=none, fill=none] at (0,-0.25) {$\omega_1$};
      \node[draw=none, fill=none] at (0,-0.75) {$\omega_2$};
      \node[draw=none, fill=none] at (1.75,-0.25) {$\beta$};
      \node[draw=none, fill=none] at (-1.5,-0.75) {$\gamma$};
      \node[draw=none, fill=none] at (1.5,-0.75) {$\delta$};
      \end{tikzpicture}
    \end{center}
  \end{column}
  \begin{column}{0.1\textwidth}
    \vspace{2cm}
    \[\color{teal} \Mapsto\]
  \end{column}
  \begin{column}{0.45\textwidth}
    \begin{center}
      \begin{tikzpicture}
      \node[draw=none, fill=none] at (0, 3.5) {S};
      \node[draw=none, fill=none] at (0.1, 1) {A};
  
      \draw (-1.5,0) -- (0,3.3);
      \draw (0,3.3) -- (1.5,0);

      \draw (-1,0) -- (0.1,0.8);
      \draw (0.1,0.8) -- (1,0);
  
      \draw[draw=black] (-1.5,-0.5) rectangle ++(3,0.5);
      \draw[draw=black] (-1,-0.5) rectangle ++(2,0.5);
  
      \draw [thick,densely dotted, rounded corners=1mm] (0,3.3) --  (-0.1, 1.7) -- (0.15, 1.4) -- (0.1, 1.1);
  
  
      \node[draw=none, fill=none] at (-1.25,-0.25) {$\alpha$};
      \node[draw=none, fill=none] at (0,-0.25) {$\omega_2$};
      \node[draw=none, fill=none] at (1.25,-0.25) {$\beta$};
      \end{tikzpicture}
    \end{center}
  \end{column}
\end{columns}
\end{frame}

\begin{frame}[fragile]
  \transwipe[direction=90]
  \frametitle{Алгоритм проверки пустоты КС языка}
   \begin{proof}
   Строим коллекцию деревьев, представляющих вывод в грамматике.
   
  \begin{enumerate}
    \item Инициализируем коллекцию деревом из одного узла S
    \item Добавляем в коллекцию дерево, полученное применением единственного правила грамматики из какого-нибудь дерева из коллекции, если его в нем еще нет, и самая длинная ветвь не длиннее $m$
    \item Если после окончания построения коллекции в ней существует дерево, являющееся деревом вывода некоторой цепочки терминалов, значит, язык не пуст
  \end{enumerate}
   \end{proof}
\end{frame}

\begin{frame}[fragile]
  \transwipe[direction=90]
  \frametitle{Упрощение КС грамматики: удаление непродуктивных нетерминалов}
  \textbf{Продуктивный нетерминал}:  нетерминал, для которого существует цепочка терминалов, выводимая из него ($\exists \omega \in V_T^*: A \derives \omega$)

\vfill

  \textbf{Непродуктивный нетерминал}:  нетерминал, не являющийся продуктивным
\end{frame}

\begin{frame}[fragile]
  \transwipe[direction=90]
  \frametitle{Упрощение КС грамматики: удаление непродуктивных нетерминалов}

  \begin{rutheorem}
    Для любой КС грамматики $G = \langle V_T, V_N, P, S\rangle: L(G) \neq \varnothing$
    
    можно построить эквивалентную грамматику, каждый нетерминал которой продуктивен
  \end{rutheorem}

   \begin{proof}
   Удаляем из грамматики все нетерминалы $A: L(A) = \varnothing$, а также правила, использующие их. 
   
   Полученную грамматику обозначаем $G_1$.

   Докажем, что $L(G) = L(G_1)$. 
   
   Очевидно, $L(G_1) \subseteq L(G)$.

   Докажем от противного, что $L(G) \subseteq L(G_1)$. 
   
   Предположим, что $\exists \omega \in L(G)$, но $\omega \notin L(G_1)$. 
   Тогда $S \derives \alpha_1 A \alpha_2 \derives \omega$, где $A \in V_N \setminus V_{N_1}$, но тогда $\exists \gamma \in V_T^*: A \derives \gamma $. Противоречие
   \end{proof}
\end{frame}

\begin{frame}[fragile]
  \transwipe[direction=90]
  \frametitle{Упрощение КС грамматики: приведение}

  \begin{rutheorem}
    Для любой КС грамматики, порождающей непустой язык, 
    
    можно постороить эквивалентную, для каждого нетерминала $A$ которой существует вывод вида $S \derives \omega_1 A \omega_3 \derives \omega_1 \omega_2 \omega_3, \omega_i \in V_T^*$
  \end{rutheorem}

   \begin{proof}
   Будем рассматривать грамматику без непродуктивных нетерминалов $G_1 = \langle V_{N_1}, V_T, P_1, S\rangle$.
   
   Верно: если существует $S \derives \alpha_1 A \alpha_3, \alpha_i \in V^*$, то $S \derives \alpha_1 A \alpha_3 \derives \omega_1 A \omega_3 \derives \omega_1 \omega_2 \omega_3, \omega_i \in V_T^*$
   
   Строим множество нетерминалов, встречающихся в выводах: добавляем сначала $S$, потом добавляем нетерминалы, встречающиеся в правой части правил для нетерминалов из множества. Завершаем процесс, когда больше ничего не добавить. Обозначаем полученное множество $V_{N_2}$, удаляем все правила грамматики, содержащие нетерминалы из $ V_{N_1} \setminus V_{N_2}$
   
   
   \end{proof}
\end{frame}

\begin{frame}[fragile]
  \transwipe[direction=90]
  \frametitle{Упрощение КС грамматики: приведение}
   \begin{proof}
   Получили грамматику $G_2 = \langle V_{N_2}, V_T, P_2, S\rangle$. 

   Докажем: $L(G_2) = L(G_1)$
   
   $L(G_2) \subseteq L(G_1)$, так как $P_2 \subseteq P_1$
   
   Докажем: $L(G_1) \subseteq L(G_2)$. Пусть $S \deriveg{G_1} \omega$. Все нетерминалы, встречающиеся в этом выводе содержатся в $V_{N_2}$, соответственно используются только правила из $ P_2 \derives[] S \deriveg{G_2} \omega $
   
   Так как все нетерминалы $V_{N_2}$ продуктивны, то  $S \derives \omega_1 A \omega_3 \derives \omega_1 \omega_2 \omega_3, \omega_i \in V_T^*$
   
   
   \end{proof}
   
   Грамматика $G_2$ называется \textbf{приведенной}, ее нетерминалы~--- \textbf{достижимыми}
   
   Недостижимые и непродуктивные нетерминалы называются \textbf{бесполезными}
\end{frame}

\begin{frame}[fragile]
  \transwipe[direction=90]
  \frametitle{Упрощение КС грамматики: удаление цепных правил}
  Правило называется \textbf{цепным}, если оно имеет вид $A \to B; A, B \in V_N$.
  
  \begin{rutheorem}
    Для любой КС грамматики $G=\langle V_N, V_T, P, S \rangle$ можно построить эквивалентную, не содержащую цепных правил
  \end{rutheorem}
  
   \begin{proof}
   Строим новое множество правил $P_1$. 
   
   Включаем в него все нецепные правила  $P$. 
   
   Затем добавляем в $P_1$ правила вида $A \derives \alpha$, если $A \derives B$, где $A, B \in V_N$ и $B \to \alpha$~--- нецепное правило из $P$.
   
   Замечание: достаточно проверять только цепные выводы длины меньшей, чем $V_N$
   
   Обозначим полученную грамматику за $G_1=\langle V_N, V_T, P_1, S \rangle$, докажем $L(G_1)=L(G)$
      \end{proof}
\end{frame} 
   
\begin{frame}[fragile]
  \transwipe[direction=90]
  \frametitle{Упрощение КС грамматики: удаление цепных правил}    
  \begin{proof}   
   Очевидно $L(G_1) \subseteq L(G)$
   
   Покажем $L(G) \subseteq L(G_1)$. Пусть $\omega \in L(G)$. Рассмотрим левосторонний вывод $S \derivegone{G} \alpha_0 \derivegone{G} \alpha_1 \derivegone{G} \dots \derivegone{G} \alpha_n = \omega$.
   
   Предположим $\alpha_i \derivegone{G} \alpha_{i+1}$~--- первый шаг, выполняемый посредством цепного правила в выводе; $\forall k \in [i..j]: \alpha_k \derivegone{G} \alpha_{k+1}$~--- посредством цепного правила;  $\alpha_j \derivegone{G} \alpha_{j+1}$ ~--- посредством нецепного правила
   
   Тогда $|\alpha_i| = |\alpha_{i+1}| = \dots = |\alpha_j|$, и на каждом шаге заменяется один и тот же нетерминал. Тогда $\alpha_i \derivegone{G_1} \alpha_{j+1} $ посредством правила из $P_1 \setminus P \text{, следовательно } \omega \in L(G_1)$
   \end{proof}
   
\end{frame}

\begin{frame}[fragile]
  \transwipe[direction=90]
  \frametitle{Нормальная форма Хомского}    
  КС грамматика находится в \textbf{нормальной форме Хомского}, если все ее правила имеют вид: 
  
\begin{align*}
  A &\to B C         & A, B, C \in V_N \\ 
  A &\to a           & A \in V_N, a \in V_T\\
  S &\to \varepsilon & S \text{ --- стартовый нетерминал}
\end{align*}
 
\end{frame}

 
\begin{frame}[fragile]
  \transwipe[direction=90]
  \frametitle{Нормальная форма Хомского}    
  
  \begin{rutheorem}   
    Для любой КС грамматики можно построить эквивалентную в нормальной форме Хомского
  \end{rutheorem}

     \begin{enumerate}
      \setcounter{enumi}{-1}
    \item Удаляем непродуктивные нетерминалы
    \item Удаляем цепные правила. Теперь $\forall A \to B: B \in V_T$
    \item Заменяем каждое правило $A \to B_1 B_2 \dots B_n$ на $A \to C_1 C_2 \dots C_n$
    \begin{itemize}
      \item Если $B_i \in V_N$, $C_i = B_i,$ 
      \item Если $B_i \in V_T$, добавляем также правило $C_i \to B_i$, 
    \end{itemize}  
    \item Заменяем правило $A \to C_1 C_2 \dots C_n$ на множество правил: 
    {\small 
    \begin{align*}
      A   &\to C_1 D_1 \\
      D_1 &\to C_2 D_2 \\
          &\dots \\ 
      D_{n-3} &\to C_{n-2} D_{n-2} \\
      D_{n-2} &\to C_{n-1} C_n 
    \end{align*}
    }%
  \end{enumerate}
  
        Полученная грамматика находится в НФХ и эквивалентна данной 
\end{frame}

\begin{frame}[fragile]
  \transwipe[direction=90]
  \frametitle{Пример приведения в НФХ}    
  \[G = \langle \{S, A, B\}, \{a, b\}, P, S\rangle \text{, где } P:\]
  
\begin{align*}
  S &\to bA \mid aB \\
  A &\to a  \mid aS \mid bAA  \\
  B &\to b  \mid bS \mid aBB 
\end{align*}

  \begin{itemize}
    \item $S \to b A \Mapsto S \to C_1 A; C_1 \to b$
    \item $S \to a B \Mapsto S \to C_2 B; C_2 \to a$
    \item $A \to aS \Mapsto A \to C_3 S; C_3 \to a$
    \item $A \to b A A \Mapsto A \to C_4 D_1; C_4 \to b; D_1 \to A A$
    \item $B \to b S \Mapsto B \to C_5 S; C_5 \to b$
    \item $B \to a B B \Mapsto B \to C_6 D_2; C_6 \to a; D_2 \to B B$
  \end{itemize}
\end{frame}


\begin{frame}[fragile]
  \transwipe[direction=90]
  \frametitle{Еще немного упростим}      
  
\begin{align*}
  S &\to bA \mid aB \\
  A &\to a  \mid aS \mid bAA  \\
  B &\to b  \mid bS \mid aBB 
\end{align*}

\vfill 

\begin{align*}
  S   &\to C_b A \mid C_a B \\
  A   &\to a \mid C_a S \mid C_b D_1 \\
  B   &\to b \mid C_b S \mid C_a D_2 \\
  D_1 &\to A A \\
  D_2 &\to B B \\
  C_a &\to a \\
  C_b &\to b 
\end{align*}
\end{frame}

\begin{frame}[fragile]
  \transwipe[direction=90]
  \frametitle{Алгоритм приведения в НФХ}      
  
  \begin{enumerate}
    \item Удалить стартовый нетерминал из правых частей правил 
    \begin{itemize}
      \item добавляется новое правило $S_0 \to S, S_0 \notin V_N, S_0$ делается новым стартовым
    \end{itemize}
    \item Избавиться от неодиночных терминалов в правых частях 
    \begin{itemize} 
      \item новое правило $C_c \to c$
    \end{itemize}
    \item Удалить длинные правила (длины больше 2)
    \item Удалить непродуктивные правила ($\varepsilon$-правила)
    \begin{itemize}
      \item Если $A \to \varepsilon$, то $A$~--- $\varepsilon$-правило
      \item Если $A \to X_1 X_2 \dots X_n, \forall i: X_i$~--- $\varepsilon$-правило, то $A$~--- $\varepsilon$-правило
      \item Заменяем $A \to X_1 X_2 \dots X_n$ на множество правил, где каждое из $\varepsilon$-правил опущено во всех возможных комбинациях, удаляем те, которые выводят $\varepsilon$
    \end{itemize}
    \item Удалить цепные правила
    \begin{itemize}
      \item Для каждой пары правил $A \to B; B \to  X_1 X_2 \dots X_n$ добавить правило $A \to  X_1 X_2 \dots X_n$, цепное правило удалить
    \end{itemize}
  \end{enumerate}
\end{frame}

\begin{frame}[fragile]
  \transwipe[direction=90]
  \frametitle{Порядок действий при приведении в НФХ}      
  
  \begin{center}
    Порядок \textbf{важен}!
  \end{center}
  
  \begin{enumerate}
    \item Удалить стартовый нетерминал из правых частей правил 
    \item Избавиться от неодиночных терминалов в правых частях 
    \item Удалить длинные правила (длины больше 2)
    \item Удалить непродуктивные правила ($\varepsilon$-правила)
    \item Удалить цепные правила
  \end{enumerate}

  \begin{itemize}
    \item 1 шаг порождает новые цепные правила, поэтому его нельзя выполнять после 5 шага
    \item Если выполнить 4 шаг перед 3 шагом, то произойдет экспоненциальный взрыв грамматики
    \item 5 шаг приводит к квадратичному возрастанию размера грамматики
    \item Наиболее эффективны порядки $1, 2, 3, 4, 5$ и $1,3, 4,5,2$
  \end{itemize}
\end{frame}

\begin{frame}[fragile]
  \transwipe[direction=90]
  \frametitle{Увеличение размера грамматики при нормализации}

\begin{center}
    Порядок \textbf{важен}!
\end{center}
  
  \begin{enumerate}
    \item Удалить стартовый нетерминал из правых частей правил 
    \begin{itemize}
      \item Увеличение на 1
    \end{itemize}
    \item Избавиться от неодиночных терминалов в правых частях 
    \begin{itemize}
      \item Увеличение на $|V_T|$ правил
    \end{itemize}
    \item Удалить длинные правила (длины больше 2)
    \begin{itemize}
      \item Увеличение не более, чем в 2 раза (для правил длины $k \geq 3$ порождается $k-1$ новых правил)
    \end{itemize}
    \item Удалить непродуктивные правила ($\varepsilon$-правила)
    \begin{itemize}
      \item Увеличение не более, чем в 3 раза
    \end{itemize}
    \item Удалить цепные правила
    \begin{itemize}
      \item Увеличение не более, чем в $O(n^2)$ (цепных правил не больше $n^2$, где $n$~--- число нетерминалов)
    \end{itemize}
  \end{enumerate}
  
  Итого: \textbf{полиномиальное} увеличение размеров грамматики при правильном порядке действий
\end{frame}

\begin{frame}[fragile]
  \transwipe[direction=90]
  \frametitle{Синтаксический анализ: алгоритм Кока-Янгера-Касами (Cocke-Younger-Kasami algorithm, CYK)}
  
  \begin{center}
    Что значит $A \to a$?
  \end{center} \pause

 \[ A \to a  \derives \omega \Leftrightarrow \omega = a \]
\end{frame}

\begin{frame}[fragile]
  \transwipe[direction=90]
  \frametitle{Синтаксический анализ: алгоритм Кока-Янгера-Касами (Cocke-Younger-Kasami algorithm)}
  
  \begin{center}
    Что значит $A \to BC$?
  \end{center} \pause
  
  \[
    A \to BC  \derives \omega \Leftrightarrow 
              \begin{cases}
                \exists \omega_1, \omega_2: \omega = \omega_1 \omega_2\\
                B \derives \omega_1\\
                C \derives \omega_2
              \end{cases}
  \] \pause

  
  \begin{center}
    Или:
  \end{center}

  \[
    A \to BC  \derives \omega \Leftrightarrow \exists k \in [0 \dots |\omega|]:
              \begin{cases}
                B \derives \omega[0 \dots k]\\
                C \derives \omega[k+1 \dots |\omega|]
              \end{cases}
  \]

\end{frame}

\begin{frame}[fragile]
  \transwipe[direction=90]
  \frametitle{Синтаксический анализ: алгоритм Кока-Янгера-Касами (Cocke-Younger-Kasami algorithm, CYK)}
  \begin{itemize}
      \item Алгоритм синтаксического анализа, работающий с грамматиками в НФХ
      \item Динамическое программирование
  \end{itemize}

  \begin{figure}
    \begin{tikzpicture}
    \draw (0,2.8) -- (-0.5, 1.7);
    \draw (0,2.8) -- (0.5, 1.7);
    \draw (-0.5, 1.3) -- (-2, 0.05);
    \draw (0.5, 1.3) -- (2, 0.05);
    \draw (-0.5, 1.3) -- (-0.2, 0.05);
    \draw (0.5, 1.3) -- (0.2, 0.05);
    \draw (0,2.8) -- (-2, 0.05);
    \draw (0,2.8) -- (2, 0.05);
    \node[draw=none,fill=none] (A) at (0, 3) {A}; 
    \node[draw=none,fill=none] (B) at (-0.5, 1.5) {B}; 
    \node[draw=none,fill=none] (C) at (0.5, 1.5) {C};
    \node[draw=none,fill=none] (w) at (-3.5, -0.2) {$\omega$:};
    \node[draw=none,fill=none] (o) at (-2.9, -0.6) {};
    \node[draw=none,fill=none, right=0.6 of o]  {i};
    \node[draw=none,fill=none, right=2.35 of o]  {k};
    \node[draw=none,fill=none] at (0.5, -0.615)  {k+1};
    \node[draw=none,fill=none, right=4.6 of o] {j};
    
    \draw[draw=black] (-3,-0.4) rectangle ++(6,0.4);
    \draw[draw=black] (-2.2,-0.4) rectangle ++(0.4,0.4);
    \draw[draw=black] (-0.4,-0.4) rectangle ++(0.4,0.4);
    \draw[draw=black] (0,-0.4) rectangle ++(0.4,0.4);
    \draw[draw=black] (1.8,-0.4) rectangle ++(0.4,0.4);

    \end{tikzpicture}
  \end{figure}
\end{frame}


\begin{frame}[fragile]
  \transwipe[direction=90]
  \frametitle{CYK}
  \begin{itemize}
      \item Дано: строка $\omega$ длины $n$, грамматика $G = \langle V_T, V_N, P, S\rangle$ в НФХ
      \item Используем трехмерный массив d булевых значений размером $|V_N| \times n \times n$, $d[A][i][j] = true \Leftrightarrow A \derives \omega[i \dots j]$
      \item Инициализация: $i = j$
      \begin{itemize}
        \item $d[A][i][i] = true$, если в грамматике есть правило $A \to \omega[i]$
        \item $d[A][i][i] = false$, иначе
      \end{itemize}
      \item Динамика. Предполагаем, d построен для всех нетерминалов и пар $\{(i', j') \mid j' - i' < m \}$
      \begin{itemize}
        \item $d[A][i][j] = \bigvee_{A\to BC}^{}{\bigvee_{k=i}^{j-1}{d[B][i][k] \wedge d[C][k][j]}}$
      \end{itemize}
      \item В конце работы алгоритма в $d[S][0][n]$ записан ответ, выводится ли $\omega$ в данной грамматике
  \end{itemize}
\end{frame}

 \end{document}
 

  
