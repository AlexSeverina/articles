\documentclass{beamer}
\usepackage{beamerthemesplit}
\usepackage{wrapfig}
\usetheme{SPbGU}
\usepackage{pdfpages}
\usepackage{amsmath}
\usepackage{mathtools}
\usepackage{cmap} 
\usepackage[T2A]{fontenc} 
\usepackage[utf8]{inputenc}
\usepackage[english,russian]{babel}
\usepackage{indentfirst}
\usepackage{amsmath}
\usepackage{tikz}
\usepackage{multirow}
\usepackage[noend]{algpseudocode}
\usepackage{algorithm}
\usepackage{algorithmicx}
\usepackage{ stmaryrd }
\usepackage{qtree}
\usetikzlibrary{shapes,arrows}
\usepackage{fancyvrb}
\usepackage{minted}
\usepackage{forest}
\usepackage{tikz-qtree}
\newtheorem{rutheorem}{Теорема}
\newtheorem{ruproof}{Доказательство}
\newtheorem{rudefinition}{Определение}
\newtheorem{rulemma}{Лемма}
\beamertemplatenavigationsymbolsempty

\setbeamertemplate{itemize item}[circle]
\setbeamertemplate{enumerate item}[circle]

\newcommand{\derive}[0]{\xRightarrow[]{*}}
\newcommand{\derives}[0]{\xRightarrow[]{}}
\newcommand{\derivek}[1]{\xRightarrow[]{#1}}
\newcommand{\deriveg}[1]{\xRightarrow[#1]{*}}
\newcommand{\derivegone}[1]{\xRightarrow[#1]{}}

\title[]{Теория автоматов и формальных языков}
\subtitle[]{Синтаксически управляемая трансляция}
\institute[]{
Санкт-Петербургский государственный электротехнический университет <<ЛЭТИ>>\\
}

\author[]{Екатерина Вербицкая}

\date{19 декабря 2019}

\definecolor{orange}{RGB}{179,36,31}

\begin{document}
{
  \begin{frame}
    \titlepage
  \end{frame}
}

\begin{frame}[fragile]
  \transwipe[direction=90]
  \frametitle{В предыдущей серии}
  \begin{itemize}
    \item Что такое язык; когда предложение принадлежит языку
    \item Классы языков
    \begin{itemize}
	  \item Регулярные
	  \item Контекстно-свободные
	  \begin{itemize}
	    \item LL(k)
	    \item LR(k), LALR(k)
	  \end{itemize}
	  \item Задаваемые PEG   
    \end{itemize}     
    \item Как задать язык
    \begin{itemize}
      \item Конечный автомат
      \item Магазинный автомат
      \item PEG
    \end{itemize}
    \item Синтаксический анализ
    \begin{itemize}
      \item Определение, принадлежит ли цепочка языку
      \item Построение дерева разбора
    \end{itemize}
  \end{itemize}
\end{frame}

\begin{frame}[fragile]
  \transwipe[direction=90]
  \frametitle{Дерево разбора --- лишь цепочка в некотором языке }
\begin{center}
\begin{tikzpicture}
\tikzset{edge from parent/.style={draw,edge from parent path={(\tikzparentnode.south)-- +(0,-8pt)-| (\tikzchildnode)}}}
\Tree [.E
[.E [.T [.P [.n ] ] ] ]
[.+ ]
[.T
  [.T [.P [.n ] ] ] 
  [.* ]
  [.P [.n ] ] ]
  ]
\end{tikzpicture}  
  \pause
  
  ~\\~
  
$[.E [.E [.T [.P [.n ] ] ] ] [.+ ] [.T [.T [.P [.n ] ] ] [.* ] [.P [.n ] ] ] ]$  
\end{center}  
\end{frame}


\begin{frame}[fragile]
  \transwipe[direction=90]
  \frametitle{Трансляция (перевод)}
  \begin{itemize}
    \item \textbf{Трансляция} --- преобразование некоторой входной строки в некоторую выходную
    \begin{itemize}
      \item $\Sigma$ --- входной алфавит, $\Pi$ --- выходной алфавит. 
      
      \textbf{Трансляцией} с языка $L_i \subseteq \Sigma^*$ на язык $L_o \subseteq \Pi^*$ 
      
      называется отображение $\tau : L_i \rightarrow L_o$
    \end{itemize}
    \item Построение дерева разбора --- простейший пример трансляции
    \item Другие примеры трансляции
    \begin{itemize}
      \item Вычисление значения арифметического выражения
      \item Преобразование арифметического выражения из инфиксной записи в постфиксную
      \item Преобразование программы на языке Java в байт-код
      \item Компиляция программ
      \item \dots
    \end{itemize}
    \item Фактически синтаксический анализ нужен для трансляции
  \end{itemize}
\end{frame}


\begin{frame}[fragile]
  \transwipe[direction=90]
  \frametitle{Схемы синтаксически управляемой трансляции}

\begin{center}
    Схема синтаксически управляемой трансляции --- пятерка $(N, \Sigma, \Pi, P, S)$ 
\end{center} 

  \begin{itemize}
    \item $N$ --- конечное множество нетерминальных символов
    \item $\Sigma$ --- конечный входной алфавит 
    \item $\Pi$ --- конечный выходной алфавит
    \item $S \in N$ --- стартовый нетерминал
    \item $P$ --- конечное множество правил трансляции вида $A \rightarrow \alpha, \beta$, где $\alpha \in (N \cup \Sigma)^*, \beta \in (N \cup \Pi)^*$
    \begin{itemize}
      \item Вхождения нетерминалов в цепочку $\beta$ образуют перестановку нетерминалов из цепочки $\alpha$
      \item Если нетерминалы повторяются больше одного раза, то их различают по индексам: $E \rightarrow E^l + E^r, E^r + E^l$
    \end{itemize}
  \end{itemize}
\end{frame}

\begin{frame}[fragile]
  \transwipe[direction=90]
  \frametitle{Выводимая пара в СУ-схеме}
  \begin{itemize}
    \item Если $A \rightarrow (\alpha, \beta) \in P$, то $(\gamma A^i \delta, \ \gamma' A^i \delta') \Rightarrow (\gamma \alpha \delta, \ \gamma' \beta \delta')$
    \item Рефлексивно-транзитивное замыкание отношения $\Rightarrow$ называется отношением выводимости в СУ-схеме, обозначается $\xRightarrow[]{*}$
    \item Трансляцией назовем множество пар $\{ (\alpha, \beta) \, | \, (S, S) \derive (\alpha, \beta), \alpha \in \Sigma^*, \beta \in \Pi^* \}$
    \item СУ-схема называется простой, если во всех правилах $A \rightarrow (\alpha, \beta)$, нетерминалы в $\alpha$ и $\beta$ встречаются в одном и том же порядке
  \end{itemize}
\end{frame}

\begin{frame}[fragile]
  \transwipe[direction=90]
  \frametitle{Пример СУ-схемы}
  
$$
\begin{array}{ccclcl}
&E& \rightarrow & E + T &, & E T + \\
& &    \mid     & T &,& T \\
&T& \rightarrow & T * F &,& T F * \\
& &    \mid     & F &,& F \\
&F& \rightarrow & n &,& n \\
& &    \mid     & ( E ) &,& E \\
\end{array}
$$

\pause ~\\~
  
$(\underline{E},\underline{E}) \derives (\underline{T},\underline{T}) \derives (\underline{T}*F, \underline{T} \, F*) \derives (\underline{F}*F,\underline{F} \, F*) \derives (n*\underline{F},n \, \underline{F} *) \derives (n * (\underline{E}), n \, \underline{E} *) \derives  (n * (\underline{E} + T), n \, \underline{E} \, T + *) \derives (n * (\underline{T} + T), n \, \underline{T} \, T + *) \derives (n * (\underline{F} + T), n \, \underline{F} \, T + *) \derives (n * (n + \underline{T}), n \, n \, \underline{T} + *) \derives (n * (n + \underline{F}), n \, n \, \underline{F} + *)  \derives (n * (n + n), n \, n \, n + *) $  
  
\end{frame}

\begin{frame}[fragile]
  \transwipe[direction=90]
  \frametitle{Обобщенные схемы синтаксически управляемой трансляции}

  \begin{center}
    Обобщенная схема синтаксически управляемой трансляции --- шестерка $(N, \Sigma, \Pi, \Gamma, P, S)$  
  \end{center}

  \begin{itemize}
    \item $\Gamma$ --- конечное множество символов перевода вида $A_i, A \in N; i \in \mathbb{Z}$
    \item $P$ --- конечное множество правил трансляции вида $A \rightarrow 	\alpha, A_1 = \beta_1, \dots, A_n = \beta_n$, где $\alpha \in (N \cup \Sigma)^*$
    \begin{itemize}
      \item $A_i \in \Gamma, 1 \leq i \leq n$
      \item Каждый символ $x$, входящий в $\beta_i$, либо $x \in \Pi$, либо $x = B_k \in \Gamma$, где $B \in \alpha$
      \item Если $\alpha$ имеет более одного вхождения символа $B$, то каждый символ $B_k$ во всех $\beta$ соотнесен (верхним индексом) с конкретным вхождением $B$
    \end{itemize}
  \end{itemize}
  

  \begin{center}
    \textbf{Входной грамматикой} назовем четверку $(N, \Sigma, P', S)$, где $P = \{ A \rightarrow \alpha \, | \, A \rightarrow \alpha, A_1 = \beta_1, \dots, A_n = \beta_n \in P  \}$
  \end{center}
\end{frame}

\begin{frame}[fragile]
  \transwipe[direction=90]
  \frametitle{Выход обобщенной СУ-схемы}
  \begin{itemize}
    \item Для каждой внутренней вершины дерева, соответствующей нетерминалу $A$, с каждым $A_i$ связывается одна цепочка 
    \begin{itemize}
      \item Такую цепочку назовем значением (трансляцией) символа $A_i$
    \end{itemize}     
    \item Каждое значение определяется подстановкой значений символов трансляции данного элемента $A_i = \beta_i$, определенных в прямых потомках вершины
    \item \textbf{Трансляция}, определяемая данной схемой --- множество $\{(\alpha, \beta)\}$
    \begin{itemize}
      \item $\alpha$ имеет дерево разбора в данной входной грамматике
      \item $\beta$ --- значение выделенного символа $S_k$
    \end{itemize}
  \end{itemize}

\end{frame}


\begin{frame}[fragile]
  \transwipe[direction=90]
  \frametitle{Пример обобщенной СУ-схемы: дифференцирование}
$$
\begin{array}{ccclcl}
&E& \rightarrow & E + T &,& E_1 = E_1 + T_1 \\
& &             &       &,& E_2 = E_2 + T_2 \\
& &      |      &     T &,& E_1 = T_1, \ E_2 = T_2  \\

&T& \rightarrow & T * F &,& T_1 = T_1 * F_1 \\
& &             &       &,& T_2 = T_1 * F_2 + T_2 * F_1\\
& &      |      &     F &,& T_1 = F_1, \ T_2 = F_2 \\

&F& \rightarrow & ( E ) &,& F_1 = (E_1) \\
& &             &       &,& F_2 = (E_2) \\
& &      |      & sin(E)&,& F_1 = sin(E_1) \\
& &             &       &,& F_2 = cos(E_1) * E_2 \\
& &      |      & cos(E)&,& F_1 = cos(E_1) \\
& &             &       &,& F_2 = -sin(E_1) * E_2 \\
& &      |      &    x  &,& F_1 = x, \ F_2 = 1 \\
& &      |      &    n  &,& F_1 = n, \ F_2 = 0  
\end{array}
$$
\end{frame}

\begin{frame}[fragile]
  \transwipe[direction=90]
  \frametitle{Транслирующие грамматики}
  \begin{itemize}
    \item КС-грамматика, терминальный алфавит которой разбит на два множества: входных и выходных символов
    \item \textbf{Транслирующая грамматика} --- пятерка $(N, \Sigma_i, \Sigma_o, P, S)$
    \begin{itemize}
      \item $N$ --- алфавит нетерминалов
      \item $\Sigma_i$ --- алфавит входных терминалов
      \item $\Sigma_o$ --- алфавит выходных терминалов
      \item $S \in N$ --- стартовый нетерминал
      \item $P = \{ A \rightarrow \alpha \}, \alpha \in (\Sigma_i \cup \Sigma_o \cup N)^*$ --- множество правил вывода
    \end{itemize}
  \end{itemize}
\end{frame}

\begin{frame}[fragile]
  \transwipe[direction=90]
  \frametitle{Пример транслирующей грамматики}
$$
\begin{array}{cccl}
&E& \rightarrow & E + T \, \{ + \} \\
& &    |        & T   \\
&T& \rightarrow & T * F \, \{ * \} \\
& &    |        & F  \\
&F& \rightarrow & n \, \{ n \} \\
& &    |        & ( E ) \\
\end{array}
$$
\pause ~\\~

$E \derives E + T \, \{ + \} \derives T + T \, \{ + \} \derives P + T \, \{ + \} \derives n \, \{ n \} + T \, \{ + \} \derives n \, \{ n \} + T * P \, \{ * \} \, \{ + \} \derives n \, \{ n \} + P * P \, \{ * \} \, \{ + \} \derives n \, \{ n \} +  n \, \{ n \} * P \, \{ * \} \, \{ + \} \derives n \, \{ n \} +  n \, \{ n \} * n \, \{ n \} \, \{ * \} \, \{ + \}$

\pause
\begin{itemize}
  \item Если вычеркнуть все выходные символы, получим $n + n * n$
  \item Если вычеркнуть все входные символы, получим $n \, n \, n + *$ --- постфиксная запись выражения
\end{itemize}
\end{frame}


\begin{frame}[fragile]
  \transwipe[direction=90]
  \frametitle{Постфиксная транслирующая грамматика}
  \begin{itemize}
    \item Если выходные символы встречаются только в конце правил, транслирующая грамматика называется постфиксной
    \item Это требование формально не выдвигается: транслирующие грамматики могут быть не постфиксными
    \item На практике постфиксные транслирующие грамматики удобнее
  \end{itemize}
\end{frame}


\begin{frame}[fragile]
  \transwipe[direction=90]
  \frametitle{Атрибутная транслирующая грамматика}
  \begin{itemize}
    \item Входной алфавит --- алфавит лексем
    \begin{itemize}
      \item Лексема характеризуется \textbf{типом} и \textbf{значением}
    \end{itemize}
    \item Транслирующая грамматика описывает перевод только \textbf{типа} лексемы
    \begin{itemize}
      \item Это существенно снижает выразительность формализма
    \end{itemize}
    \item Для борьбы с этим недостатком предложены атрибутные грамматики
    \begin{itemize}
      \item Модификация транслирующих грамматик, снабженная атрибутами
      \item Выходные символы транслирующих грамматик --- транслирующие символы
      \begin{itemize}
        \item Нетерминалы, которые раскрываются в $\varepsilon$, и в момент раскрытия выполняют связанные с ними действие
      \end{itemize}
    \end{itemize}
  \end{itemize}
\end{frame}


\begin{frame}[fragile]
  \transwipe[direction=90]
  \frametitle{Атрибут}

\begin{center}
    \textbf{Атрибут} --- дополнительные данные, ассоциированные с грамматическими символами
\end{center}
  \begin{itemize}
    \item Если $X$ --- символ, а $a$ --- его атрибут, то значение $a$ в узле дерева, помеченном $X$, записывается как $X.a$
    \item Узлы дерева могут реализовываться как записи или объекты, а атрибуты --- как поля
    \item Атрибуты могут быть любого типа 
    \item Если в каждом узле дерева атрибуты уже вычислены, оно называется \textbf{аннотированным}
    \item Процесс вычисления этих атрибутов называется \textbf{аннотированием} дерева разбора
  \end{itemize}
\end{frame}

\begin{frame}[fragile]
  \transwipe[direction=90]
  \frametitle{Вычисление атрибутов не всегда возможно}
$$
\begin{array}{clcl}
&A \rightarrow    B   & & A.s = B.i \\
&                     & & B.i = A.s + 1 
\end{array}
$$  
\end{frame}


\begin{frame}[fragile]
  \transwipe[direction=90]
  \frametitle{Синтезируемый атрибут, S-атрибутная грамматика}
  \begin{itemize}
    \item Атрибут, значение которого зависит от значений атрибутов детей данного узла или от других атрибутов этого узла, называется \textbf{синтезируемым}
    \item Если в транслирующей грамматике используются только синтезируемые атрибуты, она называется \textbf{S-атрибутной грамматикой}
    \item Аннотирование дерева разбора S-атрибутной грамматики возможно путем выполнения семантических правил снизу вверх (от листьев к корню)
  \end{itemize}    
  
\end{frame}


\begin{frame}[fragile]
  \transwipe[direction=90]
  \frametitle{Пример S-атрибутной грамматики}
$$
\begin{array}{ccclll}
&S  & \rightarrow & E       &                                & S.val = E.val \\~\\

&E_0& \rightarrow & E_1 + T & \{ ADD.res = op_1 + op_2 \} & ADD.op_1 = E_1.val \\
&   &             &         &                                & ADD.op_2 = T.val \\
&   &             &         &                                & E_0.val = ADD.res \\~\\ 

&E  & \rightarrow & T       &                                & E.val = T.val \\~\\

&T_0& \rightarrow & T_1 + F & \{ MUL.res = op_1 * op_2 \} & MUL.op_1 = T_1.val \\
&   &             &         &                                & MUL.op_2 = F.val \\
&   &             &         &                                & T_0.val = MUL.res \\~\\ 

&T  & \rightarrow & F       &                                & T.val = F.val \\
&F  & \rightarrow & n       &                                & F.val = n.val \\
&F  & \rightarrow & (E)     &                                & F.val = E.val \\

\end{array}
$$  
\end{frame}

\begin{frame}[fragile]
  \transwipe[direction=90]
  \frametitle{Наследуемый атрибут, L-атрибутная грамматика}
Атрибут, значение которого зависит только от атрибутов братьев узла слева или атрибутов родителей, называется \textbf{наследуемым}
    
\vfill 

Грамматика называется \textbf{L-атрибутной}, если каждый наследуемый
атрибут узла $X_j$ в правиле $A \to X_1 \dots X_n$ зависит только от: 

\begin{itemize}
  \item Атрибутов узлов $X_1 \dots X_{j-1}$ (братья слева)
  \item Наследуемых атрибутов узла $А$ (предок)
\end{itemize}

\vfill 

Синтезируемые атрибуты тоже разрешены

\vfill 

Любая S-атрибутная грамматика является L-атрибутной
\end{frame}

\begin{frame}[fragile]
  \transwipe[direction=90]
  \frametitle{Пример L-атрибутной грамматики}
$$
\begin{array}{ccclll}
&D  & \rightarrow & TL      &                                & L.inh = T.type \\
&T  & \rightarrow & int     &                                & T.type = integer \\
&T  & \rightarrow & real    &                                & T.type = real \\~\\

&L_0& \rightarrow & L_1, id & \{ENTRY.type=(k, v)\} & L_1.inh = L_0.inh \\
&   &             &         &                                & ENTRY.k = id.text \\
&   &             &         &                                & ENTRY.v = L_0.inh \\~\\ 

& L & \rightarrow &  id     & \{ENTRY.type=(k, v)\} & ENTRY.k = id.text \\
&   &             &         &                                & ENTRY.v = L.inh \\
\end{array}
$$  
\end{frame}

\end{document}
