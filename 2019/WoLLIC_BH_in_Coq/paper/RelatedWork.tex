\section{Related Works}
\label{sec:rel-work}

There is a big number of works in the mechanization of different parts of formal languages theory and certified implementations of parsing algorithms and algorithms for graph database querying.
These works use various tools, such as Coq, Agda, Isabelle/HOL, and are aimed at different problems such as the theory mechanization or executable algorithm certification.
We discuss only a small part which is close enough to the scope of this work.

\subsection{Formal Language Theory in Coq}

The massive amount of work was done by Ruy de Queiroz who formalized different parts of formal language theory, such as pumping lemma~\cite{ramos2015formalizationPumping}, context-free grammar simplification~\cite{ramos2015formalization} and closure properties~\cite{ramos2015formalizationClosure} in Coq.
The work on closure properties contains mechanization of such properties as closure under union, Kleene star, but it does not contain mechanization of the intersection with a regular language.
All these results are summarized in~\cite{ramos2016formalization}. 

Gert Smolka et al. also provide a big set of works on regular and context-free languages formalization in Coq~\cite{smolka2017regular,smolka2013regular,kaiser2012constructive,smolkaHofmann2016}.
The paper~\cite{smolkaHofmann2016} describes the certified transformation of an arbitrary context-free grammar to the Chomsky normal form which is required for our proof of the Bar-Hillel theorem. 
Initially, we hoped to reuse these both parts because the Bar-Hillel theorem is about both context-free and regular languages, and it was the reason to choose results of Gert Smolka as the base for our work.
But the works on regular and on context-free languages are independent, and we are faced with the problems of reusing and integration, so in the current proof, we use only results on context-free languages.

\subsection{Formal Language Theory in Other Languages}

In the parallel with works in Coq there exist works on formal languages mechanization in other languages and tools such as Agda or Isabelle/HOL.

Firstly, there are works of Denis Firsov who implements some parts of the formal language theory and parsing algorithms in Agda.
In particular, Firsov implements CYK parsing algorithm~\cite{firsov2014certified,firsov2016cfl} and Chomsky Normal Form~\cite{firsov2015certified}, and some other results on regular languages~\cite{10.1007/978-3-319-03545-1_7}.

There are also works on the formal language theory mechanization in Isabelle/HOL~\cite{1885-16399,barthwal2010formalisation,10.1007/978-3-642-13824-9_11} by Aditi Bartwall and Michael Norrish.
This work contains basic definitions and a big number of theoretical results, such as Chomsky normal form and Greibach normal form for context-free grammars. 
As an application of the mechanized theory authors, provide certified implementation of the SLR parsing algorithm~\cite{10.1007/978-3-642-00590-9_12}.

\subsection{Certified Algorithms}

Additionally, we want to mention some works on certified applied algorithms based on the formal language theory.
Certification is required in different areas, and it is a reason to work on theory mechanization.

The first area where languages intersection may be applied is language constrained path querying in structured data (for example in graphs or XML).
There exist works on certification of the core of XQuery~\cite{10.1007/978-3-642-25379-9_21}. 
XQuery is a W3C standard for path querying in XML, extended for graph querying. 
Another result is work on certified Regular Datalog querying in Coq~\cite{certifiedPrologGraphQuerying}. 
Inspired by these results, our work may be a base for certified context-free path querying algorithm.

Another area which grows fast is certified parsers and parser generators based on different algorithms for different language classes~\cite{10.1007/978-3-642-28869-2_20,bernardy2016certified,Lopes2016CertifiedDP,Gross2015ParsingPA}. 
