\section{The Chomsky Normal Form}
\label{sec:cnf}

The important aspect of our proof is that any context-free language can be described with a grammar in Chomsky Normal Form (CNF) or, equally, any context-free grammar can be converted to the grammar in CNF which specifies the same language.
Let us recall the definition of CNF.% and the algorithm for conversion of an arbitrary context-free (CF) grammar to CNF.
{\definition[Chomsky Normal Form]
{A context-free grammar is in CNF if:
\begin{itemize}
\item the start nonterminal does not occur in the right-hand side of any rule,
\item all rules are of the form: $N_i \rightarrow t_i$, $N_i \rightarrow N_j N_k$ or $S \rightarrow \varepsilon$ where $N_i,N_j,N_k$ are nonterminals, $t_i$ is a terminal and $S$ is the start nonterminal.
\end{itemize} 
}}

%Transformation algorithm has the following steps.
%\begin{enumerate}
%\item Eliminate the start nonterminal from the right-hand sides of the rules.
%\item Eliminate rules with nonsolitary terminals.
%\item Eliminate rules which right-hand side contains more than two nonterminals.
%\item Delete $\varepsilon$-rules.
%\item Eliminate unit rules.
%\end{enumerate}

As far as Bar-Hillel theorem operates with arbitrary context-free languages and the selected proof requires grammar in CNF, it is necessary to implement a certified algorithm for the conversion of an arbitrary CF grammar to CNF.
We wanted to reuse existing mechanized proof for the conversion.
We chose the one provided in Smolka's work and discussed it in the context of our work in section~\ref{sec:solka-generalized}.



